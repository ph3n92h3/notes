\documentclass{article}

\usepackage[a4paper,scale=0.8]{geometry}

\usepackage{ctex}
\setCJKmainfont{Noto Serif CJK SC}
\setCJKsansfont{Noto Sans CJK SC}
\setCJKmonofont{Noto Sans Mono CJK SC}

% \usepackage{algorithm2e}
% \usepackage{amsfonts}
% \usepackage{amsmath}
% \usepackage{amssymb}
% \usepackage{cancel}
% \usepackage{emoji}
\usepackage{extarrows}
% \usepackage{float}
% \usepackage{framed}
% \usepackage[colorlinks]{hyperref}
% \usepackage{mathrsfs}
\usepackage{mathtools}
% \usepackage{minted}
% \usepackage{multicol}
\usepackage{physics}
% \usepackage{pifont}
% \usepackage{unicode-math}
% \usepackage{upgreek}
% \usepackage{xcolor}

\newcommand{\rme}{\mathrm{e}}
\newcommand{\rmi}{\mathrm{i}}

\title{temp}
\author{桜井\ 雪子}
% \date{}

\begin{document}

\maketitle

\section{力学}

\subsection{最小作用量原理导出运动方程}

\[ S[q(t)] = \int_{t_1}^{t_2} \dd{t} L(t, q, \dot{q}) \]

\[ \delta S \simeq - \int_{t_1}^{t_2} \dd{t} \qty[\dv{t} \qty(\pdv{L}{\dot(q)}) - \pdv{L}{q}] \]

\[ \dv{t} \qty(\pdv{L}{\dot(q)}) - \pdv{L}{q} = 0 \]

\subsection{诺特定理}

\[ \tilde{t} = t + \delta t, \tilde{q} = q + \delta q \]

\[ \delta_s \coloneqq \tilde{t} - t, \delta_s q(t) \coloneqq \tilde{q}(t) - q(t) \Rightarrow \delta_s \dot{q} = (\delta_s q)^{\cdot} \]

\[ \Delta q(t) \coloneqq \tilde{q}(\tilde{t}) - q(t) \xlongequal{\delta_s \coloneqq \tilde{t} - t}  \tilde{q}(t + \delta_s t) - q(t) \xlongequal{Taylor} \tilde{q}(t) + \delta_s t \dot{\tilde{q}}(t) - q(t) \xlongequal{\delta_s q(t) \coloneqq \tilde{q}(t) - q(t) \Rightarrow \dot{\tilde{q}} = \dot{q} + (\delta_s q)^{\cdot}} \delta_s q + \delta_s t \dot{q} \]

\[ \Delta \dot{q}(t) \coloneqq \dv{\tilde{q}(\tilde{t})}{\tilde{t}} - \dv{q(t)}{t} = \qty(\delta_s q)^{\cdot} - \delta_s t \ddot{q} \neq (\Delta q(t))^{\cdot} \]

\[ \Delta L = \cdots = \dv{t} \qty[\pdv{L}{\dot{q}} \delta_s q + L \delta_s t] \equiv \dv{F}{t} \]

\[ Q \coloneqq \qty[\pdv{L}{\dot{q}} \delta_s q + L \delta_s t] - F = p \delta_s q + L \delta_s t - F \Rightarrow \dv{Q}{t} = 0 \]

\textit{Quantum Field Theory by Mark Srednicki} Section 22 中对作用量变分的写法更简洁。

\section{量子场论}

\subsection{Mark Srednicki notation}

\[ \qty[\varepsilon_{ab}] = \qty[\varepsilon_{\dot{a} \dot{b}}] = \mqty[0 & -1 \\ +1 & 0], \qty[\varepsilon^{ab}] = \qty[\varepsilon^{\dot{a} \dot{b}}] = \mqty[0 & +1 \\ -1 & 0], \sigma^{\mu}_{a \dot{a}} = (I, \vec{\sigma}), \bar{\sigma}^{\mu a \dot{a}} = (I, - \vec{\sigma}) \]

\[ g_{\mu\nu} \sigma^{\mu}_{a \dot{a}} \sigma^{\nu}_{b \dot{b}} = - 2 \varepsilon_{ab} \varepsilon_{\dot{a} \dot{b}}, \varepsilon^{ab} \varepsilon^{\dot{a} \dot{b}} \sigma^{\mu}_{a \dot{a}} \sigma^{\nu}_{b \dot{b}} = - 2 g^{\mu \nu} \]

\subsubsection{Supersymmetry}

\begin{align*}
    \qty[Q_{aA}, P^{\mu}] = 0, & \qty[Q_{\dot{a}A}^{\dagger}, P^{\mu}] = 0 \\
    \qty[Q_{aA}, M^{\mu\nu}] = \qty(S_{L}^{\mu\nu})_{a}^{\ c} Q_{cA}, & \qty[Q_{\dot{a}A}^{\dagger}, M^{\mu\nu}] = \qty(S_{R}^{\mu\nu})_{\dot{a}}^{\ \dot{c}} Q_{cA} \\
    \qty{Q_{aA}, Q_{bB}} = Z_{AB} \varepsilon_{ab}, & \qty{Q_{aA}, Q_{\dot{a}B}^{\dagger}} = - 2 \delta_{AB} \sigma_{a\dot{a}}^{\mu} P_{\mu}, Z_{AB} + Z_{BA} = 0; \mathcal{N} = 1 \Rightarrow Z_{AB} = 0
\end{align*}

\end{document}
