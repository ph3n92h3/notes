\documentclass{article}

\usepackage[a4paper,scale=0.8]{geometry}

\usepackage{ctex}

% \usepackage{algorithm2e}
\usepackage{extarrows}
% \usepackage{float}
\usepackage{framed}
\usepackage{minted}
\usepackage{physics}
% \usepackage{unicode-math}
\usepackage{upgreek}

\newcommand{\rme}{\mathrm{e}}
\newcommand{\rmi}{\mathrm{i}}

\title{光学复习}
\author{桜井\ 雪子}
\date{}

\begin{document}
\maketitle

\section{几何光学}

\begin{itemize}
    \item 符号法则中,球反射的像距是特例
\end{itemize}

\subsection{球面折射}

$$
    \frac{n'}{s'}+\frac{n}{s}=\frac{n'-n}{r},\ V=-\frac{ns'}{n's}
$$

\subsection{球面反射}

$$
    \frac{1}{s'}+\frac{1}{s}=-\frac{2}{r},\ V=-\frac{s'}{s}
$$

\subsection{薄透镜}

$$
    f=\frac{n}{\frac{n_L-n}{r_1}+\frac{n'-n_L}{r_2}},\ f'=\frac{n'}{\frac{n_L-n}{r_1}+\frac{n'-n_L}{r_2}}
$$ $$
    n=n'\Rightarrow f=f'=\frac{1}{\left(\frac{n_L}{n}-1\right)\left(\frac{1}{r_1}-\frac{1}{r_2}\right)}
$$ $$
    V=-\frac{ns'}{n's}
$$

\subsection{其他放大率}

\subsubsection{纵向放大率}

$$
    \alpha\equiv\frac{\mathrm{d}x'}{\mathrm{d}x}=\frac{n'}{n}\beta^2
$$

\subsubsection{角放大率}

$$
    \beta\gamma=\frac{n}{n'}
$$

\begin{itemize}
    \item 几何光学作图法
\end{itemize}

\section{光的干涉}

\subsection{基础}

$$
    I=I_1+I_2+2\sqrt{I_1I_2}\cos(\Delta\varphi)
$$ $$
    \Delta\varphi=\frac{2\pi}{\lambda}\delta
$$ \begin{itemize}
    \item 相长:$\Delta\varphi=2k\pi,\ \delta=k\lambda$
    \item 相消:$\Delta\varphi=(2k+1)\pi,\ \delta=(k+\frac{1}{2})\lambda$
\end{itemize}

\subsection{杨氏双缝}

\subsubsection{基础}

$$
    \delta=d\cdot\sin\theta=d\cdot y/D,\ \Delta y=\frac{\lambda D}{d}
$$
注意,这里说的是一条亮线到相邻亮线的间距,是周期 \begin{itemize}
    \item 光源上下移动:直接相似三角形
    \item 空气入射到水的反射光,有半波损失
\end{itemize}

\subsubsection{空间相干性}

\begin{itemize}
    \item 光源的极限宽度——用光源上下移动的方法算出来的,移动距离使得条纹移动距离等于条纹间距
\end{itemize}

\subsection{薄膜干涉}

\subsubsection{等倾}

$$
    \delta=2nh\cos\gamma=2h\sqrt{n^2-n_1^2\sin^2i}
$$
这里 $\gamma$ 是折射角,暂时没有计入半波损失,考虑时,需要考虑两个界面上的~
\begin{itemize}
    \item 干涉圆环级次内大外小,厚度增大,中心处有圆环冒出
    \item 透射光干涉的光程差,也是上面那个东西,不过 $n_1\to n_2$
\end{itemize}

\subsubsection{等厚}

\begin{itemize}
    \item 光程差,还是上面那个,垂直入射时(暂时没有考虑半波损失):$$
              \delta=2nh
          $$
    \item 牛顿环 $$
              \delta=2\times h+\frac{\lambda}{2}=2\times \frac{r^2}{2R}+\frac{\lambda}{2}=\frac{r^2}{R}+\frac{\lambda}{2}
          $$ 暗环:$\delta=(k+1/2)\lambda$,包括 $r=0$
\end{itemize}

\subsection{迈克尔逊干涉仪}

$$
    \delta=2d
$$
注意这个 $2$,以及薄膜干涉中的 $2$

\subsection{干涉条纹可见度、时间相关性}

\begin{itemize}
    \item 光源非单色:$k_{max}=\lambda/\Delta\lambda, \delta_{max}=\lambda^2/\Delta\lambda$
    \item 相干时间:$\Delta\tau_0=\delta_{max}/c$
\end{itemize}

\section{光的衍射}

\subsection{菲涅尔衍射}

\subsubsection{半波带法}

\begin{itemize}
    \item $a_i\approx\frac{a_{i-1}}{2}+\frac{a_{i+1}}{2}$
    \item $A(P)=\frac{a_1}{2}\pm\frac{a_n}{2}$,奇、偶
    \item 振幅矢量法:将一个半波带细分(半圆)
\end{itemize}

\subsubsection{圆孔}

计算露出的半波带数 $$
    r_k=r_0+\frac{k\lambda}{2},\ r_k^2=\rho_k^2+(r_0+h)^2,\ R^2=\rho_k^2+(R-h)^2\\\Rightarrow k_{max}=\frac{\rho^2}{\lambda}\left(\frac{1}{r_0}+\frac{1}{R}\right)
$$

\subsection{夫琅禾费衍射}

\subsubsection{单缝}

$$
    A_p=A_0\frac{\sin u}{u},\ u=\frac{\pi b}{\lambda}\sin\theta
$$
其中 $b$ 缝宽,$\theta$ 观察角度
\begin{itemize}
    \item 极值 \begin{itemize}
              \item 极值条件:$\sin u=0,\ u=\tan u$
              \item 主极大:$u=0$
              \item 次极大:$u=\tan u,\ u\neq 0,\ \sin\theta\approx(k+1/2)\lambda/b$
              \item 极小:$\sin u=0,\ u\neq 0,\ \sin\theta=k\lambda/b$
          \end{itemize}
    \item 亮纹 \begin{itemize}
              \item 中央极大,第一级暗纹:$b\sin\theta_1=\lambda$,**半**角宽度:$\Delta\theta_0=\arcsin(\lambda/b)$,线宽度:$2f\lambda/b$
              \item 其他亮纹:$\Delta\theta_k=\lambda/b,\ \Delta x_k=f\lambda/b$
          \end{itemize}
\end{itemize}

\subsubsection{圆孔}

\begin{itemize}
    \item 艾里斑半角宽/一级暗环衍射角:$\sin\theta_1=0.61\times\lambda/r=1.22\times \lambda/d$
    \item 光学仪器最小分辨角 $\theta_R=1.22\lambda/d$,分辨本领 $R=1/\theta_R$
\end{itemize}

\subsection{光栅}

$$
    A_p=A_0\frac{\sin u}{u}\times\frac{\sin N\nu}{\sin \nu},\ u=\frac{\pi b\sin\theta}{\lambda},\ \nu=\frac{\pi d\sin\theta}{\lambda}
$$
其中,$b$ 透光缝宽,$d$ 光栅常数
\begin{itemize}
    \item 单缝衍射因子:$\sin u/u$,决定主极大强度
    \item 缝间干涉因子:$\sin N\nu/\sin \nu$,决定主极大位置
    \item 相邻主极大之间有 $N-1$ 个极小和 $N-2$ 个次极大
\end{itemize}

\subsubsection{条纹}

\begin{itemize}
    \item 亮条纹:$d\sin\theta=k\lambda$ \begin{itemize}
              \item 最大级次:$\theta=\pm\pi/2$
              \item (各个)主极大(半?)角宽度:$\Delta\theta=\lambda/Nd\cos\theta$
          \end{itemize}
    \item 暗条纹:$d\sin\theta=\lambda\cdot m/N,\ m=1,2,\dots,N-1,N+1\dots,\ m\neq kN$
    \item 缺级:$\sin u=0,\ \sin\nu =0$ \begin{itemize}
              \item 如果 $(a+b)/b=3/1=k/n$,则 $k=3,6,9,\dots$ 缺级
              \item 如果 $(a+b)/b=3/2=k/n$,则 $k=3,6,9,\dots$ 缺级
          \end{itemize}
\end{itemize}

\subsubsection{分辨本领}

\begin{itemize}
    \item 角色散率:$\mathrm{d}\theta/\mathrm{d}\lambda=k/d\cos\theta$
    \item 线色散率:$\mathrm{d}l/\mathrm{d}\lambda=f\mathrm{d}\theta/\mathrm{d}\lambda=kf/d\cos\theta$
    \item 光栅的分辨本领:$R=\lambda/\Delta\lambda=kN$
\end{itemize}

\subsection{X 射线衍射}

\begin{itemize}
    \item 布拉格方程:$\delta=2d\sin\theta=k\lambda$
\end{itemize}

\section{光的偏振}

\begin{itemize}
    \item 五种偏振态:自然光、线偏振光、部分偏振光、椭圆偏振光、圆偏振光 \begin{itemize}
              \item 迎着光的传播方向观察(同一场点),电矢量沿顺(逆)时针转,右(左)旋光
              \item 区分五种偏振态:先用偏振片,再用四分之一波片
          \end{itemize}
    \item 马吕斯定律:$I=I_0\sin^2\alpha$
\end{itemize}

\subsection{反射和折射光}

\begin{itemize}
    \item 菲涅尔公式
\end{itemize}

\subsubsection{反射光}

$$
    \frac{A_{s1}'}{A_{s1}}=-\frac{\sin(i_1-i_2)}{\sin(i_1+i_2)},\ \frac{A_{p1}'}{A_{p1}}=\frac{\tan(i_1-i_2)}{\tan(i_1+i_2)},\ n_1\sin i_1=n_2\sin i_2
$$
\begin{itemize}
    \item 垂直入射或掠入射时,……
    \item 垂直振动多于平行振动
    \item 布儒斯特定律:$i_1+i_2=\pi/2(\text{i.e. }\tan i_1=n_2/n_1)\Rightarrow A_{p1}'/A_{p1}=0$,线偏振
\end{itemize}

\subsubsection{折射光}

$$
    \frac{A_{s2}}{A_{s1}}=\frac{2\sin i_2\cos i_1}{\sin(i_1+i_2)},\ \frac{A_{p2}}{A_{p1}}=\frac{2\sin i_2\cos i_1}{\sin(i_1+i_2)\cos(i_1-i_2)},\ n_2\sin i_2=n_1\sin i_1
$$
\begin{itemize}
    \item 平行振动多于垂直振动
    \item 自然光以布儒斯特角入射时,折射光仍是部分偏振光: $$
              A_{s2}=A_{s1}\cdot 2\sin^2i_2,\ A_{p2}=A_{p1}\cdot\tan i_2
          $$
    \item 玻璃片堆 $$
              A_{s}^{(2n)}=A_s\cdot\sin^n(2i_2)\to 0,\ A_{p}^{(2n)}=A_p
          $$
\end{itemize}

\subsection{双折射}

\begin{itemize}
    \item o 光 e 光都是线偏振光
    \item 光轴:一个特殊方向,光线延此方向传播不发生双折射
    \item 主截面:晶体的光轴与表面法线组成的平面
    \item 主平面:晶体光轴与光纤组成的平面
    \item 垂直入射线偏振光,振动平面与主截面夹角为 $\theta$,o 光振动垂直于主截面:$A_0=A\sin\theta,\ A_e=A\cos\theta,\ I_o=n_oA_o^2=n_oA^2\sin^2\theta,\ I_e=n_eA_e^2=n_e(\alpha)A^2\cos^2\theta$,出射之后仍有:$I_o/I_e=\tan^2\theta$
    \item 对于光轴,负晶体(方解石) $v_{\perp}>v_{\parallel}$,正晶体(石英)相反
    \item 作图法:惠更斯原理,波前
\end{itemize}

\subsection{晶体偏振器件}

\begin{itemize}
    \item 尼科尔棱镜是一个偏振片
    \item 四分之一波片、二分之一波片:使得 o 超前 e 这么多个波长(相位超前这么多个 $2\pi$)
\end{itemize}

\subsection{旋光效应}

\begin{itemize}
    \item $\theta=\alpha d$
    \item 迎着光线观察,顺(逆)时针旋转,右(左)旋物质
\end{itemize}

\end{document}