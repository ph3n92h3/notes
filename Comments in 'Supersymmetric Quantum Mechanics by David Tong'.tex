\documentclass{article}

\usepackage[a4paper,scale=0.85]{geometry}

\usepackage{ctex}

\usepackage{amsmath}\DeclareMathOperator{\Tr}{Tr}
\usepackage{amssymb}
\usepackage{bbm}
\usepackage{bm}
\usepackage{derivative}
\usepackage{extarrows}
\usepackage[colorlinks]{hyperref}
\usepackage{paracol}\columnratio{0.6}
\usepackage{physics2}\usephysicsmodule{ab, braket}

\newcommand{\rmi}{\mathrm{i}}
\newcommand{\rme}{\mathrm{e}}

\title{Comments in \textit{Supersymmetric Quantum Mechanics} by David Tong \\ David Tong 《超对称量子力学》中的评论}
\author{桜井\ 雪子}
\date{}

\begin{document}

\maketitle
\tableofcontents

\paragraph{翻译工具}\url{https://pot-app.com} \& \url{https://translate.google.com/}

\setcounter{section}{-1}
\section{Introduction}

\begin{paracol}{2}
It will come as no surprise to hear that there is a close relationship between mathematics and physics. Yet, for many centuries, the relationship was more than a little one sided. There was, in the language of marriage counsellors, a lack of equitable reciprocity. Physicists took, but gave little in return. Admittedly there were exceptions, some of them rather important like Newton's development of calculus. Nonetheless, it remains true that mathematics is a tool that us physicists cannot live without, while many mathematicians have no more use of physics than they do of chemistry or botany.
\switchcolumn
听到数学和物理学之间存在密切的关系也就不足为奇了。然而,几个世纪以来,这种关系不仅仅是一点点片面的。用婚姻顾问的话说,缺乏公平的互惠。物理学家索取了,但几乎没有给予任何回报。诚然,也有例外,其中一些相当重要,比如牛顿对微积分的发展。尽管如此,数学仍然是我们物理学家离不开的工具,而许多数学家对物理学的使用并不比对化学或植物学的使用更多。
\switchcolumn*

In the last few decades, this narrative has started to change. Physicists have been giving back. As our understanding of quantum field theories has grown, we have un- covered increasingly sophisticated mathematical structures lurking within. These are largely, but not exclusively, the structures that arise in geometry and topology. Using physicist's methods and techniques to solve quantum fields theories has revealed connections to these mathematical ideas. Initially this gave new ways of deriving results well known to mathematicians. But, as the quantum field theories became more in- volved, so too did the mathematics until physicists were able to discover new results that came as a complete surprise to mathematicians. Prominent among these is an idea called mirror symmetry, a novel relationship between different manifolds.
\switchcolumn
在过去的几十年里,这种说法开始发生变化。物理学家一直在回馈。随着我们对量子场理论理解的加深,我们发现了潜伏在其中的日益复杂的数学结构。这些主要但不完全是几何和拓扑中出现的结构。使用物理学家的方法和技术来解决量子场理论已经揭示了与这些数学思想的联系。最初,这为数学家提供了推导结果的新方法。但是,随着量子场理论变得越来越复杂,数学也变得越来越复杂,直到物理学家能够发现令数学家完全惊讶的新结果。其中最突出的是一种称为镜像对称的想法,这是不同流形之间的一种新颖关系。
\switchcolumn*

You might reasonably wonder what advantage physicists have over mathematicians in this game. After all, we're certainly not smarter. (At least, not most of us.) And yet, there are times when we are able to leapfrog mathematicians and then turn around and present them with new results that sit firmly within their area of expertise. This seems unfair, like physicists have some kind of secret weapon that mathematicians are unable to wield. And we do. In fact, we have two. The first is the path integral. The second, a wilful disregard for rigour.
\switchcolumn
您可能有理由想知道在这个游戏中物理学家比数学家有什么优势。毕竟,我们当然并不聪明。(至少,我们大多数人不是。)然而,有时我们能够超越数学家,然后转身向他们展示完全属于他们专业领域的新结果。这似乎不公平,就像物理学家拥有某种数学家无法使用的秘密武器一样。我们确实这么做了。事实上,我们有两个。第一个是路径积分。第二,故意无视严格性。
\switchcolumn*

These two weapons are not unrelated. The path integral approach to quantum field theory has so far evaded attempts to be placed on a rigorous footing, at least beyond quantum mechanics. This means that most often the physicist's approach to these questions does not meet the mathematician's bar for proof. Physics is perhaps better thought of as an idea generating machine, giving new insights into areas of mathematics that can subsequently be proven using more traditional methods. Happily, in most cases, these subsequent proofs have turned out to be much more than an exercise in dotting i's and crossing $\hbar$'s. Mathematicians take their own path to a problem, developing new ideas along the way, and these then feed back into our understanding of quantum field theory. Over the past few decades this process has resulted in a harmonious and extraordinarily fruitful relationship between communities of physicists and mathematicians.
\switchcolumn
这两种武器并非毫无关联。迄今为止,量子场论的路径积分方法还没有被建立在严格的基础上,至少超出了量子力学的范围。这意味着物理学家解决这些问题的方法通常不符合数学家的证明标准。物理学也许更适合被认为是一种想法产生机器,它为数学领域提供了新的见解,随后可以使用更传统的方法来证明。令人高兴的是,在大多数情况下,这些后续的证明不仅仅是给 i 加点和给 $\hbar$ 画叉的练习。数学家们走自己的路来解决问题,一路上提出新的想法,然后这些想法反馈到我们对量子场论的理解中。在过去的几十年里,这一过程在物理学家和数学家群体之间建立了和谐且卓有成效的关系。
\switchcolumn*

This interaction has revolutionised certain areas of mathematics. For example, it's difficult to envisage a thriving field of symplectic geometry without mirror symmetry. But it has also changed what we mean by "mathematical physics". Towards the end of the 20th century, this was viewed as a rather a dry subject and mostly involved bringing a mathematician's level of pedantry to bear on problems that physicists care about, but with little insight flowing back into the underlying physics. Now, this situation has been reversed, with interesting and exciting ideas flowing in both directions. To emphasise the shift of focus, this new activity is sometimes rebranded "physical mathematics".
\switchcolumn
这种相互作用彻底改变了数学的某些领域。例如,如果没有镜像对称性,就很难想象辛几何领域会蓬勃发展。但它也改变了我们所说的“数学物理”的含义。到了 20 世纪末,这被认为是一门相当枯燥的学科,主要涉及用数学家的迂腐水平来解决物理学家关心的问题,但很少有洞察力回到底层物理学。现在,这种情况已经发生了逆转,有趣且令人兴奋的想法在两个方向流动。为了强调焦点的转移,这项新活动有时被重新命名为“物理数学”。
\switchcolumn*

Much of this interplay between physics and mathematics takes place in the arena of supersymmetric field theories. (There are important exceptions, Witten's Fields medal winning work on knot polynomials in Chern Simons theory among them.) Supersymmetric theories are a class of quantum field theories that have a symmetry relating bosons and fermions. There is, so far, no experimental evidence that supersymmetry is a symmetry of our world. But supersymmetric theories have a number of special prop- erties that allow us to make much more progress in solving them than would otherwise be possible. It is often in these solutions to supersymmetric field theories that we find results of interest to mathematicians.
\switchcolumn
物理学和数学之间的这种相互作用大部分发生在超对称场论领域。(有一些重要的例外,其中包括陈西蒙斯理论中关于结多项式的威滕菲尔兹奖获奖作品。超对称理论是一类量子场论,具有与玻色子和费米子相关的对称性。到目前为止,还没有实验证据表明超对称性是我们世界的对称性。但是超对称理论有许多特殊的属性,使我们能够在解决它们方面取得比其他方式更多的进展。通常,在这些超对称场论的解中,我们发现了数学家感兴趣的结果。
\switchcolumn*

The purpose of these lectures is to take the first first few steps along this journey. Sadly we will not reach the heights of the subject like mirror symmetry or knot invariants, both of which require quantum field theories in higher dimensions ($d = 1 + 1$ and $d = 2 + 1$ respectively). Instead, we will restrict ourselves to $d = 0 + 1$ dimensional quantum field theories, also known as quantum mechanics. We will study a number of examples of supersymmetric quantum mechanics and, in solving them, recover some of the highlights of 20th century geometry, including ideas of de Rham, Hodge, Morse, Atiyah and Singer.
\switchcolumn
这些讲座的目的是在这一旅程中迈出最初的几步。遗憾的是,我们无法达到镜像对称或扭结不变量那样的高度,这两者都需要更高维度的量子场论(分别为 $d = 1 + 1$ 和 $d = 2 + 1$)。相反,我们将把自己限制在 $d = 0 + 1$ 维量子场理论,也称为量子力学。我们将研究一些超对称量子力学的例子,并在解决它们的过程中恢复 20 世纪几何学的一些亮点,包括德拉姆、霍奇、莫尔斯、阿蒂亚和辛格的思想。
\switchcolumn*

I should warn you that the level of rigour when addressing the more mathematical aspect of these lectures will be mediocre at best. Anyone with a real interest in these ideas is encouraged to learn both the underlying mathematics and physics to truly appreciate how the two connect. But that is not the path we will take here. Instead, these lectures will assume only a basic knowledge in differential geometry (at the level, say, of my lectures on General Relativity.) We will then use supersymmetric quantum mechanics as a vehicle to take us deeper into the mathematician's territory, allowing us to take a peek at some of the beautiful vistas that await.
\switchcolumn
我应该警告你,这些讲座中涉及数学方面的严谨程度充其量也只是平庸。我们鼓励任何对这些想法真正感兴趣的人学习基础数学和物理,以真正理解两者之间的联系。但这不是我们在这里要走的道路。相反,这些讲座将仅假设微分几何的基础知识(例如,在我关于广义相对论的讲座的水平上)。然后,我们将使用超对称量子力学作为工具,带我们更深入地进入数学家的领域,使我们能够欣赏一些等待着的美丽景色。
\switchcolumn*

\section{Introducing Supersymmetric Quantum Mechanics}
\switchcolumn
\section*{Introducing 超对称量子力学}
\switchcolumn*

\subsection{Supersymmetry Algebra}
\switchcolumn
\subsection*{超对称代数}
\switchcolumn*

\subsubsection{A First Look at the Energy Spectrum}
\switchcolumn
\subsubsection*{能谱初探}
\switchcolumn*

In this section, we discuss some basic facts about supersymmetric quantum mechanics. Our focus will be on a simple class of quantum mechanical systems that, while they have a certain elegance, won't exhibit any deep mathematics. Instead, we will treat them as a proving ground, allowing us to build some intuition for supersymmetry while developing a number of useful calculational techniques. We'll then bring these to bear on problems with a deeper mathematical pedigree in Section 3.
\switchcolumn
在本节中,我们讨论有关超对称量子力学的一些基本事实。我们的重点将是一类简单的量子力学系统,虽然它们具有一定的优雅性,但不会表现出任何深奥的数学。相反,我们会将它们视为试验场,使我们能够对超对称性建立一些直觉,同时开发一些有用的计算技术。然后,我们将在第 3 节中将这些内容应用于具有更深入数学谱系的问题。
\switchcolumn*

As an aside: there's only one other place in physics where we care about the over- all value of the ground state energy, and that's the cosmological constant in general relativity. So far, sadly, no plausible link has been found between the value of the cosmological constant and the supersymmetry algebra.
\switchcolumn
顺便说一句:物理学中只有另一个地方我们关心基态能量的整体价值,那就是广义相对论中的宇宙常数。遗憾的是,到目前为止,我们还没有发现宇宙常数的值和超对称代数之间存在合理的联系。
\switchcolumn*

Finally, one last piece of terminology. If a ground state with energy $E = 0$ exists, then we say that supersymmetry is unbroken. If the ground state has energy $E > 0$ then we say that supersymmetry is broken. This language is really adopted from higher dimensions where symmetries that do not leave the vacuum invariant are said to be "spontaneously broken". In the present context we say that supersymmetry is broken if the vacuum is not annihilated by the supercharges: the connection to symmetries will become clearer as we proceed.
\switchcolumn
最后,最后一个术语。如果存在能量 $E = 0$ 的基态,则我们说超对称性未破缺。如果基态的能量 $E > 0$,则我们说超对称性被破坏。这种语言实际上是从更高维度采用的,在更高维度中,不保持真空不变量的对称性被称为“自发破缺”。在目前的情况下,我们说如果真空没有被超电荷湮灭,超对称性就会被打破:随着我们的继续,与对称性的联系将变得更加清晰。
\switchcolumn*

\subsection{A Particle in a Potential}
\switchcolumn
\subsection*{势中的单粒子}
\switchcolumn*

\subsubsection{Ground States}
\switchcolumn
\subsubsection*{基态}
\switchcolumn*

The magic of supersymmetry means that, at least for the ground state, the Schrödinger equation has morphed from a challenging second order differential equation into a pair of decoupled, first order differential equations. Note that this same trick doesn't work to figure out the excited states of the theory. We can't solve for the whole spectrum. But we can solve for the ground state.
\switchcolumn
超对称的魔力意味着,至少对于基态,Schrödinger 方程已经从一个具有挑战性的二阶微分方程变成了一对解耦的一阶微分方程。请注意,同样的技巧无法弄清楚理论的激发状态。我们无法解决整个频谱。但是我们可以解决基态。
\switchcolumn*

Usually in a double well potential, the particle can lower its energy by tunnelling through the barrier and sitting in a superposition of both states. But that's not the case here because the two wavefunctions live in different components of spin space. This kills the possibility for tunnelling. Instead, the supersymmetric set-up is closer to our naive, classical guess of the ground states, with a Gaussian around each minima giving a good approximation to the ground state. Our arguments above tell us that the energy of this two-fold degenerate ground state is necessarily $E > 0$ We will say more about tunnelling in this system and how to compute the actual energy in Section 2.2.
\switchcolumn
通常在双势阱中,粒子可以通过隧道穿过势垒并处于两种状态的叠加来降低其能量。但这里的情况并非如此,因为这两个波函数位于自旋空间的不同组成部分。这消除了隧道效应的可能性。相反,超对称设置更接近于我们对基态的朴素经典猜测,每个最小值周围都有一个高斯分布,可以很好地近似基态。我们上面的论点告诉我们,这个两倍简并基态的能量必然是 $E > 0$ 我们将在第 2.2 节中详细介绍该系统中的隧道效应以及如何计算实际能量。
\switchcolumn*

We started with three states that we thought had the smallest energy - one for each minima - but only one survives as the true $E = 0$ ground state. The other two states must have some small, but non-zero energy. These states are the Gaussian localised in the middle vacuum, and the combination of states localised on the outside minima that is orthogonal to the ground state. Although it is far from obvious from staring at the potential, supersymmetry tells us that the energies of these states must be degenerate.
\switchcolumn
我们从我们认为能量最小的三种状态开始——每个极小值对应一种状态——但只有一种状态能够作为真正的 $E = 0$ 基态存在。其他两个状态必须具有一些小但非零的能量。这些状态是位于中真空的高斯状态,以及位于与基态正交的外部最小值上的状态组合。尽管从盯着势来看还远不明显,但超对称性告诉我们这些状态的能量一定是简并的。
\switchcolumn*

\subsubsection{The Witten Index}
\switchcolumn
\subsubsection*{Witten 指标}
\switchcolumn*

Before we proceed, a few comments. Since $\mathcal{I}$ doesn't depend on $\beta$, you might wonder why we don't just set $\beta = 0$ and consider $\mathrm{Tr} (-1)^F$. Indeed, often the Witten index is written in this way as shorthand, but it's a dangerous thing to do. The quantity $\mathrm{Tr} (-1)^F$ is an infinite series of $+ 1$ and $- 1$ and by pairing terms together in various ways you can get any answer that you like. Including $\rme^{- \beta H}$ in the definition acts as a regulator for this sum, rendering it finite. Of course, it's a familiar regulator because it also appears in the partition function in statistical mechanics.
\switchcolumn
在我们继续之前,先发表一些评论。由于 $\mathcal{I}$ 不依赖于 $\beta$,你可能想知道为什么我们不直接设置 $\beta = 0$ 并考虑 $\mathrm{Tr} (-1)^F$。事实上,Witten 指数经常以这种方式写成速记,但这是一件危险的事情。数量 $\Tr (-1)^F$ 是 $+ 1$ 和 $- 1$ 的无限级数,通过以各种方式将术语配对在一起,您可以获得您喜欢的任何答案。定义中包含 $\rme^{- \beta H}$ 作为该总和的调节器,使其成为有限的。当然,它是一个熟悉的调节器,因为它也出现在统计力学的配分函数中。
\switchcolumn*

The same arguments that show $\odv*{\mathcal{I}}{\beta} = 0$ also show that $\mathcal{I}$ is independent of the parameters of the Hamiltonian $H$. This was demonstrated in the examples above although, as we also saw, it comes with a caveat: if you change the Hamiltonian too dramatically then you can lose states in your Hilbert space and this will change $\mathcal{I}$. This happens for the particle on a line whenever we change the power of the leading term in $h(x)$.
\switchcolumn
显示 $\odv*{\mathcal{I}}{\beta} = 0$ 的相同论证也表明 $\mathcal{I}$ 独立于哈密顿量 $H$ 的参数。这在上面的示例中得到了证明,但正如我们也看到的那样,它带有一个警告:如果您将哈密顿量更改得太大,那么您可能会丢失希尔伯特空间中的状态,这将改变 $\mathcal{I}$。每当我们改变 $h(x)$ 中首项的幂时,这条线上的粒子就会发生这种情况。
\switchcolumn*

The Witten index counts the difference between the bosonic and fermionic $E = 0$ states. However, in the simple examples considered above, it actually counts the number of $E = 0$ states, positive if they're bosonic, negative if they're fermionic. One might wonder if, in practice, it always does this. Indeed, there's is some intuition that suggests this is the case. If there's no good reason for pairs of states to be stuck at $E = 0$ then, as you vary parameters in the potential, it's tempting to think that they will be lifted to $E > 0$.
\switchcolumn
Witten 指数计算了玻色子和费米子 $E = 0$ 状态之间的差异。然而,在上面考虑的简单示例中,它实际上计算 $E = 0$ 状态的数量,如果它们是玻色子,则为正,如果它们是费米子,则为负。人们可能想知道,在实践中,它是否总是这样做。事实上,有一些直觉表明情况确实如此。如果没有充分的理由让状态对停留在 $E = 0$,那么当你改变势能参数时,很容易认为它们会被提升到 $E > 0$。
\switchcolumn*

However, it's not difficult to exhibit examples where, for example, $\mathcal{I} = 0$ but there are a pair of bosonic and fermionic $E = 0$ states. A particularly simple example arises from particle moving on a circle $\mathrm{S}^1$ of radius $R$. The supercharge (1.5) and Hamiltonian (1.7) take the same form as before and are characterised by a periodic function $h(x) = h(x + 2 \pi R)$. We can follow our earlier footsteps to find a two parameter family of ground states labelled by $\alpha, \beta \in \mathbb{C}$
\switchcolumn
然而,展示例子并不困难,例如,$\mathcal{I} = 0$,但存在一对玻色子和费米子 $E = 0$ 状态。一个特别简单的例子是粒子在半径为 $R$ 的圆 $\mathrm{S}^1$ 上移动。增压 (1.5) 和哈密顿量 (1.7) 采用与之前相同的形式,并以周期函数 $h(x) = h(x + 2 \pi R)$ 为特征。我们可以按照之前的脚步找到一个由 $\alpha, \beta \in \mathbb{C}$ 标记的二参数基态族
\end{paracol}

\[ \Psi(x) = \alpha \begin{pmatrix}
        \rme^{- h} \\ 0
    \end{pmatrix} + \beta \begin{pmatrix}
        0 \\ \rme^{+ h}
    \end{pmatrix} \]

\begin{paracol}{2}
This time, because the particle lives on a circle, there is no issue with the normalisability of the wavefunction. We see that the system has two linearly independent $E = 0$ ground states for any choice of $h$. Yet, because one ground states lives in $\mathcal{H}_B$ and the other in $\mathcal{H}_F$ , the Witten index of this system is $\mathcal{I} = 0$. The potential (in blue) and wavefunctions (in orange and green) for $h(x) = \sin(x / R)$ are shown in Figure 4.
\switchcolumn
这次,因为粒子生活在一个圆上,所以波函数的归一化性不存在问题。我们看到,对于 $h$ 的任何选择,系统都有两个线性独立的 $E = 0$ 基态。然而,由于一个基态位于 $\mathcal{H}_B$ 中,另一个基态位于 $\mathcal{H}_F$ 中,因此该系统的 Witten 指数为 $\mathcal{I} = 0$。 $h(x) = \sin(x / R)$ 的势(蓝色)和波函数(橙色和绿色)如图 4 所示。
\switchcolumn*

For this particle on the circle, the pair of states sticks at $E = 0$ as we change the parameters of $h$, even though these ground states are not protected by the Witten index. One might wonder if there's a deeper reason for this. There is and it's related to the deeper mathematical concept of cohomology. We'll look at this further in Section 3.
\switchcolumn
对于圆上的这个粒子,当我们改变 $h$ 的参数时,这对状态保持在 $E = 0$,即使这些基态不受 Witten 指数的保护。人们可能想知道这是否有更深层次的原因。确实存在,并且它与更深层的上同调数学概念有关。我们将在第 3 节中进一步讨论这一点。
\switchcolumn*

Finally, one last comment before we move on. The manipulations of the Witten index rely on the discreteness of the energy spectrum. There are more subtle situations, where a particle moves on a non-compact space without a potential, where the energy spectrum is continuous and, despite the bose-fermi degeneracy in the spectrum, strange things can happen that mean that $\mathcal{I}$ does, in fact, depend on $\beta$. We will not encounter situations of this kind in these lectures.
\switchcolumn
最后,在我们继续之前,还有最后一条评论。Witten 指数的操纵依赖于能谱的离散性。还有更微妙的情况,其中粒子在没有势能的非紧空间上移动,能量谱是连续的,尽管谱中存在玻色费米简并性,但可能会发生奇怪的事情,这意味着 $\mathcal{I}$ 事实上,依赖于 $\beta$。在这些讲座中我们不会遇到这种情况。
\switchcolumn*

\subsection{The Supersymmetric Action}
\switchcolumn
\subsection*{超对称作用量}
\switchcolumn*

There is one fairly large omission in our discussion so far. As presented above, super- symmetric Hamiltonians have a nice algebraic structure. But we have no inkling of why supersymmetry has anything to do with symmetry!
\switchcolumn
到目前为止,我们的讨论中有一个相当大的遗漏。如上所述,超对称哈密顿量具有良好的代数结构。但我们不知道为什么超对称性与对称性有任何关系!
\switchcolumn*

Usually in quantum mechanics, Hermitian operators that commute with the Hamiltonian correspond to conserved quantities and conserved quantities come, via Noether's theorem, from symmetries. This suggests that perhaps $Q + Q^{\dagger}$ is somehow the conserved charge associated to a symmetry. But what symmetry?
\switchcolumn
通常在量子力学中,与哈密顿量交换的厄米算子对应于守恒量,而守恒量通过 Noether 定理来自对称性。这表明 $Q + Q^{\dagger}$ 可能是与对称性相关的守恒电荷。但什么是对称性呢?
\switchcolumn*

Often the Lagrangian framework is a better starting point when looking for symmetries. To this end, we would like to introduce a Lagrangian for our supersymmetric theory of a particle on a line. We know well how to think of position and momentum in the Lagrangian setting. But how do we incorporate the discrete $\mathbb{C}^2$ factor in the Hilbert space that gave us the all-important $\mathbb{Z}^2$ grading?
\switchcolumn
在寻找对称性时,Lagrangian 框架通常是一个更好的起点。为此,我们想为直线上粒子的超对称理论引入拉格朗日。我们很清楚如何考虑 Lagrangian 设置中的位置和动量。但是我们如何将离散的 $\mathbb{C}^2$ 因子合并到希尔伯特空间中,从而为我们提供最重要的 $\mathbb{Z}^2$ 分类呢?
\switchcolumn*

The answer is that we should turn to fermions. In higher dimensions, adding a fermion to a Lagrangian gives another field. But in quantum mechanics, fermions simply offer a different way of describing some discrete aspect of the physics.
\switchcolumn
答案是我们应该转向费米子。在更高维度中,将费米子添加到拉格朗日量中会产生另一个场。但在量子力学中,费米子只是提供了一种不同的方式来描述物理学的某些离散方面。
\switchcolumn*

Note that their kinetic terms are first order, like the Dirac action that we met in Quantum Field Theory, albeit without the intricacies of gamma matrices. We will first show that this action is equivalent to the supersymmetric Hamiltonian (1.7) describing a particle with an internal degree of freedom moving on a line. We'll then understand how to think of the supercharges $Q$ in the Lagrangian formulation.
\switchcolumn
请注意,它们的动力学项是一阶的,就像我们在量子场论中遇到的 Dirac 作用一样,尽管没有 gamma 矩阵的复杂性。我们将首先证明这个作用相当于描述具有内部自由度的沿直线运动的粒子的超对称哈密顿量(1.7)。然后我们将了解如何考虑 Lagrangian 公式中的超荷 $Q$。
\switchcolumn*

There is, however, a small subtlety awaiting us. We think of the Lagrangian as a classical object in which $x$ and $\dot{x} = p$ be placed in any order. Relatedly, $\psi$ and $\psi^{\dagger}$ are viewed as "classical Grassmann variables" in the action, which means that if one moves past the other then we just pick up a minus sign. But in the Hamiltionian, these are all to be thought of as quantum operators and, because of the commutation relations (1.15), ordering matters. Which ordering should we take?
\switchcolumn
然而,有一个小微妙之处等待着我们。我们将拉格朗日函数视为一个经典对象,其中 $x$ 和 $\dot{x} = p$ 可以按任意顺序放置。相关地,$\psi$ 和 $\psi^{\dagger}$ 在动作中被视为“经典格拉斯曼变量”,这意味着如果一个移动超过另一个,那么我们只会拾取一个负号。但在哈密尔顿量纲中,这些都被视为量子算子,并且由于交换关系 (1.15),排序很重要。我们应该采取哪种顺序?
\switchcolumn*

In most other contexts, there is no way to fix this ambiguity and it reflects the fact that there are different ways to quantise a classical theory. However, for us, we do have a way to fix the ambiguity since the resulting Hamiltonian should be supersymmetric.
\switchcolumn
在大多数其他情况下,没有办法解决这种歧义,它反映了一个事实,即有不同的方法来量化经典理论。然而,对我们来说,我们确实有办法解决歧义,因为所得的哈密顿量应该是超对称的。
\switchcolumn*

\subsubsection{Supersymmetry as a Fermionic Symmetry}
\switchcolumn
\subsubsection*{超对称性作为费米子对称性}
\switchcolumn*

Note that these swap bosonic fields $x$ for fermionic fields $\psi$. This is the characteristic feature of supersymmetry that distinguishes it from other symmetries. For this to make sense, the infinitesimal transformation parameter $\epsilon$ must be a Grassmann valued object.
\switchcolumn
请注意,这些将玻色子场 $x$ 交换为费米子场 $\psi$。这是超对称性区别于其他对称性的特征。为了使这一点有意义,无穷小变换参数 $\epsilon$ 必须是 Grassmann 值对象。
\switchcolumn*

\subsection*{The Supercharge is a Noether Charge}
\switchcolumn
\subsection*{超荷是 Noether 荷}
\switchcolumn*

Finally, we can make good on our promise and see that the supercharges $Q$ and $Q^{\dagger}$ are indeed Noether charges for supersymmetry. Usually when the action has a symmetry, we can construct the Noether charge by allowing the transformation parameter to depend on time. Things are no different here. We vary the action with $\epsilon = \epsilon(t)$. There are two steps where things differ from our previous calculation: first when we vary the kinetic terms, and again at the last where we see that the variation of the Lagrangian is a total derivative which requires an integration by parts.
\switchcolumn
最后,我们可以兑现我们的承诺,看到超级电荷 $Q$ 和 $Q^{\dagger}$ 确实是超对称性的 Noether 荷。通常当作用具有对称性时,我们可以通过允许变换参数依赖于时间来构造诺特电荷。这里的情况并没有什么不同。我们用 $\epsilon = \epsilon(t)$ 来改变操作。有两个步骤与我们之前的计算有所不同:首先,我们改变了动力学项;最后,我们看到 Lagrangian 的变化是全导数,需要分部分积分。
\switchcolumn*

It's slightly odd that the variation of the action involves $\dot{\epsilon}^{\dagger}$ but not $\dot{\epsilon}$. We can trace this to our choice of fermion kinetic term $\psi^{\dagger} \dot{\psi}$, which is asymmetric between $\psi$ and $\psi^{\dagger}$. We could instead start with the more symmetric choice
\switchcolumn
有点奇怪的是,动作的变化涉及 $\dot{\epsilon}^{\dagger}$ 而不是 $\dot{\epsilon}$。我们可以将其追溯到我们对费米子动力学项 $\psi^{\dagger} \dot{\psi}$ 的选择,它在 $\psi$ 和 $\psi^{\dagger}$ 之间是不对称的。我们可以从更对称的选择开始
\switchcolumn*

We can now go full circle. In the operator framework of quantum mechanics, the Noether charges generate the symmetry. Again, supersymmetry is no different.
\switchcolumn
现在我们可以回到原点了。在量子力学算子框架中,诺特电荷产生对称性。同样,超对称性也不例外。
\switchcolumn*

\subsection{A Particle Moving in Higher Dimensions}
\switchcolumn
\subsection*{在更高维度中运动的单粒子}
\switchcolumn*

\subsubsection{A First Look at Morse Theory}
\switchcolumn
\subsubsection*{Morse 理论初探}
\switchcolumn*

This means that our supersymmetric quantum mechanics will describe a particle moving in $\mathrm{R}^n$ with $2^n$ internal states.
\switchcolumn
这意味着我们的超对称量子力学将描述一个在 $\mathrm{R}^n$ 中运动且内部状态为 $2^n$ 的粒子。
\switchcolumn*

There's a useful geometrical way to think about these states. At the top of the pyramid depicted above we have wavefunctions that look like $\phi(x) \ket{0}$: these are just functions over $\mathrm{R}^n$.
\switchcolumn
有一种有用的几何方法来思考这些状态。在上面描述的金字塔顶部,我们有看起来像 $\phi(x) \ket{0}$ 的波函数:这些只是 $\mathrm{R}^n$ 上的函数。
\switchcolumn*

At the next level, the wavefunctions look like $\phi(x) \psi(x)^{\dagger i} \ket{0}$ and come with an internal index $i = 1, \dots, n$. We usually think of objects on $\mathrm{R}^n$ that carry such an index as vectors. However, as we now explain, the anti-symmetric nature of the Grassmann variable means that it's much more natural to think about these states as one-forms on $\mathrm{R}^n$.
\switchcolumn
在下一个级别,波函数看起来像 $\phi(x) \psi(x)^{\dagger i} \ket{0}$ 并带有内部索引 $i = 1, \dots, n$。我们通常将 $\mathrm{R}^n$ 上带有此类索引的对象视为向量。然而,正如我们现在所解释的,格拉斯曼变量的反对称性质意味着将这些状态视为 $\mathrm{R}^n$ 上的一种形式更为自然。
\switchcolumn*

All of this suggests that we should make the identification between Grassmann variables and forms
\switchcolumn
所有这些都表明我们应该对格拉斯曼变量和形式进行认同
\end{paracol}

\[ \psi^{\dagger i} \longleftrightarrow \odif{x^i}\wedge \]

\begin{paracol}{2}
\subsubsection*{The Supersymmetric Hamiltonian}
\switchcolumn
\subsubsection*{超对称 Hamiltonian}
\switchcolumn*

The generalisation of the story above is now the following: for each negative eigenvalue $\lambda_k < 0$, we should excite the corresponding collection of fermions $e^j_k \psi^{\dagger j}$. Meanwhile, for each positive eigenvalue $\lambda_k > 0$, we should just leave well alone: we're better off in the unexcited state. At a given critical point $x = X$, the semi-classical ground state then sits in the part of the Hilbert space given by
\switchcolumn
上述故事的概括如下:对于每个负特征值 $\lambda_k < 0$,我们应该激发相应的费米子集合 $e^j_k \psi^{\dagger j}$。同时,对于每个正特征值 $\lambda_k > 0$,我们应该不管它:我们在非激励状态下会更好。在给定的临界点 $x = X$ 处,半经典基态位于由下式给出的希尔伯特空间部分
\switchcolumn*

In the geometrical language, this means that the ground state wavefunction is a $p$-form, where $p = \mu(X)$ is the Morse index.
\switchcolumn
在几何语言中,这意味着基态波函数是 $p$ 形式,其中 $p = \mu(X)$ 是 Morse 指数。
\switchcolumn*

Note that we're not assuming that all critical points of $h$ correspond to true $E = 0$ ground states of the theory. It may well be that some get lifted to non-zero energy and, later in these lectures, we'll put in some effort to understand when this happens. But that's not relevant for computing the Witten index since any such states must get lifted in pairs and so cancel out.
\switchcolumn
请注意,我们并不假设 $h$ 的所有临界点都对应于理论的真实 $E = 0$ 基态。很可能有些能量会被提升到非零能量,在这些讲座的后面,我们将努力理解这种情况何时发生。但这与计算 Witten 指数无关,因为任何此类状态都必须成对提升,从而抵消。
\switchcolumn*

The same formula (1.29) also holds for our earlier model with a single $x$ and $\psi$. There a maximum of $h$ was necessarily followed by a minimum, so the sum over critical points could never exceed $+ 1$ or drop below $- 1$. Now, however, we could have multiple ground states. For example, we could have a situation where all the critical points $X$ have $\mu(X)$ even. In this case, they all contribute $+ 1$ to the Witten index and each of them must correspond to a true, $E = 0$ ground state of the system.
\switchcolumn
相同的公式 (1.29) 也适用于我们之前具有单个 $x$ 和 $\psi$ 的模型。$h$ 的最大值后面必然有最小值,因此临界点的总和永远不会超过 $+ 1$ 或低于 $- 1$。然而,现在我们可以有多个基态。例如,我们可能会遇到这样一种情况,即所有临界点 $X$ 都有 $\mu(X)$ 为偶数。在这种情况下,它们都为 Witten 指数贡献了 $+ 1$,并且它们中的每一个都必须对应于系统的真实 $E = 0$ 基态。
\switchcolumn*

\subsubsection{More Supersymmetry and Holomorphy}
\switchcolumn
\subsubsection*{更多超对称和全纯}
\switchcolumn*

Hamiltonian that can be written in this form is said to have $N = 2 q$ supersymmetries, with the $2$ because each $Q$ is complex. In this convention, the kind of quantum mechanics that we considered up until now is said to have $N = 2$ supersymmetry. (I should warn you that the nomenclature for counting supersymmetry generators in quantum mechanics is not completely standard: things settle down once we go to higher dimensional quantum field theories.)
\switchcolumn
可以写成这种形式的哈密顿量据说具有 $N = 2 q$ 超对称性,其中 $2$ 是因为每个 $Q$ 都是复数。在这个惯例中,我们到目前为止所考虑的量子力学被认为具有 $N = 2$ 超对称性。 (我应该警告你,量子力学中计算超对称发生器的术语并不完全标准:一旦我们进入更高维的量子场论,事情就会稳定下来。)
\switchcolumn*

At first glance, it looks like these are simply different Hamiltonians. However, all is not lost: these two Hamiltonians coincide if the function $h(x, y)$ obeys
\switchcolumn
乍一看,这些似乎只是不同的哈密顿量。然而,一切并没有丢失:如果函数 $h(x, y)$ 服从(以下条件),这两个 Hamiltonian 重合
\end{paracol}

\[ \pdv{h}{x^i, x^j} = - \pdv{h}{y^i, y^j} \text{ and } \pdv{h}{x^i, y^j} = \pdv{h}{y^i, x^j} \]

\begin{paracol}{2}
There's a much nicer way of writing these conditions: as we will now see, they are telling us that $h(x, y)$ is related to a holomorphic function.
\switchcolumn
有一种更好的方式来编写这些条件:正如我们现在所看到的,它们告诉我们 $h(x, y)$ 与全纯函数相关。
\switchcolumn*

\subsubsection*{Complex Variables}
\switchcolumn
\subsubsection*{复数变量}
\switchcolumn*

Here the word "complex" is in inverted com- mas because our original Grassmann variables were already complex; we just introduce different linear combinations
\switchcolumn
这里“复数”这个词用引号引起来,因为我们最初的格拉斯曼变量已经是复数了;我们只是引入不同的线性组合
\switchcolumn*

Supersymmetric Lagrangians of this kind, involving complex scalar fields and fermions, are usually referred to as Landau-Ginzburg theories. This is a nod to the Landau-Ginzburg theories that we met when discussing phase transitions in Statistical Physics. But it's not a very good nod. In particular, the theory (1.25) with just a single supersymmetry is just as much related to the kinds of models that Landau and Ginzburg considered but is never given this name in the context of supersymmetry. It's best to think of the name "Landau-Ginzburg" for the Lagrangian (1.33) as merely a quirk of history and forget that the term is also used elsewhere in physics.
\switchcolumn
这种涉及复标量场和费米子的超对称拉格朗日量通常被称为 Landau-Ginzburg 理论。这是对我们在讨论统计物理学中的相变时遇到的 Landau-Ginzburg 理论的认可。但这并不是一个很好的点头。特别是,只有一个超对称性的理论(1.25)与 Landau 和 Ginzburg 考虑的模型类型同样相关,但在超对称性的背景下从未被赋予这个名称。最好将拉格朗日量 (1.33) 的名称“Landau-Ginzburg”视为历史的一个巧合,而忘记该术语在物理学的其他领域也有使用。
\switchcolumn*

The Landau-Ginzburg Lagrangian depends on a single holomorphic function $W (z)$. This is known as the superpotential. The fact that extended supersymmetry comes hand in hand with holomorphy and associated ideas in complex analysis is extremely important. We will not discuss quantum mechanics with $N = 4$ supersymmetry in these lectures, but it's not for want of interesting content. In particular, there is a beautiful relationship to a form of complex geometry known as "Kähler geometry" that underlies many of the most interesting results in this subject.
\switchcolumn
Landau-Ginzburg Lagrangian 依赖于单个全纯函数 $W (z)$。这被称为超潜力。扩展的超对称性与复分析中的全纯性和相关思想齐头并进,这一事实极其重要。在这些讲座中,我们不会讨论具有 $N = 4$ 超对称性的量子力学,但这并不是因为缺乏有趣的内容。特别是,它与一种被称为“Kähler 几何”的复杂几何形式有着美妙的关系,它是该学科中许多最有趣的结果的基础。
\switchcolumn*

Furthermore, when we go to higher dimensional field theories, supersymmetry generators are associated to spinors and these necessarily have more than one component. This means that in, for example, $d = 3 + 1$ dimensions, the simplest supersymmetric theories have the form (1.33) and are based on complex, rather than real variables. In that context, the holomorphy of the superpotential goes a long way towards allowing us to solve some complicated features of supersymmetric quantum field theories. This is covered in some detail in the lectures on Supersymmetric Field Theory.
\switchcolumn
此外,当我们研究高维场论时,超对称发生器与旋量相关联,并且它们必然具有多个分量。这意味着,例如,在 $d = 3 + 1$ 维度中,最简单的超对称理论具有 (1.33) 的形式,并且基于复变量而不是实变量。在这种情况下,超势的全纯性对于让我们解决超对称量子场理论的一些复杂特征大有帮助。超对称场论讲座对此进行了详细介绍。
\switchcolumn*

\subsubsection*{The Ground States}
\switchcolumn
\subsubsection*{基态}
\switchcolumn*

We know from our discussion in Section 1.4.1 what we should do next: we compute the Morse index for each critical point, meaning the number of positive eigenvalues of the Hessian of $h$. But this is trivial for a holomorphic function $W (z)$.
\switchcolumn
从第 1.4.1 节的讨论中我们知道下一步应该做什么:我们计算每个临界点的 Morse 指数,即 $h$ 的 Hessian 矩阵的正特征值的数量。但这对于全纯函数 $W (z)$ 来说是微不足道的。
\switchcolumn*

We learn that in theories with $N = 4$ supersymmetry, every critical point of $W$ is a true $E = 0$ ground state of the quantum theory.
\switchcolumn
我们了解到,在具有 $N = 4$ 超对称性的理论中,$W$ 的每个临界点都是量子理论的真正 $E = 0$ 基态。
\switchcolumn*

\subsubsection{Less Supersymmetry and Spinors}
\switchcolumn
\subsubsection*{更少超对称和旋量}
\switchcolumn*

It's also possible to consider theories with less supersymmetry than our starting point. In fact, this is easy to achieve. We return to our theory with $N = 2$ supersymmetry and impose a reality condition on the Grassmann variables
\switchcolumn
也可以考虑比我们的起点具有更少超对称性的理论。事实上,这很容易实现。我们回到 $N = 2$ 超对称性的理论,并对 Grassmann 变量施加现实条件
\end{paracol}

\[ \psi^{\dagger i} = \psi^{i} \]

\begin{paracol}{2}
Real quantum mechanical Grassmann variables like this are called Majorana modes or Majorana fermions.
\switchcolumn
像这样的真正的量子力学格拉斯曼变量被称为 Majorana 模式或 Majorana 费米子。
\switchcolumn*

This is usually referred to as $N = 1$ supersymmetry. (You will sometimes see the terminology $N = 1 / 2$ supersymmetry in the literature, counting complex supercharges rather than real.)
\switchcolumn
这通常称为 $N = 1$ 超对称性。(有时您会在文献中看到术语 $N = 1 / 2$ 超对称性,计算的是复杂的增压而不是真实的。)
\switchcolumn*

Here our interest lies in a very specific property of these theories: how should we think of the internal degrees of freedom generated by the real fermions $\psi^{i}$?
\switchcolumn
这里我们的兴趣在于这些理论的一个非常具体的属性:我们应该如何思考由实费米子 $\psi^{i}$ 生成的内部自由度?
\switchcolumn*

This means that the fermions in this theory should be viewed as gamma matrices! The Clifford algebra has a unique irreducible representation of dimension $2^{n / 2}$ if $n$ is even and $2^{(n - 1) / 2}$ if $n$ is odd. This strongly suggests that the internal degrees of freedom of the particle described by the action (1.34) have something to do with spinors on $\mathbb{R}^n$.
\switchcolumn
这意味着该理论中的费米子应该被视为 gamma 矩阵!如果 $n$ 为偶数,则 Clifford 代数具有维度 $2^{n / 2}$ 的唯一不可约表示;如果 $n$ 为奇数,则具有 $2^{(n - 1) / 2}$。这强烈表明由作用 (1.34) 描述的粒子的内部自由度与 $\mathbb{R}^n$ 上的旋量有关。
\switchcolumn*

This is precisely the dimension of a Dirac spinor on $\mathrm{R}^{n}$.
\switchcolumn
这正是 $\mathrm{R}^{n}$ 上狄拉克旋量的维数。
\switchcolumn*

There is more to say about these spinors. Under a rotation in $\mathrm{R}^{n}$, the Dirac spinor transforms in the representation generated by $\Sigma^{ij} = \frac{1}{4} \ab[\gamma^i, \gamma^j]$. (See the lectures on Quantum Field Theory for more details of this.) However, in even dimension, as we have here, this is not an irreducible representation. It is composed of two smaller representations known as chiral spinors or Weyl spinors.
\switchcolumn
关于这些旋量还有更多要说的。在 $\mathrm{R}^{n}$ 的旋转下,Dirac 旋量变换为 $\Sigma^{ij} = \frac{1}{4} \ab[\gamma^i, \gamma^j]$。 (有关更多详细信息,请参阅量子场论讲座。)然而,在偶数维中,正如我们在这里所看到的,这不是一个不可约的表示。它由两个较小的表示形式组成,称为手性旋量或 Weyl 旋量。
\switchcolumn*

These arise because we can always construct an operator $\hat{\gamma}$ that is analogous to $\gamma^5$ in four dimensions.
\switchcolumn
出现这些问题是因为我们总是可以构造一个类似于四个维度中的 $\gamma^5$ 的运算符 $\hat{\gamma}$ 。
\switchcolumn*

In the context of our supersymmetric quantum mechanics, this $\hat{\gamma}$ operator has a very natural meaning. The eigenvalues are simply states with an even or odd number of $c^{\dagger}$ operators excited. In other words, this plays the role of our fermion number.
\switchcolumn
在超对称量子力学的背景下,这个 $\hat{\gamma}$ 运算符具有非常自然的含义。特征值只是由偶数或奇数个 $c^{\dagger}$ 运算符激发的状态。换句话说,这扮演了我们的费米子数的角色。
\switchcolumn*

This means that $\hat{\gamma}$ determines whether states live in $\mathcal{H}_B$ or $\mathcal{H}_F$.
\switchcolumn
这意味着 $\hat{\gamma}$ 决定状态是否位于 $\mathcal{H}_B$ 或 $\mathcal{H}_F$ 中。
\switchcolumn*

The punchline of this argument is that quantising real fermions, appropriate for $N = 1$ supersymmetry, gives Dirac spinors on $\mathbb{R}^{n}$, at least for $n$ even. These have dimension $2^{n/2}$. Meanwhile, while quantising complex fermions, appropriate for $N = 2$ supersymmetry, gives forms on $\mathbb{R}^{n}$. These have dimension $2^n$. We'll have use for quantum mechanics with $N = 1$ supersymmetry in Section 3.3 where we discuss the Atiyah-Singer index theorem.
\switchcolumn
这个论点的要点是,量子化实费米子,适用于 $N = 1$ 超对称性,给出 $\mathbb{R}^{n}$ 上的 Dirac 旋量,至​​少对于 $n$ 偶数。它们的维度为 $2^{n/2}$。同时,在量子化复费米子时,适用于 $N = 2$ 超对称性,给出了 $\mathbb{R}^{n}$ 上的形式。它们的维度为 $2^n$。我们将在第 3.3 节中讨论 Atiyah-Singer 指数定理,使用具有 $N = 1$ 超对称性的量子力学。
\switchcolumn*

As an aside, clearly the construction of spinors and forms on $\mathbb{R}^{n}$ from Grassmann degrees of freedom is closely related. This also suggests that you can take $2^{n/2}$ different Dirac spinors and bundle them together to look like forms. Such a construction is called Kähler-Dirac fermions. It won't play a role in these lectures, but arises in a number of other areas of physics including topological twisting of field theories and lattice gauge theory where it goes by the name of staggered fermions.
\switchcolumn
顺便说一句,显然,来自 Grassmann 自由度 的 $\mathbb{R}^{n}$ 上的形式和旋量的构造是密切相关的。这也表明您可以采用 $2^{n/2}$ 不同的 Dirac 旋量并将它们捆绑在一起以看起来像形式。这种结构称为 Kähler-Dirac 费米子。它不会在这些讲座中发挥作用,但会出现在物理学的许多其他领域,包括场论的拓扑扭曲和晶格规范理论,其名称为交错费米子。
\switchcolumn*

\subsubsection*{The Case of $n$ Odd: A Subtle Anomaly}
\switchcolumn
\subsubsection*{$n$ 为奇数的情况:微妙的异常}
\switchcolumn*

We still have to understand the case of $n$ odd. Here there is a surprise. Quantum mechanical theories with an odd number of Majorana modes don't make any sense! They are an example of what is sometimes called an \textit{anomalous} quantum theory: a seemingly sensible classical theory that cannot be quantised.
\switchcolumn
我们仍然需要理解 $n$ 奇数的情况。这里有一个惊喜。具有奇数个 Majorana 模式的量子力学理论没有任何意义!它们是有时被称为\textit{异常}量子理论的一个例子:一种看似合理但无法量子化的经典理论。
\switchcolumn*

For us, this means that theories with $N = 1$ supersymmetry are restricted to describe a particle moving in an even dimensional space, like $\mathbb{R}^n$ with $n$ even.
\switchcolumn
对我们来说,这意味着具有 $N = 1$ 超对称性的理论仅限于描述在偶数维空间中移动的粒子,例如 $\mathbb{R}^n$ 和 $n$ 偶数。
\switchcolumn*

\section{Supersymmetry and the Path Integral}
\switchcolumn
\section*{超对称和路径积分}
\switchcolumn*

In addition to making the symmetry aspect of supersymmetry manifest, the Lagrangian description of the quantum mechanics has one additional advantage: it allows us to bring the path integral to bear on the problem.
\switchcolumn
除了使超对称性的对称性变得明显之外,量子力学的 Lagrangian 描述还有一个额外的优点:它允许我们用路径积分来解决问题。
\switchcolumn*

We'll make plenty of use of the path integral in later studies of supersymmetric systems. The purpose of this section is to understand some of the basic properties of the quantum mechanical path integral and how we can use it to compute quantities of interest in supersymmetric theories.
\switchcolumn
我们将在以后的超对称系统研究中大量使用路径积分。本节的目的是了解量子力学路径积分的一些基本属性,以及如何使用它来计算超对称理论中感兴趣的量。
\switchcolumn*

\subsection{The Partition Function and the Index}
\switchcolumn
\subsection*{配分函数和指标}
\switchcolumn*

Our goal now is to manipulate (2.2) so that the left-hand-side looks like the partition function $Z$. There are a number of differences that we need to fix.
\switchcolumn
我们现在的目标是操纵 (2.2),使左侧看起来像配分函数 $Z$。我们需要修复许多差异。
\switchcolumn*

where now the boundary conditions just tell us that we should integrate over all possible closed paths. Equivalently, we can implement this condition by insisting that we work in periodic Euclidean time, so that $\tau$ is a coordinate on a circle $S^1$, with
\switchcolumn
现在边界条件只是告诉我们应该整合所有可能的闭合路径。等价地,我们可以通过坚持在 Euclidean 周期时间内工作来实现这个条件,因此 $\tau$ 是圆 $S^1$ 上的坐标,其中
\end{paracol}

\[ \tau \equiv \tau + \beta \]

\begin{paracol}{2}
Although we've derived this punchline in the context of quantum mechanics, it also works in quantum field theory. If you want to compute the thermal partition function of any quantum field theory, you simply need to work in periodic, Euclidean time. This will tell you information about the equilibrium properties of the system at temperature $T = 1 / \beta$.
\switchcolumn
尽管我们是在量子力学的背景下得出这个妙语的,但它也适用于量子场理论。如果你想计算任何量子场理论的热配分函数,你只需要在周期性的 Euclidean 时间中工作。这将告诉您有关系统在温度 $T = 1 / \beta$ 时的平衡特性的信息。
\switchcolumn*

\subsubsection{An Example: The Harmonic Oscillator}
\switchcolumn
\subsubsection*{例子:谐振子}
\switchcolumn*

That leaves us with the first infinite product in (2.5) to deal with. And that's more tricky because it diverges. To better understand such terms, we should really go back and dissect the path integral to figure out where it came from. (For example, the partition function should be dimensionless but this term has dimension of $[\text{Energy}]^{2 \infty}$ which is a hint that we didn't define our measure very well.) However, in the spirit of this course we're going to treat this term as blithely as possible. And, for those physicists of a blithe disposition, there are few tools more useful than zeta function regularisation.
\switchcolumn
这让我们需要处理 (2.5) 中的第一个无限乘积。这更加棘手,因为它存在发散。为了更好地理解这些项,我们真的应该回顾并剖析路径积分以找出它的来源。(例如,配分函数应该是无量纲的,但这项的维数为 $[\text{Energy}]^{2 \infty}$,这暗示我们没有很好地定义我们的度量。)但是,本着本课程的精神,我们将尽可能轻松地对待这个术语。而且,对于那些性格开朗的物理学家来说,没有什么工具比 zeta 函数正则化更有用了。
\switchcolumn*

However, $\zeta(s)$ is defined for all values of $s$. The idea is that we use this to give meaning to divergent sums. For example, we could think of the sum of all positive integers as $\zeta(- 1) = - 1 / 12$. Although these zeta function games seem rather inane when you first meet them, the magic is that they tend to give the right answers when used to regulate divergences in quantum field theory. (For example, in the lectures on String Theory we first invoked the unconvincing $\zeta(- 1) = - 1 / 12$ argument to compute the critical dimension of the string, and then spent a significant amount of time rederiving this using conformal field theory techniques where the divergences were absent.)
\switchcolumn
然而,$\zeta(s)$ 是为 $s$ 的所有值定义的。我们的想法是,我们用它来赋予发散的总和意义。例如,我们可以将所有正整数的总和视为 $\zeta(- 1) = - 1 / 12$。尽管这些 zeta 函数游戏在你第一次见到时看起来相当愚蠢,但神奇的是,当它们用于调节量子场论中的发散时,它们往往会给出正确的答案。(例如,在弦理论讲座中,我们首先调用了令人难以置信的 $\zeta(- 1) = - 1 / 12$ 参数来计算弦的临界尺寸,然后花费大量时间使用不存在发散的共形场论技术重新推导它。)
\switchcolumn*

\subsubsection{Fermions: Periodic or Anti-Periodic?}
\switchcolumn
\subsubsection*{费米子:周期还是反周期?}
\switchcolumn*

Relatedly, there are two natural partition functions that we could construct for fermions. In addition to the thermal partition function $\Tr \rme^{- \beta H}$ , we could also consider the quantity $\Tr (-1)^F \rme^{- \beta H}$. In supersymmetric quantum mechanics, $\Tr (-1)^F \rme^{- \beta H}$ is the Witten index and is necessarily an integer. But, for a general fermionic system it is just a different way to sum the states, weighted by an extra minus sign. I'll refer to the quantity $\Tr (-1)^F \rme^{- \beta H}$ as an "index" in both supersymmetric and non-supersymmetric theories, although strictly this terminology should be reserved for the former case.
\switchcolumn
相关地,我们可以为费米子构造两个自然配分函数。除了热分配函数 $\Tr \rme^{- \beta H}$ 之外,我们还可以考虑数量 $\Tr (-1)^F \rme^{- \beta H}$。在超对称量子力学中,$\Tr (-1)^F \rme^{- \beta H}$ 是 Witten 指标,并且必然是整数。但是,对于一般的费米子系统来说,这只是一种不同的状态求和方式,由一个额外的负号加权。我将把数量 $\Tr (-1)^F \rme^{- \beta H}$ 称为超对称和非超对称理论中的“指标”,尽管严格来说这个术语应该保留给前者的情况。
\switchcolumn*

It seems plausible that inserting a factor of $(-1)^F$ in the trace would flip the sign of the fermion as we go around the Euclidean temporal circle. But which boundary condition corresponds to the thermal partition function, and which to the index?
\switchcolumn
当我们绕 Euclidean 时间圆运行时,在迹中插入一个因子 $(-1)^F$ 会翻转费米子的符号,这似乎是合理的。但是哪个边界条件对应于热分配函数,哪个边界条件对应于指标呢?
\switchcolumn*

As always, the right answer can be found by going back to first principles and looking at how one constructs the path integral from small, but finite, time steps.
\switchcolumn
与往常一样,通过回到第一原理并研究如何从小而有限的时间步构建路径积分,可以找到正确的答案。
\switchcolumn*

\subsubsection*{The Fermionic Oscillator}
\switchcolumn
\subsubsection*{费米子谐振子}
\switchcolumn*

Clearly the index isn’t independent of $\beta$ for this simple model: that is only true for supersymmetric systems.
\switchcolumn
显然,对于这个简单模型,该指数并不独立于 $\beta$:这只适用于超对称系统。
\switchcolumn*

It will not have escaped your attention that the path integral calculation was a lot of work to get the partition function for a two state system. However, as we come to consider more complicated quantum mechanical models, including higher dimensional field theories, the path integral starts to come into its own and, ultimately, is much more convenient than canonical quantisation.
\switchcolumn
您一定会注意到,为了获得二态系统的配分函数,路径积分计算需要大量工作。然而,当我们开始考虑更复杂的量子力学模型,包括高维场论时,路径积分开始发挥作用,并且最终比规范量子化方便得多。
\switchcolumn*

\subsubsection{The Witten Index Revisited}
\switchcolumn
\subsubsection*{重新审视 Witten 指标}
\switchcolumn*

It’s useful to understand why, from the path integral perspective, the Witten index is always an integer for supersymmetric theories. After all, something magical must happen where we do an infinite dimensional integral but, regardless of the parameters in the integrand, we always get an integer. How does this come about? The answer is a rather special property of supersymmetric path integrals known as localisation.
\switchcolumn
从路径积分的角度来看,理解为什么 Witten 指数对于超对称理论总是整数是有用的。毕竟,当我们进行无限维积分时,一定会发生一些神奇的事情,但无论被积函数中的参数如何,我们总是得到一个整数。这是怎么发生的?答案是超对称路径积分的一个相当特殊的属性,称为局域化。
\switchcolumn*

The bosonic field $x(\tau)$ is always periodic: $x(\tau) = x(\tau + \beta)$. But that means that the supersymmetry transformations (2.14) only hold if $\psi$ is also periodic: $\psi(\tau) = \psi(\tau + \beta)$.
\switchcolumn
玻色子场 $x(\tau)$ 始终是周期性的:$x(\tau) = x(\tau + \beta)$。但这意味着超对称变换(2.14)仅在 $\psi$ 也是周期性的情况下成立:$\psi(\tau) = \psi(\tau + \beta)$。
\switchcolumn*

As we’ve seen, if we wish to compute the thermal partition function $Z = \Tr \rme^{- \beta H}$ using the path integral then we must give the fermions anti-periodic boundary conditions. But, in doing so, we break supersymmetry. In contrast, if we wish to compute the Witten index $\Tr (-1)^{F} \rme^{- \beta H}$ then the path integral enjoys supersymmetry. This makes intuitive sense. In general, the full partition function $Z$ is no easier to compute for a supersymmetric theory than a non-supersymmetric theory. But the Witten index is much easier. And, from the path integral perspective, this manifests itself because of the transformations (2.14).
\switchcolumn
正如我们所看到的,如果我们希望使用路径积分计算热配分函数 $Z = \Tr \rme^{- \beta H}$,那么我们必须给出费米子反周期边界条件。但是,这样做就打破了超对称性。相反,如果我们希望计算 Witten 指标 $\Tr (-1)^{F} \rme^{- \beta H}$ 则路径积分具有超对称性。这很直观。一般来说,超对称理论的完整配分函数 $Z$ 并不比非超对称理论更容易计算。但 Witten 指标要容易得多。并且,从路径积分的角度来看,这由于变换而显现出来(2.14)。
\switchcolumn*

The extra term in the integrand has a special form because it is itself a supersymmetry variation. To see this, it’s useful to use the supersymmetry generators that we introduced in (1.21). With a rescaled potential $\lambda h$ and Euclidean time, these become
\switchcolumn
被积函数中的额外项具有特殊形式,因为它本身就是超对称变体。为了看到这一点,使用我们在 (1.21) 中引入的超对称生成器很有用。通过重新调整势 $\lambda h$ 和欧几里得时间,这些变为
\switchcolumn*

(Note that there’s no danger of a boundary term here because $\tau$ parameterises a circle and all fields are periodic.)
\switchcolumn
(请注意,这里不存在边界项的危险,因为 $\tau$ 参数化了一个圆,并且所有域都是周期性的。)
\switchcolumn*

But we also know that the action is invariant under supersymmetry and, as we showed in (1.22), this can be written as $\mathcal{Q}^{\dagger}_{\lambda} S_E = 0$. This means that our final expression is a path integral of a total supersymmetry variation,
\switchcolumn
但我们也知道,作用在超对称性下是不变的,正如我们在 (1.22) 中所示,这可以写成 $\mathcal{Q}^{\dagger}_{\lambda} S_E = 0$。这意味着我们的最终表达式是总超对称性变化的路径积分,
\end{paracol}

\[ \odv{\mathcal{I}}{\lambda} = \int \mathcal{D} x \mathcal{D} \psi \mathcal{D} \psi^{\dagger} \mathcal{Q}^{\dagger}_{\lambda} \ab(- \rme^{- S_E} \oint \odif{\tau} h' \psi) \]

\begin{paracol}{2}
The integrand is said to be $Q$-exact. The all-important point is that the integral of any $Q$-exact quantity always vanishes.
\switchcolumn
据说该被积函数是 $Q$ 精确的。最重要的一点是任何 $Q$ 精确数量的积分总是消失。
\switchcolumn*

We’re left with the bosonic functional derivatives $\fdv{}/{x(t)}$. Here we have a total derivative, albeit of a functional kind and we would expect such an integral to be given by the boundary term. The question is: what should we consider to be the boundary of this functional space? Large $x(t)$? Wildly varying $x(t)$? Either way, the boundary term vanishes. This is because there is an exponential suppression from the action $\rme^{- S_E}$ that asymptotes quickly to zero for anything that you might reasonably consider to be the boundary. The upshot of these arguments is that
\switchcolumn
我们剩下玻色子泛函导数 $\fdv{}/{x(t)}$。这里我们有一个全导数,尽管是函数类型,并且我们期望这样的积分由边界项给出。问题是:我们应该将什么视为这个功能空间的边界?$x(t)$ 很大?$x(t)$ 变化很大?无论哪种方式,边界项都会消失。这是因为 $\rme^{- S_E}$ 操作存在指数抑制,对于您可能合理地认为是边界的任何内容,它都会快速渐近到零。这些争论的结果是
\switchcolumn*

Now we’re in business. Because the Witten index is independent of $\lambda$, we can calculate it in the limit $\lambda \to \infty$. Here the potential term in the action suppresses all contribution except for the a finite number of constant maps,
\switchcolumn
现在我们开始做生意了。因为 Witten 指标与 $\lambda$ 无关,所以我们可以在 $\lambda \to \infty$ 的极限下计算它。这里,动作中的势项抑制了除了有限数量的常量映射之外的所有贡献,
\end{paracol}

\[ x(\tau) = X \text{s.t.} h'(X) = 0 \]

\begin{paracol}{2}
There are the critical points of $h$. The phenomenon of an integral - in this case an infinite dimensional functional integral - receiving contributions from just a handful of points is known as \textit{localisation}. It is a property of supersymmetric path integrals that is not shared by most other quantum systems.
\switchcolumn
$h$ 存在临界点。积分(在本例中为无限维泛函积分)仅从少数点接收贡献的现象称为\textit{局部化}。这是大多数其他量子系统所不具备的超对称路径积分的性质。
\switchcolumn*

Indeed, taking the $\lambda \to \infty$ limit should be viewed as suppressing the non-linear inter- actions in the potential. The statement that the Witten index is independent of $\lambda$ is equivalently to saying that the one-loop approximation is, in fact, exact.
\switchcolumn
事实上,采用 $\lambda \to \infty$ 限制应该被视为抑制势能中的非线性相互作用。Witten 指标独立于 $\lambda$ 的说法相当于说单圈近似实际上是精确的。
\switchcolumn*

\subsection{Instantons}
\switchcolumn
\subsection*{瞬子}
\switchcolumn*

Much of our story so far has revolved around understanding the structure of ground states in supersymmetric systems. A common theme - one familiar from other quantum mechanical models - is that the existence of multiple classical ground states does not necessarily mean that there are multiple quantum ground states.
\switchcolumn
到目前为止,我们的大部分故事都是围绕着理解超对称系统中的基态结构展开的。一个常见的主题(在其他量子力学模型中很常见)是多个经典基态的存在并不一定意味着存在多个量子基态。
\switchcolumn*

In this section, we develop a more hands-on understanding of how ground states are lifted. Once again, our tool of choice will be the path integral and, as we will see, this provides a particularly direct way to think about quantum tunnelling and related phenomena. We will explore how this works in some detail, first in ordinary quantum mechanical systems and then in those that exhibit supersymmetry.
\switchcolumn
在本节中,我们将更深入地了解基态是如何提升的。我们选择的工具将再次是路径积分,正如我们将看到的,这提供了一种特别直接的方式来思考量子隧道效应和相关现象。我们将首先在普通量子力学系统中,然后在表现出超对称性的系统中,详细探讨它是如何工作的。
\switchcolumn*

Tunnelling is particularly easy to understand from the path integral perspective. It arises from paths that start at one minima and end up at another. If the parameters in the potential are such that we can do a semi-classical analysis, then the amplitude for tunnelling is dominated by the classical paths that minimise $S_E$. There is a rather cute way of finding these paths.
\switchcolumn
从路径积分的角度来看,隧道特别容易理解。它产生于从一个最小值开始并以另一个最小值结束的路径。如果势能中的参数使得我们可以进行半经典分析,那么隧道效应的幅度将由最小化 $S_E$ 的经典路径主导。有一种相当可爱的方法可以找到这些路径。
\switchcolumn*

We write the action (2.18) by completing the square
\switchcolumn
我们通过配平方来写出作用量(2.18)
\switchcolumn*

If we fix the end points $x_i$ and $x_f$ to be two distinct minima, then the action is minimised when this inequality is saturated with the most stringent $\pm$ sign. This means that if $h(x_f) > h(x_i)$, we should solve the equation
\switchcolumn
如果我们将端点 $x_i$ 和 $x_f$ 固定为两个不同的最小值,那么当这个不等式用最严格的 $\pm$ 符号饱和时,动作就会最小化。这意味着如果 $h(x_f) > h(x_i)$,我们应该求解方程
\end{paracol}

\[ \odv{x}{\tau} = \odv{h}{x} \]

\begin{paracol}{2}
Solutions to this equation are known as \textit{instantons}. The name is chosen (by 't Hooft) to mimic the names give to particles but, as will see, these solutions are not localised in space but in (Euclidean) time and so occur just for an instant. If $h(x_f) < h(x_i)$, we should solve the other equation
\switchcolumn
该方程的解称为 \textit{瞬子}。这个名称(由 't Hooft 命名)选择是为了模仿粒子的名称,但是,正如我们将看到的,这些解决方案并不局限于空间,而是局限于(Euclidean)时间,因此只发生一瞬间。如果 $h(x_f) < h(x_i)$,我们应该解另一个方程
\end{paracol}

\[ \odv{x}{\tau} = - \odv{h}{x} \]

\begin{paracol}{2}
Solutions to this equations are called \textit{anti-instantons}. They interpolate between the two vacua in the opposite direction to instantons.
\switchcolumn
该方程的解称为\textit{反瞬子}。它们以与瞬子相反的方向在两个真空之间进行插值。
\switchcolumn*

For more general $h(x)$ the exact solution of the instanton may be harder to come by but its simple to get an intuitive feel for its properties. Viewed from the usual perspective of Lagrangian dynamics, the Euclidean action (2.18) describes a particle moving in a potential $- V(x)$. This is shown on the right-hand side of Figure 6 for the double well potential. The instanton (or anti-instanton) describes a particle that starts at one maximum of $- V(x)$ at $\tau \to - \infty$ and then rolls down and up to another maximum, reaching the peak only at $\tau \to + \infty$.
\switchcolumn
对于更一般的 $h(x)$ 来说,瞬子的精确解可能更难获得,但对其属性有一个直观的感受很简单。从拉格朗日动力学的通常角度来看,欧几里得作用(2.18)描述了一个粒子在势$-V(x)$中运动。这显示在图 6 右侧的双阱电势中。瞬子(或反瞬子)描述了一个粒子,它从 $- V(x)$ 的一个最大值开始,在 $\tau \to - \infty$ 处,然后向下和向上滚动到另一个最大值,仅在 $\tau \to - \infty$ 处达到峰值 $\tau \to + \infty$。
\switchcolumn*

If $V(x)$ has multiple minima, then we can only find solutions to the instanton equations (2.20) and (2.21) that interpolate between \textit{neighbouring} minima. This is because these are first order equations of motion, and once you sit at a critical point of $h$ you have necessarily stopped. That doesn’t mean that there is no tunnelling between multiple vacua: indeed, as we’ll see shortly, in non-supersymmetric quantum mechanics it is usually approximate solutions to the classical equations of motion that dominate proceedings.
\switchcolumn
如果 $V(x)$ 有多个最小值,那么我们只能找到在 \textit{邻近} 最小值之间插值的瞬子方程 (2.20) 和 (2.21) 的解。这是因为这些是一阶运动方程,一旦到达 $h$ 的临界点,您就必然会停止。这并不意味着多个真空之间不存在隧道效应:事实上,正如我们很快就会看到的,在非超对称量子力学中,通常是主导过程的经典运动方程的近似解。
\switchcolumn*

\subsubsection{Tunnelling}
\switchcolumn
\subsubsection*{隧穿}
\switchcolumn*

Let’s first remind ourselves what we qualitatively expect from the ground states. Around each minima, the potential looks like a harmonic oscillator (2.23) and we can then construct approximations to the ground states as Gaussian wavefunctions, localised around each of the minima
\switchcolumn
让我们首先提醒自己,我们对基态的定性期望是什么。在每个最小值周围,势看起来像一个谐振子 (2.23),然后我们可以将基态近似构造为高斯波函数,位于每个最小值周围
\switchcolumn*

For any even potential $V(x) = V(- x)$, the energy eigenstates are also eigenstates of the parity operator, meaning that they are either even or odd functions. A better approximation to the low-lying energy eigenstates must therefore be
\switchcolumn
对于任何偶势 $V(x) = V(- x)$,能量本征态也是奇偶算子的本征态,这意味着它们要么是偶函数,要么是奇函数。因此,对低位能量本征态更好的近似是
\switchcolumn*

But the true ground state of any quantum system has no node, meaning that $\psi(x) \neq 0$ for any finite $x$. (Given a wavefunction $\psi(x)$ with a node, we can consider $\ab|\psi(x)|$ and then smooth out the cusp to lower the expected energy.) So it must be that $\psi_{+}(x)$ is an approximation to the ground state, while $\psi_{-}(x)$ is an approximation to the first excited state.
\switchcolumn
但任何量子系统的真实基态都没有节点,这意味着对于任何有限的$x$,$\psi(x)\neq 0$。 (给定一个带有节点的波函数 $\psi(x)$,我们可以考虑 $\ab|\psi(x)|$,然后平滑尖点以降低预期能量。)所以它一定是 $\psi_ {+}(x)$ 是基态的近似值,而 $\psi_{-}(x)$ 是第一激发态的近似值。
\switchcolumn*

So to determine the ground state energy, we just need to compute the path integral and extract the large $T$ behaviour. We can then find the ground state energy in the exponent.
\switchcolumn
因此,为了确定基态能量,我们只需要计算路径积分并提取大 $T$ 行为。然后我们可以找到指数中的基态能量。
\switchcolumn*

It should be thought of as a measure of the difficulty in getting up and over (or, more precisely, through) the barrier between the two minima.
\switchcolumn
它应该被视为衡量站起来并越过(或更准确地说,穿过)两个最小值之间障碍的难度的指标。
\switchcolumn*

The semi-classical approximation is valid whenever we can ignore the $\mathcal{O}(\delta x^3)$ contributions relative to the $\delta x^2$ contributions in the path integral. To understand the circumstances under which this holds, we should look more closely at the action and identify a dimensionless coupling constant $g$ which multiplies all higher order terms. Perturbation theory is then valid when $g \ll 1$. A simpler way to view things is to rescale the potential $h(x) \to \lambda h(x)$. Then the semi-classical approximation is valid in the limit $\lambda \gg 1$ where we have a steep potential, with deep minima. Under this rescaling, the action of the instanton (2.26) becomes
\switchcolumn
只要我们可以忽略路径积分中相对于 $\delta x^2$ 贡献的 $\mathcal{O}(\delta x^3)$ 贡献,半经典近似就有效。为了理解这种情况成立的情况,我们应该更仔细地观察该作用,并确定一个无量纲耦合常数 $g$,它乘以所有高阶项。当 $g \ll 1$ 时,微扰理论有效。一种更简单的看待事物的方法是将势 $h(x) \ 重新调整为 \lambda h(x)$。那么半经典近似在极限 $\lambda \gg 1$ 中是有效的,其中我们有一个陡峭的潜力,具有深的最小值。在这种重新调整下,瞬子 (2.26) 的作用变为
\switchcolumn*

This is the requirement that we will use for the semi-classical approximation to be valid. The results that we will get below will receive corrections of order $1 / S_{\text{inst}}$.
\switchcolumn
这是我们要使半经典近似有效的要求。我们将在下面得到的结果将得到阶为 $1 / S_{\text{inst}}$ 的修正。
\switchcolumn*

In the language of quantum field theory, neglecting the higher order $\delta x^3$ terms is tantamount to computing one-loop diagrams but not two-loop or higher. In normal circumstances, we would be doing perturbation theory around the classical vacuum $x(\tau) = \pm a$, in which case we would have $V'' = \omega^2$, a constant. The difference here is that we’re now doing perturbation theory around the background of the instanton profile.
\switchcolumn
在量子场论的语言中,忽略高阶 $\delta x^3$ 项相当于计算单环图,但不能计算双环或更高阶。在正常情况下,我们将围绕经典真空 $x(\tau) = \pm a$ 进行微扰理论,在这种情况下,我们将有 $V'' = \omega^2$,一个常数。这里的区别在于,我们现在正在围绕瞬子轮廓的背景进行微扰理论。
\switchcolumn*

The kind of instanton calculations that we’re performing here are often referred to as \textit{non-perturbative}. This refers to the fact that tunnelling phenomena of this kind can’t be captured by perturbation theory around any single vacuum. However, the phrase "non-perturbative" is also a little misleading: we’re still doing perturbation theory, just around a non-trivial solution.
\switchcolumn
我们在这里执行的瞬子计算通常称为 \textit{非微扰}。这是指这样的事实:这种隧道现象无法通过任何单一真空周围的微扰理论来捕获。然而,“非微扰”这个词也有点误导:我们仍在研究微扰理论,只是围绕着一个不平凡的解。
\switchcolumn*

On the left-hand side, we've taken the tunnelling to happen over a time $T$; ultimately we will be interested in taking $T \to \infty$. We have also stressed that we’re computing the contribution to the tunnelling from a single instanton and we’ll subsequently see that this is just part of the story.
\switchcolumn
在左侧,我们假设隧道发生的时间为$T$;最终我们会对将 $T \to \infty$ 感兴趣。我们还强调,我们正在计算单个瞬子对隧道效应的贡献,随后我们将看到这只是故事的一部分。
\switchcolumn*

There is, however, a catch. In the background of the instanton, there is always one eigenvalue that is zero. Viewed naively, this would seem to tell us that the determinant vanishes, giving an infinite amplitude for tunnelling. This, it turns out, is not an infinity that we should try to regulate away, but instead an infinity that’s means we should think more carefully about what we’re calculating. Our first task, therefore, is to understand the physics behind this zero eigenvalue.
\switchcolumn
然而,有一个问题。在瞬子的背景中,总是有一个特征值为零。单纯地看,这似乎告诉我们行列式消失了,为隧道效应提供了无限的振幅。事实证明,这并不是一个我们应该试图调节掉的无穷大,而是一个意味着我们应该更仔细地思考我们正在计算的东西的无穷大。因此,我们的首要任务是理解零特征值背后的物理原理。
\switchcolumn*

Understanding zero modes is an important part of any instanton computation. They typically arise, as in the present case, because the instanton solution is not unique, but labelled by a number of parameters known as \textit{collective coordinates}. For us, the instanton profile has a single collective coordinate, $\tau_1$. Any fluctuation, like (2.29), that can be thought of as a variation of a collective coordinate necessarily has zero eigenvalue. These fluctuations are called \textit{zero modes}.
\switchcolumn
了解零模式是任何瞬子计算的重要组成部分。它们通常会出现,就像在本例中一样,因为瞬子解不是唯一的,而是由许多称为 \textit{集体坐标} 的参数标记。对于我们来说,瞬子轮廓有一个集体坐标,$\tau_1$。任何波动,如(2.29),可以被认为是集体坐标的变化,必然具有零特征值。这些波动称为 \textit{零模式}。
\switchcolumn*

In the present case, the existence of the zero mode can be traced to the fact that the underling quantum mechanics enjoys time translation symmetry, while any particular instanton profile does not. In quantum field theory (or statistical field theory), we would refer to the zero mode as a "Goldstone boson" for time translational symmetry.
\switchcolumn
在目前的情况下,零模式的存在可以追溯到以下事实:基础量子力学具有时间平移对称性,而任何特定的瞬子解则不然。在量子场论(或统计场论)中,我们将零模式称为时间平移对称性的““Goldstone 玻色子”。
\switchcolumn*

There are three things to take away from this. First, there are some slightly messy pre-factors that we’ve absorbed into $K$, which now include a ratio of the harmonic oscillator and instanton determinants. The exact expression for this ratio will not be particularly important in what follows and we won’t make any attempt to compute it. However, the advantage of writing this as a ratio of determinants is that it makes it clear that it differs from $1$ only due to physics in a region of width $1 / \omega$ where the instanton profile is non-trivial, and $V''(x_{\text{inst}})$ differs from $\omega^2$. We’ll see the utility of this shortly.
\switchcolumn
从中可以得出三件事。首先,我们将一些稍微混乱的前置因素吸收到 $K$ 中,其中现在包括谐振子和瞬子决定因素的比率。这个比率的确切表达式在接下来的内容中不会特别重要,我们不会尝试计算它。然而,将其写为行列式之比的优点在于,它清楚地表明它与 $1$ 的不同仅是由于宽度为 $1 / omega$ 的区域中的物理现象,其中瞬时子解是非平凡的,并且 $V ''(x_{\text{inst}})$ 与 $\omega^2$ 不同。我们很快就会看到它的用处。
\switchcolumn*

Second, the amplitude is suppressed by a factor of $\rme^{- S_{\text{inst}}}$ . This is a characteristic feature of tunnelling in quantum mechanics. Finally, we see that the tunnelling amplitude from a single instanton has the slightly odd $T \rme^{- T}$ behaviour. It turns out that the correct interpretation of this comes by considering not a lone instanton, but a whole slew of them.
\switchcolumn
其次,振幅被 $\rme^{- S_{\text{inst}}}$ 因子抑制。这是量子力学中隧道效应的一个特征。最后,我们看到单个瞬子的隧道振幅具有稍微奇怪的 $T \rme^{- T}$ 行为。事实证明,对此的正确解释不是考虑单个瞬子,而是考虑整个瞬子。
\switchcolumn*

\subsubsection{The Dilute Gas Approximation}
\switchcolumn
\subsubsection*{稀薄气体近似}
\switchcolumn*

In the calculation above, we restricted to a single instanton solution that interpolates from one classical ground state to the other. However, we know that the interesting part of this instanton profile takes place over a region that is exponentially localised within a width $\sim 1 / \omega$. That means that if we take an instanton, followed a long time later, by an anti-instanton, followed some time later still by another instanton, then this \textit{almost} solves the classical equation of motion. It’s not an exact solution because there are no exact classical solution with these properties. But, if the instantons and anti-instantons are separated by a distance $L \gg 1 / \omega$, then the action of a string of $n$ such objects is roughly
\switchcolumn
在上面的计算中,我们限制为从一个经典基态插值到另一个经典基态的单个瞬子解。然而,我们知道这个瞬子分布的有趣部分发生在宽度 $\sim 1 / \omega$ 内呈指数局部化的区域。这意味着,如果我们采用一个瞬子,很长一段时间后是一个反瞬子,再过一段时间又是另一个瞬子,那么这个 \textit{几乎} 就是经典运动方程的解。这不是一个精确解,因为没有具有这些性质的精确的经典解。但是,如果瞬子和反瞬子之间的距离为 $L \gg 1 / \omega$,那么一串 $n$ 这样的物体的作用大致为
\switchcolumn*

In computing the amplitude $\braket[3]{- a}{\rme^{- H T}}{a}$, we should sum over all possible numbers of instantons and anti-instantons. We just need one more instanton than anti-instanton to ensure that we end up in the opposite vacuum from where we started. In other words, $n$ should be odd in (2.34).
\switchcolumn
在计算幅度 $\braket[3]{- a}{\rme^{- H T}}{a}$ 时,我们应该对所有可能的瞬子和反瞬子数量进行求和。我们只需要比反瞬子多一个瞬子,以确保我们最终处于与开始位置相反的真空中。换句话说,(2.34) 中的 $n$ 应该是奇数。
\switchcolumn*

We see the effect of summing over the dilute gas is to exponentiate the one-instanton contribution $K T \rme^{- S_{\text{inst}}}$.
\switchcolumn
我们看到对稀气体求和的效果是对单瞬时贡献 $K T \rme^{- S_{\text{inst}}}$ 求幂。
\switchcolumn*

We can also do a similar calculation to evaluate the amplitude $\braket[3]{+ a}{\rme^{- H T}}{+ a} = \braket[3]{- a}{\rme^{- H T}}{- a}$ for returning to our original vacuum. Everything is the same, except that we should now take the number $n$ of instantons and anti-instantons to be even. Of course, $n = 0$ is allowed.
\switchcolumn
我们也可以进行类似的计算来评估幅度 $\braket[3]{+ a}{\rme^{- H T}}{+ a} = \braket[3]{- a}{\rme^{- H T}}{- a}$ 返回我们原来的真空。一切都是一样的,只是我们现在应该将瞬子和反瞬子的数量 $n$ 取为偶数。当然,$n = 0$ 是允许的。
\switchcolumn*

We see the promised energy splitting, proportional to the characteristic tunnelling amplitude $\rme^{- S_{\text{inst}}}$.
\switchcolumn
我们看到所承诺的能量分裂,与特征隧道振幅 $\rme^{- S_{\text{inst}}}$ 成正比。
\switchcolumn*

Of course, if we really want to do a good job then we should roll up our sleeves and compute the ratio of determinants that sits in $K$. But we can see the key piece of physics without doing this: the splitting of energy levels scales as $\rme^{- S_{\text{inst}}}$.
\switchcolumn
当然,如果我们真的想做好工作,那么我们应该卷起袖子计算 $K$ 中的决定因素的比率。但我们无需这样做就可以看到物理的关键部分:能级分裂的尺度为 $\rme^{- S_{\text{inst}}}$。
\switchcolumn*

\subsection{Instantons and Supersymmetry}
\switchcolumn
\subsection*{瞬子和超对称}
\switchcolumn*

It’s now time to return to supersymmetric quantum mechanics. It turns out that there is a deep relationship between instantons and supersymmetry, both in quantum mechanics and in higher dimensional quantum field theories. The two make for perfect bedfellows. In this section, we will start to get a hint of where this relationship emerges from. We’ll also see that the existence of fermions brings some important technical differences to the tunnelling calculation that we did in the last section.
\switchcolumn
现在是时候回到超对称量子力学了。事实证明,无论是在量子力学还是在高维量子场理论中,瞬子和超对称性之间都存在着深刻的关系。两人堪称完美的同床异梦。在本节中,我们将开始了解这种关系从何而来。我们还将看到费米子的存在给我们在上一节中进行的隧道计算带来了一些重要的技术差异。
\switchcolumn*

We need to briefly pause to think about what this determinant means because, in contrast to the bosonic fluctuations (2.27), it’s not the determinant of a Hermitian operator. The operator used to be Hermitian, back when we were living in real time, where it was $\ab(+ \rmi \odv{}/{t} - h'')$. But the Wick rotation ruined that property.
\switchcolumn
我们需要短暂地停下来思考一下这个行列式的含义,因为与玻色子涨落(2.27)相比,它不是厄米算子的行列式。当我们生活在实时间时,该运算符曾经是 Hermitian 算符,即 $\ab(+ \rmi \odv{}/{t} - h'')$。但 Wick 旋转破坏了这一特性。
\switchcolumn*

\subsubsection{Fermi Zero Modes}
\switchcolumn
\subsubsection*{Fermi 零模}
\switchcolumn*

We could have anticipated the existence of this fermionic zero mode on symmetry grounds. Recall that we could trace the bosonic collective coordinate $\tau_1$ to time translational symmetry since, while the action is invariant under time translations, any given instanton profile is not. Similarly, the fermionic collective coordinate can be traced to a fermionic symmetry which is, of course, supersymmetry. If we look again at the transformation rules (2.14) for the fermions in Euclidean time, then we see something rather nice:
\switchcolumn
我们可以基于对称性预期这种费米子零模式的存在。回想一下,我们可以将玻色子集体坐标 $\tau_1$ 追踪到时间平移对称性,因为虽然作用在时间平移下是不变的,但任何给定的瞬子轮廓却不是。类似地,费米子集体坐标可以追溯到费米子对称性,这当然是超对称性。如果我们再次查看 Euclidean 时间中费米子的变换规则(2.14),那么我们会看到一些相当不错的东西:
\end{paracol}

\[ \delta \psi = \epsilon \ab(\odv{x}{\tau} + h'), \delta \psi^{\dagger} = \epsilon^{\dagger} \ab(- \odv{x}{\tau} + h') \]

\begin{paracol}{2}
The supersymmetry transformations have, hidden within them, the instanton and antiinstanton equations (2.20) and (2.21)! This, it turns out, is a beautiful feature of supersymmetry, and one that persists as we look both to more complicated theories and to more complicated instantons and other solitons. For now, we note that if we take an instanton obeying $\dot{x} = h'$ and hit it with a supersymmetry transformation, then $\psi$ will turn on while $\psi^{\dagger}$ will not. But, because supersymmetry is a symmetry, the action of the solution doesn’t change when $\psi$ turns on. This is the fermi zero mode (2.38) that we identified above.
\switchcolumn
超对称变换中隐藏着瞬时方程和反瞬时方程(2.20)和(2.21)!事实证明,这是超对称性的一个美丽特征,并且当我们研究更复杂的理论以及更复杂的瞬子和其他孤子时,这一特征仍然存在。现在,我们注意到,如果我们采用一个服从 $\dot{x} = h'$ 的瞬子,并用超对称变换击中它,那么 $\psi$ 将打开,而 $\psi^{\dagger}$ 将不会打开。但是,由于超对称性是一种对称性,因此当 $\psi$ 打开时,解的作用量不会改变。这就是我们上面确定的费米零模式 (2.38)。
\switchcolumn*

You might be nervous that we seem to have broken reality. In the background of an instanton, the fermion $\psi$ has a zero mode, but $\psi^{\dagger}$ does not. Indeed, the equation of motion for $\psi^{\dagger}$ is $D^{\dagger} \psi^{\dagger} = 0$ and $D^{\dagger}$ has no zero mode in the background of an instanton. Conversely, in the background of an anti-instanton $\psi^{\dagger}$ has a zero mode, while $\psi$ has none. This issue is commonplace for fermions in Euclidean time (or, more generally, in Euclidean space) and arises because D is not Hermitian. It’s best to think of $\psi$ and $\psi^{\dagger}$ as independent degrees of freedom in Euclidean time. Only when we Wick rotate back to real time (or Minkowski space) do the reality and Hermiticity properties of various operators manifest themselves again.
\switchcolumn
你可能会感到紧张,因为我们似乎打破了实性。在瞬子的背景下,费米子 $\psi$ 具有零模式,但 $\psi^{\dagger}$ 没有。事实上,$\psi^{\dagger}$ 的运动方程为 $D^{\dagger} \psi^{\dagger} = 0$ 并且 $D^{\dagger}$ 在背景中没有零模式瞬子。相反,在反瞬子的背景下,$\psi^{\dagger}$ 具有零模式,而 $\psi$ 则没有。这个问题对于 Euclidean 时间(或者更一般地说,在 Euclidean 空间)的费米子来说很常见,并且出现的原因是 D 不是厄米的。最好将 $\psi$ 和 $\psi^{\dagger}$ 视为 Euclidean 时间中的独立自由度。只有当我们 Wick 旋转回实时间(或 Minkowski 空间)时,各种算子的实性和厄米性才会再次显现出来。
\switchcolumn*

The upshot is that the instanton breaks one half of the supersymmetries: $Q^{\dagger}$ is broken and generates a fermionic zero mode, while $Q$ survives. Objects, like instantons, which have the property of preserving some fraction of the supersymmetry are known as \textit{BPS}. (The initials stand for Bogomolnyi, Prasad and Sommerfeld, but what they actually did was only vaguely related to supersymmetry and the meaning of the initials BPS has evolved over the years.)
\switchcolumn
结果是瞬子破坏了一半的超对称性:$Q^{\dagger}$ 被破坏并生成费米子零模式,而 $Q$ 幸存下来。物体,比如瞬子,具有保留部分超对称性的特性,被称为 \textit{BPS}。 (缩写代表 Bogomolnyi、Prasad 和 Sommerfeld,但他们实际上所做的只是与超对称性模糊相关,而且缩写 BPS 的含义多年来一直在演变。)
\switchcolumn*

Although $Q$ is unbroken, it is not totally redundant. It actually relates the collective coordinates $\tau_1$ and $\eta$ of the instanton, a kind of "zero dimensional" supersymmetry. This interpretation won’t be important for now.
\switchcolumn
尽管 $Q$ 没有被破坏,但它也不是完全冗余的。它实际上关联了瞬子的集体坐标 $\tau_1$ 和 $\eta$,这是一种“零维”超对称性。这种解释现在并不重要。
\switchcolumn*

Let’s now return to our tunnelling computation. We know what to do: rather than integrate over all fermion modes, we isolate the zero mode and treat it separately, choosing instead to integrate over the fermionic collective coordinate $\eta$. As before, we pick up a Jacobian factor which, because the fermion zero mode (2.38) has the same functional form as the bosonic zero mode (2.29), is the same value $J = \sqrt{S_{\text{inst}}}$ that we computed in (2.31). But Jacobians for Grassmann integration come as $1 / J$, rather than $J$ so this actually cancels our original bosonic contribution.
\switchcolumn
现在让我们回到隧道计算。我们知道该怎么做:我们不是对所有费米子模式进行积分,而是隔离零模式并单独处理它,选择对费米子集体坐标 $\eta$ 进行积分。和之前一样,我们选取​ Jacobian 因子,因为费米子零模 (2.38) 与玻色子零模 (2.29) 具有相同的函数形式,所以它的值相同 $J = \sqrt{S_{\text{inst} }}$ 我们在 (2.31) 中计算。但是 Grassmann 积分的 Jacobian 行列式是 $1 / J$,而不是 $J$,所以这实际上取消了我们最初的玻色子贡献。
\switchcolumn*

We now have a ratio of determinants, both with zero eigenvalues omitted. There is no need to do our previous trick of introducing the harmonic oscillator amplitude (2.32). Indeed, part of the reason for doing that previously was to make manifest the $\frac{1}{2} \hbar \omega$ ground state energy but, as we’ve seen, the analogous semi-classical energy in supersymmetric quantum mechanics is exactly zero.
\switchcolumn
现在我们有了行列式的比率,两者都省略了零特征值。无需执行我们之前引入谐振子振幅(2.32)的技巧。事实上,之前这样做的部分原因是为了显示 $\frac{1}{2} \hbar \omega$ 基态能量,但是,正如我们所看到的,超对称量子力学中类似的半经典能量正好为零。
\switchcolumn*
\subsubsection{Computing Determinants}
\switchcolumn
\subsubsection*{计算行列式}
\switchcolumn*

In non-supersymmetric theories, it can be very challenging to compute the determinants in the background of an instanton. In contrast, in supersymmetric theories it is trivial because the ratio of determinants precisely cancels!
\switchcolumn
在非超对称理论中,计算瞬子背景下的行列式可能非常具有挑战性。相反,在超对称理论中,这是微不足道的,因为行列式的比率恰好抵消了!
\switchcolumn*

This cancellation is entirely analogous to our previous observation that the ground state energy in a supersymmetric vacuum is zero, since the $+ \frac{1}{2} \hbar \omega$ from the harmonic oscillator is precisely cancelled by a $- \frac{1}{2} \hbar \omega$ from the fermions. Here we see a similar cancellation persists about a BPS instanton configuration. This is a lesson that also transfers to higher dimensional quantum field theories, where it is often the case that all perturbative contributions cancel between bosons and fermions when evaluated about BPS backgrounds.
\switchcolumn
这种抵消完全类似于我们之前的观察,即超对称真空中的基态能量为零,因为来自谐振子的 $+ \frac{1}{2} \hbar \omega$ 被来自费米子的 $- \frac{1}{2} \hbar \omega$。在这里,我们看到 BPS 瞬子配置仍然存在类似的取消。这个教训也适用于更高维的量子场理论,在这种情况下,当评估 BPS 背景时,玻色子和费米子之间的所有微扰贡献通常都会抵消。
\switchcolumn*

In fact, this is to be expected given our earlier discussion of supersymmetric quantum mechanics. From Section 1.2, we know that the semi-classical ground states $\ket{- a}$ and $\ket{+ a}$ lie in different spin sectors or, equivalently, in different components of the Hilbert space factorisation $\mathcal{H} = \mathcal{H}_B \oplus \mathcal{H}_F$. This means that there can be no tunnelling from one state to another and the path integral realises this by introducing a lone fermion zero mode.
\switchcolumn
事实上,考虑到我们之前对超对称量子力学的讨论,这是可以预料到的。从第 1.2 节,我们知道半经典基态 $\ket{- a}$ 和 $\ket{+ a}$ 位于不同的自旋扇区,或者等效地,位于 Hilbert 空间分解 $\mathcal{H} = \mathcal{H}_B \oplus \mathcal{H}_F$ 的不同分量中。这意味着不能存在从一种状态到另一种状态的隧道效应,并且路径积分通过引入孤费米子零模式来实现这一点。
\switchcolumn*

\subsubsection{Computing the Ground State Energy}
\switchcolumn
\subsubsection*{计算基态能量}
\switchcolumn*

The Hamiltonian analysis of Section 1.2 told us more about this system. We know, for example, that the two states localised in different minima remain true ground states of the system but their energy is lifted above zero (i.e. supersymmetry is broken for a cubic $h(x)$). It is possible to see this from the path integral. We just need a small tweak of our previous analysis.
\switchcolumn
1.2 节的哈密顿分析告诉我们更多关于这个系统的信息。例如,我们知道,位于不同最小值的两个状态仍然是系统的真实基态,但它们的能量被提升到零以上(即三次 $h(x)$ 的超对称性被打破)。从路径积分中可以看出这一点。我们只需要对之前的分析进行一些小小的调整。
\switchcolumn*

{\footnotesize This sidesteps an annoying subtlety. If you compute the matrix element for $\braket[3]{L}{Q}{R}$ directly then, at leading order, the result will vanish. This is because, after Wick rotation to Euclidean time, $Q$ is proportional to the instanton equations and so vanishes when evaluated on the instanton. (This follows from the fact that the supersymmetry transformation (2.39) is proportional to the instanton equation.) You then have to work to higher order to find the non-vanishing ground state energy. Computing the matrix element of $[Q, h]$ avoids this headache.
\switchcolumn
这回避了一个恼人的微妙之处。如果直接计算 $\braket[3]{L}{Q}{R}$ 的矩阵元素,则按前导顺序,结果将消失。这是因为,在威克旋转到欧几里德时间之后,$Q$ 与瞬子方程成正比,因此在对瞬子求值时消失。(这是根据超对称变换 (2.39) 与瞬子方程成正比这一事实得出的。)然后,您必须工作到更高阶才能找到不消失的基态能量。计算 $[Q, h]$ 的矩阵元素可以避免这个令人头疼的问题。
\switchcolumn*}

There’s another lesson lurking in the calculation above. To compute the energy $E_0$, we didn’t need to invoke the dilute gas approximation; it was sufficient to look at a single instanton. Indeed, viewed the right way it was \textit{necessary} to look at just a single instanton. This is because the single instanton is BPS, meaning that it is invariant under one-half of the supersymmetries, and therefore has just a single fermion zero mode. However, a string of instanton-anti-instanton pairs does not have this property: it breaks both $Q$ and $Q^{\dagger}$ and therefore has two fermion zero modes, rather than just one. This is a special property of BPS instantons in supersymmetric theories that is closely related to the localisation of the path integral that we saw previously.
\switchcolumn
上面的计算中还隐藏着另一个教训。为了计算能量 $E_0$,我们不需要调用稀气体近似;观察一个瞬子就足够了。事实上,以正确的方式来看,\textit{有必要}只看一个瞬子。这是因为单个瞬子是 BPS,这意味着它在一半超对称性下是不变的,因此只有一个费米子零模式。然而,一串瞬子-反瞬子对不具有此属性:它同时破坏 $Q$ 和 $Q^{\dagger}$,因此具有两个费米子零模式,而不仅仅是一个。这是超对称理论中 BPS 瞬子的一个特殊性质,与我们之前看到的路径积分的局域化密切相关。
\switchcolumn*

We’ll revisit instanton calculations of this kind in Section 3.2 where we discuss Morse theory. It will turn out that these kind of calculations underlie many of the key ideas in that context.
\switchcolumn
我们将在第 3.2 节中讨论莫尔斯理论,重新审视这种瞬子计算。事实证明,此类计算是该背景下许多关键思想的基础。
\switchcolumn*

\subsubsection{One Last Example: A Particle on a Circle}
\switchcolumn
\subsubsection*{最后一个例子:圆上的粒子}
\switchcolumn*

We briefly discussed this model in Section 1.2.2 where we showed that, despite its similarities to the double well potential, it actually has two zero energy ground states, given by $\rme^{+ h} \ket{0}$ and $\rme^{- h} \psi^{\dagger} \ket{0}$. The puzzle that we’d like to address here is: why aren’t these states lifted from the perspective of the path integral?
\switchcolumn
我们在第 1.2.2 节中简要讨论了这个模型,其中我们表明,尽管它与双阱势相似,但它实际上具有两个零能量基态,由 $\rme^{+ h} \ket{0}$ 和$\rme^{- h} \psi^{\dagger} \ket{0}$。我们在这里要解决的难题是:为什么不从路径积分的角度解除这些状态?
\switchcolumn*

The novelty is that we have two different instanton solutions in this case, corresponding to the two different ways to go around the circle. The first instanton has $\dot{x} > 0$, the second $\dot{x} < 0$.
\switchcolumn
新颖之处在于,在这种情况下我们有两种不同的瞬子解决方案,对应于两种不同的绕圈方式。第一个瞬子 $\dot{x} > 0$,第二个 $\dot{x} < 0$。
\switchcolumn*

So it’s clear what the solution to our puzzle must be. These two instantons must contribute with opposite signs, so that they cancel out in the matrix element
\switchcolumn
所以我们的难题的解决方案一定是什么就很清楚了。这两个瞬子必须具有相反的符号,以便它们在矩阵元素中抵消
\end{paracol}

\[ \braket[3]{+ \frac{\pi R}{2}}{Q}{- \frac{\pi R}{2}} \]

\begin{paracol}{2}
that we care about, leaving the energy of both states at zero. The question is: how does this minus sign arise in the computation?
\switchcolumn
我们关心的,使两种状态的能量都为零。问题是:这个负号在计算中是如何出现的?
\switchcolumn*

This, it turns out is subtle. A rerun of the calculation above shows that there’s nowhere obvious that this sign could appear. The non-obvious place is, it turns out, in the definition of the determinants. The cancellation that we derived in Section 2.3.2 is really
\switchcolumn
事实证明,这一点很微妙。重新运行上面的计算表明,没有明显的地方可以出现这个迹象。事实证明,不明显的地方在于行列式的定义。我们在第 2.3.2 节中导出的取消实际上是
\end{paracol}

\[ \frac{\det D}{\sqrt{\det D^{\dagger} D}} = \pm 1 \]

\begin{paracol}{2}
Figuring out which sign we get is not so straightforward. For now, we’ll content ourselves with the observation that, by answer analysis, the signs must be opposite for the two instantons that traverse the circle in different directions. We’ll give a prescription for computing this sign in Section 3.2 when we discuss Morse theory.
\switchcolumn
弄清楚我们得到的是哪个信号并不那么简单。现在,我们将满足于这样的观察:通过答案分析,对于以不同方向穿过圆的两个瞬子来说,符号必须相反。当我们讨论 Morse 理论时,我们将在 3.2 节中给出计算这个符号的处方。
\switchcolumn*

\section{Supersymmetry and Geometry}
\switchcolumn
\section*{超对称和几何}
\switchcolumn*

In this section, we will begin our journey into the territory of mathematicians. Our strategy is to think about the physics of a particle moving on a manifold. As this section progresses, we will learn that the quantum ground states of this particle encode some precious information about the manifold.
\switchcolumn
在本节中,我们将开始进入数学家的领域。我们的策略是考虑粒子在流形上运动的物理原理。随着本节的进展,我们将了解到该粒子的量子基态编码了有关流形的一些宝贵信息。
\switchcolumn*

Lagrangians of the form (3.1) are commonplace in physics, both in quantum mechan- ics and in higher dimensional quantum field theories. They often go by the unhelpful name of a \textit{sigma model}. Sometimes they are called non-linear sigma models to reflect the fact that, unless $g_{ij}$ is constant, the equations of motion will be non-linear. The name "sigma model" is utterly unilluminating; it dates from one of the first such models written down by Gell-Mann and Levy to describe the dynamics of mesons. (Somewhat comically, Gell-Mann and Levy were building on an earlier model that described both pions and an extra meson known as the "sigma". They then wrote down an improved model that described just the mesons but chose to name it after the missing particle. And the name stuck.)
\switchcolumn
(3.1) 形式的 Lagrangian 在物理学中很常见,无论是在量子力学还是在高维量子场理论中。它们通常使用无用的名称 \textit{sigma 模型}。有时它们被称为非线性 sigma 模型,以反映这样一个事实:除非 $g_{ij}$ 恒定,否则运动方程将是非线性的。“sigma 模型”这个名字完全没有启发性。它可以追溯到 Gell-Mann 和 Levy 为描述介子动力学而编写的第一个此类模型。(有点滑稽的是,Gell-Mann 和 Levy 正在建立一个早期模型,该模型描述了介子和一个称为“sigma”的额外介子。然后他们写下了一个改进的模型,该模型仅描述了介子,但选择以缺失的名称命名它粒子。这个名字就这样被保留下来了。)
\switchcolumn*

The manifold $M$ is known as the target space. For much of what we do below, the story will be simplest if $M$ is a compact, orientable manifold and we’ll assume this to be the case in what follows.
\switchcolumn
流形 $M$ 称为目标空间。对于我们下面所做的大部分工作,如果 $M$ 是一个紧的可定向流形,那么故事将是最简单的,我们将假设接下来的情况就是这种情况。
\switchcolumn*

Strictly speaking, the metric $g_{ij}(x)$ in the Lagrangian should be viewed as the pull back of the metric from $M$ to $W$. As we saw in earlier courses covering differential geometry, strictly speaking the sigma model only describes the particle in a patch of the manifold $M$ that is covered by the coordinates $x^i$. One might think that to understand more subtle topological issues, we should be willing to consider overlapping patches. Perhaps surprisingly, it will turn out that this is not necessary, at least in these lectures.
\switchcolumn
严格来说,Lagrangian 中的度规 $g_{ij}(x)$ 应被视为度规从 $M$ 到 $W$ 的拉回。正如我们在之前介绍微分几何的课程中看到的那样,严格来说,sigma 模型仅描述由坐标 $x^i$ 覆盖的流形 $M$ 的补丁中的粒子。人们可能会认为,为了理解更微妙的拓扑问题,我们应该愿意考虑重叠的补丁。也许令人惊讶的是,事实证明这是不必要的,至少在这个讲义中是这样。
\switchcolumn*

There is certainly a lot of interesting physics in the geodesic equation. But it's challenging to extract any interesting mathematical statements about the manifold $M$ from knowledge of these geodesics. In particular, at any given time, the particle knows only about its immediate surrounding, yet any point looks much the same as any other locally. This means that the state of the particle cannot know anything about the global properties of the manifold. To extract any such information, we would need to know about the entire history of the particle.
\switchcolumn
测地线方程中肯定有很多有趣的物理原理。但从这些测地线的知识中提取关于流形 $M$ 的任何有趣的数学陈述是具有挑战性的。特别是,在任何给定时间,粒子只知道其周围环境,但任何点看起来都与本地任何其他点非常相似。这意味着粒子的状态无法了解流形的全局属性。为了提取任何此类信息,我们需要了解粒子的整个历史。
\switchcolumn*

This can be contrasted with the situation in quantum mechanics. Now the wavefunction spreads over the manifold $M$, which suggests that the state of the particle may well know about some of the manifold’s quirks. In particular, the state of a quantum particle may be sensitive to the topology of $M$. Ultimately, we will see that this is indeed the case, at least when we consider supersymmetric extension of our theory. But, for now, let’s push on can consider the quantum theory associated to the non-supersymmetric Lagrangian (3.1).
\switchcolumn
这可以与量子力学中的情况形成对比。现在波函数分布在流形 $M$ 上,这表明粒子的状态可能很好地了解流形的一些奇怪之处。特别是,量子粒子的状态可能对 $M$ 的拓扑敏感。最终,我们将看到情况确实如此,至少当我们考虑理论的超对称扩展时。但是,现在,让我们继续考虑与非超对称 Lagrangian (3.1) 相关的量子理论。
\switchcolumn*

Already here, things are not so straightforward because the metric $g_{ij}$ depends on $x^i$ and these don’t commute with $p_i$. Different choices of ordering give different quantum Hamiltonians and so different theories.
\switchcolumn
在这里,事情已经不是那么简单了,因为度规 $g_{ij}$ 取决于 $x^i$ 并且他们不与 $p_i$ 对易。不同的排序选择会产生不同的量子 Hamiltonian,从而产生不同的理论。
\switchcolumn*

There is no right or wrong choice here. But we can narrow down our options by requiring that the resulting theory has certain desirable properties. Given that we’re interested in the geometry of $M$, it makes sense to search for a Hamiltonian that is covariant with respect to changes of coordinates on $M$. In other words, to stick as closely as possible to differential geometry. The action of the momentum on the wavefunction is, as usual, $p_i = - \rmi \partial_i$, so the Hamiltonian should be a second order differential operator with terms that involve no more than two derivatives acting on the metric.
\switchcolumn
这里没有正确或错误的选择。但我们可以通过要求所得理论具有某些理想的特性来缩小我们的选择范围。鉴于我们对 $M$ 的几何形状感兴趣,搜索与 $M$ 上的坐标变化协变的 Hamiltonian 是有意义的。换句话说,尽可能地遵循微分几何。像往常一样,动量对波函数的作用是 $p_i = - \rmi \partial_i$,因此 Hamiltonian 应该是二阶微分算子,其项涉及不超过两个作用于度规的导数。
\switchcolumn*

We should also decide what Hilbert space we want our operators to act on. The obvious choice is to take the wavefunctions $\psi(x)$ as functions over $M$, with the norm given by
\switchcolumn
我们还应该决定我们希望算符作用在哪个 Hilbert 空间上。显而易见的选择是将波函数 $\psi(x)$ 作为 $M$ 上的函数,范数为
\switchcolumn*

Now we have our Hamiltonian (3.4) describing a quantum particle roaming around on a manifold $M$. What do we do with it? As physicists, our natural inclination is to find the spectrum of the Hamiltonian. We would typically expect that the particle has a unique ground state, with an infinite tower of excited states. This prompts two interesting questions: first, is it possible to calculate this spectrum? Second, what can we do with this information?
\switchcolumn
现在我们有了 Hamiltonian (3.4) 来描述在流形 $M$ 上漫游的量子粒子。我们用它做什么?作为物理学家,我们的自然倾向是找到 Hamiltonian 的谱。我们通常期望粒子具有独特的基态,具有无限的激发态塔。这就提出了两个有趣的问题:首先,是否可以计算这个能谱?其次,我们可以用这些信息做什么?
\switchcolumn*

Both of these questions are interesting, although neither is easy. In general, it is a difficult problem to determine the spectrum of the Hamiltonian (3.4). Which properties of the manifold can be reconstructed from this spectrum is reminiscent of the famous question "can you hear the shape of a drum?". Mathematicians have spent much time on this question. It is known, for example, that two manifolds may have the same spectrum even though they are not isometric. The first examples are 16-dimensional tori, but subsequent examples have been found in any dimension $n \geq 2$. In fact, it’s known that two manifolds may share the same spectrum even if they have different topology (e.g. their fundamental group may be different). All of which is to say that the problem of a quantum particle moving on a manifold $M$ is certainly interesting, but thinking as a physicist provides no particular advantage. We will now see that this situation changes (for the better!) when we introduce supersymmetry.
\switchcolumn
这两个问题都很有趣,但都不容易。一般来说,确定 Hamiltonian(3.4)的谱是一个困难的问题。可以从该频谱中重建流形的哪些属性,这让人想起著名的问题“你能听到鼓的形状吗?”。数学家们在这个问题上花费了很多时间。例如,众所周知,两个流形可能具有相同的频谱,即使它们不是同构的。第一个例子是 16 维环面,但随后的例子已在任何 $n \geq 2$ 维度中找到。事实上,众所周知,两个流形可能共享相同的谱,即使它们具有不同的拓扑(例如,它们的基本群可能不同)。所有这些都表明,量子粒子在流形 $M$ 上运动的问题当然很有趣,但作为物理学家的思考并没有提供特别的优势。现在,当我们引入超对称性时,我们将看到这种情况发生了变化(变得更好!)。
\switchcolumn*

\subsection{The Supersymmetric Sigma Model}
\switchcolumn
\subsection*{超对称 Sigma 模型}
\switchcolumn*

\subsubsection{How to Show that the Sigma Model is Supersymmetric}
\switchcolumn
\subsubsection*{如何证明 Sigma 模型是超对称的}
\switchcolumn*

Conceptually, it’s straightforward to demonstrate the supersymmetricness of the sigma model: you just vary the action, use the transformations (3.7), and show that it vanishes. In practice, you end up with a tsunami of terms. Here’s some help to guide you along the way.
\switchcolumn
从概念上讲,证明 sigma 模型的超对称性很简单:您只需改变作用量,使用变换(3.7),然后证明它消失了。在实践中,你最终会遇到海啸般多的各种项。这里有一些可以指导您前进的帮助。
\switchcolumn*

First, when implementing the supersymmetry transformation it’s useful to set $\epsilon = 0$ and just keep the $\epsilon^{\dagger}$ terms in the variation. There’s no subtlety here: it’s just means that we only have to keep track of half the terms in the variation. The other half are then fixed by ultimately requiring that the action and its variation are real. In particular, setting $\epsilon$ means that we have $\delta \psi = 0$ while $\delta \psi^{\dagger} \neq 0$.
\switchcolumn
首先,在实现超对称变换时,设置 $\epsilon = 0$ 并在变分中保留 $\epsilon^{\dagger}$ 项非常有用。这里没有什么微妙之处:它只是意味着我们只需要跟踪变体中的一半项。然后通过最终要求动作及其变化是真实的来固定另一半。特别是,设置 $\epsilon$ 意味着我们有 $\delta \psi = 0$ 而 $\delta \psi^{\dagger} \neq 0$。
\switchcolumn*

Second, there’s a familiar trick, described in the lectures on Quantum Field Theory, that is used to compute the conserved charges associated to any symmetry: we do local variations, instead of global variations. To this end we promote $\epsilon^{\dagger} \to \epsilon^{\dagger}(t)$. We will then find the supercharges multiplying the $\dot{\epsilon}^{\dagger}$ terms in the variation of the action.
\switchcolumn
其次,量子场论讲座中描述了一个熟悉的技巧,用于计算与任何对称性相关的守恒荷:我们进行局部变化,而不是全局变化。为此,我们提升 $\epsilon^{\dagger} \to \epsilon^{\dagger}(t)$。然后我们会在作用量变分中发现超荷乘以 $\dot{\epsilon}^{\dagger}$ 项。
\switchcolumn*

There are two different kinds of terms in this expression. The first take the form $\psi^{\dagger} \psi \dot{\psi}$. Gathering them together, we find that they come multiplying $\nabla g = 0$. The second take the form $\dot{x} \psi^{\dagger} \psi \psi$. The first of these involve combinations of the connection that gather together to give $\partial \Gamma + \Gamma^2$. But this is the definition of the Riemann tensor and is cancelled by the final term above. The upshot is that, for a global variation with $\dot{\epsilon}^{\dagger} = 0$, we have $\delta S = 0$: the action is supersymmetric.
\switchcolumn
该表达式中有两种不同类型的项。第一个采用 $\psi^{\dagger} \psi \dot{\psi}$ 形式。将它们聚集在一起,我们发现它们相乘 $\nabla g = 0$。第二个采用 $\dot{x} \psi^{\dagger} \psi \psi$ 形式。第一个涉及联络的组合,这些联络聚集在一起给出 $\partial \Gamma + \Gamma^2$。但这是 Riemann 张量的定义,并被上面的最后一项取消。结果是,对于 $\dot{\epsilon}^{\dagger} = 0$ 的全局变分,我们有 $\delta S = 0$:作用是超对称的。
\switchcolumn*

\subsubsection{Quantisation: Filling in Forms}
\switchcolumn
\subsubsection*{量子化:填充形式}
\switchcolumn*

Quantising the sigma model needs a little care due to operator ordering issues.
\switchcolumn
由于算符顺序问题,量子化 sigma 模型需要小心一些。
\switchcolumn*

The tricky commutator is, it turns out, the one between bosons and fermions. This is best described in the terms of the mechanical momentum as opposed to the canonical momentum,
\switchcolumn
事实证明,棘手的对易子是玻色子和费米子之间的那个。这可以用机械动量而不是正则动量最好地描述,
\switchcolumn*

As we already advertised in Section 1.4.1, this is a very familiar structure in geometry: it arises for totally anti-symmetric $(0, p)$ tensor fields, also known as \textit{p-forms}. This prompts the identification
\switchcolumn
正如我们在第 1.4.1 节中已经宣传的那样,这是几何中非常熟悉的结构:它出现于完全反对称的 $(0, p)$ 张量场,也称为 \textit{p 形式}。这提示了如下的认同
\switchcolumn*

States in the Hilbert space of supersymmetric quantum mechanics are no longer just functions over the manifold $M$ , but now all forms over the manifold $M$ . States of the kind $f(x) (\psi^{\dagger})^{p} \ket{0}$ correspond to $p$-forms. We denote the space of $p$-forms over $M$ as $\Lambda^{p}(M)$.
\switchcolumn
超对称量子力学 Hilbert 空间中的状态不再只是流形 $M$ 上的函数,而是还包括流形 $M$ 上的所有形式。$f(x) (\psi^{\dagger})^{p} \ket{0}$ 类型的状态对应于 $p$ 形式。我们将 $M$ 上的 $p$ 形式的空间表示为 $\Lambda^{p}(M)$。
\switchcolumn*

This relation between Grassmann variables and forms, identifying $\psi^{\dagger i} \longleftrightarrow \odif{x}^i \wedge$ provides the key link between supersymmetry and more interesting aspects of geometry. From this, many lovely geometrical facts follow. For example, we can ask: what is the geometrical interpretation of $\psi^i$? From the commutation relation $\ab\{\psi^i, \psi^{\dagger i}\} = g^{ij}$, it clearly acts as a map $\psi^{i}: \Lambda^p(M) \mapsto \Lambda^{p - 1}(M)$.
\switchcolumn
Grassmann 变量和形式之间的关系,认同 $\psi^{\dagger i} \longleftrightarrow \odif{x}^i \wedge$ 提供了超对称性和几何学更有趣的方面之间的关键联系。由此,许多可爱的几何事实随之而来。例如,我们可以问:$\psi^i$ 的几何解释是什么?从交换关系 $\ab\{\psi^i, \psi^{\dagger i}\} = g^{ij}$ 来看,它显然相当于一个映射 $\psi^{i}: \Lambda^p (M) \mapsto \Lambda^{p - 1}(M)$。
\switchcolumn*

The fact that $F$ is conserved means that Hamiltonian evolution doesn’t mix up forms of different degrees: energy eigenstates lie in a particular $\Lambda^{p}(M)$. The fermion number $F$ also provides the grading that splits our Hilbert space into bosonic and fermionic pieces: $\mathcal{H} = \mathcal{H}_B \oplus \mathcal{H}_F$. These comprise of even and odd forms respectively.
\switchcolumn
$F$ 守恒的事实意味着 Hamiltonian 演化不会混合不同程度的形式:能量本征态位于特定的 $\Lambda^{p}(M)$ 中。费米子数 $F$ 还提供了将 Hilbert 空间分为玻色子和费米子部分的分级:$\mathcal{H} = \mathcal{H}_B \oplus \mathcal{H}_F$。它们分别由偶形式和奇形式组成。
\switchcolumn*

Finally, we come to the supercharges $Q$ and $Q^{\dagger}$ themselves. The presence of the momentum operator means that these act as derivatives, while the fermions ensure that they also map $Q: \Lambda^p(M) \mapsto \Lambda^{p + 1}(M)$. But there is a very natural object in differential geometry with these properties: it is the exterior derivative
\switchcolumn
最后,我们来看看超荷 $Q$ 和 $Q^{\dagger}$ 本身。动量算子的存在意味着它们充当导数,而费米子确保它们也映射 $Q: \Lambda^p(M) \mapsto \Lambda^{p + 1}(M)$。但是微分几何中有一个非常自然的对象具有这些属性:它是外导数
\switchcolumn*

This adjoint operator annihilates functions $d^{\dagger} f = 0$ for $f \in \Lambda^0(M)$. This is to be expected since it follows from $\psi^i \ket{0} = 0$. Similarly, the exterior derivative itself annihilates top forms, $d\odif{\omega} = 0$ for all $\omega \in \Lambda^n(M)$. Before we go on, note that the correspondence $Q \equiv d$ and $Q^{\dagger} \equiv d^{\dagger}$ is the reason that we chose to define $Q$ to be the supercharge involving $\psi^{\dagger}$ rather than, as in Section 1, in terms of $\psi$.
\switchcolumn
对于 $f \in \Lambda^0(M)$,伴随运算符湮灭了函数 $d^{\dagger} f = 0$。这是可以预料的,因为它是从 $\psi^i \ket{0} = 0$ 得出的。类似地,外导数本身消除了顶级形式,对于所有 $\omega \in \Lambda^n(M)$,$d\odif{\omega} = 0$。在我们继续之前,请注意 $Q \equiv d$ 和 $Q^{\dagger} \equiv d^{\dagger}$ 的对应关系是我们选择将 $Q$ 定义为涉及 $\psi^{\dagger}$ 的超荷的原因,而不是如第 1 节中那样,用 $\psi$ 表示。
\switchcolumn*

This is the \textit{Laplacian} operator in differential geometry. It is clear from its definition in terms of $d$ and $d^{\dagger}$ that it is a prime candidate for a supersymmetric Hamiltonian; in some sense everything that we’ve done above is just to realise this possibility in terms of Grassmann variables $\psi$ and $\psi^{\dagger}$.
\switchcolumn
这就是微分几何中的 \textit{Laplacian} 算子。从 $d$ 和 $d^{\dagger}$ 的定义可以清楚地看出,它是超对称 Hamiltonian 的主要候选者;从某种意义上说,我们上面所做的一切只是为了通过 Grassmann 变量 $\psi$ 和 $\psi^{\dagger}$ 来实现这种可能性。
\switchcolumn*

The Laplacian is positive definite, as befits a supersymmetric Hamiltonian.
\switchcolumn
Laplacian 是正定的,适合超对称 Hamiltonian。
\switchcolumn*

Note, however, that in the absence of supersymmetry there was always the option to add the $\alpha R$ term in (3.3) to the Hamiltonian. Supersymmetry removes this ambiguity.
\switchcolumn
但请注意,在不存在超对称性的情况下,始终可以选择将 (3.3) 中的 $\alpha R$ 项添加到 Hamiltonian 中。超对称性消除了这种歧义。
\switchcolumn*

\subsubsection{Ground States and de Rham Cohomology}
\switchcolumn
\subsubsection*{基态和 de Rham 上同调}
\switchcolumn*

This discussion also tells us that there are three kinds of states in the Hilbert space: those for which $\ket{\phi} = Q \ket{\alpha}$ or $\ket{\phi} = Q^{\dagger} \ket{\beta}$, which sit in supersymmetric pairs. And those for which $Q \ket{\phi} = Q^{\dagger} \ket{\phi} = 0$ which are the supersymmetric ground states. This means that any state $\ket{\phi} \in \mathcal{H}$ has a unique decomposition as
\switchcolumn
这个讨论还告诉我们,Hilbert 空间中存在三种状态:$\ket{\phi} = Q \ket{\alpha}$ 或 $\ket{\phi} = Q^{\dagger } \ket{\beta}$,它们位于超对称对中。那些 $Q \ket{\phi} = Q^{\dagger} \ket{\phi} = 0$ 的超对称基态。这意味着任何状态 $\ket{\phi} \in \mathcal{H}$ 都有一个独特的分解:
\end{paracol}

\[ \ket{\phi} = Q \ket{\alpha} + Q^{\dagger} \ket{\beta} + \ket{\omega} \]

\begin{paracol}{2}
There is an important comment to make here. The Hodge decomposition theorem is not a trivial statement in mathematics. It took Hodge much of the 1930s to prove and, even then, needed corrections from Weyl and Kodaira. Yet the statement about the decomposition of states in the Hilbert space (3.11) follows trivially from the structure of supersymmetric quantum mechanics! What’s going on?
\switchcolumn
这里有一个重要的评论。Hodge 分解定理并不是一个简单的数学陈述。Hodge 花了 20 世纪 30 年代的大部分时间来证明这一点,即便如此,也需要 Weyl 和 Kodaira 的纠正。然而,关于 这个讨论还告诉我们,Hilbert 空间(3.11)中状态分解的陈述是从超对称量子力学的结构中得出的!这是怎么回事?
\switchcolumn*

Shortly we will "prove" other theorems in geometry where we will make use of the physicist’s secret weapon, the path integral. Here, however, the power of physics comes only from our blatant disregard for anything approaching rigour. In geometry, the space of differential forms is \textit{not} a Hilbert space because the inner product (3.9) is not complete. In quantum mechanics, we deal with this by restricting attention to $L^2$ forms but then one has to worry whether the exterior derivative acts solely within this space. All of these are subtleties that we sweep under the rug in physics, but present the real challenge behind the proof of the Hodge decomposition theorem.
\switchcolumn
很快我们将“证明”几何中的其他定理,其中我们将利用物理学家的秘密武器——路径积分。然而,在这里,物理学的力量仅仅来自于我们对任何接近严谨的事物的公然漠视。在几何中,微分形式的空间 \textit{不是} Hilbert 空间,因为内积 (3.9) 不完备。在量子力学中,我们通过限制对 $L^2$ 形式的关注来处理这个问题,但随后人们必须担心外导数是否仅在这个空间内起作用。所有这些都是我们在物理学中隐藏起来的微妙之处,但却是 Hodge 分解定理证明背后的真正挑战。
\switchcolumn*

\subsubsection*{Cohomology}
\switchcolumn
\subsubsection*{上同调}
\switchcolumn*

There is another way to view the ground states in terms of \textit{cohomology}. As we’ve seen, the exterior derivative $d$ (or equivalently the supercharge $Q$) maps us from $d: \Lambda^p(M) \to \Lambda^{p + 1}(M)$. We can depict this in terms of what mathematicians call a \textit{chain complex}
\switchcolumn
还有另一种方法可以根据 \textit{上同调} 来查看基态。正如我们所见,外导数 $d$(或等效的超荷 $Q$)给了我们映射 $d: \Lambda^p(M) \to \Lambda^{p + 1}(M)$。我们可以用数学家所说的 \textit{链复合体} 来描述这一点
\end{paracol}

\[ \Lambda^0(M) \xlongrightarrow{d} \Lambda^1(M) \xlongrightarrow{d} \Lambda^2(M) \xlongrightarrow{d} \Lambda^3(M) \xlongrightarrow{d} \dots \]

\begin{paracol}{2}

Because $d^2 = 0$, the image of one map necessarily lies in the kernel of the next. The idea of cohomology is that it’s interesting to look more closely at the difference between the kernel and image.
\switchcolumn
因为 $d^2 = 0$,一个映射的像必然位于下一个映射的核中。上同调的想法是,更仔细地观察核和像之间的差异是很有趣的。
\switchcolumn*

There are a number of interesting things about these Betti numbers. First, this counting of cohomology classes is just another way of counting the ground states in quantum mechanics, and the Betti numbers can equally well be viewed as counting harmonic forms.
\switchcolumn
关于这些 Betti 数有很多有趣的事情。首先,这种上同调类的计数只是量子力学中计算基态的另一种方式,Betti 数同样可以被视为计数调和形式。
\switchcolumn*

There’s an analogy here with gauge symmetry that is worth highlighting. In Maxwell theory, the gauge potentials $A$ and $A + \odif{\alpha}$ are physically equivalent as they are related by a gauge transformation. If we want to pick a representative of this equivalence class then we need gauge fixing condition that picks out one particular choice of $A$. For cohomology, the equivalence class $[\omega]$ relates $\omega \sim \omega + \odif{\alpha}$. A representative of this class can be picked by the "gauge fixing condition" $\mathrm{d}^{\dagger} \omega = 0$. This then picks out the harmonic forms as special.
\switchcolumn
这里有一个与规范对称性的类比值得强调。在 Maxwell 理论中,规范势 $A$ 和 $A + \odif{\alpha}$ 在物理上是等效的,因为它们通过规范变换相关。如果我们想选择这个等价类的代表,那么我们需要规范固定条件来挑选 $A$ 的一个特定选择。对于上同调,等价类 $[\omega]$ 联系了 $\omega \sim \omega + \odif{\alpha}$。此类的代表可以通过“规范固定条件”$\mathrm{d}^{\dagger} \omega = 0$ 来选取。然后,它会挑选出特殊的调和形式。
\switchcolumn*

Any manifold $M$ with dimension $\dim(M) = n$ always has $b_0 = 1$ and $b_n = 1$. The zero forms are just functions over the manifold, and any constant function over $M$ is clearly harmonic, but cannot be written as $d(\text{something})$ as there are no$p = - 1$ forms. Similarly, the volume form $\text{Vol} = \star 1$ provides the harmonic top form.
\switchcolumn
任何维度为 $\dim(M) = n$ 的流形 $M$ 始终具有 $b_0 = 1$ 和 $b_n = 1$。零形式只是流形上的函数,$M$ 上的任何常数函数显然都是调和的,但不能写成 $d(\text{something})$ 因为没有 $p = - 1$ 形式。类似地,体积形式 $\text{Vol} = \star 1$ 提供了谐波顶部形式。
\switchcolumn*

Other Betti numbers come in pairs with $b_p = b_{n - p}$, a relationship that follows from Poincaré duality. It turns out that all these higher Betti numbers are non-vanishing only if the manifold $M$ has some interesting topology. To explain this, we need to remove the co in cohomology.
\switchcolumn
其他 Betti 数与 $b_p = b_{n - p}$ 成对出现,这种关系源自 Poincaré 对偶性。事实证明,只有当流形 $M$ 具有一些有趣的拓扑时,所有这些较高的 Betti 数才不会消失。为了解释这一点,我们需要去掉“上同调”中的“上”。
\switchcolumn*

\subsubsection*{Homology}
\switchcolumn
\subsubsection*{同调}
\switchcolumn*

Here we give a brief overview of how the de Rham cohomology, and associated harmonic forms, contain information about the topology of the manifold $M$.
\switchcolumn
在这里,我们简要概述 de Rham 上同调以及相关的调和形式如何包含流形 $M$ 的拓扑信息。
\switchcolumn*

We can see this in two simple examples shown in Figure 9. There we depict two manifolds of dimension two: the sphere $M = \bm{\mathrm{S}}^2$ and the torus $M = \bm{\mathrm{T}}^2$. On each we’ve drawn a one-dimensional submanifold $C$ as a red line. For $C \subset \bm{\mathrm{S}}^2$, this submanifold is the boundary of a disc $C = \partial D$. For $C \subset \bm{\mathrm{T}}^2$ there is no such bounding manifold $D$. This reflects the fact that there is interesting topology in the torus, but not in the sphere.
\switchcolumn
我们可以在图 9 所示的两个简单示例中看到这一点。其中我们描绘了两个二维流形:球体 $M = \bm{\mathrm{S}}^2$ 和环面 $M = \bm{\mathrm {T}}^2$。在每一个上,我们都绘制了一个一维子流形 $C$ 作为红线。对于$C \subset \bm{\mathrm{S}}^2$,这个子流形是圆盘$C = \partial D$的边界。对于 $C \subset \bm{\mathrm{T}}^2$ 来说,不存在这样的边界流形 $D$。这反映了这样一个事实:环面中有有趣的拓扑,但球体中没有。
\switchcolumn*

Indeed, there are actually two different topo- logically non-trivial submanifolds of the torus: in addition to the circle $C$ shown in Figure 9, there is also the circle $C'$ that winds in the way shown on the right.
\switchcolumn
事实上,环面实际上有两个不同的拓扑非平凡子流形:除了图 9 中所示的圆 $C$ 之外,还有按右侧所示方式缠绕的圆 $C'$。
\switchcolumn*

The algebraic structure of these topologically non-trivial submanifolds is identical to those of forms. In particular, a boundary of a boundary is always vanishing, which we write as $\partial^2 = 0$. This, obviously, is the strikingly reminiscent of the exterior derivative relation $d^2 = 0$. We use this to define homology groups using $\partial$ analogous to the cohomology groups that we defined previously using $d$. The \textit{homology group} $H_p(M)$ is the equivalence class of closed $p$-dimensional submanifolds that are not themselves the boundary of a ($p+1$)-dimensional manifold. In particular, two submanifolds $C_1$ and $C_2$ lie in the same cohomology class if one can be smoothly deformed into the other. In terms of equations, this mean that difference is a boundary,
\switchcolumn
这些拓扑非平凡子流形的代数结构与形式的代数结构相同。特别是,边界的边界总是为零,我们将其写为$\partial^2 = 0$。显然,这令人震惊地让人想起外导数关系 $d^2 = 0$。我们使用 $\partial$ 来定义同调群,类似于我们之前使用 $d$ 定义的上同调群。\textit{同调群} $H_p(M)$ 是闭合 $p$ 维子流形的等价类,这些子流形本身不是 ($p+1$) 维流形的边界。特别是,如果两个子流形 $C_1$ 和 $C_2$ 可以平滑地变形为另一个,则它们位于同一上同调类中。就方程而言,这意味着差异是一个边界,
\switchcolumn*

The relationship between homology and cohomology is more than just an analogy. The spaces $H_p(M)$ and $H^p(M)$ are dual to each other, and hence isomorphic. This statement, known as \textit{de Rham’s theorem}, is not straightforward to prove but it’s easy to get some intuition for how it works.
\switchcolumn
同调和上同调之间的关系不仅仅是类比。空间 $H_p(M)$ 和 $H^p(M)$ 是对偶的,因此是同构的。这个命题被称为 \textit{de Rham 定理},并不容易证明,但很容易对其工作原理有一些直觉。
\switchcolumn*

The upshot of these arguments is that the ground states of the supersymmetric sigma model (3.6) are determined by the topology of the target space $M$. Heuristically, the quantum particle can minimise its energy by spreading its wavefunction over topologically non-trivial submanifolds of $M$.
\switchcolumn
这些论点的结果是超对称 sigma 模型 (3.6) 的基态由目标空间 $M$ 的拓扑决定。启发式地,量子粒子可以通过将其波函数分布在 $M$ 的拓扑非平凡子流形上来最小化其能量。
\switchcolumn*

Finally, I should mention in any logical presentation, homology precedes cohomology. Our physics approach has lead us to introduce these in an inverted order.
\switchcolumn
最后,我应该在任何逻辑表达中提到,同调先于上同调。我们的物理方法使我们以相反的顺序引入这些。
\switchcolumn*

\subsubsection{The Witten Index and the Chern-Gauss-Bonnet Theorem}
\switchcolumn
\subsubsection*{Witten 指标和 Chern-Gauss-Bonnet 定理}
\switchcolumn*

In Section 1, we learned that there is something special about the Witten index in supersymmetric quantum mechanics. Recall that this is defined by $\Tr (- 1)^{F} \rme^{- \beta H}$ and counts then number of supersymmetric ground states, up to a sign.
\switchcolumn
在第一节中,我们了解到超对称量子力学中的 Witten 指数有一些特殊之处。回想一下,这是由 $\Tr (- 1)^{F} \rme^{- \beta H}$ 定义的,然后计算超对称基态的数量,直到一个符号。
\switchcolumn*

We can see from our discussion of quantum mechanics why this is a topological invariant. We know that the Witten index is robust against any small change of the parameters in the quantum mechanics. In the present case, that means that if we vary the metric $g_{ij}$, at least within reason so that we avoid singularities, then the Witten index should remain unchanged. But that means that object $\chi(M)$ defined in (3.13) must be independent of the the choice of metric: it can be depend only on cruder aspects of $M$, specifically its topology.
\switchcolumn
从我们对量子力学的讨论中我们可以看出为什么这是一个拓扑不变量。我们知道,Witten 指数对于量子力学中参数的任何微小变化都是稳定的。在目前的情况下,这意味着如果我们改变度规 $g_{ij}$,至少在合理范围内以避免奇点,那么 Witten 指数应该保持不变。但这意味着(3.13)中定义的对象 $\chi(M)$ 必须独立于度规的选择:它只能依赖于 $M$ 的更粗略的方面,特别是它的拓扑。
\switchcolumn*

Finally, note that the sigma models provide many other examples in which the Witten index vanishes but there are, nonetheless, ground states with $E = 0$. For example, the sigma model on $\bm{\mathrm{S}}^3$(or, indeed, any odd dimensional sphere) has $\chi(\bm{\mathrm{S}}^3) = 0$ but there are two ground states, one the constant function corresponding $b_0 = 1$ and the other the volume form corresponding to $b_3 = 1$. These ground states are also protected by topology, this time by the cohomology rather than the cruder Euler character.
\switchcolumn
最后,请注意,sigma 模型提供了许多其他示例,其中 Witten 指数消失,但仍然存在 $E = 0$ 的基态。例如,$\bm{\mathrm{S}}^3$(或者实际上任何奇数维球体)上的 sigma 模型有 $\chi(\bm{\mathrm{S}}^3) = 0$ 但有两种基态,一种是对应 $b_0 = 1$ 的常数函数,另一种是对应 $b_3 = 1$ 的体积形式。这些基态也受到拓扑的保护,这次是通过上同调而不是更粗糙的 Euler 特征标。
\switchcolumn*

\subsubsection{The Path Integral Again}
\switchcolumn
\subsubsection*{再次路径积分}
\switchcolumn*

As we saw in Section 2, there is a straightforward description of the Witten index in terms of the path integral.
\switchcolumn
正如我们在第 2 节中看到的,根据路径积分可以直接描述 Witten 指数。
\switchcolumn*

We know that the Witten index is independent of $\beta$. We will use this to compute the path integral in the limit $\beta \to 0$. The key idea is that, in this limit, any non-trivial excitations around the Euclidean circle costs an increasing amount of action and so we can restrict ourselves to constant configurations, where the path integral reduces to a normal integral.
\switchcolumn
我们知道 Witten 指数与 $\beta$ 无关。我们将用它来计算限制 $\beta \to 0$ 的路径积分。关键思想是,在这个限制下,Euclidean 圆周围的任何非平凡激发都会花费越来越多的作用量,因此我们可以将自己限制在恒定配置,其中路径积分减少到正常积分。
\switchcolumn*

where we now see explicitly that in the limit $\beta \to 0$, the modes with $\dot{x}$ and $\dot{\psi}$ non-zero are heavily suppressed. The path integral then reduces to the ordinary integral
\switchcolumn
我们现在明确地看到,在 $\beta \to 0$ 的限制下,$\dot{x}$ 和 $\dot{\psi}$ 非零的模式被严重抑制。然后路径积分减少为普通积分
\switchcolumn*

The magic of the Chern-Gauss-Bonnet theorem is that a global topological object, $\chi(M)$, is described in terms of an integral of local data, the Euler density. The magic of supersymmetric quantum mechanics is that it gives a straightforward derivation of this result, with the only real complication the combinatoric factors that arise from Grassmann integration. This is first example where a deep mathematical result can derived in a different way using the path integral. It won’t be the last.
\switchcolumn
Chern-Gauss-Bonnet 定理的神奇之处在于,全局拓扑对象 $\chi(M)$ 是用局部数据的积分(Euler 密度)来描述的。超对称量子力学的神奇之处在于,它给出了这个结果的直接推导,唯一真正复杂的是 Grassmann 积分产生的组合因子。这是第一个使用路径积分以不同方式得出深层数学结果的示例。这不会是最后一次。
\switchcolumn*

\subsection{Morse Theory}
\switchcolumn
\subsection*{Morse 模型}
\switchcolumn*

We stick with our $N = 2$ supersymmetric sigma model (3.6), describing a particle moving on a manifold $M$. The novelty is that we now also include a potential $h(x)$ over the manifold. The resulting supersymmetric theory is a combination of the sigma model and the kind of theories we considered in Section 1.4.1,
\switchcolumn
我们坚持使用 $N = 2$ 超对称 sigma 模型 (3.6),描述在流形 $M$ 上移动的粒子。新颖之处在于我们现在还在流形上包含了一个潜在的 $h(x)$ 。由此产生的超对称理论是 sigma 模型和我们在第 1.4.1 节中考虑的理论类型的组合,
\switchcolumn*

In the absence of the potential, we know that the ground states of the supersymmetric quantum mechanics spread over cycles of $M$. However, when we add a the potential $h$, the wavefunctions get squeezed and, as the potential gets larger, the wavefunctions are increasingly localised at the minima of the potential. We know that the Witten index can’t change. But, more strongly, the total number of $E = 0$ ground states doesn’t change either and, even in the presence of the potential, is given by the Betti numbers of the manifold.
\switchcolumn
在没有势能的情况下,我们知道超对称量子力学的基态分布在 $M$ 的圆内。然而,当我们添加势能 $h$ 时,波函数会受到挤压,并且随着势能变大,波函数会越来越集中在势能的最小值处。我们知道 Witten 指数不能改变。但是,更强烈的是,$E = 0$ 基态的总数也不会改变,即使存在势能,也由流形的 Betti 数给出。
\switchcolumn*

We saw in Section 3.1.2 that the ground states are determined by the cohomology of $Q$. But the cohomology when $h \neq 0$ is isomorphic to the cohomology when $h = 0$. We simply take the wavefunctions in the latter case and multiply them by $\rme^{- h}$. Indeed, this is the form of the wavefunctions (1.12) that we found back Section 1.2 when considering a particle on a line.
\switchcolumn
我们在第 3.1.2 节中看到,基态是由 $Q$ 的上同调决定的。但 $h \neq 0$ 时的上同调与 $h = 0$ 时的上同调同构。我们只需将后一种情况的波函数乘以 $\rme^{- h}$ 即可。事实上,这就是我们在第 1.2 节考虑直线上的粒子时发现的波函数 (1.12) 的形式。
\switchcolumn*

The fact that the number of supersymmetric ground states is independent of $h$ means that something interesting must be going on. Because if we crank up $h$ to be very large, the ground states are localised around the minima of the potential $V = \ab|\partial_i h|^2$. This means that there must be some relationship between these minima and the topology of the manifold. This relationship goes under the name of \textit{Morse theory}.
\switchcolumn
超对称基态的数量与 $h$ 无关这一事实意味着一定正在发生一些有趣的事情。因为如果我们将 $h$ 调到非常大,基态就会集中在势能 $V = \ab|\partial_i h|^2$ 的最小值附近。这意味着这些最小值和流形的拓扑之间必定存在某种关系。这种关系被称为\textit{Morse 理论}。
\switchcolumn*

Consider the situation where we scale the Morse function $h(x) \to \zeta h(x)$, and subsequently send $\zeta \to \infty$. In this limit, the physics is entirely dominated by the critical points of the potential and, at the semi-classical level, the ground state wavefunction is localised at the critical point $x = X$. That’s not to say that all critical points are necessarily true $E = 0$ ground states; there may well be tunnelling of the kind that we discussed in Section 2.3 that lifts putative ground states in pairs. But the true ground states must be contained within the set of critical points.
\switchcolumn
考虑这样的情况:我们缩放 Morse 函数 $h(x) \to \zeta h(x)$,然后取极限 $\zeta \to \infty$。在此限制下,物理学完全由势的临界点主导,并且在半经典水平上,基态波函数局域于临界点 $x = X$。这并不是说所有临界点都必然为真 $E = 0$ 基态;很可能存在我们在 2.3 节中讨论的那种隧道效应,它成对地提升假定的基态。但真正的基态必须包含在临界点集合内。
\switchcolumn*

We also need to figure out what’s going on with fermions. This is the same calculation that we already met in Section 1.4.1. There, we learned that we should look at the eigenvalues of the Hessian $\partial_i \partial_j h$,
\switchcolumn
我们还需要弄清楚费米子发生了什么。这与我们在 1.4.1 节中已经遇到的计算相同。在那里,我们了解到我们应该看 Hessian 矩阵 $\partial_i \partial_j h$ 的特征值,
\switchcolumn*

We learn that the semi-classical ground state sits in the sector with $\mu(X)$ fermions excited. In other words, the semi-classical ground state at the critical point $x = X$ is a $p$-form with $p = \mu(X)$.
\switchcolumn
我们了解到半经典基态位于 $\mu(X)$ 费米子激发的扇区中。换句话说,临界点 $x = X$ 处的半经典基态是 $p$ 形式,其中 $p = \mu(X)$。
\switchcolumn*

Already we learn something striking. We can compute the Witten index by simply summing over the critical points $X$, just as we did in (1.24). The novelty is that we know that, for our supersymmetric sigma model, the Witten index tells us the Euler character of the manifold $M$. This means that we can compute the Euler character of $M$ from the critical points of a function over $M$,
\switchcolumn
我们已经学到了一些惊人的东西。我们可以通过简单地对关键点 $X$ 求和来计算 Witten 指数,就像我们在 (1.24) 中所做的那样。新颖之处在于,我们知道,对于我们的超对称 sigma 模型,Witten 指数告诉我们流形 $M$ 的 Euler 特征标。这意味着我们可以从 $M$ 上函数的临界点计算 $M$ 的 Euler 特征标,
\switchcolumn*

In fact, we can say more than this. The total number of critical points may well be more than the total number of $E = 0$ ground states, since states can be lifted in pairs. But the number of critical points can never be smaller than the number of ground states! Suppose that there are $m_p$ critical points $X$ with Morse index $p = \mu(X)$. This can be no less than the number of ground states associated to $p$-forms, so
\switchcolumn
事实上,我们可以说的还不止这些。临界点的总数很可能比 $E = 0$ 基态的总数多,因为状态可以成对提升。但临界点的数量永远不能小于基态的数量!假设有 $m_p$ 个临界点 $X$,Morse 指标 $p = \mu(X)$。这不能少于与 $p$-forms 相关的基态数量,所以
\switchcolumn*

Nice as this is, it’s possible to do better. We can, in fact, recover the original Betti numbers $b_p$ from an understanding of the critical points and the relationships between them. In the rest of this section we explain how.
\switchcolumn
尽管这很好,但还可以做得更好。事实上,我们可以通过理解关键点及其之间的关系来恢复原始的贝蒂数 $b_p$。在本节的其余部分中,我们将解释如何进行。
\switchcolumn*

\subsubsection*{A Simple Example: The Two Sphere}
\switchcolumn
\subsubsection*{一个简单的例子:二维球面}
\switchcolumn*

This is shown in the left-hand side of Figure 10. Clearly there are two critical points of the height function: at the bottom of the sphere where it is a minimum and at the top of the sphere where it is a maximum. The Morse index is $\mu = 0$ and $\mu = 2$ respectively, so from the discussion above we know that these ground states are associated to 0-forms and 2-forms. We also know that ground states localised around these minima must be exact $E = 0$ states.
\switchcolumn
如图 10 的左侧所示。显然,高度函数有两个临界点:在球体的底部,它是最小值,在球体的顶部,它是最大值。Morse 指标分别为 $\mu = 0$ 和 $\mu = 2$,因此从上面的讨论我们知道这些基态与 0-形式和 2-形式相关。我们还知道,围绕这些最小值的基态必须是精确的 $E = 0$ 状态。
\switchcolumn*

Now we deform the system. We could change the Morse function $h$ but, for illustrative purposes, it is simplest if we instead change the metric on the sphere. We’ll turn it into the bean shape shown in the right-hand side of Figure 10, keeping the same height function $h = z$. This time there are four critical points, one with $\mu = 0$, two at the top with $\mu = 2$, and the saddle point in the middle with $\mu = 1$.
\switchcolumn
现在我们对系统进行变形。我们可以更改 Morse 函数 $h$,但出于说明目的,如果我们更改球体上的度量,则最简单。我们将其变成图 10 右侧所示的豆形,并保持相同的高度函数 $h = z$。这次有四个临界点,一个$\mu = 0$,顶部两个$\mu = 2$,中间的鞍点$\mu = 1$。
\switchcolumn*

My wife thought it important to point out that these are not drawn to scale.
\switchcolumn
我的妻子认为有必要指出这些并不是按比例绘制的。
\switchcolumn*

For the bean shaped metric, we know that two of four semi-classical ground states must be lifted to have $E > 0$. Clearly, it should be the 1-form and some combination of the two 2-forms that gets lifted. Our goal now is to understand how this works, both in the case of the kidney bean and more generally. We will see that much of the technology that we will need has already been covered in the supersymmetric instanton calculation of Section 2.3
\switchcolumn
对于豆形度规,我们知道必须提升四个半经典基态中的两个才能使 $E > 0$。显然,应该是 1 形式和两个 2 形式的某种组合被提升。我们现在的目标是了解它是如何发挥作用的,无论是在芸豆还是更一般的情况下。我们将看到我们需要的大部分技术已经在第 2.3 节的超对称瞬子计算中涵盖了
\switchcolumn*

\subsubsection{Instantons Again}
\switchcolumn
\subsubsection*{再次瞬子}
\switchcolumn*

The exact energy eigenstate localised around $x = X_a$ will be denoted as $\ket{\Psi_a}$. Some of these states will persist as zero energy states when all quantum corrections are taken into account. Meanwhile others will be lifted but, as we saw in Section 2.3, will remain as low lying states, with energies of order $\rme^{- S_{\text{inst}}}$. Our goal is to understand this spectrum.
\switchcolumn
位于 $x = X_a$ 附近的精确能量本征态将表示为 $\ket{\Psi_a}$。当考虑到所有量子修正时,其中一些状态将持续为零能量状态。与此同时,其他的将被提升,但正如我们在 2.3 节中看到的,将保持为低位状态,其能量为 $\rme^{- S_{\text{inst}}}$。我们的目标是理解这个频谱。
\switchcolumn*

\subsubsection*{The Instanton Equations}
\switchcolumn
\subsubsection*{瞬子方程}
\switchcolumn*

It’s clear that we are now back in the realm of the quantum tunnelling calculations that we performed in Sections 2.2 and 2.3. To start, we can study the instantons in a sigma model with potential.
\switchcolumn
很明显,我们现在回到了第 2.2 节和第 2.3 节中执行的量子隧穿计算领域。首先,我们可以研究有势的 sigma 模型中的瞬子。
\switchcolumn*

From our previous calculation in Section 2.3, we know that the fermion zero modes play a crucial role in supersymmetric instanton calculations. So our next question: how many fermi zero modes does our instanton have? To answer this, we look at the linearised fermion equation of motion. Here "linearised" means that we drop the Riemann tensor term in (3.14), and the connection term in $\nabla t$.
\switchcolumn
从我们之前 2.3 节的计算中,我们知道费米子零模在超对称瞬子计算中起着至关重要的作用。所以我们的下一个问题:我们的瞬子有多少个费米子零模?为了回答这个问题,我们研究线性化费米子运动方程。这里“线性化”意味着我们去掉(3.14)中的 Riemann 张量项,以及 $\nabla t$ 中的联络项。
\switchcolumn*

We want to know how many solutions each of these equations have in the background of an instanton. In fact, we really just want to know the difference between the number of solutions to these equations. This is because if both $D$ and $D^{\dagger}$ have zero modes then they will most likely be lifted by the non-linear terms in the action. And, indeed, generically, this will happen. However if there are unpaired zero modes of, say $D^{\dagger}$, then these must be saturated in some other way in the path integral.
\switchcolumn
我们想知道这些方程在瞬子背景下有多少个解。事实上,我们只是想知道这些方程的解的个数之间的差异。这是因为如果 $D$ 和 $D^{\dagger}$ 都具有零模式,那么它们很可能会被作用量中的非线性项提升。事实上,一般来说,这种情况将会发生。然而,如果存在不成对的零模式,例如 $D^{\dagger}$,那么这些模式必须在路径积分中以某种其他方式饱和。
\switchcolumn*

\subsubsection*{Instantons and the Relative Morse Index}
\switchcolumn
\subsubsection*{瞬子和相对 Morse 指标}
\switchcolumn*

We've now played the "Grassmann integration" card twice: once in (3.18) to argue that we should get contributions only between vacua that have relative Morse index $1$, and again above to argue that we should only get contributions from instantons with $\mathcal{I}(D^{\dagger}) = 1$. Clearly we need these two different arguments to coincide.
\switchcolumn
现在,我们已经打了两次“Grassmann 积分”牌:一次是在(3.18)中,认为我们应该只在具有相对 Morse 指标 $1$ 的真空之间获得贡献,而在上面再次论证我们应该只从 $\mathcal{I}(D^{\dagger}) = 1$ 的瞬子中获得贡献。显然,我们需要这两个不同的论点一致。
\switchcolumn*

The upshot is that the number of bosonic collective coordinates is equal to the number of fermi zero modes, and both are counted by the relative Morse index. A slicker way of saying this is to note that bosonic and fermionic zero modes are related by the unbroken supersymmetry $Q^{\dagger}$ in the background of an instanton.
\switchcolumn
结果是玻色子集体坐标的数量等于费米零模的数量,并且两者都是通过相对 Morse 指标来计数的。更巧妙的说法是,玻色子和费米子零模通过瞬子背景中的未破缺的超对称性 $Q^{\dagger}$ 相关。
\switchcolumn*

For our purposes, we want to consider instantons that interpolate between critical points with relative Morse index $1$. Here the sole bosonic collective coordinate is the obvious one: the time $\tau_1$ at which the instanton does its business of interpolating from one critical point to the other. This is the collective coordinate that we met previously in Section 2.2.
\switchcolumn
出于我们的目的,我们希望考虑在相对 Morse 指标为 $1$ 的临界点之间进行插值的瞬子。这里唯一的玻色子集体坐标是显而易见的:瞬时子执行从一个临界点插值到另一个临界点的时间 $\tau_1$ 。这是我们之前在 2.2 节中遇到的集体坐标。
\switchcolumn*

Although not of immediate utility, we can also get a feel for where the other bosonic col- lective coordinates may come from when $\Delta \mu >1$. Consider the height Morse function on the round sphere $\bm{\mathrm{S}}^2$. We know that there are two critical points at the south and north pole with Morse index $0$ and $2$ respectively. Correspondingly, the instanton that interpolates from the south to the north pole has two collective coordinates: one is the time $\tau_1$ at which the instanton makes the jump, the other is the angle $\phi$ of the trajectory as shown in the figure. In this example, the second collective coordinate is obvious because it arises due to a symmetry. But the arguments above tell us that, perhaps surprisingly, this second collective coordinate persists even when we deform the sphere, or potential, so that there’s no longer a rotational symmetry.
\switchcolumn
虽然没有立即实用性,但我们也可以了解当 $\Delta \mu >1$ 时其他玻色子集体坐标可能来自哪里。考虑圆球 $\bm{\mathrm{S}}^2$ 上的高度 Morse 函数。我们知道,南极和北极有两个临界点,Morse 指数分别为 $0$ 和 $2$。相应地,从南极插值到北极的瞬子有两个集体坐标:一个是瞬子跳跃的时间 $\tau_1$,另一个是轨迹的角度 $\phi$,如图所示图。在这个例子中,第二个集体坐标是显而易见的,因为它是由于对称性而产生的。但上面的论点告诉我们,也许令人惊讶的是,即使我们使球体或势变形,第二个集体坐标仍然存在,因此不再存在旋转对称性。
\switchcolumn*

\subsubsection{Completing the Instanton Computation}
\switchcolumn
\subsubsection*{完成瞬子计算}
\switchcolumn*

Let’s start in the vacuum $\ket{\Psi_a}$ localised at the end of the instanton trajectory at $X_a$. There are $\mu(X_a)$ negative eigenvalues of the Hessian and their eigenvectors span a $\mu(X_a)$-dimensional space that we call $V_a$. The ground state $\ket{\Psi_a}$ is associated to a $\mu(X_a)$-form and this induces an orientation on $V_a$.
\switchcolumn
让我们从位于 $X_a$ 瞬子轨迹末端的真空 $\ket{\Psi_a}$ 开始。Hessian 矩阵有 $\mu(X_a)$ 负特征值,它们的特征向量跨越 $\mu(X_a)$ 维空间,我们称之为 $V_a$。基态 $\ket{\Psi_a}$ 与 $\mu(X_a)$ 形式相关联,这会导致 $V_a$ 上的定向。
\switchcolumn*

The tangent to the instanton trajectory at $X_a$ lies in the space $V_a$ of negative eigenvectors. Let us call this tangent vector $v$. Generically, $v$ will coincide with the eigenvector with largest negative eigenvalue $\lambda_k$ (or smallest $\ab|\lambda_k|$) since this is usually the unique direction for which the eigenvalue flips sign by the time we reach $X_b$ at the bottom. We denote the subspace of $V_a$ that is orthogonal to $v$ as $\tilde{V}_a$. There is a natural orientation on $\tilde{V}_a$ that comes from taking the interior product $\iota_v \Psi_a$.
\switchcolumn
$X_a$ 处瞬子轨迹的切矢位于负特征向量的空间 $V_a$ 中。我们称这个切矢为 $v$。一般来说,$v$ 将与具有最大负特征值 $\lambda_k$(或最小的 $\ab|\lambda_k|$)的特征向量一致,因为这通常是当我们在底部到达 $X_b$ 时特征值翻转符号的唯一方向。我们将与 $v$ 正交的 $V_a$ 子空间表示为 $\tilde{V}_a$。$\tilde{V}_a$ 上有一个自然的方向,它来自于内积 $\iota_v \Psi_a$。
\switchcolumn*

For example, consider again the deformed bean-shaped sphere shown in Figure 12. There are two instanton trajectories that interpolate from the minimum $X_b$ at the bottom to the saddle point $X_a$ in the middle. At $X_a$, the tangent vectors to the two different instanton trajectories point in different directions, and that means that each instanton trajectory induces opposite orientations $\iota_v \Psi_a$ on $\tilde{V}_a$. Correspondingly, one instanton will have $n_{\gamma} = + 1$ and the other $n_{\gamma} = - 1$, and the two cancel out in (3.23). This same argument explains why the ground states are not lifted for the double well on a circle that we discussed in Section 2.3.4.
\switchcolumn
例如,再次考虑图 12 中所示的变形豆形球体。有两个瞬子轨迹从底部的最小值 $X_b$ 插值到中间的鞍点 $X_a$。在 $X_a$ 处,两个不同瞬子轨迹的切矢指向不同的方向,这意味着每个瞬子轨迹在 $\tilde{V}_a$ 上产生相反的方向 $\iota_v \Psi_a$。相应地,一个瞬子的 $n_{\gamma} = + 1$ 而另一个 $n_{\gamma} = - 1$,并且两者在 (3.23) 中抵消。同样的论点解释了为什么我们在 2.3.4 节中讨论的圆上的双阱的基态没有被提升。
\switchcolumn*

\subsubsection{The Morse-Witten Complex}
\switchcolumn
\subsubsection*{Morse-Witten 复合体}
\switchcolumn*

Let’s recap. A Morse function gives us a collection of critical points. There are $m_p$ critical points $X$ with Morse index $p = \mu(X)$ and, associated to each, there is an energy eigenstate $\ket{\Psi_a}$ and an associated $p$-form. These can be thought of as a basis for an $m_p$-dimensional space that we will call $C^p$.
\switchcolumn
让我们回顾一下。Morse 函数为我们提供了临界点的集合。有 $m_p$ 个临界点 $X$,Morse 索引 $p = \mu(X)$,每个临界点相关联一个能量本征态 $\ket{\Psi_a}$ 和一个相关的 $p$-形式。这些可以被认为是一个 $m_p$ 维空间的基,我们将其称为 $C^p$。
\switchcolumn*

Here the "almost resolution" of $\mathbbm{1}$ is because we've neglected all higher energy states. But their overlap with the low lying states $\ket{\Psi_a}$ is exponentially suppressed and can be ignored. This means that we're left with an expression for the action of $Q$ among the critical points. Neither the factor of $\sqrt{2 \pi}$, nor the instanton action, are important for our present purposes and can be absorbed into the normalisation of the states.
\switchcolumn
这里“几乎归一化”为 $\mathbbm{1}$ 是因为我们忽略了所有更高的能态。但它们与低能的态 $\ket{\Psi_a}$ 的重叠受到指数压低,可以忽略不计。这意味着我们留下了临界点中 $Q$ 的作用的表达式。 $\sqrt{2 \pi}$ 因子和瞬子作用对于我们当前的目的来说都不重要,并且可以被吸收到状态的标准化中。
\switchcolumn*

This is a map $Q: C^p \to C^{p + 1}$. More abstractly, it can be viewed as a map between spaces of critical points. And, importantly, it satisfies $Q^2 = 0$. This means that we can define a chain complex (strictly a cochain complex), known as the \textit{Morse-Witten complex}, or sometimes the \textit{Morse-Smale-Witten complex}
\switchcolumn
这是一个映射 $Q: C^p \to C^{p + 1}$。更抽象地说,它可以被视为临界点空间之间的映射。而且,重要的是,它满足 $Q^2 = 0$。这意味着我们可以定义一个链复合体(严格来说是一个上链复合体),称为 \textit{Morse-Witten 复合体},有时也称为 \textit{Morse-Smale-Witten 复合体}
\end{paracol}

\[ 0 \longrightarrow C^0 \xlongrightarrow{Q} C^1 \xlongrightarrow{Q} \dots \xlongrightarrow{Q} C^n \xlongrightarrow{Q} 0 \]

\begin{paracol}{2}

The cohomology of $Q$ describes the $E = 0$ ground states of the system or, equivalently, the Betti numbers.
\switchcolumn
$Q$ 的上同调描述了系统的 $E = 0$ 基态,或者等效地,Betti 数。
\switchcolumn*

\subsection{The Atiyah-Singer Index Theorem}
\switchcolumn
\subsection*{Atiyah-Singer 指标定理}
\switchcolumn*

We now turn to a second application of supersymmetric quantum mechanics. We will study a version of supersymmetric quantum mechanics that yields the \textit{Atiyah-Singer index theorem}. Before introducing the physics, we first explain what problem the index theorem addresses.
\switchcolumn
我们现在转向超对称量子力学的第二个应用。我们将研究超对称量子力学的一个版本,它产生 \textit{Atiyah-Singer 指标定理}。在介绍物理之前,我们首先解释指标定理解决的问题。
\switchcolumn*

The only solutions to (3.24) are constant spinors. That’s a bit boring and, in $\mathbb{R}^n$, more than a bit non-normalisable. Things get more interesting when the fermion lives on a curved manifold $M$.
\switchcolumn
(3.24) 的唯一解是常数旋量。这有点无聊,而且在 $\mathbb{R}^n$ 中,有点不可归一化。当费米子生活在弯曲流形 $M$ 上时,事情会变得更加有趣。
\switchcolumn*

We can then ask: how many solutions there are to the Dirac equation (3.26)? This is where the Atiyah-Singer index theorem comes in. It relates the number of solutions to the Dirac equation to the topology of the underlying manifold. The purpose of this section is to give a physics derivation of the index theorem from supersymmetric quantum mechanics.
\switchcolumn
那么我们可以问:Dirac 方程(3.26)有多少个解?这就是 Atiyah-Singer 指标定理的用武之地。它将 Dirac 方程的解数与基础流形的拓扑联系起来。本节的目的是从超对称量子力学中给出指标定理的物理推导。
\switchcolumn*

\subsubsection{The $N = 1$ Sigma Model}
\switchcolumn
\subsubsection*{$N = 1$ Sigma 模型}
\switchcolumn*

Our quantum mechanics of choice has half the supersymmetry of the models that we’ve considered until now in this section. That is, we will have $N = 1$ supersymmetry with a single real supercharge $Q$. We met some simple theories of this kind already in Section 1.4.3.
\switchcolumn
我们选择的量子力学具有本节到目前为止我们所考虑的模型的一半超对称性。也就是说,我们将拥有 $N = 1$ 超对称性和单个实超荷 $Q$。我们已经在 1.4.3 节中遇到了一些此类简单的理论。
\switchcolumn*

The key difference is that the Grassmann variables are now Majorana modes
\switchcolumn
主要区别在于 Grassmann 变量现在是 Majorana 模
\switchcolumn*

Although the action (3.28) has fewer interaction terms, it also has less symmetry. In particular, because the fermions are real we no longer have the $U(1)$ symmetry that rotated the phase of the fermions. For our $N = 2$ sigma model (3.6), this symmetry ensured that the energy eigenstates had a fixed number, $p$ of excited fermions. Now that we no longer have this symmetry, we expect the energy eigenstates to involve a mixture of different fermions. The only protection we have comes from the $(- 1)^F$ symmetry that categorises states into $\mathcal{H}_B$ with an even number of fermions and $\mathcal{H}_F$ with an odd number.
\switchcolumn
尽管作用量 (3.28) 的相互作用项较少,但它的对称性也较低。特别是,因为费米子是实的,所以我们不再具有旋转费米子相位的 $U(1)$ 对称性。对于我们的 $N = 2$ sigma 模型 (3.6),这种对称性确保能量本征态具有固定数量 $p$ 的受激费米子。现在我们不再具有这种对称性,我们期望能量本征态涉及不同费米子的混合物。我们唯一的保护来自 $(- 1)^F$ 对称性,它将状态分类为具有偶数个费米子的 $\mathcal{H}_B$ 和具有奇数个费米子的 $\mathcal{H}_F$。
\switchcolumn*

which is closely related to the Clifford algebra (3.25): the relationship between the fermions and gamma matrices involves a vielbein to accommodate the presence of the metric: $\psi^i = e_{a}^{\ \ i} \gamma^a$. This is telling us that, upon quantisation, the fermions will give $2^{n/2}$ states which can be viewed as a Dirac spinor $\chi$ living on the manifold $M$. While quantisation of the $N = 2$ sigma model (3.6) gave us $p$-forms over the manifold, now we have a spinor.
\switchcolumn
与 Clifford 代数 (3.25) 密切相关:费米子和 gamma 矩阵之间的关系涉及一个 vielbein 来适应度规的存在: $\psi^i = e_{a}^{\ \ i} \gamma^a$。这告诉我们,在量子化时,费米子将给出 $2^{n/2}$ 状态,可以将其视为存在于流形 $M$ 上的 Dirac 旋量 $\chi$。虽然 $N = 2$ sigma 模型 (3.6) 的量化为我们提供了流形上的 $p$ 形式,但现在我们有了一个旋量。
\switchcolumn*

We can now see why this quantum mechanics is of interest. The ground states are specified by solutions to the Dirac equation which is exactly what we want to count. Moreover, we know how to count ground states in supersymmetric quantum mechanics, at least up to sign: we use the Witten index.
\switchcolumn
我们现在可以明白为什么这种量子力学令人感兴趣了。基态由 Dirac 方程的解指定,这正是我们想要计算的。此外,我们知道如何计算超对称量子力学中的基态,至少精确到符号:我们使用 Witten 指数。
\switchcolumn*

\subsubsection{The Path Integral Again}
\switchcolumn
\subsubsection*{再次路径积分}
\switchcolumn*

\subsubsection{Adding a Gauge Field}
\switchcolumn
\subsubsection*{添加规范场}
\switchcolumn*

\subsection{What Comes Next?}
\switchcolumn
\subsection*{接下来是什么?}
\switchcolumn*
\end{paracol}

\end{document}
