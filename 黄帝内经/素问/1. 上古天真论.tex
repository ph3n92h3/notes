\subsection{上古天真论}

昔在黄帝,生而神灵,弱而能言,幼而徇齐,长而敦敏,成而登天。乃问于天师曰:余闻上古之人,春秋皆度百岁,而动作不衰;今时之人,年半百而动作皆衰者,时世异耶,人将失之耶?歧伯对曰:上古之人,其知道者,法于阴阳,和于术数,食饮有节,起居有常,不妄作劳,故能形与神俱,而尽终其天年,度百岁乃去。今时之人不然也,以酒为浆,以妄为常,醉以入房,以欲竭其精,以耗散其真,不知持满,不时御神,务快其心,逆于生乐,起居无节,故半百而衰也。

夫上古圣人之教下也,皆谓之虚邪贼风,避之有时,恬惔虚无,真气从之,精神内守,病安从来。是以志闲而少欲,心安而不惧,形劳而不倦,气从以顺,各从其欲,皆得所愿。故美其食,任其服,乐其俗,高下不相慕,其民故曰朴。是以嗜欲不能劳其目,淫邪不能惑其心,愚智贤不肖不惧于物,故合于道。所以能年皆度百岁,而动作不衰者,以其德全不危也。

帝曰:人年老而无子者,材力尽邪,将天数然也。歧伯曰:女子七岁,肾气盛,齿更发长;二七而天癸至,任脉通,太冲脉盛,月事以时下,故有子;三七,肾气平均,故真牙生而长极;四七,筋骨坚,发长极,身体盛壮;五七,阳明脉衰,面始焦,发始堕;六七,三阳脉衰于上,面皆焦,发始白;七七,任脉虚,太冲脉衰少,天癸竭,地道不通,故形坏而无子也。丈夫八岁,肾气实,发长齿更;二八,肾气盛,天癸至,精气溢写,阴阳和,故能有子;三八,肾气平均,筋骨劲强,故真牙生而长极;四八,筋骨隆盛,肌肉满壮;五八,肾气衰,发堕齿槁;六八,阳气衰竭于上,面焦,发鬓颁白;七八,肝气衰,筋不能动,天癸竭,精少,肾藏衰,形体皆极;八八,则齿发去,肾者主水,受五藏六府之精而藏之,故五藏盛,乃能写。今五藏皆衰,筋骨解堕,天癸尽矣。故发鬓白,身体重,行步不正,而无子耳。帝曰:有其年已老而有子者何也。歧伯曰:此其天寿过度,气脉常通,而肾气有馀也。此虽有子,男不过尽八八,女不过尽七七,而天地之精气皆竭矣。帝曰:夫道者年皆百数,能有子乎。歧伯曰:夫道者能却老而全形,身年虽寿,能生子也。

黄帝曰:余闻上古有真人者,提挈天地,把握阴阳,呼吸精气,独立守神,肌肉若一,故能寿敝天地,无有终时,此其道生。中古之时,有至人者,淳德全道,和于阴阳,调于四时,去世离俗,积精全神,游行天地之间,视听八达之外,此盖益其寿命而强者也,亦归于真人。其次有圣人者,处天地之和,从八风之理,适嗜欲于世俗之间,无恚嗔之心,行不欲离于世,被服章,举不欲观于俗,外不劳形于事,内无思想之患,以恬愉为务,以自得为功,形体不敝,精神不散,亦可以百数。其次有贤人者,法则天地,象似日月,辩列星辰,逆从阴阳,分别四时,将从上古合同于道,亦可使益寿而有极时。
