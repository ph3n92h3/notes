\subsection{四气调神大论}

春三月,此谓发陈,天地俱生,万物以荣,夜卧早起,广步于庭,被发缓形,以使志生,生而勿杀,予而勿夺,赏而勿罚,此春气之应养生之道也。逆之则伤肝,夏为寒变,奉长者少。
夏三月,此谓蕃秀,天地气交,万物华实,夜卧早起,无厌于日,使志无怒,使华英成秀,使气得泄,若所爱在外,此夏气之应养长之道也。逆之则伤心,秋为痎疟,奉收者少,冬至重病。
秋三月,此谓容平,天气以急,地气以明,早卧早起,与鸡俱兴,使志安宁,以缓秋刑,收敛神气,使秋气平,无外其志,使肺气清,此秋气之应养收之道也,逆之则伤肺,冬为飱泄,奉藏者少。
冬三月,此谓闭藏,水冰地坼,无扰乎阳,早卧晚起,必待日光,使志若伏若匿,若有私意,若已有得,去寒就温,无泄皮肤使气亟夺,此冬气之应养藏之道也。逆之则伤肾,春为痿厥,奉生者少。

天气,清净光明者也,藏德不止,故不下也。天明则日月不明,邪害空窍,阳气者闭塞,地气者冒明,云雾不精,则上应白露不下。交通不表,万物命故不施,不施则名木多死。恶气不发,风雨不节,白露不下,则菀稾不荣。贼风数至,暴雨数起,天地四时不相保,与道相失,则未央绝灭。唯圣人从之,故身无奇病,万物不失,生气不竭。逆春气,则少阳不生,肝气内变。逆夏气,则太阳不长,心气内洞。逆秋气,则太阴不收,肺气焦满。逆冬气,则少阴不藏,肾气独沈。夫四时阴阳者,万物之根本也。所以圣人春夏养阳,秋冬养阴,以从其根,故与万物沈浮于生长之门。逆其根,则伐其本,坏其真矣。

故阴阳四时者,万物之终始也,死生之本也,逆之则灾害生,从之则苛疾不起,是谓得道。道者,圣人行之,愚者佩之。从阴阳则生,逆之则死,从之则治,逆之则乱。反顺为逆,是谓内格。是故圣人不治已病,治未病,不治已乱,治未乱,此之谓也。夫病已成而后药之,乱已成而后治之,譬犹渴而穿井,鬭而铸锥,不亦晚乎。