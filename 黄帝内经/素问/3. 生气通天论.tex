\subsection{生气通天论}

黄帝曰:夫自古通天者生之本,本于阴阳天地之间,六合之内,其气九州九窍五藏十二节,皆通乎天气。
其生五,其气三,数犯此者,则邪气伤人,此寿命之本也。
苍天之气清净,则志意治,顺之则阳气固,虽有贼邪,弗能害也,此因时之序。
故圣人传精神,服天气,而通神明。
失之则内闭九窍,外壅肌肉,卫气散解,此谓自伤,气之削也。

阳气者若天与日,失其所,则折寿而不彰,故天运当以日光明。
是故阳因而上,卫外者也。
因于寒,欲如运枢,起居如惊,神气乃浮。
因于暑,汗烦则喘喝,静则多言,体若燔炭,汗出而散。
因于湿,首如裹,湿热不攘,大筋緛短,小筋弛长,緛短为拘,弛长为痿。
因于气,为肿,四维相代,阳气乃竭。

阳气者,烦劳则张,精绝辟积,于夏使人煎厥。目盲不可以视,耳闭不可以听,溃溃乎若坏都,汨汨乎不可止。
阳气者,大怒则形气绝,而血菀于上,使人薄厥。
有伤于筋纵,其若不容,汗出偏沮,使人偏枯。
汗出见湿,乃生痤痱。
高梁之变,足生大丁,受如持虚。
劳汗当风,寒薄为齄,郁乃痤。

阳气者,精则养神,柔则养筋。开阖不得,寒气从之,乃生大偻。
陷脉为瘻,留连肉腠。俞气化薄,传为善畏,及为惊骇。
营气不从,逆于肉理,乃生痈肿。
魄汗未尽,形弱而气烁,穴俞以闭,发为风疟。
故风者,百病之始也,清静则肉腠闭拒,虽有大风苛毒,弗之能害,此因时之序也。
故病久则传化,上下不并,良医弗为。
故阳畜积病死,而阳气当隔,隔者当写,不亟正治,粗乃败之。

故阳气者,一日而主外,平旦人气生,日中而阳气隆,日西而阳气已虚,气门乃闭。
是故暮而收拒,无扰筋骨,无见雾露,反此三时,形乃困薄。

歧伯曰:阴者,藏精而起亟也,阳者,卫外而为固也。
阴不胜其阳,则脉流薄疾,并乃狂。
阳不胜其阴,则五藏气争,九窍不通。
是以圣人陈阴阳,筋脉和同,骨髓坚固,气血皆从。
如是,则内外调和,邪不能害,耳目聪明,气立如故。

风客淫气,精乃亡,邪伤肝也。
因而饱食,筋脉横解,肠澼为痔。
因而大饮,则气逆。
因而强力,肾气乃伤,高骨乃坏。
凡阴阳之要,阳密乃固,两者不和,若春无秋,若冬无夏,因而和之,是谓圣度。
故阳强不能密,阴气乃绝,阴平阳秘,精神乃治,阴阳离决,精气乃绝,因于露风,乃生寒热。

是以春伤于风,邪气留连,乃为洞泄。
夏伤于暑,秋为痎疟。
秋伤于湿,上逆而咳,发为痿厥。
冬伤于寒,春必温病。四时之气,更伤五藏。

阴之所生,本在五味,阴之五宫,伤在五味。
是故味过于酸,肝气以津,脾气乃绝。
味过于咸,大骨气劳,短肌,心气抑。
味过于甘,心气喘满,色黑肾气不衡。
味过于苦,脾气不濡,胃气乃厚。
味过于辛,筋脉沮弛,精神乃央,是故谨和五味,骨正筋柔,气血以流,凑理以密,如是,则骨气以精,谨道如法,长有天命。
