\subsection{腰痛问题、虚胖和瘦弱问题、保持警惕的重要性}

\paragraph*{前言}

最近提问白发问题的戒友有不少,白发要恢复大概需要一年左右,经常看我经验帖的戒友,应该会看到一个案例,有个戒友是戒色两百多天白发全部变黑,而我当初是用了一年多时间,现在我的白发极少,大概一到两根,完全属于正常范围。白发要恢复唯有坚持,并且要:一,严格控制遗精次数;二,尽量杜绝 YY;三,注重养生之道。这三方面的功夫到了,白发自然是能恢复的,就怕很多人戒色了,遗精控制不了,YY 控制不了,养生方面又不懂,而且还经常破戒,这样白发要恢复,难度较大。

有的戒友,特别是症状还不算严重的戒友,会觉得一个戒友有十几种症状匪夷所思,这样的戒友我只能说阅历尚浅体会不深,我到现在看过几千个戒友案例,亲自沟通的戒友上千个,其中大部分的人其实身体的症状都很多,只是挑了几个比较明显的症状来问我。多看看受害者案例其实就明白了,有十几种乃至几十种身体不适的戒友大有人在。有的戒友感觉很细腻很敏感,能把那些不适的感觉都描述出来,而有的戒友后知后觉或者表达能力有限,无法把全部的不适都描述出来。那些伤得不重的戒友,或者后知后觉的戒友,就会觉得不可思议,这其实很好理解,当你伤到那种程度,自然会明白我说的都是真的,没体会过的人无法理解那种感受。你没吃过盐,就永远不知道盐的具体味道,听别人描述再多也显得苍白。

另外我再强调点,我并不是一开始戒色就成功,我刚开始戒色在初中时代,我那时什么都不懂,又没学习戒色文章的意识,又没人指点我,我那时就是瞎戒和强戒,根本不懂如何专业戒色,结果就是屡戒屡败,失败过无数次,在怪圈中挣扎了很多年,搞得一身的症状,如果我刚开始就能戒掉,就不会有以后那么多症状了。可惜我在十几年后才觉悟到学习戒色文章的重要性,还好不算太晚,现在身体经过几年的调理也已经恢复正常,当然和十几岁时无法比,因为现在我都三十岁了。

关于动机我再阐述下,我写经验帖,一不为名,二不为利。完全是公益性质的,不求任何回报,我也不需要任何人崇拜我,有的戒友会叫我大师,我从来没放在心上,我和各位戒友都是平等的,我就像一个指路人,指路人和问路人是平等的关系,而不是居高临下的关系,我就是这样的一个平凡的指路人,你要问我如何戒色,我可以给你指一条明路。我希望帮到大家,仅此而已。

下面步入正题。这季就腰痛问题、虚胖和瘦弱问题、保持警惕的重要性详细论述一下,具体如下。

\subsubsection{腰痛问题}

中医:腰为肾之府。

腰痛问题在戒友当中非常普遍,相信大部分的戒友都有腰酸腰痛的经历,SY 后身体的一个表现就是:腰膝酸软,严重点的就会出现疼痛症状。当然出现腰痛的原因很多,一般 SY 导致的腰酸腰痛,戒掉一段时间并且注意休养,腰酸腰痛的情况是会慢慢缓解的,如果你的腰痛问题持久困扰你,我建议你最好去医院做个检查,看看是否有器质性的病变,比如腰椎盘问题和肾结石等。生活中有腰椎盘突出的人还是很多的,比如在搬运重物时不注意正确的姿势,就可能落下病根。

SY 后的腰痛还有一个表现,就是在弯腰久了以后,直不起来,感觉异常酸痛,我以前 SY 后洗衣服,每次洗完腰都非常酸痛,要休息一段时间才能缓过来,所以 SY 后会让你的腰变得很脆弱,本来弯腰不会出现酸痛问题,但由于你有 SY 恶习,弯腰久了就会容易出现酸痛的表现。这种弯腰久了,腰部出现酸痛的现象,有很多戒友都有反映过,我自己也有切实的体会,另外,SY 后人就软掉了,这时候去运动就要格外小心,因为在软掉的身体状态下从事剧烈运动,是非常容易受伤的,这个我也深有体会,以前频繁 SY 后去打篮球,骨折过三次,崴脚无数次。而且那种受伤都很莫名其妙,是自己把自己给扭伤了。

对于腰痛问题,其实很好解决,如果没器质性病变,一般戒色后注意休养,腰痛问题一般会不药而愈的。如果腰酸腰痛问题持续,建议去做个检查,没器质性病变就配合中药调理一下,如果有器质性病变,那就积极治疗,并且坚持戒色和养生,这样恢复才比较快。男人的腰实在是太重要了,一定要好好保护好自己的腰。

\subsubsection{虚胖和瘦弱问题}

再来谈下虚胖和瘦弱问题。

SY 后一般会出现失调情况,表现在体重上,不是虚胖就是瘦弱。当然有的人虽然 SY,但有良好的运动习惯和饮食习惯,倒不一定会出现这个问题。我以前虽然频繁 SY,但我热爱运动,在学校的时候作息饮食相对比较规律,就没出现过体重失调的状况。

\paragraph{虚胖}

在中医上虚胖一般有两种原因:脾虚湿阻和脾肾阳虚。

一个人如果热爱运动,他的胖也不会是虚胖,虽然胖,但感觉也很强壮,并不是那种肉很松垮的虚胖。为什么热爱运动的人一般不会出现虚胖呢?因为中医:脾主四肢。\textit{脾主身之肌肉(《素问》)。} 即脾气健运,则肌肉丰盈而有活力。如脾有病,则肌肉萎缩不用。一个人热爱运动,可以起到健脾的作用,这样出现虚胖的可能性就比较小,脾气健运,肌肉就会富有弹性有活力。很多人会认为虚胖是脂肪的问题,其实不然,虚胖其实是脂肪和肌肉都软塌塌的,感觉没有弹性,很松垮的感觉。

要如何治疗虚胖呢?吃中药当然是一种方法,可以调理你的内脏。但最关键的还是建立良好的运动习惯和饮食习惯,这点太重要了!你可以不吃中药,但只要做好后面两点,依然可以把体重减下来,甚至可以变成有型的肌肉男。我以前曾做过健身教练,在减脂方面积累了丰富的经验,自己也有减脂的亲身经历。如果你有点健身的常识,肯定会知道一句话:三分练,七分吃。要增加肌肉,光练不行,关键是吃练都要到位,这样增肌才比较快。其实这句话对于减脂也是一样的,很多会员过来一顿狂跑狂练,但是回家后无法管住自己的嘴,吃得太多,把在健身房消耗的热量又给全部吃回来了,而且脾主四肢,一般运动完,胃口都会变好,特能吃,结果是怎么练怎么跑,就是瘦不下去,其实如果能严格控制饮食摄入,不需要练得太苦,体重一样会下来。

对于虚胖戒友,我的建议就是:

\begin{itemize}
    \item 积极锻炼,以有氧运动为主,可以再做些力量训练。
    \item 严格控制饮食摄入,高热量的食物,夜宵零食等都要忌口。
\end{itemize}

这两点做到位了,你的体重肯定会下来,如果经过一个阶段的努力,体重没下来,你就要反省这 2 点哪点没做好,没做到位。

另外,做些力量训练可以让你的肌肉更有型,很多戒友会迷恋力量训练,会想让自己变得更强壮,更 MAN,这种想法相当普遍,强壮的身体的确能给人一种安全感,而且自己也更自信,一般强壮的男人性能力也更强。但请记住一句话:福兮祸所伏!中医讲究欲不可强。很多人练强壮了,反而更会乱来,结果是年轻时放纵,四十岁以后就要买单了,到时候就苦大了。强壮并不等于健康和长寿,很多强壮的人都有很多毛病,甚至活不过五十岁,而很多注重养生的人,力量指标很低,但依然可以活到八十岁以上。

对于虚胖的戒友,我还有一点建议就是,你想减脂,必须让自己懂得更多,这和戒色一样,都讲究专业。你有了专业知识,就更容易入门,更容易成功。学习在任何一个行业,任何一件事上,都显得异常重要,不学不知道,学了就懂了,再加上自己的不断实践,不断领悟,要成功就不难了,所谓会者不难,难者不会,必须多学习。等你掌握了足够多的减脂专业知识,很多事情就会变得简单,如果你懂得不够深刻,在很多选择上就会犹豫或者茫然。

\paragraph{瘦弱}

还有一类戒友会出现瘦弱的情况。

瘦弱和体质有一定关系,当然和 SY 导致的内脏功能紊乱也有密切联系,瘦弱的人胃口不好,吃不下去,也不消化,怎么吃也长不胖,这是一类;还有一类,就是很能吃,但就是吃不胖,这种人很可能消化吸收不好,或者是基础代谢太厉害。还有的戒友慢性腹泻,这种情况就更容易掉体重了。

瘦弱的人建议看中医调理下,然后积极锻炼,注重养生之道,慢慢是能调理过来的。

\subsubsection{保持警惕性的重要性}

最后谈下保持警惕性的重要性。

不少戒友戒了上百天,结果还是破了,原因是什么?很多戒友在成功戒除一段时间后,容易出现放松警惕的现象,自以为成功了,没问题了,于是放松了,一,是放松对戒色文章的学习。二,就是放松了警惕。人一旦放松了警惕,就容易破戒,戒色就像走钢丝,是要每天保持警惕的,因为一不小心,你就下去了,一不小心,你就破戒了,好像鬼使神差一般。

我戒到现在依然每天保持警惕,就像麻雀的警觉,人走近麻雀,麻雀就飞掉了,麻雀是不会让人靠近它们的。戒色也是如此,戒到一定时间进入稳定期后,并不意味着不会破戒了,你只要放松警惕,依然会破戒。所以我们要让自己保持在警惕的状态,不要离开戒色文章,应该多看受害者案例。这种习惯最后就像每天刷牙一样,不一定需要多长时间,但你每天看看戒色内容,绝对有利于你保持住警惕的状态。这对于彻底戒色是必须需要的意识。

当然有的人觉悟极高,他既不看戒色文章,又不看受害者案例,他每天把自己过去的痛苦回忆一遍,这样也可以让他保持警惕。还有一种人也可以保持警惕,那就是有宗教信仰的人,戒色属于戒律的一种,如果他严守戒律,也是不会去破戒的,因为他知道,破戒比死难受,他有这个觉悟。

\paragraph*{结语}

有戒友会问到,为何我周围的人都有 SY,为何他们身体还好,我怎么就一身的症状。这个问题有不少戒友问到过,这个也不难理解,人的体质因人而异,有的人先天体质很好,底子厚,又热爱运动,作息饮食规律,没有其他不良嗜好,这种人可以伤个几年,恶果出来相对较晚。而有的人天生体弱多病,体质不佳,又不爱动,喜欢久坐有网瘾或者烟瘾,经常熬夜,再加上频繁 SY,这类人出现症状就比较早,也比较严重。

还有一点,很多人虽然表面上看着还好,其实他也有很多症状,只是表面上你看不出来而已,有一个词叫“暗疾”。生活中有暗疾的人相当多,这都属于隐私,他不说,你还以为他很健康,其实不然。就拿我来说吧,我那时有前列腺炎,但别人并不知道,我也没跟朋友说过,而我在运动场上也生龙活虎,别人都以为我身体很好,其实不然,只有我自己知道我因为染上 SY 恶习,身体出现了很多不适症状,只是这些症状不是很严重,还能忍着,戒一段时间,这些症状都会有所缓解,所以那时并未引起足够的重视。而且,我得精索十几年,自己都不知道,因为我是轻微精索,并无特别的不适感,但已经得上了,去检查 B 超才知道得上了。关于精索问题,我前面写过一篇文章,不少戒友看过那篇文章就去检查了,结果不少戒友都检查出了精索,只是以前自己并不知道。

其实,因为 SY 恶习,多多少少都会出点症状的,即使你体质超好,没有其他不良嗜好,但随着你年龄的上升,很多潜伏的问题就会暴露出来,出来 SY 迟早是要还的。身体早垮掉未必是坏事,因为在你年轻时就垮掉,然后觉悟后彻底戒掉,身体的恢复能力还是不错的,就怕是四十岁以后身体垮掉,那时身体的恢复能力已经大不如前了,恢复的速度就会很缓慢。

再来谈谈夏季做固肾功出汗的问题,现在炎炎夏日,稍微一动就汗流浃背,有时即使什么都不动,汗水都像下雨一样。我现在依然每天坚持练习固肾功,没有一天间断的,我的做法就是做的时候赤膊,并且准备好一条毛巾,做一组流汗了,马上擦掉,然后再做,这样出汗的困扰就能避免,否则刚洗完澡,一做固肾功,衣服又湿了,很不舒服,也不利于入睡。另外,夏季尽量少吹空调,吹空调吃冷饮都伤阳气,对身体的恢复很不利。切记。

最近有戒友向我反映“天数党”的问题,我个人倒不反对每天签到,但我觉得签到就像是打卡上班,你不能来打卡却不上班,具体对应到戒色上,就是你来签到,签到的目的是什么呢?签到的目的,其实就是每天学习戒色文章和多看受害者案例警醒自己,不能签到了就完了,那样你的觉悟和定力并没有提升,还是停留在强戒的层次。

一般新人容易犯这个错误:光签到不学习,看到别人开帖签到,自己也开个帖签到,却不注重戒色文章的学习,我当年戒色,也喜欢记录天数,那时是记录在笔记本上,我那时最多熬了 28 天,后来我彻底觉悟后,根本没想过天数,也没有了煎熬的感觉,戒到现在都没破过,因为我彻底觉悟了,并且每天保持警惕意识。我不记得我戒了多少天,但我记得我从几月份开始戒的,我的建议就是:不要太在意天数,而是要注重学习来提高戒色觉悟和戒色意识,这才是戒色的正道。
