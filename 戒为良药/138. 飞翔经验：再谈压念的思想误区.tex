\subsection{再谈压念的思想误区}

\paragraph{前言}

这季前言分享几个案例。

\begin{case}
    前段时间的频繁破戒使我的心脏又出现了问题,不定时的刺痛,惶恐,其他的毛病也出现,尿频,尿余沥不尽,眼睛迎风流泪,肿胀的感觉,太阳下都睁不开,频繁出错,做不好一些简单的事情,记忆力下降,上个楼梯弯个腰突然闪痛,手脚心发烫,多梦,阴虚火旺,昏昏沉沉没有精力,痘痘爆发,气色很差,乱七八糟的症状说不完,总之戒友们加油吧,引以为戒。之所以没出现那么多症状是因为还没伤到那个临界点,过了临界点真的就苦大了。我想戒色一是深刻明白邪淫的危害,让自己预见未来,看到继续破戒的下场;二是看一些恢复帖和经验帖,清楚明白戒色对身体以及生活各个方面的好处;三是练习断念,觉察念头,怂恿类、图像、回忆、YY 全部要断掉,如果前两项不学习的话,后面断念也断得不利索,犹犹豫豫。
    \subparagraph{附评} 现在是冬季,冬季破戒伤害更明显一些。冬藏精,要加强养生之道,避免破戒。冬天气温下降,本来就容易出现症状反复,如果再加上频繁破戒,身体肯定受不了。破戒后不要过于自责,自己一定要注意休养,在破戒后认真反省和总结,平时一定要做好日课,有一个好的日课,才能保持住良好的戒色状态。症状爆发是有一个临界点,过了那个临界点,各种症状就来了,到时苦不堪言,心脏也不好了,脾胃也不好了,头部也难受了,心理也失调了,精力也不行了,脑力也下降了,前列腺也发炎了,各种不适。破戒就像从黄金变成铁渣,这是这位戒友后来和我说的,后来他加强了养生——食疗和有氧运动,坚持了快二十天,活力又回来了,身体症状改善很多,毕竟年轻,恢复速度比较快,加上会养生,症状恢复很多。他自己也总结了,戒色必须认识危害、学习经验帖、做好断念,这三样是必不可少的。深刻认识危害,才有决心来戒;学习经验帖,才能提升觉悟;做好断念,才能做到不破。失败是暂时的,谁没失败过?关键要学会从失败中崛起,避免犯类似的错误,不断强大自己的实力,这样才能越戒越好。
\end{case}

\begin{case}
    今天是我戒色 202 天的日子了,很开心我能戒这么久。通过这段时间的戒色,我的身体确实好了不少,戒色前我身高 172,体重只有 110 斤左右,看上去像营养不良的病人。通过戒色加锻炼,短短几个月的时间我的体重已经涨到了 126 斤,涨了整整 16 斤,整个人看起来给人很结实的感觉。戒色这段时间,我的精神也比以前好很多,失眠的现象也减少了,胆子也变大了,敢在课堂上主动回答老师的问题了。最让我开心的是我的精索静脉曲张也好了很多,戒色前我的左侧精索静脉曲张很严重,肉眼可以看到蚯蚓状的静脉,在戒了大概三个月左右的时候,我惊喜地发现我用肉眼已经看不到蚯蚓状的静脉了。虽然我知道我的精索静脉曲张还没有完全好,但是这的确是我身体变得更健康的一个好兆头。戒色这段时间我也有很多次想要破戒,但是都被我自己给努力克制下来了,我觉得这和我经常看《戒为良药》有很大关系,《戒为良药》给了我很大的帮助,也让我对戒色有了更深一步的了解。所以我希望新来的戒友都可以去看看《戒为良药》,相信我你绝对会受益匪浅。还有我觉得戒色最重要的一点,就是要做到念起即断,念起不随,念起即觉,觉之即无,在刚有欲望出来的瞬间就把它给 pass 掉,很多戒友可能会觉得这很难,但其实你只有多看看《戒为良药》和戒色文章,平时多加练习,时间久了你自然就会很容易断了邪念。戒色道阻且长,我们一起努力,做一个堂堂正正的男人!!!
    \subparagraph{附评} 学生党戒友能戒 202 天,很不错!整体改善明显,身心都朝着好的方向在变化,精气神上来了,人真的不一样,从里到外都不一样,那种自信和底气,那种活力和精力,真的太棒了。精气神是男人最好的化妆品,正气是男人最好的外衣!前几天看到一个戒友发帖分享自己的精索静脉曲张恢复情况,一侧消失,一侧缩小。坚持戒色养生,精索静脉曲张是有望缩小和消失的,能缩小已经不错了,这说明戒色的确是有很大好处的,对于缓解疾病有着明显的效果。这个案例的戒友,他的精索也缩小了,这的确是一个好兆头,说明身体在恢复了,也可以说是一个喜讯,让自己更有信心坚持戒色。古人云:“人生功名事业,寿夭穷通,皆自少年基始,而戒淫为第一。盖血气未定,最易沾涉邪淫。迨至日复老成,虽知怨悔,已追悔莫及。”古人很有戒色意识,懂得戒色的重要性,这是在守护自己最宝贵的能量!肾精是男人最重要的战略资源!这种资源,岂能被色情所抢夺?!有精无精,天壤之别!人不在帅,有精则灵!有精,整个人都被盘活了,就像汽车有了油一样,汽车没油,就像一堆废铁,人没精,就是一个废人。有了能量,就有精力去干事,就能干成很多事,就有那种强大的斗志和勇气,所以戒色保精实乃人生第一修为。
\end{case}

\begin{case}
    今天戒色 195 天了,恢复情况如下:\begin{itemize}
        \item 我的精神恢复得特别好,没戒色前我整天没精神,整天想睡觉,戒色到现在我每天都很有精神。
        \item 脑力恢复得也特别好,从前记不住的东西戒色后都能记住了。从前没耐心看书,现在有耐心看好书了。
        \item 我以前有社恐,走在路上都不自在,总是感觉有人在看我,戒色 195 天我彻底恢复了。
        \item 戒色前严重失眠,戒色到 195 天我彻底恢复了。
        \item 戒色前特别痛苦,感觉活着太痛苦了,戒色后我每天都能感觉到纯净的大快乐,我才知道原来人活着不邪淫的生活这样美好。
        \item 戒色后我脾气变好了很多,戒色前我跟别人是三天一小吵,十天一大吵,几月一次干架。看谁都不顺眼,就想对人发脾气。戒色到现在我脾气明显变好了。人际关系变好了,就在戒色这 195 天内我结交了几个真心的好朋友。
    \end{itemize} 还有很多戒色后恢复的症状,我只写这几个比较重要的吧,我想说的是,我戒色 195 天后我整个人都变好了,这种改变是从内到外真正的改变。所以不要质疑戒色,不要听所谓的无害论。那些说无害论还以邪淫为荣的人只是还没有伤到我们的程度,等伤到我们的程度他哭都来不及了!只有坚持戒色,我们才能王者归来,才能英雄崛起,做回到真正的铁血真男儿,才能逆袭人生,改变贫贱早死的命运!成功戒色十年后你回头看,你会发现选择戒色是你一生最最英明神武的决定,没有之一。因为戒色使你从撸管病夫变成了英雄,王者归来后真正改变了自己的命运。生当做人杰,死亦为鬼雄。活好这一回,别留下遗憾!加油!
    \subparagraph{附评} 戒色 195 天,这位戒友恢复情况很可观,他的这些症状都比较常见,很多人都有,好好坚持戒色养生,慢慢症状就会改善,很多好变化在悄然发生,到时身心都会焕然一新。纯净的大快乐多么美好,看看那些小孩子,每天开开心心,无忧无虑,活得多么纯真美好。我们也可以回归那个纯净纯善的心灵状态,重拾遗失的美好与快乐,那种快乐显然与性无关,这是纯净灵魂独有的大快乐,因为纯净,所以快乐。这个世界上还有比手淫爽百倍的事情,那就是做纯净的自己,安住于真我。多么快乐,多么美好,多么自由。一位戒友说:“戒了五十天我感觉变化好大,首先是人自信了,在学校一晚上睡五个小时第二天照样有精神,头发也茂密了,耳鸣好了,人际关系也好了很多,学长还说我变帅了,正能量满满。”另外一位戒友说:“今天去银行办理个业务,戒色 42 天了,可以坦然自在地坐在小姐姐面前,对答如流,那一刻就感觉,一切的付出都是值得,戒色让生活更美好。”坚持戒色,人际关系也会随之改善,因为内心祥和了、喜悦了,人际关系自然就改善了,那种底气和自信,可以让人光明磊落地和人交流。邪淫是苦不是乐,邪淫的路越走越窄,越走越苦!戒色的路越走越宽,越走越光明,越走越开心。还有一位戒友说:“我现在感觉到,其实接触大自然比接触黄更快乐,那种太阳生起的信心,站在高山上的豪迈,看到树木的宁静,柳树垂条的摆动,湖波荡漾的感觉,嫩绿的小草,奇花异草的绽放等等,有时候一阵清风扑面而来,都能特别欢喜和满足,这在接触黄的过程中是无法体会的,有时候我们真的要多学习孩子,他们那个时候先天意识还比较强,是那么纯真善良美好!”那种发自内心的欢喜和满足,是纯净的孩子才能感觉到的,在戒色后有望重新找回这种美好的感觉,多么可贵。总有人相信无害论,等到症状爆发,就会知道无害论是弥天大谎,到时就知道戒色了。真正的智者都在戒色,真正的愚夫则在邪淫,还为自己找很多冠冕堂皇的借口,最后被症状打脸,就知道真相了。这位戒友说得很好,从病夫到英雄,上演王者归来!不忘初心,牢记使命,正己化人,无私奉献!做一个真正纯粹的利他行者!
\end{case}

\begin{case}
    回首这几个月的历程,学习《戒为良药》获益良多,《戒为良药》让我学会如何克服自己欲望,如何控制自己的念头,教我如何做人,这些知识一般人是看不到的,也可以说这可能就是我人生最大的财富吧!我的内心变得更自由更快乐了,每天都有大快乐充斥着我,莫名其妙地就感到自由轻松快乐,比起邪淫时整天压抑担心受怕,简直是一个天一个地,心境完全不一样,现在我很快乐,这种大快乐也不是每分每秒都有,因为有时也因为生活琐事、情绪不好等原因导致大快乐暂时消失,但大部分时间还是快乐的,无比感恩飞翔前辈的《戒为良药》再造之恩!
    \subparagraph{附评} 戒色能让人重获“无因之乐”!这四个字是我最近看修行的文章看到的,我顿觉很好。小时候,我们常常处于无因之乐的状态,没有特别的原因,就是内心很开心、很快乐,伴随纯净的感觉,还有一种深深的满足和愉悦,多么美好的感觉,内心感到自由和快乐。小孩子就在这个状态,等到发育了,内心开始起邪念了,开始看黄了,就从那个天堂般的状态失落了,变得很压抑,很痛苦,内心很难开心起来,脸上也没了纯真无邪的笑容,活得很惶恐,等到症状爆发,身心都备受煎熬。真的是一个天,一个地,我深深体会过两者的区别,撸管貌似爽到了,其实亏大了,撸完就会觉得没意思,感到空虚和悔恨,人也会变得脾气暴躁,充满戾气和负能量,各种不顺和倒霉就跟着来了。\textit{专气致柔,能婴儿乎?(《道德经》)} 老子为何推崇“婴儿”?因为婴儿的心灵处于非常纯净的状态,那种状态是和道相合的,是充满大快乐的,一旦追求欲望、追求快感,就会从那个纯净美好的状态中跌落下来,以为快感是快乐,于是拼命追求快感,其实快感不是真正的大快乐,快感是玻璃,真正的大快乐是钻石!两者是有区别的。大家发现没有,人在感受快感的时候,很少会笑,因为那不是一个真正快乐的状态,笑不出来。完事后,内心状态很糟糕,就像被抢劫了一样,两个腰子空空如也,心魔大笑而去,留下一脸沮丧的撸者。这不是享受,这是被奴役的状态。远离色情,戒掉恶习,做回纯净美好的自己,到时就能再次感受到纯净的大快乐了,那是久违的童年感觉,在纯净的简单中蕴藏着异常丰富的美好体验,再次睁开清澈明亮的眼睛,用纯真的眼神看着这个世界,突然,你笑了,没有原因,无因之乐!哈哈!笑得就像一个纯真无邪的孩子,那么开心,那么喜悦,那么孩子气,仿佛自己是全天下最开心的一个人。当我们内心真的纯净了,你就会发现天空很蓝,草儿很绿,花儿很美,就连空气都弥漫着快乐的味道。\textit{成为一个喜悦的人,是你能为自己和身边人做的最好的事。……愿你知晓这种满足感——你让所触及的一切都变得快乐。(萨古鲁)} 当你喜悦了,开心了,别人也会受到感染,大家一起开心,把这种喜悦的能量传递开来,人际关系就会变得和谐融洽,其乐融融。
\end{case}

下面进入正文。

这季是关于压念的,压念是修心方面比较常见的误区,很多大德都提到过,修心不是压制念头,而是化解念头。在真正学会修心前,很多人都有过压念的思想误区,在实战中也会试图压念,这样只会导致烦恼和挫败。之前关于压念的问题,我已经多次讲到过了,有的戒友还不是很明白,这季会再详细深入地讲一下,这季会把压念的问题彻底讲透,通过研读这一季,大家都会知道压念到底是怎么一回事,从而避免出现压念。有些人把断念误解成了压念,当然戒不好,压念是错误的操作,只有明白了什么是真正的断念,什么是压念,才能避免陷入误区。

\paragraph{什么是压念}

所谓压念就是主观想压制念头,不想让念头起来,有这种压制的想法,就是压念。正确的操作是觉察念头,不是压制念头。打个比方,比如有两个人看到地上有一个弹簧竖在那,甲用脚去踩弹簧,他想压住弹簧,不想让弹簧起来,结果越压,反弹越强,他感到挫败。乙没有去踩弹簧,而是找来了电焊工具,瞬间熔化了弹簧。大家看到这,应该都明白了吧!甲有压弹簧的想法,乙没有,乙用的是另一种方法,瞬间让弹簧消失。这里的弹簧好比就是念头,不要去压,而是觉察,觉察即消灭,觉察即降伏,念头来了,也可以马上念佛持咒,或者思维对治等,总之不能让念头连续下去,不能让念头起势,要斩绝萌芽!

再举几个类似的比喻:\begin{itemize}
    \item 压念就像你想刹车,却错把脚放在了油门上,结果可想而知,压念是错误的操作。
    \item 家中进了一个强壮的小偷,无法压制他,因为他很强壮,他会极力反抗。只需去看他,他就做贼心虚,自己跑掉了。
    \item 火烧起来了,这时不能用一堆纸去压火,而是要用灭火器!
\end{itemize}

通过这几个比喻,大家应该都知道压念是怎么一回事了。很多人搞不清什么是压念,看了这几个比喻,应该一目了然了。

记住:不能压,而是看!这是两个动作,压是压不住的,会导致挫败和烦恼,而看很轻松,一觉即空,一觉即灭。刚开始做不到觉察即消灭,发现邪念上头了,可以马上念口诀,或者念佛持咒,功在平时,平时练到条件反射的程度,到时邪念上头时,就能立刻断除。

\paragraph{我的压念经历}

我初中站养生桩时,说要排除杂念,于是念头一来,我就想压住念头,不想让念头起来,结果念头跑得更凶,冒得更猛,根本压不住,我感到很挫败,后来用了数息法摄念,勉强能减少妄念。我那时不会觉察,不懂得如何正确断念,所以那个阶段的我因为压念感到了挫败。

看很多大德的文章,都会提到降伏、战胜、征服自己的妄念,这类词汇并没有错,修行的确是要断除妄念的,关键要正确理解,避免出现压念。刚开始很容易出现压念,当发现念头起来了,就想压住它,不让它起,结果起得更猛。压念是完全错误的操作,正确的操作是看,是觉察,或者念口诀、念佛来转,也可以思维对治。

\paragraph{为什么压念是错的}

因为压念会导致念头反弹,会让人感到压抑和挫败,会感到焦虑和紧张。压念无法让念头消失,反而会让念头跑得更凶!

\paragraph{不要抗拒念头,不要害怕念头}

如果对念头有抗拒心理,那就容易出现压念,不要抗拒念头,而是懂得通过觉察来化解念头。也不要害怕念头,古德云:“不怕念起,就怕觉迟。”要从容应对,镇定自若。

\paragraph{不压、不随}

\textit{念起即觉,不压不随。(元音老人)} 一觉就完事了,一点不压抑,很轻松。我现在断念很有把握,也比较轻松,一记觉察就搞定了,念头疯狂进攻时,几记觉察的组合拳就把心魔打趴下了。

\paragraph{如何避免压念}

\begin{itemize}
    \item 要认识到压念是思想误区,是错误的操作;
    \item 不要抗拒念头;
    \item 强化觉察力,不要去压,而是去看,去觉察念头。
\end{itemize}

\paragraph{断念是化解,而不是压制}

\textit{愿我恒时观此心,烦恼妄念初生时,毁坏自己他众故,立即强行而断除。(《修心八颂》)} 观心断念是化解,而不是压制念头。很多人误解了断念,以为断念就是压制念头,结果就变成了压念,在实战中感到很压抑、很挫败,断念是化解念头,觉而化之,这点要明白。压是压不住的,而觉察可以化解之。

\paragraph{倓虚大师的经历}

\begin{quotation}\it
    有一次,我到谛老那里去请教,顶完礼之后,他老先问:“你用的功夫怎么样?”

    “没别的!”我说,“最初坐的时候,妄想直起,像海里的波浪一样,前浪逐后浪,后浪逐前浪的不断,抑制也抑制不住,心里很著急。后来我不抑制它,反而用观照力来观它,观看妄想究竟从何处起?这样一观,妄想就没了;没了又起,起来再观。时间长久,慢慢的妄想就不起了,心里也很恬静、很自然了。”

    “嗯——”谛老说,“你算会用功咧!就这样好好回去修吧,以后可以不用再来。”
\end{quotation}

\subparagraph{分析} 倓虚大师说:“后来我不抑制它,反而用观照力来观它。”刚开始倓虚大师也想抑制念头,这就是在压念,当然抑制不住,心里也着急,会感到挫败。后来倓虚大师用观照力来观它,这就是观心断念,一观,妄想就没了,没了又起,起来再观,慢慢妄想就不起了。谛老也肯定他了,说他会用功了。

\paragraph{断不掉导致烦恼,该怎么办?}

刚开始断力差,断速慢,这时要懂得调整心态,避免烦恼,多鼓励自己,同时要加强学习和练习,不断积累实战的实力。在断力差时,以不随为主,就是不要跟随念头,不跟随,就不会陷进去。\textit{降伏贪嗔痴,不是一朝一夕就能做到的,得靠串习。(希阿荣博堪布)} 万事开头难,刚开始断不掉,是因为水平低,这时不应气馁,而是应该认真总结和反省,不断提升自己的觉悟,积累自己的实力,到时实力够了,就能断掉了,到时就会发现战胜心魔的感觉实在太爽了!

\paragraph{几个误区,以及澄清}

\subparagraph{断念治标不治本}

有人认为:断念并不能从根本上解决问题,必须要通过每天不断行善积累正能量。

在之前的文章里,我讲到过行善和修心同为戒色根本,行善的确很重要,但一定要认识到断念的重要性,断念是实战的核心和根本,在《四十二章经》里佛陀讲到戒色的问题,佛言:“\textit{若断其阴,不如断心,心如功曹,功曹若止,从者都息。}”佛陀为什么不说行善,不说修身,偏偏要说“断心”!以佛的智慧,肯定会一针见血地指出问题的核心和根本!请大家思考一个问题,什么是本?本是心!要治本,就要观心断念!真正治本的是断念!必须抓住这一核心,否则就会走入误区。当然,行善和修身都很重要,但最关键还是断心!否则做得再多,邪念一上头,马上就会垮掉。不是说行善或者念佛持咒后,邪念就没了,淫欲种子在八识田里,还会经历很多次的翻种子,邪念还是会上头的,到时断不掉,就会疯狂破戒!提倡行善积德是很好,但一定要注意断恶,断恶修善是两方面,不可偏于一方,断意恶尤为关键。

\textit{最上的立断(黄念祖老居士)},观心断念是很多大德都提倡的法门,是治本的方法,是可以从根本上化解淫欲心的。有些戒友刚开始水平低,所以无法做到,这不能怪断念口诀,通过研读断念的文章,加上自己不断坚持练习,慢慢就能做到了。

\subparagraph{忘掉戒色}

戒色不是戒烟,烟是身外之物,可以丢掉可以忘记,而淫欲种子就在八识田里,它会自动冒出来,想忘记戒色,但是心魔却会疯狂进攻,所以说忘记戒色是行不通的,这是一种逃避,自欺欺人,是对戒色原理认识不清。记得好几年前有些人就说要忘记戒色,结果基本都破戒了。我想说的是,你想忘记戒色,但是心魔一直惦记着你呢!当你忘记戒色,放松警惕,到时心魔一攻就破!如果忘记戒色可以成功,那也不需要戒色文章了,那些学生党在接触戒色文章前,基本都忙于学业,忘记戒色,但是还是会破戒。为什么?因为一到周末或者看到诱惑时,就会起邪念,如果缺少断力,就会沦陷。所以说,忘记戒色不可取,也行不通。忘记戒色就像打仗时忘记敌人一样,完全放松警惕,这样能不战败吗?!

\subparagraph{偏离修心,脱离实战,想靠转移注意力和充实生活来戒色}

在真正学会修心之前,基本都是靠转移注意力和充实生活来戒色的,记得我大学时戒色,那时没有学会修心,就是靠转移注意力和充实生活,早睡早起,积极锻炼,专注于学业,虽然也能坚持一段时间,但是最终还是独处时被心魔怂恿了,不知断除,反而听信,结果就是一连串的破戒,那时破戒的教训很惨痛,就是因为不会修心。靠转移注意力和充实生活来戒色,属于比较初级的戒色方法,治标不治本,是不懂修心之前的戒色尝试,必然会失败。

\subparagraph{观心断念只有上根才能掌握}

这是很明显的误导,我自认不是上根,但我也掌握了,很多戒友也都掌握了,我个人认为观心断念比学开车要容易,关键是正确理解和坚持练习,不管是学车还是学一门乐器,都需要经常练习,才能熟能生巧、熟极自神,观心断念也是如此。观心断念最普适,总摄诸法,任何法门都离不开观心。

\begin{quote}\it
    三界之中以心为主,能观心者究竟解脱,不能观者究竟沉沦。(《大乘本生心地观经》)
\end{quote}

观心断念是修行的大总持,总摄诸法,大德多有开示,并不是某些居心不良之人所能诋毁的。观心断念不是治标的方法,它是从起心动念上修,是最治本的一个方法,完全可以化解淫欲心。\textit{过由心造,亦由心改,如斩毒树,直断其根,奚必枝枝而伐,叶叶而摘哉?大抵最上治心,当下清净;才动即觉,觉之即无。(《了凡四训》)} 观心断念就是从心上对治,是最治本的方法。刚开始断不掉,或者越断越多,那是因为断力差,随着坚持练习,自然渐入佳境,到时就会越断越快,越断越强。天天练习观心断念,保持警惕,是必须的,就像奥运冠军天天训练一样,或者学生党天天学习一样,要取得成功,必须要努力付出。

小结:最近一位资深戒友提到了近两年戒色界出现的一些思想误区,其中就包括对观心断念的刻意贬低,另外就是过于抬高“充实生活”、“行善积德”等的作用。他说得挺有道理,有正知见的戒友肯定知道那类文章存在误导,戒色界是有一些走偏的文章,某些居心不良之人也在故意贬低和诋毁观心断念,用压念失败的案例反对观心断念,我们应该远离那类文章,坚定对观心断念的信心。一个人即使失败了,只要信心和立场不动摇,依然还有望崛起,如果信心和立场动摇了,那就难办了,所以远离诋毁文章、坚定信心和立场,显得尤为重要,戒色成功是留给信心和立场特别坚定之人的。如果听信诋毁的文章,不仅自己会受到误导,丧失对观心断念的信心,甚至还会深陷诋毁诽谤之深坑,诚可悲矣!我不愿看到这样的情况,所以告诫大家一定要远离贬低和诋毁的文章,以免被带到沟里,那样下场就惨了。

\paragraph{断念是实战的根本}

一位戒色两年多的戒友的体会:“戒色的核心是什么?这里简单说下,我本人现在信佛,当然不可否认行善修德的威力和帮助,可我更想强调的是,光靠行善提升正气是远远不够的,我刚开始也有这样的误解,觉得正气足了就没有邪念了,以为精满就不思淫了,但没有那么简单,真正要想戒得好,还是要学会断念,断念才是王道,断念是正行,其他的都是助行,正行助行都不能少,离了谁都不行,但最要紧的还是要学会断念。这是一切的根本。不学会断念,正气再足,也有马失前蹄的一天,断念才是根本的根本的根本,不戒到一定程度无法理解到这一步,不是说行善不重要,已经有太多前辈强调行善的重要性了,可我得强调比行善更重要的是断念,断念不难,但是得靠努力练习才行。我们必须有断念的实力,就像剑客要有剑,士兵要有枪一样,不能只靠福报,福报得有,但还是得有自己的实力,光有福报没有保护自己的功力,迟早遇到强大的欲望而破戒。”

\subparagraph{分析} 不管行善多少年,做了多少件善事,还会发现一个现象:邪念会自动上头!断不掉,就会破戒!实践证明,光靠行善是不行的,如果光靠行善就可以,那些大德也不用开示观心断念了,就直接让大家行善,不就行了!虽然行善可以增加正能量,但行善并不能让你一个邪念不起,既然邪念会起,就必须做好断念,否则肯定会破戒的。我戒色修善九年多,邪念还是会上头的,如果我不做好断念,肯定会破戒!行善的确重要,我也一直在强调行善积德,但断念才是王道!实战差,必败无疑!大家一定要认清这一点。

\paragraph{修心诀——千金不换的金口诀}

\begin{quotation}\it
    念起即觉,觉之即无;修行妙门,唯在此也。(圭峰禅师)

    千万个修抵不过我一觉,觉则心空,此是最上福德。(《般若花》)
\end{quotation}

断念口诀最核心的一个字就是“觉”,这是最精髓的一个字。参透这个字,练习觉察,就能主宰内心。觉察永远是第一步,第一步强,直接就可以消灭念头。不管是念佛持咒还是思维对治,都是以观心为前提的,比如一个邪念来了,发现它来了,马上念佛持咒来转,前提就是发现它来了,这就是在观心,在觉察。

\paragraph{勇猛断念}

一位资深戒友写道:“戒色就是一人与万人敌,需要的是大勇猛、大决心、大魄力,必须够狠、够强硬,刀刀见血,敢于在万念丛中杀出一条血路,杀出戒色的黎明!”“必须重视断念实战,因为这是戒色的核心,一切的一切,不管你看了多少戒色文章,哪怕上万篇,最终的体现还是实战的那零点几秒。”

这位资深戒友戒了两年了,他这两句写得很给力,断念是要勇猛,是要有决心,有了决心,才肯狠断!缺少决心,必然犹豫,甚至贪恋,不肯断!断念那一下必须够狠、够快、够强硬!不管看了多少戒色文章,最终就看那一下,那一下不行,肯定破戒!那一下软弱,必然被心魔奴役!威力是什么?一下就搞定!那一下,含着很多东西,是千锤百炼的产物,简洁的背后,是无数次的练习,也是综合觉悟的体现。

\paragraph{磨刀不误砍柴工,刻意练习的重要性}

《刻意练习》这本书里面有句著名的话:“\textit{从新手到大师,你需要刻意练习!}”通过刻意练习,你也可以成为某领域的高手。比如 NBA 史上最伟大的三分射手之一雷·阿伦,当别人羡慕上帝赐给他这个天赋的时候,他却很生气。他说:“不要低估我每天付出的巨大努力,不是一天两天,是每一天!”奥运冠军都在刻意练习,甚至节假日都不休息,还在努力训练,他们的训练科学、专业而系统。你和高手的距离,全在刻意两字,高手会刻意学习、刻意练习、刻意总结、刻意反省、刻意完善自己。任何一个领域,要想取得成功,必须要刻意练习,必须要努力付出。戒色高手都有刻意练习观心断念,想在任何领域取得成就,都离不开有效的方法和大量的刻意练习。

有次苏炳添和偶像乔丹一起参加耐克的活动,活动开始前,他还到酒店健身房,练了会慢跑。后来乔丹知道了,亲口对苏炳添说:“现在的你就是当年的我。”真正的成功者都在刻意练习!都极其注重刻意练习!都是真正的练家子!

我想说的是,刻意和执著并没有错,没有刻意练习,没有执著的投入,很难取得成就,大家看看那些成功人士就知道了,他们对成功很执著,努力奋斗,渴望成就一番大事业。有句话说得好:“执著是成功的一种素质。”有的人可能会说,那佛法为什么讲不要执著,所谓不要执著是内心不要执著,但是事情还是要尽力而为,做完了就放下。不能误解不要执著,以为不要执著就不要去行善了,不要念佛持咒,也不要观心断念了,这样会走偏的。

\paragraph{最高段位}

最高段位就是直接觉察消灭!一觉即空!觉之即无!

当你的断念水平练到了很高的层次,其实还能继续提高,就像跑进了十秒,还能继续提高一样,这是一个不断精进、不断进步的过程。

很多东西必须练到,才能真正明白,你以为自己懂了,其实懂得很浅,只有在不断练习的体会中,你才能有入木三分的领悟。

高人可以指点你,但不能代替你,谁练谁得!

\paragraph{保持警惕而不要过于紧张}

戒色切忌过于紧张,就像过马路时不能过于紧张,要自然,但是却警惕着,也像开车一样,不能过于紧张,自己要学会调整。如果过于紧张,必然会慌乱,会焦虑,会失去章法和控制,很容易出事。有的戒友就是过于紧张,一定要学会调整,保持适当的放松、自然和安定。看到异性应该光明磊落,大大方方的,如果有邪念,及时断除即可,不要过于紧张,也不要走极端。戒色一定要避免走极端,内心要祥和稳定,保持平衡,不能过于紧张,避免出现强迫和走极端,自己要善于调整。

\paragraph{王炸组合}

戒色有王炸组合,思维对治 + 断念口诀 = 王炸组合!思维对治 + 念佛持咒 = 王炸组合!两个王炸组合,都需要有思维对治,因为戒色必须通过思维对治贪恋,可以思维邪淫危害和不净观等。对治贪恋非常重要,有些人不肯断,就是因为贪恋快感、贪恋女色,如果平时经常思维对治,就不会贪恋,就能做到狠断!对治贪恋是极为重要的一环!很多人就是这一环没重视,这一环出了问题。

\paragraph{那个看}

\begin{center}\it
    向内看 \\ 最奇怪的事情发生了! \\ 念头消失了! \\ 这时你突然顿悟 \\ 你不是念头,而是那个看! \\ 看消灭念!
\end{center}

这是我曾经写的一首小诗,如果能悟透这首小诗,就会有一个很大的进步。从思维者转变到了观察者,这是身份的转变,不仅认识到自己是观察者,还发现了看消灭念!这个看,威力十足!念头无法承受这一看!这一看,犹如激光炮一样,瞬间消灭念头。断念是王道,提升觉察力是王道的核心!

\paragraph{一句了然超百亿}

\begin{quote}\it
    要在“观自己的念头”上用功。(黄念祖老居士)
\end{quote}

黄念祖老居士这句开示直指核心,修行在哪里用功?就是观照!观心!最核心的一个字就是观,就是觉,就是照,就是发现,就是知道,都是一个意思。

一句了然超百亿,悟透了这一句,你就知道该在哪里用功了,因为你真正把握了核心与精髓!

不知道核心,就在外围打转,不管是戒色还是修行,都不得力。

\paragraph{到底要不要和妄念作斗争}

\begin{quotation}\it
    吾应乐修断,怀恨与彼战,似嗔此道心,唯能灭烦恼。(《入行论》)

    一星之火,可以燎原;罅漏之水,可以决堤。吾谓淫念亦然,立地起念,即要立地一刀斩断,著不得一些游移,容不得一毫缱绻,否则魔愈深,势愈炽。(《家庭宝筏》)

    非将死字挂在额颅上,决难令妄想投降。妄想既不能投降,则妄想成主,本心成奴。……当杂念初起时,如一人与万人敌,不可稍有宽纵之心。否则彼作我主,我受彼害矣。若拌命抵抗,彼当随我所转,即所谓转烦恼为菩提也。……所言格物者,格,如格斗,如一人与万人敌;物,即烦恼妄想,亦即俗所谓人欲也。与烦恼妄想之人欲战,必具一番刚决不怯之志,方有实效。否则心随物转,何能格物?”“战之一字,关系甚深,人欲天理之际,若不以力战,则理被欲蔽,俾理必隐而欲必著矣。(印光法师)

    经过成千上万次的斗争,由一开始的念起不觉,到念起能觉,由觉而不能转,到一觉即转。……除习气,犹如人天交战,此必百战而可克胜,原非一朝一夕之功。(元音老人)
\end{quotation}

从这些开示中,我们知道与妄念作斗争,是完全正确的!必须要力战!\textit{孔子曰:“我战则克!”}

绝大多数的开示都提倡作斗争!少部分的开示会说不要斗争,那是怕你压念,从不随的角度来讲的。总体而言,都是提倡斗争,提倡力战!要降伏其心!

\paragraph{王阳明先生谈断念}

\begin{quotation}\it
    常如猫之捕鼠,一眼看着,一耳听着。才有一念萌动,即与克去。斩钉截铁,不可姑容,与他方便。不可窝藏,不可放他出路,方是真实用功。方能扫除廓清,到得无私可克,自有端拱时在。

    此须识我立言宗旨。今人学问,只因知行分作两件,故有一念发动,虽是不善,然却未曾行,便不去禁止。我今说个知行合一,正要人晓得一念发动处,便即是行了。发动处有不善,就将这不善的念克倒了,须要彻根彻底不使那一念不善潜伏在胸中。此是我立言宗旨。

    人若知这良知诀窍,随他多少邪思枉念,这里一觉,都自消融。真个是灵丹一粒,点铁成金。(王阳明先生)
\end{quotation}

能克己,方能成己!克己的关键在于克念!因为念头导致行为!

圣贤教育最核心讲的不是修身,而是修心!因为心是根本!“克念作圣”这四个字就说明了一切,为何不是“行善作圣”或者“修身做圣”?而是“克念作圣”?因为修心是核心,观心断念就是克念,就是修心。大家都知道知行合一,行一般解释为实践、执行力,其实王阳明先生讲的行,最核心的一个意思,就是——断念!“我今说个知行合一,正要人晓得一念发动处,便即是行了。”在念头起来时,就要行了!这个行指的就是“克念”!

\paragraph{实战的四种情况}

实战时有四种情况,第一种,起了念头,马上就去看黄,没有任何抵抗,这是最菜鸟的表现;第二种表现,念头来了,会去抵抗,但是断力不行,无法斩断,稍微挣扎一下又被心魔拿下,还是破戒;第三种,有时可以斩断,有时做不到,断念水平忽上忽下,不稳定,这类人也容易再次破戒;第四种,是真正的断念高手,完全是压倒性的胜利,胜负没有悬念,这类高手才是真正的戒色刀客,具备超一流的断念水平。

我们可以对照一下自己处于哪种情况,坚持练习,不断总结,不断深入领悟,就能越来越强!到时就能做到压倒性的胜利!

\paragraph{败了再打}

\begin{quotation}\it
    佛法的仗,是正念与邪念打,败了再打,时久自然邪不胜正。(慈舟法师)

    多败绩,败愈多,战愈力,自是敢战而拼死,予始胜。(紫柏尊者)

    妄想刚强,久战自服,必无疑也。(莲池大师)
\end{quotation}

大德不畏失败,屡败屡战,越战越勇!不断培养自己的实力,最后必然获得胜利!就像一个闯关游戏,不断失败,不断打,总结经验教训,学习高手操作,不断练习,最后成功闯关,必无疑也。真正的戒者,不因暂时的失败而气馁,反而会给自己鼓劲,告诉自己继续坚持下去,不断强大起来,最后的胜利必然属于自己!

\paragraph{只要将猛,不怕贼强}

我以前的一次断念实战,心魔三幅图像接连上来,我三记觉察就消灭了,一记比一记狠,三记觉察就像一套组合重拳,一记上勾拳,一记右摆拳,一记后手直拳,直接把心魔干趴下了,不敢进攻了。有的戒友说自己越断念头越多,因为他缺少威慑力,有威慑力,就像有了大杀器一样,敌人不敢轻举妄动,因为敌人知道,只要敢进攻,必然被消灭!他们怕了!我们要强化觉察力,越来越犀利,越来越强大,到时就能断出威慑力!只要将猛,不怕贼强!横刀立马!唯我戒色大将!正所谓:万千邪念杀奔来,一声断喝皆齑粉!

\paragraph{断念才是硬道理}

戒色方面最硬的道理就是断念!因为最后就看实战表现!不是看你阅读了多少戒色文章,也不是看你记了多少戒色笔记,也不是看你把戒色文章写得多么有道理,而是看最后那一下,实战的那一下,最根本的那一下!记住:断念才是硬道理!放之四海而皆准!万流归海,万道归宗,都归到断念那一下。一个刀客,就看他出刀的那一下,要极具爆发力;一个剑客,就看他挥剑的那一下,要凌厉果决!

\paragraph{案例反馈}

\begin{case}
    飞翔哥!自从我顿悟断念口决的重要性,我就又开始背诵断念口诀了!以前我也背过,可背着背着就不想背了,也就不背了!今天睡醒了就躺床上背了好像二百左右的时候,有个念头说久卧伤身,我就坐起来继续背诵,一口气背到了五百遍,以前我都没有过一口气背到五百遍,总是背到三百左右就很累了就不背了!我在背口诀的过程中,背了一会冒出来念头,过了一会又冒出来不同的念头,就是这样一直冒出来念头!发现被念头带跑,我就把注意力放在口诀上,不去管念头,那个念头就没了!
    \subparagraph{分析} 只要真的下功夫练习,就会有所收获,关键要持续地投入,坚持练习,贵恒!背诵口诀或者念佛时,会被念头带跑,及时发现拉回,慢慢觉察力就上去了。其实最核心的就是提升觉察力,觉察力就是念头的克星。觉察力强到一定程度,断念就会变得很轻松,一觉即灭!背口诀宜小声默念,避免劳累,如果感觉累,可以分两组或者三组,组间休息一下,应保质保量地完成。这位戒友在练习的过程中发现念头冒出来了,这个发现就是在观心了,发现被带跑,及时拉回,拉回的速度会越来越快,觉察力会越来越强,背诵的质量也会水涨船高。有了练习的不断积累,加上深入研读观心断念的文章,到时实战水平就会有一个很大的飞跃。
\end{case}

\begin{case}
    断念口诀就是我对治邪念的无上法宝!
    \subparagraph{分析} 这位戒友真正体会到了口诀的威力,真正练强了之后,对治邪念,真的很轻松,不仅可以对治邪淫的念头,其他的妄念也可以一样对治。观心断念是大总持,总摄诸法,是一个总的法门。
\end{case}

\begin{case}
    平时不练习,等欲望一起来就克制不住,该怎么办?
    \subparagraph{分析} 平时不练习,战时多流精!等到邪念上头时,断不掉,结果就会被附体,身不由己,一发不可收拾。只有平时好好练习,认真备战,以实战为核心,不断提升自己,真正强大起来后,才能立于不败之地!否则实战时就会一触即溃,再次掉入怪圈。
\end{case}

\begin{case}
    飞翔老师您好!请问一下您一个问题,自从学习了您教的断念实战,我不断地练习觉察,目前已经戒色两百天了,前段时间我躺在床上看抗战搞笑之类的喜剧片,片中突然跳出了擦边信息,当时心里并不是起了很强大的淫念,反而是一种很微妙的贪恋感觉,而且还伴随着怂恿我去看,当时我立刻觉察到了,但并没有彻底断除,念头起来了我还犹豫了两三秒,然后我想起了“迷住我的不是这个肉身,而是妄念,我是纯粹的觉知,念头不是我”这句话,然后再觉察,邪念就消失了。
    \subparagraph{分析} 在这个色情泛滥的时代,对境的考验真的很大,我们看电视或者上网时,都要小心谨慎,做好视线管理,加强修心。这位戒友出现了微妙的贪恋感觉,还有心魔的怂恿,这时他犹豫了,不过好在马上思维对治了,色不迷人人自迷,为什么会迷?就是认为诱惑好,如果认为诱惑脏乱差,还会迷在里面吗?要学会转变观念,这可以有效对治贪恋。认识到自己是纯粹的觉知,不是念头,这在实战中有很大威力,因为这样一思维,就彻底不认同念头了。
\end{case}

\begin{case}
    我们戒色就是要训练自己提高这个自动拦截念头,也就是觉察念头的意识,只要一觉察到念头,就可以迅速地转化消灭了,这个意识必须要强。虽然我对理论的知识掌握得还可以,但是在实战中的顿悟,我觉得才是王道,我记得《菜根谭》说,“念头起处,才觉向欲路上去,便挽从理路上来。一起便觉,一觉便转,此是转祸为福、起死回生的关头,劝莫轻易放过。”这更加符合我的理解,我一定会更加深入下去的,感恩飞翔老师,没有您,我根本学不到手淫的危害,也学不到修心断念的知识,这门知识确实是宝贵,让内心宁静,能主宰自己的感觉太爽了!昨天晚上我做了一个梦,我梦到我十四岁之前,我看着那个男孩子,不知道他为什么那么开心,我从来没有见过一个小孩和我十四岁那样子的开心,总是喜欢蹦蹦跳跳,调皮捣蛋的,不知道他为什么精力那么充沛,我感觉到他的纯净,也感觉到了邪淫之后的污染。我一定会奋发图强,彻底战胜心魔,虽然我不能再让那个少年回来了,但是我可以让自己的内心保持宁静,有这种宁静和纯净的心灵我就知足了,感恩飞翔老师。
    \subparagraph{分析} 不管是《菜根谭》还是《了凡四训》,都有提到“觉”,念头肯定会上来的,及时觉察,才能做到不破,这是最实战的操作。我一直在强调实战,因为我深知,偏离实战的后果就是疯狂破戒。理论知识很重要,但一定要结合实战来体会,在实战后马上回到断念的文章或者笔记,这时研读必然会有所顿悟。修心断念的知识是最宝贵的,在学校里是学不到的,这门知识是最顶级的,是金字塔的尖顶,这门知识是圣贤教育的核心和精髓,能够学到修心断念的知识,并且正确掌握,就有望主宰自己的内心,甚至证悟宇宙人生的实相。能够学到这门宝贵的知识,真的有福了!好好珍惜吧。
\end{case}

\begin{case}
    飞翔老师,最近一个月学习《戒为良药》,发现断念修心很重要,比以往戒色要舒服好多,意淫一来就断开,也不会担心遗精后再破戒。
    \subparagraph{分析} 刚开始学习戒色文章,往往把握不住重点,等到经历了一些破戒,再加上看到前辈如此强调断念实战,到时就会醒悟——断念实战真的太重要了,破与不破,往往就在一念之间,能够及时断掉,就没事。有了断念的能力,戒色就会顺利很多。
\end{case}

\begin{case}
    我在努力地克制自己,但却在闲下来不够充实的时候看黄破戒,最近半年算是我最成功的了,四十天破了一次,然后两个二十天各破一次,我自己的总结是忙的时候很充实,也顾不上这么多,但是一闲下来,就很容易破戒,干不过心魔。
    \subparagraph{分析} 这就是靠转移注意力、充实生活来戒色的弊端。人总有闲下来的时候,不可能一直忙,独处时,心魔很容易进攻,所以古人强调慎独。靠转移注意力、充实生活来戒色,这并没有把握戒色的核心,包括外在的修身也是,都是治标不治本,本是心!偏离观心断念,去充实生活,这样只能管一时,迟早会破戒的。
\end{case}

\begin{case}
    个人体会是在邪念上来的时候一定要念佛号,这时候念佛更考验一个人啊,会感觉到邪念在心里的冲击,但念下去邪念的力量就弱了。要是能达到觉之即无的境界,断念应该会更轻松。
    \subparagraph{分析} 邪念上来时,可以觉察,也可以念佛持咒,两者都可以。我平时以觉察为主,当然念佛持咒也很好。念佛断念的原理,就是以一念代万念,发现邪念来了,马上念佛来转,速度一定要快!光有念佛日课还不行,一定要把念佛用于断念,因为实战才是根本。有的人说自己念了几百万的佛号了,为何还破戒?因为邪念上来时,他没念佛!他平时虽然有日课,但是实战时却没念!那是不行的!脱离实战,必破!
\end{case}

\begin{case}
    飞翔哥,我想跟您分享一个开心的事,就是我最近悟懂了一个道理,就是您在 117 季将的“不认一切念头”这个开示对我真的很有帮助,我现在对于一般的邪念都是能做到念起即断,而之前对于怂恿类的念头有时我会跟随,但是现在对于怂恿念头,只要我开启“不认一切念头”的警惕意识,我就会去识别它是什么类的念头,如果是邪念然后立刻断除。对于这个顿悟,我真的很开心,在此也再一次感谢飞翔哥一直默默的付出,现在我每天都学习一篇《戒为良药》,然后做笔记,提高自身觉悟,做到正己化人。
    \subparagraph{分析} 不认一切念头,这是我在实战中的一个体会,当时就觉得这个领悟很重要,因为不认一切念头,而认纯粹的觉知,这样就和念头彻底拉开了距离。所有的念头都不是我,我是纯粹的觉知,我可以觉察念头,我不是念头。有了这一层的认识和体会,念头的力量就会下降很多。大家发现没有,如果认了念头,跟了念头,就等于给念头继续下去的动能,如果我不认念头,不随念头,念头就会消失。
\end{case}

\begin{case}
    刚才我看了《戒为良药》第 99 季,我看完这一段,突然好像悟到了什么一样,然后我按照所说的观照,念头上来了,我看着念头,突然念头就消失了,我这算不算断念了?
    \subparagraph{分析} 这位戒友做到了,找到感觉了,虽然他还比较无知。记住:看消灭念!去看念头!而不是去压念头!看和压,是两个动作,仅仅是看,念头就会消失。刚开始可能做不到,或者偶尔能做到,随着坚持练习,强化觉察力,就能每次都做到!
\end{case}

\begin{case}
    分享下今天早上的实战,醒来坐在床上玩手机,下载点开了一个疑是色情 app,看到加载出来的图片,就立马把软件卸载了!这时候以贪恋感觉为主、邪念为辅,瞬间上脑企图控制我,我用断念口诀加思维对治化解了这次实战,我用断念口诀的同时观想狗看美女的情形!那种上脑企图控制我的贪恋感觉,瞬间就消退了,邪念也跟着消退了!这种思维对治贪恋,感觉太有用了!贪恋感觉上脑是瞬间,而我用这种方法消灭贪恋感觉也是瞬间!最后总结疑是色情 app 不要去下载确认!
    \subparagraph{分析} 这个案例的反馈很好。这位戒友用了“王炸组合”——思维对治 + 断念口诀!对境时,很容易出现贪恋的微妙感觉,这时应该及时思维对治。这位戒友观想狗看美女的情形,狗看美女是没有分别的,狗不会觉得美女好,而去贪恋美女。狗看到美女,很淡定,就像没看到一样。实战时懂得用“王炸组合”,必然有绝佳的实战效果。这位戒友的警惕性也很强,看到不良图片,马上就卸载了。他最后也总结了,只是要疑似色情,就不要去下载,这是实战的总结,很好。
\end{case}

\begin{case}
    戒色的核心就是观心断念,就是实战的那一下!认真学习《戒为良药》提高综合觉悟,好好练习观心断念,努力落实戒色十规,十规做到就可以成功。
    \subparagraph{分析} 很多戒友都认识到了观心断念的重要性,都认识到了实战那一下的重要性,也认识到了戒色十规的重要性。那些资深戒友能戒好几年,就是因为觉悟高、实战强,真正落实了戒色十规。我戒到现在九年多,其实也就是因为落实戒色十规。
\end{case}

\begin{case}
    我理解的核心就是,发现念头,这几个字,我每天都在练习发现念头,不知这样对不对?以我的理解,其实那种方法,最终都是为了更快地发现念头,也只有不断发现念头才能提高觉察力。
    \subparagraph{分析} 这位戒友的理解很正确,所谓发现念头,其实就是觉察念头,发现、知道、觉察,这三个词在观心方面,其实是一个意思。就是为了更快地发现念头,觉察念头。
\end{case}

\begin{case}
    又看到飞翔哥的新季了,开心!还有一件事报告给飞翔哥,今天戒色满一年了。可以尝试一下念南无观世音菩萨圣号,我每次就是靠这个断念的,一起邪念马上连续念南无观世音菩萨,把邪念扼杀在摇篮中。念南无观世音菩萨,一方面是靠这个断念,转邪念为清净,一方面是靠佛力加持,\textit{若有众生多于淫欲,常念恭敬观世音菩萨,便得离欲。(《观世音菩萨普门品》)}
    \subparagraph{分析} 念观世音菩萨圣号很殊胜,念佛不仅在平时要坚持日课,关键是实战时要念。反过来讲,如果平时没日课,实战时才念,也不行。平时要念熟,实战时才能发挥威力。日课和实战,同等重要!看元音老人开示,我就知道念佛持咒要用于断念,其实很多大德都开示过,一定要注重断念实战。这位戒友做得很好,一起邪念,马上念观世音菩萨圣号,这样就能断除邪念,他戒色满一年了,很不错。
\end{case}

\begin{case}
    破戒了,经过一晚上的不断反思,我总结出我的问题如下:首先,无聊的时间增多了,大四下没有功课,呆在宿舍的时间开始变多。其次,警惕性下降,明知许多电影和电视剧有擦边内容,但还是义无反顾地去看,破戒的种子早已埋下。再者,疏于学习戒邪淫知识,贴吧也不常来了,一个戒色老兵主动放弃自己的阵地和武器,这无异于自杀。最后,断念之刃不再锋利。不知道从什么时候开始,我的断念水平就停滞不前甚至是下降,之前我能取得这么好的成绩,离不开强大的断念能力。破戒后,连续破戒的情况一再出现,就是因为断念不行了。戒色是一辈子的事情,在这个邪淫泛滥的年代,警惕性要永远保持,一定要坚持练习断念并应用到实战中,戒色吧要常来,戒色知识的学习一天也不能落下,戒色的同时要养成良好的生活学习工作习惯,时刻要牢记,戒色是一个系统工程。送给所有戒友一句话,无论你戒多久,别以为你稳了。
    \subparagraph{分析} 这是一位资深戒友的破戒总结,之前他戒得很不错,人生也完成了逆袭,后来渐渐失去了好状态,如何保持好状态是一门学问。自己戒色成功后,应该要常来贴吧帮助戒友,也要坚持行善积德,不断培养自己的正能量,同时,断念一定要坚持练习。断念不行,肯定会破戒,不管戒多久,都要保持警惕!不应该远离戒色,不应该忘掉戒色,否则之前积累的很多东西都会变得生疏,自己也会放松警惕,最后破戒也就必然了。
\end{case}

\begin{case}
    飞翔老师,观心断念真的太重要太重要了!之前一直是戒油子,连一个礼拜都很难戒到,最多坚持十几天肯定破,自从意识到观心断念的重要性,一有不良思想,马上觉察到,然后就断掉了,这种感觉真的就是“底气”,现在戒了 44 天,虽然不多,但真的是最近几年最多的一次了。
    \subparagraph{分析} 这位戒友也意识到观心断念的重要性了,为何之前一直是戒油子?戒油子的特点就是实战差!嘴皮子有两下,实战一下都没有!邪念一来,一触即溃。貌似都懂,其实假懂,似懂非懂!一旦认识到断念的重要性,并且开始练习,就真正入门了,就能越戒越好,就能告别戒油子的状态了。所以,一定要立足断念实战和对境实战!实战强了,就有底气和把握了。
\end{case}

\begin{case}
    飞翔哥,非常感恩您,我手淫十年,后来通过学习您的《戒为良药》,练习修心,戒了 633 天了,由于我经常练习断念,所以我对邪念很敏感,邪念一起就断了!但总是被负面念头带跑,感觉自己德行欠佳,心里总是会响起对别人的负面评判,以及各种负面念头,我该如何对治这些负面情绪啊?求飞翔哥赐教!
    \subparagraph{分析} 这位戒友戒得很不错,断念很厉害了,但是他没有把断念的范围扩大化,应该要断除所有负面的念头,心魔代表负面,所有负面的念头都是心魔的表现,都要立刻断除。对别人负面评判,是心魔惯用的套路,就是让你充满负能量。所以大德告诫我们,要多看别人优点,存好心,说好话。不要对别人进行负面评判,对于其他各种负面念头,也要学会识别并断除,自己平时要坚持学习圣贤教育,不断提升自己的德行。
\end{case}

\begin{case}
    飞翔哥!我练习断念的时候,我感受到了那个空,纯粹的空,我只感受到一瞬,就再也感受不到了!我感受到它的一刹那就很害怕,因为我感觉没有我了!没有念头了!就是感受到空了,瞬间感到害怕,然后空就不见了,念头又回来了,我只是感受到了一瞬间的空,可能是我第一次有意识地感受到空,所以感到害怕吧!我知道那个空很可贵!如果在心魔来的时候!安住这个空,简直就是无敌的,就像飞翔哥说的无敌模式一样!是啊!我确实不习惯无念的状态,因为这是我第一次感受到这个空!
    \subparagraph{分析} 这个案例挺好,那个纯粹的空,也就是纯粹的觉知,才是真我。因为过去一直认念头为自己,所以会害怕空。我当初试着安住这个空时,也感到莫名的害怕,因为我觉得我没了,后来不断学习大德开示,加上不断安住,慢慢就习惯了,解除了对念头的认同,我就是那个纯粹的觉知,纯粹的空。这个空,并不是死空,而是有觉知的,所以根本不用害怕。我对安住的感觉有四个分类,一,感到害怕;二,感到无聊;三,感到美妙;四,感到神圣。感到害怕是最开始的阶段,因为过去顽固地认同念头为自己,所以突然安住空,就会感到害怕,慢慢适应了就好了。然后继续安住空,会感到无聊,因为习惯于做各种事情,各种忙,当安住于空,什么也不干,什么也不想,会有无聊的感觉。继续安住空,就有美妙的感觉了,感到纯净、美好、自由、喜悦和满足,才发现,原来空这么美妙和神奇,这是最纯粹的意识状态,安住于真我,真的奇妙无比。第四个感觉就是神圣,纯粹的觉知就是每个人内在的神性,有了这层认知后,再安住于真我,就会有神圣的感觉,就是托利讲的“神圣的临在”。有了强大的断力后,随时都可以切入空,我已经安住无数次了,安住越深,越美妙。
\end{case}

\begin{case}
    一直到今年九月份,接触到您的《戒为良药》,如茅塞顿开,我才意识到戒色最重要的是戒意淫,断念修心做好了,戒色就成功了一大半,至今我已经戒了八十天。
    \subparagraph{分析} 真正抓住了本,就能取得突破,本是心,念头是行为的先导,把念头断了,也就不会邪淫了。当然综合觉悟也很重要,但最后就看断念的那一下,那一下既是觉悟的体现,也是平时练习的体现。
\end{case}

\begin{case}
    我今天中午不小心就破戒了,当时一个人在家,心魔来了,挡都挡不住,我真的一直想转移注意力,但是感觉当时的身体已经不是我的了。
    \subparagraph{分析} 为何挡不住?因为断念不行!我独处时,心魔也会来,最近几天,邪淫的回忆就浮现了,我立刻断除了,就没事。心魔来了,再想转移注意力,就难了,一旦邪念起势,就会控制身体,到时就身不由己了,就像提线木偶。转移注意力可以作为断念的辅助,如果仅仅靠转移注意力,那肯定是不行的。
\end{case}

\begin{case}
    请问飞翔大哥,强迫症患者遇到强迫性的心魔念头该怎么办,比如说脑子里一直出现一张过去看过的很暴露的图片,越去关注越对抗,越出现,搞得自己很痛苦,厉害的时候坐立难安。感恩飞翔大哥的付出,您辛苦啦!
    \subparagraph{分析} 强迫的邪念或者图像出现时,不要去对抗,一对抗就变成了压念。本来斗争、对抗、抵抗等词是没问题的,但很多人不会断念,就会搞成压念,结果搞得自己很烦恼。当断力不行时,就做好不随,发现强迫的念头出现时,不要跟随即可。或者马上念佛、念口诀,念一段时间就清净了。其实真正会断念了,这些都不是问题,觉察力强了,一觉即灭,配合思维对治,那就是王炸!
\end{case}

\begin{case}
    你戒不掉是因为你不会观心断念,戒色唯一得核心就是观心断念,我以前不会观心断念也是当了五年戒油子。现在学会观心断念戒到今天第一百天,所以戒色一定要学会观心断念这个终极武器!
    \subparagraph{分析} 当了五年戒油子,就是因为不会观心断念,断念实战差,结果就是越戒越差。靠其他方法也许能戒一段时间,但如果偏离了断念实战,最终肯定失败。而且失败后,再用原来的方法,就不管用了。所以一定要学会观心断念,这是大总持,这也是基本功。
\end{case}

\begin{case}
    接触飞翔大哥的帖子到现在两年了,没有手过一次,意淫也控制得很好,没看过黄。关键一点是自己的戒色决心大。邪淫这十几年,受的痛苦太多,挣扎的时间太长,摆脱痛苦的愿望太强了。
    \subparagraph{分析} 决心强,加上坚持学习和练习,就能越戒越好,把意淫控制好了,基本就成功了一大半了,其他养生、控遗、德行、情绪管理等方面再跟进,就能戒得比较稳定了。
\end{case}

\begin{case}
    刚刚做了淫梦,然后醒来忍不住看黄了,好后悔啊,这是戒色近两年来第一次看黄。我也是每天都有日课,早晚各念一遍清净明诲章,甚至每天都念一部地藏经,没想到还会忍不住看黄,对飞翔哥所说的立足于实战为核心,又有了新的认识。
    \subparagraph{分析} 很可惜,毕竟戒了两年了,警惕性真的很重要,戒色的过程中会发生各种情况,一定要提高警惕,做好观心断念。立足于实战,才能立于不败之地,一念之差,就可能看黄,就可能破戒。这位戒友虽然有日课,但还是栽了跟头,说明警惕性的重要性,一定要时刻警惕邪念上头。他做了淫梦,醒来后肯定有微妙的想看黄的念头,他应该立刻断除,这样就没事。但他听信了那个念头,这就是断念实战没做好,希望他吸取实战的教训。
\end{case}

\begin{case}
    我大多数时间能够觉察到邪念,但是就是不想断,还是觉得看黄爽,我每次破戒都要看黄,昨晚从凌晨 12 点一直撸到早晨 5 点多,不断地找新的黄片,然后不断地厌倦,另外还打游戏,杀戮的,整个人的心态感觉崩了。有时候邪念来了没觉察到,意淫了一会儿,然后就直接去找黄了,一点心理防线都没有,来到戒色吧将近两年半了,笔记记了厚厚的五本,然而一直在走形式,念头来了不能及时觉察,不能及时断掉。
    \subparagraph{分析} 不想断,觉得看黄爽,这就是贪恋心理。思维对治没做好,就会导致贪恋心理还是那么强,要多思维邪淫危害、不净观等,要坚持对治贪恋。这点极为重要!因为贪恋,所以不肯断,通过思维对治,就能做到狠断。熬夜看黄撸管,有猝死的风险,一定要小心。这位戒友的状态很差,问题就是贪恋,还有影响戒色状态的就是打游戏。断念实战做得很差,当然屡戒屡败!问题出在自己身上。笔记做了,但是吸收率如何?就要打个问号了!一直走形式,念头来了,不能及时断,还那么贪恋,这样只会沦为戒油子。希望这位戒友能够认真反省,对治贪恋 + 狠断,争取早日突破怪圈。
\end{case}

\begin{case}
    飞翔老师,我也做到觉察即消灭了,但是还达不到纯熟的条件反射,实际感受:念头生起那一刹那也可以说萌芽的时侯,立即看大脑里的念头,就会消失得无影无踪,这就是念起即觉,觉之即无。不怕念起,就怕觉迟,这就是修心,降伏心魔,感恩飞翔老师。
    \subparagraph{分析} 这位戒友也做到了,虽然还未达到纯熟的境地,但已经找到正确的实战感觉了。很不错,希望他再接再厉。其实,实战感觉很简单,就那么一看,念头就消失了。随着强化觉察力,这个实战感觉就能越来越强。刚开始不稳固,不断强化,就能越来越稳固,越来越娴熟了。
\end{case}

\begin{case}
    如果天天持咒,不注意远离黄源,一旦邪念來了,不及时断掉,一样会破。
    \subparagraph{分析} 的确是这样,这位戒友一语中的!虽然天天念佛持咒,但是邪念还是会来的,坚持念佛持咒,有一段时间内心是非常清净的,有些人就以为这种清净会一直保持下去,其实不然,因为淫欲种子并未消除,戒到一定时间会猛烈翻种子的,加上外境的诱惑,也是很容易让人贪恋和起邪念的,所以即使天天念佛持咒,还是要注意远离黄源,做好观心断念,这是实战的根本。
\end{case}

\begin{case}
    每个人想要戒色,真正要戒得长久,必须要意识到,断念是必须要掌握的,我以前并没有重视,在不会断念以前,三个月是极限,所以千万不要轻视断念。(现在戒色 502 天)
    \subparagraph{分析} 实战强,才能戒得长久,因为实战是核心!脱离实战,就是花架子,念头一来,就会垮掉。不会断念以前,很多人一个月都撑不过去,我以前不会断念,最高戒了 28 天,真正会断念了,戒到现在九年多,一次未破。
\end{case}

\paragraph{总结}

这季详细深入地讲了压念的思想误区,相信大家都应该彻底明白了,也知道如何避免压念了。在实战中,不要试图去压念,而是去觉察念头,觉察力强,念头直接就消失了。或者马上念口诀、念佛号来转,这也是可以的。断念要熟练,功在平时!平时不磨刀,用时不给力!平时有空就磨刀,把断念之刃磨得锋利无比,到时实战时就非常强悍了!完全是压倒性的胜利!真正的断念不是压念,而是觉而化之,或者念口诀、念佛号来转,明白了吧!

这季扩展延伸讲了很多关于断念的内容,加上一些案例的集中分享,相信大家对断念会有更深入、更细致的理解和体会,这是值得反复研读的一季。我之所以能做到九年零破,就是因为注重实战,断念实战和对境实战,我的戒色理论体系是立足于实战的,以实战为核心,这样才能立于不败之地!一位戒友说:“一念没做好就破戒了。”的确是这样,破戒就在一念之间,不断掉,就会被附体,陷入疯狂。实战的教训真的很深刻,过去我被心魔虐了无数次,就是因为不会断念,当我开始研究心魔的套路,开始练习观心断念,开始落实戒色十规,才突破了怪圈。只要真正去落实,相信大家都会成功的,失败是暂时的,只要不断完善自己的觉悟,强化实战水平,最后的胜利必将属于我们。加油!

下面分享一首诗歌。

\begin{poem}[神圣的临在]
    \begin{multicols}{3}
        \begin{center}~\\
            契入内心 \\ 那份神圣的感觉 \\ 简单地存在 \\ 没有念头 \\ 纯粹地存在 \\ 充满力量 \\ 这不可思议的瞬间 \\ 就是临在的片刻 \\ 体味内心的自由与美好 \\ 感到无限欢喜 \\ 一切是那么神奇 \\ 仿佛第一次看到这个世界 \\ 临在的人 \\ 就是一个奇迹 \\ 临在的片刻 \\ 千万朵花儿 \\ 一起绽放 \\ 千万颗星星 \\ 一起闪耀 \\ 千万个孩子 \\ 一起欢笑 \\ 一个神圣的存在 \\ 双眼里燃烧着 \\ 永恒之爱 \\ 整个天地 \\ 整个宇宙 \\ 都被爱点燃
        \end{center}
    \end{multicols}
\end{poem}

下面推荐一本书。

\begin{book}[《格言联璧》]
    此书集先贤警策身心之语句,垂后人之良范。全书主要内容包括学问类、存养类、持躬类、敦品类、处事类、接物类等,是成己成人之宝筏,希圣希贤之阶梯也。该书说理之切、举事之赅、择辞之精、成篇之简,皆萃古今。每一条事理内涵丰富,广博精微,言有尽而意无穷,先哲的聪明智慧和无限期望尽在这妙语联珠之中。一册在手,揣摩研读,细心体会,必能驾驭人生的真谛,游刃于生活之间,既能修身齐家,又能报效社会,不失为难得的济世良药,人生指南,因而其成书问世后即为宫廷收藏,流传民间,远播海外,成为影响深远、读者众多、历久不衰的蒙学读本。戒色后应该不断提升和完善自己的涵养、品格和德行,这非常关键,厚德载物,德之不修,戒之不稳。戒色十规里专门强调学习圣贤教育,就是为了让大家从圣贤教育里得到启发,学到为人处世的道理,学到修心修身的道理,自身的德行能够越来越好,这样不仅对于戒色有利,对于自己的生活、学业、工作和为人处世,都极为有利。“谦退,是保身第一法;安祥,是处事第一法;涵容,是待人第一法;洒脱,是养心第一法。”这四句我很喜欢,其他的句子,比如“大事难事看担当,逆境顺境看襟度。临喜临怒看涵养,群行群止看识见。”我也很喜欢,《格言联璧》里有很多好句,的确是一本很好的书。这季推荐给大家,希望大家都能不断提升自己的德行,扩充自己的德量,戒到最后拼的就是德,德行是地基,地基一定要稳固扎实。
\end{book}
