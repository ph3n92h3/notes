\subsection{必知戒色文章之红五类}

\paragraph*{前言}

这季前言分享两个答疑案例。

\begin{case}
    飞翔哥您好,很感谢您写的文章,我有一个问题想请教一下,我已经戒了两年了,一次没手过,现在就是频遗,在吃后遗症治疗吧主开的药,但是我觉得多半是自己心理导致的,所以我不想吃药,但是就怕好不了。自从戒了之后就一直频遗,导致现在还没好,心理更别说了,干什么事总走神,胡思乱想,脑子一刻也停不下来,睡觉也是,好像就是强迫思维一样,我不知道是肾精不足导致的还是心理的问题,现在很难受,有时候总觉得身体问题不大,毕竟戒了两年了,一般的症状都没有了,但还是头昏脑涨,注意力不集中等等,真的要疯了!求您救救我

    \textbf{答} 我控遗那季你可以看看,频遗不克服,是万难恢复的。我一般建议固肾功加吉祥卧,如果实在不行,还可以去看看中医调理,克服频遗关,身体才能恢复正常。加油!

    \textbf{附评} 这位戒友戒得还是很不错的,能够戒到两年不撸一次,很难得。但是这位戒友的恢复不是很理想,脑力下降,还有神衰的症状表现。主要原因就是频遗无法克服,中医有讲到久遗八脉皆伤,频遗属于病态,危害不容小觑。肾精不足是可以导致心理失调的,伤精的表现分为生理和心理两方面,是对身心健康的双重摧残。这位戒友出现的强迫思维,就考虑和伤精有关,很多伤精的戒友都有强迫思维的表现,脑子一刻也停不下来。一般坚持戒色养生,然后注意心理调整,强迫思维是会慢慢恢复正常的,有一个过程,一般都需要三个月乃至半年以上。我那时也强迫思维,经常想一个问题,疯狂地想,还喜欢想病,乃至疑病,可以想到坐立不安的程度,后来戒色半年以上,这种心理失调的表现就消失了,自己就好了。关于控遗我写过好几季,控遗也是门学问,各种控遗方法都可以试试,选择适合自己的坚持练习,我一般推荐固肾功和吉祥卧,但是很多戒友做不到位,那就难以保证效果了,吉祥卧看似容易,然而很多戒友遗精醒来时都是直腿,所以如何给腿定型就是吉祥卧的最大难点。铁板桥和站桩也可以试试,铁板桥也有固肾腰的效果,站桩可以强身补气,气足自可摄精不遗。如果试过很多控遗的功法都不行,那中药调理也是很不错的选择,也有戒友通过中药调理成功控制了遗精频率。总之,频遗这关一定要克服,否则身体恢复就很难保证了,有的频遗戒友,不但恢复不理想,而且又新添了很多症状,搞得他异常苦恼。所以,我们一定要想方设法克服频遗关,克服了频遗关,身体恢复才有基础。
\end{case}

\begin{case}
    飞翔哥辛苦了,我大半年没破过戒了,但从不敢放松警惕,现在每天晚上睡觉前会看一遍下在手机里的飞翔戒色视频,这个视频是飞翔哥戒色心得的浓缩,对潜意识的净化作用是很强的,我想很多戒友并未很好地利用这个视频呢。

    \textbf{附评} 这位戒友令我印象深刻,我还记得他刚开始的发言,现在他都戒色大半年了。他是那种脚踏实地的戒友,认真学习,真正落实到自己的行动中去。他的警惕意识修得非常好,我看过很多破戒的案例,很多破戒的人,你去看他们的警惕意识,真可谓一塌糊涂。要破戒多少次,才能意识到警惕意识的重要性呢?有位戒友说得极好:“戒色的每一天,都不要离开戒色文章!”每天看戒色文章,一方面是增长戒色觉悟,把戒色思想内化成自己的戒色意识,另外,就是通过学习让自己保持高度警惕,时时刻刻都在提醒自己保持警惕,放松警惕的话,下一个破戒的就是自己!这不是开玩笑的,网络上到处是诱惑,到处是黄色地雷,能不警惕吗?擦边图、擦边新闻那么多,一不小心就中招了!警惕意识最好每天都要修,多看戒色文章,多看受害者案例警醒自己,一般看一篇文章可以管三天,过了三天,警惕意识就可能会出现下降。就像手机充电,充一次管三天,你必须再次充电,这样才能延续下去。戒色也是如此,很多人戒到最后都放松了警惕,结果就是破戒。武谚有云:一日不练十日空!古语还说到:一日不练手生,三日不练拳脚全荒。所以,刚开始戒色,每一天都不要离开戒色文章,你一离开戒色文章,一方面觉悟上不去,另外就是容易放松警惕。即使你进入戒色稳定期,还是要保持高度警惕的。记得有一位戒了两年半的老戒友,他就是有一天突然出现了一个“试定力”的念头,对于这类心魔怂恿的念头,他没有足够的警惕,于是听从了这个念头,结果就破戒了,他异常后悔。

    警惕的内容一般分两种:

    \begin{enumerate}
        \item 对外在诱惑的警惕;(特别是擦边内容)
        \item 对自己邪念的警惕。(还包括心魔的怂恿,比如试定力、试性功能等)
    \end{enumerate}

    警惕意识真的是要专修的,一定要引起足够的重视,我戒到现在每天都在看受害者案例,一方面是我研究戒色的需要,另外也有利于我保持良好的警惕性。这个案例的戒友,他的学习态度真的非常认真,每天睡前都在看戒色视频,很多人看一遍就不看了,其实不少戒友都是看视频来警醒自己的,戒色视频有其独特的作用,有些图片的文字是我心得的提炼和浓缩,多看多思考,对于提升觉悟是有一定帮助的。态度决定高度,这位戒友有着认真的学习态度,他能戒大半年没破过戒,我一点不奇怪,当初我就很看好他,他就是那种脚踏实地认真学习的人。他说“我大半年没破过戒了,但从不敢放松警惕”,大家看看他的警惕意识,再对照一下自己的警惕意识,你觉得差距在哪呢?
\end{case}

下面步入正文。这季就戒色文章的种类及其作用和大家做个分享,戒色文章是应该做个分类,这样大家看戒色文章才可以更有针对性,效果也会更好,具体如下。

\subsubsection{纯论述危害的文章}

专门论述危害的文章还是很多很多的,新人来到戒色吧,一定要先认识撸管各方面的危害,认识必须深刻到位。很多新人来到戒色吧,大脑里还残留着无害论,因为基本人人都被无害论洗脑过,所以一看到危害的文章,可能脑子一下无法转变过来,所以只能多看,慢慢熏陶,然后一定要把无害论从大脑里清除出去。还有一些戒友,虽然知道撸管有害,但是认识一点不深刻、不全面,甚至有的戒友会认为撸管不会导致那么多症状,这其实和他的阅历还有自身体验的局限有关,撸管导致的症状如果细分的话,估计有上百条,肾虚百病丛生,绝非虚言。

戒任何瘾都离不开谈危害,不管是酒瘾、烟瘾、毒瘾、赌瘾等,都要大谈特谈危害,如果你对危害认识不深刻,你就不会下大决心去戒。只有当你深刻认识到撸管的危害,和撸管后果的严重性,这时你才会真正发出大决心来戒色。所以,看危害的文章对于新人发心戒色是非常有帮助的,而对于老戒友,危害的文章和案例则可以起到很好的警醒作用,我到现在还时常看危害的文章和案例,特别是受害者的真实案例,我基本天天都在看,力量源自真实!看真实的受害者案例,你就会想,如果我不戒或者我破戒,我的下场也会和他一样,所以受害者案例的警醒作用是很强大的。人有一个弱点,那就是好了伤疤忘了疼,所以还是要多看危害的文章和案例来警醒自己,让自己保持良好的警惕意识。我们就像非洲大草原上的羚羊,心魔就是猎豹,如果我们不警惕,那肯定会被心魔吞噬,不管你戒了多久,只要你放松警惕,等待你的肯定就是破戒。

这里要特别注意的就是,认识危害并不代表戒色就能成功!很多人会觉得论述危害的文章没用,其实他误解了此类文章的作用,论述危害的文章,它的主要作用就是让新人真正深刻认识到撸管的危害,毕竟很多新人脑子里面的思想误区很多,而且对撸管危害的认识很肤浅或者很片面,然后此类文章最重要的作用就是让新人发出戒色的决心;对于老戒友来说,多看危害的文章和案例可以让老戒友保持很好的警惕性。所以,论述危害的文章其作用是很大的,但想要戒色成功,认识危害远远不够,还需要全面学习其他的戒色文章,综合提高觉悟。只看危害的文章是难以戒色成功的,比如有些学中医的戒友,知道危害不比你多?但是他们也没有戒色成功。所以,了解撸管的危害只是第一步,第一步很关键也很重要。

{\it 认识危害是第一步,但认识危害并不代表戒色成功;

认识危害有利于新人发出戒色的决心;

认识危害可以让你看清撸管的真相;

认识危害有助于你时常警醒自己;

认识危害可以让你预见到撸管的后果;

认识危害有助于老戒友提高警惕;

认识危害可以解释身体出现的症状;

认识危害也可以用于对治意淫,

当意淫的念头一起,

马上思维危害来对治和降伏。

总之,认识撸管的危害格外重要,

但如果你以为认识危害就能戒色成功,

那么你就把戒色想简单了,

这仅仅是第一步。}

\subsubsection{论述戒色方法、戒色原理与规律的文章}

有句话叫:不是不可能,只是我们暂时还没有找到方法。戒色方法,也就是怎么戒。我一直提倡的就是通过学习提高觉悟,觉悟降伏心魔。千万不要强戒和盲戒,我指出的道路就是专业戒色之路,通过学习来获得戒色成功。诸如破戒的类型,如何断意淫,如何控制遗精频率,遗精后要注意什么,什么是戒断反应、欲望休眠期、症状反复期、戒色稳定期,情绪管理的重要性等等,这些你都要搞清楚。我通过自己的不断研究和总结,已经把我的成功经验写在《戒为良药》里了,不少人已经通过看这本书戒到一年以上了,半年以上的也非常多。我可以问心无愧地说,我戒到现在都没有破,是真真正正戒色成功了,当然这个成功,只是“到现在”为止,戒撸是一辈子的事情,我发愿坚持到生命的最后,真正做到彻底戒掉撸管恶习。

好的戒色文章很多,其他成功的戒色前辈也有自己的心得体会和方法,每个人的方法都会有所差别,但肯定有许多的共同点,成功的戒友都是相似的。我的戒色文章就是我的戒法,如果你对我有信心,那你可以专研我的戒色文章,按照我的戒色理论来戒色,当然你也可以看别的前辈的戒色文章,汲取精华的戒色思想,如果他已经戒色成功了,比如戒色一年或者二年以上了,那么基本就是可信的,否则他还没戒到半年,就说自己成功了,这种成功就要打个问号了,甚至只戒了一个月就说自己成功了,那可信度就更小了,自己还摸着石头过河,自己都没摸清楚戒色成功之路,然后就来教别人,那很可能会产生误导的情况。戒色后一个月左右,还在欲望休眠期内,很多人都会以为自己成功了,其实过了欲望休眠期就是破戒高峰期,这就是戒色的规律,这条规律来自于我十几年的亲身体验,也来自于广大戒友的案例反馈,很多人一开始容易轻敌,觉得戒色很简单很轻松,殊不知,欲望只是暂时休眠了,等你身体有所恢复,魔考就会来临!到时,就是验证你觉悟的时候,觉悟过关就不会破戒,觉悟不够肯定会破戒的。

所谓戒色成功,没有一定的时间标准,有的说要三年,有的说一年,我一般按照进入戒色稳定期来判断是否成功,有的戒友悟性极高,进入戒色稳定期相对较快,比如戒色半年就进入稳定期了,进入戒色稳定期有一个标志,那就是煎熬感消失,意淫极少。但最好应该要戒色一年以上,如果戒色三年了,那可信度就很高了。选择哪位前辈的文章,还是要看你自己的眼光了。

认识撸管的危害是戒色的第一步,第一步很关键,第一步也可以说是基础。而第二步则更关键,因为这二步做好了,你就可以戒色成功了。真正掌握了戒色方法、戒色原理与规律,戒色成功就是水到渠成的事情。

\subsubsection{论述养生恢复类的文章}

很多人是戒色成功了,但是身体恢复不理想。所以我一般都提倡的是“戒色养生”,我把这两个词经常放在一起说。我们戒色,也要注意养生,两手抓,两手都要硬!养生类的文章也是戒色文章的一部分,毕竟几乎每个戒友都会关注如何让身体更好地恢复,关于如何恢复,我也写过好几季,有介绍过自己的恢复经验,也介绍了一些养生功法。我一般都建议广大戒友要多学习养生的文章,多看养生的讲座视频,努力提升自己的养生意识,这对于恢复是有积极意义的。记得有一位戒友和我说,他说他的养生意识很强,但是每天久坐十个小时,这就自相矛盾了,养生意识强怎么还会久坐十个小时,久坐伤肾伤脾,容易导致气血瘀滞,也可能加重前列腺炎的症状。后来我建议他每四十分钟起来活动一下,不要一坐几个小时不动,那样对于恢复很不利。我们在戒色的同时,一定要注重养生,一定要在恢复方面下功夫,同样两位戒友,养生做得好,恢复速度就加倍,反之久坐不动,或者还有熬夜,那即使戒一年,也可能恢复很不理想。

\subsubsection{图片视频类文章}

有些戒友会觉得图片或者视频没用,这其实是偏见。我记得我戒色之初,也很喜欢看图片,但那时戒色图片少得可怜。戒色图片和视频有着很好的宣传效果,也可以令新人发心戒色,有的戒色图片会让人有眼前一亮的感觉,或者突然明白了一个道理。但是,戒色图片和视频就像开胃菜,开胃菜就是开胃菜,不能代替主菜!主菜就是戒色文章,还是要深入系统地学习戒色文章,广泛吸收和理解戒色知识,一定要掌握戒色的原理和规律,这才是最关键的。有的戒友通过看图片和视频来保持警惕,这也是有一定效果的,不少戒友受到诱惑,然后马上看戒色视频,欲望就下去了。所以,戒色图片和视频还是有一定效果的,不过话又说回来,如果你只看戒色图片和视频,而不看戒色文章,这样也难以戒色成功,因为你忽视了主菜,长觉悟主要还是要靠主菜!图片和视频是具有一定的作用,能够带给人鼓舞和力量,可以激励大家更坚定地戒色,属于戒色的助力,但图片和视频无法取代戒色文章,这点大家要认识清楚。

\subsubsection{佛教类戒邪淫文章}

佛教类文章,戒色吧还是时有转载的,可以接引有佛缘的戒友,佛教的戒邪淫文章的确是一种莫大的加持!学佛可以让人开大智慧。佛教的戒色文章一般就分两步走,你如果能做好这两步,即使不学习专业戒色文章,你也可以成功的。如果你既懂得佛教戒邪淫,又懂得专业戒色,那就是如虎添翼!

这两步就是:

\begin{enumerate}
    \item 邪淫的过患(促使你认识危害,并发心戒色)
    \item 控制念头!(直接进入实战!)
\end{enumerate}

我曾经说过,戒色就是念头与念头之间的战争!佛教修心其实就是如此,修心就是修念头!佛教虽然不说什么欲望休眠期、戒断反应等,但是佛教抓住了最最根本之处,也就是抓住了核心,那就是控制念头!这是很简洁的两步,一,认识危害,发心戒色,和专业戒色并无两样,认识危害都是共通的;二,就是念头实战!什么叫念头实战呢?就是邪念一起,马上用佛号一转,元音老人曾经讲过,佛号就是武器,要用佛号打退妄念!其他法师也说过,以一念代万念!不管什么念头出来,就用佛号一转。意淫一出现,马上提起佛号,意淫自然化为无形,还有就是“念起即断,念起不随,念起即觉,觉之即无”,我总结的这个断意淫口诀,其实就是脱胎于佛教。这个口诀如果能用熟练,也是可以降伏意淫的,这是一种觉照的功夫,你要学会看住念头,你要学会观察自己的念头,向内看,很多戒友都是意淫很久了,才突然发现自己原来在意淫,这时候就很难断了,因为意淫已经壮大了,就像火烧大了难扑灭,星星之火好灭,燎原之势就难灭了。所以断意淫贵早。不怕念起,就怕觉迟。

其实专业戒色,最后也要进入实战,也要看住念头,这和佛教都是共通的,所以戒色的最高层次就是修心。看一万本游泳教材,不如下水学习游泳!佛教戒色就是注重实战!可以说一步到位,而专业戒色研究的内容更多一些,两者各有千秋。如果你真能牢牢掌控念头,其实也不用学专业戒色的内容。而学专业戒色的内容,最终还是要回到念头的实战上。养兵千日用兵一时,一切的一切都要上战场来检验,这个战场不在别处,就在你的脑海里,当邪念来犯,就看你能不能打败它!很多人根本没有抵抗力,或者抵抗几下就沦陷了,如果你真能战胜邪念,那是可以守住戒色成果的,很多人之所以阵亡,就是打不过邪念,这就像格斗游戏,你打不过就是失败,打过了就是过关!

不管何种戒色文章,专业的或者宗教的,最后的实战,都是念头与念头之间的战争,禅宗里面常常讲:「打得念头死,许汝法身活」!你即使学过一千篇戒色文章,最后的检验还是念头之间的战争,看你是否能「降伏其心」!邪念一出来,或者遭遇到了诱惑,是否能够做到念起即断!是否能够对境不动心。真的有觉悟,还是觉悟不过关,一检验就出来了。如果你的觉悟真的达到了,那么即使魔考一百次,你也能够顺利过关,就怕你的觉悟还存在缺陷,那么肯定会破戒的。

\begin{multicols}{3}\it
    平时练兵,

    上战场一检验就分出高下了;

    实战时

    就知道自己真实水平如何;

    平时学习戒色文章,

    当意淫一出现,诱惑一出现,

    就看你的反应了!

    如果你有强大的觉悟,

    那么必然可以降伏之。

    反之,必破无疑!
\end{multicols}

\paragraph*{总结}

了解戒色文章的种类及其作用,可以有针对性地看戒色文章,比如撸管的危害你已经了解很深了,那么危害的文章可以少看一些。可以把主要精力放在研究戒色方法、戒色原理与规律上,你也可以把精力放在养生恢复上。有些戒色文章也不是纯论述危害的,也讲到了戒色方法与恢复方法,属于综合性的文章。对于各类戒色文章,我们要有一个清晰的认识,但不要低估任何一种戒色文章的作用,其实每种戒色文章都有相应的作用,只是看你自己的需要了。如果你对撸管的危害认识肤浅,那你就应该多看危害方面的文章和真实案例,如果你对戒色方法和理论一窍不通,那么你就应该多学习相关的戒色文章来提升觉悟水平。

\paragraph*{补充}

有时看戒色文章,很多人会经历一次“大觉醒”,撸管的想法和冲动刹那间就消失得无影无踪了,心瘾似乎没了,这时候很多人就以为自己戒色成功了。其实不然,这个世界上没有任何一篇戒色文章能够让你回到“婴儿意识”,也就是再也不起邪念,这是不可能的。很多戒色文章可以让你大觉醒,可以让你进入欲望休眠期,心瘾只是暂时消失了。但千万要记住,这只是休眠状态,就像火山的休眠一样,过了欲望休眠期就是破戒高峰期,到时候就会迎来魔考,这就是戒色的规律。很多新人不懂得这个规律,到时候就会出现一连串的破戒,在欲望休眠期内,很多人会变得过度自信,甚至自傲乃至自狂,他会觉得戒色很简单嘛,有什么难的。我之前的文章也讲到过这个问题,很多新人往往在欲望休眠期内发帖说自己成功了,然后没过多久魔考一来,他就破戒了。之前他会觉得自己的戒色思想是最好的,破戒之后才知道其实自己的觉悟还很低,根本就经不起魔考的检验。

SY 的瘾是很强大的,其他瘾,比如烟瘾、酒瘾、毒瘾、赌瘾、网瘾等,也很厉害,其中最难戒的应属毒瘾,但最容易上瘾的,SY 可以排在前面,基本都是一次上瘾,一发不可收拾。而且戒 SY 有一个最大的难点,很多人戒烟成功了,就是无法戒除 SY,戒 SY 的难度在于,烟、酒、毒品、电脑等,都属于外在的物质,比如把你扔在一个荒岛上,你还能抽烟喝酒吗?你还能赌博吗?你还能玩电脑吗?你还能买到毒品吗?都不可能了。但是,你还可以 SY,你还可以 YY,因为 JJ 就长在你身上,YY 就在你脑海里,即使把你五花大绑,你依然可以 YY。这就是最难点之一。还有一个难点就是无害论的泛滥,比如抽烟,一般人都知道香烟盒上有吸烟有害健康的字样,也是提醒你抽烟的危害,你戒烟的话,家人一般还会表示支持。但是当今无害论泛滥,你戒撸很可能会遇见很多不理解乃至误解的情况,所以戒撸的难度就更大了一些,这也是一个黄毒泛滥的年代,诱惑到处都是,防不胜防,撸管材料很好搞!人人都有手机,一上网到处都是。所以,SY 的瘾较之于其他的瘾,有其自身的难点,不是那么好戒的,必须专业系统地学习戒色文章,深入掌握戒色的原理与规律。

还有的戒友会说,你们戒色是渐悟,我戒色是顿悟。顿悟是有醍醐灌顶之功效,自感心里豁然开朗,对于一件事或者一个道理突然有所领悟。快速直入究极之觉悟,称为顿悟;依顺序渐进之觉悟,称为渐悟。顿悟是好,但他以为一顿悟就成功了,其实不然。禅宗讲的顿悟,不是一悟便休,还是要悟后真修的,悟后绵密保任除习气,还要在境上磨炼!练到对境不动心才可以。所以,有些戒友误解了顿悟的含义。以为一顿悟就彻底戒色成功了,真的是这样吗?其实戒色没那么简单的,一次顿悟可以让心瘾暂时消失,这倒是真的,但是魔考迟早还会来的,即使进入戒色稳定期,偶尔还是会出现魔考的,所以还是不能放松警惕!多少顿悟的戒友最后还是破戒了,知道为什么会破戒吗?说到底还是觉悟上存在缺陷!

\begin{multicols}{3}\it
    不管你看过多少戒色文章,

    不管你经历过怎样的顿悟,

    最终都要回到实战上,

    所谓实战,

    就是念头之间的战争,

    当遭遇诱惑,

    当意淫出现,

    当怂恿的念头出现,

    就看你怎么办了!

    实战检验一切!

    魔考迟早会来临,

    到时候就见分晓了。
\end{multicols}

另外,很多戒友喜欢速成,恨不得看一篇戒色文章就能彻底戒掉,现在似乎是一个速成的年代,但是大家仔细想想,你学完小学要几年呢?你学完高中要几年呢?你学完大学要几年呢?戒色是一项系统工程,急于求成的心态是要不得的,还是要脚踏实地,扎扎实实地学习戒色文章来提高觉悟,所谓欲速则不达,慢工才能出细活。罗马不是一天建成的,很多戒友心浮气躁,连戒色也是如此,一直想走捷径,想搞速成,结果呢?其实,少走弯路即是捷径,多学习前辈经验即可少走弯路。不要幻想看一二篇戒色文章,就能彻底戒掉。你问问成功的戒友,他们看了多少戒色文章,有的戒友光戒色笔记本都有好几本,还有的戒友甚至把《戒为良药》看了十遍以上,戒到现在一年多,都还没有破戒。看一二篇戒色文章,可以让你发出戒色的决心,可以让你很快进入欲望休眠期,但千万别以为自己已经成功了,你只是在欲望休眠期内,过了欲望休眠期就是破戒高峰期,魔考肯定会来临的。关于欲望休眠期,不只我的戒法有专门提出过,很多戒色前辈都有过类似的表述,就是肾气稍微一恢复,人就会破戒。欲望休眠期也就在一个月左右,到时候就很容易出现破戒的情况,而在欲望休眠期内,可以戒得很轻松。我们戒色一定要认识到欲望休眠期的存在,这就是戒色的原理与规律,非常关键,切记。

现在到处都在搞速成,比如很多宣传鼓吹七到十天保证说一口流利的英语,如果这样的话,所有的大中小学都不要开英语课了,在这里强化几天就行了,何必那么麻烦呢?戒色也是如此,不要指望一蹴而就,要注重学习和积累,不断完善自己的觉悟,这样才能最终戒掉撸管恶习。我的戒法就是主张脚踏实地,扎扎实实提高觉悟,当你觉悟达到了,自然可以降伏心魔,从开始学习戒色文章到觉悟强大到能够降伏心魔,这中间有一个过程,这个过程就像姚明从 1.5 \unit{\m} 长到 2.26 \unit{\m},你会认为姚明是一夜之间长到 2.26 \unit{\m} 的吗?这不是速成可以达到的,这是一个不断积累的过程,量变产生质变的过程,这才符合戒色成功的规律。

古德有云:今日之顿,亦昔日之渐也。即使你是顿悟,也不可能逃出量变产生质变的规律,就像烧开水,不可能一下达到沸点 100 \unit{\degreeCelsius},而是慢慢加热达到沸点,顿悟也不代表彻底成功,顿悟后依然要进行念头的实战,即使你进入了戒色稳定期,还是要进行念头的实战。说到底,戒色就是念头与念头之间的战争,正念弱,邪念强,必然破戒;通过学习让正念变强大,邪念自然就弱了。正念不断变强大,最后就能“降伏其心”,如果你想让自己永远不起一个邪念,回到“婴儿意识”,那你就不是凡人了。佛教有讲到伏惑与断惑,制伏所起之惑,而使之一时不起,称为伏惑;断绝惑种,而使之毕竟不生,称为断惑。又说到“断见惑如断四十里流,况思惑乎。纵令见地高深,以烦惑未断,仍旧轮回。再一受生,退者万有十千,进者亿少三四。”断惑是很难的,能把它暂时伏住已经不易,我现在能做到的就是伏住淫欲心,邪念一起,我能马上把它降伏住,但淫欲的种子并未彻底断掉。如果你真能把淫欲这个种子彻底断掉,比如戒色三年没有起一个邪念,那境界是相当高了,说实话,目前我还没有达到这种境界。

附禅宗过三关:

\begin{description}
    \item[破初关,又叫法身边事] 已知禅为何物,并有所体会,进入“见道”位。但定力不足,妄念、执着不时现前,要时时觉照,有如牧牛,渐除习气,此时进入“保”的阶段,又叫“守道”。
    \item[破重关,又叫法身正位] 习气基本除净,心已不随境转,一片空明,无内无外,但尚有法执,佛见未除。
    \item[破牢关,又叫法身向上] 进入“任”的阶段,已经“了道”,无修、无证,无智、无得,无凡、无圣,随缘无住、任运自在、方便善巧觉悟有缘众生。
\end{description}

由上可知,禅修的三关,初关是开悟的境界,重关是证道的阶段,牢关则是彻底的解脱。

这季创作了五首戒色诗歌,我的诗歌基本都是口语诗,很好懂,奉献给大家。

\begin{poem}[美女凶猛]
    \begin{multicols}{4}
        \begin{center}~\\
            跑了无数次医院 \\ 花掉上万的医药费 \\ 动过两次手术 \\ 吃过几十种药物 \\ 他最终得出一个结论 \\ 美女是一种废器 \\ 而且越美越废 \\ 越美废得越狠 \\ 美与废成正比 \\ 美女貌似柔弱 \\ 但柔弱能胜刚强 \\ 一身的凶器 \\ 废人杀人于无形 \\ 用最舒服的方式废你 \\ 让你废得心甘情愿 \\ 美女猛于虎 \\ 美女猛于车祸 \\ 有了这一层认识 \\ 不禁毛骨悚然 \\ 拉你下地狱的 \\ 就是你最喜欢的美女 \\ 贪皮肉,下地狱! \\ 铜柱铁床伺候!
        \end{center}
    \end{multicols}
\end{poem}

\begin{poem}[衣冠撸兽]
    \begin{multicols}{4}
        \begin{center}~\\
            你变了 \\ 好好照照镜子 \\ 好好看看你现在的样子 \\ 你的确变了 \\ 眼神变了 \\ 气色变了 \\ 整个气场也变了 \\ 沉迷撸管 \\ 让你开始腐烂 \\ 由内而外的腐烂 \\ 灵魂的腐烂 \\ 肉身的腐烂 \\ 就像一个烂苹果 \\ 一身的撸气 \\ 一身的邪气 \\ 因为撸管 \\ 你已经不再新鲜 \\ 一种变质过期的感觉 \\ 一种令人厌恶的感觉 \\ 撸管的危害 \\ 好好照照镜子吧 \\ 你都让自己恶心了
        \end{center}
    \end{multicols}
\end{poem}

\begin{poem}[抢精时代]
    \begin{multicols}{4}
        \begin{center}~\\
            中国在不知不觉间 \\ 已经进入了抢精时代 \\ 生活中,网络上 \\ 到处是抢精犯 \\ 真可谓防不胜防 \\ 既然被抢了 \\ 那肯定要出症状 \\ 这是必然的 \\ 身体是本钱 \\ 肾精是本钱的本钱 \\ 而美女专抢肾精 \\ 专门抢你的本钱 \\ 人活着,精没了 \\ 那就是行尸走肉 \\ 双眼无神,目光呆滞 \\ 气色似鬼,症状缠身 \\ 当你意识到 \\ 这是一个抢精时代 \\ 你就要提高警惕了 \\ 我想问你 \\ 你今天被抢了吗?
        \end{center}
    \end{multicols}
\end{poem}

\begin{poem}[我退出]
    \begin{multicols}{4}
        \begin{center}~\\
            成千上万 \\ 成万上亿的人 \\ 在加入撸管大军 \\ 这是一股超强大的洪流 \\ 裹挟着你前进 \\ 不容你思考对与错 \\ 无害论已经把你砸晕 \\ 只要你能勃起 \\ 你就能加入 \\~\\
            每一天都有人发育 \\ 每一天都有人加入 \\ 他们迫不及待地想加入 \\ 与他们的疯狂加入 \\ 形成鲜明对比的是 \\ 我毅然选择退出! \\~\\
            我宣布,我退出! \\ 我退出撸管界! \\ 当疯狂的撸流 \\ 朝着自毁的方向 \\ 盲目撸动时 \\ 我大喝一声 \\ 我退出!
        \end{center}
    \end{multicols}
\end{poem}

\begin{poem}[感觉]
    \begin{multicols}{4}
        \begin{center}~\\
            你看母猪有感觉吗? \\ 你看母鸡有感觉吗? \\ 你看母狗有感觉吗? \\~\\
            你看母猪不会有性冲动! \\ 你看母鸡不会有性冲动! \\ 你看母狗不会有性冲动! \\~\\
            跳出人的局限, \\ 对美女一样没感觉!
        \end{center}
    \end{multicols}
\end{poem}

这季继续推荐一本书:

\begin{book}[《净土资粮信愿行》,大安法师]
    大安法师是净宗祖庭——江西庐山东林寺方丈、代住持、《净土》杂志主编、净宗研究生班导师。大安法师原来是对外经济贸易大学教授,2001 年出的家,大安法师的开示我看过不少,主要以视频为主,《净土资粮信愿行》在网上也有视频版,我目前看的是书籍,真是非常好的一本开示,在接引初机方面也非常好,我在这季推荐给有缘的戒友,大家可以看看大安法师的开示。因为法师原来是大学老师,所以世间法和出世间法都非常通达,大安法师的讲座很有亲和力的,我个人比较喜欢。希望有缘之人不要错过。南无阿弥陀佛!
\end{book}
