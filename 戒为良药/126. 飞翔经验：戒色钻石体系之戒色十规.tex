\subsection{戒色钻石体系之戒色十规}\label{126}

\paragraph*{前言}

一位哲人说过:“所有的真理都要经过三个阶段:首先,受到嘲笑;然后,遭到激烈的反对;最后,被理所当然地接受。”看了这句话很多戒友应该很有感触,深以为然,戒色吧的确受到了很多误解、嘲笑、诽谤、诋毁和反对,记得刚开始的几年反对的声音还比较少,主要以 75 党为主,那个阶段也就几十万人,引起的关注也不多,比较小众化。后来发展壮大了,所谓树大招风,关注的人多了自然反对就多了,毕竟我们面对的是几千万乃至几亿被无害论洗脑的人,有害的真理一时让很多人难以接受,所以自然就会反对。这也符合常理,有接受的肯定就有反对的,表面看反对是一件坏事,但我认为反对也是一种宣传,属于反向宣传,让更多的人知道戒色吧,知道戒色这件事,等到撸出症状自然会来戒色吧,才发现戒色吧并不像外界所诽谤的那样,戒色吧是一个充满正能量的贴吧,大家互相鼓励,互相支持,无私奉献,是一个拯救人的贴吧。\textit{即使只有少数人相信,真理依旧是真理!(甘地)} 我相信只要我们坚持宣传和推广戒色理念,将来必定会被社会大众理所当然地接受,人们会真正认识到色情与手淫的危害,并且自觉戒除。我想说的是:你可以对戒色理论嗤之以鼻,认为是胡说八道的歪理邪说,但是当症状爆发,你就知道什么是手淫的代价了!到时自然会开始戒色,因为不戒不行了,身体吃不消了。在初中时我就知道手淫不好了,因为每次手淫后总感觉身体不大舒服,有一种被掏空的感觉,撸了一两年后,身体就开始出现一些症状了,勃起不坚、早泄、腰痛、腿软、尿频、脑力下降、体能变差、鼻炎加重、痘痘增多、精神萎靡等,那时就隐约知道手淫的危害了,毕竟事实摆在那,只是当时年纪尚小,比较无知,也无人告知手淫的危害,所以认识比较肤浅和模糊,也不懂怎么戒色,尝试很多次都没戒掉,一次次的沉沦让原本纯真的少年沦为了行尸走肉,学习成绩也大幅下滑。戒色是很重要的,从少年时代就要开始戒色,要懂得控制自己的欲望。无害论迟早会被淘汰,有害的真理会被大家普遍接受,这是一种趋势,国外近些年来也在大力宣传色情与手淫的危害,他们也认识到了。星星之火可以燎原,让我们坚持下去,帮助更多的孩子,更多的人。唤醒他们,帮助他们,让戒色的正能量传遍神州大地!

下面分享一些案例。

\begin{case}
    不知不觉戒色一年多,回想起一年多前的自己瞬间就有流泪的冲动,那是被自己感动了,一年多的默默坚持,虽说不上有翻天覆地的变化,但我却时不时感受到了童年时的蓝天白云回来了,和原来那样清新生动,让人无限遐想,感慨万千。一年前在公园散步都是自然而然地走在路的最边上,感觉不想让人看见,如今却自然而然地走在路中间,内心明亮坦荡,我深知过去十几年来砍伐自己,不是短时间能弥补,但我看到了一条光明之路,我要一直走下去,一直走下去。近期心魔涌动,各种威逼利诱,确实让我内心起了波澜,也更加明白自己还是多么弱小,我定会加倍警惕,时刻提防,明年我觉得将会是个大有作为的一年,加油!

    \textbf{附评} 生命中清澈如许、纯真喜悦而又光明坦荡的一段时光,就是童年的美好时光,是邪淫剥夺了你内心纯净美好的感受,让你沉沦在色情的粪坑里如蛆虫一般上下翻滚。戒色是内心的净化,修善是正能量的累积,让你再次感受到童年时的蓝天白云,白得那么纯粹,蓝得那么醉人,你的眼睛倒映着蓝白相间的天空,那么唯美,那么震撼,用心感受纯净灵魂的大美!高能量的人总是那么积极、乐观、快乐、精神爽利兼神采飞扬,让你的心灵成为一个纯洁的圣殿,让我们永远沉醉于高频的快乐境界!一颗阴暗猥琐的心永远托不起一张纯真灿烂的笑脸,一个自甘堕落的撸者永远不知道戒色有多爽!这位戒友逆袭了,童年的神奇感觉又回来了,纯真的眼眸再一次睁开,单纯地凝望着这个世界,无限的美好在内心中逐渐展开。他变得自信坦荡了,我邪淫的时候也总是走在路的最边上,还不敢看别人眼睛,缺少对视的光明磊落,缺少那种底气和自信。坚持戒色修善,一种力量、自信和底气逐渐在内心增长,你能感觉到内心微妙的好变化,有一种气壮山河的感觉,体内真气澎湃,一股豪情万丈油然而生。一旦发生破戒,那种内心的力量、自信和底气又会突然消失,就是这么微妙。这位戒友用到了“砍伐”这两个字,很到位的表述,色字头上一把刀,沉迷手淫就是在用色刀砍伐自己,这是一种伤害,不是享受,愚夫会被短暂的快感所欺骗,看不到之后的恶果和报应,而智者看得很远,所以坚决不犯邪淫。戒色修善是一条光明的大路,邪淫是一条无比崎岖的布满地雷的危险之路,我们要坚定地走光明的大路,严防被心魔再次拖入那个魔咒般的怪圈。当你的能量水平提升了,你就可以在学业或事业上有一番作为了,你也有强大的勇气去面对人生的各种挑战了。
\end{case}

\begin{case}\label{126case2}
    我放纵自己将近二十年,从小学五年级就学会了这种恶习,初中以后就开始有些频繁了,病症也一个接一个地浮现出来,当时也没在意,也不会和恶习扯上关系!直至今年六月份,当病魔危机到生命那一刻,我才有所察觉,不是不报时候未到!我患上了高血压以及心脏神经官能症,简直可以用痛不欲生来形容!记得发病前,我有过一天三次,当时没什么事,隔天也没事,但第三天我便感觉头有些疼,吃过一些止痛药却不见好转,于是去检查,测量血压得到的结果是,高压 164!低压 121!心率 139!我永远都会记得这三个数字!我形容一下当时的感觉,就像脖子上顶着一颗球,然后有人拼了命地往球里打气,直到打爆为止!医生诊断是高血压以及心率过速,开了一些药物!吃过药之后也不见有什么显著效果,我记得当时最舒服的时候,也就起床后的几分钟,过了那几分钟我又开始浑身难受了。顶着无药可救的病痛,又去另一家医院,检查结果都一样,就是高血压引起的头疼,但为什么吃了降压药却不见任何效果呢?换了好几种降压药通通无效!甚至我的家人也怀疑我就是在装病!我放弃了,不再去医院检查了,我自己上网查有没有我这种情况的,结果还真有!后来我才弄明白原因,高血压是分为原发性和继发性,我应该是后者,是因为其他脏腑器官出了问题,引起的血压上升,我第一个想到的就是肾!于是我便开始适度节制,也是从那时开始接触适度无害的,之前真没听过,但是通过我的以身试法、顶风作案的结果证明,简直就是扯淡!!!就像您说的那样,恶习有着高度成瘾性!前两周我还是可以一周一次,但第三周就变成了三次,心里话说,多一次也没事,结果第二天就变成,反正昨天都那样了,那就再来一次吧!那时我还没能彻底认清危害,顶着痛苦过着度日如年的生活,后来听说运动可以降压,我便下载了运动软件,开始了与人攀比步数的日子,记得最疯狂一次,我步行三十公里,四万多步,拿到了当日第一,但这样过度的运动不出一月,我的膝盖就废了!我躺在床上不能走路,还要顶着心脑血管的痛苦时,我绝望了!我甚至觉得我活不过今年,我想过轻生!活着太招罪啦!我应该也有抑郁症吧,对别人我会保持微笑,强装镇定,对自己我只会保持安静,只有沮丧!直到去年我来到了戒色吧,就仿佛找到了家!很多戒友都在推荐您的《戒为良药》于是我也找来看看,刚看了一个开头,就仿佛一面镜子立在我的面前,就好像高考公布了分数,就像是侦探抓住了凶手,我的天呐!这内容说的不就是我吗!!!从此我便痛下决心,此生不再放纵自己,这才对得起自己!至今也快五个月啦,我没有破戒,我要用我的过去来警醒那些后来的人!别等到像我这样,老底儿都快挥霍空了,才悔恨当初!你放纵的每一次,都要用十倍甚至百倍的努力才能换回来,请别再祸害自己了!

    \textbf{附评} 放纵二十年,也是该还了,伤到一定程度身体就开始拉警报了,神经症爆发后那真的是生不如死、痛不欲生,严重的患者更是度日如年。根据大量案例的反馈和我的亲身体会,伤精症状有时是会延迟的,不一定是放纵的第二天出现症状,可能是第三天或者几周后再爆发出来。快感是短暂的,痛苦是漫长的,痛苦是快感的 N 倍!当痛苦降临,你就要开始“享受”报应了!非常惨的下场!非常惶恐乃至恐慌绝望的体验,绝对的灰暗,绝对的暗无天日,更惨的是不被家人理解,以为你装病,表面看着你还算正常,其实你的身体已经被邪淫掏空了,千疮百孔,完全就是豆腐渣工程。千里之堤毁于蚁穴,举鼎之男毁于纵欲,一次次纵欲为将来的大崩溃埋下了伏笔,到时候身体突然就垮掉了,病来如山倒!健身界的网红“巨臂哥”婚后半年离世,网上有说用药过量所致,因为他有长达二十多年的类固醇等激素服用史,但我觉得和婚后纵欲也是密不可分的,新婚多放纵,再加上之前的放纵史,再强壮的男人也经不起如此的耗损。巨臂哥在视频中说过:“坏事最终会找到你!”感觉他这句话不仅是在说滥用类固醇,也是在说沉迷色情,最终坏事真的找上了他,非常可悲的结局,巨臂虽然震撼,但也无法抵挡死亡的降临,与其练出震撼的巨臂,不如练出巨臂般强悍的觉察力!觉察力强悍,才能真正主宰内心!

    当你五脏失调了,肯定会影响到血压,高血压的中医分型:

    \begin{description}
        \item[肝阳偏盛型] 表现为头痛、性情急躁、失眠、口干苦、面红目赤等;
        \item[肝肾阴虚型] 表现为头部空虚感、头痛、眩晕、耳鸣、面部潮红、于足心热、腰膝无力、易怒、心悸、乏力、失眠、健忘等;
        \item[阴阳两虚型] 有严重的眩晕,走路感觉轻浮无力,心悸气促,面部或双下肢水肿,夜尿多,记忆力减退,畏寒肢冷,腰膝酸软,胸闷、呕吐或突然晕倒等症状出现。
    \end{description} 这位戒友一天三次,三天后血压超标了,正常的血压范围是收缩压在 90 - 140 \unit{\mmHg},舒张压在 60 - 90 \unit{\mmHg},正常成年人安静时的心率平均在 75 次/分左右(正常范围 60 - 100 次/分之间)。他三项都超了,不少人疯狂手淫后这三项都会超标,特别是连续多次后很容易出现超标。这位戒友之前也相信适度无害论,然而实际操作就从一周一次变成一周多次了,手淫很容易一发不可收拾,一定要彻底掏空才过瘾,往往撸一次感觉不爽,还不死心,要连着来两次甚至三次,这样身体怎能吃得消?这位戒友想到了锻炼,但是没有养生意识,过度的运动让他膝盖废了,不能走路,真是雪上加霜,可见养生意识有多么重要!好在他遇见了戒色吧,戒色吧就像黑暗中强大的灯塔,照亮了他人生的道路,也给了他重生的希望。“请别再祸害自己了!”这是多么振聋发聩的告诫,彻底惊醒手淫梦中人,喊醒那些执迷不悟的人,召唤他们迷途知返、回头是岸。这是多么痛的领悟,请别再祸害自己了!不仅为了自己,也要多为家人着想。
\end{case}

\begin{case}
    我曾经戒了八个多月,过年期间不小心破戒了,破戒后不到两个月的时间,已经连破十八次了。破戒后,我每天学习戒色文章、做笔记、读诵心经、四种决定清净明诲、楞严咒、南无观世音菩萨圣号,最近又开始放生。我比没破戒之前更努力了,但是依然止不住破戒,找不回曾经的状态。我真的努力了,连病了身体不舒服,都止不住破戒。我很迷茫,学习戒色文章、持经诵佛、放生我都在努力做,但还是突破不了现在的状态。我的突破口在哪?我到底该怎么办?

    \textbf{附评} 这位戒友之前戒得还是很不错的,可惜不小心破戒了,这一破戒,就难以找回良好的戒色状态了。破戒后一定要认真总结和反省,关键一定要认识到观心断念的重要性,努力提升实战水平,他是做了很多功课,也学习了戒色文章,念经念佛放生,但是最终的实战表现是否有提升呢?这就要打个问号了。也许你做了很多功课,但是当邪念一上来,那种图像画面在脑海中一浮现,还是断不掉,那依然会破戒的,虽然做了很多功课,但没有围绕实战去提升,那还是等于零!训练一定要以实战为核心,不是说念了多少经、多少佛号、放了多少生,就能戒掉的,关键还是修心,不能停留在表面形式。就像有些人也练武,练套路,练架子,看着也像那么一回事,但是实战时还是抱头挨打的份,为什么会这样?因为他们的练习没有紧密地结合实战!他们完全是脱离实战的。念佛号也要把握要领,要知道念佛号也是断念的一种方式,一念代万念,来达到斩断邪念连续的实战效果。我们戒色,不管是走专业戒色的路线还是信仰戒色的路线,一定要深刻地认识到核心,注重提升实战表现,否则学习了很多戒色文章,做了很多功课,然而实战表现却没有实质的提升,当遭遇心魔时,还是和过去一个样,那只有被心魔虐的份,根本没有还手之力。心魔就像灭霸一样强大,要战胜心魔就必须比它更强大,心魔虽然强大,但并非不可战胜!这位戒友说“止不住破戒”,那种连续破戒的状态的确很疯狂,很身不由己,我深深体验过那种无助的被奴役的状态,破戒后就是空虚和悔恨。心魔满脸狰狞,挥舞着鞭子抽打在猥琐的躯壳上,心魔大声吼道:“给我全部撸出来!”一位戒友说:“每次说要戒,邪念来了,感觉人都是疯的,抓了狂地找视频找资源,感觉自己变了个人。”不要放出心魔,否则你会进入“三疯状态”,疯狂找黄、疯狂看黄、疯狂手淫,那个猥琐的身影又开始忙碌起来了……无尽的荒唐、无尽的堕落、无尽的颓废、无尽的空虚、无尽的悔恨。戒色后我们要定期考察一下自己,就是自己的断念实战水平是否有所进步和提升,这非常关键,如果没有明显的提升,那就考虑思想认识上很可能存在误区。比如你做了很多笔记,也看了很多戒色文章,也有念佛持咒放生等,但你的实战水平还是菜鸟级别,那就要认真反省了。我不看你读了多少戒色文章,念了多少经多少佛号,我只看你实战的那一下!那一下到底行不行?到底狠不狠?你要具备金刚断力!你的气势要像刚完成了史诗级的绝杀一样!惊天的壮举!那种眼神、那种气魄、那种犀利!!!够硬气!够铁血!够硬汉!!!请记住一点:一切的一切都必须围绕实战来展开!理论再强,实战一触即溃,那还是垃圾一个!在网上看过一篇军事文章,里面讲到:“紧紧扭住战斗力生成这一龙头,始终立足实战狠抓军事训练,有效摔打磨砺部队,不断提高应急应战能力。”我们要立足实战抓学习、抓训练,学习和练习就是为了提升实战水平,任何的戒色功课也是为了实战的那一下,千万不可脱离实战来做功课,否则必将失败!
\end{case}

\begin{case}
    我手淫十年了,昨天破戒三次,半夜躺在床上玩手机心脏突然剧痛,喘气都很困难,血压心率急剧上升,高压 170多、低压 110、心率 117,给我吓一跳,身体肯定出问题了,要不然血压不可能瞬间升高,还有昨天自己做了炸土豆片,油占半个锅,盐放多了齁嗓子。半夜去了医院挂了急诊科,简单说了一下自己的身体症状然后去做了心电图,结果没什么特别的,就是心率过快,在医院又量了一下血压比上次高了十个点,高压 183,医生给我开了四种降压药,有一个是降低心率的,还有一个是中药,在医院医生就给我吃了一片含在舌根底下,回家再量就恢复正常了。

    \textbf{附评} 破戒三次加上熬夜,心脏抽了,三项也超标了,饮食方面油和盐也放多了,吃盐过多也容易导致高血压。手淫十年应该只有二十五岁左右,还是很年轻,但身体也招架不住疯狂的耗损和熬夜,很多年轻人都比较无知,对手淫危害没有任何认识,而且还被无害论洗脑了,也缺少养生意识,熬夜久坐,暴饮暴食,所以现在二十多岁身体垮掉的非常多。前段时间聊过一个高三的戒友,去年他的头发还很多,今年头顶就开始稀疏了,他也就十七八岁,撸龄应该有好几年了,他心里很慌,担心自己秃顶。手淫的伤害是逐步累积的,短时间可能不太明显,但累积到一定量就会突然爆发症状,然后形成一种趋势,要扭转这个颓势必须坚持戒色养生,还需要一定的恢复时间。案例 \ref{126case2} 的戒友伤精史有将近二十年,并且已经撸出神经症,吃药降压已经很难立刻奏效了,毕竟伤得深了,沉疴难起。这个案例的戒友吃降压药还有效,但内伤已经造成了,之后再撸很容易出现这方面的问题,长期吃降压药也有一定的副作用,可能会导致身体其他方面的不适。高血压已成为严重威胁国人健康的“隐形杀手”,目前不少高血压患者遇到降血压两个瓶颈,第一,血压降不下来;第二,血压降下来了,却稳不住。高血压是一种常见病和多发病,此病一般起病缓慢,患者早期常无症状,或仅有头晕、头痛、心悸、耳鸣等症状,表面上看是一种独立的疾病,实际上是引发心、脑血管和肾病变的一个重要的危险因素,如果治疗不当就会病变成为较严重的脑中风、心肌梗死和肾功能衰竭等这些常见高血压并发症。其实我们的身体是很脆弱的,年轻时就要注意保养了,盲目无知很可怕,什么也不懂,就知道疯撸,疯狂掏空自己,压根儿不知道会引发什么严重的后果,完全无知,无知者无畏,等到症状爆发才知道怕!撸管刚开始会觉得爽,后来会撸到怕!撸到一次都伤不起!\textit{舍利子是人修行持戒而有的,不杀生、不偷盗,主要是不邪淫。因为不邪淫,则自己本身宝贵的东西,不会丢了。那“宝贵的东西”是什么呢?我相信每个人自己都知道,什么东西是自己生命的根本,我不需要说太多。你如果没有邪淫的行为,舍利子自然会光明灿烂,比钻石都坚固。(宣化上人)} 我曾经也很无知,后来才知道肾精是多么宝贵,也知道了持戒其实就是在保护自己,孔子说君子有三戒,第一戒就是“戒之在色”,戒色才能保住能量,有了能量就能干成很多事情,手淫不仅泄掉一个人宝贵的能量,也泄掉一个人的斗志和正气,脾气也会变得暴躁,容易和人吵架,负能量爆表。从少年时代就要懂得戒色的意义和价值,这点至关重要,身体是革命的本钱,把身体撸垮了,一切免谈!
\end{case}

\begin{case}
    我今年 29 岁,SY 已经有十多年了。就在前年我查出得了痛风,当时是蒙得不知道是什么病,随后就关注了百度痛风贴吧。大家可能不知道痛风是什么样的感受,我可以简单地描述一下:

    \begin{itemize}
        \item 短则几天,多则几个星期,失去行动能力,只能卧床靠人照顾,基本等于瘫痪的一个废人,更不要提学习和上班了;
        \item 通风发作是 24 小时的,发作关节处红肿,剧痛无比,只能靠吃止痛药和激素药来镇痛;(大家可以查一下止痛药和激素药对人体伤害有多大)
        \item 通风恢复期要靠吃降酸药来控制血液里的尿酸值,常用药物有:别嘌醇、苯溴马隆、非布司他等。前两种药物副作用极大,别嘌醇伤肝,苯溴马隆伤肾。非布司他副作用相对来说小一点,但是新药价格昂贵。如果吃非布司他维持降酸每个月大概需要一千才吃得下来,长期吃下来肝肾已废。
    \end{itemize} 为什么我在戒色吧提痛风呢?因为从我关注本吧以来,开始戒色,我的痛风再也没有发作过,而且戒色文章提到的其他症状也在逐渐改善,我就不多说了。相信肯定有师兄和我一样有 SY 型痛风,望大家赶快戒色回归健康生活,也希望飞翔师兄把我的经历收录到《戒为良药》里,警戒他人。

    \textbf{附评} 痛风又称“高尿酸血症”,是一种因嘌呤代谢障碍,使尿酸累积而引起的疾病,属于关节炎的一种,又称代谢性关节炎,患者多于三四十岁以后发病,一般间歇性发作,主要表现为拇趾、踝及指关节等部位红肿,且伴随针刺、刀割般的锐性疼痛,多数患者会伴发高烧,痛风有急性痛风和间歇性痛风。不得上这个病,一般不会去主动了解它,除非你是做医生的,或者家里有痛风病人。完全陌生的症状世界,在自己得上后才有了切身的痛苦体验,从这位戒友描述中已经能感受到那种痛苦了,剧痛无比,失去行动能力,就像瘫痪的废人,还要吃各种药物,费钱还有副作用,这种生活太无望。失去行动能力对普通人而言很可怕,现在看来,能自如行走,能跑能跳是多么幸福啊!身体垮掉后,连自如行走都成了莫大的奢侈。痛风的中医辨证分型有好几种,脾肾阳虚型痛风和肝肾阴虚型痛风等,手淫十多年得了痛风,体验到了那个可怕的症状世界,这个案例很值得收录,很多戒友对于痛风一无所知,看了这个案例应该会有所警醒。可喜的是,这位戒友在戒色后痛风再也没有发作过,这就说明了戒色对于身体康复是有极大好处的,也可以省下很多医药费,从而减轻了家庭的负担,自己的人生也重新充满希望了。一位戒友感慨地说:“我发现‘戒为良药,不泄为补’这八个字可以省下很多医药费,没明白这几个字的意思之前,我白白花掉了几万块!”现在看病吃药也贵,有的人看病都看不起,吃药都吃不起,检查也很花钱,多跑几趟医院就知道一个好身体实在太重要了,健康的时候不觉得,等到身体不行了才知道健康是多么宝贵,肾为先天之本,肾精充足,五脏六腑皆旺,抗病能力强,身体强壮,则健康长寿。反之,肾精匮乏,则五脏衰虚,多病早夭。古人很有“保精”意识和“宝精”意识,第一个保,是保护肾精能量,要保住它,不可妄泄,这是养命的能量,岂可随意撸掉?另外一个宝贝的宝,就是要懂得肾精是人体的宝贝、至宝,要懂得珍惜能量,当宝贝一样珍惜。所谓:“丹田养就长命宝,万两黄金不予人。”
\end{case}

下面步入正文。

有戒友建议我写个更加明确的体系以便大家去执行,其实戒色界已经有一些体系了,每种体系各有千秋。我之前也总结过十五条戒色成功的要点、戒色心髓十颂等,已经比较系统了,之前的总结是几年前的事情了,这季我将做一个更深入、更完善、更全面、更系统的总结。戒色十规的每一条都饱含力量,每一条都是经过无数实践验证所总结出来的,每一条都那么重要,每一条都戳中了要点,每一条都极具针对性。这十规是我今年对戒色的一次巅峰总结,也是这么多年来戒色经验的一次巅峰总结,戒色十规给你戒色成功的力量!吃透戒色十规,落实戒色十规,戒色成功将指日可待!这是一个更明确、更系统、更专业、更科学的体系,这个体系可以让你最快地把握戒色的核心和实战的精髓。这个体系最重要的一个特点,就是它已经非常非常成熟和完善了,就像成熟的果子等待你的采摘、品尝、消化和吸收,它之所以成熟,是因为这些精华要点久经验证,资深戒友一看就知道这个体系的价值和分量。这个体系不仅参考了前人和今人的各种戒色方法、各种戒色体系的优点,也收录了我自己对戒色深刻的理解、认识和领悟,是一个立足于实战的戒色体系。我相信好的体系会事半功倍,会让人更快地把握戒色的原理和规律,更快地进入良好的戒色状态,也能够更好地激发大家的戒色动力和热情。可能有戒友会问体系为何以钻石来命名,因为钻石纯净、坚固、永恒,代表高品质,也象征着真我。

以下逐一解析戒色十规。

\subsubsection{远离黄源:戒色后一定要注意远离黄源,删除手机和电脑里的不良资源。}

真正做到手机和电脑里不存一张黄图、一个黄片,有的戒友戒色后还舍不得删除自己的不良资源,这怎么行呢?要有壮士断腕的魄力,黄图和黄片是害人的东西,一定要下狠心删除,删光删净,一个不留!不管你有多少 \unit{\giga\byte} 的黄片,几百 \unit{\giga\byte} 或者几个 \unit{\tera\byte},都要坚决删掉,拿出与邪淫生活彻底决裂的气势,删掉!删掉!!全部删掉!!!必须对自己狠一点!戒色的决心更坚决一些!我比较欣赏的四个字就是:死地则战!陷入死地没有退路了,就会进入破釜沉舟、以一敌十、拼死作战的状态。删光一切就是在破釜沉舟,拿出大干一场的架势,杀出重围,杀出一条血路,大劈大砍大杀,刀光闪耀,血肉横飞,杀声震天,英勇杀敌的英雄,顽强的斗志,血战的悲壮!“凡守城将士,必英勇杀敌,战端一开,即为死战之时!”如果有碟片或者黄书,也要立刻销毁,绝对不能留在身边。平时上网不看擦边图和擦边新闻,看到诱惑马上避开,尽量不要在床上玩手机,躺着容易放松警惕,很多人都是躺着玩手机然后一步步沦陷的。对于擦边图更要警惕,有的新人会觉得看看擦边图没事,其实擦边图的诱惑也非常厉害,也很容易让人陷进去,而擦边新闻则往往是新奇吸引人的标题,吸引你去点击,内容也容易让人产生邪念,所以要格外警惕。手机上网看小说或者看新闻,这个举动暗含很大的危险,因为网上到处都有黄雷,诱惑真的很多,弄不好就会中招!上网就像进入了雷区,进入了地雷阵,这时候必须高度警惕,诱惑的内容跳出来,不要去看,不要聚焦,马上关掉。网络是把双刃剑,以前看黄还要购买碟片,而现在不出家门半步就可以在网上看到大量的黄片,黄片根本不是什么福利,黄片就是害人的毒品,国家也在扫黄,黄赌毒,黄排在第一位,淫秽内容的腐蚀性比浓硫酸还强烈,对心灵的毒化和污染极其严重。网络也有好的一面,沟通的确方便了,我们也可以看到很多大德的开示,聆听到宝贵的教诲,在这个网络时代,升华和堕落的机会就摆在我们面前,就看自己如何把握了。

\subsubsection{学习戒色文章:每天坚持学习戒色文章,多做笔记,多复习笔记,提高吸收率。}

大家都是学生党过来的,应该知道学习是多么重要,老师也会划重点,也会要求记笔记。孔子韦编三绝,为读《周易》孔子把串连竹简的牛皮带子都给磨断了几次,读书勤奋度可见一斑。记得我以前屡戒屡败就是因为不懂得学习戒色文章,只是自己一味靠毅力强戒,每次都撑不过一个月,后来我开始学习戒色文章,学习大德开示,这样才有了真正的突破和飞越。学习的重要性不言而喻,学习可以纠正思想误区,可以显著提高戒色觉悟,所谓戒色觉悟就是对戒色的理解和认识程度,一定要先搞懂戒色的原理和规律,这样才能戒得专业和稳定。真正读懂戒色文章非常重要,有的人走马观花,自认为懂了,其实是假懂或者懂得很浅,没有深入理解戒色文章的精髓,根据我的体会,戒色文章要反复看,戒色笔记也要多记多复习,这样吸收率才能上去,看一遍和看十遍的感觉是完全不同的,有的笔记我都看了几十遍乃至上百遍,依然会有新的感悟和收获。

有时看到一条笔记会非常兴奋,因为顿悟了,以前看那条笔记没多大感觉,随着阅历和实战体会的加深,就会突然对某条笔记产生格外强烈的感觉,一下就悟进去了,自己也能举一反三,融会贯通,非常兴奋非常狂喜非常振奋的体验。有时悟懂一个道理可以兴奋一整天,相信不少资深戒友都有过顿悟的体验。戒色一定要开窍,不开窍,不明理,怎么也进不去,能开窍,能明理,很快就能契入。每个人的悟性是不同的,有的戒友很有天赋,也很勤奋,注重学习和积累,他们的觉悟提升非常快,就像坐电梯一样。有的戒友可能比较迟钝,但应该相信勤能补拙,只要注重积累,坚持下去,迟早会迎来顿悟的。有段时间我每天都有顿悟,一连串的顿悟,就像一个个烟花在头脑里爆炸出绚烂的火花,积累到一定程度就会出现顿悟的爆发,悟道是那么兴奋和快乐,有时懂一个道理真的比赚一百万还高兴,那个道理不是普通的道理,是可以主宰内心、降伏心魔、改变命运的道理,这个道理是无价的。学习戒色文章尽量不要中断,要养成良好的学习习惯,坚持下去,从学习中找到乐趣,这点很关键,如果很忙没时间看戒色文章,那也应该挤时间听听戒色录音或者看看戒色笔记,这对于保持良好的戒色状态是非常重要的。

李昌镐八岁那年,在全州和曹薰铉下了两盘让三子的指导棋,曹薰铉败了。这时曹薰铉就意识到这个孩子是一个天才。从 1984 年到 1992 年,共有八年,在曹家的那些日子,李昌镐每一晚都在认真读棋谱,读老师家中的藏书,电灯夜夜都在子夜才熄灭。曹薰铉家中的数千册棋书,李昌镐读了三遍。有人笑着对曹薰铉说:“你家里有一个贼,每天都在偷你的东西。”曹一笑了之。李昌镐的用功是出名的,李昌镐的成功就是靠他超常的用功,他的学习能力超强。要在一个领域成为顶尖的人物,必定要付出大量的努力和巨大的投入,那些奥运冠军都经过艰辛刻苦的长年的系统训练。要进入稳定的戒色层次,坚持学习戒色文章是必须的,不学习,觉悟就无法提升,就不会有更深入、更透彻的理解和认识。一位戒友说:“好多人当时是懂了的,但一做题目马上就不会,就像对境马上投降,必须自己经过消化与练习,知识才会是自己的,而通过自己写出知识点,一遍又一遍利用这个知识去实践,多重复学习,并写出自己的感悟,才会把知识真正内化为自己的东西。”学习戒色文章要提高吸收率,一遍遍看,一遍遍复习,再写写自己的心得、总结和归纳,这样就可以把知识点真正内化成自己的实战意识,在实战中真正体现出来。

\subsubsection{练习断念:建议每天背诵断念口诀五百遍以上(默念即可),一定要正确理解口诀的含义,严格按照口诀的意思去做,有佛缘的戒友也可以念佛持咒。}

断念是戒色实战的核心,观心断念真的太重要了,怎么强调都不过分,强调多少遍都不嫌多。实战强,才是真的强,要做实战派戒者,不能做理论的巨人,实战的矮子!空谈派我见多了,夸夸其谈,谈理论一套一套的,实战时稀烂,心魔一来,一触即溃,还是菜鸟的水平。断念到底有多重要?\textit{有人患淫不止,欲自断阴。佛谓之曰:若断其阴,不如断心。心如功曹,功曹若止,从者都息。邪心不止,断阴何益?(《四十二章经》)} 古人戒色和现代高度相似,戒色吧不少新人嚷着要剁屌、剁手的,发毒誓谁都会,但是真的有效果吗?那个发毒誓的人也许没过几天就把毒誓抛之脑后,又开始撸起来了,心魔附体后,什么都不管不顾了,什么危害、什么道理都抛之脑后,只知道撸出来,像着了魔一样。佛陀说:“若断其阴,不如断心。”把阴茎断了,不如把心断了。心就是心念、念头。《四十二章经》这句话可谓一语中的,把戒色最核心的内容揭示出来了,戒色要注重断念,不是靠行善,不是做俯卧撑,而是直接对治邪念,行善可以作为辅助,但最关键的还是要学会直接断念。不断念,就会被附体,念头在头脑里安装运行后,你就不是你了,你已经沦为了疯狂的撸管肉机!\textit{才有一念萌动,即与克去。斩钉截铁,不可姑容,与他方便。不可窝藏,不可放他出路,方是真实用功。(王阳明《传习录》)} 断念要斩钉截铁,不可贪恋和犹豫,要有一股狠劲,坚决果断,断念要快、要狠!\textit{念头起处,才觉向欲路上去,便挽从理路上来。一起便觉,一觉便转,此是转祸为福、起死回生的关头,切莫当面错过。(《菜根谭》)} 我们要学会提升自己的觉察力,觉察力就是极具威力的武器,念头一上来,一记觉察就消灭了。\textit{大抵最上者治心,当下清净;才动即觉,觉之即无。(《了凡四训》)} 这一觉,就像快刀出鞘,瞬间斩断邪念,不怕念起,就怕觉迟,一定要快觉!丁愚仁老师:“当念头起时,及时地观照这个念头,念头就会化为乌有。”观照即觉察,念头怪无法承受你的觉察力,你的觉察力就像激光炮一样,可以在刹那间摧毁入侵的念头怪。圣贤教育极其注重和强调断念实战,所谓“克念作圣”,一定要战胜邪念!真正学会观心断念了,也就真正把握了戒色实战的核心,不断强化自己的觉察力就可以立于不败之地!真正强大、真正有力量的东西就是你的觉察力!!!就是那个觉!就是那个知!那个看!这个世界最奇怪的事情就是当你去看念头时,念头突然消失了,怎么会这样?!这个消失的效果太令人震惊了,太不可思议了!就像强烈的阳光可以让水滴消失,同样地,你的觉察之光也可以让念头瞬间消失!这就是原理,这是最伟大的发现,这个发现比所有诺贝尔奖的发现还要重要无数倍,这是最顶级的发现。这个世界上最伟大的奇迹,最奇妙的事情,就是当你看向念头时,念头突然消失得无影无踪,那个看威力十足,那个看就是死亡之瞪、杀灭之瞳!念头怪经不起这一看,它根本承受不住这一看!前提是你的觉察力要足够强,强到一看念头,念头就消失!

反复练习断念就是为了实战的那一下,那一下必须狠,必须强硬,必须果断,否则就会被心魔附体。之前看过《远邪十法》,第一条就是:清心地——于平时防犯淫念的生起,一起念即予觉断。觉断这个词很好,一觉就断,问题就是你不觉,或者后知后觉,练习就是要做到早觉,一起念就觉断。国外一位戒友说:“其中最难的一个是停止所有肮脏的念头,不要放纵它们。你无法控制它们出现,但你可以在它们出现后控制其发展。不要压制念头,这会反向加强它们。”国外有些资深戒友也认识到了断念的重要性,所以他们也能戒得很好,我们无法控制念头出现,因为念头会自动冒出自动入侵,但我们可以控制其发展,不让其发展壮大。有的戒友会误解断念为压念,断念不是刻意压制念头,那样会适得其反,越压越起,断念是觉而化之,是化解念头,一觉就化除了。断念需要找对感觉,悟性高的人很快就能找对感觉,悟性差的人可能几年都找不对感觉,所以开窍非常重要,一开窍就懂了,一旦找对感觉就会觉得断念其实很容易,就那么一觉、一知、一看,邪念就会消失无踪,觉察力就是核武器!!!每个人都有这件最厉害的武器,可以说是顶级装备,但很多人却不知道怎么用它,一旦学会了,就会所向披靡,战无不胜!断念最惊人的面向之一就是它惊人的容易,不可思议的容易,比吃饭喝水还要容易,吃饭要拿筷子,喝水要拿杯子,而断念只需一觉、一知、一看,念头就消失了,真的是会者不难,一旦找对感觉,就会感叹实在太容易了,过去怎么不知道呢?!惊人的容易,惊人的简单,不去压念,只是一觉,就完事了,念头来源于空性,一觉察就消融回空性,就这么奇妙,就这么简单,实战就在零点几秒的刹那。当第一次找对感觉,甚至会很兴奋,因为突然懂了,突然会了,绝对是戏剧性的开悟,之后只需继续强化那个实战的感觉,就能越来越强!这是让念头消失的实战技法或艺术,绝对的amazing!外国人经常说这个词,我也很喜欢这个词的发音和说这个词时的语气,意思是:令人惊异的;(尤指)令人惊喜(或惊羡、惊叹)的。It's absolutely amazing!(这太不可思议了!)谁会想到这么容易?就这么一看,一切都解决了。关键是会看,找对那个感觉,强化那个看,你就是无敌的!念头怕你的觉察力,觉察力越强,它越怕!它根本无法承受强大的觉察之光,就像激光一样把它消灭!实战的神操作就是一记觉察!一记觉察消灭一切!!!这是顶级高手的操作,尽享杀念的快感!杀念就是爽!杀念的快感胜过手淫千百倍!!!弄懂原理,找对感觉,强化练习,就能精于此道,决胜实战!从初学乍练、登堂入室、渐入佳境、出神入化到登峰造极,从刚开始断念勉勉强强到后来压倒性的胜利,这需要一个持续进步的过程,需要很高的悟性,也需要不断地努力、勤奋和精进。

一位戒友在帖子里说:“几个月之前我在想飞翔为什么十几年的邪淫历史能一次性成功戒除,他是人我也是人,我为什么做不到,后来我又反省,飞翔虽然说写了这么多前辈经验,但是他一直强调断念实战,也一直强调修心,我当时天天反复看《戒为良药》就是戒不掉,后来我心里一狠,我也天天背诵断念口诀,自从背诵之后到现在我没有破过,我现在每天都坚持,一天都不能放松。”这位戒友之前也反复学习戒色文章,但是戒不掉,后来他终于开窍了,知道练习断念口诀了,刚开始要熟背断念口诀,到后来只需按照口诀的意思去做即可,直接观心断念,觉察消灭念头。就像刚开始我们会熟背乘法口诀,到后来就不用背诵了,只需按照口诀去做即可。我们一定要注重断念实战,如果我没有把握核心,我肯定戒不掉的,我之所以能戒到现在还未破,就是因为把握了戒色实战的核心——观心断念!我在文章里强调了无数遍,每一遍我都试图组织新的语言、新的比喻、新的角度、更集中的总结,为的就是唤醒大家对断念实战的重视,以及让大家对断念实战有更深入、更全面、更细腻的理解、体会和认识。

\begin{case}[练习断念]
    飞翔哥,我发现我对念头的掌控越来越强,现在每天我都活得很快乐了,人际关系大大改善,和以前简直天堂比地狱。走路敢抬头了,别人吓唬我我都没反应了,甚至会劝别人向善,这在以前根本想都不敢想(以前在别人面前跟个孙子似的就怕别人吓唬我),和漂亮的女的都敢对视了,清楚地看清自己的每一个念头,有时候我一个不经意的善意的举止都会让身边的同学脸上充满喜悦,周围的人和我在一起都面带笑容,很开心的样子,总之戒色的好处就不一一说了。发誓一辈子再也不犯邪淫,在这里要对你表示深深的谢意,要不是你的“修心就是修念头”这几个字,恐怕我现在早就走极端了。

    \textbf{分析} 主宰内心,主宰念头,获得对内心的强大统治力!对念头的掌控越来越强!不再是那个被心魔随便虐的菜鸟了!有些戒友领悟力很强,进步神速,实战的关键要领一点就透,真正领悟后,只需不断强化练习即可,强化那个觉、那个知、那个看,不断强化!强化!再强化!极致强化!其他一些戒友则需要相当的学习积累才能顿悟实战的精髓,虽然领悟较晚,但只要真正领悟了,其实也是一样的,就像有的人学车快,有的人学车慢,等到学会了其实都一样,所谓熟能生巧,熟极自神,真正熟练后都能做到一觉即断!犀利无比!这位戒友也提到了走路敢抬头了,不怕别人吓唬了,也敢对视了,中医讲肾主恐,肾气不足就容易害怕、恐惧,戒色后肾气养足了就是一派气壮山河的气象,有自信了、有底气了,不怕了,人际交往也如鱼得水了,别人能感觉到你散发出来的能量,戒色修善的人能量很高,可以给别人带去正能量,带去喜悦,带去美好的体验。一个戒色的人可以影响周围的人,一个邪淫的人也会影响周围的人,我们选择戒色不仅是为了自己,也可以为别人带去积极正面的影响。这位戒友顿悟了“修心就是修念头”,很多人对修心的概念很模糊,其实心就是念头啊!这是一个关键的领悟,戒色就是要在念头上下功夫,俞净意公之前虽然行善,但不懂得修心,命运依然凄惨,后来他得到点化,终于知道要断除邪念,这样命运才逆袭了。我们一定要把握核心,核心不是行善,而是修心,很多人虽然行善,但内心还是不干净,行善固然重要,但更重要的是修心,把握这个核心,你就会越戒越好,偏离这个核心,虽然也能戒一段时间,但终究会破戒的。
\end{case}

\begin{case}[练习断念]
    我感觉最难对付的就是图像,如果不狠断,一旦那种图像霸占了你的头脑,极易破戒。我也意识到自己的问题了,识别能力差,觉察的还不够快,断力也不行,念头来了不断而是跟随强化,还在犹豫贪恋。

    \textbf{分析} 这个案例很典型,说了几个常见的问题:识别差、断念慢、跟念头、犹豫贪恋,都是比较典型的问题,只有克服了这几个问题,才能在实战中取胜。断念是该狠一点,要像惊天排球大血帽,让心魔看看你的强硬,你的立场!告诉心魔这是你的地盘,这是禁飞区!戒色更需要气势,就像一条龙暴扣燃爆全场,彻底摧毁心魔的不良企图!上演荡气回肠史诗级的绝杀,球进哨响,万人沸腾!拿出强悍无比的气势来戒色!这是你死我活的战斗!不少人戒得太软,戒得死气沉沉,缺少一股冲劲和狠劲,缺少血战到底的烈士气概。要让心魔知道你在防守端的存在感真的是无敌强,念来即扇,念来即盖!无一例外!劈头盖脸把心魔彻底盖趴在地上,一双眼睛死死瞪住心魔,让它瞧瞧你的强悍,你的强硬!你的血性!你的不屈!杀死比赛!杀进总决赛!形成对心魔的强大威慑力!做断念的狠角色!让心魔闻风丧胆!让心魔知道,只要敢动,必将被 KO!一记觉察犹如一记后手重拳直接把心魔打成躺尸!彻底统治内心的八角笼!图像袭脑极快,瞬间就霸占了你的头脑,然后一帧一帧演变成小黄片在你头脑里播放,我们必须狠断!反应要快,切忌犹豫贪恋,当断不断反受其害!我们要做断念的大内高手,修炼成断念九段!九段制是围棋的划分,也可以用在断念实战能力的划分。对于各种会导致破戒的念头也要心知肚明,要增强识别能力,识别差也是一个很大的问题。
\end{case}

\subsubsection{做好慎独:独处时很容易破戒,戒色后一定要做好慎独,保持警惕,防意如城。}

慎独,语出《中庸》:“\textit{莫见乎隐,莫显乎微,故君子慎其独也。}”当独自一人而无别人监督时,也要做到表里一致,不做见不得人的事情。慎独,其实就是“慎心”,要看住自己的念头,独处时心魔容易入侵,很容易滋生各种邪念,独处时又缺少潜在的监督,为放纵提供了条件。古人讲君子慎独,不欺暗室。\textit{诚于中,形于外,故君子必慎其独也。(《大学》)} \textit{慎独则心安。(曾国藩)} 古人智慧很高,知道慎独的重要性,一个人的修养不仅仅是在人前的表现,关键是独处时是不是也能把持住自己,这点极为重要。慎独讲究个人道德水平的修养,看重个人品行的操守,\textit{慎其家居之所为。(东汉郑玄注《中庸》“慎独”)} 一般理解为在独处无人注意时,自己的行为也要谨慎不苟,慎独是一种自律,慎独是一种坦荡,慎独是一种修为境界。在独自无人监督的情况下,凭着高度自觉,按照一定的道德规范行动,而不做任何有违道德信念之事。很多人表面好像是正人君子,背地里却一直在沉迷看黄手淫,完全表里不一,这就是伪君子,真正的正人君子必须是光明磊落,表里如一。大家可以回忆下自己的破戒过程,破戒时基本都是在独处时发生的,很少会出现几个人一起手淫,基本都是一个人偷偷手淫,这是一个非常隐秘的恶习,如果你突然撞见一个人在手淫,对视的刹那你会深深感到人性的复杂,那个眼神里有太多复杂的内容。在独处时要格外保持警惕,独处意味着潜在监督的缺失,警惕意识一定要提起来。

\begin{case}[做好慎独]
    飞翔老师好!我踏上戒色道路也有接近两个月了,我是一名学生,一到周末就容易起念,从而破戒,我该怎么克服这种怪圈呢?

    \textbf{分析} \textit{欲戒淫行,必自戒淫念始。淫念起,则淫行随之矣。(《寿康宝鉴》)} 如果你能及时断除念头,那就不会发生破戒,一个人也许看了上千篇的戒色文章,但最后还是看实战的那一下。那一下的含金量极高,所有的功夫和实力就体现在那一下,那一下惊天地泣鬼神,那一下软弱,必定被心魔附体,\textit{欲火入心,犹如鬼著。(《大集经》)} 我做学生党时,周末也经常破戒,平时要学习,注意力主要在学习上,而到了周末有大量独处的时间,一个人也不可能总是在忙,总有休息独处的时刻,独处时心魔很容易入侵,会出现想看黄、想撸的微妙感觉和想法,如果不懂得断念或者断念不力,那就很容易破戒。周末破戒是非常多见的,一到周末很多人都会破戒,周末就像一道关口,很多人都倒在了这道关口。要克服这种怪圈,必须具备强大的觉悟和强悍的断念能力,念头一起就断掉,而且要断得够狠、够快,形成强大的威慑力,这样心魔就不敢轻举妄动,如果你断得很勉强,那心魔就看到了攻破你的希望,如果你是压倒性的胜利,那心魔就会自知难以得逞,进攻频率会大幅减少。最近有戒友向我反馈,说虽然断念了,但是念头一会又起来了,好像断不完一样,一波接着一波,就像守着一个山头,下面的敌人一波波地往上攻。如果你足够强硬、强悍,火力足够猛烈,敌人就不敢轻举妄动,因为动就等于送死!心魔也是如此,你必须足够强大,强大到心魔怕你,强大到心魔不敢轻举妄动,动就杀!动就给它来一下狠的!让它尝尝厉害!
\end{case}

\begin{case}[做好慎独]
    身为一名学生党,深受其害,但还是到了周末忍不住看黄,我发现放假千万不要独处,太危险了,欲望随之萌生,而且断念一定要快,昨天再次破戒,我才意识到是断念太慢导致的,两边思想斗争最终失败,一定要断念快啊!

    \textbf{分析} 这也是一位学生党,同样是周末,学生党平时也可能破戒,但主要以周末破戒和假期破戒为主,这位戒友说:“放假千万不要独处”,独处的确很危险,他也意识到了。我们肯定会有独处的时候,关键是提升自己的实战能力,断念强大,就不怕独处,我现在也有很多独处的时候,但我不再破戒了,因为强大后就不怕独处了,反而会很享受独处的时刻。说到底,还是要强大,你强,心魔就弱,你弱,心魔就强,心魔本来很强大,当你强过心魔后,心魔就显得弱了。记得初中时打格斗游戏,第一次打 BOSS 感觉 BOSS 超强大,后来越打越有经验,这时候就有战胜 BOSS 的把握了,面对强攻也能从容应对了。强过心魔后,你的内心自然就会有一种必胜的把握,很多人都没有这种把握,因为他们还不够强大,学习和练习就是为了使自己强大起来,强大了才有主宰权!弱小注定被虐!
\end{case}

\begin{case}[做好慎独]
    我发现每到独处、时间充裕的时候就会面临一场恶战,念头一个接着一个冒出来,前面几波感觉断念可以,可是越到后面越来越难断,直到心魔形成燎原之势,身体的控制权被夺走,然后剩下满目疮痍!

    \textbf{分析} 这就是不够强大的表现,前几波攻击还能顶得住,后来就不行了,顶不住了,被攻陷了。如果足够强大,就可以直接把心魔干回老巢!拿火力来比喻,一支步枪难以抵挡成群的攻击,当你手里有火箭炮或者火神机枪,那局面就完全不同了,敌人看到你这样强,也不敢轻举妄动了,大杀器绝对令敌人胆寒,令敌人闻风丧胆!当大杀器出现在战场上,那绝对是敌人的噩梦,会产生极大的威慑力!当你火力超猛,威力十足,对心魔来说就是个噩耗,你的实力足以震慑心魔、碾压心魔,让心魔恐惧,让心魔颤抖,戒色要用实力说话!!!长板桥,别名又称当阳桥。说到长坂桥,会让人想到张飞喝退曹军追兵的典故,张飞怒目横矛,立马于桥上,大喝曰:“燕人张翼德在此,谁敢来决一死战。”声如巨雷,吓得曹军旋马而走,夏侯杰当场毙命,众将亦一起往西奔逃,弃枪丢盔者不计其数。有诗云:“长坂桥头杀气生,横枪立马眼圆睁。一声好似轰雷震,独退曹家百万兵。”长坂桥一战显示了张飞在战场上的勇猛霸气,有万夫不当之勇,“猛张飞”真是了得!《三国志》作者陈寿评:“关羽、张飞皆称万人之敌,为世虎臣,并有国士之风。”戒色就是一人与万念战,我们要做“万念敌”,拿出张飞般的勇猛气魄,拿出决一死战的架势,就是横!就是猛!就是不要命!拿出横扫千军的表现,只要你足够猛,根本就不怕念头上来,上来就是找死的!你的强悍勇猛会形成强大的威慑力,这种威慑力足以秒杀一切!战胜一切!降伏一切!正所谓:戒色战场猛张飞,一夫当关万念灭!
\end{case}

\subsubsection{视线管理:对境实战时要做好视线管理,看到诱惑马上避开,不聚焦、不停留、不回看。}

《论语》专门讲到了视线管理,\textit{子曰:“非礼勿视!”} 视线管理在对境实战时是重中之重,要做到目不邪视,所谓心正目正,心邪目邪,金刚正眼,不视邪秽!不管是上网还是生活中看见诱惑要马上避开,避色如避箭,人言是牡丹,佛说是花箭,射人入骨髓,死而不知怨。看到诱惑时,视线不聚焦、不停留、不回看,不试图看清,避免第二眼沦陷、第二眼着魔!对待诱惑要冷!好好体会冷的感觉,要冰冷,好像不感兴趣一样。视线管理是非常重要的,刚开始视线肯定容易粘上去,视线会贪婪地“抓”图片,想看清,想看仔细,特别是习惯性地聚焦敏感部位,这是“熟路”,何为熟路?就是你过去一直反复这样做,所以越做越熟,都快形成条件反射了,看见女人就盯着敏感部位看,非常贪婪。不管是网上还是生活中都有无数次的机会让你对境实战,对境是一大考验,真金不怕火炼,对境表现真的很能看出一个人戒色实战意识的高低。

有时我能感觉到诱惑图片的强大吸引力,余光我能感觉到有一股强大的拽力试图把我的视线拽过去,图片出现在电脑页面的左侧或者右侧,但我不去聚焦,马上转移视线,如果图片出现在电脑正中央,我会转动鼠标快速下拉页面或者关闭页面,总之我不会停留在图片上,停留是很容易陷进去的。过去的第一反应肯定是停留,想看清,那种看的欲望很强,后来有了戒色实战意识,就知道避开了。所有的撸者都在停留,都在聚焦,都在盯着看,而戒色高手正好相反。根据我的研究,视线管理是需要练习的,刚开始总是粘上去,总是停留,随着一次次在实战中磨练,视线的粘性开始下降,不再抓图片了,不再盯着敏感部位看了,这就是进步的表现。\textit{常与自己逆,便是进功。(元音老人)} 有的戒友对境时不够警惕,虽然知道不能去看,但还是忍不住会去看一两眼,这也是时常发生的事情,那个瞄一眼的速度非常快,所以要提高警惕,严防视线粘上去,严防停留。面对诱惑,第一反应要从盯着看变成避开,不要飞蛾扑火,要学会避开色弹!这是一个色弹横飞的世界,一不小心就阵亡了。小时候打战机类游戏,里面一个很重要的技巧就是“躲子弹”,相信很多戒友都深有体会,能躲子弹就能幸存,躲子弹是一门很高深的技术,不仅要求反应要快,也需要丰富的实战经验,我们戒色也是如此,要学会“躲色弹”,戒色老兵能在对境时幸存,就是因为他能躲,他会躲,他有躲避的实战意识,而戒色新人往往缺少经验,看到色弹不知躲避,还迎着上,结果直接就被干掉了,新兵蛋子要多向老兵学习,看看老兵是怎么躲的。如果对境时做得不够好,自己要及时反省和总结,不断优化和提升自己的实战表现,在下一次实战时要尽量做好,这点很关键,不大可能一下就做得很好,肯定是有时做得好,有时做得不够好甚至很差,自己要善于反省,这样视线管理就能逐渐越做越好,进入比较稳定的状态。

\subsubsection{养生恢复:加强养生意识,严格控遗,适量锻炼,做有氧运动和养生功法。}

戒色是系统工程,养生恢复也是系统工程,两手抓,两手都要硬。戒色后肯定会面临一个恢复的问题,年纪轻恢复快,伤精程度浅的人也恢复快,懂得养生的人恢复也快。戒色后要严格控遗,然后要加强养生恢复,养生是一门很深的学问,要注意很多细节,大家戒色后一定要多学习养生的文章,努力提升自己养生的意识,这样对于恢复是很有利的,可以大大促进身体的恢复,否则一边戒色,一边做着伤身的事情,这样就会减慢恢复的速度,甚至感觉恢复很不理想。症状严重的话应该配合积极治疗,然后三分治疗,七分戒色养生,作息饮食要规律,尽量不要熬夜久坐,平时适量锻炼,可以做做有氧运动,比如慢跑、快走、球类、骑行等运动,简单来说,有氧运动是指强度低且富韵律性的运动,其运动时间约三十分钟或以上。有氧运动可以缓解压力,可以调节心情,化解不良情绪,具有宣泄功能,可以释放心里的压抑,忘却烦恼,同时也可以带来身心上的愉悦。有氧运动还有减脂、预防疾病、增加心肺功能等作用,好处很多,但也要注意适量锻炼,不可运动过度,否则对身体也会带来一定的伤害,身体虚弱的话可以从散步开始,然后再快走、慢跑等,一定要注意控制出汗量,最好控制在微汗的程度,不要每次都大汗,中医讲大汗伤阳,经常大汗对身体恢复不利,也容易进入运动疲劳期,不利于坚持。身体好的话也可以适当做做力量训练,但训练后一定要加强修心,因为力量训练会导致雄性激素分泌增加,欲念很可能会随之增多,所以要加强警惕。身体差的话不建议力量训练,\textit{肾主精与骨,用力举重则伤肾。(《类经·卷十三》)} 劳力过度可损伤肾气,很多人不明白养生的原理,想通过力量训练来痊愈,结果身体越练越差,懂得养生之理就知道身体差时应该静养,避免用力举重,等身体恢复很好了,才可以适当做一下力量训练,这点认识很重要,很多人就是不懂这个理,才吃了这个亏。

有压力可以通过运动来排解,不能用撸管来发泄,撸管虽然可以短暂地缓解压力、忘却烦恼,但会让你陷入恶性循环,将来会面临更大的压力和烦恼。在促进恢复方面,我们也可以做做养生功法,比如养生桩、静坐、八段锦、五禽戏、六字诀等,养生桩或者静坐应该修炼一项,让自己入静可以接受宇宙能量,大家可以看看《灵性的实相》这个视频,我看了很多遍,依然觉得很好,很受启发。里面讲到:“你所有的念头都止息了,我们称为无念的状态,这是一个静心的状态。在这个状态下,我们开始接收源源不绝的宇宙能量,坐的时间越久,我们接受的宇宙能量就越多。”“静坐的时候,我们可以获得充沛的宇宙能量,宇宙能量在能量体内的所有能量管内流通,宇宙能量强大的流势可以清洗所有的以太破洞,以太破洞清洗干净以后,我们就摆脱了疾病。”养生桩和静坐都非常好,我都尝试过,现在以静坐为主,静坐一会就感觉被充电了一会,感觉神清气爽,精力倍增,静坐时也可以练习观心断念,只有观心断念的能力强了,才能更好地入静,否则还是妄念纷飞,静不下来。曾国藩的日课就有静坐:每日不拘何时,静坐四刻,正位凝命,如鼎之镇。静坐不仅是一种修行,也是一种养生,更是一种修养,成大事者有静气,这种静气就养之在平时。

戒色后在饮食方面也要格外注意,平时以素食为主,肉少吃,条件允许的话就吃全素,吃素也要注意营养均衡和全面,我现在吃素感觉身体挺好的,刚开始有点不适应,后来慢慢就适应了。吃素可以减少欲念,肉吃多了,那种念头就会增多,而且容易生气。酒尽量少喝或者别喝,最好不要抽烟,烟酒色这三样太毁男人了,小时候我们这三样都不沾,长大了很多人这三样都沾上了,如果有应酬需要,酒还是尽量少喝,喝酒乱性很可怕,喝酒容易出问题,所以要格外警惕小心,酒后嫖娼更可怕,一定要谨慎。一位戒友说:“葱姜蒜真的不能吃,刚才吃了点,欲望马上就上来了,好险啊!”戒色后葱姜蒜韭菜辣椒要少吃或者不吃,以前有次我吃了韭菜,没过多久那种念头就上来了,而且上来得很猛,好像打了兴奋剂一样,如果不是及时断掉,就很可能被攻陷了。吃了葱姜蒜,晚上也容易遗精,所以最好别吃。

饮食是非常关键的,很多人都没有意识到饮食对他们的身心灵会有多大的影响,我们吃的每种食物都有一个特有的振动力,我们要吃高频的食物。

法国的电磁学专家 André Simoneton 在 20 世纪三四十年代做了一个测试特定食物电磁波的实验。他发现,为了保持身体健康,人们必须保持 6500 \unit{\angstrom}(\unit{\angstrom} 是光学计量单位)的振频。

\begin{enumerate}
    \item 第一类为蔬菜和水果,这个类别有着最高的振频,约 6500 - 10000 \unit{\angstrom}。其中包括:鲜果,蔬菜,橄榄油,坚果,葵花籽,椰子,大豆,花生,榛子和全谷物。
    \item 第二类中,振频在 3000 - 6500 \unit{\angstrom}。包括熟菜,牛奶,黄油,鸡蛋,蜂蜜,熟鱼,花生油和葡萄酒。
    \item 第三类中,振频非常弱,低于 3000 \unit{\angstrom}。包括熟肉,香肠,咖啡,果酱,乳制奶酪和精面面包。
\end{enumerate}

参见 Simoneton 的实验结论,可以看出食用了低于 6500 \unit{\angstrom} 的食物将会阻碍甚至削弱我们的振动力和健康。因此,建议多吃高振动力的食物,多吃蔬菜水果,少食肉类。

\subsubsection{情绪管理:不良情绪或者狂欢情绪容易导致破戒,所以要做好情绪管理,保持心平气和。}

情商比智商更重要,情商(Emotional Quotient)通常是指情绪商数,简称 EQ,主要是指人在情绪、意志、耐受挫折等方面的品质,总的来讲,人与人之间的情商并无明显的先天差别,更多与后天的培养息息相关。提高情商是提高控制情绪的能力,从而增强自我管理、理解他人及与他人相处的能力。哈佛格言:一个人在社会上最后占据什么地位,绝大部分取决于非智力因素。很多成功人士他们的智商真的不高,甚至有的只有初中文化,但他们的情商奇高,很会做人,很会管理自己的情绪,那些智商高的人才反而成了他们的手下员工。单有智商而缺乏情商的人,往往不能很好地控制自己的情绪,一个人如果无法很好地控制自己的情绪,那么不仅伤害自己,也会把不良情绪发泄给别人,把别人当出气筒,这样就会导致人际关系紧张,不得人心。情绪管理对于一个人的生活、工作有着重大意义,不良情绪的危害真的很大,不少戒友都有情绪破戒的现象,比如考试失败、恋爱失败、找工作失败、被老板骂、被老师骂、被家人骂、生活中遇见各种挫折等等,这时候一定要及时调整自己的情绪,要学会看开放下,对于别人的错误也要学会宽容和原谅,上次一位戒友发帖嚷着说要报仇,他有这种心态就无法戒色成功,戒色要保持心平气和,不怨天尤人,不生气,知足常乐养喜气,懂得宽恕别人,这样心态平稳了,才能戒得稳定。反之,如果心态不稳,负面心态很重,那就很容易导致破戒,遇见挫折和逆境能够及时调整情绪,保持心胸开阔,这点是成功戒者必备的心理素质。有位戒友说:“遇到不顺的事情就会自动出现撸管的想法。”生活中肯定会有各种挫折和不顺,我们要坦然面对,勇于克服,不能拿撸管来发泄或者逃避,很多人撸管都是在逃避现实,不想去面对现实。上次一位戒友也提及了狂欢破戒的情况,遇见特别高兴的事情,就拿撸管来庆祝了,这也要值得警惕,狂欢容易滋生放纵的心理。情商高的人会激励自己,也会激励别人,在遭遇挫折、陷入低潮的时候,他会提醒自己要勇于面对,相信困难和挫折只是暂时的考验。

\begin{case}[情绪管理]
    我一直为情绪破戒而烦恼,我的理解是,不管是谁,肯定会有情绪不好的时候,我最近一直在挑战难度很高的事情,导致情绪起伏很大。但就算我在情绪出现波动的初期就立刻转化,可是我还是必须要去想,去解决这个问题呀,我还是会情绪波动呀,最后还是照样破戒,而且很多大事也不可能说开心就立刻开心呀,我有做情绪管理的笔记,但觉得还是离掌握差十万八千里。

    \textbf{分析} 挑战高难度的事情,很可能会有屡次失败的经历,也许要失败很多次才能成功一次。关键还是情绪调整能力,要知道失败只是暂时的,很多发明的产品在批量生产前都经历过多次的失败,工程师一直在不断完善和改进,最后出来的版本就是比较成功的一个版本。上次看电视,说研制的新一代动车要达到某个高标准,结果实验失败了,工程师马上研究失败的原因,看得出他们脸上不大高兴,毕竟失败了,但他们还是在继续完善和改进,这样最后成功的喜悦也是格外振奋的。挑战难度高的,最后成功时成就感也很大,但也要面对容易失败的局面,一开始自己就要摆正心态,就算失败也要坦然面对,不要有太大的情绪波动,更不能把不良情绪用撸管来发泄,调整情绪不一定要立刻开心,关键是坦然,自己能够接受失败和挫折,能够以一个比较理性的态度来面对。
\end{case}

\begin{case}[情绪管理]
    此刻在写这段文字的同时,心情是极其复杂的,戒了七个月,就这样宣告失败,这段时间情绪上的不稳定,从而导致戒色状态的不稳,一旦打破了那种良好的戒色状态,再难回归平静,一旦跟随念头,之后的一个月甚至几个月都难以调整过来,唉!破戒后今天一天情绪都很低落,脑子再次处于一种浑浑噩噩的状态中,提不起精神,很后悔很后悔,还是自身落实不到位,放松了警惕,才导致现在的局面,在此奉劝那些戒了半年乃至一年以上的戒友,不要放松警惕,戒色每天都是在走钢丝,一旦松懈就会坠入万丈深渊。

    \textbf{分析} 这位戒友的文字可谓发自肺腑,是实战后的深刻感悟。戒了七个月不容易,一旦情绪不稳定,人就容易出现破戒,情绪不稳定是很可怕的,因为随时会出现不好的事情,所以稳住情绪、稳住心态是非常重要和关键的。比如运动员有时情绪过于激动导致发挥失常,教练会马上提醒他调整情绪,不要激动,放平心态。围棋中提到了“平常心”,对于胜负得失的淡然。古力:“平常心可能大家会以为是消极的态度。但我认为恰恰相反,平常心是一种非常积极的态度,它是让我们去控制内心非常功利浮躁的心态,让我们以平和的心态去面临即将到来的事物和局面。”古力说得很好,心态平稳了才能发挥出最佳的水平。心态一浮躁,情绪不稳定,这样内心动荡了,就容易出现对自己不利的局面。境随心转,心态是至关重要的,\textit{宠辱不惊,看庭前花开花落;去留无意,望天上云卷云舒。……风来疏竹,风过而竹不留声;雁渡寒潭,雁去而潭不留影。故君子事来而心始现,事去而心随空。(《菜根谭》)} \textit{养得胸中一种恬静。(曾国藩)} 古人修养真的很高,内心处于非常祥和平静的状态,这样才能达到非常高的境界。我之前一直强调情绪管理,因为实在太重要了,情绪不稳定会导致戒色状态不佳,从而影响观心断念,影响警惕性,是一环扣一环的,如果你情绪很稳定,这样戒色状态也会变得很稳定。阿法狗震惊了世界,两位世界顶尖的围棋高手除了技术层面以外,也输在了情绪的控制上。阿法狗没有情绪,而人类情绪波动很大,这也许是人类的一个弱点,如果懂得及时调整情绪,让内心恢复平稳,这样才能发挥出自己最佳的水平。戒色一定要注重情绪管理,情绪的稳定不仅对养生恢复有利,也可以让你处于比较好的戒色状态。
\end{case}

\subsubsection{改过迁善:要勇于改正自己的不良习气,要克服一切负面的心态,多行善积德。}

\textit{君子以见善则迁,有过则改。(《周易·益》)} 戒色就是在改过,改掉恶习,同时要修善,去掉负能量,增加正能量,有了正能量心态和处境就会有很大的变化,负能量缠身处处都容易不顺和倒霉,负能量重的人脾气也很暴躁,人际关系也很紧张。改过的范畴是很大的,任何负面的心态和行为都要改正,比如嗔恨心、贪心、嫉妒心、报复心、傲慢自负、盛气凌人、炫耀自夸、偏执偏激、怨天尤人、损人利己、自私自利、爱慕虚荣等。我们要多孝顺父母,多帮助别人,要有“积善意识”,这个意识极端重要,记得我在戒色前每天过得浑浑噩噩,混吃等死,觉得人生很没意义,只想着看黄纵欲,满足自己日渐变态的欲望,其他什么都不想管,那是一种非常自私阴暗的内心状态。后来我学习了圣贤教育,开始懂得行善、积善的重要性,积善之家,必有余庆,积不善之家,必有余殃。古人的积善意识之强,真的是登峰造极!勿以善小而不为,勿以恶小而为之,\textit{莫轻小善,以为无福。水滴虽微,渐盈大器,小善不积,无以成圣。莫轻小恶,以为无罪,小恶所积,足以灭身。(《法句经》)} 每一个善举都是一个发光的金子,一定要懂得积善!虽然戒色的核心是修心,但行善也是非常之重要,行善也有助于修心和稳定戒色状态。给戒友答疑或宣传戒色就是在积善,戒色前辈都懂得积善,很多前辈都推荐日行一善。行善的方式有很多种,百善孝为先,先把父母孝顺好,有的戒友也在捡垃圾、让座、扶共享单车、捐款、捐衣物和放生等,参加各种爱心公益活动,做志愿者,做义工,劝人向善、劝人戒邪淫等等,自己应该要具备坚持行善和积累正能量的意识。过去邪淫就是在积累负能量,负能量必将导致痛苦,几秒的快感换来无限的痛苦和悔恨,不仅毁掉身体健康,还漏掉自己的福报,而行善是在积累正能量,积累福报,\textit{做一个有福人。(秦东魁老师)} 邪淫是大漏洞,这个漏洞一定要堵上,然后要力行善事,广积阴德,凡为善而人知之,则为阳善;为善而人不知,则为阴德。要做到心善、语善、视善、行善,身口意三门皆善,止于至善。袁了凡先生发誓要行善事三千条,以报天地祖宗之德!我们要学习袁了凡先生的弘愿和发心,真正拿出大决心来改过迁善。戒色的同时也不可沉迷游戏,网游最好不要玩,益智类或者运动类的游戏也不可沉迷,平时不要赖床,尽量早起,每天反省、总结,不足之处应及时改正和调整。

\subsubsection{培养德行:首重“谦德”,常修八德:孝、悌、忠、信、礼、义、廉、耻,多发感恩心、谦卑心、恭敬心、惭愧心、忏悔心。}

小戒靠忍,大戒靠悟,至戒靠德,德不配位,必有灾殃!戒色要懂得培养和提升自己的德行。我国民间有云:“道德传家,十代以上;耕读传家次之,诗书传家又次之;富贵传家,不过三代。”不少人可能比较在意颜值,我以前也是,太看重颜值而忽视德行才是根本,我后来领悟到一个人的颜值迟早会随着岁月的流逝而下降,这是必然的,再漂亮的人也会变老,而一个人的德行修养才是历久弥香的。以德立身、以德立威、以德立人、以德服众、以德报怨、以德为本、以德为先、以德治国、以德兴国,最后来一个“以德戒色”!观人以德,而不是以貌取人。小戒,戒个十几天或者一个月左右,不学习戒色文章,一味靠毅力强忍也能做到;大戒,戒个一年或者几年,这需要坚持学习戒色文章,领悟戒色的道理;至戒,戒个十年以上或者终身修为,那就要进入培养和提升德行的高层次了,德行不够,肯定会败下阵来,就这么奇怪,不管你能力有多强,只要你一骄傲,一自负,就容易被攻破!培养德行首重“谦德”,谦乃保身第一法,是非常重要的德行,我们要常修谦德,谦者,众善之基;傲者,众恶之魁。戒色一定要注重修谦德,不可骄傲自满,所谓骄兵必败。如果你想进入最高层次,那就必须要注重德行的培养。传统武术的文章也特别强调武德,武德不行,功夫就练不上去,练到一定程度就会进入瓶颈,因为德行跟不上,德行不行就会限制住自己,更高层次的道理就无法领悟。我自己的体会和理解就是德行可以提升振动频率,振动频率上去了,自然和高层次的道理是匹配的,很容易就能理解。如果频率无法匹配,那就很难领悟那一层的道理,总是误解和曲解,难以真正悟进去。毛主席有一个化名叫“李德胜”,不仅是“得胜”的谐音,我觉得也有注重德行之意,不拿群众一针一线,有德行的部队才能打下天下。

积德、勤俭可以兴家旺家;败德、妄为必定败家亡家,邪淫就是在伤身败德,在肆意妄为,后果肯定是症状缠身,败家亡家。戒色后多发感恩心、谦卑心、恭敬心、惭愧心、忏悔心,这五心要多发,戒色先培德,戒色先做人,六守:守愚、守浅、守谦、守仁、守敬、守诚。守愚,不耍小聪明安分守己;守浅,不管学了多少,总觉得自己很浅薄,还需继续努力;守谦,保持谦虚,不起骄傲的念头;守仁,做一个有仁德的人,仁爱、仁慈、仁义,仁者无敌;守敬,培养恭敬心,一分恭敬,得一分利益;十分恭敬,得十分利益。诚恳恭敬,忠厚端肃,主敬则身强!懂得尊重人,绝不看不起人;守诚,\textit{诚者天之道也,诚之者人之道也。(《礼记·中庸》)} \textit{欲修其身者,先正其心;欲正其心者,先诚其意。(《大学》)} 立身处世,当以诚信为本,做人做事也要有诚意,是一种真实不欺的美德。\textit{有一秘诀,剀切相告,竭诚尽敬,妙妙妙妙。(印光大师)} 至诚感通,真诚恭敬到极处,自然就感通。王阳明先生原名王守仁,别号阳明,突出一个仁字。丁愚仁老师,有愚和仁两字,大智若愚,返璞归真,纯真质朴,平易近人,亲和的感觉。这“五心六守”希望广大戒友好好践行,绝对受益匪浅。“孝、悌、忠、信、礼、义、廉、耻”这八个字是做人的根本,也是我们的大成至圣先师孔子他老人家的德育内容的全部精髓,也是人生的八德。“仁义礼智信,温良恭俭让”,这十个字也很重要,仁义礼智信为五常之道,五常是做人的起码道德准则,温良恭俭让是温和、善良、恭敬、节俭、谦让这五种美德,是传统美德的重要内容。与人为善,节俭生活,上敬下和,敦伦尽分,闲邪存诚,诸恶莫作,众善奉行。给我强烈震撼的还有四个字,那就是:非德勿行!行为和想法要符合德行,符合仁义,能够做到问心无愧,这点太重要了,非德勿行这四个字,每个字都有万吨的重量,人生就应该严格按照这四个字去做,努力活出圣贤的教诲。

\subsubsection{学习圣贤教育:让你的人生进入真正崇高而光明的境界,步入戒色大成澄明之境,戒出生命的大格局大气象。}

君子有造命之学,修身立命,积德累功,慈心于物,忠孝友悌,正己化人。看来看去还是圣贤教育最有味道,而且是越看越有味道,相隔几千年还是那么有道理,还是那么对。那个智慧,那个德行,那个磅礴的正气,那个深度和厚度,还有那个感人至深的慈悲,真的让我五体投地,泪流满面,总有一种力量会让人泪流满面,圣贤教育就有这个感染力,老祖宗那个智慧和德行真的太深厚了。戒色后应该认真学习圣贤教育,这样可以获得真正的大智慧,圣贤教育散发着璀璨的光芒,打开圣贤的开示,就能感受到那种高超的智慧和涵养,按照圣贤的教诲去践行,就能进入真正崇高而光明的境界,绝对不可思议的境界,和邪淫堕落的生活,简直一个天一个地,差别真的太大了,邪淫就像生活在暗无天日的地牢里,那个地牢里有电脑、手机、黄片,你就在里面撸啊撸射啊射,掏空自己身体的精华,把人生过成一坨屎。邪淫的世界是非常阴暗的低振频世界,戒色后懂得修善,开始学习圣贤教育,你开始感受到一个截然不同的高振频世界,你会感受到与邪淫时完全不同的快乐、轻松与自由,光明与美好,纯粹与高尚,圣洁与高贵,灵魂变得格外轻盈,整个人有一种灵动的感觉,清澈而充满灵气,焕然一新,这是多么久违的感觉,可以追溯到童年,记得儿时我们散发着灵性的光辉,有一种想奔跑、想跳跃的欢快。睁开你的双眸,整个世界再一次变得神奇起来,你充满喜悦的眼睛一如童年时那般纯真与美好。老子在《道德经》提到“复归于婴儿”,回归纯净纯善、纯真无邪的内心状态,这种状态是多么可贵,一旦开始邪淫,就会从这个状态跌落。戒色和学习圣贤教育给我们提供了一个机会,就是让自己回到最纯粹、最美好的状态,一旦你重新进入这个状态,你才知道真正的快乐与满足是什么!

圣贤教育我推荐《论语》《弟子规》《易经》《太上感应篇》《道德经》《了凡四训》《菜根谭》《传习录》《寿康宝鉴》《安士全书》《俞净意公遇灶神记》《文昌帝君戒淫宝训》《关帝圣君觉世真经》,秦东魁老师的两本:《你是自己的命运设计师》、《运气提升法则随身自查手册》。有佛缘的话还可以看看佛法方面的书籍和开示,元音老人、黄念祖老居士、宣化上人、印光大师、虚云法师、憨山大师、星云法师、梦参法师、本焕长老、大安法师、丁愚仁老师、犟牛居士、希阿荣博堪布、顶果法王、法王如意宝、大宝法王等的开示都是很不错的,大德有很多,选择自己有信心的,和自己有缘的。宣化上人法汇 app 相当殊胜,有缘的戒友可以下载学习,这个 app 做得很精致,内容也很全面,真的很不错。我们对于圣贤教育要有一个开放、包容和接纳的态度,摈弃过去的一些误解、成见和偏见,这样才能获得真正的法益,对于自己目前还不能接受的,应该求同存异,一个尊重和包容的学习态度真的很重要。另外推荐三本灵性书籍《当下的力量》《终极自由之路》《幕后:一位觉者的实修日记》,这三本我之前的文章结尾都推荐过,真的是非常好的书籍,对于提升灵性、认识真我很有帮助,这个世界还有比性更爽一亿倍的事情,那就是认识真我,安住于真我!戒色最终要进入更高的道途,去认识真正的自己,活出你的真我。

《素书》开篇的一段话:

\begin{quote}\it
    贤人君子,明于盛衰之道,通乎成败之数,审乎治乱之势,达乎去就之理。故潜居抱道,以待其时。若时至而行,则能极人臣之位;得机而动,则能成绝代之功。如其不遇,没身而已。是以其道足高,而名重于后代。
\end{quote}

意思是说,贤明能干的人物,品德高尚的君子,都能看清国家兴盛、衰弱、存亡的道理,通晓事业成败的规律,明白社会政治修明与纷乱的形势,懂得隐退仕进的原则。因此,当条件不适宜之时,都能默守正道,甘于隐伏,等待时机的到来。一旦时机到来而有所行动,常能建功立业位极人臣。如果所遇非时,也不过是淡泊以终而已。也就因此,像这样的人物常能树立极为崇高的典范,名重于后世。

这段话我个人比较喜欢,戒色修善以待天时,即使此生不遇,也能淡泊名利,尽己之力为社会为国家做出自己应有的贡献。真正的戒者对修行有更高的追求,所以能淡泊名利,以悟道为乐。恬淡、洒脱、豁达,一份超然的心境,宁静悠远的氛围,就像屹立于高山之巅,俯瞰尘世,内心所有的东西都放下了,归于空灵。不断修心、悟道、行善、提升德行,步入戒色大成澄明之境,安住内心的纯粹与美好,连结本源那神圣的力量,你就是纯粹的觉知,你就是那颗闪耀着璀璨光芒的钻石!

\paragraph*{总结}

戒色不单单是戒除手淫恶习,而是整体生命的改造或重建,树立正确的三观,过一种高度自律和负责的正能量的人生,这样的人生更有意义,更有价值,更有幸福感。戒色十规的横空出世相信会帮助更多的人认识戒色的本质,掌握戒色的原理和规律,也会帮助大家少走弯路,能够尽早摆脱恶习而戒色成功。这十规其实大家并不陌生,我之前的文章都有讲过,只是这次的总结更深入、更完善、更全面、更系统,戒色十规的整体架构已经非常成熟和完善了,相信广大戒友看了这戒色十规都会深深明白这十规的价值和分量,这是我这么多年的学习、研究、答疑、实战体会的巅峰总结,希望大家好好推广戒色十规,想到《戒为良药》就应该想到戒色十规。不以规矩,无以成方圆,戒色的确需要一定的规矩和规范,这样就能很快步入正轨,当然我不会勉强大家去执行,关键还是要靠自觉。问题不在于你学了多少,而是你真正理解了多少,吸收了多少,落实了多少!要看你落实的力度、质量和程度!必须下大决心,拿出超强的行动力!坚信自己一定能成功!别人可以,为什么你不可以?!戒色十规主要走专业戒色路线,断念口诀不需要信佛也可以背诵和掌握,其他一些戒色体系可能需要念佛持咒,这对于一些戒友可能不大好接受。断念口诀是一个修心诀,这个口诀也不是我发明的,在大德开示中就能看到,这个口诀其实是帮助大家提升觉察力,强大的觉察力可以让念头瞬间消失,即使你不信佛,你也可以练习觉察力,所以这个口诀普适性更广一些,当然我也很尊重念佛持咒、思维对治等修心方式,不管何种修心方式,都是为了断念,都是不让念头连续下去。

曾经我活得像具行尸走肉,惶惶不可终日,享受不到内心真正的安宁,在心魔驱使下一次次疯狂掏空自己,缓缓抬起头望着镜子里的自己,是那么灰暗、无神和颓废,眼睛中毫无生气,看不到生命能量的流动,两个严重下垂的睾丸一上一下荒诞地挂着,浑身散发着恶臭,我彻底厌恶和憎恨这个堕落的自己。是戒色修心和学习圣贤教育让我产生了极大的蜕变,我降伏了心魔,主宰了内心,我看到了全新的人生风景,因邪淫而黯淡的生命也重新绽放出绚丽的光彩。成为一个纯净、纯粹、纯真的单纯存在!纯净就是最伟大的力量,是时候拿回你的力量了!此刻,静静地感受内心的纯净,内心的美好,满溢着神圣源头的耀眼光芒……宛如一颗旋转的璀璨美钻,挥洒着圣洁而灵性的光辉。纯净、庄严,散发着纯真美好的能量,脸上有了最美的表情,眼睛清澈而明亮,如此纯净、如此美好,内心充满着自由、爱、喜悦和平安。你在外面所遍寻不到的一切,都在那儿,只需净化心灵,就能找回那种感觉。重新拥有纯净的力量,属于你的力量!你的人生将会被全然改写,还是那句话:你真正渴望的不是黄片,不是手淫,不是任何形式的纵欲,而是做回纯净的自己,这是最深的渴望,只有做回纯净的自己,你才会真正快乐起来。有一个大秘密很多人还不知道,那就是:空里面有无限的喜悦和满足,让内心空净,就能体验到各种美妙的感受,喜悦和满足本自具足,不假外求。戒色后爱笑了,发自内心的喜悦,时常感受到幸福的满足感,我已经掌握了这个大秘密,现在把它分享给大家,真诚祝福每一位戒友都能完成净化和蜕变,生活在纯粹的美好和喜悦中!享受纯净生命的每一刻!你就是奇迹!!!

戒色十规:

\begin{description}
    \item[远离黄源] 戒色后一定要注意远离黄源,删除手机和电脑里的不良资源;
    \item[学习戒色文章] 每天坚持学习戒色文章,多做笔记,多复习笔记,提高吸收率;
    \item[练习断念] 建议每天背诵断念口诀五百遍以上(默念即可),一定要正确理解口诀的含义,严格按照口诀的意思去做,有佛缘的戒友也可以念佛持咒;
    \item[做好慎独] 独处时很容易破戒,戒色后一定要做好慎独,保持警惕,防意如城;
    \item[视线管理] 对境实战时要做好视线管理,看到诱惑马上避开,不聚焦、不停留、不回看;
    \item[养生恢复] 加强养生意识,严格控遗,适量锻炼,做有氧运动和养生功法;
    \item[情绪管理] 不良情绪或者狂欢情绪容易导致破戒,所以要做好情绪管理,保持心平气和;
    \item[改过迁善] 要勇于改正自己的不良习气,要克服一切负面的心态,多行善积德;
    \item[培养德行] 首重“谦德”,常修八德:孝、悌、忠、信、礼、义、廉、耻,多发感恩心、谦卑心、恭敬心、惭愧心、忏悔心;
    \item[学习圣贤教育] 让你的人生进入真正崇高而光明的境界,步入戒色大成澄明之境,戒出生命的大格局大气象。
\end{description}

下面分享一首戒色诗歌。

\begin{poem}[万念敌]
    \begin{multicols}{2}
        \centering~\\
        轮回中宿命的对决 \\ 一次次倒在了心魔的铁蹄之下 \\ 沦为了疯狂的撸囚 \\ 不顾一切地掏空自己 \\ 眼中的纯净光辉逐渐暗淡 \\ 耷拉着眼皮的猥琐男 \\ 浮现在镜子里 \\ 感到挫败、空虚和悔恨 \\ 曾经纯洁的少年 \\ 竟然沦落到如此不堪的地步 \\ 实在让人唏嘘 \\ 他感到无奈 \\ 因为心中那股力量实在很强大 \\ 一次次把他拖入怪圈 \\ 念头再次悄无声息地 \\ 出现在他的脑海之中 \\ 把他带跑,让他沦陷 \\ 这次他顿悟了前辈反复强调的重点 \\ 那就是——断念实战!!! \\ 对!断掉就不会破戒! \\ 断不掉就会被附体! \\ 回想每一次破戒 \\ 真的就是这样! \\ 终于明白了! \\ 他开始练习断念口诀 \\ 逐渐领悟口诀的奥义 \\ 一遍一遍地练习 \\ 结合实战体会 \\ 领悟越来越深 \\ 某天晚上,心魔不请自来 \\ 一幅图像浮现在脑海中 \\ 唰!瞬间就消灭了! \\ 几乎在出现的刹那 \\ 就被觉察之光划掉了 \\ 关羽、张飞万人敌 \\ 他要做万念敌! \\ 一夫当关万念莫开! \\ 抬头望向浩瀚的星空 \\ 他融入了纯粹的觉知 \\ 戒色修心是生命中 \\ 最浓重的色彩 \\ 最传奇的经历 \\ 最荡气回肠的 \\ 绝杀!!!
    \end{multicols}
\end{poem}

下面推荐一本书。

\begin{book}[《心灵午夜密谈》]
    作者有幸结识了印度心灵大师萨古鲁,并与他共度一个星期的时间。在每个万籁俱寂的夜晚,他们在无人小岛的篝火旁讨论生命、死亡和命运的人生课题。萨古鲁带给她的是内心的平静和感悟,她称之为“改变一生”。萨古鲁对生命和灵性的感悟会带给你真正的感动,它包含了生活的真义,聆听在静谧的夜晚充满智慧的话语,将打开你对世界认识的另一扇窗。这是一本需要多花些时间研读的书,参透心灵,领悟人生,揭示生命的本质,解答终极的疑惑。萨古鲁的讲话深刻、意境深邃,为我们带来深入的洞察、超越的逻辑及始终如一的智慧。此书以其优雅而简朴的风格,将你带进意识探索的神秘领域,从而让你领会到一个更高的生命真相。这本书之前有戒友推荐过,我自己也专门看过,看的是电子版,做了 180 条笔记,的确是一本给人以巨大启示的好书,萨古鲁的样子是那样超乎寻常,显得既古老又年轻,庄重而又光彩焕发,而且显得极其优美。对修行感兴趣的戒友可以看看这本书,相信会有所收获。
\end{book}
