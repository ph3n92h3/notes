\subsection{《当下的力量》笔记分享与解析}

\paragraph{前言}

这季前言谈下行善与修心的问题,到底是行善是戒色的根本,还是修心是戒色的根本?如果你看过很多戒色文章,一定知道不少戒色文章都在大力提倡行善,这并没有错,行善可以大大增加正能量,正能量上去了,对戒色很有利。但如果只强调行善而忽视修心,那还是可能会破戒的,有些人善根深厚,的确很能行善,做了很多善事,但是后来他们还是破戒了,就是没有真正懂得修心。我看过很多大德的开示,他们讲修心,也讲行善,修心和行善,两者不可偏废,就像鸟的两只翅膀,古代马车的两个轱辘。从某种角度来讲,善和德的确是根本,所谓厚德载物,善和德是地基,但从更高的角度来讲,修心则是根本,善和德的修炼也属于修心的范畴之内,而且修心更强调直接对治邪念,这是行善所不具备的。总体而言,修心是核心,行善是辅助,在更高层面行善也并入修心。诸恶莫作,众善奉行,这八个字相信大家都知道的,断恶是放在前面的,恶中之魁就是意恶!俞净意公之前只知行善,而不知断除邪念,命运依然凄惨,后来经过灶神指点才恍然大悟。本名叫俞都,后来自别号“净意道人”,即自净其意。修心就是断除邪念,多发善念,修心本身就包括了善的部分,由善念而有善行,其根本还在于心,《四十二章经》:“心如功曹,功曹若止,从者都息。”《金刚经》:“降伏其心。”古圣先贤说过“克念作圣”,心念是问题的根本,戒色要懂得修心,真正掌握观心断念,也要懂得行善积德,行善与修心都是非常重要的,两者同为戒色的根本。还有的人的戒色思路就是提倡充实生活、忙起来、转移注意力,这也是不懂得修心,充实生活等只能作为辅助,有的人生活非常充实,也很忙,但这只能管一时,再充实的生活也有独处的时候,再忙的人也会有休息的时候,到时就可能会破戒。学生党对此应该深有体会,平时在学校忙于学业,生活很充实,但是周末回家独处时间多了,就破戒了,心魔专门在周末的时候疯狂攻击学生党。有的人从早忙到晚,累得撸不动,只想马上上床睡觉,但是当第二天睡醒后心魔攻击他,他就破了。充实生活、忙起来、转移注意力,这些只能作为辅助,如果你把这些当成核心而忽视修心,那迟早会被心魔攻陷的,这点一定要认清。在我看来,想靠忙起来戒色的人,其实还是戒色的门外汉,因为他不懂修心,以为忙起来就可以戒掉,等破戒几次后他就知道忙起来并不能戒掉,忙起来也许能坚持一段时间,但最终还是会破的。

以下是戒友“广正”的体会。

\begin{quote}\it
    修心是主要,行善是次要。之前我是个戒油子,总是戒戒破破,后面戒过一百多天,也算小有成绩,那个一百天我说一下怎么戒的,就是行善、放生、孝顺,学习正能量,早睡早起,戒了游戏,电视新闻都不去看,那种状态下戒了一百多天。但是那个一百多天其实不算真正意义上地戒,虽然我做到了一些正能量的事情,充实了自己的生活,但是我一闲下来有煎熬感,紧张不安,因为我不懂得觉察断念,每次邪念反反复复出现,很煎熬,我得拼命地皱着眉头去排除那种念头,很煎熬,因为不懂断念,越排越有挫败感。那种煎熬真的很难受,因为不懂得观心断念。这里还想说一点,就像我一般,如果你之前戒戒破破是个戒油子,你通过某种方法戒了一段时间,你就会认同这种戒色方法,因为你之前没戒过这么久啊,哪怕只是几个月,对一直破戒的你来说也是很高了,你就会很认同这种方法。认同之后其实看其他的文章就没有那么好契入了,就像我,我刚开始是认同依靠正气去戒色,虽然时不时有邪念有煎熬感,但是因为我认同了正气这种方法,我就会想这方面的观点,我当时的观点就是:有邪念是因为我的正气还不足,还需要加强,正气很足断念就没煎熬感了。可是后来一次次的邪念让我很紧张不安,我意识到这不像是真正的戒色,我强行地依靠正气可以说是忍了将近一百多天,那个阶段我看飞翔老师的文章很难看进去,因为我认同了依靠正气戒色。但是后来我破戒了,就是因为念头,一次比一次猛,由于我没有任何观心断念的练习,导致连续破戒。我后来发现了,就是念头!戒色就是要修心,你需要断邪念,还需要断怂恿,还需要断微妙的感觉,微细的压缩过的念头,一个比一个强大,这是我后来领悟到的,如果只靠正能量而没有修心的功夫是不行的,总有一天会败下阵来。但是我也没有把正能量抛弃掉,修心 + 正能量才可以,光走偏哪一个都是不行的,这是我一点浅薄的体会。
\end{quote}

广正这段体会很能说明问题,是他的切身体会,光靠正气和行善是不行的,有的戒色文章把正气和行善宣扬得很厉害,认为只要大力行善就可以戒掉。我之前已经发现了这类戒色文章的问题,但毕竟是正能量的文章,我也不好说什么,但我心里很清楚这类文章的缺陷,那就是靠正气虽然可以帮助你戒一段时间,但迟早会破戒的,因为邪念会一次次入侵,会不断冒出来,断念能力不行的人迟早会被心魔攻破,正气并不能化邪念,能化除邪念的是你的断念能力,正气是助力,仅有助力而不具备直接断念的能力,依然还会失败。做公益的犯邪淫,为什么会这样?就是因为不懂得对治自己的邪念!他们虽然做了很多善事,但是当邪念袭脑时,却不知立刻断除,结果还是沦为了心魔的傀儡,做出了让人很不齿的事情。念头肯定会上来,猛烈地上来,疯狂地上来,如果断念慢,就会有煎熬感,如果断念不力,就会被附体,身不由己地破戒。即使正气十足的人,依然还会起邪念,因为念头会自动冒出来,邪淫回忆、图像、怂恿、微妙感觉、性幻想、各种意淫等,一波一波地攻击你,企图把你拿下。我戒到现在,依然还会遭到心魔的攻击,修心是持久战,能让我做到不破的就是因为我的断力足够强大,能够瞬间降伏心魔。当然我也很注重行善,行善增加的正能量是很大的助力,会让内心有一种崇高的感觉,这让戒色状态更稳固、更扎实、更深厚。戒色是系统工程,应该把握核心,分清主次,真正的戒色高手都是在念头上直接操作的,正是强悍的断念能力让他们立于不败之地。大德强调行善、改过和做人的道理,这些的确非常重要,但是当你听到大德讲断念的内容时,真正识货的人会屏息凝神,生怕错过一个字!因为这是最高的实战秘籍,不是随便能听到的,很多人即使听到了,也听不懂,断念实战是最终的检验,来不得半点虚假,关于断念的指导是最为关键的部分,要反复学习,深入领悟,真正吃透。不管看了多少戒色文章,做了多少善事,念了多少佛号,最后就看实战那一下,请记住,念头肯定会再次上来,心魔一直在虎视眈眈,念头上来的一刹那,就看你的实战表现了!

下面分享几个案例。

\begin{case}
    飞翔老师,我戒到现在两年了,一直很稳定,每天都学习戒色文章,养生,断念训练。和心魔斗了无数次,都过来了,一般来说一到破戒高峰期,我能感觉到心魔大举来犯的时候,我会先读一篇戒色文章,然后加上断念保护,很快就能击溃心魔的进攻,然后就感觉到很快它就退了,这个过程也就几个小时左右,没什么压力,剩下的散兵游勇也构不成威胁。可是这几天心魔几乎每天都在疯狂进攻我,大规模的那种,即使看戒色文章也没以前作用那么大了,心魔就跟疯了一样,怂恿、图像、微妙的感觉。虽然我都顶住了,但是它老这样疯狂进攻我,也让我心里挺闹腾的。希望您能给我解答下,两年了心魔从来没这么疯狂过。

    \textbf{附评} 这位戒友戒了两年了,很不容易,印光大师开示过:“等到道力渐深,藏在八识里面的多生根本习气,被功夫逼迫出来,或欲念横飞,或妄心乱起,力量甚大,非比寻常。”戒到一定时候肯定会猛烈翻种子,心魔会疯狂进攻,我也经历过多次这样的情况,关键还是战斗力和威慑力!战斗力极为强大才能产生威慑力,如果你战斗力强悍,心里根本是不怕的,来多少,灭多少,你有绝对的战胜把握,就像孔子说的:“我战则克!”虽有一定战斗力能顶住,但威慑力不行,心魔还会疯狂进攻,一波进攻下去了,另一波进攻又上来了,的确让人心里挺闹腾的,而且感觉很累。就像拳击比赛,如果两者实力相当,就会你来我往,你打退对方,过一会对方又上来了,如果你具备 KO 的实力,一拳把对方打挺尸了,那就不会上来了。我最近一次断念实战,心魔三幅图像接连上来,我三记觉察就消灭了,一记比一记狠,三记觉察就像一套组合重拳,一记上勾拳,一记右摆拳,一记后手直拳,直接把心魔干趴下了,不敢进攻了。一位戒友说:“早上感受到了心魔来袭,我将它断掉了,但因为断得不够狠,邪念不断地冒出,一次更比一次强烈。”他就是断力差了点,还不够狠,断念一定要快、严、烈、狠!敌硬,我更硬,敌强,我更强,犀利的眼神,勇猛的断力,彻底把心魔打服打怕!打得心魔不敢轻举妄动,这股杀气这股狠劲一定要有!!!戒色要有一股杀光一切邪念的铁血杀气,手撕心魔的杀气!向内杀,杀自己的念,冒出就杀!杀!杀!杀!杀气十足,杀气冲天!杀尽始安居,杀光了心里就太平了!这是你死我活的战斗!敢于亮刺刀,敢于拼刺刀!血战到底,绝对强硬!不怕贼强,只要将猛!断念必须非常猛利,横刀立马,斩杀邪念!譬如百万军中单刀直入,横握金刚王,触目不容存一法!加强狠劲的训练,必须对念头狠!你断念时的狠劲要让心魔颤抖,要让心魔知道你的厉害,知道你的狠,知道你是狠角色,知道你不好惹,吓死心魔!威震敌胆!!!

    三种实战的情况:

    \begin{enumerate}
        \item 水平很差,被心魔碾压,真可谓“一触即溃”,心魔一来,毫无招架之力,胜负没有任何悬念。
        \item 水平相当,陷入缠斗,就像打拳击一样,打退了还上来,更加疯狂地进攻你,企图把你拿下。
        \item 实力超出一大截,压倒性的胜利,并且能形成强大威慑力,心魔闻风丧胆,不敢轻举妄动。
    \end{enumerate}

    另外一位戒友给我留言:“飞翔老师,我无聊躺着玩手机时被黄毒回忆以图像形式袭脑之后,我进行了抵抗,可是黄毒接二连三接连不断地对我进行色弹攻击,我跟心魔之间的斗争就像辩论会一样你来我往,最后心里还是久久不能平静,特别痒痒特别想破,在心魔怂恿下看黄破戒了,您觉得我的破戒根本原因是什么?以前看的黄片回忆以画面形式袭脑,断了以后还不断浮现,不停地骚扰,怎么办?我记得您说过能回忆起来的属于印象比较深刻的黄毒,这种印象比较深的抵抗起来比较吃力,最终不敌心魔,欲火中烧看黄破戒了。”

    我的回复:根本原因就是图像袭脑时,没有及时断掉。你是抵抗了,只是断力不行,你能抗住一会,但是接二连三的攻击,你却顶不住。平时要好好练习断念,磨刀在平时,用刀在实战,要加强断力,要狠一点。心魔攻势很猛,关键你要具备强大的战力,才能顶得住。不怕念起,就怕觉迟,它不断浮现,我这勇猛狠断,打得心魔不敢轻易进犯。你的实力必须高出心魔一大截,这样才能具备威慑力,有了威慑力,心魔就不敢随便动你了。
\end{case}

\begin{case}
    真正看 H 是初二,当时偷看了家里的录像带,我记得很清楚,脑海中有一个声音告诉我:“完了,这孩子完了!”同学说我整天愁眉苦脸,感觉自己就像游戏中的人物,刚刚吃了加血,结果就被心魔一刀砍了,总是处在快死的边缘,怎么可能把学习搞好!邪淫就像一根铁链锁,越缠越紧,自己很难解脱,任它摆布,痛苦不堪。几十天的戒色之后,我强烈地感觉到,我不是个坏人,我是一个很有良知的人,我要让那个单纯、善良、上进的少年挣脱邪淫的枷锁,重新站立起来!短短几十天,在人生中是多么的短暂啊,但是我的感觉太棒了!首先是脸部的感觉,不像以前总是干瘪瘪的,开始饱满了,然后是眼睛,以前总是睁不开,现在工作了一天,晚上回家路上还是圆睁睁的,像是一股强大的力量充斥着,不知疲劳。走在马路上,树木、花草、河流、蓝天、白云,一切都是鲜活而明亮的,说话声音不沙哑了,浑厚有力,最关键是有上进心了,感觉一定要做出点什么!以前迷恋成功学,说什么要设立目标,要行动,要主动,要细节,要积极,现在终于明白了,男人要经营好自己的人生,根本上是要戒色!我看了我做得好的同学的照片,全部都是两眼炯炯有神,没有眼袋,没有黑眼圈,眼睛下面有一条细细的眼线,才最终明白,原来人生的秘密全部在这里,要戒色,要纯洁,要善良,要精力充沛,要积极上进,要把有限的精力放在有意义的事情上,而不是每天想着找 H、SY、YY,想睡觉!宝贵的精液是人身的能量块,没有能量的人,没有了精力,没有了阳气,两个黑眼圈,就如同还没有彻底死亡的鬼,怎么可能在人世间做出贡献!

    \textbf{附评} 这个案例的戒友是 81 年出生的老哥,37 岁了。他在帖子里说:“从 97 年开始 SY,到现在有 21 年了,人生中最宝贵的 20 年被邪淫毁了。现在的我,没有存款,与妻女两地分离,创业屡屡失败,在家靠父母救济,人生跌落到了低谷,就差上街要饭了!高血压,失眠,没有精力,怕冷,每天犯困,睁不开眼,腰疼,哈欠连连,看书看不进去,头晕,头疼,不喜欢社交,害怕看人的眼睛,容易感冒,反应迟钝,健忘,双眼无神。”戒了几十天,他的改变就很大了,他这段文字写得很好,体会很细腻。戒色后有一种焕然一新的感觉,能量上去了,感觉就完全不同了,“走在马路上,树木、花草、河流、蓝天、白云,一切都是鲜活而明亮的”,在沉迷邪淫后,看这个世界都会觉得灰暗,觉得了无生气,因为自己是灰暗和了无生气的,所以看这个世界就会带着这种感觉。境随心转,当心净化了、干净了、纯粹了、祥和了,看这个世界就截然不同了,小时候看这个世界会觉得很神奇,花花草草似乎有一种明亮的色彩,有一种鲜活感。进入发育期开始手淫后,那种鲜活和明亮的内心感受就消失了,内心变得沉重而不快乐,感觉也变得迟钝了,树还是那个树,花还是那个花,绿色还是那个绿色,但已经感受不到那种鲜活的振动频率了,内心充斥着各种负面的感受。这位戒友戒色后,自言感觉太棒了,精力变足,双眼圆睁,“像是一股强大的力量充斥着,不知疲劳。”并且有了上进心,戒色后的身心感受就像破车换新车一样,也像高度近视戴上了眼镜,瞬间这个世界变得清晰了、鲜活了、明亮了,很多细节都能感受到了,你会感叹,原来这个世界是这样的,为什么之前感受不到呢?一位戒友说:“戒色一年,整个人换新。无论是心理、精神还是身体,我都感受到了前所未有的愉悦,每天脑中想的都是健康美好的事,不再像过去一样。整个人充满清新活力,身体变好,没有之前的体弱多病,学习工作都是火力全开,人也精神了不少。心胸狭隘、自私的毛病和心理轻了许多,微笑出现在脸上,看到花花草草有一种幸福的感觉,仿佛回到花季雨季。”戒色的感觉就是这么棒!戒出全新的自己!充满活力,充满朝气,充满正能量、充满底气与自信,这就是你真正想要的感觉。

    成功学我也看过,有些理论是给人不少启发,但是巧妇难为无米之炊,肾精不足,你干什么也不得劲,容易走霉运,容易失败,因为邪淫后整个能量场都变得极其不好,振动频率严重下降,很容易感召不好的人事物。有些戒友说撸后就有不好的事情发生,的确是这样,起了那种不好的念头,感召的就是不好的事情,家庭也容易变得争吵,脾气也会变得暴躁,内心极其烦躁。能干成大事的人,都懂得管理自己的能量,特别是性能量,真的不能随便耗损,再强的男人疯狂纵欲后都会烂泥扶不上墙,我前段时间看了泰森和道格拉斯那场比赛的击倒镜头,那个泰森让我诧异,你会发现他的肌肉维度小了近乎三分之二,就像没练过拳击的人,然后眼神看上去没杀气了,显得呆滞无神,自信和霸气完全看不见,走路都不太灵活,结果惨遭 KO,牙套都被打飞了,那场比赛前泰森疯狂纵欲,结果败给了名不见经传的无名小卒。戒色的确是人生的一大秘密,很多人都不知道这个秘密,毛主席早年听课笔记《讲堂录》里写到:“淫为万恶本,而意淫之为害,比实事尤甚,当懔懔然如在深渊,若履薄冰。”蒋介石:“欲立品,先戒色;欲除病,先戒欲。色欲不戒,未有能立德、立智、立体者也。避之犹恐不及,奈何有意寻访也!”1919 年 2 月,蒋介石在福建曾勉励自己:“好色为自污自贱之端,戒之慎之!”晚清第一名臣曾国藩也戒色,他深知沉溺于此是会妨碍事业的,严重损坏一个人的精气神,一分精神一分事业,十分精神十分事业。近代这三位举足轻重的人物都选择戒色,可见戒色之重要性,真的是君子第一修为。戒色后良知会慢慢复苏,会厌恶那个邪淫的自己,良知复苏后更觉得邪淫是不对的,男人应该把有限的精力和时间放在真正有意义的事情上,而不是沉迷邪淫,疯狂掏空自己!
\end{case}

\begin{case}
    谢谢飞翔老师,《戒为良药》我从第一季一直看到最新的一季,每季都看,目前戒色一年零八个月,期间没 SY 过。最近我和同事去参加培训,他看了一部电视剧,我也跟着看,开头的部分有些诱惑容易让人浮想联翩,差点破戒,到现在意淫的念头还会冒出来,戒色真的是如履薄冰,避色如避箭!稍有不慎,满盘皆输!

    \textbf{附评} 戒色一年零八个月,这个战绩很不错,现代戒色的难度比古代大多了,人人一部手机,找片比地上找石子还容易,过去看黄要买碟片,也很容易看腻掉,而现在资源太多了,可以提供无限的新鲜感,让你疯狂沉迷其中,不能自拔,很多人都像打了鸡血一样疯撸,简直不要命,丧心病狂。不少人都提到一定要有新鲜感才撸得起来,才能彻底兴奋起来,网上又有无限的新鲜片子可供浏览下载,这就会导致深度沉迷和严重上瘾。上次一位戒友说起了上世纪七八十年代的人,那是个没有网游和网络色情的年代,那个时代的年轻人精气神很足,那个年代的年轻人也许也会染上手淫恶习,但成瘾程度应该会轻不少,而现在这个时代的色情资源会最大化地刺激你的欲望,还有各种变态的片子来诱导你变态,这个时代戒色的难度可想而知,古人戒色会容易很多,古人没手机没电脑没网络,女性穿衣服也很保守,而今人戒色真的要做到“火中生莲”,在充斥诱惑的环境中生出一朵清净的莲花,这是很不容易的。即使戒色后不再主动去找黄看黄,但生活中各种擦边内容也会层出不穷,所以一定要保持警惕,对境时真的很考验人,不管你看的是什么戒色方法,最后的实战永远是对境时和起念时,要做到“内不随念转,外不为境迁。”外避内断,这四个字要严格做到位。对境时要“抓眼贼”,什么叫抓眼贼?在对境的一刹那,眼睛往往是贪看的,一刹那间就被吸引上去了,一下就着相了,着在了敏感部位上,那个过程只有零点几秒,即使要点关闭了,还在关闭前瞄一眼,这就是“眼贼”,对境时要抓眼贼,严格做到戒色十规的第五条——视线管理,看到诱惑马上避开,不聚焦、不停留、不回看。这种实战意识不是短期内练成的,是成千上万次的实战中越练越精的,对于很多人而言,第一反应永远是贪看,永远是眼贼,要做到第一反应是避开,是需要不断磨练的,也许刚开始十次有八次是眼贼,通过不断反省和实战总结,慢慢做到十次有五次是眼贼,然后渐渐做到十次有一两次眼贼,最后就能基本做到立刻避开了,而且心无波动,这是很高的境界,对境不动心,为大力量人!举起一吨的杠铃,不算大力量人,对境不动心,才是真正的“大力士”!平时不要去看擦边的图片、视频、小说,这个警惕心一定要有,真的是稍有不慎,满盘皆输!同事在看电视剧,他也在边上看,当看到诱惑时应该马上转移视线,或者站起来走动一下,不要再看那个电视剧,这时要严密监视内心的念头,因为看到诱惑时,邪念往往会紧跟着上来。即使当时没起邪念,也要警惕几天内心魔利用你看到的诱惑图像来攻击你,那幅图像已经瞬间印入了你的记忆,然后短期内会不断浮现。对境时是极大的考验,大家都知道滴滴车司机强奸杀人案,发生了很多起,就是在车厢内司机对境时起了邪念,他没断除,反而听从助长了那个念头,又是在密闭的空间内,很容易引发犯罪。不管戒多久,对境时一定要保持警惕,这点要切记!戒色十规要严格落实,戒色十规是紧密围绕实战、以实战为核心的一个体系,第 126 季要反复研读,也可以下个有声读书软件,反复听,反复吸收,要把戒色要点真正内化成实战意识。
\end{case}

\begin{case}
    赶上了直播,真开心!彻底戒色三年半,外加科学健身,现在充满底气,执行力强,身材标准,思维敏捷。时刻保持警惕,不断学习,不断提高觉悟,让觉悟始终以绝对优势降伏心魔!第三遍《戒为良药》看到第四十二季啦,每次都有新收获!

    \textbf{附评} 这位戒友的状态颇佳,戒色三年半是非常不错的成绩,一般过了三年就会具有相当的稳定度,但也不可掉以轻心、放松警惕,他的学习状态非常好,能够温故而知新,这样就能不断提升自己的觉悟。戒色的好处是非常多的,撸者往往只看到手淫一时的爽,迷恋短暂的快感,其实那种快感非常虚无缥缈,当时好像感觉很舒服,但过后就是失落和空虚,乃至悔恨,多巴胺下降后,好像整个人生都没意义了,什么事情都不感兴趣了,内心陷入一片悲惨的境地,再看之前的片子,一点感觉都没了,好像垃圾一样。症状出来后就是不舒服,太不舒服了,简直难受死了,甚至生不如死!就是因为贪恋手淫那点“舒服”,最终换来大大的“不舒服”,大大的痛苦和惶恐。肾精能量丢失后会出现很多坏变化,整个人的气场都会被削掉一大块,撸后照镜子就会感觉到那种不良的变化,好像某种能量被抽走了,有一种衰败的迹象开始出现了。戒色后有一种自由感、摆脱感、轻松感、喜悦感、明亮感,充满底气和自信,敢于直视,敢于奋斗自己的人生,撸者往往缺少直视的力量,不敢看别人眼睛,好像做贼心虚一样,我过去就是如此,后来戒掉后,有一种光明磊落的直视力量,好像能看穿别人的内心,这种直视的力量是很可贵的,这是真正有底气和自信的表现,这种底气不是来自钱,而是来自戒色,来自修心,来自降伏心魔,来自积攒肾精。戒色后你甚至敢和一头老虎对视,而手淫时你甚至不敢和一条狗对视,这就是最明显的改变,戒色后一种内在的力量开始增长,只要不破戒,这股力量就会积累得越来越雄厚,越来越磅礴,越来越深不可测。整个人神气饱满,双目光明,眉宇间一派浩然正气,眼中闪烁着纯净的光芒,充满灵气,直视时非常有力量。戒色后发现自己再度成为一个小孩,眼睛再度变得跟小孩的眼睛一样清澈明亮,重新像孩子那样以惊喜的眼光来看待这个处处都是奇迹的世界。耶稣说:“只有那些像小孩的人才能够进入神的王国。”像小孩一样纯净明澈的状态才是你最向往的,因为这是最贴近本真的一个状态,简单、纯粹!充满力量!

    国外一位戒友说:“我永远不会忘记我在戒色中最先获得的三个好处:我睡得更少了(大约六小时自然醒);整天精力充沛;我对音乐的感受更好并且我的想象力也因为它而提升。之后我明白了在戒色前我所走的路是多么具有破坏性。我答应自己:无论我感觉多么糟糕,无论我经历什么,我都不会再回到以前的生活方式。在戒色之前,我感觉自己已经死了,只不过是一个迷失的灵魂,正在看着别人过着他们的生活,这真是一种可怕的感觉。而在我的内心深处,我知道生活应当比现在这样更有意义。我经常觉得有什么不对劲,但又说不出来是怎么回事!在我曾经认为色情是正确的时候,我根本不知道它对我造成了多大的伤害,也没有人告诉我色情是如何对心灵、身体、精神产生危害的。从灵性层面来说,当你努力保留自己的生命力,并且让色情滚出你的生活时,你的身体、心灵和精神会重新连接起来,你的频率就会上升。你的频率越高,你的感觉就会越好!你戒色时的频率可以给你的生活带来好的环境,因为你散发出什么,宇宙就会给你什么。一旦你踏上戒色旅程,你便开始散发出你天性中本有的积极能量!”

    这位国外戒友说得很好,戒色后睡得更少了,因为身体充电效率明显提升了,整天精力充沛,邪淫时睡十个小时都感觉累,都爬不起来。戒色后感受力增强了,对音乐、对大自然有更细腻、更美好的感受。当一个人沉迷色情时,往往会认为这样的生活很爽,于是疯狂纵欲,等到亲身体验到邪淫带来的负面影响和苦果时,这时才幡然醒悟,才开始反省过去的生活方式。纵欲最后的结果必定是自我毁灭,对心灵、身体、精神都会造成巨大的伤害,当戒掉色情后,振动频率就会上升,懂得行善积德,频率上升会更显著,频率越高,感觉就会越好,到时就会体验到真正的大快乐,有一种发自内心的幸福感。
\end{case}

\begin{case}
    飞翔大哥,自从发现自己生活中种种问题的根源是由于邪淫后,我下了大决心去戒邪淫,现在已经初步取得了一定的成果,可是每每想到自己曾经所做的错事,以及自己曾经遭受的痛苦,我心里就很难受很痛苦,也一直在自责悔恨,可这样对我现在的生活有很大影响,往往一想到那些事情,我就什么事也干不了了,一直陷在那些痛苦里面,同时还加重了社恐,我现在该如何去看待那些曾经的错误,如何去解决这些心理问题呢?希望飞翔大哥能够帮我解答,感激不尽!

    \textbf{附评} 这位戒友提出的问题很典型,相信每位戒友都会遇见这种问题,我自己也多次遇见这种情况,过去做过的错事会反复在脑海中跳出来,然后情绪一下被带至懊悔、自责、悔恨,甚至有些愤怒,恨自己当初的无知,恨自己做了那么不堪的事情,不仅给自己造成不良影响,也给他人造成了一定的伤害,有些错事真的是不堪回首,一回想就会陷入深深的自责之中,有种无地自容的感觉。人在自私状态和邪淫状态时会做出很多错事,有些错事做过了也就忘记了,但有些错事错得太荒唐、太龌龊,甚至太邪恶,这种错事往往会留下深刻的印象。从更高的角度来讲,反复的懊悔是心魔的一种策略,目的在于扰乱你的内心,让你的内心陷入混乱,这样心魔就容易得逞,修行方面专门讲到了懊悔是干扰内心平静的一大障碍,我们要学会克服。人做了错事肯定会懊悔,人非圣贤孰能无过,关键要真诚忏悔,下决心不再去犯,所谓“忏悔得安乐,忏悔得清凉,忏悔得自在,忏悔则心安”。孔子强调“忠恕”,忠诚、忠厚,也要懂得宽恕,不仅宽恕别人,也要懂得宽恕自己,以前我对忠比较好理解,对恕理解得不太深刻,后来才真正认识到恕的意义和价值,基督教也讲宽恕,为什么圣贤都强调宽恕呢?因为宽恕后,内心才能恢复平静,才能更好地契入真我。如果不宽恕别人的错误,你就会想着报仇,内心充满嗔恨,甚至想杀掉对方,这是极其负面的心态,是具有毁灭性的心态,这样只会增加自己的负能量,导致振频下降,内心很不平静,会被那类报仇、怨恨的想法驱使,做出很多出格的事情。如果懂得宽恕,内心就会恢复平静,你宽恕别人了,那类报仇、怨恨的念头就会消失,最终你会发现,你真正的敌人不是外在的,而是那类报仇、怨恨的念头,是那类念头在反复折磨你,让你痛苦。如果宽恕别人了,内心也就放下了,恢复平静了。有的人你对他好,对他有恩,最后他给你来个恩将仇报,这时往往会很愤怒,但遇见这种情况也要学会宽恕,没必要拿别人的错误来惩罚自己,别人做的恶事自会有报应,根本不需要你去出手,天自报之。另外,就是要懂得宽恕自己,这也是非常重要的,如果你不宽恕自己,就会怨恨自己,陷入自责的心态,那种懊悔的情绪会对生活造成很大影响,一想到错事就会陷入内心的痛苦,什么也干不了,严重的甚至要抓头发、拍桌子、砸东西,面部表情痛苦,那叫一个悔恨啊!戒色之后我们一定要懂得宽恕自己,这也是我的经验之谈,那类错事经常会自动跳出来,企图让我陷入痛苦和懊悔,渐渐我就发现了这是心魔的诡计,不再理睬了,一跳出来就及时断除,不去想它了,内心也就恢复平静了。宽恕自己并不是纵容自己犯错和作恶,而是让内心恢复平静祥和,并且警醒自己不要再犯那类错误。懂得宽恕是一种很高的智慧,智者会运用宽恕的力量来让自己的内心恢复平静,一个宽恕的想法可以摆平很多负面的念头,让自己的心胸变得开阔,让内心变得祥和稳定。宽恕他人,是一种豁达,是一份大度,体现了自己的仁厚,宽恕别人,就是善待自己;宽恕自己,就能从自责、自恨、懊悔、纠结、痛苦的漩涡中获得解脱,内心不再消极、混乱和无力,不再自我否定,而是变得积极、有序而强健,宽恕自己,让内在恢复祥和,心地也会渐渐地广阔开朗起来。戒色之后要学会运用宽恕的力量,这是非常重要的技巧,生活中遇见的很多问题都可以在圣贤教育中找到答案,之前你看不懂,是因为你的体会不深,等到体会深了,再去学习,就会明白圣贤为什么要这样说,里面真的有很深刻的道理,极具大智慧。
\end{case}

下面步入正文。

记得大概一年多以前,有位戒友建议我谈下《当下的力量》和断念之间的关系,当时我觉得他的建议很好,之后也是在等待时机成熟,现在谈这个主题的时机已然成熟,就在这季和大家分享,当然不仅仅是谈到断念,而是把《当下的力量》一和二,还有《修炼当下的力量》这三本书的精华笔记和大家做一个详细的分享,这是很难得的机会,也是很特别的一季,这是众所期待的一季,绝对重量级的一季。《当下的力量》这本书我之前的文章有推荐过,很多戒友都看过这本书,这本书是一本灵修的书籍,灵修即“灵性的提升与修炼”。在国外叫灵修,在国内一般叫修行、修道,就是为了提升你的灵性,你的振动频率。关于灵修,网上有很多负面的报道,主要就是针对某些人动机不纯,借灵修之名骗财骗色,这类败类在每个行业都有。灵修本身是没问题的,不过我更喜欢说修行,修心,这样更有传统文化的味道。

《当下的力量》是由埃克哈特·托利(Eckhart Tolle)所写,托利生于德国。伦敦大学毕业后,他在剑桥大学担任研究学者和导师。29 岁那年,一次意外的经历彻底改变了他的生活,让他的人生道路发生了全新变化。接下来的几年,他致力于解释、整合和深化这种内在心灵的变化。托利的教法简单而深刻,与古代灵性导师一脉相承,用清晰明了的语言传达一个永恒的真理:我们可以在当下摆脱痛苦,获得宁静,进入自在平和的世界。目前,托利在世界各地广泛旅游讲学,努力把自己的领悟与意识力量带给世界各地的人们。托利是我个人比较喜欢的灵性导师,很有喜剧和幽默的天赋,如果他不当灵性导师,一定可以成为一位著名的喜剧大师。我个人也比较喜欢憨豆先生(Mr. Bean),高中时代就很喜欢看他的喜剧短片,我觉得托利和憨豆先生有那么一点相似,两人都是高学历,表情都很丰富,都特有喜剧细胞,总能不经意把你逗笑,托利咯咯地笑时,就像纯真无邪的孩子,特别有趣。有时托利沉默时,一个眼神都能让人笑起来,那个眼神的含义很有趣。有时他的眼神有某种害羞的感觉,有时则比较深沉,颇具智慧的感觉,有时则很神秘,像是在谈一个终极奥秘,有时的表情颇有孩子淘气的感觉。托利是高级知识分子,有学者的气质,也有孩子的纯真,这点很难得。

一位修行者描述托利:“说实话,他就是一驼背小老头,踟蹰地从后台出来,坐在光秃秃的台子上,往下一瞅乌泱泱的都是人。外貌协会的人肯定看不上他,但是,他内在的自性之光,他深沉宁静的临在感,他深谙人性之后的诙谐表达,却让我不得不爱上他!两年前我在挪威向他提问时,不由自主地先评论一句:‘你好可爱’,他的脸刷就红了。哇,真是好可爱,经常像天真的小孩子一般咯咯笑。他模仿小我的神经质的状态简直是太逗了,就像看相声小品一样,不过他的幽默是高雅而含蓄的。”很多人看了托利好几本书,看书和看视频的感觉是有所不同的,我建议大家去看看托利的视频,视频现场的那种感染力和氛围是书中难以感受到的。我看了很多托利的视频,视频中的托利给人的感觉是多样化的,有时西装革履,带几分英俊帅气;有时穿着朴素,有居家男人的感觉;有时穿的衣服有传道的特点,显得神圣而神秘。有年轻的托利,也有上了年纪略带驼背的托利,年龄跨度是比较大的,但诉说的真理一直未变,虽然诉说的方式与语气有所不同,但重点和核心永远是一样的,演讲时有即兴的灵感,还有和观众互动,这也是很大的看点。

什么是戒色的最高境界?有的人说是忘记戒色,这类观点很容易让人放松学习,放松警惕,实不可取。有的人说是不戒而戒,养成新习惯,转移注意力,充实自己的生活,这类方法能让人坚持一段时间,但忽视了修心的重要性,也不是最高境界。有的人说是行善积德,这是修福的境界,还不是最高的境界,有的人会说奋斗自己的人生,即使通过自己的奋斗当上了亿万富翁,住豪宅开豪车娶美妻,内心还是隐隐有某种不满足,不少富人活得也很苦,有的也得上了焦虑症、强迫症、抑郁症,还有跳楼自杀的。那到底什么是戒色的最高境界,戒色的最高境界是和修道直接接轨的,戒色的最高境界有两层,第一层,正己化人,活出圣贤的教诲;第二层,认识真我(the true self),活出真我——纯粹的觉知。这就是戒色的最高境界,每位前辈对最高境界的理解有所不同,在我这里,能够达到真我的层面才算是最高境界,在真我以下的层面都不能算最高境界。修行有句名言:“不识本心,学法无益。”本心即真我,不认识真我,学来学去还在外围,核心没有真正的认识和掌握。

《当下的力量》其实揭示的就是你的本心,让你认识自己的真正身份,不是身体,也不是念头,而是纯粹的觉知。从认识真我的角度来讲,《当下的力量》的确意义非凡,此书为《纽约时报》畅销书排行榜第一名,被翻译成三十多种文字,畅销全球,被誉为我们这个时代最具影响力的心灵启迪之书。一本灵性的书籍能如此畅销实属罕见,很重要的一点,就是《当下的力量》没有明显的宗教倾向,但却说着宗教最高的真理,而且托利有心理学的背景,这点也容易让人接受,最具戏剧性的一点就是:托利 29 岁时绝望到极点,觉得痛不欲生,产生了自杀的念头,但就在彻底绝望之时,他获得了顿悟。这个故事情节对于很多人而言颇受激励,因为现代很多人的确处于类似的心理困境,很多人也曾有过自杀的想法,看到托利的经历很容易产生共鸣,很多抑郁症患者通过学习《当下的力量》走出了心理的困境,获得了新生,这本书的确帮到了很多人,此书深者得其深,浅者得其浅,开卷有益。有的网站把《当下的力量》列为心理学书籍,也是有一定道理的,但从根本来讲,这是一本灵修的书籍,其内涵之深刻远超任何心理学的书籍。

资深戒友“青玄苍”和“谦为师”都提到了《当下的力量》,他们已经在这方面有相当深的领悟了,有这种领悟的人,在广大戒友中属于较少的那一类,位于戒色金字塔的顶端,这类极具悟性的戒友,就是戒色大班级的尖子生。也许你看过很多戒色文章,几百篇甚至上千篇,但真正提及这块内容的几乎很少,而这块内容就是修行王冠上的蓝宝石,是最具分量、最有价值的一部分内容。很多戒友看了《戒为良药》第 90 季有了顿悟,我那时也是写了很多戒色文章,看了很多大德开示以及很多灵性书籍之后才等到时机成熟,在第 90 季谈了这块内容,在第 99 季又加深谈了一下,有不少戒友都给了反馈,他们对这块内容还是很感兴趣的,受到了很大的启发。这次第 128 季正好结合《当下的力量》把这块内容再谈一下,希望能够带给大家一些新的体会。托利说过:“从本质上来说,灵修只有一种,但是它也许用许多不同的形式来表达。”真理只有一种,关于真我的真理被打造成了许多不同的钥匙,就等着有缘人来发现它。我最初的领悟来自于根本上师元音老人的直指,正是由于元音老人的慈悲加持我才能理解其他任何关于真我的开示,真的是一通百通,永远顶礼和感恩根本上师元音老人!能够师从元音老人,绝对是大因缘,大福报,真的太幸运了,元音老人是如云中黄山般的磅礴存在,又平易近人、亲切和蔼如邻家老爷爷,大家如果有机会应该好好研读元音老人的开示,真的是大手笔大气魄的大宗师,能够那样直接说法,一点也不拐弯抹角,真的是太罕见了。

下面一些概念先弄清楚,这些概念弄清楚了,才能更好地理解这本书,否则连基本概念都很模糊,肯定无法正确理解。

\begin{multicols}{2}
    \begin{itemize}
        \item 思维/念头/心智/头脑/大脑(Mind)
        \item 纯意识(Pure consciousness)
        \item 纯粹的觉知(Pure awareness)
        \item 本体或存在(Being)
        \item 那无限和永恒的真我(The true Self)
        \item 小我(Ego)
    \end{itemize}
\end{multicols}

小我(Ego)等同于Mind(思维),1、6 是同一个意思,2、3、4、5 是同一个意思,纯意识就是纯粹的觉知,就是本体,就是纯粹的存在,就是真我!

Mind 的翻译基本有这五种,其中前四种翻译都比较常见,翻成“大脑”较少见。《当下的力量》纸质书是把 Mind 翻成“大脑”的,这点不是很准确,但也还可以。Mind 最常见的翻译是“头脑”,其次是“思维”、“心智”,但都指向同一个根本意思,那就是——念头(thought)。

\begin{quote}\it
    小我由思维活动组成,只有不断地进行思考它才能存活。‘小我’这个词对不同的人来说有不同的含义,但是在这里,我所指的是虚假的自我,它是我们无意识地认同于思维而产生的。(托利)
\end{quote}

凡夫把思维认作自己,为什么会这样?因为思维当中有一种念头最贼!那就是“我念”,“我念”是最贼的冒充!什么是“我念”?“我念”就是我字打头的念头,比如“我要喝水”、“我要睡觉”、“我要出去玩”,我要怎么怎么样,“我念”会产生一种虚假的自我感,让人把“我念”当做自己。而你真实的身份不是“我念”,而是纯粹的觉知,大家可以当下试验下,你主动起个“我念”,然后你可以看着念头,不认同它,不跟随它,这时候你可以很清楚地知道,“我念”不是你,你是那个观察者,那个纯粹的觉知,可以看着念头而不卷入其中。

\begin{quote}\it
    由认识贼到打倒贼、清除贼,最后贼也归顺了,就是所说的“全妄归真”。在这个贼当家的时候,什么荒唐的事情都可能干得出来,它是贼嘛,贼当家,不可能出好主意。应知这个心是贼,贼给你出不了好主意。这个是贼呀!那是认贼作子,由于认贼,那人修行就是煮沙做饭。自胜者,就是胜过了这个贼。所以《四十二章经》:“慎勿信汝意,汝意不可信。”你现在咱们这个当家作主的是妄心。你认贼作子,大家都是在妄心用事。在为贼做奴婢,所以这可是可怜可悯者。就是在这,我们自个的这个妙明真心,就好像在运动中靠边站了。他也没有什么,但是他靠边站,没有发言权呀!这发号施令的是你的敌人。你不知道,这个贼,这个强盗,这个敌人,你不知道,他正是你自己妄心,所以咱们可怜。就迷,迷在此处。认贼作子,相信妄心,拿贼当儿子,最后绝对不能成就。你认贼为子呀,你以为是你自己的心哪,结果是什么呢,那是贼呀,那是害你的贼呀!(黄念祖老居士)
\end{quote}

黄念祖老居士这段话谈得太好了,太恳切了,太一针见血,也太到位了,真的谈到了关键处,谈到了骨子里。两位在家大德,一南一北,上海元音老人和北京黄念祖老居士,这两位在家大德真的让我五体投地,泪流满面,真的是太慈悲,太伟大了。得遇恩师,无上福报!

黑客帝国最经典的台词:That you are a slave,Neo. Like everyone else you were born into bondage. Born into a prison that you cannot smell or taste or touch. A prison for your mind.(尼奥,你是一个奴隶。和其他人一样,你生来就被奴役。你生下来就是监狱,你闻不到、尝不到、摸不到的监狱,你头脑的监狱。)这里的 mind 翻作头脑,指的就是思维、念头、心智。而头脑本身也是一个多义词,有时指念头,有时指两耳之间的空间,有时也指思考能力、脑筋、头绪等,要看具体语境来判断。

Mind 既是工具又是监狱,那些大德也要起念头做事,但他们是念头的主人,念头不会束缚他们,对他们而言,念头是工具。而凡夫则把念头当做自己,跟着念头跑,这样就会造成很大的困扰和束缚,这就是大德说的“作茧自缚”,大家都有这样的体会,脑中的念头喋喋不休,胡思乱想,一会想到这,一会想到那,几乎一刻不停,上个厕所,刷个牙,都被念头牢牢占据着,这就是念头的奴隶,不认识真我,反而认贼作子,跟着念头跑。

小我与心魔的区别,小我的含义更广,泛指所有念头,心魔则指的是负面的念头。念头的确强大,而负面念头则显得更加强大,特别是嗔恨心、嫉妒心、意淫等,会带来非常强烈的内心感受。念头虽然这么强大,但你能说清念头是什么东西吗?大家可以问下自己这个问题,你已经起过无数个念头了,但你从来没弄清念头是什么东西,念头甚至谈不上是一个东西,它是脑海中一种很飘忽的感觉,一会出现了,一会又消失了,就像幽灵一样。人活着需要起念去做事,说话、读书、买东西都需要起念头,看看那些高楼大厦,看看那些高端的科技产品,看看手中的手机,看看神舟七号飞船,看看那些汽车、高铁,是什么造出来的?其实最初都是源自念头,工程师、设计师在那起念头,在那设计,然后最后由工人去建造,建造时也要起念头,需要克服很多困难。念头真的是非常强大,它的能力似乎是无限的,它可以构思出无限种东西,无限种组合,但还有一种东西比念头还要厉害,还要强大,那就是你的——觉察力!也就是观察力、观照力,这是来自真我的力量。觉察力可以消灭念头,这样你就能主宰内心了。

大家应该都看过李世石 VS 阿法狗的海报,还有柯洁 VS 阿法狗升级版的海报,人工智能机器人坐在对面,中间一个棋盘,从更高的修道层次来讲,坐在我们对面的就是 mind system (心智系统),这个系统非常强大,说坐在对面不太准确,因为它就在我们的脑海里,说坐在对面只是一种形容。我们最终都要战胜自己的念头,学会安住于真我,安住于真我一会就会产生妙不可言的感受,虽然刚开始有点无聊和平淡,但随着安住的程度加深,就能体会到了。Mind system 是对真我的一种限制、遮蔽和压制,我们要战胜这个心智系统,做回念头的主人。有的戒友说他已经基本能降伏心魔了,但其他妄念层出不穷也令他感到烦恼,心魔是负面的念头,能够降伏心魔,其他的妄念也容易降伏,降伏的原理是相通的,最终你会发现,所有的念头,不管恶念还是善念都要断除,这样才能契入真我,就像六祖慧能大师说的“不思善、不思恶”,动念即乖,举心即错。我们可以利用念头做事,但不要被念头束缚住,做完事就安住纯粹的觉知。

下面开始分享精华笔记,附上我的解析。

\subsubsection{《当下的力量》(一)}

\begin{quote}\it
    后来,常常有人走过来对我说:“我要你所拥有的东西,你能把它给我吗?或者告诉我怎么得到它?”我说:“你已经有了,只是你感觉不到它的存在而已,因为你的思维产生了太多的噪声。”
\end{quote}

\subparagraph{解析} 真我本来就存在,不需要别人给你,只需要别人告诉你,指示给你,而你有足够的勇气来承当,就像托利说的,因为思维产生了太多的噪声,所以感受不到那个“宁静的”。

\begin{quote}\it
    曾经,有个乞丐在路边坐了三十多年。一天,一位陌生人经过。这个乞丐机械地举起他的旧棒球帽,喃喃地说:“给点儿吧。”陌生人说:“我没有什么东西可以给你。”然后他问:“你坐着的是什么?”乞丐回答说:“什么都没有,只是一个旧箱子而已,自从我有记忆以来,我就一直坐在它上面。”陌生人问:“你曾经打开过箱子吗?”“没有。”乞丐说,“有什么用?里面什么都没有。”陌生人坚持:“打开箱子看一看。”乞丐这才试着打开箱子。这时令人意想不到的事情发生了,乞丐充满了惊奇与狂喜:箱子里装满了金子。
\end{quote}

这个故事是《当下的力量》里的,很有名的一个故事,类似的故事有不少,总之就是告诉你,你已经拥有了宝藏,那个宝藏就是真我,关键就是认出它。

\begin{quote}\it
    你将会了解到如何从思维的枷锁中解放出来,进入一个开悟的意识状态,并理解如何在你的生活中维持这种状态。
\end{quote}

\subparagraph{解析} 你是“那个思考者后面的知晓者”,你看见念头,你知晓念头,你觉察念头,但你不是念头,你是觉知,纯粹的觉知。思维是枷锁,觉察能切开这个枷锁,彻底释放你,让你感受到真正的自由。

\begin{quote}\it
    你不等于你的大脑(You are not your mind.)
\end{quote}

\subparagraph{解析} 这是《当下的力量》中振聋发聩的一句话,可谓狮子吼,唤醒群迷,mind 翻作大脑不太准确,但也还行,翻作头脑更好一些。

托利:“最为关键的一步是:从对思维的认同中摆脱出来。”这条不是很容易做到的,因为一般人都在认同思维为自己,那个认同相当坚固,很多人即使当面听到了大德的开示,也很难转变过来,就是因为那个认同太坚固,太顽固了,有的人一辈子都转不过来。

托利:“认同思考,就是说,你从思考的内容和活动中获取自我的感觉,因为你认为,如果你停止思维活动,你将不复存在。”当你停止思维活动了,其实你还是存在,而且这种存在更加纯粹。纠正思想误区,就能从错误的认同中摆脱出来。

托利:“只有停止思维认同,你的思维才会丧失它的力量,本体才会以你原来的本性显露出来。”当你不再认同思维了,思维的力量就会减弱很多,这个我是深有体会的,不再认同它,而是观察它。

托利:“你越是认同自己的思维,你就越感到痛苦。或者可以这样说:你越是接受当下,你受的苦就越少,也越能从小我思维中解脱出来。”认同思维,跟着念头跑,最后肯定会失调,肯定会感受到痛苦。

认清思维不是自己,这点太关键了,这个关键点如果能认清,那个最坚固的错误认同就被打破了,一个全新的你就诞生了。

\begin{quote}\it
    本体是超越那些受限于生死的各种生命形式而永在的“至一生命”(One Life)。作为无形的、不灭的本质,本体不仅超越了所有生命形式,更深深地根植于每种生命形式之中。也就是说,作为你最深的自我和真实的本质,你可以在每个当下接触到它。别试着去掌握它的含义,别试着去理解它。只有当你的思维处于静止时,你才会领会它的真正含义。当你的思维处于静止时,当你的注意力完全集中在当下时刻时,你就会感觉到本体,但是从心智上我们永远无法理解它。
\end{quote}

\subparagraph{解析} 本体就是纯粹的觉知,从心智上是无法理解的,这种状态只能体验,你读到这段文字,当下就可以体验那种纯粹的状态。如果你试着用念头理解它,就偏离了它。

\begin{quote}\it
    认同于你的思维,它使人们进行强迫性的思考。不能停止思考是一个可怕的烦恼,由于几乎每一个人都遭受着此种痛苦,而我们又无法意识到这一点,所以这就成了一件很正常的事情。这种不停的思维活动使你无法达到内心的宁静状态。同时,它创造了一个虚假的自我,不断投射出恐惧和苦难的阴影。
\end{quote}

\subparagraph{解析} 史上最糟糕的认同:认同思维念头是自己,跟着念头跑。不能停止思考的确是一个可怕的烦恼,很多人都没意识到这是一种痛苦,喋喋不休的头脑,混乱的头脑,胡思乱想的头脑,让人身不由己,如果被负面念头操控,那更加可怕。

\begin{quote}\it
    思维已经变成了一种疾病。当事情失去平衡时,这种疾病就会发生。注意:如果思维被正确利用的话,它将是一个超强的工具;但如果利用不当,它的危害则相当大。准确地说,不是你利用思维的方式不对—基本上你根本没有利用它,而是它在利用你。这就是一种病态。你认为你就是你的思维、你的大脑,其实这只是种幻觉,这个工具已然控制了你。
\end{quote}

\subparagraph{解析} 这段说了思维的两面性,一面是工具,另外一面就是疾病、监狱、束缚、奴役,就像一匹野马,在驯服前,它会带着你疯跑,不受你控制,也像一只疯猴上蹿下跳,不听你指挥,只有驯服后,才能正确利用它。

\begin{quote}\it
    让我这样问你吧:无论何时,当你想从思维中解放出来的时候,你能做到吗?你找到了停止思考的那个按钮吗?
\end{quote}

解析:起念谁都会,但停止就很难了,就像开车一样,想开就开,想停就停,想左转弯就左转弯,想刹车就刹车,一切听你指挥,你要马上停下来,就马上停下来。但对于念头,你却没有这个掌控力,往往被念头带跑,一会东,一会西,停不下来,身不由己。要得到这个主宰力,就要提升自己的修心水平。

\begin{quote}\it
    从思维中解放出来的开始就是认识到你不是一个思考问题的实体——思考者,认识到这一点能使你很好地观察这个思考者。在你观察这个思考者时,一个更高层次的意识就被激活了。
\end{quote}

\subparagraph{解析} 托利说过要“观察思考者”,“倾听你脑袋中的声音并作为一个观察者的临在”。成为思维的观察者,也就是要学会看自己的念头,当你看念头了,你就不是念头了,你和念头之间拉开了一个距离,你可以观察念头而不陷入念流之中。思考者代表念头,而观察者不是指观察的人,而是观察这个行为本身。

\begin{quote}\it
    当一种思维止息时,你会在自己的心智流中体验到一种思维的中断——思维空白。首先,这种空白是短暂的,或许仅几秒钟,但是渐渐地它们会变得长久些。当这种空白出现时,你在内心会感觉到一种静止和宁静的状态。于是你开始感觉到与本体合二为一的自然状态,通常这种状态受到思维的蒙蔽而模糊,多加练习之后,你的这种平和与宁静的感觉会加深。实际上,这种深度是无止境的。你同样会感觉到一种来自你内心深处的喜悦。
\end{quote}

\subparagraph{解析} 当思维停止,本体就显现了,当思维活跃,本体就被遮蔽了。学会安住于本体,只有在必要的时候才使用思维,通过觉察就可以进入本体,随着坚持练习,安住能力会逐渐加强,刚开始念头会一次次把你拉离本体,到后来觉察力变强后,就慢慢稳固了,被拉离的次数会减少很多,你的安住也会变得深邃、深厚,那种深度是无止境的,越深越妙。

\begin{quote}\it
    在这种空白中,你高度警惕,注意力高度集中,但是你没有在思考。
\end{quote}

\subparagraph{解析} 那个空白必须是有觉知的,不是昏迷的,而且是高度的警惕,注意力高度集中,是这样一种状态,思考不见了,但你还存在,纯粹的存在。

\begin{quote}\it
    随着你进一步深入这种“无念”的状态(东方人的说法),你会认识到一种纯意识状态。在这种状态下,你会如此喜悦地感受你的临在:所有的思维、情绪、身体以及整个外部世界,在与你的本体比较之下,都不那么重要了。然而这不是一种自我(selfish)的状态,而是一种无我(selfless)的状态。它超越了你原先认知的自己(yourself)。
\end{quote}

\subparagraph{解析} 记得我第一次听见“无我”这个词,心里真的有点害怕,当时心想:没有我了,那是多么可怕和恐怖啊!后来才知道无我方显真我,无我的我,是小我,假我,不是真我。那个小我是在念头之中,当处于无念的状态,就摆脱了那个小我,从而进入了一种纯意识的状态。托利:“无念是有意识但没有思维。”就是这种状态,这是最标准的定义。

\begin{quote}\it
    每次,当你在思维中创造空白时,你的意识就会变得更强。
\end{quote}

\subparagraph{解析} 每次你这样做,安住于那个空白,安住于那个无念的状态,你的意识就会变得更强,更明亮,那种强度是可以不断加强的。

\begin{quote}\it
    超越你的思维,你的大脑只是一个工具。它是被用来处理特殊任务的,当这个任务完成时,你就让它处于休止状态。因此可以说,人们 80\% ~ 90\% 的思维不仅是重复的,而且是无用的,甚至由于思维的运作障碍和消极的本质,大部分思维都是有害的。如果你观察你的思维,你就会发现这是真的,这还导致了你生命能量的严重损耗。
\end{quote}

\subparagraph{解析} 思维只是一个工具,在有必要的时候才用它,用完了就安住纯粹的觉知,就像手机用完了就让其黑屏。如果念头恶仆凌主,把你带跑,让你陷入胡思乱想,就会导致很大的痛苦。一天中大部分的思维是无意义、无用的,就是胡思乱想,这样也会导致能量损耗,所以我们要学会修心,克服没必要的胡思乱想,学会保存自己的能量。

\begin{quote}\it
    所有真正的艺术家,不管他们是否知道,都是在无念的、内在宁静的状态下进行创作。即使最伟大的科学家都声称他们的创造性突破来自于无念状态。对美国最著名的科学家(包括爱因斯坦在内)的调查令人吃惊,调查结果显示,“在那个短暂的、决定性的创造本身的过程中,思维只起到了小部分的作用”。所以我可以说,绝大部分人不具有创造性,不是因为他们不懂得如何去使用思维,而是他们不懂得如何停止思维。
\end{quote}

\subparagraph{解析} 这段话的后两句一针见血,不是不懂得如何去使用思维,而是不懂得如何停止思维,被思维的野马带跑,根本停不下来,这是一种奴役和束缚。修心就是要主宰内心,做念头的主人,学会停止自己的思维,该用时就用,该停时就停,自己要有绝对的主宰权!真正的艺术家和科学家都会有意无意地安住于无念的状态,无念的状态是灵感爆发之源泉,安住一小会,灵感自动就涌现了。

\begin{quote}\it
    在你的意识之光下,这种无意识的模式很快就会投降了。
\end{quote}

\subparagraph{解析} 这段其实讲的就是断念实战,也就是断念口诀的最后四个字:觉之即无。保持觉察,无意识的跟念模式很快就会投降,关键是要发展觉察力的强度,激光可以射穿物体,而手电筒却做不到,因为激光有强度,强度具备了,自然就能做到。

\begin{quote}\it
    真正的力量是在你的内心深处,而此刻你就已经拥有了它。
\end{quote}

\subparagraph{解析} 真正的力量就是觉察力,观照力,这种力量可以切断念头造成的束缚,每个人都拥有这个力量,这是比念头更为强大的力量,要学会强化这个力量。

\begin{quote}\it
    任何时候,当你有能力观察你的思维时,你就不会再落入它的陷阱,这时,另外一个不属于思维的东西就来临了:观察者的临在。
\end{quote}

\subparagraph{解析} 托利说“更为有力量的东西”——宁静的观察者。观察思维,不陷入念流,观心既是基本功,也是最终极的功夫,修心从学会观心开始。

\begin{quote}\it
    通过练习,你自我观察的能力以及监控你内在状态的力量将会得到加强。
\end{quote}

\subparagraph{解析} 上次看到一位戒友说:“最近对断念上了一个台阶,对念头有了更高的敏感。”通过一段时间的练习,他的觉察力变强了,开始出敏感了,这就是练习带来的进步。练习使人强大,苏炳添能够在 2018 年爆发,就是因为他的训练强度和训练质量上了一个全新的台阶,比以往加强了许多,并且很注重训练后的恢复,恢复好才能扛得住高强度的训练负荷。国家队的训练非常科学、专业和系统,在这种训练体系下出的人才,必定具备很强的实力。

\begin{quote}\it
    你的注意力一旦放松,思维就会趁虚而入。
\end{quote}

\subparagraph{解析} 乘虚而入,这个虚就是注意力涣散了、分心了、走神了,注意力一弱,念头就进来了,就像门卫保安打瞌睡了,小偷强盗就进去了。我在实战中的体会,就是注意力稍微一松懈,也就零点几秒,一下就切换到了无意识的跟念模式,所以要求很高的警惕性,还有就是反应要快,觉察要快,实战中慢一点,就会被念头带出去老远。

\begin{quote}\it
    无论何时,当你观察你的思维,你会把意识从你的思维形式中撤离。结果,观察者——超越形式的纯意识——变得更为强大,而思维的形式结构则变弱了。
\end{quote}

\subparagraph{解析} 这段话很好,说到底还是要观察,要看住念头,不要无意识地跟着念头跑,很多人跑了一大段才反应过来,这时已经有点迟了,开始欲火中烧,有煎熬感了。你越能观察自己的念头,就越能摆脱那种跟念跑的模式,就越有主宰权。观察者越强大,那个无意识的思维模式就会越弱,所以要不断提升自己的觉察力。

\begin{quote}\it
    你必须遭受一定深度的痛苦或损失才会被灵性世界吸引。
\end{quote}

\subparagraph{解析} 痛苦虽然难受,但也是转变和提升的契机,沉迷于欲望是活在“鸡粪层”,而灵性的世界则是充满光辉和辉煌,做回最纯净最纯粹的自己,才能真正快乐起来。纯粹的觉知是每个人内在的神性,要活出它。不要开发自己的兽性,那样会带来很大的痛苦。

\begin{quote}\it
    当你改变了,你的整个世界就改变了,因为世界只是你内在的反映。
\end{quote}

\subparagraph{解析} 外在的世界是一个全息投影,一张全息图,根源就是你的内在。你可以仔细体会下自己的起心动念带来的外在改变,记住那个微妙的过程,当你断恶修善,提升了自己的振动频率,这时候你的人生境遇就会大为改变。巴夏说过:“你在现实世界中的一切体验,都是你的投射,都是你的信念系统、情绪、思维模式、行为方式的投射,都源于你内在所创建的振频。物理现实就是如此精确的全息构架,你存在于哪个振频,你选择哪个振动状态,你就会进入哪个实相。”

\subsubsection{《当下的力量》(二)}

\begin{quote}\it
    人类的现状:迷失在思考之中
\end{quote}

\subparagraph{解析} 认同念头为自己,跟着念头跑,就会迷失在思考中,如果认识到念头只是工具,真我是纯粹的觉知,并且学会观心断念,这样才有望出离思考的迷宫。

\begin{quote}\it
    大多数人一辈子禁锢在自己的思维牢狱里,他们从未超越那受制于过去的、狭隘的、源于大脑的自我感觉。
\end{quote}

\subparagraph{解析} 这就是可悲的现实,很多人一辈子都无法认识到思维是一个牢狱,一直把牢狱当成自己,认贼为子,非常可悲的状态。

\begin{quote}\it
    进入那个比思想更为深邃的意识层面。
\end{quote}

\subparagraph{解析} 纯粹的意识非常深邃,思想与之相比显得非常浅薄,思想就像一只异常活跃的疯猴(crazy monkey),修行方面专门把思想比作疯猴,需要觉察的雄狮来震慑这只疯猴。

\begin{quote}\it
    当你发觉头脑里有一个声音总是假扮你,而且喋喋不休时,你便从对思考的无意识认同中苏醒了过来。
\end{quote}

\subparagraph{解析} 发现这个真相,那就是那个“我念”一直在冒充你,假扮你,喋喋不休,胡思乱想,通过这种方式来压制你的真我,限制纯意识。纯意识本来是无限的,但被喋喋不休的思想压制在非常有限狭隘的范畴内。

\begin{quote}\it
    真正珍贵的是你的本体,你心底最深处的“我是”,也就是意识本身。
\end{quote}

\subparagraph{解析} 一亿吨黄金也比不上你的本体,本体打开后遍一切处,整个宇宙都是你。

\subsubsection{《修炼当下的力量》}

\begin{quote}\it
    初步的自由解脱,就是了解到你不是那个思考者。在你开始观察那个思考者的那一刻,就启动了意识的更高层次。然后你就会了解:有一个智性的广大领域是超越思想的,思想只是其中极小的一个面向。你同时也了解到,所有真正重要的事物,如美丽、真爱、创造力、喜悦、内在平安,都是超越心智而生的。这时,你就开始觉醒了。
\end{quote}

\subparagraph{解析} 《大乘心地观经》:“汝等凡夫,不观自心,是故漂流生死海中。”又说:“能观心者,究竟解脱,不能观者,究竟沉沦。”徐恒志老居士说过:“要行住坐卧,时时警惕,刻刻不离观照,方能成就。”所有的大德都在强调观心,观察那个思考者,不要认同思维。

\begin{quote}\it
    阻止我们体验这个连结的最大障碍就是与心智的认同,因而造成强迫性的思考。无法停止思考是个可怕的折磨,但我们无法意识到这点,因为几乎所有人都在为此受苦,所以大家都以为这是理所当然的。没完没了的心智噪音阻止你找到那份与本体无法分离的内在定静,也创造了由心智制造的虚假自我,投射出恐惧和苦难的阴影。
\end{quote}

\subparagraph{解析} 这条说到了问题的根源,认同心智,跟着念头跑,深陷念流之中,无法停止思考是个可怕的折磨。特别是被负面思想纠缠,那更是会导致悲剧的结果,几乎所有人都在受思想的苦,但很少有人意识到这一点。

\begin{quote}\it
    当一个思想销声匿迹时,你会经历到心智续流中的一个间断——无念的间隙。起初,这个间隙可能很短,也许几秒钟而已,但是它们会逐渐变长。当这些间隙发生时,你会感到内在有某种定静和平安。你开始感觉到你与本体合一的自然状态,这种状态通常会被心智遮蔽起来。随着不断练习,定静和平安的感受会加深。事实上,它的深度没有尽头。你也会感受到一种内在深处散发出来的微妙喜悦,那就是本体的喜悦。
\end{quote}

\subparagraph{解析} 思想本质上是对真我的一种遮蔽,就像乌云遮蔽太阳。但思想也是一种工具,在必要时要使用它,要有觉知地思考,就像开车时完全在你掌控之中,不能让思想带着你跑。思考分两种,一种有觉知地思考,一种无意识地跟着跑。第一种是你在使用工具,第二种简直就是监狱、束缚和奴役。无念的间隙就在两念之间,这也是丁愚仁老师经常强调的要点,好好深入体会这个间隙(gap),明就仁波切也有提到这个 gap,这个 gap 价值无量。

\begin{quote}\it
    我们可以称这个虚假的自我为小我(ego),它由心智活动组成,而且只能经由不断地思考而存活。
\end{quote}

\subparagraph{解析} 小我由念头组成,当你处于无念状态,它就像幽灵一样消失了,剩下的就是真我。那个“我念”会产生一种虚假的自我感,是最贼的冒充。

\begin{quote}\it
    开悟意味着超越思想。在开悟的状态中,必要时你还是会用心智思考,但会比以前更专注、更有效率。你大部分是为了实际用途而使用心智,而且你摆脱了不由自主的内在对话,因而感受到内在的定静。
\end{quote}

\subparagraph{解析} 不由自主的内在对话,这种情况大家都有体验过,脑袋里喋喋不休,充满了各种念头,特别是评判的念头,这个好,那个坏,总是在评判。大德特别强调不要评判,只要静默,静默就是唯一的答案。生活中必要时使用心智,使用完了记得回归于静默,安住于纯粹的觉知。

\begin{quote}\it
    在你的意识之光中,无意识的模式很快就会瓦解。
\end{quote}

\subparagraph{解析} 这条和《当下的力量》中的一条说得类似,那条是投降,这条是瓦解。最关键的一点就是开启觉知,开启你的意识,去观察你的念头,这样就能瓦解那个无意识跟念跑的模式。

\begin{quote}\it
    在你能够稳定地在临在状态中安住下来之前——也就是说,在你能够全然地有意识之前——你会有一阵子在意识与无意识之间、临在状态与心智认同状态之间来来回回转换。你会一次又一次地失去当下,然后又转回去。最终,临在会成为你主要的状态。
\end{quote}

\subparagraph{解析} 这是修行必须要经历的过程,会有一个拉锯战,就像拔河一样,双方你来我往,当你的实力更强大了,就开始一边倒了。刚开始会一次次被拉离本体,然后你一次次回来,然后渐渐地,随着你的觉察力增强,被拉离的次数就减少了,安住能力在变强,就像太极推手一样,功夫深厚了,别人就推不动了。大力士龙武推太极宗师陈小旺,推不动,因为龙武在陈小旺身上找不到发力点,在发力的一瞬间就会滑掉,所以就算有再大的力气也使不上。关键还是要功夫深,久练自化,熟极自神,观心断念要练到出神入化的地步,进入化境的层次。

\begin{quote}\it
    试着把更多意识带进你的生活。这样一来,你临在的力量就增加了。它会在你的内在和周围创造一个高振动频率的能量场,无意识、负面性、不和谐或暴力都无法进入这个能量场并存活,就像黑暗无法在临在之光中生存一样。
\end{quote}

\subparagraph{解析} 临在的力量增强了,也就是觉察力、觉知力增强了,这样你的振动频率就会有一个大的跃升,所有负面的思维模式都会被大大削弱,逐渐瓦解。

\begin{quote}\it
    每当需要一个答案、一个解决方案,或一个创意时,停止思考一会儿,并把注意力放在你内在的能量场,觉察那份内在的定静。当你重新开始思考,你的思考就会变得鲜活而有创意。
\end{quote}

\subparagraph{解析} 这是一个获得创意的秘诀,很多艺术家、设计人士、科学家,他们在获得好创意时,往往是在一小会的无念状态之后,突然就有灵感了。真正的智能是源头智能,源头智能就是你的真我,就是纯意识,纯粹的觉知,学会安住它,它的力量它的创意是无限的。

\begin{quote}\it
    你愈是认同自己的心智,就愈是受苦。
\end{quote}

\subparagraph{解析} 背离真我就会导致受苦,事实上,背离真我就是最大的痛苦,心智最终会导致失调和痛苦。很多千万富翁和位高权重的人他们的内心也痛苦,认同心智,背离真我,最后的结果就是痛苦,只有活出真我才能真正幸福起来,真正快乐起来。不少有钱人都说过:“为什么我这么有钱,但还是那么不快乐,内心还是很痛苦。”因为他们背离了真我,背离真我就要受苦。

\begin{quote}\it
    保持临在,保持有意识,时时警觉地守护自己的内在空间。
\end{quote}

\subparagraph{解析} 要时时警觉,保持警惕,防止被念带跑,陷入无意识的跟念模式。保安——保护真我,安住本性;保洁——保护真我,清洁内心。这两个工种从灵性的角度来讲,有着特殊的含义。有的戒友会说时时警惕是不是活着太累了,他把警惕误解为了过度紧张,其实不然,就像过马路,虽然警惕但也相对放松,开车也一样,保持一定的警惕但也比较放松,是这样一种状态。

\begin{quote}\it
    所谓自由、救赎和开悟,就是知道你自己是思考者之下的本体、心智噪音下的定静,以及痛苦之下的爱和喜悦。
\end{quote}

\subparagraph{解析} 这段写得好。你不是思考者,你是观察者,你可以观察思考者,你是思考者之下的本体,心智是一种噪音,噪音消失后,那是什么?

\begin{quote}\it
    当你臣服于当下本然因而全然临在时,过去就失去了它的力量。被心智所遮蔽的本体范畴就会开启。突然之间,你的内在会升起极大的定静——一种深不可测的平安感。在那平安之中,有着极大的喜悦。在那喜悦之中,有着爱。而在最深处的核心,是神圣的、不可测量的、无以名之的“那个”(that)。
\end{quote}

\subparagraph{解析} 本体就是纯粹的、无条件的爱,本体是一种定静的感觉,非常深厚,一种深不可测的平和感,其中有喜悦涌现,非常开心的感受,就像童年时那种时刻向外冒的新鲜喜悦,看着花花草草感觉特美好,好幸福的感觉。

\begin{quote}\it
    你必须在某个层面上受到重挫,或是经历一些重大的损失或痛苦,才会被灵性的向度吸引。再不然,就是你的成功让你感觉空虚、没有意义,反而让你受挫,你才会走向灵性。
\end{quote}

\subparagraph{解析} 这条《当下的力量》也讲到了,不过这条讲得更扩展些,一个人必须遭受痛苦折磨,受到重挫,才可能反省人生,否则他在那条轨道上是很难回头的。另外成功后也可能感到空虚,没有意义,成功时虽然伴随着喜悦感和成就感,但成功者也会感受到某种失落,因为一个目标完成了,还有下一个目标等待他去追寻,而这种追寻几乎是无止境的。世俗的成功的确是这样的,并不能提升你的灵性,只会让你陷入更大的欲望之中,赚了一千万,就想着五千万,赚了五千万就想着一亿,赚了一亿就想着十亿,赚了十亿就想着一百亿,这个欲望是无止境的。即使获得了再大的成功,还有某种不满足,为什么?因为真正的满足来自于活出真我,而不是追寻外在的一切,真我是内在的,它等待你去发现它,活出它。

\begin{quote}\it
    超越了心智创造的对立面之后,你就会成为一座深不可测的湖。你生命的外在情境(生命中发生的事),就是湖水的表面,随着周期和季节不同,有时风平浪静,有时波涛汹涌。然而在湖的最深处,永远平静无波。你就是这整座湖,而不只是湖面,你与自己内在的最深处相连,那儿永远是绝对的宁静。
\end{quote}

\subparagraph{解析} 本体是深不可测的湖,也有把本体形容为大海的,而心智只是波浪,在大海深处是平静无波的,非常宁静,非常深沉,在大海表面会有各种波浪。

\begin{quote}\it
    小我是很狡猾的,所以你必须提高警觉,保持高度临在。
\end{quote}

\subparagraph{解析} 小我的确非常狡猾,有一袋子的诡计把你拉出临在状态,所以要警惕,看住自己的念头,严格做到念起即断。

\begin{quote}\it
    去感受,而不是去思考!
\end{quote}

\subparagraph{解析} 这条简洁有力,直戳要点,临在状态是去感受,而不是去思考,比如当你仰望星空,你的思维停止了,你被震撼到了,在那个瞬间只是纯粹的觉知,纯粹的感受,就是那种状态。觉知是非常非常单纯的东西,一旦那种状态被思考遮蔽了,就变得不单纯了,各种妄念把你缠缚起来。记得看过一个视频,托利对一个年轻人说:“那个觉知,就是真正的你。”那个年轻人迷茫的表情把我逗乐了,托利颇有点对牛弹琴的感觉,普通人顽固地认同念头和身体为自己,分不清觉知和念头的区别,即使大师去点,也很难把他立即点开,可能需要一段时间才能慢慢觉悟和转变过来。

\paragraph{总结}

托利在《宁静在说话》(Stillness Speaks)里讲到:“一旦你意识到在你的脑袋里有一个声音,它总是假装是你,从来都不停止说话,那你就正在从对持续思考的无意识认同之中苏醒。当你留意那个声音,你会发现你并不是那个声音,那个思考者,而是能够觉知到声音的那个。发现你自己是那声音背后的意识,就是自由。”认清思维不是自己,认清“我念”在冒充你,不再认同念头,而是开始观察念头,这就是极大的进步。很多大德都提到了这个“贼”,这个一直在冒充你的贼,首先要认清这个贼,然后要降伏这个贼,阿罗汉有一个名字就叫“杀贼”,杀内心之贼。普通人都在认贼作子,跟着贼跑,这个贼是带跑高手,稍不警惕,就会被它带跑,陷入无意识的跟念模式。

当年我读《当下的力量》,读得很入迷,划了很多重点,当读到一个词,我心中有了微妙的感觉,那是来自小我的微妙抗拒,那个词就是“臣服”,大家看到这个词,是不是内心也有着某种微妙的抗拒,小我会说:“我为什么要臣服,我就不想臣服。”会有这种微妙的抗拒,其实臣服的真正含义不是叫你去跪下向某个权威臣服,也不是失败或认输,臣服的真意是臣服于你的真我,成为真我。小我很讨厌臣服,臣服意味着小我的消失,所以会有这种抗拒。当年我看到“无我”感到害怕,看到“臣服”感到抗拒,当我觉悟提升后,有了正知正见,就完全没了害怕和抗拒,害怕和抗拒往往是由于误解导致的。《当下的力量》这本书很不错,可以作为戒色后提升灵性的第一读本,很多戒友都在看这本书,也在推荐这本书,这本书可以帮助你提升到更高的认识层次,一般的戒色文章是不会讲到这个层次的,因为这需要极高的认识水平,而《当下的力量》专门讲这个层次,所以很值得学习与研读。不少人第一次读到《当下的力量》,感觉似懂非懂,看得糊里糊涂,虽然接触到了这本书,但时机未到,所以很难看懂。当达到一定的认识水平,对修心有了较深的体悟后,再度捧起这本书,就会发现字字珠玑,这时才能真正看懂。为何前辈特别强调复习?因为很多内容要看到一定程度才能真正看懂,是需要一定的积累才能达到成熟的时机,通过反复学习,终有一天会真正看进去真正看明白的,到时就会拍案叫绝,兴奋异常,不禁感叹:“之前都理解错了,没抓住重点,现在才真正看懂,原来是这么一回事。”

戒色不仅仅是戒掉手淫,而是全面的改造,整体的提升,并且有着更高的意义。也许你通过戒色养生获得了良好的精力和脑力状态,让你在学业或事业上有良好的表现;也许你通过戒色养生恢复了身心健康,不再经常往医院跑,不再是药罐子,不再被病痛折磨;也许你通过戒色养生提升了自己的精气神,找到了一位满意的眷属,有了自己的孩子,人生也比较幸福。这些的确是戒色带来的好处,但是没有认识真我,没有活出真我,你的内心总会感觉少了些什么,那个最高的价值和目的没有实现,即使让你拥有全世界的财富,你也会感觉到某种不满足,一定要实现真我才能带来真正彻底的满足。真我是永恒的,只有实现那个永恒,你才会真正满足,否则就会有无止境的追寻,你一直在追这追那,追到了还是不满足,其实你真正在追的是真我,只不过投射到了错误的对象上去,真我也不需要追,只需要发现,它本来就在你内心,回过头来认出它即可。在射精的一刹那,你处于无念的状态,你真正渴望的是无念的真我,而不是快感,我们可以通过觉察进入真我,根本无需手淫,手淫伤身败德,极易上瘾,必须要彻底戒掉。人们疯狂地追求性,其实就是对真我的一种渴望或回归,只是采取了错误的方式。很多人即使认识了真我,也无法安住,因为断念不行,一直被念头带跑,所以仅仅认识真我还不行,还必须修炼出强大的断力,这样才能稳固地安住于真我。

真我其实是非常非常简单的,真的是大道至简,因为顽固地认同于念头,所以才看不到那个简单的真我。十七世大宝法王:“最终我们会回归到那个一开始让我们感到无聊的状态,然后发现原来这个就是它。所谓的认出心的真正本质,意思就是回到你的那个初始的既无聊又不吸引人,既不高深也不新奇的心理状态,同时能够真正地认出———它,就是你一直在寻找的。为什么我们还是无法认出自心的真正本质呢?是因为它太深奥、太困难了,还是它太简单、太容易了呢?过去的禅修大师们曾经说过,我们之所以无法认出自心的真正本质,不是因为它太高深了,而是它太简单了,简单到令人无法相信的地步。心的真正本质就仅仅是我们当下本来的状态,它是我们无所造作的自然状态。”心的真正本质就是真我,那个纯粹的状态一开始让人感觉有点无聊,很平淡,的确不吸引人,但是随着安住的加深,就能体会到不可思议的深邃,不可思议的真味,不可思议的美妙,那个质感超强烈!人有一个倾向,那就是不相信简单的,喜欢复杂的有挑战的,能够相信简单的人,必须是特别单纯的人,甚至是有点傻的人,世智辩聪的人很难相信,因为他脑子里的那套知识理论已经把他堵死了,一有心机,一有怀疑,一有思考,那肯定不行,那个简单纯粹的状态只能体验,所有的描述就是为了让你实际去体验。

来到这个世界上,能够发现真我就没有白来一趟,我希望我的戒色文章能够把大家带到真我的层面,让大家认识到自己的真我。这是个灵性大觉醒的时代,我看到很多大师都在分享真理的讯息,然而这个时代也是一个大堕落的时代,网络上到处都有黄毒,这是最好的时代,也是最坏的时代,觉醒和堕落的机会都非常大。戒色后懂得修心,具备了一定的观心断念能力,已经可以基本主宰内心了,这时候就很容易契入真我。这的确需要一定的修心基础,有了扎实的基础,才可能安住,否则根本无法做到。戒色吧很多戒友已经具备了修心的基础,所以谈这个主题很多戒友都能接受,也能活出真我了。活出真我是戒色最高的层次、最高的境界,也是人生最高的价值、最高的目的,一灯燃千灯,一灯传万灯,元音老人把心灯传给了我,我希望把这盏心灯续传给广大戒友,让大家都能认识真我,活出真我。真我是最伟大的力量,活出它!这是你最深的渴望,也是你最崇高的使命!!!

下面分享一首戒色诗歌。

\begin{poem}[原力觉醒]
    \begin{multicols}{3}
        \centering~\\
        念头是什么 \\ 它是脑海中一种飘忽的感觉 \\ 一会出现,一会又消失了 \\ 就像幽灵一样 \\ 你感知到一个念头 \\ 一幅图像 \\ 或者更微妙的感觉 \\ 当“我念”出现时 \\ 你把它认同为自己 \\ “我念”是一种冒充 \\ 它给你一种虚假的自我感 \\ 当你认清了这一点 \\ 发现了那个狡猾阴险的冒充 \\ 你就不再相信念头了 \\ 而是开始观察念头 \\ 并且安住于纯粹的觉知 \\ 你的真实身份 \\ 你的真我 \\ 这时你就觉醒了 \\ 这是真正原始的力量 \\ 源头的智能 \\ 它比念头更为强大 \\ 它是纯粹的存在 \\ 绝对的存在
    \end{multicols}
\end{poem}

下面推荐一本书。

\begin{book}[《遇见智慧》,秦东魁]
    秦东魁老师以其亲身经历,生动、深刻、细腻地讲述了他和他遇见的故事。他带领我们跳脱身处的环境,用全新的视角来观察、讲述东方智慧,寻找幸福密码。从而让我们认识到:如何做到与子女和,与财富和,与天地和,遇见智慧,遇见新的自己。书中讲到:“人生祸福的核心:了解念头,观察念头,控制念头”、“念头是行为的先导,所以自己能控制念头,就能掌握自己的人生”,学会断念、控制自己的念头真的很重要,人生祸福的核心就在于念头,你的念头是好还是坏?是正能量还是负能量?我们一定要学会修心,断除邪念,多发善念,强化自己的正能量和责任感。书中还讲到:“我们平时做事情,要用良心对待所有人,起心动念都想着要符合天道。”这句话说得很好,起心动念要很小心,一定要符合天道。秦老师的书籍和视频讲座我都有推荐,的确是一位不可多得的良师益友,希望大家好好学习秦老师的教诲。
\end{book}
