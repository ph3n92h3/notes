\subsection{戒断反应、JJ 偏向问题、精液颜色改变}

\subsubsection{戒断反应}

戒断反应我以前的文章有讲过,最近问戒断反应的新人很多,我有一种恍惚的感觉,时空错位的感觉,戒色吧已经进入了轮回,不断有新人加入进来,问的问题都是前辈问过的问题,而以前的新人已经变成了老戒友,老人带新人,这种局面已经出现了。关于戒断反应,这季再好好讲讲,以便消除新人的顾虑,坚定他们的戒色信心,我想还是很有必要的。

戒色后,在思想方面,要过的第一关就是无害论!而在身体方面,要过的第一关就是戒断反应。绝大多数戒友在刚开始戒色后的一个月内都会出现戒断反应,没戒色前症状并不明显,戒色后反而出症状了,一出症状很多人就慌了,马上想到了禁欲有害,马上想到了禁欲对前列腺不好,结果又掉进了 SY 陷阱。其实这就是戒断反应,一般坚持戒色,戒断反应会逐渐缓解乃至消失的。如果新人不知道戒断反应,不了解戒断反应,那就很可能会起退心了。因为他慌了,他怕了,的确,被戒断症状缠着的感受不好过,但只要坚持,戒断反应会过去的。出现戒断反应时,要注意休养,不要太劳累,可以做些适量的锻炼。一般过了戒断反应,身体恢复会越来越好的,当然前提是你要学会养生之道,积极锻炼,这样恢复才有保障,不能老坐着,那样恢复进度会很慢的。

我研究过很多成瘾的行为,一般成瘾的行为戒断后,都会出点症状,戒断反应在瘾界比较普遍。

比如戒烟:吸烟者在强制戒烟之后也可能会出现诸如焦躁不安、心急、胸闷、咳嗽、短暂健忘、无精神、发胖、发抖、失眠、食欲增强、吐黑灰色痰、血压升高以及心律不齐等戒断反应,会产生极大的痛苦。但是这种反应大多数会随着体质恢复逐渐消失。

再比如戒酒:戒酒后精神症状表现为焦虑、抑郁、易激惹,重者出现幻觉、错觉、妄想、意识障碍等;神经系统症状表现为心悸、胸闷、大汗淋漓等植物神经症状,重者可有震颤、抽搐、癫痫样发作等;胃肠道症状表现为恶心、呕吐、腹痛腹泻等。

戒断反应属于适应性反跳,一般坚持戒色,戒断症状会逐渐消失的。一般戒色后的戒断反应表现如下:

\begin{description}
    \item[情绪障碍] 情绪不佳,急躁易怒,颓废悲观,兴趣丧失。
    \item[睡眠障碍] 戒断后反映睡眠障碍的戒友很多。以入睡困难,失眠多梦比较多见。
    \item[躯体症状] 以泌尿系统疾病为常见,比如前列腺炎加重。
\end{description}

还有的戒友会出现腰酸腰痛,全身无力,浑身难受,晨勃消失、脱发增多、精液改变等。

出现戒断反应不要慌,不要怕,好好坚持戒色,注意休养,很快就过去了,这是一道坎,大家都是这样过来的。

\subsubsection{JJ 偏向问题}

下面谈下 JJ 偏向问题。

我聊过上千个戒友,其中反映 JJ 偏向的不多,但时不时会冒出来几个,这个问题的确不容小觑,会影响到一个人的自信。大家先看两个案例:

\begin{case}[JJ 偏向问题]
    男性,25 岁,未婚,无性生活,阴茎弯曲偏左,弯曲角度很大,好几年了,因为不好意思就一直没到医院治疗。
\end{case}

\begin{case}[JJ 偏向问题]
    飞翔大哥,我 JJ 向左弯 20 \unit{\degree} 左右,尿线向左偏一点,都是原来 SY 用右手造成的,怎么办?
\end{case}

我遇见的戒友中有 JJ 偏左的,也有 JJ 偏右的,有的甚至偏右 45 \unit{\degree},国外有一篇戒色文章就说到,SY 会导致 JJ 发生偏向问题,但也要看具体情况而言,因为其他很多人也有 SY 恶习,但他们并未出现这个问题,我想 SY 应该是一个诱因,出现 JJ 偏向应该是很多因素共同作用才会出现这种情况。比如你在 JJ 发育时撸管,撸管方向偏左或者偏右,或者摩擦时习惯往一个方向用力,在 JJ 发育时这样很可能就会影响到 JJ 的生长方向,就像一棵小树,在它还小时把它弄弯,结果会怎样呢?它就会顺着歪劲长,长大了也是歪的,而大树则没这个问题,因为大树的生长方向已经基本定型了。

出现这个问题如何治疗,那要看是不是影响你的自信,还有是否会影响到将来的性生活,如果影响到了,一般医生会建议做矫正手术的。所以有这方面问题的戒友,应该及时就医治疗,听取医生的专业建议。同时,不要再 SY 了,好好养身体,SY 导致的症状太多了。等出了症状,其实有点晚了,中医讲究治未病,我们应该防患于未然!

\subsubsection{精液颜色}

最后谈下精液颜色。

一般 SY 伤了肾气以后,精液也会随之发生改变,反映精液改变的极多,应该是人人都会遇见的一个问题。一般会发生如下几种情况:

\begin{adjustwidth}{-1.5em}{-1.5em}
    \begin{multicols}{2}
        \begin{itemize}
            \item 颜色的改变,变得很黄或者黄绿色
            \item 精不液化,出现结晶体或者果冻状
            \item 血精,出现血精的原因很多,以炎症和结石较常见
            \item 精液偏稀薄偏少,黏稠度下降
            \item 精液排出过多过浓,也属病态,提示炎症的存在
            \item 精子质量下降(无精、死精、活力低、畸形等)
        \end{itemize}
    \end{multicols}
\end{adjustwidth}

我看过的文章,提到精液的正常颜色,一般有两种说法:

\begin{multicols}{2}
    \begin{itemize}
        \item 乳白色或者淡黄色
        \item 灰白色或者淡黄色
    \end{itemize}
\end{multicols}

如果严格地说,乳白色也不正常,提示炎症的可能性。从大家的提问来看,出现果冻状和结晶体比较常见,其次就是变黄,如果是淡黄色,是正常的,如果变得很黄,或者出现黄绿色了,那就要引起重视了。还有一些戒友会出现血精的现象,精发红或者带粉红,出现血精的情况建议及时就医检查治疗。

刚开始戒色,精液是容易出现改变的,可以看作戒断反应。随着坚持戒色养生,精液的质量会慢慢好起来的,还有就是不注意养生也容易出现精液的改变,比如酗酒抽烟,久坐熬夜,吹空调等。养成良好的生活习惯是非常重要的,对健康有直接的影响。

中医理论有讲到:“种子之法,男必先养其精。”要想优生优育,男人应该好好保养身体,该戒的都戒掉,把精子质量养好,这样生出来的孩子才会比较健康。很多戒友有去做精子检查,有部分戒友出现了弱精症,出现无精症的也有,还有的就是精子活力不行,死精,畸形的不少。一般导致精子质量不行的主要原因,就是前列腺炎或者精索静脉曲张,而 SY 是可以导致这两个疾病的,然后就会影响到你精子的质量。很多人沉迷 SY,把精子质量搞得很差,有的出现了不孕不育,有的虽然还有怀孕的功能,但是继续 SY 下去,弄不好也会出现不孕不育。所以,戒掉 SY 是有百利无一害。不戒掉,有百害无一利。如果真要说 SY 有什么好处的话,那就是可以发泄你的情绪和压力,可以让你暂时放松下来,但结果就是陷入恶性循环,得不偿失。

希望大家能好好坚持戒色养生,把精子质量养好,把最好的自己留到结婚后,千万不要还没结婚,就把自己给废了。精就是种子,种子好,将来你的后代才会更健康。如果种子不好,你的后代就可能体质不佳,出现多病甚至夭折的可能。

\paragraph*{后记}

在回帖时,有看到戒友讲到:男女居室,人之大伦。孤阴不生,独阳不长,人道不可废者。这个戒友引用了古代房事养生的理论,有一定道理。但如果阅历不够深厚,没有自己的思考和理解,就会认为何必禁欲呢?不是说孤阴不生,独阳不长吗?禁欲是不是会导致短寿呢?不碰女人,阴阳不调和,那不就是“独阳不长”吗?如果我没有广泛的阅历,也会这样认为,但我深入中医理论,发现这句话并不是绝对的,是有前提条件的。

中医有讲到:精少则病。也讲到:肾水无封藏太过之病,肾水愈能封藏,阳根愈坚固也!大家看到这句话肯定会觉得,这句话不是和“独阳不长”矛盾的吗?我举个案例大家就明白了,虚云法师一生没碰女人,活到了 120 岁,佛门活过百岁的有不少,如果说“独阳不长”,那和尚不碰女人,怎么可能活过百岁?道教修炼,也提到了要完全禁欲,自古没有漏精的神仙。\textit{点滴精液不漏,即是登天行梯。(白玉蟾)}

所以,“独阳不长”并不是绝对的真理,并不适合修道者,对于凡夫还是比较适合的,因为凡夫修心不到位,会有 YY,大家都知道,YY 憋着不好,YY 本身就是暗漏,很多戒友 YY 后就出现了症状的反复,而修道者修心功夫到位,并不会存在这个问题。

不过,凡夫也不能有念头就行房事,还要看自己的身体状态,如果身体虚着,已经症状缠身了,是要禁欲养身体的,等把肾气养足,彻底恢复健康后,才能节制地过性生活。古代很多名医看病,都要求患者远房帏的,就是远离房事,否则身体是万难痊愈的,药的疗效也会大打折扣。现代很多中医不讲了,很遗憾。
