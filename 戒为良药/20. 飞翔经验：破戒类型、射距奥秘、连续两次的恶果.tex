\subsection{破戒类型、射距奥秘、连续两次的恶果}\label{20}

\paragraph*{前言}

最近吧里有帖《光戒是没用的,伤精患者唯一的复原之路在此》,有不少戒友跑过来问我静阳子说得对不对,一看文章内容,很多戒友慌了,不静坐就好不了,这到底是真是假?我看了静阳子的发言,他推荐的“因是子静坐法”很好,打坐的确是补元气的一大妙法。当然打坐也分很多种,也有很多讲究,我自己每天也打坐的,但我学的是南怀瑾静坐。打坐的确对于戒色后身体的恢复很有帮助,但并不是唯一的恢复方式,也并不适合所有人,我聊过不少焦虑症患者,一坐就头晕得不行,后来我建议站养生桩,身体就恢复得越来越好了。

有人喜欢打坐,有人更喜欢站桩,有人则喜欢八段锦和六字诀,有人练习太极拳,有人练习易筋经,中国的养生功法很多,可以选择适合自己的功法。对于静阳子说的“运动锻炼更是扯淡”,我不敢苟同,虽然中医有讲到劳则耗气,但也讲到动则升阳,中医并不反对运动,而是提倡适量运动,一味打坐并不是最好的恢复方式,要动静结合,华佗的五禽戏就是让人们动起来,但华佗并没有叫大家举石锁,华佗五禽戏的运动量和运动方式是有益于身体健康的。

\begin{quote}\it
    人体欲得劳动,但不当使极耳,动摇则谷气得消,血脉流通,病不得生。譬如户枢,终不朽也。(华佗)
\end{quote}

华佗这句话其实就是告诉我们运动的法则:要运动,但不要过度。在身体虚的时候,更要注意运动适量。我相信中国任何一个名中医都不会反对运动锻炼,他们反对的只是过度运动。就算是道士,也并不是完全靠打坐养生,很多有名的道士都习武,这样动静结合,更有利于养生修道。

静阳子说的“光戒是没用的”,这句我倒是比较认同,因为戒色是地基,养生才是造房子。光戒的确远远不够,必须学会养生之道,建立自己的养生意识,这样对恢复才比较有利。静阳子推荐的静坐法其实就是养生的一个法门,这个法门并没有错,大家可以尝试着练习。尝试了,你才知道效果如何,适不适合自己。当然,容易出偏的功法一定要谨慎练习,否则没毛病也会练出毛病来。

如何更好更快地恢复的确是戒色后的一个问题,但更重要的是做到彻底戒色,这才是最关键的。否则谈再多的恢复也等于零,就像一边吃中药,一边 SY,上补下漏,根本无法真正痊愈。

让身体恢复的法门很多,静坐只是其中一个法门而已,前段时间土豆吧主推荐的拜忏也非常好,土豆吧主就是靠拜忏恢复了健康。法门很多,请选择适合自己的,有佛缘的可以尝试拜忏。另外,除了养生功法,多做适量的有氧运动对身体恢复也是极其有利的,比如慢跑或者快走等。

相信静阳子的初衷没错,他大部分的观点也没问题,就是语气上给人盛气凌人之感,大家可能不大能接受。他是为了大家好,但如果换一种平和的方式,也许大家更容易接受些。

对于静阳子,我表示感谢,他的心是好的,希望他能更好地帮助到大家。这里的氛围不是敌对,而是善意的沟通,通过沟通来互相促进。

下面言归正传。这季就破戒类型,射距奥秘,连续两次的恶果这三个方面详细论述一下,具体如下。

\subsubsection{破戒类型}

破戒是每天都在发生的事情,破戒后往往很沮丧,这是不难理解的,当初戒色是雄心万丈,决心极大,没过多久就产生了动摇,一而再,再而三地破戒,自己都对自己失望了。身体症状告诉你不能再放纵了,再放纵就废掉了,但瘾告诉你继续 SY 吧,绝大多数戒友都是被瘾牢牢控制着,像一个提线木偶,成了欲望之奴,身不由己。明知道不对,但就是停不下来,在怪圈中苦苦挣扎。

失败并不可怕,可怕的是不学习,不学习觉悟和定力就不会提升。不学习的强戒,注定失败!有戒友会问,有人是一次戒除的吗?说实话,我到现在阅戒友上千,没见过哪个人是一次戒除成功的,都是反复破戒,不断总结经验教训,不断学习戒色文章提高定力和觉悟,这样才慢慢戒掉的。有人会说,我见过有人一次戒除的,其实,一次戒除这个概念有点模糊,是彻底觉悟后一次戒除还是第一次戒色就成功?我就是彻底觉悟后一次戒除成功的,我也属于一次戒除成功,但并不是第一次戒色就成功,之前也破戒过无数次。我第一次想戒色要追溯到初中了,那时候就觉得 SY 这个行为很龌龊,SY 后体质明显下降,十几岁时我就想戒色了,但那时没戒色资料,没人引导,也没学习戒色文章的意识,就是一味地强戒,结果就是不断破戒。我想大多数人尝试第一次戒 SY 都是在十几岁时,因为人是有戒色本能的,身体会告诉你不能再放纵了。而尝试第一次戒色,很多方面都无任何经验,属于戒色菜鸟级别,一没戒色知识,二没戒色意识,在这种情况下想一次戒除简直是天方夜谭。打个比方,让一个从没学过开车的人第一次开车,结果是什么呢?结果肯定是开得歪歪扭扭,甚至撞车。

戒色就像学开车,你必须掌握一整套戒色的理论和方法,然后熟能生巧,这样才有望戒色成功。你要说这个世界上是否真的有人第一次戒色就成功,我想即使有,这种人也是千里挑一的,善根极其深厚,而且前提是,在第一次戒色时,他就接触到了大量的戒色文章,并且善于学习,悟性极高,就像念书跳级一样,别人十年都没搞懂的道理,他一个月就弄明白了,这种人拥有不世出的戒色天赋,千里挑一不为过,甚至是万里挑一。

戒色必须专业,这季我总结了一些破戒的类型,大家可以参照地警惕自己,所谓“知己知彼百战百胜”,你知道了破戒类型,就会尽量保持警觉。就像知道了骗子的伎俩,就不会轻易上当,因为你知道那是骗术,不是骗钱,而是骗精!

\begin{multicols}{2}
    \begin{description}
        \item[情绪破戒] 无聊情绪、愤怒压抑情绪、烦躁情绪、厌倦情绪等
        \item[疑惑破戒] 疑惑产生动摇,然后破戒
        \item[诱惑破戒] 图和视频乃至 H 段子
        \item[赖床破戒] 非常多见,必须纠正
        \item[压力破戒] 各方面的压力
        \item[YY 破戒] 一般 YY 是破戒的前奏,包括回忆和幻想
        \item[放松警惕破戒] 以为成功了,其实一放松就易破戒
        \item[狂欢破戒] 狂欢时容易放纵,所谓得意忘形
        \item[试定力破戒] 千万不可去试
        \item[晨勃破戒] 晨勃后摸 JJ,从而破戒
        \item[试性能力破戒] 特别是早泄阳痿患者,一试就破
        \item[鼓动破戒] 被邪友带坏,去不良场所
        \item[喝酒破戒] 酒是色媒人
        \item[遗精后破戒] 遗精容易产生思想动摇,从而破戒
        \item[周末破戒] 周末一个人无聊,心魔容易跑出来
        \item[补药破戒] 补药吃多了,不注重修心,容易破戒
        \item[吃肉破戒] 肉吃多了,不注重修心,容易助长欲望而破戒
        \item[无害论破戒] 看了无害论,容易破戒
        \item[有女友破戒] 有女友对定力的要求更高,也容易破戒
        \item[有老婆破戒] 有老婆对戒色定力要求也很高。
        \item[习惯性破戒] 当破戒变成习惯,产生强大惯性,后果很可怕
        \item[假期破戒] 比如暑假和寒假,比较空闲,容易失控
        \item[邪法破戒] 不管是 PC 肌抑或别的所谓 JJ 增大法,极易破戒
    \end{description}
\end{multicols}

\subsubsection{射距奥秘}

下面再来谈下射距的奥秘。

射距就是射精距离,很多戒友反映刚开始 SY 的几年,射距很远,到后来就越来越不行,是顺着 JJ 流淌下来的,明显后劲不足,没有冲力了,这其实就是肾气亏损的一个表现。精,一看颜色,二看量,三看浓度,四看射距。射距越远,就说明这个人肾气越足。这个道理非常像射箭,精就好比箭,而肾气则是射箭的那个人,肾气不足,射箭无力,自然就射不远了。养足肾气,射距才会增加。一旦射距变短乃至是流淌下来的,其实就是身体在给你信号了,告诉你肾气已经不足了,不能再放纵了。如果此时你没读懂身体的信号,一意孤行,结果真的就是不见棺材不掉泪。

很多人只看到精,却没看见背后“射箭的那个人”已经不行了,精相对比较容易恢复,好好休息几天,吃得好点,就能有所恢复,但背后“射箭的那个人”要恢复就很慢了,可能要几个月甚至更长的时间才能恢复元气。所以,大家不能把眼光仅仅局限在精,而要看到背后“射箭的那个人”。那个人已经不行了,射不远了,就说明要学会戒色了,不能再放纵了,要学会养生了,否则等待你的只有症状和医院。

那个射箭的人就是“肾气”,既然他射不动了,射不远了,疲惫了,就不要再让他射了,让他好好休养生息吧,否则真会累出病来,切记!

最后谈下连续两次的恶果。

如果你去过植物神经紊乱吧,就会知道 sg 552 这个病友(案例 \ref{sg 552}),他就是长期熬夜 + 连续两次 SY,导致的发病,出现了濒死感。连续两次实在是太伤身体了,中医早就有讲到:欲不可强。\textit{强力施泄,便成劳损。(《千金方》)} 特别是连续两次,真是大忌讳。连续两次也非常容易出现前列腺炎和早泄阳痿的倾向,这其实就是人性的弱点贪婪导致的,对欲望的贪婪无度。要满足自己,结果是害了自己!

很多人之所以连续两次,其实就是一个原因:一次不过瘾,一次太短了,没爽够。正是这种对欲望的贪婪,导致了自毁。这种方式的自毁实在太多太多了,我们一定要时刻牢记中医欲不可强的教诲,不能去做傻事,你不知道这个道理,就会无知者无畏,结果害惨了自己,如果你知道了这个道理,还这样放纵,那完全就是自取灭亡!很多人无法战胜心瘾,就容易犯这种错误,就像中了魔法一样,破戒后才会悔悟。所以,我们一定要防微杜渐,好好管住自己的念头,尽量杜绝 YY,否则等火烧起来,就不是你所能控制的了。

\paragraph*{结语}

夏季来临,不少戒友反映脱发增多,于是有些慌张,其实导致脱发的因素很多,季节因素就是其中一个原因,我最近脱发量也由原来的五根以内变成二十根左右,这其实就是季节导致的,并不是戒 SY 导致的,所以在这一点上大家一定要认识清楚,否则容易产生信心动摇,从而转变成 SY 行为。我在导致破戒的因素中就讲到了“疑惑破戒”,如果你心中有疑惑,破戒的可能性就会很大。一般夏季人体汗腺油脂都分泌旺盛,加上紫外线强烈,会对头发造成损伤,这样就容易出现脱发增多的现象,不仅人脱发,你观察下动物,比如宠物狗,夏季脱毛也很厉害。

有位戒友说得好,戒色是一个系统工程。必须让自己戒得尽可能专业,必须让自己懂得更多,要让自己变成戒色内行才行,否则不学习,永远是外行,总是处在戒色菜鸟级别,要戒色成功就非常难了。当大家有了学习意识后,经过一段时间的学习,你会发现自己的觉悟和定力都有了提升,只要坚持学习,彻底开悟就不会遥远,彻底开悟后保持警惕,就是彻底的戒色成功。加油!
