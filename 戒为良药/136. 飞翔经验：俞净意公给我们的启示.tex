\subsection{俞净意公给我们的启示}

\paragraph*{前言}

最近有戒友反馈练习口诀时,头部出现不适,比如一念口诀头就痛,念佛也有类似的反馈,就是太用力了,注意力太过集中于头部,久而久之,气血不通,不通则痛。练习口诀或者念佛时,要注意放松,内心不可急躁,不可过于用力,可以小声默念,口念耳闻,注意力放在听口诀上。一定要注意放松,不要紧张,小学时我们念乘法口诀,何曾出现头痛过?一旦过于用力,过于集中于头部,时间长了,头部肯定会出现不适的。不宜大声念,因为大声伤气,一直大声念,身体容易吃不消,而且大声也会很累,不利于坚持。我练习到现在,未曾出现头部不适,因为我很注意放松,这样就没事。千万不能绷着头使劲地念,这样容易导致心火上炎、头痛、失眠等,自己一定要善于调整,不可勉强。放松地念,轻松地念,就没事。如果头部出现不适了,我建议先暂停一段时间,不要练习口诀了,自己注意休养和调整,转移一下注意力,去大自然中走一走,看看风景,放松一下心情,等身体恢复正常再说。这段时间就直接练习观心,看自己的念头,不要跟随。

\textit{观心法门是正修心之路。(南怀瑾先生)} \textit{观照是正行,打坐是助行,观照是不可忽视的正行!(元音老人)} 观心是基本功,也是最终极的功夫,这个功夫传承了几千年,无数祖师大德都在强调这个功夫。\textit{夜深人静独坐观心,始知妄穷而真独露,每于此中得大机趣。(《菜根谭》)} \textit{唯观心一法,总摄诸法,最为省要。(达摩祖师)} 观心,就是看念头,练习觉察力,刚开始熟背断念口诀,经过“走神 - 拉回 - 走神 - 拉回”的反复训练,走神的次数在减少,拉回的速度越来越快,这就是觉察力在变强了。发现自己在胡思乱想,在你发现的那一刹那,就是在观心,你发现自己跟着念头跑了,那一刹那的觉就是观心,就是在练习这个觉,强化这个觉!走神的次数不断减少,看、觉的力量在不断增强,降伏其心,不在话下。觉的力量有了,就能做到一觉即断,也就是断念口诀的后四个字:觉之即无。刚开始是念起不觉或者觉而不断,因为刚开始水平低,断力不足,实战意识差,经过学习和练习,断力变强了,实战意识也有了,到时就可以轻松断念了。随着你越来越强,就会形成强大的威慑力,邪念不敢轻易进犯。断念高手的眼神可以吓退一波邪念、图像、微妙感觉、怂恿念等,那个眼神够凶、够狠,充满凛冽的杀气,仿佛在说:“敢上来,就斩立决!”就是这股狠劲!那个眼神足以秒杀一切邪念!就像一个狠人站在中央,周围一群歹徒恶棍想上来,狠人一个眼神,他们就齐刷刷退下,不敢上来,因为他们知道上来就是送死!\textit{吾应乐修断,怀恨与彼战,似嗔烦恼心,唯能灭烦恼。(《入行论》)} 这里的嗔恨是一种杀敌的决心,同仇敌忾,上阵杀念,下马学佛,杀念时要勇猛!学佛时要慈悲!不怕贼强,只要将猛!

练习观心断念不可急于求成,这是一个慢功夫,就像练习打字一样,刚开始打得很慢,但每天坚持练习,渐渐就快了,刚开始还要看键盘,后来直接就盲打了,速度越来越快,越来越熟练,关键就是坚持练习,每天进步一点,坚持一百天,就会发现自己已经今非昔比了,已经具备降伏心魔的实力了。一位戒友说:“练习观心断念现在获益的两种体验,第一:在每天坚持断念口诀五百遍,差不多一个月后,当我起邪念时我能立刻就觉察到邪念来了,有时甚至会邪念一起,心中不由自主地念起断念口诀。第二:随着断念口诀的加深,我脑海中的断念之刃越来越锋利,我好几次亲身懂得了飞翔前辈说的念起即觉、觉之即无是何种感觉了,觉察即消灭,觉察即照破,邪念一起即觉,一觉即无!这种感觉亲身体会后实在神奇!坚持断念口诀的练习,就能体会到念起即觉、觉之即无,就知道断念的殊胜。”不管是断念口诀、思维对治、念佛持咒、一字诀(呸、断、咄)等,都是为了断念!断念是实战核心和根本,偏离实战,必败无疑!有的人刚开始不重视断念,甚至看了一些误导的文章,对断念存在误解和偏见,等到失败很多次以后,在实战中一次次被虐,才发现原来观心断念如此重要!看似再好的方法,如果脱离实战,那就存在巨大的缺陷!我一直在强调实战,重视实战,以实战为核心,实战强,才是真的强!实战弱,被心魔虐成狗!一切看实战!不管何种断念方式,都要坚持日课,坚持练习,才可能有持续的进步,才能最终降伏心魔。

我的戒色文章直指戒色实战的核心——观心断念!不会和大家兜圈子,我的方法就是直指!简单明了,一针见血!观心断念是最治本的,因为观心断念是总原则,总摄诸法,最为省要,直接单刀直入,直指核心!\textit{念起即觉,觉之即无,修行妙门,唯在此也。(圭峰禅师)} \textit{千万个修,抵不过我一觉。觉则心空,此是最上福德。(《般若花》)} \textit{随他多少邪思枉念,这里一觉,都自消融。真个是灵丹一粒,点铁成金。(《传习录》)} 问:当念头起来的时候,要如何控制它?怎么样不跟坏念头跑?宣化上人:念起即觉,觉之即无。无数的高僧大德都在讲这个“觉”,这个观心断念!练习断念口诀就是为了培养大家内在的觉察力,觉察力变强了,就可以降伏心魔了。真正的断念高手,内在会有一种很强的力量感,这种力量感不是举起两百公斤杠铃的力量感,而是一种内心的力量感、主宰感,这种力量感来自于不断强化觉察力!在练习的过程中可能会出现破戒,因为水平还不是很高,出现破戒,不要气馁,调整好心态,继续坚持练习,练到一定程度,自然可以降伏心魔,主宰内心。

刚开始不少新人会试图压念,他们没有正确理解断念口诀,压念就是念头起来了,想压住念头,不让念头起来,压念是压不住的,反弹会更厉害,断念是觉而化之,一觉即灭,轻松得很。当实力具备了,自然可以做到。我初中站桩时就是压念,那时站桩时念头来了,我就拼命想压住它,就是不想让它起来,结果念头起来得更猛,跑得更凶,就像脑袋里有一个博尔特,在那疯跑,怎么压也压不住,我感到很挫败。现在我已经真正掌握了断念口诀,不会去压念,只一觉,就完事了,非常轻松。看见念头,念头就没了!这个实战操作,元音老人在讲,丁愚仁老师在讲,明就仁波切也在讲,无数的高僧大德都在讲这个实战操作,实战最核心的一个字,就是——觉!好好领悟这个觉,练习这个觉,就可以在实战中立于不败之地!刚开始熟背断念口诀,到了高级阶段就不用背了,因为觉察力已经练出来了,可以直接觉察消灭了。初级阶段的操作:念头来了 $\to$ 马上念口诀来转(原理类似于念佛号来转),平时要有日课,把口诀念熟;高级阶段的操作:念头来了 $\to$ 一觉即灭(非常轻松)。

下面分享一些案例。

\begin{case}
    戒色 200 天了,谈谈自己戒色后的一些变化,勉励各位吧友。16 岁开始手淫,导致自己以后的人生一片昏暗,一事无成,还带来了很多疾病,严重的阳痿、早泄、前列腺炎、前列腺疼痛、精神萎靡不振。身体快垮了,29 岁了,快 30 了,却什么都没有,心里的痛苦只有自己知道。这些年自己经历了什么,不说了,大多数戒友都有相同的地方。我想说说在自己快要绝望的时候,遇到了戒色吧,看了《戒为良药》,才恍然大悟,原来这些年所经历的痛苦,根源在于邪淫,却不自知。自从看了《戒为良药》,知道了自己百病缠身的根源在于邪淫后,开始下定决心戒色,一定要恢复自己的身体,找回遗失的美好,重新恢复光明人生。每天都坚持看戒色文章,学习了很多戒色知识和方法。

    \begin{itemize}
        \item 断手淫意淫杜绝一切黄源,把所有有关黄的东西全部销毁了。
        \item 开始早睡早起,坚持晨跑,锻炼身体,做固肾功、站桩、面壁蹲墙功等等。
        \item 饮食方面,吃清淡的,很少吃肉辛辣刺激性食物。
        \item 每天坚持看戒色文章,学习戒色知识、方法,坚持看书学习提升自己。
        \item 开始孝敬父母,做力所能及的善事,帮吧友回答问题,劝人戒色,在公司帮助身边的同事等等,只要是自己能做的善事都去做。
    \end{itemize}

    现在身体在稳步恢复,阳痿好了,前列腺不疼了,龟头没有以前敏感了,没有了腰酸背疼,整个人精神了很多,眼睛发亮,面色红润了,皮肤变好了,变得喜欢交际了,整个人是发自内心地开心,走在路上都想笑,运气也变好了,总会有意外的收获。也只有受到邪淫伤害的人,才能体会到我现在的心情。现在的我只想拼命工作干事业,努力恢复身体,迎接明天幸福的生活,迎接那个她。同时,也告诫吧友们,戒色一定要有决心,一定要尽全力,才有希望成功。

    \textbf{附评} 戒了 200 天,这位戒友完成蜕变了,戒色、养生、行善,他都做到了,而且做得很好,身体稳步恢复,身心状态有了质的飞跃。看到他的蜕变,我感到由衷的高兴,他说:“变得喜欢交际了,整个人是发自内心地开心,走在路上都想笑。”这种状态多么好啊!戒色获得了纯净的大快乐,能量水平有了极大的提升,感觉人生由灰暗变得色彩斑斓了,有了某种童年时的神奇感觉,身心轻盈、愉悦,充满活力和精力,这种状态实在太棒了。身体好了,干事业也有劲头了,未来可期;正能量起来了,开始感召好的人事物,生活会顺利很多。邪淫时,感觉生活罩上了灰蒙蒙的滤镜,看一切都是灰暗颓废的,戒到一定时间,这层滤镜消失了,突然感觉周围又变得彩色鲜活了,脸上也有了久违的笑容。“当你走出邪淫的泥沼,你永远不会孤独,因为有万千戒友在默默鼓励你、支持你,这给了你莫大的勇气和力量。抬起头,走出黑暗,用你的双眼去迎接璀璨的光明。在黑暗的尽头,是蔚蓝的天空,是明媚的阳光和鸟语花香的世界,还有孩子纯真无邪的笑容。恍忽间明白,原来戒色如此美好,如此快乐。”我相信,只有经历过磨难的人,才能够更好地懂得人生的道理,也才能珍惜这来之不易的幸福,总有一天,你的坚持会让脚下的路走向开阔,走向美好的未来。
\end{case}

\begin{case}
    第一说身体,去年八月开始,戒了 99 天,加上运动锻炼,身体恢复了一些,不用穿很多衣服,手脚也不总是冰凉的,身体有多余的热量,三十多岁了,第一次感觉过了个舒服的冬天。在冬天没抗住又破了,然后就是断断续续破,春天难受,整个人像病怏怏的一样,然后到夏天就更难受了,夏天热又不敢吹电扇,进空调间就难受,你说多痛苦!吹电扇根本不敢吹,吹得骨头疼,进空调间呆一会,那就像瞬间得病一样,直接犯鼻炎,直打喷嚏,流鼻涕,痛苦就不用说了,我算是体会到了在什么时候破的,当时不觉得什么,到下个周期一定会反映出来的。撸了十多年了,身体江河日下,说真的,以前就根本不知道戒色,真是羡慕现在年轻的二十岁左右的,你们是幸运的,好好戒下去,身体肯定会恢复的,人生会越来越好。第二说倒霉和运气,我说因为邪淫我损失了几百万,你信吗?今年35岁了,一事无成,现在连工作都找不到,在家啃老,谁能想到我三十岁的时候我是一家有二十人公司的老板,开着豪车,住着一百多万的房子,还有一个更重要的身份,财政局正式的公务员,这一切都是真实拥有过的,可是现在什么都没有了,我一直都在思考为什么?我觉得我运气都不错的,但是就是抓不住,随便怎么样,我都不会落到现在这样的地步,可是现在的局面就是这样,让我痛苦万分,直到我学了一点点佛,我才逐渐明白一切的因果,就是因为邪淫,同样是做生意,别人做了几年没出过问题,我做的几年中,大大小小的问题都出现过,再说遇到的人,那就全部是邪淫的,全部是拉我入坑的,让我辛苦挣的钱都一点点耗尽。

    \textbf{附评} 这位戒友分享的两点让我感同身受,第一点身体,本来通过戒色养生锻炼,身体恢复了一些,的确感觉身体变好了,后来他在冬天破了,春天就难受,到了夏天更难受,不敢吹电扇,不敢吹空调,身体虚弱时,真的一点风寒都扛不住。身体虚弱后,卫气会变差,\textit{卫者,水谷之悍气也,其气慓疾滑利。(《素问》)} 卫有保卫、卫护之义。卫气属阳,卫气生于水谷,源于脾胃,出于上焦,行于脉外,其性悍,运行迅速而滑利。具有温养内外、护卫肌表、抗御外邪、滋养腠理、开阖汗孔等作用,卫气不固会导致疾病发生,一点风寒都扛不住,稍微吹一点电扇和空调,身体立马就会出现不适,我之前也有这个体验,真的是一点电扇空调都不能吹,吹过之后,身体就开始难受,我相信很多戒友也应该有这个体会。这位戒友也提到了周期的问题,比如现在破的戒,也许短期内感觉不明显,但是过个十几天、二十天身体就开始感觉不行了。有大周期和小周期之分,大周期是以年来算的,中医讲过:冬不藏精,春必病温。冬天我们身体的气机收敛,应该藏精,如果冬天不能藏精,到了春天的时候,正气不足,就很容易感染温病。人体周期参数:体力周期(约为 23 天一个循环)、情绪周期(约 28 天一个循环)、智力周期(约 33 天一个循环)、习惯周期(约 21 天一个循环)、人体血液更新周期(约 18 天一个循环)。人体是有周期的,这个周期破戒,到了下个周期身体就会很难受,开始爆发疾病。有的人说自己现在不是好好的嘛!没什么大问题,其实他这样放纵,不断掏空自己,到了一定的周期,身体必然会垮掉的。这位戒友第二点,谈到了运气,运气这个词有点虚无缥缈,我更喜欢用运势这个词,邪淫会影响运势,这是公开的秘密,能量场一差,运势自然会变差,会感召各种倒霉的事情。犯邪淫者,求财不成;犯邪淫者,家庭不合;犯邪淫者,疾病缠身;犯邪淫者,官业必堕。之前这位戒友的福报还是很大的,几百万的家产,当老板,还是公务员,好事都让他赶上了,但是邪淫后,他遭遇了滑铁卢,做生意失败,公务员也没了,35 岁,一事无成,在家啃老,这种局面让他痛苦万分。做生意最怕的就是嫖娼!和客户喝酒后经常去声色场所,不仅自己嫖娼,还请客户嫖娼,这样削福报太猛,果报很惨!有的人嫖娼几次,领导就开始看他不顺眼了,开除他了,我之前两个同事就是这样的。嫖娼的果报来得非常快,对财运、官运的影响最大。有的人撸坑还没出来,又掉进了嫖坑,嫖娼的果报更惨!千万不可去嫖娼!犯了邪淫,就像捅了大窟窿,多大的福报都会漏掉的,就像一浴缸的水,下面塞子拔了,没多久就漏光了。就算是亿万富翁也是玩不起的,不少亿万富翁就是因为邪淫导致人生运势变差,资金链断裂,负债几千万,压力大得睡不着觉,甚至自杀,一死了之。恶不积,不足以灭身,积恶必败!这个案例的戒友算是处于人生的低谷,好在他现在已经知道自己的问题所在,并且又戒了一百天,他这种情况就应该力戒邪淫,行善积德,守正待时!这样才有望东山再起。
\end{case}

\begin{case}
    戒色真的是太棒了,太爽了,这才是真正的大爽,手淫所谓的快感与之相比真的是弱爆了,完全比较不了。接触戒色至今两年了,这两年让我彻底地蜕变,我更加自信阳光了,最主要的是我真正地看清楚这个世界,这个世界是那么的美,处处充满了美,那是一种纯粹的大美。以前我的世界充满了灰暗处处都是不如意,现在就不一样了,我感觉我从地狱回到了天堂。这一点也不夸张,这是我自己的真实感受,我在这里保证我说的每一句话都是真的。希望各位师兄弟坚持戒色,能接触戒色吧可以说是你最大福报,让我们一起筑建一个充满正气的神州!加油!

    \textbf{附评} 戒色的爽不是那几秒快感可以比拟的,这是灵魂的高频振动,是高级的大爽!手淫是低级的爽,蛆吃屎的爽,“大可笑,大可笑,好汉多迷屎尿窍。”智者知道真正的快乐来自于纯净的心灵,来自于崇高的善德,因为纯净和善良是高频,邪念是负面低频,负面低频必然会导致灰暗、沉重和痛苦,这是必然的。宇宙中的一切都有自己的振动频率,人生的秘密在于提升振动频率,小到粒子,大到整个宇宙,万物都在振动之中。振动频率上去了,纯粹的大快乐自然会到来。当我们将自己提升到一个较高的振动频率之时,一切都会变得神圣、微妙与喜悦。要提升自己的振动频率,最关键的就是净化自己的心灵和行善积德!在一个自然美景的地方感受到的振动,与在一个肮脏、受污染的地方所感受到的振动是截然不同的。我们每个人都有自己独特的振动,心灵纯净的人,会让人感到喜悦与美好,看看孩子就知道了;心灵肮脏的人,会让人感到厌恶和恶心,只想远离。高频率对应的就是轻盈、美好和喜悦,这是真正的快乐,不是邪淫短暂的快感,如果你是一个有觉悟有智慧的人,很快就能发现两者之间的巨大区别,快感不是真正的大快乐。高频率是一个神奇的状态,就像四岁孩子看世界,充满了神奇,感觉世界那么美,那么不可思议。有句话是这样说的:“生活中从不缺少美,而是缺少发现美的眼睛。”我要说的是:“真正缺少的是纯净的心灵,心灵纯净了,自然发现美。”一位戒友说:“戒色到 220 天我明显感觉到自己浑身是胆,内心中总是充满一股英雄豪气和王者霸气,斗志也很强了。以前邪淫时从未有过这种气概。这种王者气概才算是真正的铁血男子气概!”有了这种气概和斗志,就能做成大事,真正的戒者,气吞万里如虎!

    另外三位戒友的反馈:

    \begin{itemize}
        \item “现在每天我过得特别安宁,像小时候一样。舒服的衣服会让我快乐,蓝天白云会让我快乐,鸟叫声会让我愉悦,这或许就是真正的快乐吧。手淫时我无时无刻不痛苦,走在大街上没来由的恐惧,不敢和别人起冲突,不敢和女生说话,被除了家人的所有人嫌弃。”(附评:安宁这个词很好,内心的安宁与美好,多么久违的感受。自己会因为简单的事物而感到快乐,这种感受在手淫后就失去了,戒色后又找回来了。手淫不是真的快乐,撸者的内心是很痛苦的,因为肾精耗损了,负能量变重了,自然会感到痛苦。)
        \item “戒色之后整个人是十足的大爽,正能量爆棚,心灵经过洗涤,重新变得纯粹和安宁,看到什么东西莫名想笑,这种感觉实在美好!”(附评:十足的大爽!这五个字形容得很贴切!不戒色,不知道戒色有多爽!纯粹、安宁、美好、愉悦,看到什么东西莫名想笑;手淫时期,看到什么东西都觉得不顺眼,内心莫名烦躁,充满嗔恨和恐惧,真的是一个天堂一个地狱。)
        \item “戒 382 天,早上起来的感觉整个不一样了。平静,安详,精力充沛,不可阻挡。所有事都 hold 住。”(附评:手淫后早上起不来,感觉睡不够,精神萎靡不振,戒色后身心感觉整个不一样,就像充满电一样,有着雄厚澎湃的能量,即使学习工作一天,也不感觉累,精力像开了挂一样,势不可挡。)
    \end{itemize}
\end{case}

\begin{case}
    哈哈哈哈哈哈!!!我明白了那个觉!我顿悟了那个纯粹的觉知,那个安住!万分亿分感谢飞翔恩师!如果不是您,我也学习不到那么多殊胜的无价开示,真的是无价之宝!给我什么我都不换!这次顿悟说来话长,今天和家人外出游玩,回来后累挺了,正好吃了些肉,然后因为太累就放松了警惕,独处躺在了床上。结果心魔祭出了一幅恋癖图像,我警觉,但因为觉察力不够 + 放松警惕,没能一拳把心魔打挺尸,心魔连续祭出十多幅恋癖图像,强烈的破戒冲动如魔鬼一样进入了我的脑中。我在几天前顿悟了学习的重要性,我的觉悟让我勉强保持了理智。然后心魔又祭出了一个“我要去看黄”的极阴险的念头(以“我”字开头的念头藏得极深,想要彻底认清它很难),我的理智几近消失,只差一丝丝就去执行那个看黄指令了。但是!我想起了飞翔老师您在《戒为良药》\ref{135} 的开示,我突然意识到那个念头不是我!(那种感觉用任何语言去形容都显得苍白)我觉察消灭了它!然后又上来了几幅恋癖图像和看黄的念头,我又觉察消灭,由于平常只学习文章做笔记复习而疏于练习断念口诀,我觉察的速度时快时慢,但我仍然打败了心魔!!!这在我破戒之后的六十多天是从未有过的!(对于我来说打多少个感叹号都不够啊,太激动了!太开心了!)然后我又看了一遍您的《戒为良药》\ref{135},感悟极深,里面决胜实战的关键点真的真的太多了,感觉需要做的笔记用一整天都记不完啊!(激动得都语无伦次了)快看完时我突然就自己真正安住于纯粹之中了,发现那种状态真是太美好了,太幸福了。全部看完以后,我就忽然顿悟了那个觉,当时那叫一个激动啊,就差飞到天上去了。觉得我当时是这十四年来最幸福最开心的时候了,激动得全身的细胞都在欢呼雀跃。我顿悟了能够主宰内心的知识!(能够主宰内心我也就能主宰人生了)以上我的所有话语我都感觉不满意,语言在亲身经历面前实在是显得太过渺小和无力。真的是这样,只有真正悟进去了才有那种身临其境的美妙感觉(重点在于反复看),看戒色文章最重要的就是悟了。飞翔老师,您是我见过最好、最可爱的人了!您对我的恩情太大了啊!感动流涕!您拯救了我的一生,也间接拯救了我的家人,教会我修心的道理,您是真正的智者,您是活菩萨啊!我以后也要注重学习圣贤教育,多修德行,觉断恶念。我第一次戒色戒了一百五十天左右,最后失败就是因为骄傲念头没有及时觉察消灭而放松学习练习,觉悟下降得很厉害。戒色如逆水行舟,停止学习练习,觉悟就在退步,就无法降伏心魔了。我再也不想被骄傲的念头控制了,我不会松懈的!我再也不要被神经症折磨了,我看到了光明!

    \textbf{附评} 这是上季的一个反馈,上季是会让人顿悟的一季,顿悟之乐,实在妙不可言。在实战中,这位戒友突然意识到那个念头不是我!这个领悟真的可以力挽狂澜,本来他已经听信了心魔的怂恿,心魔冒充他,以“我”字开头的念头向他发送破戒指令,他本来就要执行这个指令了,突然意识到那个念头不是自己!然后几下觉察把心魔打败了,他非常激动,战胜心魔的确很爽!整个人非常振奋!就像打了大胜仗一样!实战后,他回到了 \ref{135},感悟极深!实战后,要马上回到实战的文章,这时学习文章,会有入木三分的理解,乃至顿悟!之前看文章,没有多少深入的体会,实战后再看,一下就看明白了,一下就抓住重点了,一下就看进去了。所以,我很强调实战后马上看断念的文章,这时往往会产生觉悟的突破,是一个飞跃的关键时段。平时往往看不进去,实战后再看,真的是每个字都打进脑袋深处了,真正悟进去了。断念实战的水平要有质的飞跃,最关键的一点就是完成身份的转换,从思考者转变到观察者,念头不是自己,从而不认同念头,否则身份不转换过来,很容易跟着念头跑。身份转换成观察者,加上不断强化觉察力,熟悉心魔的各种进攻套路,做到知己知彼,这样就可以轻松决胜实战。这位戒友也提到了骄傲的问题,骄兵必败,戒者要培养谦德,平时做人就要恭敬谦虚,一个敬字,一个谦字,这两个字受用无穷。德乃戒之基,我们一定要好好完善和提升自己的德行。在我眼里,每一位众生都是活菩萨,唯我一人是愚夫!很荣幸能为各位戒友服务,希望每一位戒友都能超过我!我做垫脚石也没有关系。
\end{case}

下面进入正文。

这季是关于俞净意公遇灶神记的分享,中国传统文化里面关于改造命运的典范,一个是《了凡四训》的袁了凡先生,另一个就是《俞净意公遇灶神记》的俞净意公,这两个就像两座高峰一样,是必读的两本书。《了凡四训》我在 \ref{129} 分享过了,《了凡四训》的确太经典了,不知多少人都在学习《了凡四训》,《了凡四训》和《俞净意公遇灶神记》各有千秋,从改造命运的原理和方法上,《了凡四训》无疑更系统,但《俞净意公遇灶神记》说了一个修行人最切中要害的问题,那就是——意恶!很多人修行流于表面,流于形式,虽然也做了很多善事,但是命运却不好,甚至还因此怨天尤人,怀疑古圣先贤的教诲,其实问题出在自己身上。

修行的关键是要从起心动念上修,\textit{南阎浮提众生起心动念,无不是业,无不是罪。(《地藏经》)} 十界图当中就是一个“心”字,心也就是念头,十界图太著名了,相信很多戒友都看过,我每次看十界图都很有感触,当中为何是一个“心”字,不是一个“身”字,或者其他什么字,因为心是根本,修行的根本是修心,必须学会断除恶念,多发善念。外在的修身属于辅助,关键的核心还是修心,念头要正,念头一邪,必然衰败,所谓心正则兴旺,心邪必衰败!

俞公之前就是流于形式,专务虚名,虽然也做了不少善事,但是心里还是有很多的邪念,还怨天尤人,这样命运肯定就不好了,他还不知道原因,后来灶神亲自来点化他,他这才醒悟过来。这个故事很有启发,因为很多人都是在犯这个错误,不知从心地上改,虽然做了不少善事,但是却不知对治邪念,导致最后命运不济。善事要做,邪念也要对治,这是双管齐下的,不能以为做了善事,邪念就没了,不是这样简单的,如果这样简单,那些祖师大德仅仅强调行善即可,也不必谈对治邪念的方法了。有些做公益的人最后犯了邪淫,为何如此?他们善事做了很多,但是却不知对治邪念,仅仅行善是不够的,必须学会对治邪念。种上庄稼还是要除草的,你问问农民,他们除不除草?你到庄稼田里看一看,就会发现那些田里还是有杂草,农民伯伯还是要定期除草的。有的人会认为光行善就行了,这肯定是不行的,我记得之前一位戒友写了一些文章,专门强调行善,最后他却向戒友借钱不还,还振振有词,他光强调行善了,却没对治自己的意恶!意恶重,命运肯定凄惨,俞公后来叫俞净意公,就是为了提醒自己净化意念,从起心动念上修,后来他的命运逆袭了。

《俞净意公遇灶神记》有一部连续剧,八集,我几年前就看过了,受到了良好的熏陶,很喜欢里面的古风场景。《俞净意公遇灶神记》的文字比《了凡四训》短很多,却能拍成八集,实属不易。从儿时一直拍到了老年,内容很丰富,儿时和少年时是那么纯真,俞公的房子在竹林里,鸟叫得很好听,那个居住环境我很喜欢。俞公少年时代和同学在学堂里一起读书,那个场景我很感动,少年的纯真让我感慨万千,那么意气风发。这部连续剧很不错,建议大家看一下。看了文章,再加上看了连续剧,这样就比较生动活泼了,仿佛回到了那个年代,仿佛自己成了俞公,经历了他的喜怒哀乐。

文章开始说,俞公与同庠生十余人结文昌社,惜字放生,戒淫杀口过,行之有年。古代读书人都祭拜文昌帝君,文昌帝君又名文星神,是中国民间和道教尊奉的掌管士人功名禄位之神。在古代,文昌帝君是每个读书人都要拜的,全国各地都有文昌阁,不仅仅是求保佑,还有感恩文昌帝君的教化。《文昌帝君阴骘文》《文昌帝君训饬士子戒淫文》《文昌帝君戒淫圣训》《文昌帝君戒淫宝训》《文昌帝君元旦劝孝文》《文昌孝经》,这几篇绝对是经典中的经典,值得反复研读。

\begin{quote}\it
    文昌帝君云:“广行阴骘,上格苍穹。人能如我存心,天必赐汝以福。……欲广福田,须凭心地。行时时之方便,作种种之阴功。利物利人,修善修福。正直代天行化,慈祥为国救民。……诸恶莫作,众善奉行。永无恶曜加临,常有吉神拥护。近报则在自己,远报则在儿孙。百福骈臻,千祥云集,岂不从阴骘中得来者哉?……天道祸淫,其报甚速,人之不畏,梦梦无知,苟行检之不修,即灾殃之立至。……倘能持正而不邪,自尔名归而禄得。……惟淫恶之报,天律最严。……孽海茫茫,首恶无非色欲;尘寰扰扰,易犯唯有淫邪。拔山盖世之雄,坐此亡身辱国。绣口锦心之士,因兹败节隳名。始为一念之差,遂至毕生莫赎。……谨劝青年志士,社会名流,发觉悟之心,破色魔之障。芙蓉白面,不过带肉骷髅;芍药红妆,乃是杀人利刃。纵对如玉如花之貌,当存若姊若妹之心。未犯者宜防失足,曾行者及早回头。更望展转流传,迭相化导。必使在在齐归觉路,人人共出迷津。……何谓第一事?孝者百行之原,精而极之,可以参赞化育,故谓之第一事。”
\end{quote}

从文昌帝君的教诲中,我们可以得知,文昌帝君格外注重行善积德、戒邪淫和孝顺,古代的读书人必读文昌帝君的教诲,所以古人很有行善积德的意识、戒色的意识和孝顺的意识,这三个意识,古人从小就开始培养了,根扎得特别深,而现代人这三个关键的意识都很淡薄了,现在中国在复兴,希望古圣先贤充满智慧的教导也能得到很好的传承,我辈要好好发扬和提倡。

古代的一个例子:

\begin{quote}\it
    唐皋少年时某日在窗下读书,有一位少女听到书房内书声嘹亮,心想此人后日必成大器,于是将纸窗用舌头舐破一个洞,目送秋波,想藉此互通情意,唐皋一见即刻将破洞补起来。并题诗说:“舐破纸窗容易补,损人阴德最难修。”后来考中状元,名播天下。贤哉唐皋,力戒邪淫,丝毫不犯。
\end{quote}

看看古人的戒色意识有多强!面对邪淫的机会,力拒于千里之外,做光明磊落的君子,不做邪淫放纵的无耻之徒。唐皋知道邪淫的事情损人阴德,会遭报应,所以严正拒绝,绝不去犯。最后他考中了状元,和他的高尚品德有很大关系,如果他把持不住自己,和女子私通,耗损自己的肾精,必然会影响考试状态,搞得自己灰头土脸,丧失光明正大之气。古人之所以戒色意识这么强,和学习圣贤教育密不可分,尤其尊崇文昌帝君的教诲,力戒邪淫,立场格外坚定!面对邪淫的诱惑,具有刚正的气象和严格自律的精神,绝不苟且。

下面开始分享读书笔记与解析。

\subsubsection*{读书笔记与解析}

\begin{quotation}\it
    公心异其人,执礼甚恭,因言生平读书积行,至今功名不遂,妻子不全,衣食不继,且以历焚灶疏为张诵之。

    张曰:“予知君家事久矣。君意恶太重,专务虚名,满纸怨尤,渎陈上帝,恐受罚不止此也!”
\end{quotation}

\textbf{解析} 灶神自称姓张,看上去很有仙风道骨,俞公说自己读书积行,积行就是积累善行,但是功名不遂,家庭不幸,命运不济,甚至衣食不继,潦倒贫困。灶神一针见血地指出了俞公的问题所在:意恶太重,专务虚名,还怨天尤人。意恶这两个字是文章里的重点,简明扼要地直指俞公的核心问题。当灶神说出“意恶”这两个字,估计俞公心头一惊,犹如当头棒喝。他也许从来没听过“意恶”这个词,这时却如雷贯耳,直接被震到了!灶神继续指出俞公的问题。

\begin{quote}\it
    社中每月放生,君随班奔逐,因人成事,倘诸人不举,君亦浮沉而已,其实慈悲之念,并未动于中也。且君家虾蟹之类亦登于庖,彼独非生命耶?
\end{quote}

\textbf{解析} 灶神继续指出其他问题,俞公做善事心不真诚,没有慈悲之念,也只是随众人而已,自己并未真心行善。虽然参与放生,但自己家里却还是杀生,这明显有悖于慈悲。灶神指出了行善的关键,行善的心一定要真诚,要有慈悲心,真放生,首先要戒杀生。放生合天心,放生顺佛意,放生免三灾,放生离九横,放生寿命长,放生官禄盛,放生子孙昌,放生家门庆,放生无忧恼,放生少疾病。放生时,自己的心情是特别愉悦的,非常开心舒畅的体验,但一定要如法放生。生命被宰杀时,其苦无量,其悲无涯,苦苦挣扎,恐怖万分,其痛苦不可言说。你把它从生死线上解救出来,是何等慈悲的精神,使其远离生死的恐怖,使其从无助无奈的苦痛哀鸣中解脱,获得安乐幸福,给那些可怜众生消除生死恐惧的最大灾难,实是最大的善举,这就是无畏施。行无畏施者,消灾免难获得平安,自然得长寿健康的大福报。

\begin{quote}\it
    若“口过”一节,君语言敏妙,谈者常倾倒于君;君彼时出口,心亦自知伤厚,但于朋谈惯熟中,随风讪笑,不能禁止,舌锋所及,怒触鬼神阴恶之注,不知凡几!乃犹以笃厚自居,汝谁欺?欺天乎!
\end{quote}

\textbf{解析} 这条是关于口过的,群居防口,独坐防心。要有口德,不能随意讪笑讽刺别人,做人要厚道。在人多的时候,往往容易犯口过,说到兴起了,就管不住自己。这点我们也要注意,要存好心,说好话,注意口德。开口讥诮人,不惟丧身,足亦丧德。修炼口德,就是修炼自己的气场,口德好才能运势好。恶语伤人六月寒,语言切勿刺人骨髓,戏谑切勿中人心病,伤人以言,甚于刀剑。古人说:口能吐莲花,也能吐蒺藜。我们一定要注意口德,负能量的语言对人的伤害非常大,也对自己的阴德损耗非常大。做人,嘴巴要留口德,不要尖酸刻薄,口业的力量很大,善语得善报,恶语得恶报。不要说话伤害别人,不要说话侮辱别人,不要说话讽刺别人。

\begin{quote}\it
    邪淫虽无实迹,君见人家美女子,必熟视之,心即摇摇不能遣,但无邪缘相凑耳。
\end{quote}

\textbf{解析} 古人云:“见美色时,而起心私之,其心田即暗,中正之心已邪,则光明正大之神遂失,若人时时存邪念,积久而邪气蛊惑于身心,即小人矣。”俞公见到人家的美女子,眼睛就盯着看,内心也有不良想法,这就属于眼恶、意恶!虽然没有实迹,但这种眼恶、意恶也会增加负能量,见了美女子就盯着看,这明显是好色的表现,这种行径明显不是正人君子的作风。圣贤告诫我们非礼勿视,这就是视线管理,不能盯着看,更不能有意恶,这点要格外注意,戒色十规也专门讲到了视线管理。

\begin{quote}\it
    于私居独处中,见君之贪念、淫念、嫉妒念、褊急念、高己卑人念、忆往期来念、恩仇报复念,憧憧于胸,不可纪极。此诸种种意恶,固结于中,神注已多,天罚日甚;君逃祸不暇,何由祈福哉。
\end{quote}

\textbf{解析} 这条说得更直接,古人特别强调慎独,慎独则心安。而俞公独处时,各种邪念充满了他的脑袋,这都是意恶!修行一定要断意恶!从起心动念上修!意恶重,怎能有好报?逃祸不暇,还想着祈福,这不荒唐吗?因为不知道自己的意恶是如此重,还以为自己做了很多好事,应该可以获得很大的福报,殊不知,意恶重也是在积累恶业,是会受到惩罚和报应的。一直在向天地宇宙发送恶念,吸引来的遭遇肯定是倒霉的,就像对山谷喊:“你是坏蛋”,回过来的也是“你是坏蛋”,你就是那个发射源,如果一直发送负面低频的信息,这样人生肯定会遭遇各种不顺、倒霉和坎坷。善的心念招感善报,恶的心念招感恶报。\textit{凡人举念,关系最重,发机虽微,果报甚大。(《修心铭》)} 发念之机在十恶,则三途之业报已成,发念之机在十善,则人天之业报已成。《感应篇》所说的“夫心起于善,善虽未为,而吉神已随之;或心起于恶,恶虽未为,而凶神已随之。”断恶修善的关键在于控制起心动念。在身、口、意三业方面,意业最重要,因为念头是行为的先导。邪正之分,在于一念之间,一念得正,人斯正矣;一念入邪,人斯邪矣。《菜根谭》讲守口须密,防意要严!“防意不严,走尽邪蹊。”铭曰:“防心如城,守念如门;邪正出入,一念修分。”修行的核心就是修心,在起心动念上去修,不要停留于表面形式,关键还是对治念头,恶念起时,要立刻断掉!恶最重要的是意恶,所以祖师大德常常教人“从根本修”,把恶的念头统统断除。

\begin{quote}\it
    平生善行善言,都是敷衍浮沉,何尝有一事着实!且满腔意恶,起伏缠绵,犹欲责天美报,如种遍地荆棘,痴痴然望收嘉禾,岂不谬哉!
\end{quote}

\textbf{解析} 俞公行善敷衍浮沉,缺少真心和恒心。这和他满腔意恶是分不开的,充满意恶的人,行善必然虚伪、浮夸,不真心。俞公还想着上天给自己好报,这肯定是办不到的,心里如此污秽险恶,岂能有好报?!遍地荆棘是种不出好庄稼的,要种出好庄稼,首先要把杂草除干净。满腔意恶之人,负能量太重,这种人还想得到好报,这是不可能的。君子以正位凝命,心一定要善,一定要正,内心一定要干净,这样才能得好报。

\begin{quote}\it
    君从今后,凡有贪淫、客气、妄想、诸杂念,先具猛力,一切摒除,收拾干干净净,一顾念头,只理会善一边去;若有力能行的善事,不图报,不务名,不论大小难易,实实落落耐心行去;若力量不能行的,亦要诚诚恳恳,使此善意圆满。
\end{quote}

\textbf{解析} 灶神给建议了,断除负面念头,把心地收拾干净,起念就起善念,做善事不图报,不务名,脚踏实地去做,诚诚恳恳。这才是真心做善事的样子,也就是无私奉献的精神,不图报,不为了名利,这才是高质量的善行,这才能感动天地,如果为了名利而行善,这就大打折扣了。动机要崇高——无私!自私的动机,老天是看不上的。很多人虽然也行善,但他的动机是自私自利,是贪图福报,这样的动机就比较低级了,我们行善一定要有崇高的动机,要学会无私奉献,无私利他,不求任何回报,纯粹无私地行善。

\begin{quote}\it
    第一要忍耐心,第二要永远心,切不可自惰,切不可自欺,久久行之,自有意外效验。
\end{quote}

\textbf{解析} 这条就是讲忍耐心和恒心的,不可懒惰懈怠,不可自欺,要恒久力行,久而久之,自然有效验。一时的热情,很多人都有,有的人初心甚猛,后来就不行了,所以贵在坚持,贵在恒心,不中断,这样不断积行,自然效果非凡。把初心变成恒心,养成习惯,不断坚持下去,心中有信念,行动有力量,相信积累和坚持的力量,终将会迎来质的飞跃。\textit{勤字功夫,第一贵早起,第二贵有恒。(曾国藩家训)} \textit{古今庸人皆以惰败,才人皆以傲败。(《曾国藩家书》)} 要取得成功,必须戒惰戒傲,保持勤奋谦虚,有忍耐心,不浮躁,有恒心,这样才能成事。

\begin{quote}\it
    初行之日,杂念纷乘,非疑则惰,忽忽时日,依旧浮沉。因于家堂所供观音大士前,叩头流血,敬发誓愿:善念真纯,善力精进,倘有丝毫自宽,永堕地狱。每日清晨,虔诵大慈大悲尊号一百声以祈阴相。从此一言一动、一念一时,皆如鬼神在旁,不敢欺肆。
\end{quote}

\textbf{解析} 俞公听了灶神的建议,开始去做了。刚开始依旧浮沉,后来他下大决心了,家里供了观音菩萨,也发了毒誓。关于毒誓的问题,有的戒友也在发,然后破戒后,他们很恐慌,怕自己遭到报应,我个人认为,有些毒誓真的不能随便发的,因为你自己没把握,弄不好会置自己于危险的处境。所以发毒誓不能随便学的,要慎重。念佛持咒是很殊胜的,有佛菩萨的加持,如果戒来戒去,一直不太理想,也可以试试念佛持咒,买个念佛计数器,每天从一千声佛号开始。可以念阿弥陀佛,也可以念观音菩萨圣号。一定要知道日课必须结合实战,邪念一来,马上转成佛号。我看到不少人也念佛,但是日课脱离实战,结果还是破戒。所以一定要注重断念实战,不能仅仅停留于日课,最终是要看实战表现的。

\begin{quote}\it
    凡一切有济于人、有利于物者,不论事之巨细、身之忙闲、人之知不知,力之继不继,皆欢喜行持,委曲成就而后止。随缘方便,广植阴功,且以敦伦、勤学、守谦、忍辱,与夫因果报应之言,逢人化导,惟日不足。每月晦日,即计一月所言所行者,就灶神处为疏以告之。持之既熟,动即万善相随,静则一念不起。如是三年,年五十岁,乃万历二年甲戌会试,张江陵为首辅,辍闱后,访于同乡为子择师,人交口荐公,遂聘赴京师,公挈眷以行。张敬公德品,为缓列入国学。万历四年丙子,附京乡试,遂登科,次年中进士。
\end{quote}

\textbf{解析} 俞公广植阴功,正己化人,坚持三年,后来中了进士。\textit{故吉人语善、视善、行善,一日有三善,三年天必降之福。(《太上感应篇》)} 三年是一个标准,要坚持行善三年,而且是不图回报、不求名利的行善,完全无私奉献,这样去做,肯定会有好报,德行也会随之大幅提升。当然,前提必须是制心——断除意恶!俞公的修心功夫有了飞跃,能做到“静则一念不起”,很不容易,说明他对念头的掌控已经到了很高的程度了。这条也提到了敦伦、勤学、守谦、忍辱,这四条非常重要,《了凡四训》也强调谦德之效,这两本书都在强调谦德,说明谦德的重要性。俞公断意恶、力行善、修德行,坚持三年,人生逆袭了。人交口荐公,有了良好的口碑,这是行善积德得来的口碑,德品提升了,要飞黄腾达了。

\begin{quote}\it
    公居乡,为善益力;其子娶媳,连生七子皆育,悉嗣书香焉。公手书遇灶神记、并实行改过事以训子孙,身享康寿八十八岁,人皆以实行善事回天之报云。
\end{quote}

\textbf{解析} 俞公继续力行善事,家族命运也改变了,文章提到俞公找回了早年走失的儿子,那个父子相认的场景很是感人,俞公太激动了,儿子终于找回来了,让我想起了《等着我》这个节目,多年失散的亲人终于团聚了,那种心情实在太激动了,无以言表,那种猛烈的情感,在相见的那一刻彻底爆发了。后来其子娶媳,儿孙满堂,俞公也得了高寿,他的人生也算圆满了。一人行善,会影响一方的人都有行善积德的意识,看到别人行善得了好报,也增加了自己行善的信心和动力。俞公之前的命运比了凡先生惨多了,前后应试七科,皆不中,生五子四子病夭,唯一的儿子还走失了,生四女仅存其一,妻以哭儿女故,二目皆盲。俞公潦倒终年,贫窘益甚,他没意识到是意恶害了自己,还自认为无大过,等到灶神来点化他,他这才明白自己的问题出在哪,于是下大决心改过、行善,人生真的逆袭了。人意恶重的时候,那个气质和气场相当不好,有一股邪气、怨气、晦气、小家子气,惶惶如丧家之犬,连续剧中这点表现得很好,后来俞公逆袭了,那个气质和气场那是相当好,大气、从容、有德行,满面的善相和德相。

再分享一个古代的故事:清崇明黄永爵,相者决其无子,寿止六十,后有南洋一舟,遇风将覆,黄急出银十两,买渔舟救之,全活十三命。复遇相士骇曰:“君满面阴骘纹,必有盛德,不特有子,且登大魁,己亦上寿矣。”后果生子,名振凤,中康熙己未会魁,己寿九十余善终。命运是可以改的,我们一定要懂得行善积德,你一行善,你的能量场马上就开始变好了,你有意恶,你的能量场马上就变差了。断除意恶,力行善事,这就是改造命运的方法,光行善还不行,必须断除意恶!断恶修善是并重的,不能偏于修善,而不断恶!断恶是前提,否则修善也会虚伪、浮沉。一位戒友说:“行善一直有,可是心瘾超重,多的时候一天好几次,痛不欲生,如果能早点学习,早两年明白观心断念的重要性,而不是心瘾超重的现在,可能我早就戒掉了。”行善是需要的,但一定要学会观心断念,这是修行的核心,不管是断念口诀、思维对治还是念佛持咒,目的都是为了断念,清净自心,观心断念是真正的核心和根本,学会观心断念,才算真正入门。

\paragraph*{总结}

王阳明临终遗言:“\textit{此心光明,亦复何言!}”意恶重,此心污浊、黑暗、肮脏,这样前途必然不容乐观。\textit{心体光明,暗室中有青天;念头暗昧,白日下有厉鬼。(《菜根谭》)} 有一个词叫“佛口蛇心”,最近听到一段开示,就是讲这个问题的,虽然口念佛号,但是心如蛇蝎,这样修行肯定是不行的。什么是蛇心呢?就是负面念头,比如嗔恨、贪婪、嫉妒、傲慢、自私自利、虚伪、争名夺利等,自古以来,多少修行人都是这个问题,修来修去,都没有对治自己的负面念头,还是那么自私狭隘,搞得负能量还是很重。俞公也是这个问题,之前他没有意识到自己的意恶,还自认为自己做了善事,应该得到好报,当命运不济,他就怨天尤人,其实问题就出在他自己身上,灶神点化了他,让他意识到了自己的问题所在。俞公最后逆袭了,是励志的榜样,值得我们学习。我们戒色一定要学会对治负面念头,不仅是邪淫的念头,而是所有负面的念头都要断除,这点极为重要。负面念头多,德行必然差,德行有亏,是很容易破戒的,因为德乃戒之基,地基不稳,戒色大厦很容易坍塌,德不配位,必有灾殃。俞净意公给我们的启示很深刻,我们要认真反省,看看自己还有哪些负面的念头,要坚决断除,然后多发善念,多行善事,不断增加自己的正能量,这样才能越戒越稳固,人生才会越来越好。

下面分享一首诗歌。

\begin{poem}[梦遇俞公记]
    \begin{multicols}{3}
        \centering~\\
        恍恍然我进入了梦乡 \\ 穿越到明嘉靖年间 \\ 江西俞公的住所 \\ 那间竹林小屋 \\ 场景是如此熟悉 \\ 我敲了敲门 \\ 就像当年灶神敲门一样 \\ 俞公打开了门 \\ 我表明来意 \\ 想亲自拜会一下他 \\ 听他讲一下过去的经历 \\ 他把我迎进屋 \\ 相对而坐 \\ 俞公娓娓道来 \\ 讲到伤心处时老泪纵横 \\ 语重心长地说:“要断意恶啊! \\ 那些年意恶把我害惨了啊!” \\ 我表示同情 \\ 就在这时突然场景急速向后退 \\ 我睁开双眼回到了 21 世纪 \\ ——我的床上 \\ 俞公那句:“要断意恶啊!” \\ 犹回荡在耳畔
    \end{multicols}
\end{poem}

下面分享一本书。

\begin{book}[《佛光菜根谭》,星云大师]
    《佛光菜根谭》是星云大师几十年来弘扬佛法及教化人心的智慧精华。星云大师自幼熟读中国古代典籍,熟习中国传统文化,并把这些积淀融合到佛教立身处世的理念和智慧之中。本书字字珠玑,言简意赅,文意通俗而不失庄严,文字平实中见优美,是大众丰美的精神食粮。让您在工作闲暇之余,能品味人生的指南,分享到佛教菁华。《佛光菜根谭》代表了大师思想的缩影,阅读此书可进而获得大师的圆融的智慧,带领您迈向成功人生。星云法师的语录我很欣赏,比如:

    \begin{quotation}
        德行善举,是永不失败的投资;慈悲大爱,是永不战败的盾牌。

        以诚感人,人也报之以诚;以德服人,人也报之以德。

        有大胸襟,方有大格局;有大格局,方能成大事。

        宽恕能恢宏气度,包容能促进和谐,慈悲能助长道德,喜舍能增加人缘。

        心善,则触目所及皆真善美;心恶,则言行举止皆贪瞋痴。

        看手相,看面相,不如看心相。

        人生最大的敌人,不是别人,而是自己;人生最大的胜利,不是制敌,而是克己。
    \end{quotation}

    读《佛光菜根谭》,可以懂得很多修行的道理和为人处世的道理,对于指导人生很有意义。星云法师的《佛光菜根谭》有好几本,应该是一个系列,都可以看看。人生一大快事,就是学习圣贤教育,可以学到大智慧,有了大智慧,人生就能过得更幸福、更从容、更有价值。
\end{book}
