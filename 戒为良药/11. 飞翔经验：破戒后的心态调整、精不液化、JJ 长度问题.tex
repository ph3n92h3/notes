\subsection{破戒后的心态调整、精不液化、JJ 长度问题}

\subsubsection{破戒后的心态调整}

戒色吧每天都有破戒的戒友,破戒后发帖总是后悔得不行,有些戒友会把破戒叫“阵亡”,这种叫法我觉得很好,戒色就像一场战役,大家都是一条战壕里的兄弟,看到别人阵亡的确很痛心,当然更要引以为戒,保持警惕,粉色子弹不断从对面阵地射过来,戒色文章就是我们的防弹衣,我们必须多学习戒色文章提高觉悟和定力等级,否则如果你放松警惕,下一个阵亡的可能就是你了。看到不良图片和视频马上要避开,畏色如畏虎,避色如避箭,这种防范意识一定要有。要戒掉,意识一定要强,意识是通过学习获得的,有了意识,有了警惕,戒色才有望成功。

对于破戒的戒友,也不用灰心丧气,没有人可以一次戒除成功,在成功前必定有无数次的失败,我在彻底戒除前,也失败了无数次,我已经记不清我破过多少次戒了,我有十几年都没走出那个怪圈,因为那时的我处在无知的状态,没人指点我,没人告诉我该怎么做,后来我通过学习开悟后,定力就上去了,大概学习了一年多的戒色文章和养生知识,认识有了极大的飞跃,有了更深刻的思考和认识,是在这种思想状态下才彻底戒掉的。

一般戒色后容易犯两类错误:

\begin{multicols}{2}
    \begin{itemize}
        \item 只戒不学
        \item 只戒不养
    \end{itemize}
\end{multicols}

只戒不学就是强戒和盲戒,强戒和盲戒注定失败,因为定力没提高,欲望休眠期过后就会进入破戒高峰期,到了破戒高峰期,破戒的欲望挡都挡不住,往往是连续破戒,前功尽弃。我在前面的文章中无数次地提到了学习戒色文章的重要性,因为不学习你就不知道,不学习你觉悟就不会提高,不学习就无法开悟。当你学到一定程度,定力就会升级,定力上去了,就能降伏心魔,否则见一次心魔,就失败一次。刚开始戒,难免定力等级低,假设心魔等级 100,你定力等级只有 15,你怎么能和心魔抗衡?只有通过学习你的定力才会上去,定力到了一定程度,心魔就动不了你了。失败不可怕,可怕的是不学习,不学习的结果就是,还会继续不断地破戒,出不了怪圈。

只戒不养在戒友中也相当普遍,很多戒友缺乏的正是养生意识,养生意识也是通过学习来获得的,多看名家讲座视频和养生类书籍,这对于身体的恢复是大有好处的。很多人光戒不养生,照样久坐熬夜抽烟,不注重情绪管理,这样的人恢复情况不会乐观,也容易出现反复,而有些戒友戒掉后积极锻炼,注重养生之道,不熬夜不久坐,这样恢复就比较快了。纵欲是伐生,戒色是养生,但戒色只是养生的第一步,光戒是不行的,光戒远远不够,因为伤肾气的方式不仅仅只有 SY 一种,熬夜、久坐、生气、吃冷饮、吹空调等,同样也很伤肾气。

中医专门有讲到五劳七伤:

\begin{description}
    \item[五劳] 久视伤血,久卧伤气,久坐伤肉,久立伤骨,久行伤筋,是谓五劳所伤。
    \item[七伤] 大饱伤脾,大怒气逆伤肝,强力举重久坐湿地伤肾,形寒饮冷伤肺,忧愁思虑伤心,风雨寒暑伤形,恐惧不节伤志。
\end{description}

五劳七伤也要注意避免,养生是一门很深刻的学问,是点滴的功夫,各方面都要注意,等你养生的意识提高后,对你身体更好地恢复是极其有利的。因为你知道了什么该做,什么不该做,伤肾气的事情绝对不做。

破戒后的类型:

\begin{itemize}
    \item 越挫越勇,斗志依然
    \item 信心受到打击,不积极了
    \item 起了退心,做了逃兵
\end{itemize}

戒友在破戒后要好好忏悔下,但不要太自责,心理压力也不要太大,一切可以重新开始,把戒色文章捡起来是真的,虽然破戒了,但你要看到,只要坚持戒色文章的学习,终有一天定力会达到的,定力修到了,自然就会彻底戒掉,就像爬山一样,只要坚持下去,终究会爬到山顶的,很多人因为中途劳累就不爬了,真正有决心和勇气的人会坚持到底,绝不会轻易放弃戒色。

经过我的研究,很多戒友之所以放弃,是因为对戒断反应认识不足,一般戒掉后都会出症状,坚持戒色,按时作息,积极锻炼,戒断症状就会消失的,很多人即使戒了半年乃至一年也是会出现反复的,出现反复很正常,不用担心,即使正常人也有身体不适的时候,这和外感有关,还和季节的转换有关,每个季节人体的阳气水平都不同,还有遗精后也容易出现反复,所以要尽量减少遗精次数。另外,有老婆和女友的戒友也容易动摇决心,因为有女人是戒色的一大障碍,有了女人,身体症状就很难恢复了。还有一部分戒友,对 SY 的恶果认识不深,还在认同适度无害论,这类戒友也比较容易起退心。

\subsubsection{精不液化}

下面再来谈下精不液化的问题。

精不液化的问题出现得也比较多,也比较普遍,射出来的东西像果冻或者结晶体,很多戒友看到了都吓一跳,很恐慌,其实这种精不液化的现象非常多,在戒色吧,几乎每天都能看见有这类提问。引起精不液化的主要病因是前列腺炎,精不液化是会影响到生育功能和精子质量的,很多人去检查不是弱精、死精、就是畸形的精子,甚至是无精。

在中医来讲,精不液化的主要病因如下:

\begin{adjustwidth}{-3.5em}{-3.5em}
    \begin{multicols}{2}
        \begin{itemize}
            \item 先天肾阳不足,或后天失养,大病久病,戕伐肾阳
            \item 寒邪外袭,损伤肾阳,均可使精寒凝,不得液化
            \item 酒色房劳过度,频施伐泄,或劳心太甚,或五志化火
            \item 平素嗜食辛辣、醇甘厚腻,湿热内蕴,或外感湿毒
            \item 过食寒凉冷饮,损伤脾阳,或他病伤及脾阳,脾虚及肾
            \item 气虚血瘀或血淤体质,精窍淤阻,精亦不液化
        \end{itemize}
    \end{multicols}
\end{adjustwidth}

一般戒友出现精不液化的主要原因就是第三点:纵欲过度。纵欲过度后就容易出现这类问题,而 SY 只要开始,就会一发不可收拾,几乎没人能做到不过度,这个“度”也没几个人能知道。所以必须彻底戒掉 SY 恶习,否则身体就万难痊愈了,生育功能都会受到影响,我在贴吧回答问题,已经遇见无数因为 SY 恶习而导致无法生育的案例了,唯有戒掉 SY,学会养生,养足肾气,精子质量才有望恢复,否则真的就废掉了,怎么办?!

\subsubsection{JJ 长度问题}

下面来谈下 JJ 的长度问题,JJ 的长度是个敏感话题。男人对 JJ 长度是很在意的,普遍认为越大越好,JJ 大者有自尊,JJ 短小者自卑。我以前也这样认为,后来我悟道后就不这样认为了,JJ 长度里面有大玄机,不是想象的那么简单,如果不悟道,很少有人知道有这么回事,很少有人知道其背后的深刻道理。今天我就来谈谈这个话题。

JJ 是属肝的,因为肝主筋,很多戒友有反映 SY 后 JJ 短小了,没自信。其实 SY 的确能影响到 JJ 的发育,因为肝肾同源,SY 伤肾也伤肝,而 JJ 属肝,所以会影响到 JJ 的长度,但是影响 JJ 长度的还有基因等其他因素,如果你基因不行,再加上 SY 恶习,这样出现 JJ 短小的可能性就比较大了。

JJ 的长度一般 12 \unit{\cm} 就可以了,10 \unit{\cm} 也还行,JJ 不需要太大,JJ 太大容易招邪,《柳庄神相》里就有讲到:JJ 粗大者主下贱,大者招凶,人必贱。这和大家的认识完全相反,大家普遍认为越大越好,越大越爽,以大为好,以大为强,其实这种认识是比较肤浅的,为什么粗大者反而不好呢?为什么粗大者反而下贱呢?这里面就涉及到中医医理了。如果不懂医理,就会在认识上继续错下去,可能到死都不明白这个道理。

我曾经把 SY 比作购物,花的是肾气,购买的是短暂的快感,而 JJ 粗大者一般“购买力”都比较强,性欲比较旺盛,表面看这似乎是好事,但福兮祸所伏,购买的快感越多,花掉的肾气就越多,这样身体垮掉的可能性就越大。别看他现在强,因为纵欲,将来有他好受的,将来弄不好会早泄阳痿,各种疾病缠身,所以说欲不可强,越强越亏。粗大者招邪,这个邪指的就是邪淫,中医的圣经《黄帝内经》第一章《上古天真论》就专门谈到了纵欲的危害。古代名医经过上千年的经验总结,深知邪淫的危害,我看了很多医案,从古至今,因为纵欲身体垮掉的例子实在太多,简直多如牛毛,如果你有机会深入了解,就会知道邪淫到底有多厉害了,洪水猛兽其实并不夸张,当你肾气充足时,感觉不会很明显,当你肾气耗损到一定程度,恶果就会越来越明显,随着年龄的上升,各种疾病都会找上门来。肾气足,万邪熄,肾气虚,百病丛生!

那又为何“粗大者主下贱”呢?因为中医:肾上通于脑,SY 伤肾必伤脑力,注意力和记忆力乃至意志力都会不同程度下降,一个人脑力不行,各方面就不容易成功,主下贱也就必然了。那些真正的成功人士往往都是严格自律之人,不自律的人只会成功一时,不会长久。性放纵、性混乱直接造成的就是脑力不行,犹太人有严格的性禁忌,对性有极其严格的规定,这样的种族智商就比较高,全世界诺贝尔奖获奖者中 22\% 是犹太人。

得道之人绝对不会认为性欲强是好事,因为中医:虚则亢。虚则亢再发展下去就是早泄阳痿,各种疾病都会找上门来,这就叫“亢龙有悔”,得道之人也绝对不会认为 JJ 粗大是好事,反而会认为 JJ 粗大是坏事,背后的道理相信看了我这篇文章的戒友都会了解。
