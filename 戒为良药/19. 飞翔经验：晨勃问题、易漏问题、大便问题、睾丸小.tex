\subsection{晨勃问题、易漏问题、大便问题、睾丸小}

\subsubsection{晨勃问题}

晨勃是戒友普遍关心的问题,很多戒友都把晨勃作为自己肾气是否恢复的标志,有晨勃就高兴,没晨勃就心事重重,患得患失,搞得一天都没好心情。其实身体是否恢复,从很多方面都可以表现出来,你自己照照镜子,眼睛是否有神采、气色如何、整个精气神如何、握力如何、腿是否有力,其他的不适症状是否缓解乃至消失,这些都可以看做身体是否恢复的指标,而晨勃只是众多恢复指标中的一个而已。

只要坚持戒色,晨勃是会恢复的,如果你只有十几岁,那晨勃恢复相对比较快,如果上了 25 岁或者伤得比较严重,那晨勃恢复就比较慢了。有的戒友是戒色养生两百多天才恢复晨勃,我当初大概用了一年多才恢复晨勃,因为那时我已经 27 岁了,十几年的纵欲经历让我的肾气亏损严重,所以我用了一年多才恢复晨勃。我现在虽然恢复晨勃了,但已经不可能和十几岁时相提并论了,那时在发育期,吃得下,并且吃得好,那时晨勃质量高,只要 SY 不是很频繁,一周可以晨勃五天左右。但这种晨勃频率也只维持了两年左右,也就是刚发育那会。后来 SY 频繁,晨勃就很少了。现在我晨勃的频率大概是一周三次左右,有时质量高,有时质量一般,我现在一周吃五天素,所以一周能有三次晨勃已经很不错了。我以前出现过早泄和阳痿倾向,一般有早泄阳痿倾向的人,晨勃要恢复很难,至少需要大半年以上了。

我现在基本不关心晨勃了,有最好,没有也无所谓了,因为身体恢复并不是晨勃一个指标,其他方面我恢复得很好,而且影响晨勃的因素也很多,一般如下:

\begin{multicols}{2}
    \begin{description}
        \item[饮食因素] 补肾食物的摄入量
        \item[锻炼因素] 适量的有氧运动和力量训练是可以帮助晨勃恢复的
        \item[情绪因素] 不良情绪会影响到晨勃,比如生气、压力。
        \item[年纪因素] 青春期更容易晨勃,年纪越大,晨勃越容易消失
        \item[体质因素] 先天体质好的人容易有晨勃
        \item[疾病因素] 没其他慢性疾病的人更容易有晨勃
        \item[撸龄因素] 一般有五年以上的人晨勃容易消失
        \item[季节因素] 不同季节晨勃的频率和强度都有差别
        \item[每月因素] 每个月的不同时间段,人体阳气是不同的,出现晨勃也会不同
        \item[憋尿因素] 有时憋尿可以导致晨勃的发生。
        \item[劳累因素] 过于劳累会伤到肾气,晨勃容易消失
        \item[睡眠因素] 睡眠质量高的人,更容易有晨勃
        \item[频遗因素] 频繁遗精后,晨勃容易消失
        \item[戒断因素] 刚开始戒,会有欲望休眠期,晨勃易消失,坚持下去又会恢复
        \item[] 是否有其他不良生活习惯,比如熬夜久坐抽烟,都会伤到肾气,影响到晨勃
    \end{description}
\end{multicols}

晨勃是指无 YY 的晨勃,如果有 YY,就不叫晨勃了,那是你念头导致的勃起。晨勃其实很危险,为什么说晨勃危险呢?一般人并不知道这个道理。因为很多人定力尚浅,一有晨勃就挺高兴的,然后会趁着勃起之势,去摸自己的 JJ,这一摸就完蛋了,不知不觉就又开始 SY 了,这种晨勃破戒的情况在周末是非常普遍的。我现在晨勃,一不在乎,二不去看,三不去摸,任其自生自灭,如果你很在意晨勃,弄不好就会走火入魔。你不在意它,它自己就下去了,不会导致破戒,很多戒友有赖床习惯,这样如果有晨勃,就非常容易破戒了。切记!

很多人会说自己的晨勃时有时无,其实这个很正常的,你以为你永远都在发育期吗?就是你在发育期,要做到每天都晨勃也很难,因为影响晨勃的因素有很多,不可能一年 365 天,天天晨勃,那是不现实的,为何晨勃会时有时无呢,这和季节、生活习惯、饮食、年纪都有密切关系,一般吃得好的人,并且消化吸收能力强的人,更容易有晨勃。以前我经常吃荤菜,那段时间我晨勃相对多些,现在我一周吃五天素,晨勃相对较少了,但是心里却清净很多,不容易起邪念。戒色期间吃素不错,有利于控制欲望,吃肉多容易助长欲望,如果定力尚浅,那就很容易出现破戒。吃素是一种选择,要看个人的接受程度,我建议广大戒友从减少吃肉开始,能够完全吃素我也是非常支持的。

\subsubsection{易漏问题}

下面谈下易漏问题。

易漏问题,指的是长期纵欲,肾气不固,在很多情况下都容易漏精,具体如下:

\begin{multicols}{2}
    \begin{itemize}
        \item 小便后漏出
        \item 大便后漏出
        \item 和女友打电话漏出
        \item 滑精
    \end{itemize}
\end{multicols}

这种易漏问题,打个比方大家就明白了,就好比一个水龙头用久了,关不紧了,就老漏。小便后漏出的情况其实挺多的,其实就是肾气不固的表现,大便后漏出的情况更普遍,特别是刚开始戒色时,更容易出现这种情况。还有不少戒友反映,和女友打电话,没聊到敏感内容,但也漏了,这其实和潜在的条件反射有关,有女友的戒友,要戒色比较难,所以最好和女友沟通一下,尽量避免婚前性,自己则要加强修心,把握好交往分寸。滑精的戒友也比较多,如果偶尔一次滑精,倒不必太担心,如果持续滑精,那就要及时就医治疗了,另外,建议多做固肾功。一般坚持戒色养生,积极锻炼,小便后和大便后漏出的情况会自动消失的,我以前也出现过大小便后漏出的情况,后来我坚持做固肾功,这种情况就极少出现了。

当然,漏出的情况不止这四种,比如 YY 和紧张也会导致漏出的情况,即使你肾气充足,沉迷于 YY,也会漏出。所以要坚决杜绝 YY,做到念起即断,并且多学习戒色文章提高觉悟和定力,这样才有望真正戒除 SY 恶习。

\subsubsection{大便问题}

再来谈下大便问题。

大便问题在戒色吧的帖子里也时常出现,因为中医:肾主两便,大便和小便,肾一虚,大小便就容易出问题,小便问题就不用多说了,相信有前列腺炎的戒友深有体会。

肾虚导致的大便问题一般分为两类:

\begin{itemize}
    \item 腹泻或者大便不成形
    \item 便秘
\end{itemize}

导致腹泻和便秘的原因有很多,其中就有肾虚这个原因。大家可能会记得小时候的大便都非常好,一是颜色好,二是形状好。小孩子阳气充足,童体未破,如果没有其他疾病,一般消化吸收能力都是非常好的,大了以后就不同了,一是 SY 纵欲,二是冷饮或者刺激性的食物吃多了,或者饮食不节、暴饮暴食等原因,都容易出现消化系统的功能障碍,去检查也检查不出来什么,就是功能紊乱了。小孩吃点冷饮没事,因为小孩阳气足,能化掉。纵欲以后就不行了,一吃就容易出现胃肠功能的紊乱。所以一定要注意忌口,不要贪吃冷辣的东西。如果你有腹泻或者便秘的烦恼,最好去就医治疗,然后配合戒色和养生,积极锻炼,注意保持饮食清淡,这样慢慢地就能调理过来。便秘和腹泻的问题我都经历过,便秘是 19 岁开始的,困扰了我两年多,就是因为纵欲加吃冷饮导致的,后来 26 岁又开始腹泻,就是吹空调、暴饮暴食、加上纵欲导致的,我彻底戒色后,经常艾灸足三里和神阙穴,慢慢地就调理过来了,用了一年多时间。

\subsubsection{睾丸小}

最后再来谈下睾丸小的问题。

睾丸小睾丸萎缩的问题,在戒友中也时常出现,SY 恶习导致睾丸的问题一般如下:

\begin{multicols}{4}
    \begin{itemize}
        \item 睾丸下垂,不收缩
        \item 睾丸萎缩变小
        \item 阴囊潮湿
        \item 精索
        \item 阴囊湿疹
        \item 阴囊肿大
        \item 阴囊瘙痒
        \item 睾丸疼痛
        \item 睾丸炎症
        \item 睾丸囊肿
        \item 隐睾
        \item 其他病变
    \end{itemize}
\end{multicols}

睾丸萎缩的戒友最好去医院做个检查,确诊一下病情的严重程度。

睾丸萎缩,是指男子睾丸缩小痿软病证,以一例或双侧睾丸萎缩,既小又软为特征,大多数能引起不育,多继发于腮腺炎或外伤,也有先天者。多因肾气亏乏,或病邪损伤引起。先天性睾丸发育不良者不易治愈,继发性睾丸萎缩者亦需耐心调治,中医辨证治疗可以取得一定效果。

SY 恶习导致肾气亏损,继而导致睾丸萎缩,这在戒友中屡见不鲜,所以一旦出现睾丸萎缩的情况,一要积极治疗,二要配合戒色和养生。这样把肾气养足后,才有望恢复正常。

有的戒友还会问到睾丸大小不一样的问题,一般正常人的睾丸都不是等大的,但如果差别特别大,那最好去做个检查,如果差别不大,那就不必太担心。
