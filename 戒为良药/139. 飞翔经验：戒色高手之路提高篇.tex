\subsection{戒色高手之路提高篇}

在第 120 季的文章,我分享了高手之路基础篇,那篇文章提到了将来要写个提高篇,后来有几位戒友也问起过,这季就写个高手之路提高篇,把更多的戒色心得和广大戒友做一个分享。这季内容是比较重要的,现在我写的文章都是高度的总结,精华要点的强调和提炼。其实高手最注重的都是基础要点,因为基础要点往往是最重要的,基础要点真正落实了,戒起来就有把握了,基本功一定要扎实。

\subsubsection{转贪六思维}

看过很多戒友的案例,其中贪恋的问题是比较突出的,就是一个字:贪!心里对邪淫内容的贪恋非常严重,毕竟看了那么多年,强化了那么多年,几乎都形成条件反射了,心里非常贪婪。戒色一定要转变贪恋,克服贪恋,否则不管戒多久,都难以成功。下面我列举了六种转变思维的方法,如果想对治贪恋,这六种思维应该经常思维,坚持一段时间,贪恋的问题就会大大改善。沉迷色情的人,脑中有一条色情“车辙”,国外戒色文章专门提到了这一点,只有通过不断对治,这条车辙才能慢慢淡化,最后恢复正常。

\paragraph{第一思维:邪淫的报应}

邪淫这个概念自古就有,邪淫伤身败德,手淫属于邪淫的一种。关于色情危害、邪淫危害的文章有很多,不仅是对身心健康的危害,也会影响一个人的运势和福报。上次看到一个戒友的告诫,他说还没伤到神经症的人,赶紧戒了,否则伤到神经症的程度,那种痛苦让人难以承受。现代社会很多人沉迷游戏,经常熬夜久坐,加上经常看黄手淫,十几岁就得上神经症了,这类人现在很常见了。姑且不论神经症,伤精的其他症状也是挺痛苦的,伤精会导致免疫力下降,身体失调,会导致很多的症状表现。不仅脑力下降,精力下降,还会导致诸多慢性病缠身,影响家庭和谐,影响社交能力,影响容貌气质,影响自信和底气。人一思淫,心田即暗,中正之心已邪,则光明正大之气遂失。人没了正能量,充满了负能量和戾气,磁场很不好,这样的人能不倒霉吗?总之邪淫的报应太多了,影响生活的方方面面,一定要多思维危害,警醒自己!多思维邪淫的苦,才能下最大决心去狠戒!男人必须对自己狠一点!

\paragraph{第二思维:不净观}

作为戒者,不净观也是要多思维的,因为这可以有效对治贪恋,不净观在修行方面是比较常见的内容,就是为了对治贪恋。不净观通过观想身体的种种污秽不净现象,消除自身对欲望的贪恋,是对治贪恋的关键方法,是修持的重要法门。不一定要看那些恶心的图片,自己思维一下不净观的文字即可,对于不净观要正确理解,并非不尊重女性,而是为了对治贪恋。通过坚持思维不净观,会有一个感受,那就是自己的贪恋在逐步减弱,没有之前那么强了,这就是对治的效果。

\paragraph{第三思维:没意思}

请记住一点:射完就没意思!快感过后就是巨大的空虚和无尽的悔恨,射完了再看之前的片子,就索然无味了,再诱惑的片子,射完后都是垃圾!感觉特别没意思,也感到上当受骗了,自己最宝贵的肾精已经失去了。射完后就满目疮痍,内心一片悲凉,心地荒凉,感觉特没意思,特别空虚,也没有动力去奋斗自己的人生理想了,变得很颓废,这是能量下降的表现。记住那个没意思,不要去贪恋,最后肯定没意思,肯定后悔。

\paragraph{第四思维:君子想}

男子汉大丈夫应该顶天立地,岂能躲在阴暗的角落偷偷看黄手淫?真的是龌龊猥琐至极!丢尽了祖宗的颜面!我们要做正人君子,君子第一修为就是戒色,所有的邪淫行为一定要戒掉,做正能量的君子,不做邪淫的猥琐无耻之徒。把浩然正气提起来,吾善养吾浩然之气!光明正大、光明磊落、刚正不阿!要拿君子的标准来严格要求自己!不看黄,不贪色,不犯邪淫!有高雅的品位与追求,有崇高的利他精神。大家读《论语》会发现,这里面经常出现一个词:君子。“君子”是孔子心目中理想的人格标准,君子的力量始于人格与内心,有德行,有风度,君子怀仁、好公、好义、助人、守纪、谦虚、自省、坦荡、积极向上、信念坚定、不抱怨他人、不愤世嫉俗。君子务本,主忠信,勇于面对困难,修己安人,守孝道,懂得戒邪淫。

\paragraph{第五思维:作亲想}

\textit{亲想者,见老者作母想,长者作姊想,少者作妹想,幼者作女想。欲心纵盛,断不敢于母、姊、妹、女边起不正念。(印光大师)} 作亲想,是一种提起人伦规范的思维对治,看见异性就像对待自己的亲人一样尊敬,克己复礼,修己以敬!要尊敬别人,不可邪淫别人,要提起人伦规范,不能做禽兽之举。

\paragraph{第六思维:慈悲观}

怜悯众生,要有慈悲的心态,这样也能转变贪恋,贪恋是起邪念,是低频负面的念头,慈悲是高频的心态,慈悲的人自有一种庄严。修习慈悲观,内心也会得到快乐、安宁,从而保持一种和乐之象。修习慈悲观不仅可以对治怨嗔,也可以对治邪淫的贪恋,因为慈悲,所以不会对别人起那种邪念,一方面尊重别人,另外就是于心不忍。

小结:通过这六种转贪的思维,慢慢贪恋的心态就会减弱,这需要坚持对治,坚持思维,坚持一段时间,就会发现贪恋没那么强了,已经变弱许多了,这点我深有体会,我坚持对治到现在,贪恋已经变得很弱了,不敢说一点没有,但已经很弱了,很容易就能克服。


\subsubsection{学习高手的操作}

高手之所以成为高手,就是因为高手紧抓实战!以实战为核心!实战是根本!实战不行,肯定破戒!破得一塌糊涂,被心魔虐成狗!当然戒色是综合修为,其他方面也很重要,但最重要的还是实战,就像打仗要看实战,不是看嘴皮子,最后是要上战场的,实战软弱,肯定被虐!我们要学习高手的思维对治,学习高手的实战操作,学习高手的实战心得体会,这样去借鉴,去练习,也能很快变成高手。高手的东西都是高度提炼的,一针见血,抓住实战要点,事半功倍!

\begin{case}
    来戒色吧半年多了,整整半年没有一次手淫,意淫也是一秒内就断掉,很庆幸能有这种成绩。
    \subparagraph{分析} 断念强,戒色就成功了一大半了,因为念头导致行为,能断念,就能做到不破。关于断念的理论是很多的,我已经把心魔的套路基本研究透了,这样实战时就能知己知彼,加上不断练习观心断念,断力变得越来越强,就能做到不被心魔攻破。高手的断念操作就是“判定强,断念快”,有威慑力!之所以判定强,就是已经熟知心魔的进攻套路,做到心中有数,实战不慌,冷静沉着。之所以断念快,就是平时坚持练习的结果,没有练习的积累,是无法变得那么强的!就像练级一样,不可能一天就能变得那么强,需要系统的练习积累。之所以有威慑力,就是断力强的缘故,有了威慑力,念头上来就会变少,不敢上来了。
\end{case}

\begin{case}
    最近意识到了断念的重要性,昨晚我脑子里的邪念一觉察就没了,真的很微妙很喜悦,脑子里瞬间进入了积极向上的状态,来一个念头就斩一个念头。
    \subparagraph{分析} 很多新人没意识到断念的重要性,自然不会下功夫去练习,等到他们真正意识到了,真正下功夫去练习了,才有望突破怪圈。看看高手怎么断念的,高手面对诱惑是什么表现,高手是怎么落实戒色十规的!观心断念是实战核心,也是基本功,总摄诸法,一定要充分重视起来。高手之所以成为高手,就是实战强!说句大实话,就是那些诋毁观心断念的人,也不得不使用观心断念,除非他们不想戒了。观心断念是实战根本和核心,偏离这一根本和核心,必然失败。高手的实战特点:决心强,断速快,断力大!菜鸟的实战特点:决心弱,贪恋强,断速慢,断力小。一对比就知道了差距在哪里!要学习高手的强悍操作!实战强,才能立于不败之地!
\end{case}

\begin{case}
    屡戒屡败的根本原因是断念太差,同时正气太弱,心灵得不到彻底的改变,由此我吸取教训注重断念,提升正气,戒色天数就开始突飞猛进了。
    \subparagraph{分析} 这位戒友意识到自己问题了,断念要强,正气也要强,断念靠练习,正气靠改过和行善积德,这两方面都是需要强化的,这两方面做好了,戒色天数自然突飞猛进。断念是实战核心,断念太差,自然越戒越差。正气也很重要,正气太弱,负能量太强,也很容易破戒。如果正气起来了,邪念会减少的,这样有利于戒色,所以断念和行善都是很重要的。
\end{case}

\begin{case}
    太感谢飞翔哥了,最近一个月有两次顿悟,第一次体会到了“觉”的含义,兴奋了好几天,断念水平又上去了,但是有时又总感觉少点什么,最近又理解了“看”的意义,先觉后看,念头就没有了,真的好开心。
    \subparagraph{分析} 有了顿悟就会很兴奋,一下就明白了,明白后实战水平会有一个大的提升,就像坐电梯一样。“觉”这个字是断念口诀的核心精髓,念起即觉,觉之即无。这个觉其实就是看,两个字是一个意思,觉即觉察,和看一个意思,看比较口语化,两个字都很好。断念实战的关键就是提升觉察力,觉察力强了,自然可以降伏其心!觉察力到了,一看念头,念头就没了,这就是觉之即无。能做到觉之即无,很不错!这是高阶的操作。刚开始是发现念头来了,马上念口诀或者念佛来转,进入高阶,觉察力强了,就可以做到一觉即空。
\end{case}

\subsubsection{记住:高手曾经也是失败者}

所有的戒色高手之前都曾经失败过,而且是失败过很多次,我也曾经浮躁过,甚至还放弃过,后来被症状逼得没办法,又回到了戒色这条路上,再次下大决心,再次开始戒色,通过不断学习提高觉悟,提升断念实战水平,才最终冲破了怪圈。我看到现在很多新人都比较浮躁,看不进戒色文章,失败了就消极思考,自暴自弃,这让我想起了我的曾经,那个无知的我,当年也是这样。后来我之所以蜕变了,就是因为学习,学习大德开示,学习中医理论,从大德开示里我学到了修心的方法,从而主宰了自己的内心,做到了降伏其心。通过学习中医理论,我知道了手淫导致的各种症状,背后的医理是什么,也知道了手淫的危害原来那么大。所以,学习是非常重要的,即使暂时失败,也要坚持学习,保证日课,贵在坚持,才能迎来突破。

高手曾经甚至还不如现在的你,但高手勇于尝试,善于学习和总结,不断完善自己,所以才有了后来的逆袭。高手是从失败中崛起的,经历了很多次失败,交了太多的“精费”,也曾经一直在犯同样的错误而不觉悟,后来还是通过学习获得了顿悟,一有顿悟,很容易就能突破。对于屡戒屡败的人,我的告诫就是,不要消极思考,要激励自己,勇于下决心,勇于学习,拿出血战到底的气概,一定可以逆袭的。我看过很多戒色高手,他们都有一个特质,那就是勇猛精进!干劲冲天!有这样的架势,才有望冲破怪圈。怪圈就像一张网,冲劲够足,才能冲破,如果消极思考,就会变得懈怠和无力,消极思考危害很大,一定要避免。高手有亮剑的精神,有那种勇气和决心,这样才能成事,高手有很强的执行力,敢于尝试,敢于投入,暂时的失败反而会激起他们更大的决心,我们要向高手学习。

我写这条就是为了激励大家,不要被失败打败,而是要从失败中顽强崛起,要积极思考,多鼓励和激励自己!

\subsubsection{高手是怎么做到的}

高手能戒几年,肯定是有原因的,有的新人都戒不了几天,“大神都是被虐出来的!”这句话有一定道理,至少刚开始是这样的,因为刚开始很多人觉悟低,断念差,贪恋重,实战意识都没有,脑子里还有很多思想误区,所以在这个阶段肯定会被虐。我在戒色吧看多了,刚开始大家都差不多的,但过了一段时间,有的戒友就变强了,开始慢慢成为高手了,而有的人还是没有把握关键要点,还是屡戒屡破。能成为高手的人,肯定是善于学习的人,他们善于总结和反省,不断完善和提升自己的实战表现,关键是持续学习、练习和总结,这样就能慢慢变成高手,暂时的失败不可怕,一定要不断完善自己,补强自己,优化自己的实战表现,专业而系统地戒色,严格落实戒色十规。

\subsubsection{高手注重落实}

为什么要强调落实?因为方法再好,道理再对,没有落实,一切等于零!只要去落实,就会有一个结果,或好或坏,不断完善,就能越做越好。菜鸟总是喊:“为什么我又破了?”他们一方面缺少落实,另外就是缺少恒心,要坚持落实,不断落实,这样才能有一个好的结果,高手都在落实戒色十规,戒色十规是对成功普遍规律的高度总结和概括,所有关键要点都在里面了,关键是落实到何种程度,是不是能真正做到。

\subsubsection{高手都注重积累}

请记住:没有积累,没有人可以成功!高手都在进行日课的积累!一天天,一月月,专业系统地积累。如果没有学习的积累,没人可以从小学念到大学,必须需要积累,才能不断突破。修行方面有一个词:“达量”!也就是积累到量了,量变产生质变,到时一下就飞跃了,一下就顿悟了,一下就懂了,懂了之后,愈练愈精,愈练愈强!方向越来越明确,强化的要点也越来越简洁。我观察过菜鸟的特点,一,破戒后习惯消极思考;二,浮躁;三,不懂得积累,不坚持积累;四,决心不够强;五,不注重反省和总结。高手有决心,有恒心,注重反省和总结,注重落实,注重积累,是这样一步步变成高手的!

\subsubsection{高手注意细节}

一些戒色的细节一定要注意,比如慎独、不要赖床、注意姿势、遗精因素、喝酒吃肉等,姿势的问题要引起足够的重视,卧姿比较舒服,容易放松警惕,所以卧姿时要提高警惕,尽量不要躺着上网,看到很多破戒案例都是躺着手机上网,放松警惕,看了擦边图而破戒的。坐姿或卧姿时,如果意淫没及时断除,这时应该马上站起来,这样有利于提升警惕,另外站起来也有助于消除微妙的感觉,还可以做做固肾功和提肛,有时意淫断晚了,睾丸会松,身体会出现略微的不适,这时可以来回走动一下,做做提踵、固肾功和提肛。饮食方面也需要注意,过年期间有戒友反馈吃肉过多,出现频遗,过年要注意减少吃肉,尽量不要喝酒,这都是细节,有时细节是会间接决定成败的,这方面要多注意,多总结。

\subsubsection{高手不会沉迷游戏}

沉迷游戏会导致戒色状态严重下滑,还会影响自己的学业或工作,沉迷游戏危害真的很大,之前第 134 季我专门讲了这个问题,一定要戒沉迷游戏,否则戒色很难成功。在成长的过程中,最让人上瘾的应该就是游戏和色情了,这是除了毒品以外,最让人上瘾的东西了。我现在几乎不玩游戏了,即使玩,也是玩时间很短的游戏,绝对不会去玩网游,太浪费时间和精力。前几天看了一篇戒色文章,那位戒友戒色四年,文章里面专门提到了戒沉迷游戏,沉迷游戏的人,戒色状态会急剧下滑,会变得不想看戒色文章,开始疏远戒色。如果你一玩游戏就停不下来,几个小时甚至十几个小时,这就要反省了。高手不会沉迷游戏,高手懂得保持良好的戒色状态。

\subsubsection{高手的立场特别坚定}

一个人能不能成功?决心、信心和立场是非常重要的!有的人失败就是因为信心和立场不坚固,看了一些误导的文章,或者被心魔怂恿了,他就产生疑惑,开始动摇了,结果肯定是破戒。戒色高手都是信心和立场特别坚定的人,他们会远离无害论,远离误导诋毁的文章,不断坚定自己的决心、信心和立场,这类人的善根真的很深厚,也很有分辨力,很有正见。有了决心、信心和立场,就不会随便动摇了,这样戒起来的气势当然不同凡响。我很强调信心和立场,这点太重要了,没有信心和立场,根本不会戒下去的,不是听信无害论,就是听信诋毁误导的言论,最后搞得自己充满负能量。戒色也是一个大浪淘沙的过程,真正的坚定者都留下来了,信心和立场不坚定的都被淘汰了。

\subsubsection{高手懂得及时调整状态}

戒色后状态肯定有起伏,戒色几个月后也可能遭遇戒色厌倦期,不想看戒色文章,不想练习断念等,或者生活中遭遇挫折,出现生气等不良情绪,这时一定要善于调整,保持心平气和。这点极为重要,情绪不稳定是很容易导致破戒的。我看过很多人就是情绪不稳定,结果就疯狂看黄,通过手淫发泄,重新掉入了怪圈。现在很多书籍都在讲情绪管理,这一条的确非常重要,有一个好情绪,做起事来也会顺利很多。每个人都有状态欠佳的时候,也有遭遇挫折的时候,也会遇见各种困扰和压力,这时一定要善于调整,让自己情绪稳定,避免陷入烦恼和纠结,要学会看破放下随缘,内心要保持稳定,情绪要乐观一些。这种调整能力就是一种修为,和每个人的德行也密切相关,所以平时要多学习圣贤教育,让自己在这方面的修为能够得到提升,这样遇到挑战时,才能从容应对。高手是善于调整的人,懂得及时调整状态,因为他们深知,只有内心稳定,才能戒得稳定。

\subsubsection{高手对境什么表现}

对境第一反应是避开,不要去看第二眼,不要试图看清。这就是戒色十规里面强调的视线管理。现在是网络时代,也是色情泛滥的时代,不管是网上还是生活中,对境的机会非常多,这就很考验对境实战的意识了。古德就非常强调对境,对境不动心,才是有定力的表现。现代这个时代的诱惑比古代要大很多,特别是网上,更要时刻提起对境实战的意识,一定要避开,菩萨见欲,如避火坑。手机上的一些充满诱惑的 app,建议卸载掉,因为擦边内容实在太多了,看了那些内容,很容易破戒,这点一定要注意。对境是很大的考验,网络时代的色情诱惑又是那么猛烈,层出不穷,海量的诱惑内容,就等着你陷进去。所以,在这样一个时代,就需要我们具备更强的对境实战意识,告诉自己,对境要无敌,要做到百毒不侵。

\begin{case}
    昨天晚上看社交软件,上面有热搜文字新闻的标题,有擦边文字,不是直接的擦边图片,只是热搜新闻的标题在外面显示,是没有办法关闭的,居然对这文字有了想点进去的心,并且点进去了!好在具体文字和图片还没有加载出来,然后我赶紧关掉了,不过这无疑是一次失败的对境实战!我没有立刻避开关掉,而是起了微妙的想点进去的念头,并且跟了这个非常迅疾的念头,点进去了,如果不是在加载出来之前就关掉,可能看了擦边新闻内容会引发意淫之火花,然后一步步动摇!功亏一篑!
    \subparagraph{分析} 上网一定要小心谨慎,在对境的一刹那,就可以看出实战意识到底如何,自己也要善于反省和总结,慢慢摸清规律和特点,这样防范起来才有针对性,才能不断完善自己。网上推送的内容,一是文字标题,二就是图片。在看到的一刹那,很可能会产生好奇心,然后就点进去了,这个过程非常快。那个想看的微妙感觉非常细微,瞬间产生一种想点进去看的冲动,这个冲动非常快。必须具备很强的实战意识,才能防止这一冲动,在每次对境实战后都应该反省一下,是否贪恋了?是否及时避开了?为什么要点进去?一定要认真反省,不断优化自己的对境实战表现。即使是戒色高手,有时也可能点进去,因为在浏览的过程中,警惕性可能会下降,所以必须要经常反省,遇见擦边的内容,尽量不要点进去,如果不慎点进去了,也要做好视线管理,及时退出来。现代社会戒色这么难,和诱惑多是密不可分的,诱惑不仅多,而且非常猛烈,在这个时代戒色,必须要注重对境实战,严格做好避字诀,严格做好视线管理。
\end{case}

\begin{case}
    很多戒友戒了几百天甚至一两年,因为实战意识还是不够深刻,看到擦边内容破戒,千里之提溃于蚁穴。我在 QQ 经常会看到 QQ 里的新闻,是无法关闭的,是热搜榜上的内容会在 QQ 上随机显示出来,不小心看到那标题文字,可能就会让我心里一咯噔,想点进去看看具体内容,如果点进去了就很容易会引起意淫,然后入了心魔的圈套,破戒在所难免。看到擦边图像或视频,虽然并没有点进去看,但是仅是看到那概括性的文字,看清楚那几个文字,就会起一种微妙的想点进去的念头或者直接对文字展开了联想,浮想联翩,非常危险,需要引起重视,这是心魔的高级套路。
    \subparagraph{分析} 很多标题文字就带擦边,让人浮想联翩,吊起了好奇心,很容易点进去。有的人点进去后,会马上退出来,而有的人实战意识很差,点进去后,就一步步深陷了,从搜擦边图到搜黄,越陷越深,欲罢不能。我在对境实战的总结中,也经常反省,人的第一反应往往都是好奇心,想点进去,想看个明白,这似乎是人的一种本能。我们戒色后,就要警惕这种好奇心,这种好奇心夹杂着微妙的贪恋,会让人移不开视线,越看越想看。戒色高手会反省,会优化对境实战的表现,这样才能强化实战意识,在遇见类似的情况时,做出正确的选择——避开,不去点,不去看。在无聊时也要提高警惕,要学会充实自己的生活,不要沉迷手机。第一反应是避开,还是去点、去看,这就能看出实战意识到底如何。对境表现是需要不断反省和优化的,有一个过程,一定要学会避开,要知道上网就像进入了“黄雷区”,一定要小心谨慎。
\end{case}

\begin{case}
    今天下午差点栽进去了,先是中午躺着休息时看到了擦边内容,然后下午看视频时再次看到,这就顶不住了,想要破戒的欲望很强烈,由此可知“远邪”和视线管理有多重要,这两点做不好,那就几乎没有战胜心魔的可能了,可是状态差时这两点就是做不好,定力和自制力很差,会不断地上网娱乐,视线也完全管不住,但也未必就一定会破戒,就像今天这样,在挣扎的过程中好像是突然良心发现了,猛烈呵斥自己无耻的意念和举动,然后借助大声念咒和转移注意力——与网友聊天、和爸妈打牌(思维对治 + 念佛持咒 + 转移注意力),才化解了此次危机。值得注意的是我在大声持咒的时候,有个念头突然蹦了出来,这个念头伴随着强烈的冲动而来,似乎在说我要爽一下!这个念头被我观照得清清楚楚,那可真是主动入侵,就像在头脑里住着一个淫贼,它想借我的身体来发泄扭曲和变态的欲望,我之后反省自己也是这种感觉,觉得自己刚才怎么会那么变态,犹如被附体了一般。
    \subparagraph{分析} 躺着刷手机,危险级别比较高,因为躺着容易放松警惕。刷手机肯定会看到擦边内容,那些新闻和视频,含有大量的擦边内容。一定要谨慎,有的人看到一次还顶得住,看到第二次、第三次,就渐渐顶不住了,想破戒了。这位戒友也说了,远邪和视线管理非常重要,对境时,放任自己浏览,迟早是会栽进去的,所谓常在河边走,哪有不湿鞋。定力差时,一定要减少浏览新闻和视频,这类内容会让人越看越无聊,越看越空虚,甚至会产生强迫倾向和上瘾倾向,手里一直拿着手机,吃饭时也舍不得放下。在使用手机这方面,一定要加强自律,严格管控。这位戒友之后陷入挣扎,还好最后成功化解了这次危机,他的实战水平还是有的,思维对治 + 念佛持咒 + 转移注意力,这三点他都做到了。很不错,他也具备一定的觉察力,但他也应该反省,那就是对境实战意识还不够强,所以才导致了这次危机。现在这个时代不像古代,古代没有手机和电脑,交通也闭塞,一辈子很少有机会能接触到黄源,在古代戒色,的确更容易一些,这是从外境诱惑方面来讲的。在现代,必须要特别重视对境实战,这太关键了。外境的诱惑那么猛烈,如果对境实战做不好,真的会马失前蹄,重新陷入怪圈的。最近一位老戒友总结破戒原因:玩手机游戏,刷短视频看美女,不来吧里鼓励帮助戒友。他的总结很深刻,这也是很多戒友都在犯的,沉迷游戏会严重影响戒色状态,会导致不想看戒色文章,心思全部在游戏上,不想来吧里帮助戒友,警惕性下降。刷短视频看美女就更危险了,很多人都是刷这种视频而导致破戒的,教训很深刻。我能戒掉现在,一没有沉迷游戏,我现在几乎不玩游戏;二就是没有下载短视频 app。这两点我都规避掉了,希望大家都能吸取教训,在这方面要格外注意。
\end{case}

小结:在色情泛滥的时代戒色,真的是“火中生莲”!在这个时代戒色,必须特别强调对境实战,其实自古那些大德就一直在强调对境,对境实在太重要了!有没有定力,对境时就能验出来,对境是非常大的考验。断念和对境一直是我强调的重点,因为这两个实战,实在太重要了!怎么提升这两个实战的表现,实乃重中之重,这两个实战做得好,一定会越戒越强的。而要完善和提升这两个实战的表现,需要我们坚持学习和练习,并且要不断反省和总结,不断优化,实战表现才能越来越强!

\subsubsection{永远牢记警惕}

大德有云:“学道犹如守禁城,昼防六贼夜惺惺。”群居守口,独坐防心。戒色修心也同样要保持警惕,这个时代诱惑太猛烈了,大家都懂的!手机上、网络上,海量的色情信息,对境一定要警惕,独处时也要警惕,严防邪念的进攻。警惕的螺丝是不能松的,我戒到现在依然每天保持警惕,前几天邪淫的回忆上头了,被我立刻断除了,一天上来了几回,都被我断除了,如果我放松警惕,沉迷意淫,那后果是什么?不言自明!即使念佛持咒,还是会遭遇翻种子的,所以必须保持警惕,做好断念,修心的确是持久战,不是一劳永逸的,要做好持久战的心理准备。

\begin{case}
    我戒色三年多,这几天宅在家。1 月 30 号、1 月 31 号连续两天看黄,昨天破戒了。然后我感觉我有点自暴自弃,坚持三年的养生桩,昨天我也没有做,然后昨天破戒后,熬夜玩手机玩到通宵。这是我戒色三年多第一次破戒,冷静下来我发现,我没有识别出心魔的怂恿。当时脑海中出现看黄的想法,我没有识别出这是心魔的念头,也没有及时念断念口诀,然后导致破戒。我想了想,主要是我没有用好断念之刃这把刀。以前遇到怂恿还有各种诱惑,我都是用断念之刃这把刀劈开它们。戒色三年症状基本全部恢复,今年困扰我两年的频遗也控制好了。前段时间看中医喝了一个月中药,慢前也没有反复过,这都是戒色带给我的。我现在很颓废,不过我会控制好情绪,找回以往的状态和自律。这两天没做好视线管理,还有听同事谈了邪淫话题,我没有及时断念。前天加了一个同学群,里面有人发了邪淫视频,我看了第一遍又仔细看了第二遍,控制不住怂恿念头,找黄看黄。昨天前天都有看黄的念头,没有识别出那是心魔的怂恿,心魔怂恿也是念头!没有断念!
    \subparagraph{分析} 这位戒友之前戒了三年多,相当不错的战绩!可惜他破戒了,是三年多第一次破戒,有点可惜。问题首先在于对境,听同事谈了邪淫话题,没及时断念,另外就是加群看到了邪淫视频,他看了二遍!这足以点燃欲火!对境时一定要格外警惕,这个时代的诱惑真的太多了,一定要反复强化对境实战的意识,听到同事谈邪淫话题,应该马上走开,或者善巧地转移话题。看到群里有人发邪淫视频,坚决不看!马上避开!自己也可以清除聊天记录。对境是个大考验,对境也最容易出问题,所以要不断反省和总结,提高警惕。这位戒友说自己没有识别出心魔的怂恿,其实他之前戒了三年,肯定遇见过很多怂恿的情况,但他都断掉了。这次没识别出,可能和看了二遍邪淫视频有关,那种视频真的不能看,看了之后很容易失控,因为诱惑实在太强了!太猛烈了!一定要坚决避开!如避火坑!能戒三年,说明觉悟和实战的实力都是具备的,相信这位戒友通过反省和调整,很快能找回良好的戒色状态。我们从这个案例可以看出对境和断念有多重要,这个时代的诱惑太猛了!古代戒色相对容易,现代戒色要面对海量的色情内容,这就要求我们必须提高警惕,做好对境实战!在色弹横飞的战场,只有警惕者才能生存!第一反应:避开!老兵不会撞子弹,老兵懂得避开!一定要避开!一定要避开!一定要避开!在上网前至少提醒自己三遍!这就是实战意识!实战意识一定要特别强!在战场上要避免不正常的紧张,要沉着冷静!
\end{case}

对境实战后一定要深刻反省:

\begin{multicols}{2}
    \begin{itemize}
        \item 对境时自己看了几眼?
        \item 有没有微妙的贪恋?
        \item 有没有点进去?
        \item 有没有及时避开?
        \item 有没有做好视线管理?
        \item 有没有克服好奇心?
    \end{itemize}
\end{multicols}

通过反省和总结,就能强化实战意识,优化对境实战的表现。

\subsubsection{不居功,不自傲,躬行于善,厚积在德}

高段位戒者肯定注重修德,注重培养自己的德行,因为他们知道“至戒靠德”!做了很多善事,千万不能居功自傲,要懂得让功,懂得感恩!这点非常重要!一居功,就会骄傲,一让功,就是无私。差距非常之大,高段位戒者肯定是谦和之人,“劳而不伐,有功而不德,厚之至也。”一位高中戒友说:“进入戒色吧三年了,这三年教会我的一个道理就是谦虚,骄兵必败!这是永恒的真理。”厚德载物,德行是地基,德行有亏之人,很难戒得稳定,一定要注重修德,克服嗔恨心、傲慢心、嫉妒心、自私自利的心态。戒色说到最后,肯定要提德行,因为德行跟不上,戒色的境界就上不去,也很可能出现破戒。一德,二命,三风水!德是第一位的,德不配位,必有灾殃。林则徐《十无益》第一条就谈到了“\textit{存心不善,风水无益}”。一定要存好心、说好话、行好事、做好人。坚持行善积德,培养自己的正能量,躬行于善,厚积在德!


\paragraph{总结}

这季分享了高手之路提高篇,相信大家又收获了很多心得体会,要成为戒色高手,一定要学习高手的操作,借鉴高手的经验,高手的东西都是高度提炼的,直指核心!在学习高手经验的同时,自己也要不断反省和总结,加深自己的认识,提高戒色的觉悟,强化实战意识。我的戒色体系是立足于实战的,因为我坚信:实战强,才是真的强!实战强,才能立于不败之地!我戒到现在一次未破,就因为强化了实战表现,我经历了无数次实战考验,如果实战不强,肯定早破了。断念实战和对境实战,怎么强调都不为过!实战的内容是很深的,要注意的细节有很多,自己要慢慢完善和打磨实战意识。不管戒多久,都不能放松警惕,还是要做好断念实战和对境实战,现在这个时代,对境的机会成千上万,诱惑又是那么猛烈,如果没有很强的实战意识,肯定会阵亡的。这个时代戒色的难度很大,就是因为黄源太多,诱惑太大。希望大家好好强化实战意识,坚定自己的信心和立场,行善积德,加强自律,不断反省、总结和完善,你们肯定会越戒越强的!暂时的失败不可怕,能从失败中强势崛起才是真正的强者!

下面分享一首诗歌。

\begin{poem}[我们在蓝天下等你]
    \begin{multicols}{2}
        \begin{center}~\\
            走出邪淫的阴霾 \\ 走出那些伤精的症状 \\ 走出身心的痛苦折磨 \\ 作为戒色前辈 \\ 我们在蓝天下等你 \\ 希望你带着欢乐 \\ 就像纯净的孩子一样 \\ 一蹦一跳向我们走来 \\ 蓝天白云下 \\ 有纯净美好的我们 \\ 清风拂面 \\ 那种感觉是多么的爽 \\ 我们笑了 \\ 笑得那么纯真 \\ 那么欢乐 \\ 就像一群纯真无邪的孩子 \\ 徜徉在纯净王国里 \\ 以全新的勇气 \\ 去面对未来的挑战 \\ 我们的步伐无比坚定 \\ 因为我们有正气、有活力、有能量 \\ 我们一起站在蓝天下 \\ 站在永恒的美好里 \\ 面对纯净的彼此 \\ 一起充满朝气地 \\ 奔向我们的未来
        \end{center}
    \end{multicols}
\end{poem}
