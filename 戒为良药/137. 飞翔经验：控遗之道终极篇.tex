\subsection{控遗之道终极篇}

\paragraph{前言}

上季一位戒友说自己都会背楞严咒了,为什么还会破戒?这个问题我之前其实有提到过,那就是日课脱离实战!有的戒友也念佛持咒,但还是会破戒,为什么会这样?当邪念图像上头时,如果不及时断掉,肯定还是会破戒!不管你念佛持咒了多少遍,关键还是看实战的那一下!戒色一定要以实战为核心,立足于实战!否则就像打仗时脱离实战一样,最终的命运肯定是吃败仗!虽然念佛持咒了,但是当邪念上头时,却不断掉,跟着邪念跑,或者当面对诱惑的对境时,不知避开,反而盯着看,疯狂点击,这当然会破戒!不是说念佛持咒了,邪念就不来了,根据我的亲身体会,如果真正用功去念佛持咒,比如一天几千乃至上万,坚持一段时间,内心真的会变清净的,但是继续戒下去,到了一定时间邪念又会来的,会遭遇猛烈翻种子,到时很考验断念能力,断念差,必败无疑!破得一塌糊涂!

\begin{case}
    下午一看黄就没刹住车,连破三次,天天念经、听经也没用。
    \subparagraph{分析} 不注意远离黄源,天天念经也无法阻止破戒,远离黄源是最重要的第一步,戒色十规的第一规,在这个色情泛滥的时代,必须要注意远离黄源,不远离黄源,还去看黄、看擦边图,那念佛念经也会失去效果。一看黄,那种刺激太强烈了,到时一切都抛之脑后,变得极为疯狂,不顾一切。远离黄源,就像远离毒品一样,当想看黄的微妙念头出现时,一定要立刻断除。独处时,这种想看黄的微妙念头很容易出现,一定要提高警惕,做好断念。念头是行为的先导,必须做好断念!前段时间聊过一位戒友,那天他刚打完地藏七回来,身心状态极好,感觉他充满法喜,我也被他的喜悦状态感染到了,又过了一段时间,再次聊时他却状态很差,在连续破戒。我想说的是,佛法很好,的确有加持,但我们自己也要争气啊!一定要强化观心断念,否则断念不力,就会被邪念附体,身不由己。
\end{case}

我戒到现在九年了,一次未破,我说句大实话,虽然我一直在坚持念佛持咒,也在行善积德,但是邪念图像还是会上来的,就在前几天晚上半夜醒来,突然意淫的念头就攻上来了,我立刻警觉,觉察消灭,几记觉察就摆平了。如果我不断掉,那我肯定就会破戒,所以断念极端重要!修心是持久战!不是一劳永逸的,真正的考验还在后头!所以不可放松警惕!一定要注重断念实战!戒色不能华而不“实”,这个实引申为实战的“实”!虽然有些人也在行善,也在修身,也在改掉不良习惯,也在奋斗人生,但是他们忽视了最重要的部分——断念实战和对境实战!把实战给忽视了,其他做得再多,也只能管一时,戒到某个时间点,心魔疯狂进攻,到时顶不住,就会疯狂破戒。

念佛持咒是要紧密结合实战的,比如邪念一来,马上就念佛持咒,反应一定要快!念佛持咒的原理就是以一念代万念!邪念上头,马上就念佛或者持咒,就像条件反射一样,立刻把邪念转过来,转成佛号。如果觉察力很强,则可以直接觉察消灭,也可以思维对治。总之,一定要重视断念!我极端重视断念!断念强,才能立于不败之地!理论可以几千页,实战就看那一下!断念差,被心魔虐成狗!其他一些方法强调转移注意力、充实生活等,这类方法是有严重缺陷的,那就是脱离实战!上季聊过一位学生戒友,平时在学校都戒得好好的,专注于学业,努力学习,生活充实,在为人生理想奋斗,但是到了周末,一回到家里,独处时间增多,邪念就冒出来了,想看黄,各种意淫,这时候很考验断念能力,断念差,必定会破戒。他的情况属于周末破戒,我之前专门总结过这种破戒类型,平时在学校忙于学业,似乎都把戒色这件事给忘了,但是周末回到家,邪念一来,立刻就败下阵来,破戒后他就抱怨自己为何断不掉,因为他平时不注重练习断念,实战水平很低,等到实战来时,自然一败涂地,心魔一攻就破。

\begin{case}
    老师,我现在学习量、佛法的日课量都比较满,虽然破戒次数减少了,但还是会破戒,总结原因就是实战不行,断念晚,断念不给力。老师我真的迫切地想提高实战能力,我现在深刻认识到自己的不足了,就是断念实战不行!
    \subparagraph{分析} 虽然念佛,但是邪念图像还是会入侵的,这位戒友日课比较满,但还是会破戒,原因就是实战不行,断念不力。念佛持咒是很殊胜,一方面要保证念佛的质量,不能一边念佛,一边跟着妄念跑,要专注在佛号上,发现被带跑了,马上拉回来,这就是在训练觉察力,等到邪念入侵时,也能马上念佛。另一方面,念佛持咒一定要结合实战,否则念多少遍,还是可能会破戒。我到现在念了几千万的佛号了,但是邪念图像还是会冒出来的,如果我不做好断念,等待我的就是破戒。所以不管念了多少佛号,一定要围绕实战、以实战为核心,把握住这个根本,才不会破戒。
\end{case}

戒色界有一些文章是有水分的,甚至存在误导,资深戒友应该都能看出来,脱离实战的文章是有缺陷的,脱离实战不可能戒除成功,等到邪念图像袭脑时,到时就知道脱离实战的后果是什么了。一切最终要落到实战上,实战不行,说什么都白搭!戒色一定要注重实战,好好练习观心断念,刚开始也许会屡戒屡败,因为实战水平差,继续坚持学习和练习,有了顿悟后,就会突飞猛进。戒色十规就是立足于实战的戒色体系,这是专业戒色的钻石体系,希望大家一定要认真落实,真正的戒色高手都知道十规的价值和分量,这是专业戒色的最高总结,戒色十规强调的就是实战和德行,这两个是我最看重的。实战差,必破;德行差,必败!记住这两点,放之四海而皆准!戒色不是让你一天到晚在那和邪念做斗争,而是当邪念入侵时,要能断除之!戒色不是脱离生活,每个人都有自己的生活,平时要努力奋斗,是该充实生活,但一定要注重练习断念,戒色的核心不是修身,而是修心,修心是持久战,要降伏其心!断念是疏不是堵,断念是化解,而不是压制,真正治本的是断念,修行方面最高的教法都在讲观心断念,看看那些大德的开示就知道了,强调的就是“立断”!真正具备强大的断念实战能力,这才是做到不破的真正保障!断念才是硬道理!!!

\begin{case}
    大家一定要注意实战断念啊!真的是怎么强调都不为过!昨天真的是造孽,早起锻炼,下午打球,晚上泡脚的时候听清净明诲章,因为泡脚的时候被心魔怂恿了一下,回忆到了上一次破戒没看完的H片,回忆了大概三四秒才醒悟过来要念断念口诀,就是这短短三四秒的意淫就能让人欲火焚身,太可怕了!!!大家一定要在一秒之内觉察到自己邪恶或者被心魔怂恿的念头,然后立即用断念口诀断掉,临阵磨枪是不行的啊!
    \subparagraph{分析} 这个实战案例的戒友,他用自己的亲身经历告诫大家一定要注重断念实战,实战的体会往往是最深刻的,的确是怎么强调都不为过。我为什么如此强调断念实战,因为过去我屡戒屡破,原因就是实战不行,那时一味靠毅力强戒,靠转移注意力和充实生活,既没戒色觉悟,也没断念能力,真的被心魔虐成狗!心魔攻破我,就像玩儿似的,根本没有一点难度,往往几秒内就攻破我了,那个破戒的心理过程,我重复了无数次,我自己都记不清多少次了。我那时发誓要戒,戒了一周或者十几天,某天独处,心魔就开始怂恿我看黄,帮我找破戒的借口,我那时没有戒色觉悟,以为那是自己的想法,结果就听信了,于是就疯狂破戒!这个案例的戒友是泡脚时被心魔怂恿了,回忆了没看完的H片,他断念慢了,过了三四秒才反应过来,有的强烈诱惑的回忆片段,只需几秒就能让人欲火焚身,所以断念一定要快!一定要狠!他说临阵磨枪不行,这句说到点上了,平时缺少练习,觉察力不强,断力不够,到了实战就有困难了。大家平时就要花时间去练习,保证练习日课,不断研读断念实战的文章,这样断念水平才能不断精进,到时实战时就能战胜心魔了。从这个案例,我们可以看出断念实战的重要性!最后就看那一下!那一下不行,就会疯狂破戒!拦也拦不住地疯狂破戒!不管其他你做得再多,如果那一下不行,必然还是破戒!因为最后决定破还是不破,就看那一下!实战的那一下,必须要有!必须够快、够狠!
\end{case}

下面分享一些案例。

\begin{case}
    终于撸进医院了,明天一大早就去医院泌尿外科检查,可能有肾结石、前列腺炎啥的,今天肾部位疼得我下不了床,在床上疼得死去活来,一阵阵地绞痛,从早上疼到下午,最后可能昏过去了,下午醒来后终于不疼了,半条命都没了,一天没吃饭,早上吃的一个馒头鸡蛋也全吐光了,刚赶紧出去喝了一碗粥,楼下药店的医生建议我去医院挂急诊,我想了想没去,现在去要一个多小时才能到医院,各种排队做完相关检查,估计很晚了,晚上回来到时又不方便,而且现在也不疼了,用不着急诊,号也已经挂好了,明早反正还是要去,挨一晚明早去就好了。撸管十年了吧,一直都没有戒掉,自从神经症一发作,工作没了,我又掉回了怪圈,很快又变成废人,生不如死,外人什么也看不出来。现在终究还是撸进医院了,邪淫的果报啊!大家要引以为戒。
    \subparagraph{附评} 这是肾绞痛的案例,出现肾绞痛的症状,考虑多与肾结石、肾感染、肾囊肿等有关,肾结石和个人体质、生活习惯、工作环境有关系。出现肾绞痛,应该及时就医检查治疗,不可耽误病情。这位戒友撸龄十年,已经得了神经症,沦为废人,生不如死,神经症患者的感受我太清楚了,各种奇怪的症状,各种疑病,还不被家人理解,活得很惶恐。现在他又得了肾绞痛,真可谓雪上加霜。邪淫终究是要还的,用痛苦加倍地还!人财两空!人生陷入极度的灰暗境地,苦啊!撸进医院是迟早的事情,不断掏空自己,身体迟早会垮掉的。

    中医讲到“精”的五大作用:\begin{description}
        \item[繁衍生命] 肾精,会产生生殖之精,并将遗传信息传递给下一代,这是精的重要作用。
        \item[濡养作用] 精能滋润濡养人体各脏腑形体官窍。
        \item[化血] \textit{精不泄,归精于肝而化清血。”因而肾精充盈,则肝有所养,血有所充。故精足则血旺,精亏则血虚。(《张氏医通·诸血门》)}
        \item[化气] 精可以化生为气。\textit{精化为气。(《素问·阴阳应象大论》)} 气有保卫机体、抵御外邪入侵的能力。\textit{故藏于精者,春不病温。(《素问·金匮真言论》)} 可见精足则正气旺盛,抗病力强,不易受病邪侵袭。
        \item[化神] \textit{精气不散,神守不分。(《素问·遗篇·刺法论》)} 只有积精,才能全神,这是生命存在的根本保证。反之,精亏则神疲,精亡则神散,生命休矣。
    \end{description} 肾精太宝贵了,这五大作用大家要牢记,繁衍、濡养、化血、化气、化神。精能生气,气能生神,荣卫一身,莫大于此,养生之士先宝其精,精满则气壮,气壮则神旺,神旺则身健,身健而少病,内则五藏敷华,外则肌肤润泽,容颜光彩,耳目聪明,老当益壮矣。精为身本,精为至宝,肾精就像人体的核能一样,疯狂看黄疯狂手淫,不断耗损肾精,最后的下场会很惨。
\end{case}

\begin{case}
    那个时候我已经渐渐知道了自己身体很糟糕了,不能再放纵了。记忆力越来越差,我开始忘记一些事情,比如自己明明关了门,我就会怀疑自己没关好,一定要去检查一下,经常都是这样,就像强迫症一样,而且只要遗精手淫,我的身体就会很难受,很疲倦,在短暂的快感之后,就是无尽的痛苦。我开始出现了腰疼,睡觉就疼,早上起来那会儿非常疼,我感觉不是自己醒的,而是被腰疼醒的,我开始意识到这个问题的严重性,那个时候还没有前列腺炎的症状,所以我在想怎么办,所有我可以做的事情都试过了,没办法,我知道我必须去看病了。而且大四我又准备考研,以我那个时候的精神状态根本无法应付考研,但是没办法我必须要开始治疗,所以我就去我们学校附近的一个药店坐诊的一个中医那里去看,当时我还有些钱,我真的是鼓起了勇气,我去见了一个老中医,真的是老中医,年纪已经大概六十多岁了吧。他问我哪里不舒服,我就给他说了腰疼,然后很困,还有遗精,没有精神,然后他把了我的脉就开始叹气了,说我肾气很弱,非常弱,我当时也不知道肾气是啥,就听他说,他问我有女朋友没,我说没有,他就给我说叫我不要再手淫了,一定不要再手淫了,我当时纳闷他怎么知道我手淫,但是他一脸的慈祥,我觉得是我做错了,他说这个病不能着急,要慢慢好,一定不要手淫,不然以后会成为废人。废人?我当时心里想,废人是怎么样的?真的会成为废人吗?但是我感觉他是一个有良心的中医,他告诉我药可能有点贵,问我要不要吃,我说要,就给我开了药,应该都是补肾的,那些药很贵,我也忘了我吃了多久,总共花了将近一千多元,但是我没有戒掉手淫,心里还是有侥幸的心理,我在想既然我吃了药,就可以光明正大地撸下去了,但是事实证明我错了,完全地错了,现在才知道吃药会助长性欲,然后我又手淫把它泄掉,上补下漏永远都好不了。可我那个时候不懂,我继续手淫,因为那种快感真会让人上瘾,真的。我是看到了《戒为良药》才知道了真相,现在觉得自己好无知,所以大家一定要赶快戒掉。
    \subparagraph{附评} 腰为肾之府,手淫后容易出现腰痛的表现,我初中时就深有感触了,那时手洗衣服,弯腰一会,就感觉腰痛难忍,直不起来,刷牙时也会。有时手淫后的几天,不弯腰也痛,有时还发胀,会有奇怪的跳动感,躺下时,那种跳动感尤为明显,我那时很担心。一般戒除一段时间,腰部症状就会有所缓解,手淫真的很伤腰,腰真的太重要了,腰不行,运动什么的都不敢发力。这位戒友遇见的老中医还是挺不错的,一把脉就开始叹气了,老中医心里肯定在想:“年纪轻轻,肾气怎么那么弱?只有两个可能,一,有女朋友,二,自己手淫。”老中医用了排除法,既然没女朋友,那肯定就是手淫了。老中医也提到了“废人”这个词,疯狂纵欲,最后的结果就是沦为废人,也许因为年轻,这位戒友对废人这个词没有体会,等到身体真正垮掉了,到时就知道了。之前他的思想误区很明显,还有侥幸心理,以为吃了药就可以继续撸下去了,吃药纵欲,身体只会更差。有些中药吃下去是容易助长欲望的,所以吃中药后要提高警惕,加强修心,不要上补下漏,否则身体很难恢复。每个人曾经都很无知,现在知道手淫的危害,就要好好下决心戒了,这是一场健康的保卫战。身体垮了,一切都没有指望,身体好了,精力足了,才有奋斗的渴望与斗志,才有美好的未来。
\end{case}

\begin{case}
    今年已经十七了,本是花样年华,朝气蓬勃的少年。可是因为这些年的放纵和堕落,自己变成了行尸走肉,自己照镜子都觉得猥琐不堪。事事不顺,负能量爆棚,身体不好,精神萎靡,一点朝气都没有,整天过得浑浑噩噩,死气沉沉的,整个人生就是在黑暗里!脸色差得要命,自己又变得自卑,不自信,一系列心理疾病,生活一塌糊涂,光明在哪?真正痛苦是十六岁的时候,那个时候几乎都是在求医,在医院住院中度过的,当时的那种感觉真的太无助,太痛苦了,要不是年轻恐怕早就扛不住了,刚开始身体真的没有什么症状,没想到一来就是大病,很严重的肾病,差一点就奔向尿毒症了,我吃了很长时间的中药才控制住,但是很容易复发。
    \subparagraph{附评} 十七岁的花样年华,却面临凋零的可能,十七岁,多么美好的年纪,可惜因为纵欲让生命蒙上了一层厚重的阴影,让人喘不过气来。我有一张照片,大概也是十七岁时照的,照片里的我哭丧着脸,双眼无神,也像行尸走肉,十七岁时,我已经撸了大概三年左右了。那是上世纪九十年代末,邪淫已经开始泛滥,主要是光碟,那时还需要买,现在都成免费的,找黄很容易,孩子很小就接触电脑、智能手机,很容易染上手淫恶习,因为有海量的邪淫资源,所以很容易进入疯撸的状态,简直不要命,丧心病狂,加上又沉迷网游,熬夜久坐,这样没几年身体就吃不消了,开始全面垮掉,十几岁的孩子就得上神经症,甚至更严重的疾病,现在已经比较常见了。这位戒友得的是严重的肾病,可能是肾衰竭,对于一个十七岁的孩子来说,有点残忍。他提到了刚开始身体真的没有什么症状,刚开始都这样,感觉手淫很爽,黄片那么多,好像天上掉馅饼一样。刚开始年纪还小,身体有点底子,撸后最多感觉有点累,其他没有太明显的症状,然而撸了几年后,就不对了,各种症状开始爆发了,到时就要受苦了。有句话是这样说的:“世上伤害生命的事情不止一种,而贪淫好色是最要命的。”\textit{若耗散真精不已,疾病随生,死亡随至。(《遵生八笺》)} 古人早有告诫,我一直在勤求古训,发现古人对这方面的认识非常深刻,纵欲的伤害实在太大了,很多孩子非常无知,没有人来引导他们,父母也不会告诉孩子手淫有害,因为父母也没这方面的觉悟和认识。这种情况因为网络戒色的崛起而有所改观,越来越多的人开始认识到纵欲的危害,开始自觉戒除看黄手淫,他们也认识到了戒色的重要性,戒色是君子第一修为,少之时,血气未定,戒之在色!唯戒色不负青春!
\end{case}

\begin{case}
    最近刚做完气胸的内窥镜肺大泡切除手术,在家修养,回想自己这 29 年的生命,从开始手淫就没有平顺过,而且从一个身体健壮聪明的少年变得现在一身的疾病,气胸(以后不能做剧烈运动)腰间盘突出、胃炎、鼻炎(冬天很难受)、痔疮、早泄、前列腺炎、精子质量差,很多都是慢性的疾病,也不能做重的体力劳动,甚至不能跑步打球。而且因为自己手淫后经常神不守舍,在一次交通事故中还撞死了一位老先生,现在想想真的是恨死自己,因为那天我就是邪淫后开的车。而立之年的人了也没有婚姻,一事无成。我在这里奉劝还在邪淫的师兄们,一定要痛下决心戒除邪淫!不然,自己的一生都葬送了!!我现在欲哭无泪,本来可以有一个平凡的人生的,现在的自己何去何从都不知道。但我明白,如果不彻底戒除,真的一辈子都完了,我不想做一个让人笑话的烂泥,我不能只是哭泣流泪抱怨,这些都无济于事,只有不断地洗涤自己污秽的思想,多吸取正能量的戒色知识,自己才能变好!我奉劝师兄们以我为戒,别到无路可走时才悔悟啊!一定要好好戒邪淫,做一个对得起父母对得起天地的人!
    \subparagraph{附评} 这位戒友的经历挺发人深省,活了 29 年,从手淫后就开始不顺,一身疾病,动了气胸手术,还撞死了人。回想我在手淫后的那些年,也有很多倒霉的事情,也是一身的病,也动过手术,而且还是三次,的确够倒霉的。这位戒友之前撞死了一位老先生,这估计会让他悔恨终生,邪淫后人的精神是恍惚的,注意力难以集中,浑浑噩噩的状态,在这种状态下很容易出事,很多戒友都反馈在手淫后会出现不好的事情。邪淫后容易出车祸,这是真的,绝不是闹着玩的,撞死人算是摊上大事了,估计赔了很多钱。一直邪淫,一直起邪念,可想而知气场有多差,人的状态有多糟糕,如果再不悔改,真的是一生都葬送了。《The Secret》(秘密),全书阐述了一个法则——吸引力法则。基本原理是这样的:人类所有的思维活动,都会产生某种特定的频率,而这种频率会吸引同样的频率,引发共振。因此,我们应该尽可能地摒弃一切负面的思维活动,多发正面的善念,负面的念头会感召负面的人事物,正面的善念会感召正面的人事物。一个邪淫的人,满脑子邪思邪念,他的身体会变差,他的运势也会变差,人生会充满各种坎坷,其实都是自己感召的。古圣先贤一直在教导我们改过迁善,断除负面的念头,发正面的念头,这样人生才能步入正轨,真正幸福起来。
\end{case}

\begin{case}
    大家别撸了,彻底戒除吧。4 月 29 日晚上,突然我心脏快不跳了,心慌颤抖全身不能动,呼吸困难,家人打 120 接我走的,当时我以为自己不行了,然后在县医院呆了不行,转市里最好的医院。一通检查下来,中重度肾积水,CT、心电图异常,生化十七项,彩超,提示肾积水。医生说心脏没事,但是这个肾积水要好好查查,然后天明就回家了,一天后办理住院了,当天住院,晚上偷偷溜回家,结果第二天晚上又犯了,又是全身不能动,呼吸困难,心跳 180 多。当时自己家人送的急诊,本来住院的,又开始一通检查没事,当天晚上症状大爆发,焦虑症抑郁症,心脏神经官能症,疑心病,各种问题根本停不下来。住院部又给我做心电图,心脏彩超,24 小时动态心电图。做了 N 次,没人告诉我是焦虑症。后来,CT 增强扫描,尿道造影说输尿管狭窄,然后给我手术了。手术后还是各种问题,心脏一直早博,心脏瓣膜关闭不全,各种痛苦的症状。我突然想起,发作前几天,我撸了一发……我渐渐明白了。手术后第四个月早博才感觉没有,然后去查,医生说切除左肾,怀疑肾癌,三个月复查还没有这么大,不行让我去北京看去。劝大家不要再邪淫了,真的,就算不是肾癌,我这个病没法控制,多发囊肿、肾衰竭、尿毒症就是我的将来。有因必有果。左肾一直发胀,腰疼,没想到啊!焦虑症抑郁症心脏神经官能症我都熬了过来。最后还是不给我机会。戒得太晚。切记。哎!大家,千万别再说戒不掉了,那是你还没有下决心,戒晚了,也就完了。
    \subparagraph{附评} 这个案例让人细思极恐!这位戒友的经历也给我们提了一个醒!撸管真的是要还的,也许暂时感觉不明显,但是伤到某个程度,一下就崩塌了,就像一幢大楼在你面前崩塌一样,那种场景让人感到恐怖和震撼。撸出濒死感,这种例子是比较常见的,常驻戒色吧的戒友肯定时不时会看到此类案例,呼吸困难,心脏狂跳,感觉自己快挂了,那种感受绝对让人崩溃。伤得深了,等待撸者的就是神经症爆发,惊恐发作,濒死感,各种疑病,各种难受,生不如死。这时才惊觉,原来健康的身体是如此重要,而自己之前那么无知,身体的精华都被自己糟蹋了,还心存侥幸,以为症状不会降临到自己头上,等到症状爆发了,就呜呼哀哉了,感到深深的悲哀、绝望和恐慌。这位戒友想起来了,惊恐发作前几天,他撸了一发,说是一发,也许是连续几发,或者熬夜时撸管,他渐渐明白了,纵欲的报应找上门来了,等着他去承受!医生怀疑那个囊肿是肾癌,这估计让他心里又深深惊恐了一番,很多人撸后都感觉腰痛、发胀,要引起警觉啊!他说戒得太晚!之前有位戒友叫“戒得太迟”,都是得了重病后的深刻感悟,如果早能认识邪淫的危害,早下决心戒除恶习,也许就不会这么悲惨了。去医院看看,那里有多少病人,人生病后很痛苦,得了大病更痛苦!一个健康的身体实在太重要了,必须要痛下决心戒除,不能等到大病临头才悔悟,最后也来上那么一句“戒得太晚”!\textit{姿其情欲,则命同朝露也。(名医孙思邈)} 命同朝露,这四个字实在振聋发聩,放纵自己,迟早会大难临头。\textit{淫泆无度,忤逆阴阳,魂神不守,精竭命衰,百病萌生,故不终其寿。(《养生延年录》)} 精竭就命衰了,到时百病蜂起,可畏哉!
\end{case}

下面进入正文。

戒色是系统工程,恢复也是系统工程,而恢复方面,控遗又是非常重要的基础之一,如果控遗做不好,那身体恢复进度会慢很多,如果频遗很严重,甚至会爆发新的症状。前段时间看到一个戒友的案例,他戒得还可以,但就是没有克服频遗,导致身体恢复不理想,神经症还是好不起来。其实这类戒友还是比较常见的,很多戒友都有频遗的困扰,我曾经也经历过,知道频遗的危害,不要说频遗,就是一次遗精都会导致身体不舒服,感觉能量下降的无力感和颓废感,心情也会变差,往往要好几天才能缓过来。频遗对戒色和恢复的信心是一种很大的打击,所以克服频遗是恢复的重中之重,一定要引起高度重视。

我在第 3 季就开始谈控遗了,我很早就意识到了控遗的重要性,后来的很多季都有谈到控遗。列举的 39 条遗精因素,很多戒友都知道,也被做成图片转发了无数次,频遗是恢复的拦路虎,是必须要克服的问题。根据我自身这些年的研究和体会,控遗的确是一门很深的学问,需要注意的方面有很多,刚开始戒色,很多人不会碰到频遗问题,但戒到一定时间基本都会碰到,有的人很快能克服,而有的人可能一直无法克服,搞得自己很苦恼,好不容易积攒的肾精就这样遗掉了,实在太可惜了。控遗其实就是在做能量管理,当然戒色本身就是能量管理,控遗做得好,再加上其他的养生,身体恢复就会加快很多。大家可以想一下,本来恢复不错,突然遭遇频遗问题,身体感觉差了一大截,那是怎样的落差?如果能把能量守住,不断积累,那身上的感觉真的是气壮山河、波澜壮阔,精力像开了挂一样,那种底气和自信,完全不一样。

如果出现频遗,比如一月三次以上,有的人甚至一月七八次遗精,那真的太伤身体了,之前一位戒友 36 天遗精十次,太频繁了,身体很难恢复。中医:久遗八脉皆伤!有的人真的是一次遗精都伤不起,记得我刚开始戒色,因为之前身体一直亏空,大概戒了有大半年才恢复遗精,然而一遗精,第二天就腹泻了,说明身体还是比较虚弱,未完全恢复,也说明遗精真的很伤身体,直接影响五脏六腑,遗精后的身体感觉会差几个档次,心情也会变得低落,无精打采,这时一定要注意休养,适量锻炼,调整好状态。

一般一月三次以内属于正常频率,但最好控制在一月一次,或者几个月一次,那就更好了。比如一年十二次遗精,这个数据已经很好了,很成功了,但如果能做到一年六次遗精,那就更好了。遗精越少越好,有的人甚至不想遗精,但很难做到,所以我们要做的就是严格控遗,尽可能少遗精。遗精少了,能量就保住了,能量会不断在身体里积累,去滋润五脏六腑和补养大脑,整个人都会焕然一新,走路都冒劲,腿脚有力,气宇轩昂。精满,气足,神旺,就像跑车加满了油,有大干一场的冲劲和气势,把这种状态用在学业和事业上,那真的是势不可挡,突飞猛进。

最近看到一句古代养生的句子,感觉很有道理:“精于人也,若水浮航。”有了肾精,就像水能把船托起来,这样人生的航行才能顺畅,否则肾精亏虚,再好的船也走不了,只能困在那里,是不是这个道理?养足肾精,人生就会风生水起,身体好了,精力足了,干劲也起来了,底气和自信也跟着上来了,到时面对人生的机遇时,就能把握住,面对人生的挑战或挫折时,也能以沉稳的心态去从容应对。肾气足则五脏六腑功能协调,容面红润有光泽,精力充沛,抵抗力强,不易生病,肾气虚则反之。另外一个比喻也很好,人就是一盏灯,精就是灯油,灯油(肾精)耗尽,人的寿命也就终止了。所以,一定要懂得保养肾精,爱惜肾精,不仅戒邪淫,还要尽可能少遗精,把损耗最小化。

\begin{quote}\it
    神气血脉,皆生于精,故精乃生身之本,能藏其精,则血气内固,邪不外侵。(清代医家张隐庵)
\end{quote}

\textit{肾者,作强之官,伎巧出焉。(《黄帝内经》)} 作强,是指产生强劲之力;伎,同“技”,多能之意也;巧,精巧也。\textit{盖髓者,肾精所生,精足则髓作。髓在骨内,髓作则骨强,所以能作强,才力过人也。精以生神,精足神强,自多技巧。髓不足者力不强,精不足者智不多。(《医经精义》)} 肾为作强之官,藏全身精华,为一身之根本所在。

肾精就是生命能量之源,代表的就是生机和活力,绝对是人体的大宝贝,是不能随便往外拿的。肾精要深藏,就像一个宝贝一样藏起来,这是人的核心能量,是要保护起来的,一定要有爱惜意识,不能相信无害论,不可随意耗泄。如果你家有一个传家宝,你一定会把它藏起来,放在保险箱里,或者其他安全的地方,不会随便给别人看,非常爱惜和保护这个宝贝。对于肾精,我们也要如此爱惜,坚决戒色,严格控遗,把宝贵的能量守住。中医有一个词叫“失精家”,失是丢失的失,精是肾精,失精家长期流失自己宝贵的肾精,搞得症状缠身,苦不堪言。我提出一个“惜精家”和“积精家”,要懂得爱惜自己的能量,《黄帝内经》讲“积精全神”,我们要做惜精家和积精家,不能做失精家!

之前遗精的因素总结了 39 条,已经相对比较全面了,这季完善到 65 条,是更详细的一个版本。

导致遗精的因素如下:

\begin{multicols}{3}
    \begin{itemize}
        \item 白天意淫
        \item 白天劳累
        \item 喝酒
        \item 吃肉太多
        \item 吃补药
        \item 趴着睡
        \item 裸睡
        \item 晒被子
        \item 盖太厚太重
        \item 睡前打坐
        \item 内裤太紧
        \item 顶着或者夹着被子
        \item 艾灸不当
        \item 打坐意守下丹田
        \item 熬夜久坐
        \item 睡前喝水太多
        \item 运动过度
        \item 生病
        \item 肾亏无梦而遗(滑精)
        \item 饮食偏辣偏重
        \item 紧张(包括梦魇)
        \item 挤压(包括趴睡)
        \item 生气(导致气血紊乱)
        \item 受凉(包括吃凉食、冷饮)
        \item 按摩穴位不当
        \item 炎症导致的遗精(属于中医肾亏范畴)
        \item 还阳卧(有不少遗精的反馈)
        \item 睡前泡脚(水过热)
        \item 憋尿(易致遗精)
        \item 睡时使用电热毯或取暖器
        \item 睡前有过剧烈的运动
        \item 包茎包皮过长
        \item 思虑过度,劳心太过
        \item 回笼觉
        \item 失眠再睡
        \item 大量出汗
        \item 吃中药(特别是不对症)
        \item 长时间做足疗
        \item 黑芝麻吃多了
        \item 受到诱惑刺激或者惊吓恐惧
        \item 饮食不节
        \item 恣情纵欲
        \item 业障因素
        \item 心火旺盛
        \item 心脾两虚
        \item 相火妄动
        \item 肾气不固
        \item 湿热下注
        \item 思欲不遂
        \item 心肾两虚
        \item 心肾不交
        \item 下元虚寒
        \item 阴虚火旺
        \item 痰火扰精
        \item 肝郁血瘀
        \item 睡前腹式呼吸鼓动丹田
        \item 左侧卧睡姿
        \item 坚果吃太多了
        \item 吃了葱姜蒜、韭菜、辣等
        \item 睡得太多
        \item 太阳晒多了
        \item 用了电热毯
        \item 吹了空调受寒
        \item 晚饭过饱
        \item 吃夜宵
    \end{itemize}
\end{multicols}

关于遗精的因素,要具体情况具体分析,每个人身体情况不同,每个阶段身体的情况也有所不同,有的人吃辣会遗,有的人不会,有时候吃辣没事,有时候吃辣就会遗精,这是不一定的。但只要犯了这些因素,遗精的可能性就会加大,所以还是要尽量避免。在身体尚可时,对某些导致遗精的因素可能不太敏感,所以不会出现遗精,一旦身体虚弱时,对某些因素就会开始变得敏感起来,到时就容易出现遗精。

因素的大致分类:\begin{description}
    \item[饮食] 吃肉过多、吃辣、喝酒、吃冷饮、睡前喝水太多、晚饭过饱、吃了某些中药、补药、夜宵等。一般遗精后首先怀疑的就是饮食因素,看看自己是否吃了容易导致遗精的食物,这是首先要调查的。饮食非常关键,吃了蒜、韭、葱等也容易遗精,遗精后先回忆一下自己吃了什么。
    \item[过劳] 运动过度、劳累,思虑过度,劳心太过。前段时间一个戒友去健身练力量,回家晚上就遗精了,身体虚弱,要严格控制运动量,否则一劳累,晚上很容易遗精。还有的戒友则是思虑过度,一方面学业压力,另外就是强迫思维,胡思乱想,这就是劳心太过,也会导致遗精。
    \item[内因] 内因主要就是指的身体的失调,比如肾气不固、心肾两虚、肝郁血瘀、湿热下注等,身体已经失调了,所以不用外在刺激,就会出现频遗。对于身体失调导致的频遗,应该积极就医治疗,自己也可以多做养生功法,慢慢把身体调理过来。
    \item[外因] 外因就比较多了,像饮食也属于外因,睡前泡脚过热、吹空调受凉、回笼觉、艾灸不当、太阳晒多了、睡前打坐、左侧卧睡姿、受到诱惑刺激或者惊吓恐惧、足疗时间过长等,这些都属于外因。外因也要学会避免,很多时候遗精就是外因导致的。熬夜久坐伤身,导致身体虚弱,也容易遗精。晚上不要做腹式呼吸,特别是睡前不要做,腹式呼吸鼓动丹田,很容易导致遗精。
    \item[中医论遗精] 中医论遗精的原因,多因情志失调、心有妄想、多饮酒浆厚味、房劳太过或劳心引起。主要病机有心神浮越、心肾不交、阴虚火动、疏泄无度、湿热下注、扰乱肾气、劳伤心脾、气不摄精、肾虚不藏、精关不固、阴阳两亏、下元虚惫等等,其病与五脏都有关系,中医论遗精已经包括内因和外因。中医认为,遗精是因肾脏亏虚,肾气不固,无法稳固精关,或虚火旺盛,造成脏腑湿热,扰动精室所致的,以不经过性生活,精液却频繁遗泄为临床特征的病症。本病发病因素比较复杂,主要有房事过频、先天禀赋不足、操心过度、思欲不遂、饮食没有规律、湿热侵袭等。
\end{description}

中医按照临床症状,将遗精分为以下类型:\begin{itemize}
    \item 君相火旺,劳心过度心阴暗耗,心火偏亢,心火不能下交于肾,肾水不能上济于心,心肾不交,水亏火旺,扰动精室,发为遗精。\textit{梦遗之证,其因不一样,非必尽因色欲过度,以致滑泄。大半起于心肾不交,凡人用心太过则火亢于上,火亢则水不升而心肾不交。士子读书过劳,每有此病。(《折肱漫录·遗精》)} 又心有妄想,情动于中,所欲不遂,心神不宁,君火偏亢,相火妄动,扰动精室,也可发为遗精。
    \item 湿热痰火下注,饮食不节,醇酒厚味,损伤脾胃,酿湿生热,或蕴痰化火,湿热痰火流注于下;或湿热之邪侵袭下焦,湿热痰火扰动精室,发为遗精。
    \item 劳伤心脾,心脾亏虚,或劳心太过,或体劳太过,以致心脾亏虚,气不摄精,发为遗精。
    \item 肾虚不固,先天不足禀赋素亏,或青年早婚,房事过度,或少年无知,频繁手淫,导致肾精亏虚。若致肾气虚或肾阳虚,则下元虚惫,精关不固,而致滑精。若肾阴亏虚,则阴虚而火旺,相火偏盛,扰动精室,精液自出,发为遗精。
\end{itemize}

有梦而遗精者,称为梦遗;无梦而遗精,甚至清醒时精液自出者,称为滑精,滑精更严重。

\begin{itemize}
    \item 《黄帝内经·灵枢》指出凡内伤情志、思虑过度、恐惧怵慑,均可能扰动心神、损伤精气,产生“骨酸痿厥,精时自下”的症候。
    \item 汉·张仲景在《金匾要略》中称梦遗为“梦失精”,并认为是由虚劳所引起。(张仲景认识到了遗精和虚劳有关)
    \item 隋·巢元方《诸病源候论》进一步认识到遗精是由于肾气虚的所致。《诸病源候论·虚劳病诸候》指出本病的病机有肾气虚弱和见闻感触等:“肾气虚弱,故精溢也。见闻感触,则动肾气,肾藏精,今虚弱不能制于精,故因见闻而精溢出也。”(巢元方认识到了遗精是肾虚所致,和外在的见闻感触亦有关系)
    \item 在唐宋年代,对遗精的治疗已积累了丰富的经验,孙思邈论失精仍主肾病,孙思邈《备急千金要方》中载有治遗精方达十四首;王焘著《外台秘要》收录治虚劳失精方五首,虚劳梦泄精方十首,对后世较有影响。
    \item 明清两代,通过总结前人的理论和经验,对遗精的认识和治法渐趋全面和完善,如明·张景岳说“梦遗滑精,总皆失精之病,虽其证有不同,而所致之本则一。”明确指出遗精包括梦遗和滑精,且二者的病机基本相同,并将其治法总结归纳为降火、滋阴、升举、固涩、分利、温补等七法。\begin{quote}\it
              遗精之证有九:凡有所注恋而梦者,此精为神动也,其因在心;有欲事不遂而梦者,此精失其位也,其因在肾;有值劳倦即遗者,此筋力有不胜,肝脾之气弱也;有因用心思索过度彻遗者,此中气有不足,心脾之虚陷也;有因湿热下流或相火妄动而遗者,此脾肾之火不清也;有无故滑而不禁者,此下元之虚,肺肾之不固也;有素禀不足而精易滑者,此先天元气之单薄也;有久服冷利等剂,以致元阳失守而滑泄者,此误药之所致也;有壮年气盛,久节房欲而遗者,此满而溢者也。凡此之类是皆遗精之病。”“治遗精之法,凡心火甚者,当清心降火;相火盛者,当壮水滋阴;气陷者,当升举;滑泄者,当固涩;湿热相乘者,当分利;虚寒冷利者,当温补下元;元阳不足,精气两虚者,当专培根本。(《景岳全书·遗精》)
          \end{quote} 张景岳对遗精的病因病机做了较为全面的总结,归纳出遗精之证有九种。由此可见,遗精的发病机制主要责之于心、肝、肾三脏,但其中与心肾关系最为密切。所以不论火旺、湿热、劳伤、酒色等不同病因引起,日久无不耗精伤肾。病变以阴虚火旺、心肾不交发展为肾虚不固者多见。\textit{凡脏腑之精悉输于肾,而恒扰于火。火动则肾脏封藏不固。心为君火,肝肾为相火,君火一动,相火随之,而梦泄矣。(《类证治裁》)}
\end{itemize}

\begin{quotation}\it
    遗精得之有四,有用心过度,心不摄肾,以致失精者;有因思色欲不遂,精乃失位,输精而出者;有欲太过,滑泄不禁者;有年高气盛,久无色欲,精气满泄者。(《丹溪心法》)(朱丹溪的总结也很不错,总结了四种主要的情况)

    其色心太重,妄想过度而致遗滑者,自从心肾治之,但兼脾胃者多,又当审察治之。(《古今医鉴》)

    心为君火,肾为相火,君火动于上,则相火必炽于下,故少年多有此证。然有因梦而遗者,有无梦亦遗者。因梦而遗者,其病浅;无梦亦遗,其病深。治此之法,暴起者,宜清心,久滑者,宜固肾,且各求所因调治,何患不愈。(《医学集成》)

    遗精之主宰在心,精之藏制在肾,凡人酒色过度,思虑无穷,致真元下渗,虚火流行,精气滑脱。(《证治汇补》)

    盖遗精之始,无不病由乎心,正以心为君火,肾为相火,心有所动,肾必应之,故凡以少年多欲之人,或心有妄思,或外有妄遇,以致君火摇于上,相火炽于下,则水不能藏,而精随以泄。盖精之藏制虽在肾,而精之主宰则在心,故精之蓄泄,无非听命于心。凡少年初省人事,精道未实者,苟知惜命,先须惜精,苟欲惜精,先宜净心。及其既病而求治,则尤当以持心为先,然后随证调理,自无不愈。使不知求本之道,全恃药饵,而欲望成功者,盖亦几希矣!(《景岳全书·遗精》)
\end{quotation}

张景岳对遗精的认识非常全面深刻,不仅列出了遗精九证,也认识到了要净心,净心也就是修心,让心地恢复干净,心为君火,能够主宰心了,再配合治疗,这样治疗效果才好。

\paragraph{小结}

本病是指以不因性生活而精液频繁遗泄为临床特征的病证,有梦而遗精者,称为梦遗;无梦而遗精,甚至清醒时精液自出者,称为滑精。本病的发病因素比较复杂,主要有房室不节、先天不足、用心过度、思欲不遂、饮食不节、湿热侵袭等。遗精主要在肾和心,并与肝、脾密切相关。病机主要是君相火旺,扰动精室;湿热痰火下注,扰动精室;劳伤心脾,气不摄精;肾精亏虚,精关不固。本病应结合脏腑,分虚实而治,实证以清泄为主,心病者兼用安神;虚证以补涩为主,属肾虚不固者,补肾固精;劳伤心脾者,益气摄精。平时应注意调摄心神,排除杂念,以持心为先,同时应节制房事,戒除手淫。注意生活起居,避免脑力和体力的过劳,晚餐不宜过饱,养成右侧卧习惯,被褥不宜过重,内裤不宜过紧,以减少局部刺激,并应少食辛辣刺激性食物。

\subsubsection{案例分析}

\begin{case}
    飞翔老师我来了,九月份做到了一个月以上没有遗精,神经症恢复到百分之七十以上了,眼睛看东西明亮了,只要神经症在恢复,身体其他症状都会伴随着一起恢复,身心已经越来越舒畅了,我的世界真实了,真的太舒服了!
    \subparagraph{分析} 这位戒友特别在控遗方面下了功夫,认真研读了控遗的文章,自己也有控遗的经验,九月份做到了一个月以上没有遗精,神经症恢复很多,其他症状也恢复不少,身心大为改善。从这个反馈可以看出,控遗真的极为重要!对恢复而言,控遗的作用是决定性的,控遗做得好,身体恢复就有指望了,否则出现频遗,身体状况就很难起色,就像做事情无法打开局面一样,被困住了。精气散于三焦,荣华百脉,是身体之本,人身之至宝,很多人泄精前生龙活虎,泄精后就一滩烂泥了,可见肾精之宝贵,直接关乎一个人的精力、生机和活力。遗精虽然不算破戒,但也属于泄精,也会导致身体状态下降,所以必须要严格控遗。
\end{case}

\begin{case}
    戒色七十多天了,前几天感觉不错,这两天频繁遗精,而且腰疼,手汗特别大,该怎么办?
    \subparagraph{分析} 频遗很让人烦恼,因为好不容易把肾精养起来,突然出现频遗,身体状态又下降了,某些症状也开始反复了。这位戒友之前感觉不错,但是频繁遗精后,身体就开始出症状了。戒到一定时间,肯定会遭遇频遗问题,因为导致频遗的因素非常多,一不小心就可能出现频遗问题。出现频遗不要慌,要沉着应对,要看控遗的文章,学会控遗之道,这很关键。否则有些人会因为灰心丧气而破戒,乃至破罐破摔,这样就会重新掉入恶性循环。
\end{case}

\begin{case}
    我刚遗精,睡了一下回笼觉就遗了。
    \subparagraph{分析} 回笼觉这个因素,我在 39 条因素里专门总结过,之前有不少戒友都反馈过回笼觉遗精。中医讲久卧伤气,睡得太多也可能伤身体,有时睡完回笼觉起床后会感觉很累,甚至出现头晕、头疼的情况,这主要是因为睡眠时间过长所致,也就是睡多了,也会影响到了人的精神和体力。三四点起来小便后回笼觉,遗精的概率会增加,因为小便会导致精关松动,所以更易遗精。有时我小便后也会做几个固肾功,以加固精关。小便的时候要咬牙,牙属肾,小便时轻咬后槽牙,就是为了收敛住自己的肾气,让它不外泄。小便时踮脚尖,踮脚尖能够通畅足三阴经,达到益肾的效果。
\end{case}

\begin{case}
    我都戒了一百多天了,还是频繁遗精,差不多四五天一次,搞得我都没信心了,一点都没有恢复,好像比以前还要糟糕,最近心情很低落,我怕又要掉坑里了。
    \subparagraph{分析} 这位戒友虽然戒了一百多天,但是没有克服频遗,导致信心下降,心情低落,这也会影响到戒色状态,所以控遗不管对恢复还是戒色,都是极为重要的。
\end{case}

\begin{case}
    我三十天遗八次,很痛苦,不过现在心态调整了,目前十天遗一次,好了很多。
    \subparagraph{分析} 三十天八次,属于严重频遗了,肯定很痛苦,身心状态非常糟糕,不过这位戒友及时调整了心态,然后加强了控遗,状态就好了很多。出现频遗,一定不要慌,也不要气馁,一定要把心态稳住,学会控遗。
\end{case}

\begin{case}
    终于戒了一百多天,期间从未破戒,除了刚开始第一个月的时候有频繁遗精,后面几个月都是一个月遗一两次,终于体会到那些戒色三个月以上的前辈们戒色后的感觉了,只有戒的人才知道那种无法言喻的感觉了。感觉自己运气变好了,脑力体力也都恢复了,人也变精神帅气了,从一个颓废自卑、满脸油光、气质猥琐的人变成了一个意气风发、阳光开朗的人,走在街上经常有异性投来欣赏的眼光,人缘变好了,戒色的好处真的是太多了,戒越久越好。
    \subparagraph{分析} 这位戒友也克服了频遗,刚开始一个月有频遗,后来控制到一月一两次,这样恢复就会大大加快,能量就能有效积累起来。戒了一百多天,他的状态变得非常好了,也真正体会到了戒色带来的各种好处。控遗真的很关键,如果一直频遗,能量就无法积攒起来,导致恢复不理想。
\end{case}

\begin{case}
    戒色路上有飞翔哥指引,实在是莫大的荣幸,一点都不寂寞,感谢飞翔哥。我之前戒了九个月,后来破了,然后就是一直破破戒戒,到现在,通过系统地学习,我又戒了三个月了,现在我一点都不敢放松警惕。飞翔哥的固肾功真的太有用了,之前戒的时候,我一个月遗三至四次,现在遗精间隔时间越来越长,谢谢。
    \subparagraph{分析} 这位戒友通过固肾功成功控遗,之前一个月三四次,现在估计一个月一两次,遗精频率下降了,身体恢复自然就加快了,身心感觉会提升很多。
\end{case}

\begin{case}
    飞翔哥,我今年十六岁,之前邪淫四年,自从接触您的《戒为良药》后我便发奋改过,自从接触戒色文章以来,从未破戒,所有方面都做得很好,控制遗精也在一月一次,我戒了有一年半了,身体症状基本痊愈,现在的我已经脱胎换骨。
    \subparagraph{分析} 十六岁就能戒一年半,而且还是从未破戒,这个成绩相当不错,我十六岁时连一个月都无法突破,现在的孩子可以接触到专业戒色,真的很幸福。他遗精方面也控制得很好,一月一次,这个频率控制得相当棒!他的症状基本痊愈了,自感脱胎换骨。
\end{case}

\begin{case}
    从去年 8 月 27 日一直戒到现在,中间没有破,每天生活得充实又快乐,我认为自己永远不会再破了,这两天考试完后心情烦躁,内心空虚,昨天早上遗精了,今天早上也是,今天感到欲望非常强,中午时破戒了,还看黄了!后悔不已,现在心情更低落,这几个月中我平均五六天就遗精,怎么办,这两天感冒了,胃也不舒服。
    \subparagraph{分析} 戒到一定阶段,控遗就是一个非常重要的问题了,也可以说是亟待解决的问题。控遗做不好,就会影响恢复,也会影响戒色信心,影响情绪,影响戒色状态,症状也会出现反复,所以克服频遗是当务之急。这位戒友频遗后出现破戒了,遗精后邪念容易变得活跃,欲望会增强,这时一定要提高警惕,严防心魔进攻,同时要调整好心态,加强养生。
\end{case}

\begin{case}
    各位戒友我想问问,为什么吃宵夜后晚上会遗精啊?而且是连续两次,希望前辈们指教,在下感激不尽。
    \subparagraph{分析} 吃夜宵,伤脾胃,影响消化,加上睡前摄入很多食物,也容易遗精。
\end{case}

\begin{case}
    通过我不断地总结,终于明白了,首先你必须避免频繁遗精的因素,就是吧友经常发的那 39 条因素,这些是千真万确的,都是前人的总结,因为之前我每次遗精都基本上是里面的因素造成的,很多时候自己没注意,比如运动过量劳累了,吹多了空调,这些都会导致遗精的。因为我之前很喜欢坐着看书,所以我并不知道久坐危害如此之大,久坐后经常小腹不舒服,而且那几个月也经常遗精,后来我突然再查看这 39 条因素的时候猛然醒悟了,里面有一条就是熬夜久坐,不光是熬夜,久坐一样会遗精,大家一定记住不要久坐,坐四十分钟就站起来休息一下,后来我不久坐了,遗精频率就降下来了,这是真的,因为身体太虚弱了,只要有一个因素没做好,估计都会导致遗精,还有大量吃肉、吃补药也会的,所以一定要注意,这些因素都避免了,然后在养生功上下功夫。
    \subparagraph{分析} 39 条因素就是我总结的,被转发无数次,是 2013 年总结的。遗精后自己一定要学会分析,这个过程就像侦探破案一样,找到可能导致遗精的因素,然后及时调整。控遗真的是一门学问,我戒色这些年深感控遗的学问很深,需要方方面面都注意,能够做到一个月一次遗精就非常成功了,能做到三个月一次遗精就很完美了。这位戒友也谈到了熬夜久坐问题,熬夜久坐伤身体,会导致身体失调,中医讲久坐伤脾、久坐伤内、久坐伤肾,久坐是会导致疾病的,这点我深有体会,一定要学会避免久坐,定时起来走动一下。身体虚弱时,对很多遗精的因素都会敏感,这时自己一定要学会避免这些因素,养生方面要多下功夫。
\end{case}

\begin{case}
    这个月一号二号连着梦遗,身体腿部都有些隐隐作疼了,腰那里感觉特别空的感觉,早上头都是昏的,浑身上下无力,不想动,嘴唇老是起皮,喝再多水也没用。每次控制一段时间,都感觉相貌明显好看了,遗一次立马就变丑了,这是真的。
    \subparagraph{分析} 频遗后身体状态会很糟糕,这时一定要注意休养和调整。频繁遗精对于身体恢复是一大难关,很多戒友都在和频遗作斗争,克服频遗,身体状态就会好很多,会进入良性循环,能量能够真正积攒起来。遗精后精气神下降,的确会变丑,男人最重要的是精气神,失去了精气神,再帅的人也会变得黯淡无光,五官也会出现走形。
\end{case}

\begin{case}
    我都是睡前做固肾功,一百下,前提是要有拉紧感才算数,我也戒色 39 天了,3 9天一次遗精都没有,一般每天跳绳十分钟左右。
    \subparagraph{分析} 这位戒友做固肾功控遗,效果不错,加上跳绳,对身体恢复很有利,跳绳和八段锦的一个动作有异曲同工之处,那个动作叫:背后七颠百病消。此动作正是通过颠脚跟的方式,刺激肾经系统,诱发全身震荡,柔和地按摩五脏六腑,从而起到消除百病的神奇功效。颠脚跟还可以刺激到膀胱经以及足三阴经,包括肾经、脾经、肝经,足三阴阳气一足,可以宣通阳气于上,让更多的气血上达头面。经常颠脚跟,可以有效改善脑部的气血循环,可以让大脑充满活力,而且能缓解紧张的精神压力。
\end{case}

\begin{case}
    我也是遗精频繁,大概一个月七八次,跟你说说我遗精的原因:睡前喝水太多,吃得太饱,吃夜宵吃太饱,肉吃得太多,内裤太紧,被子盖得太暖,白天意淫做黄梦遗精,睡姿不对,大部分都是仰着睡遗精的。
    \subparagraph{分析} 知道遗精的原因,就要学会避免,杜绝遗精的因素,遗精频率就会下来。晚上各方面都要注意,因为临睡了,各种因素的影响会比较明显。
\end{case}

\begin{case}
    我每天两百个固肾功,五十多天遗精一次。
    \subparagraph{分析} 五十多天一次的频率,相当好了,能够做好固肾功,并且杜绝其他导致遗精的因素,是有望做到一个多月一次遗精的。
\end{case}

\begin{case}
    开始我也是一个月遗精八次,后来坚持做固肾功,十五天没遗精。吃了点肉,又遗精了一次,总结一下,固肾功有用!
    \subparagraph{分析} 一月八次,身体都会被遗垮的,还好做了固肾功,十五天没遗,如果饮食方面加强控制一下,那就更好了。
\end{case}

\begin{case}
    我个人最近休息在家,在彻底戒色基础上,也在吃中药调理,已经吃了二十副了。手脚出汗没有那么多了,遗精次数从之前八次也降到了三次,并且越来越稳定了。
    \subparagraph{分析} 吃中药控遗也是一个选择,有些戒友是通过吃药调理,才把遗精频率降下来的。如果频遗很严重或者出现滑精,也可以看中医调理。
\end{case}

\begin{case}
    不吹空调太热睡不着,吹了一个星期遗了三次!不吹就失眠。
    \subparagraph{分析} 吹空调也容易遗精,但也是因人而异的,温度可以调高点,不要对着风口,注意保暖,不要一味贪凉。经常吹空调、吹电扇是容易导致身体不适的,也可能导致疾病,这方面一定要多注意。
\end{case}

\begin{case}
    晚上按摩太溪穴是不是会遗精啊?
    \subparagraph{分析} 穴位按摩不当是可能导致遗精或者滑精的,按摩的力道不能太重,如果刺激过重,即使早上按摩,晚上也会遗精。有些穴位晚上不宜按摩,特别是肾经的穴位,比如涌泉穴和太溪穴,按摩的手法力度把握得不好,就可能导致遗精,晚上也不宜艾灸,晚上艾灸也容易导致遗精。
\end{case}

\begin{case}
    我发现好几次了,只要晚上泡脚后,当夜肯定梦遗。昨晚就是这样,睡前泡脚后,果然遗精了。本来好好的,记得上次遗是在九月十六(有记录),距今刚满十天!如果不泡脚,可能就不会遗了,很苦恼。
    \subparagraph{分析} 晚上泡脚水温不宜过热,否则是可能导致遗精的,一方面可以控制水温,另外可以把泡脚时间提前一两个小时。睡前要十分小心,因为在睡前各种因素的影响会变大很多。
\end{case}

\begin{case}
    晚上打坐容易遗精,怎么回事?是哪里不对啊?我就是念佛号,意念守在下丹田。
    \subparagraph{分析} 晚上打坐是容易遗精的,因为腿盘起来,气血流通变慢,容易腿麻,这样容易导致遗精,另外意守下丹田也很容易导致遗精。晚上打坐遗精,这类反馈非常多见,不少人是指望打坐调理身体的,这样一频遗,身体反而更差了,真的伤不起,所以晚上要避免打坐,也不要意守下丹田。可以白天打坐,打坐完毕先按摩一下腿部,再下来走一下。
\end{case}

\begin{case}
    昨天打篮球太累,晚上睡觉遗精了。
    \subparagraph{分析} 这是过度疲劳引起的遗精,中医讲劳心太过或体劳太过,以致心脾亏虚,气不摄精,发为遗精。身体虚弱时,一定要控制运动强度和出汗量,强度太大,一方面容易累,从而导致遗精,另外就是容易受伤。大量出汗也伤身体,因为大汗伤阳,我现在运动很注意控制出汗量,有出汗的苗头了,我就降低运动强度,避免出大汗,汗者,精气也!能量也会通过出汗的方式流失掉,所以一定要严格控制出汗量。偶尔出一次大汗,问题不大,如果经常出大汗,身体是会虚掉的,会感觉很累。戒色后是应该适量锻炼,有利于促进身体恢复,我现在也在坚持晨跑,但锻炼一定要控制强度和出汗量,这点一定要注意。懂得控制,状态就会越来越好,不懂控制,很可能越练越伤。
\end{case}

\begin{case}
    昨天晚上吃肉太多,今天早上遗精了,然后刚才睡了会午觉又遗精了怎么办?以前都是一个月差不多遗精3次,戒了几个月了。
    \subparagraph{分析} 吃肉太多是可能导致遗精的,之前就有不少戒友反馈过,比如过年期间吃肉过多,就遗精了。平时吃肉过多也可能遗精,吃肉多也容易助长邪念,还容易产生嗔恨心,所以戒色后一定要减少吃肉,能吃素是最好的,我现在就吃素了,以前是从减少吃肉开始的。吃素也要注意营养全面,不可太单一。
\end{case}

\begin{case}
    我只要晚上喝水多了就会遗精,少喝就没事。
    \subparagraph{分析} 晚上对各种因素会比较敏感,特别是入睡前一个小时,一定要注意避免导致遗精的因素。晚上喝太多水会加重肾脏负担,可能导致憋尿,也会影响固摄能力,从而导致遗精。晚上应该少喝一些水,控遗的细节一定要注意。很多因素一开始不会去注意,等到遗精后才引起关注,在平时就要认真研读控遗的文章,掌握控遗之道。在遗精后应该认真查看 65 条因素,看看自己是否触犯了某条因素,如果触犯了,下次一定要注意避免。
\end{case}

\subsubsection{控遗的方法}

\paragraph{加强版固肾功}

我在第三季就向大家推荐了固肾功,我到现在每天还在做固肾功,还在每天坚持,因为我深感控遗的重要性,实在马虎不得。这季我推荐一个加强版的固肾功,就是左右脚前后分开站,比如向前迈出左脚,然后向下做固肾功,整条左腿的拉紧感会变得超强烈,左腿做好后,就换右腿做。两条腿都做好后,再做我之前推荐的原版固肾功,双腿与肩同宽的固肾功,然后还可以变换站距,与肩同宽是一种,再宽一些是第二种,更宽一些是第三种,三种的拉紧感也会有所差别,可以自己体会,然后在做固肾功时,也要注意身体重心的调整,重心调整得好,拉紧感会更强烈,效果会更好,一般而言,身体重心可以稍微靠前些,这样感觉会更强烈。

固肾功可不可以睡前做?有的人说不行,说是会导致遗精,而有的人说可以。我的答案是可以,不过是因人而异的,是否可以睡前做固肾功,要看个人的体能情况,体能差的话,一做就累,一累就遗精,而体能好的人则没事。我现在睡前一般做五十个左右,做的是加强版固肾功,配合原版,很快就能强化拉紧感,然后带着较强的拉紧感上床睡觉,这样控遗效果比较好,加强版固肾功有望做到两个月不遗精。

有腰突的人能不能做固肾功?腰突患者可能不适合弯腰做固肾功,可以站立压腿,或者采用其他的控遗方法。轻微腰突患者还是可以试试固肾功的,如果感觉做了之后,腰部症状加重,那就不要做了,自己灵活把握。

做固肾功不宜过快、过猛,以免拉伤,应该注意节奏,速度适中,找准拉紧感是关键,注意把握强度,避免劳累。

\paragraph{元音老人的控遗方法}

\begin{quotation}\it
    元音老人:漏丹是做工夫的大忌。你年纪还轻,情欲未净,免不了要遗精。现教你一法,以固精关。每日早晚各做一次。先将胸腹中浊气从口中呼出,然后闭口,提肛、以鼻吸气,吸足气后,停息(不呼不吸),然后呼气松肛。闭气的长短随其自然,不可勉强。动作要慢,越慢越好,这样做三十次。做时不可有欲念,否则亦不易收效。做时如觉胸闷,可做四至五次,放下休息、散步,数分钟后再继续做。

    问:您所授固精之法,做完一个月后是否应继续修下去?三十次做多长时间为宜?

    答:最好修下去,因你心不固。时间长短须看自己的肺活量和闭气的时间而定,不可人人一样。也不可拼命闭气,但须缓缓进行,越慢越好。(元音老人《问答集》)
\end{quotation}

修道方面把遗精叫漏丹,的确是做工夫的大忌,频繁遗精,身体会变差,很影响修行,做日课也会质量下降。元音老人这个方法,我看道家也有推荐,是一个提肛的方法,配合停息,为什么要停息?停息时,能量就开始向上走了,有还精补脑之效。做任何功法,净心是前提,不能有意淫。提肛功不宜做太多,否则也可能出现便秘、尿频等不适症状,这点要注意,前几天就看到有戒友反馈做了提肛出现便秘的问题。提肛功很好,但自己要注意调整,不宜做太多。

\paragraph{南怀瑾先生的鸟飞式}

\begin{quote}\it
    要做到不漏,有一个“鸟飞式”的方法可炼,这是对治的一味药,现在介绍给大家。每天睡觉以前,站着,脚后跟分开,前八后二(两脚后跟距离约二寸)。第一步,臀部肌肉夹紧,不是提缩肛门,肛门收缩久了会成便秘的,小腹收缩。第二步,两手作鸟飞状,自然地,慢慢地举起来,动作要柔和,嘴巴轻轻地笑开,两肩要松开,两手各在身体左右侧,不要向前,也不要向后,很自然地举起来,越慢越好。与手上举同时,把脚跟提起来,配合姿势向上。第三步,手放下来时,嘴巴轻轻闭起,同时脚跟配合慢慢放下。站着时用脚的大拇指用力,姿势一定要美,要柔和,越柔和越好,重点在手指尖。手一起来,自然有一股气到指尖,到手一转,气拉住了,会自然地下来,白鹤要起飞时,就是这个姿势。每晚睡觉以前做,开始时做十下,做时两腿肌肉会发痛,以后慢慢就好了,慢慢增加次数。(南怀瑾先生)
\end{quote}

南怀瑾先生这个鸟飞式也挺不错的,可以搜南怀瑾鸟飞式视频,那个动作很简单,关键要点就是臀部肌肉夹紧,不是提肛,是臀部肌肉收紧,配合小腹收缩,就像把能量夹住并且内收一样,这里要注意一点,就是小腹不要像腹式呼吸那样鼓动,仅仅是收缩即可,睡前鼓动小腹(下丹田),是容易导致遗精的。鸟飞式包括提踵和手上举,是比较优美的一个动作,就像白鹤起飞一样,大家有兴趣的话,也可以试试这个鸟飞式。

\paragraph{子午养生桩}

两小臂抬于腰间,与上臂成 90 度,两小臂平行,左腿向左横跨一步与肩同宽,沉肩坠肘,含胸拔背,头正身直,百会、会阴成一直线,下颌微收舌顶上腭,嘴唇微闭,全身放松,意守丹田,逆腹式呼吸,逐步达到呼吸细长、均匀,站桩时间,根据年龄与身体状况而定,由短而长,争取每次达到 30 分钟以上。

子午养生桩是我推荐的一个养生桩,效果显著,而且相比于浑圆桩要简单,而且不累,容易坚持,浑圆桩要双手架着,时间长了难以坚持,而子午养生桩是双手收于腰间,比较好坚持,上手比较快。关于意守丹田的问题,南怀瑾先生也说过,如果守得不对,很可能导致遗精加重。守窍的要妙,在于守而不守,先存后忘。假如死守一处,则易气机凝滞,淤塞不通,恐生弊端。睡前不可意守下丹田,否则很容易导致遗精,练习子午养生桩,也可以不守丹田。我现在有时也会站桩,华池水下来很快,感觉很妙,练完养生桩就感觉身心和和美美的,非常不错。养生桩可以调整五脏六腑,因为身体失调而导致的遗精问题,就会大为改善,前段时间一位戒友就是通过练习子午养生桩成功控遗的。

\paragraph{吉祥卧}

吉祥卧的方法:身体向右面侧卧,两腿合拢、微微弯曲,左手轻放在左大腿外侧上,右手拇指伸开,把右手大拇指放在耳垂后面的凹陷中。从生理学和睡眠卫生的要求来说,以双腿变屈朝右侧卧的睡姿最合适。这样,使全身肌肉松弛,有利于消除疲劳;心脏在胸腔内位置偏左,右侧睡心脏受压少,可减轻其负担。人在吉祥卧的时候,就把三条脉的会聚点“三江口”给折起来了,一折起来,就不容易漏了,你一折起来能量就往头上走了,能量很容易从这个尾巴骨升到你的后脊梁背,甚至还会从中脉往上升。

吉祥卧刚开始容易不适应,有的戒友反馈说睡着时是吉祥卧,醒来时就散了,变成仰卧了,或者左侧卧了,左侧卧是一个易漏的睡姿,容易出现遗精,右侧卧是最好的睡姿,但养成需要一个过程,刚开始双腿不适应,手放着也不适应,关键是腿部要弯曲,手部可以宽松些。坚持练习吉祥卧,慢慢就舒服了,适应了。吉祥卧可以有效减少做黄梦和遗精的可能,是值得练习和养成的一个睡姿。吉祥卧的难点就在于固定睡姿,睡着时吉祥卧,醒来也能保持吉祥卧,这是需要练习和养成的,自己也要善于想办法,之前有戒友做了一个纸盒,他睡在里面,只能曲腿睡,睡一段时间,不用纸盒,也能保持吉祥卧了,这的确是一个办法,也有绑腿的,总之一定要下决心、想办法养成吉祥卧。

\paragraph{铁板桥}

习者仰躺在两条凳上,一凳放在脚后跟部,一凳放在双肩部,使身体中段悬空,身体挺直,有如一座横架两崖的金刚铁板桥。是练武之人提升腰背整体力量的一种功法,常练可以增强体内阳气,使腰背力量强大。一般每次练三到五分钟就可以了,初学者一开始可能连一分钟也很难坚持下来,循序渐进就好。

铁板桥有利于控遗,因为强化了腰背的力量,但同样有一个体能的问题,有的戒友体能不行,做了铁板桥后感觉很累,结果反而遗精了。这种情况我建议可以试试改进版的,就是肩部靠在床上,双脚则是站在地上,这样也可以锻炼腰背。改进版的强度会下降一些,但效果也是不错的,我有时就练改进版的。

\paragraph{核心力量训练动作}

\begin{multicols}{3}
    \begin{itemize}
        \item 俄罗斯转体
        \item 平板支撑
        \item 仰卧摸膝
        \item 仰卧交替触踝
        \item 坐姿剪刀式踢腿
        \item 仰卧触踝
        \item 鸟狗式
        \item 自行车卷腹
        \item 仰卧起坐
        \item 仰卧举腿
    \end{itemize}
\end{multicols}

这些动作可以百度搜到,核心力量训练对于控遗也有着强大的效果,可以很好地锻炼腰背、腰腹力量,也可以帮助腹部塑形,可以练出腹肌。这些动作可操作性强,可以在床上做,也可以在瑜伽垫上做,都是可以的。腰腹力量强了,就不容易遗精了,肾就位于腰腹这个位置,所以核心力量训练的价值很高,我个人的体会就是做完一套核心力量训练,感觉整个人被盘活了!感觉充满了活力。刚开始不要做太多,贵在坚持,一定要根据自己体能来安排,避免过于劳累。核心力量训练近年来很流行,国外那些篮球明星也经常做核心力量训练,他们能完成高难度的动作,和核心力量训练是分不开的。戒色后健身是很好的选择,健身可以让精气神更棒,但一定要注意锻炼强度和动作的准确性,坚持健身,可以让身体重返强健,养生和健身我是很注重的。

\paragraph{忏悔改过,行善积德}

出现频遗问题,第一选择就是做养生功法,既好又不用花钱,第二选择就是看中医调理,要对症。如果第一选择和第二选择都不太理想,那就可以试试忏悔改过、行善积德、念佛持咒念经等,直接提升振动频率,提升正能量,心善了,会影响到身体,身体的情况也会跟着变好。据我所知,某些戒友就是通过大力行善改善遗精的,养生的最高境界是养心,心要善,心要慈悲,对治负能量的念头,多发正能量的念头,这样身体频率就上去了,频率上去了,身心感觉就会焕然一新,之前的一些身体失调就会自动恢复正常。有些人就是在嗔恨生气后遗精的,怨恨恼怒烦等负能量情绪或念头是很伤身体的,会导致身体失调,这样就会出现遗精。所以对治频遗,也需要注意情绪管理,避免负能量。

\paragraph{控遗注意要点}

\begin{multicols}{2}
    \begin{itemize}
        \item 首先必须净心,也是名医张景岳强调的,心是本,净心是求本之道。控遗必须要克服意淫,让内心清净。
        \item 别生气别喝酒,肉也少吃,最好全素,饮食保持清淡,辛辣刺激的食物不要吃。
        \item 晚上少喝水,晚饭少吃,睡觉不要盖太厚太暖,不要熬夜久坐。
        \item 注意避免劳累,适当锻炼,不能出太多汗,不能过于劳累。
        \item 调整好心态,负面情绪不利于身体恢复,心态要平和,不发脾气,不烦躁着急。
        \item 早睡早起,作息饮食规律,不要吃夜宵。
        \item 力行善事,心善语善视善行善,正己化人。
        \item 避免思虑过多,注意放松心情,别给自己太大的压力。
        \item 避免紧张,避免着凉,不要吃冷饮,注意保暖。
        \item 固肾功加吉祥卧,这是双保险,再加一个提肛或者养生桩,就是三保险。
        \item 遗精之后要高度重视,不然很容易出现第二次遗精,也就是连着遗两次。
        \item 遗精后要加固精关,做好固肾功。
        \item 不要贪睡,不要赖床,不要睡回笼觉。
        \item 晚上不要按摩涌泉穴,否则容易遗精。刺激穴位,手法不宜过重。
        \item 晚上不宜艾灸,晚上艾灸也可能导致遗精。
        \item 睡前泡脚水不宜过热,否则也可能导致遗精。
        \item 遗精后要注意调整心态,加强养生,同时要提高警惕,严防心魔进攻。
        \item 睡前不要打坐,也不要腹式呼吸,否则容易出现遗精。
        \item 补药不要乱吃,有些补药吃下去容易出现遗精。
        \item 遗精后症状易反复,要注意休养。
        \item 养成良好的生活习惯,改掉不良的生活习惯,避免色情刺激。
        \item 注意精神调摄,消除对遗精的恐惧、焦虑等不良心态。
        \item 内裤不宜过紧,避免对生殖器的刺激。
    \end{itemize}
\end{multicols}

\paragraph{总结}

这季是控遗之道的终极篇,相对于之前的文章更详细了,以后有机会还可以讲讲遗精的问题,这季分享的控遗方法,大家都可以试试。频遗是恢复的拦路虎,不遗为大补,频遗乃大伤,控遗是系统工程,也是一门很深的学问,一定要学会控遗。导致频遗的因素是很多的,不能单一地归为某一类因素,有的人会说遗精都是业障,或者遗精都是正气不足,还有的人因为自己是劳心太过导致的遗精,所以他就狭隘地认为遗精都是自己想多了。这些都只能代表其中一类因素,即使坚持行善积德的人,也可能出现频遗,虽然坚持行善,但如果过于劳累或者饮食不注意,又或者久坐、着凉等,那也可能出现频遗,遗精有内因、外因之分,是需要综合考虑的。把频遗关过了,恢复的进程就会加快很多,能量很容易积攒起来。每一位戒友都要学会控遗之道,也许你现在没遇见频遗问题,但将来的某一天也许就会遇见,克服了频遗,就进入了身体恢复的快车道,可以积累更多的底气和自信。

有师兄也分享过拱头功,拱头功也有一定的控遗效果,但毕竟是对着硬物拱头,一定要慎重练习,之前就有戒友反馈练了几个月,颈椎突出,或者练了之后颈椎痛。这应该是练习不得法的缘故,不能用力过度,用力的角度也有讲究,站立的重心也要注意。就像搬重物,如果姿势不对,也可能导致腰突,腰椎和颈椎一定要保护好,这两个部位太重要了。前段时间聊了一位戒友,就是帮家里干活,把腰给伤了,在搬重物的时候,脊柱尽量不要弯,要蹲下搬,不要站着弯下腰去搬,正确的姿势是先蹲下,抱着重物,然后用腿部的力量将重物搬起来,不要弯腰去搬,弯腰搬重物很容易伤到腰。练了拱头功如果出现颈椎疼痛,建议先暂停练习,等恢复正常再说。练习一定要得法,姿势要正确,否则练习不当,是可能伤身的,这点要切记!不是每一个控遗功法都要做,选择适合自己的去坚持,希望每一位戒友都能成功控遗,都能有一个比较好的恢复进度,早日恢复身心健康。

我们平时也要学会做遗精的总结,每次遗精一定要学会总结分析原因,如果是一个月一次的遗精,那就不必担心,但也要总结一下。希望每一位戒友都能做“积精家”,好好积攒宝贵的能量,能量足了就能成为精力超人,在学业和事业上就像开了挂一样,别人已经累了,而“积精家”的“电量”还很足,精力还在强势续航,有的戒友反馈工作一天都不感觉累,还是精力充沛,这种状态真的太棒了,干事业就需要这种精力状态,有精才有力,所以叫精力!大家好好积精吧!

下面分享一首诗歌。

\begin{poem}[童年的天空]
    \begin{multicols}{4}
        \begin{center}~\\
            童年记忆里的天 \\ 是那么蓝 \\ 蓝得那么纯粹 \\ 那么空灵 \\ 那么不可思议 \\ 天空的云 \\ 是那么写意 \\ 那么舒展 \\ 那么轻盈 \\ 就像一场梦 \\ 我的心 \\ 像蓝天白云一样 \\ 一尘不染 \\ 我站在天空下 \\ 仰望蓝天白云 \\ 被某种莫名的感觉 \\ 深深打动 \\ 童年,有大自然,有玩伴 \\ 有蓝天白云的陪伴 \\ 一切是那么美好 \\ 那时的我无忧无虑 \\ 纯真地度过每一天 \\ 自从邪淫后一切都变了 \\ 纯真消失了 \\ 内心变得沉重了 \\ 很难开心起来 \\ 感觉活得很压抑很累 \\ 戒色后童年的蓝天白云 \\ 又都回来了 \\ 那种感觉很熟悉 \\ 突然间就发现 \\ 那是童年的感觉 \\ 那么纯粹 \\ 那么美好 \\ 那么快乐
        \end{center}
    \end{multicols}
\end{poem}

下面分享一本书。

\begin{book}[《天堂的证据》,埃本·亚历山大(Eben Alexander)]
    埃本·亚历山大(Eben Alexander),美国知名神经外科医生。该书根据作者自己的真实经历创作,该书全部以第一人称叙述,2008 年,他患上了细菌性脑膜炎,昏迷了七天七夜。当他幸运地被抢救苏醒过来之后,他回忆起自己的濒死体验,认为那就是通往天堂的道路,通俗点说,这是一个在鬼门关转了一圈的人的回忆录。在 2012 年 10 月出版发行,连续 15 周蝉联《纽约时报》、亚马逊排行榜冠军,一经面世就在国际出版界引起轰动,之后一个月内卖出英国、德国、荷兰、俄罗斯、澳大利亚、巴西、日本、韩国等 35 国版权。濒死体验是我一直很感兴趣的,这本书分享的内容很震撼,作者是一名神经外科医生,在濒死时有了其他维度世界的体验,这本书的很多细节值得一读再读,的确是很宝贵的信息。人死不是灯灭,等脱下这件“皮大衣”,就会独自踏上后世的旅程,万般带不去,唯有业随身,活着的时候一定要好好行善积德。
\end{book}
