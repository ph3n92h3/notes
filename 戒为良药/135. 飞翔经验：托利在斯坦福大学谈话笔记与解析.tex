\subsection{托利在斯坦福大学谈话笔记与解析}

我在第 128 季分享了《当下的力量》笔记分享与解析,很多戒友很感兴趣,反响很好。戒色后在传统文化方面,我首推《了凡四训》,《了凡四训》告诉你改造命运的方法,全书分为:立命之学、改过之法、积善之方、谦德之效四个部分,很值得我们学习。“命由我作,福自己求”,一个人做到“立命、改过、积善、谦德”这四点,那么福气、运气自然会降临。在灵性提升方面,我首推《当下的力量》,其实很多修行书籍都很好,不过《当下的力量》更具有普适性,接受度是非常高的,不管你信不信佛,都可以学习一下,都会有所助益。人生最快乐的事情之一,就是在对的时间遇见对的书。《当下的力量》畅销全球,是一本不容错过的书。

托利的视频我这些年看了很多,的确受益匪浅,每隔一段时间看一下,就能获得更多的领悟,体会出更多的感觉,好的开示要反复看,经常看,也许现在看没什么感觉,但是过一段时间,当你的觉悟提升到一定程度,到时再看,就会恍然大悟,拍案叫绝,欢喜雀跃,一下就明白了。我现在已经记不清看黄的兴奋了,也对那种兴奋不感兴趣了,我现在更喜欢顿悟的兴奋,那种突然开窍、突然明白的瞬间感受,非常振奋,比中大奖还高兴。托利很有喜剧天分,是我个人比较喜欢的国外灵性导师,总能把我逗乐,托利模仿效果是一绝,把那种状态模仿得惟妙惟肖。

上季一位戒友对《当下的力量》产生了误解,认为里面的有些内容消极,其实不然,托利也有自己的家庭,也有妻子,也有自己的事业,他也在积极奋斗,并不消极。我们一定要正确理解里面的内容,不要因为某些段落而产生误解和偏见,这点很重要。我们要有求同存异的包容态度,学会圆融理解,不可根据自己的误解和偏见来随便反驳,要有尊重的态度。只有当自己有了尊重和恭敬的态度,才能从书籍中获得真正的领悟和启发。

《当下的力量》里面讲纯意识的实现,纯意识就是本体(being),being 这个词也翻作“存在”。

\begin{quote}\it
    无论何时,当你观察自己的思维,你就把意识从你的思维形式中抽离。结果,观察者——超越形式的纯意识——会变得更为强大,而思维的形式结构则变弱了。
\end{quote}

《当下的力量》里面的这段话很经典,也是我比较喜欢的一段话,这段话揭示了五个要点:\begin{itemize}
    \item 你可以观察自己的思维;
    \item 你不是思维,你是观察者;
    \item 观察者就是纯意识,是超越形式的;
    \item 每当你观察思维,纯意识就会变得更强大,思维对你的掌控会变弱;
    \item 当你开始观察思维时,就能把意识从思维中抽离。
\end{itemize} 这五个要点大家认真领会一下,这段开示非常重要,值得花几年时间去真正领悟。当你练习得更多,回过头来再读,就能领悟到更深的内容。

有些戒友可能会想:“即使我明白了我是观察者,我是纯意识,这对戒色有什么用呢?”用处太大了!当你从思维者转变到观察者,你就解除了对思维的认同,你突然领悟到自己不是思维,而是纯意识,是那个观察者,你慢慢学会了观心,观心断念是一体的,这样就能决胜戒色实战,就能真正主宰内心。有的戒友也在观心,有的戒色前辈也在文章里提倡观心,很多人都知道观心这个词,但是他们中的大多数都没有从思维者转变到观察者,身份认同没有真正切换过来,他们虽然在观心,但并没有认识到那个观察者是真我,思维不是真正的自己。这点认识太关键了,有了这点认识,就可以和大多数人拉开差距,一下就进入了真我的层次。

下面是托利的一段自述:

\begin{quote}\it
    从伦敦大学毕业后,我无所事事,晃荡了大概一年。就在这一年,我变得更不快乐、更忧郁。然后,就在我 29 岁那年,我的内在突然发生了转变:我半夜醒来——夜半惊醒,对我来说并不稀奇,强烈的忧郁感、极度的恐惧感,那夜再度向我袭来,然而这次我的心中却生起一个念头:\begin{center}
        「我再也受不了我自己了。」
    \end{center} 这念头萦绕在我心中:「我受不了我自己……」突然间,我看着这个念头——这就好像我后退一步,看着它。
\end{quote}

以上是托利的自述,就在他感到忧郁和恐惧感时,他顿悟了,他看着自己的念头,就好像后退一步,看着它,托利突然领悟到自己是观察者,不是那个念头。负面念头往往会让自己痛苦,如果不认同念头,开始观察念头,就能摆脱负面念头带来的痛苦。这是一个惊人的发现,我以前也是顽固地认同念头,非常顽固,因为认同了太久,念头一来,我就跟着跑,缺少觉知,陷进念里,有时一跑就是几个小时,感觉特累,负面念头更是让我感到痛苦和惶恐,相信大家都有过这种体验。当你陷进念里,缺少觉知,当时就会有一种感觉,那就是——自己就是念头!会产生这种错觉,好像自己和念头是一体,念头就是自己,自己就是念头,会有这种感觉。特别是含有“我”字的念头,更加强化了这种感觉,我、我、我,各种“我念”,让你误以为自己就是念头,彻底认同念头为自己。

\begin{quote}\it
    你的内在有一个维度,它比你通常认为的自己还要深,它比自我还要深——而「自我」就是多数人所相信的、由个人的过往所构成的假我。在你为当下这一刻腾出内在的空间时,你就有机会认出那个维度。(托利)
\end{quote}

那是内在空的维度,也就是纯粹觉知的维度,一个是念头的维度,一个是空的觉知的维度。人们害怕空,因为他们牢牢认同思维为自己,所以会害怕空,我以前也是这样,听到空,有点害怕,为什么会这样?因为我误解空就是什么也没有,好像死了一样让人害怕,其实这是一种误解。空的维度是纯粹的觉知,并不是死了,而是纯粹的存在,无我方显真我,一个是念头的小我,一个是觉知的真我,慢慢你就会顿悟这两者之间的区别。

这季是关于托利在斯坦福大学谈话笔记与解析,那个视频的质量挺高的,大家有空可以看看,我做了一些笔记,希望和广大戒友做一个分享。

\subsubsection{托利在斯坦福大学谈话笔记与解析}

\begin{quote}\it
    那个更深层的存在感,我现在称之为意识本体。
\end{quote}

\subparagraph{解析} \textit{我们可以将那更深、更根本的,称为「存在」、「生命感」甚至是「更深的智力」,它大过你脑中的念头。最根本的层次来说——也就是「真正的我」、「最深层的存在感」,它是存在的层次,无论你处于人生的哪个阶段,人生的境遇如何,就这个最根本的层次上来说,「真正的我」是不用添加任何东西的。(托利)}

纯粹的存在(pure being),它是源头智能,更深的智力,它比念头更厉害!它大过念头!它就是真正的我——纯意识——纯粹的觉知!认识它!我个人认为来到这个世界上,没有认识真我,真的有点白活了,活了一辈子都不知道真正的自己,非常可悲的状态。认同念头和身体为自己,是一种很强的编程,要解开这个编程,需要顿悟,即使有了顿悟,也需要一个缓慢转变的过程,这个过程大概需要一年左右,彻底转变过来大概需要好几年。很多人即使听到了关于真我的开示,也有了很好的领悟,但他们还是会习惯性地认同念头,这是过去的惯性所致,所以转变需要时间,慢慢就能做到了。

\begin{quote}\it
    每个人都熟悉这种头脑中的自言自语。有时我称之为脑子里的声音,它是自动的,强迫性的。可以说是永不停止的思绪过程。对你周围的一切发生评头论足,不停地评论。(喋喋不休的评论)
\end{quote}

\subparagraph{解析} the voice in your head!头脑中的声音!自言自语,喋喋不休,对于这个现象,相信大家都有很深的体会,但不一定对这个现象有很深的认识,它是自动的、强迫性的,它会自动冒出来,强拽着你,把你带跑,你只要一跟,就是一连串的念头,一个念头接着一个念头,没完没了!而且还充满各种评判,不停地评论。大家可以想象自己的头是一个空的房间,然后里面会响起声音,这就是 the voice in your head!自动强迫的念头,如果缺少觉知,就会被它带跑,彻底陷进去。

\begin{quote}\it
    那个脑子里的声音,大多数是负面的,大多数是负面的自言自语的反映。
\end{quote}

\subparagraph{解析} 那个声音大多数是负面的,各种负面的念头,怨天尤人,说别人坏话,说别人不好,嗔恨别人,怨恨别人,嫉妒别人,贪婪,抱怨,各种负面的评判,这是那个声音的特点——负面!大多数都是负面低频的念头,出现那种念头会让自己痛苦,感觉糟糕。大德告诉我们要断除负面的念头,多发善念,这就是在修心。

\begin{quote}\it
    意识到这脑中的自语,就已经是向前迈出了一大步。
\end{quote}

\subparagraph{解析} 很多人虽然体验了自言自语,却没有充分意识到这种现象,因为他们太认同念头为自己,跟着跑,就会感觉念头是自己。当我们意识到脑中的自言自语,就是在观心,我们看到了念头,发现我们是观察者,不是念头,这需要顿悟。能意识到脑中的自语,就是一大进步了,这是解除错误认同的第一步。

\begin{quote}\it
    头脑(mind)可以是一个很棒的工具,但是对很多人来说,生活中的 80\% 时间,头脑是一个折磨人的刑具。
\end{quote}

\subparagraph{解析} \textit{我还是会有念头,不过只有在必要时,我才动用思维。虽然念头偶而也会来来去去,但就在长久的无念当中,就在念头与念头间长久的空档之中,便是对那内在平静的美好经验。同时我也了解,这内在的平静一直都在,就算是当我过去苦于焦虑时,它也在,只不过是被焦虑掩盖、被过度活跃的心理活动所掩盖。(托利)}

有的戒友问:“认识了真我,是不是不能起念了,那不是什么事情都干不了了?”念头是工具,这是它正确的位置,该起用时要起用,学习工作都需要用到念头,只是在必要时才用,就像一台车,大部分时间是静止不动的,等到要用时才开动它,它只是一个代步工具,如果汽车不听你指挥,在马路上横冲直撞,不受你控制,你想停,它却在冲,这绝对会导致车祸!念头也是如此,它是工具,要学会正确使用它。当然它也像疯猴一样不消停,会不断冒出各种念头,把你带跑,它是带跑高手,当你具备高度的觉知,就能降伏它。

念头与念头之间,那个空档,那个平静的纯意识,那么平常,简单,纯真。Quiet the mind to see the being!(安静念头以便看到存在!)这是《终极自由之路》里的一句话,我特别喜欢,特别有感悟,真我被念头所 cover(掩盖、遮蔽),安静念头,就能看到存在!纯粹的存在!(Pure being!)念头是一种遮蔽和掩盖,这是我 2019 年加深的领悟,就是这个遮蔽的机制。纯意识本来就在,大德说我们“本来是佛”,就是被念头遮蔽、掩盖和限制,产生了错误的认同,不认识真我,造业受报,所以苦啊!

要认识这个真东西,头脑必须足够简单,单纯,甚至有点傻傻的,才能认识它!它是那么简单,只有简单纯真的人才能认识它,珍惜它,信任它。头脑一复杂,各种概念太多,就会很难认识它,也很难相信它。我从元音老人那里知道了这个真东西——纯意识!这些年一直在加深认识它,体会它,它是一个空的状态,空的维度,简单而不可思议,简单而深奥。空里面有快乐,空里面有满足,空的质感是最微妙的,最美好的,也是最无敌的。

\begin{quote}\it
    第一步,就是认出那个声音。
\end{quote}

\subparagraph{解析} 第一步就是认出那个声音不是自己,打个比方,你头脑里有一只知了在那叫,那只知了就是念头,你在听它,而你不是它。你是听者,你是观察者,你是纯粹的觉知,你不是那个声音。这个体会大家可以感受一下,走到树下,听见知了叫,我会抬起头在树上找知了,从小时候我就喜欢这样,所以这个比方很生活化。第一步,就是认出那个声音不是真正的自己,那是一个冒牌货,一直在冒充你,

\begin{quote}\it
    我们称之为,另一个维度的意识进入了,我们可以称这个维度为觉知。
\end{quote}

\subparagraph{解析} 两个维度,一个念头的维度,一个觉知的维度,我们就是要认识这个纯粹觉知的维度,它一直在我们里面,我们却一直没有认识它,这么多年一直认念头为自己,跟着念头跑,一直不知道还有这个觉知的维度,对这一点一直很迷茫,没人来告诉我们,直到我们遇见了善知识,帮我们点破,豁开正眼。

\begin{quote}\it
    认识到你的意识可能有两个维度,一个是思维的维度,还有一个更重要、更深层的维度,那是纯然觉知的维度。
\end{quote}

\subparagraph{解析} 一个是思维的维度,还有一个是更深的维度——纯粹觉知的维度。要认识这一点,要分清它们之间的区别。悟性高的人很快就能分清,悟性差的人可能需要好几年。

\begin{quote}\it
    你可以观察到其中经过的念头,那个对念头的觉知,本身不是念头,它只是能观察念头的能力,就像是照在念头上的光,我称之为纯粹的、无制约的觉知。
\end{quote}

\subparagraph{解析} 你可以站在马路边,看着川流不息的行人,那些行人好比脑海中的念头,而你是一旁的观察者,和念头拉开了一个距离。如果你认同念头是自己,跟着跑,就会和念头合而为一,感觉念头就是自己,如果你作为观察者,就能拉开距离观察念头,就像一束光照在念头上。

\begin{quote}\it
    这种不断地对思绪的认同……换句话说,你的自我感,是从念头中衍生出来的。
\end{quote}

\subparagraph{解析} 认同思维会产生一种虚假的自我感,小我(ego)是从“我念”中衍生出来的,“我念”就是我字当头的念头或者含有我字的念头,会产生这种自我感,比如我要看电视,我要出去玩,你不理我了吗?说这些句子时,就会产生一种自我感,那种自我感来自于念头,是虚假的自我感。

\begin{quote}\it
    有一个更深层的自我感,不是念头的。
\end{quote}

\subparagraph{解析} 更深层的自我感,是真我的感觉,是存在(being),它不是来自于念头,而是超越了念头。

\begin{quote}\it
    对思绪的完全认同,完全缺乏觉知。
\end{quote}

\subparagraph{解析} 缺乏觉知的结果,就是跟着念头跑,认同念头为自己,就会出现这种现象。一旦觉知了,就能主宰念头了,念起即觉,觉之即无,有了觉知,就能消灭念头,就不会被念头带跑。缺乏觉知,必然被带跑,必然陷进念里,一个念头接着一个念头,深陷其中。

\begin{quote}\it
    从持续的思绪中退出来,不是那么的困难。
\end{quote}

\subparagraph{解析} 一旦陷进念里,就会不断持续下去,要退出来,也不是特别难,关键要有觉知,有的人跟着念头跑了十几分钟,突然意识到自己已经跑偏了那么久,在意识到的那一刹那,就回到了觉知。关键要练习,在念头出现时就意识到,不能跑了那么久才发现,刚开始练习,拉回的速度比较慢,坚持练习,拉回的速度就会越来越快,拉回的速度越快,就说明觉察力越强。

\begin{quote}\it
    实际上就是,从思绪中抽离,进入当下。
\end{quote}

\subparagraph{解析} 这句很简明,当你发现自己陷进去了,就抽出来,回到纯粹的觉知,也就是托利说的当下。这注定是一场持久战,一次次陷进去,缺少觉知,跟着跑,开始练习后,还是一次次陷进去,因为惯性实在太大,觉察力还不够强大。不放弃,继续练习,一次次陷进去,一次次抽出来,抽离得越来越快,陷进去的次数越来越少,在不断的练习中,觉察力强大起来,到时就能轻松进入当下了。

\begin{quote}\it
    瞥见那个难以形容的东西,或者不是东西的东西,那个东西和注意力没有区别,你会认识到那个注意力本身就是。
\end{quote}

\subparagraph{解析} 那个东西不是个东西,是一种意识状态,是一个不是东西的“真东西”,几千年以来,这个真东西一直是修行的最核心,大德高僧们给这个真东西起了很多名字:佛性、道、菩提、本心、本来面目等等,这类名字可以起无数个,但指向只有一个——纯意识。要认识这个真东西!这个真东西是空的觉知,和注意力没区别,是纯然的注意力。

\begin{quote}\it
    触及到那个没有思绪的觉知状态。
\end{quote}

\subparagraph{解析} 用心体会那个没有思维的纯粹觉知的状态,一开始感觉很平常,很普通,慢慢就会感觉到纯意识的美好与美妙。

\begin{quote}\it
    某个超越想法的东西。
\end{quote}

\subparagraph{解析} \textit{念头来来去去,你用不着跟随它们,就在来来去去的念头中,你发现念头之间有着空间。这点非常重要,也就是发现你的内在有这样的一个东西,它是念头和念头之间的空间。思绪不再是持续不断,突然间,你觉知到其中的空间,这让你感到非比寻常的美好和力量。你找到了真正的自己,如果没有了悟自己的本体,你一生就活在虚假的自我感中。(托利)}

认识念与念之间的真东西!不要跟随念头,要学会主宰自己的念头,做念头的主人。否则跟着念头跑,内心会很乱,跟着负面念头跑,内心会痛苦,会增加自己的负能量。

\begin{quote}\it
    只是观察,而不打标签。
\end{quote}

\subparagraph{解析} 纯粹地观察,不要起念,一起念就回到了思维的维度、思维的层面。从某种角度而言,也是回到了思维的牢笼、思维的陷阱、思维的监狱,囚禁纯意识的监狱!念头的一面是工具,另外一面是监狱,就像硬币有两面,有觉知的人可以控制念头,这样念头就会成为工具,缺少觉知的人会陷进念里,这样念头就成了监狱。

\begin{quote}\it
    那秘密就在于狗不思考,它只是体验你,它没有对你贴标签,它没有概念化的思维,它是直接的体验。如果你注视一只狗的眼睛,你可以感觉到那狗的存在,没有头脑思绪的遮蔽,那种纯粹的存在。
\end{quote}

\subparagraph{解析} 直接体验,不贴标签,有点像婴儿的状态,就是看着这个世界而没有概念化的思维,那种纯粹的存在,没有思维的遮蔽。幼犬很单纯,和纯真的动物在一起,也容易唤起自己的纯真,完全真诚、完全敞开的体验。

\begin{quote}\it
    一次只需要几秒钟,让自己临在。
\end{quote}

\subparagraph{解析} 我们要学会安住真我,扎根于纯粹的觉知,一次只需要几秒钟,慢慢安住的时间会延长。因为已经有了断念的修炼,所以安住会变得简单,随时都可以进入纯粹的觉知,让自己临在。短时多次,这是安住的原则,积少成多,慢慢就能扎根于真我了。

\subsubsection{《怎样停止头脑的自言自语》笔记与解析}

附:《怎样停止头脑的自言自语》笔记与解析,这是另外一个视频。

\begin{quote}\it
    首先,你需要知道头脑里在自言自语。(self-talk in the head)不是在你完全迷失于头脑里的自言自语好几个小时候之后。
\end{quote}

\subparagraph{解析} 发现要早,及时察觉自己陷进念里了,马上反应过来,抽离出来,不要迷失在头脑里几个小时,缺少觉知。

\begin{quote}\it
    为了成为自由的,你需要在头脑里的自言自语发生的时候就意识到它。
\end{quote}

\subparagraph{解析} 在自言自语发生的刹那,就要意识到它!速度要快,这里的意识到它,也就是觉察它,这样你才能成为自由的,成为内心的主宰,不会被念头束缚。

\begin{quote}\it
    这里需要有觉知的一瞥(a glimpse of awareness),从这一瞥中,你可以看见自己想从这自言自语中解脱,看见它不是你。
\end{quote}

\subparagraph{解析} 这段话肯定会让一些有觉悟的戒友想到:念起即觉,觉之即无。讲的就是觉察、觉知,通过有觉知的一瞥,你就能从念流中摆脱出来,并且看到念头不是你,你是观察者。不管是王阳明先生,还是《菜根谭》、《了凡四训》,抑或是其他大德,都在讲这个“觉”字,这个“觉”字是实战最核心的一个字,真正理解了这个字,就能把握实战的精髓。我发现很多戒友都已经领悟到了这一点,必须强化自己的觉察力,练好“觉”,实战中把“觉”用出来。

\begin{quote}\it
    没有觉知的时候,你就是这自言自语,你就是这头脑里的声音。
\end{quote}

\subparagraph{解析} 这段说得一针见血,缺少觉知,你就会认同念头,跟着念头跑,给你的感觉就是——你就是念头,你就是头脑里的声音。当你觉知了,你突然就变成了观察者,从念流中摆脱出来了。

\begin{quote}\it
    自由就是你可以选择将你的注意力从思考中撤离出来。
\end{quote}

\subparagraph{解析} 自由来自于真正主宰自己,不做念头的奴隶——被念头控制、摆布、奴役,被邪念附体,去做坏事。能把注意力从思考中抽出来,就是自由,否则一个念头接着一个念头,就是不自由,就是束缚,就是烦恼。

\begin{quote}\it
    你将你的意识、你的注意力从思考中收回。
\end{quote}

\subparagraph{解析} 意识会陷进念里,然后就像点开了播放的按钮,一直持续下去,没完没了。当你把意识从思考中抽出来,念流就中断了,你就自由了。

\begin{quote}\it
    从这觉知中你可以选择,你可以选择不再被思考的过程占据。
\end{quote}

\subparagraph{解析} 你可以选择不思考,安住于空的意识状态,你不再被思考所占据,你可以这样选择。

\begin{quote}\it
    你将注意力从思考中抽出来。
\end{quote}

\subparagraph{解析} 能觉察,就能抽出来,反之,缺少觉察,就会一直陷在念里,不断连续下去。

\begin{quote}\it
    你无法在觉知你的呼吸的同时,进行思考。
\end{quote}

\subparagraph{解析} 这是一个法门,观呼吸,觉知呼吸的出入,思维自然就停止了。这个方法挺不错,而且简单易行。

\begin{quote}\it
    因为你看见了这类思考的无益。
\end{quote}

\subparagraph{解析} 人一天中大部分的思考,几乎是无益的,就是胡思乱想,大家一定会发现这点,看见这类无益的思考,不要再陷进去了。

\begin{quote}\it
    过了一阵,你注意到自己又陷入思考,然后你选择重新将注意力抽离出来。
\end{quote}

\subparagraph{解析} 这就是拉锯战,持久战,当发现自己又陷进去了,就将注意力抽出来,一次又一次,慢慢陷入思考的次数会减少,觉察力会变强。

\begin{quote}\it
    慢慢地,你不思考的时间会越来越长,强迫性的思考会越来越弱。而真正的思考,充满创造力的思考会越来越强。
\end{quote}

\subparagraph{解析} 通过坚持练习,不思考的时间会变长,无益的思考、强迫性的思考会减弱,有觉知的思考、充满创作力的思考会变强,思考的效率会大幅提升,更有创意。

\paragraph{总结}

\textit{有一样纯粹的东西,它不会随岁月改变,无法添一丝一毫,不会减一寸一分。[Ram Dass(拉姆·达斯)]} 它就是纯意识,每个人都应该认识它——真我——纯意识——纯粹的觉知——纯粹的存在,也就是佛性!本心!\textit{心的真正本质,简单到令人无法相信的地步。(大宝法王)} 头脑比较简单、单纯的人,才能理解和认识这个简单的“真东西”,头脑比较复杂、塞满各种概念的人,很难认识这个“真东西”。

\textit{在你之内有另一个次元,我叫它“意识的次元”,我们要在自己之内找到那个次元,并使之加深。(托利)} 托利用的是 dimension 这个词:次元、维度。通过一定的学习、思考和领悟,就能认识到两个次元的区别,会发现纯意识的次元,它就在每个人之内,不断安住它,使之加深,不断加深,变得深厚、扎实、稳固。\textit{初步的自由解脱,就是了解到你不是那个思考者。在你开始观察那个思考者的那一刻,就启动了意识的更高层次。(托利)} 从思考者转变成观察者,就进入了纯意识的次元,也解除了对思维的认同。

进入 21 世纪的网络时代,是一个邪淫泛滥的时代,也是一个意识觉醒的时代,堕落与升华的机会并存,就看自己的选择和把握了。希望每一位戒友都能认识真我,安住真我,这样觉察力会变得越来越强,对内心的控制也会越来越强,到时自然越戒越好,越戒越稳定。认识了真我,在修行方面就能抓住核心,步入正轨,如果不认识真我,很容易盲修瞎练。认识真我后,修行才刚刚开始,万里长征的第一步,还需不断练习,不断安住,不断对治妄念,才能登堂入室,深入堂奥。

分享两个案例:

\begin{case}
    戒色最重要的核心就是观心断念,最开始我的断念水平也是非常差的,我不知道怎么去断,虽然每天都看《戒为良药》,每天练习念起即断,始终断不好。但是大家一定不要放弃,熟能生巧是真理,我就每天练习,把那十六个字不断地默念,念起即断,念起不随,念起即觉,觉之则无。不仅仅是默念,还要慢慢理解它的意思,慢慢地好像 yy 就少一些了,但是还是很差。飞翔哥说的,每个人的根器不一祥,我可能就是很差的那种。我不明白如何能够观察念头,如何不被它牵走。就在我不断练习的过程中,不断看《戒为良药》关于断念的文章,突然有一天,我在看到飞翔哥的一季断念文章的时候,突然就像开了窍一样,既然念头可以观察,那我并不是念头啊!我确实体验到,自己不是念头,在我观察念头的一刹那,念头就会消失得无影无踪,神奇!太神奇了!我觉得自己如获至宝,就从那天开始我的断念水平有了一个飞跃,我可以更快地断念了,提醒大家断念真的异常重要,它的重要性几乎是戒色这场战争当中最重要的核心,但是需要我们不断地学习,不断地领悟,不断地深入。
    \subparagraph{解析} 这是戒色一年的戒友的反馈,他真正开窍了,领悟到了。万事开头难,我们过去一直认同念头,跟着念头跑,不会观察念头,这种状态持续了十几年甚至几十年,现在开始戒色了,就一定要学会观心断念,不管何种戒色方法或体系,只要偏离观心断念,必然失败!观心断念是实战最核心的内容,一定要掌握。这位戒友刚开始也很差,不明白如何观察念头,但他没有放弃,不断练习口诀,不断学习断念的文章,突然就开窍了。就是在反复练习和反复学习的过程中,突然顿悟的。练习完了,就看看断念的文章,或者在练习前,先看看断念的笔记,学和练要交替结合起来,往往练过之后再看文章,就会突然明白很多。这位戒友的顿悟很不错,他明白了自己并不是念头,也做到了觉之即无!他说太神奇了,的确是这样,念头似乎很强大,会把你带跑,但是只要你观察它,它就会失去力量,消失得无影无踪,关键是我们的觉察力要强大,提升觉察力是核心!觉察力强,就可以瞬间让它消失!这位戒友的断念水平有了一个飞跃,和他的顿悟是分不开的,就这样他戒到了一年,相信他会戒得更好。
\end{case}

\begin{case}
    记得以前有一天我睡午觉起床上厕所的时候,意识不太清醒的情况下,一个念头冒了出来,一下没有断掉,结果就破戒了,破戒了以后我很烦躁,心情非常难过,经过很长一段时间我的大脑一片空白,一个念头都没有,又过了一段时间念头又起来了。因为我以前断念都是用断念口诀的,但是这次不一样,可能是我心情不好的缘故,念头起来的时候我是向念头瞪了一眼,死死地看着念头,结果念头居然消失了,每次当念头起来,我就这样瞪眼,看着念头!当时我还不知道这就是觉察的力量,第二天直到我在《戒为良药》听到这样的一句话:念头只是工具,身体只是载体,纯粹的觉知才是本体!你不是念头,念头也不是你!念起即觉,觉之即无!当念头起来的瞬间,立马用觉察去消灭念头!看到即消灭!这个看是“向内看”的力量,不是向外看!现在我每天还在强化觉察,这就是我在破戒当中无意学会了觉察,破戒真的能反映很多问题。觉察的力量其实每个人都有,但你能不能觉悟到就看你自己了,当你悟到了,你就能和其他戒友拉开十万八千里的距离,差距就是这么大!当念头起来时,第一时间一定要立马断掉才行,不然断晚了就会很被动,小火星到时候变成大火,就会不得不破!
    \subparagraph{解析} 这位戒友也学会觉察了,在一次破戒之后偶然做到的,念头起来时念口诀,属于初级阶段,等到觉察力很强了,就不用背了,直接觉察消灭。就像乘法口诀,刚开始还需要背,等到很熟练了,直接就能写出答案。这位戒友通过之前的练习,已经具备了一定的觉察力,这次实战时偶然用了出来——向念头瞪了一眼!念头居然消失了!这就是觉察的力量,只是当时他还不知道,听了《戒为良药》才明白,所以一定要在实战后回到断念的文章,有了一定的实战体会再看断念的文章就能突然明白,突然开窍!突然就懂了!元音老人的话:“\textit{看见念头,念头就没有了。}”元音老人说观照是正行,能观照就能消灭念头。邪念就像贼一样,狠狠瞪它一眼,贼就消失了!觉察力就是权杖,觉察力够强,就能真正主宰内心!你就有主宰权!实战时要狠一点,特别是决心和勇气要强烈,就像上阵杀敌一样,如一人与万人敌,杀出一条血路!李嗣业是唐朝名将,那句“挡嗣业刀者,人马俱碎!”,很是霸气!在千钧一发之际,李嗣业和他的陌刀队又一次挽狂澜于既倒,率领陌刀队“如墙前进”,所向披靡,一手扶起了摇摇欲坠的大唐江山。我们断念也需要这股狠劲和霸气!要有肃杀之气!威严之气!强悍之气!彻底粉碎心魔的进攻!太极拳有一句话:“懂劲后,愈练愈精。”你真正懂了原理,就能越练越精,越战越强!在实战中立于不败之地!学会了觉察就是这么强大,如果能从思考者转变成观察者,那就会变得更加强大!你悟到了,实战水平就会有一个飞跃!你不是念头,你是纯粹的觉知,你可以用觉察力消灭念头,从而真正主宰自己!邪念会自动冒出来,必须立刻断除,否则就会破戒。前段时间看到有戒友说,只要看破女色就不用断念了,他以为看破女色,念头就不会起来了,他想得太简单了,念头图像会自动冒出来的,戒色必须要断念的,就算你天天念佛持咒,念头还是会冒出来,最终是要看实战的!实战强!才能真正做到不破!偏离实战,必破!看一千篇戒色文章,不练习断念,必破!念几百万的佛号,不用佛号断念,必破!断念是实战的根本,断念差,什么方法什么体系都救不了你;断念强,你就能成为真正的赢家,真正的主宰!没有人一开始就是戒色高手,高手之路是崎岖坎坷的,一定会经历多次失败,然后不断反省,不断总结,不断学习,不断坚持,即使经历了多次失败,还是不气馁,还在不懈地坚持,突然有一天,顿悟来了,量变产生质变,实战水平有了很大的飞跃,开始反转心魔控制了,到时戒几百天乃至几年,都是可以做到的,所以贵在坚持!保证日课不中断,失败也不气馁,继续坚持,完成量的积累,终究会迎来质的飞跃!
\end{case}

下面分享一首诗歌。

\begin{poem}[我觉故我在]
    \begin{multicols}{3}
        \begin{center}~\\
            过去我认同思维 \\ 跟着念头跑 \\ 跑了那么多年 \\ 真的跑得太累了 \\ 晚上睡了一觉 \\ 休息好了 \\ 白天继续跟着念头跑 \\ 我发现大家都是如此 \\ 人人都在头脑里 \\ 进行着一场马拉松 \\ 认同念头 \\ 跟着念头跑 \\ 这是一场旷日持久的马拉松 \\ 一跑就是十几年、几十年 \\ 当我认同思维 \\ 我感觉我就是念头 \\ 念头就是我 \\ 当我遇见了善知识 \\ 知道了纯粹的觉知才是真正的我 \\ 我知道我终于可以歇一歇了 \\ 跑了那么久,真的太累了 \\ 歇在纯粹的觉知里 \\ 多么美好,多么安逸 \\ 我觉故我在! \\ 我可以觉察念头 \\ 我可以主宰内心 \\ 我是纯粹的觉知 \\ 我不是念头 \\ 真正的自由 \\ 就是安住于纯粹的觉知 \\ 活出本然的真我
        \end{center}
    \end{multicols}
\end{poem}
