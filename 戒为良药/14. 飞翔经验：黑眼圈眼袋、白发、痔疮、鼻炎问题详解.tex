\subsection{黑眼圈眼袋、白发、痔疮、鼻炎问题详解}

\paragraph*{前言}

最近在戒色吧浏览帖子,发现一个好现象,很多戒友的觉悟有了明显的提升,从回答问题就可以看出来。其实这些戒友通过学习就是在“长觉悟”,在围棋界把棋艺的进步叫“长棋”,在戒色方面就叫“长觉悟”,觉悟上去了,定力自然就上去了,通过不断学习戒色文章和养生知识,境界就会明显高于戒色新人,这样坚持学习下去,不断开悟,离彻底戒色就不会遥远了。

戒 SY 修的就是定力等级,定力等级达到了,自然而然就会戒掉,就怕不学习和放松警惕,那样永远戒不掉。

新人最大的问题就是存在很多思想误区,任何的怀疑和犹豫不定必须要克服掉,通过提问来消除心中的疑虑,这样戒色的信心才会坚固,否则心中还有疑问和困惑,这样是不利于戒色的。比如很多新人出现戒断反应时就想到了禁欲有害,认为会憋出毛病,如果在这个问题上认识不清,那就很容易出现破戒,其他思想误区还有很多,这些思想误区必须得到纠正,要戒色必须先改造思想,否则根本就无法戒色成功。

下面进入正题。

\subsubsection{黑眼圈眼袋}

黑眼圈和眼袋这两个问题我都有过,眼袋问题是十九岁开始的,黑眼圈在初中时就有过,因为我热爱运动,所以黑眼圈问题并不严重,根据我的仔细观察和体验,热爱运动的人出现黑眼圈的情况比较少,即使出现,只要积极锻炼和注意休息,黑眼圈很快能下去,就怕不爱动又熬夜久坐,这样的人黑眼圈就比较顽固了。我推荐大家的运动方式最好是阳光下的有氧运动,中医认为,晒太阳可以温煦体内的阳气,是养生必不可少的手段。晒太阳能够帮助人体获得维生素 D,可以提高人体的免疫力,预防各种疾病。

晒太阳也是有技巧的,在红日初升的清晨,适宜把两个手掌心(劳宫穴)对着太阳,做深呼吸,这样可以养心、肺之阳。在红日当空的午时,尤其是冬天的午时,是晒太阳最宝贵的时间,适宜把帽子脱掉,让阳气从头顶(百会穴)吸收进去,这样可以养心脑之阳,然后低下头,让阳光从颈后(风池穴)吸收进来,风池穴是人体卫外阳气的源头。

当然,晒太阳也要注意一个量,也不能晒得太多。一天一小时左右即可,不要整天猫着腰在电脑前久坐,这样对身体恢复是很不利的。要多接触大自然,多晒太阳,多运动。

在中医来讲,黑眼圈多因肾气虚损、精气不足、脉络失畅、目失所养所致。所以黑眼圈要恢复,一定要注意养生,积极锻炼,这样黑眼圈恢复才比较快。

记得高考后那个暑假我 SY 比较频繁,暑假过后照镜子就发现,有眼袋了!一有眼袋,人就显老几岁,感觉没精神很萎靡。那时的我虽然没有多少养生知识,但我知道运动可以帮助我身体恢复,于是大学时我经常慢跑,一般每次跑二十圈左右,跑得很慢。这样跑了二十天,再照镜子:眼袋没了,皮肤也好了很多,气色明显好了。但是,那时候我定力尚浅,还处在强戒盲戒阶段,于是没多久又开始 SY,一 SY 再照镜子,眼袋又回来了,所以那几年我比较苦恼,和眼袋斗争了很久,但眼袋问题一直解决不了,因为我还一直在 SY,本来好不容易通过运动和休息把眼袋消下去了,但一放纵,眼袋就又出来了,不仅眼袋出来,脸部气色也下降不少,就像过期的水果,那种枯萎颓败的感觉。我那时有了解过祛眼袋手术,有外切口和内切口两种,同时我也了解到,祛眼袋手术并不能帮你彻底祛除眼袋,还是有可能会复发的,如果你继续 SY,眼袋还是会出来的。

眼袋的出现多是因为胃燥化水功能出现衰退,胃机能差,承泣穴、四白穴阻塞造成的。而中医认为:肾为胃关!\textit{肾者,胃之关也,关门不利,故聚水而从其类也。上下溢于皮肤,故为浮肿。浮肿者,聚水而生病也。(《素问》)} 肾不仅为胃关,肾还藏五脏六腑精华之气,SY 导致肾虚,肾一虚,全身都容易出问题,眼袋问题仅仅是肾虚的一个表现而已,虽然眼袋是在胃经上,位于承泣穴的位置,但是根本原因还是肾的问题,肾一虚,五脏功能就紊乱了。

现在我彻底戒掉 SY 和 YY 后,眼袋问题已经远离我了,虽然不是非常平,但也属于和年龄相符的状态。眼袋一旦出现,要恢复的前提就是彻底戒色,否则真的很难恢复。

\subsubsection{白发}

下面谈一下白发问题。

我也有过白发困扰,但我不是少白头,青春期我 SY 频繁,但那时我作息规律,营养不错,白发有,但是很少。后来我又 SY 又熬夜,而且饮食不规律,有时一天只吃一顿,这样没多久,我白发就冒出来了很多,左边和右边各几十根,不仅出现白发还开始大量脱发,每天一百根以上,那时的我真的快崩溃了,我知道这就是报应,自作自受,怪不了别人,只能怪自己放纵自己。

导致白发的因素:

\begin{itemize}
    \item 精神因素精神紧张、忧愁伤感、焦虑不安、恐慌惊吓等都是造成少白头的原因。现代医学认为,不良的精神因素,会造成供应毛发营养的血管发生痉挛,使毛囊、毛球部的色素细胞分泌黑色素的功能发生障碍,影响黑色素颗粒的形成和运送。
    \item 营养失调实验证明,黑鼠如果一直进食缺乏叶酸、泛酸、维生素等的食物,鼠毛便会变成灰白色。另外,头发色素颗粒的颜色,往往和它含的金属有关。黑头发中的色素颗粒含有铜、钴、铁等元素,假如缺少这些元素,往往出现白发。此外。缺少蛋白质、严重营养不良等,也可长白发。
    \item 患慢性疾病一些人患有植物神经功能失调、甲状腺功能亢进、肺结核、伤寒、内分泌障碍等,也会出现白发。这是因为疾病破坏或干扰了毛囊、毛球色素细胞的生长发育,使它失去分泌黑色素的能力,阻碍黑色素颗粒的形成。
    \item 遗传因素少年白发也有一定的先天因素,在父母或家族血统中有类似的情况发生。
    \item 生活恶习如熬夜、久坐、纵欲,这些习惯都伤肾气,中医:发为肾之华。有这些恶习的人更容易出现白发。
\end{itemize}

我现在的头发已经恢复,戒色半年后,我能看到发根长出来的是黑色的头发。我非常注重养生,这也是我恢复的有利因素,很多人只戒不养,要恢复相对较难。黑色食物我也吃过,但并没有刻意去吃,黑芝麻吃过,黑豆吃过,香菇也常吃。我戒色比较彻底,又注意养生,这才是我能恢复的根本原因。在精神方面,我没给自己太多压力,情绪管理我做得很好,基本很少有不良情绪,不生气,不抱怨,不嫉妒,以一种心平气和的心态度过每一天。很多戒友都会急躁易怒,这很正常,因为肾气不足,情绪上的表现就容易急躁易怒,所以一定要注意情绪管理,让自己的情绪保持稳定,不良情绪就像地震一样,对身心是有很大的负面影响的,危害实在不容小觑,对身体恢复很不利。记得以前我隔壁邻居家的男孩,比我大几岁,他就是少白头,他没遗传基因,家里吃得也很好,但他家庭不太和睦,父母经常吵架,搞得他压力非常大,经常处在家庭暴力的阴影中,这其实就是导致白发的精神因素,家庭不和睦,心理压力过大。

白发要恢复,道理其实很简单,但做起来并不简单,就是反其道行之!杜绝那些导致白发的因素,彻底戒色,学会养生之道,这样白发问题就能慢慢恢复,有一个过程,你必须在养生方面懂得更多,更深刻,这样恢复的可能性才比较大,否则光戒不养,远远不够,具体的养生方法我在 \ref{12} 有专门提到,大家可以看看。

\subsubsection{痔疮}

下面再来谈下痔疮问题。

痔疮问题也是很普遍的问题,俗话说,十男九痔!还有一句话是这样说的:十女十痔。女人得痔疮的比例也非常高,因为女性有特殊生理期,更容易气血双亏,再一久坐,痔疮问题就会找上门来了。我得痔疮是在二十岁左右,当时我频繁 SY,然后有一段时间我坐沙发上,沙发是下陷的,这样坐了一段时间,就感觉下面有东西了,从那时起我就得痔疮了,去医院检查是内痔,配的药涂了就能消下去,但不能去根,痔疮也给我生活带来了很大的烦恼,冷辣不能吃,所以那时我在吃的方面很小心,一吃辣就会发作,大便容易下血。后来我学中医才知道,SY 会导致气虚,你气足的时候坐沙发没事,因为气推血行,气足坐沙发不会引起气血瘀滞,但是当你一旦气虚了,再坐下陷的沙发,那就容易导致气血瘀滞了,痔疮就容易得了。

痔疮的解剖学原因:人在站立或坐位时,肛门直肠位于下部。由于重力和脏器的压迫,静脉向上回流颇受障碍。

我现在痔疮已经彻底好了,我是无心插柳柳成荫,我痔疮好的过程有点奇特,本来我以为除了做手术,痔疮是不会好的。当然做手术后也是可能复发的,因为从医学角度来说,痔疮是由于静脉血液回流不畅所致,所以,手术只能把你现有的痔疮给解决掉,但是保证不了你血液回流不畅的问题能恢复。

我记得我两年前开始中药泡脚,用的是艾叶,淘宝上买的,当时我看了单桂敏的博客,里面提到了根治痔疮的方法,就是艾灸痔疮。方法是坐在一个挖空的凳子上,下面用艾灸熏烤痔疮,这样大概一周左右,痔核就会脱落,痔疮就痊愈了,当然这个过程有点痛苦,不是每个人都能接受的。我看了以后就没敢尝试,当时我就艾叶泡脚,没想到连续艾叶泡脚三天,我就感觉痔疮外翻了,变大了,我就知道是艾叶的温经通络在起作用,当痔疮外翻时,我的确很痛苦,坐都坐不下去,一坐就痛,只能歪着坐。那时我已经开始艾灸,当然没敢直接艾灸痔疮,那时我艾灸神阙穴,也就是肚脐眼,泡脚三天痔疮外翻,第四天艾灸肚脐眼,艾灸了四十分钟,我就感觉到痔疮要破了,我马上一提肛,痔疮果然破了,痔核掉了,从那以后我痔疮就彻底好了,再没犯过。

有痔疮困扰的戒友可以试试我这个方法,当然,治疗痔疮的方法很多,我只是把我的经验拿出来和大家分享,希望能帮到大家。

\subsubsection{鼻炎}

最后再来谈下鼻炎的问题。

我是过敏性体质,是一个过敏性鼻炎患者,有时打喷嚏可以连续打十几个,我记得从小时候开始,我鼻子就不好,那时虽然没 SY,但是因为先天不足,先天就容易外感风寒,所以我小时候经常感冒,还得了哮喘。但是,那时候我鼻炎没那么严重,进入青春期以后,开始频繁 SY,鼻炎就突然加重了,一年鼻子也很少有通气的时候,很痛苦。用了很多治疗方法都不见效,后来去做了下鼻甲切除手术,才缓解了不少,但还是不行,一到季节转换就容易出问题。

那时候我处在无知的状态,根本不知道是 SY 加重了我的鼻炎,现在学习中医医理后就知道了,中医:精虚鼻渊。精气一虚,就容易导致鼻炎,因为肾主纳气,肺主出气。先天不足的人再一 SY,鼻炎就会加重,因为先天不足代表的意思就是伤不起了,每个人体质不同,有人先天禀赋深厚,不伤到一定程度是不会出症状的,像我就是先天不足,稍微一伤,就出症状了。记得有位戒友说过,原来体质超好,冬天赤膊都不觉得冷,后来频繁 SY 伤了肾阳后,夏天都手脚冰凉,就是肾阳虚了。你一年两年 SY 身体没问题,但只要继续 SY,伤到一定程度肯定会出症状,有的人热爱运动,作息规律,症状出来就轻微,但是无论身体如何强壮,到了四十岁以后身体都会走下坡路,到时候很多潜在的疾病,也就是隐疾,就会集中爆发出来,就是年轻时放纵太过导致的。年轻时阳气足不觉得,年纪一上去,很多毛病就显现出来了。

我现在依然有鼻炎,但相对以前要好很多了,以前几乎每天都难受,现在很少有难受的时候了,而且现在我学会了艾灸,鼻子不舒服时艾灸合谷穴和足三里,很快就能恢复通畅。其实因为 SY 导致鼻炎加重的人有很多,我认识的朋友当中就有好几个,我表弟以前鼻子没问题,后来也学会了 SY,没几年,也得上了鼻炎。SY 导致肾气亏损,身体的抵抗力就会大大下降,内虚后,外邪就容易乘虚而入,这样很多疾病的发病率就提高了。所以,一定要戒掉 SY,养足肾气。否则真的会百病丛生!切记。
