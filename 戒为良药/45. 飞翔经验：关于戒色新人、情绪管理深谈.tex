\subsection{关于戒色新人、情绪管理深谈}

\paragraph*{前言}

这季前言继续分享几个戒友的树疗反馈:

\begin{case}[树疗反馈]
    呵呵,飞翔哥,树疗真的实在是太神奇了!我大概直坚持一个星期,感觉我的脸明显的亮了很多,我朋友都说我现在看起来像孩子,呵呵,谢谢你啊!太开心了!要是我早几年接触到戒色吧就好了。哎!遗憾。

    \textbf{附评} 这个戒友已经体会到了树疗的神奇效果了,明亮的改变是显而易见的,提亮效果明显,一亮,整个人的精神面貌就会为之一振。其实还有其他的感觉,比如变干净了,变细腻了,脸庞变饱满了,是一系列的微妙好变化。
\end{case}

\begin{case}[树疗反馈]
    飞翔大哥,今天我又在学校的校园里抱了一会柳树,这次的感觉也挺明显的,我不断换着脸颊贴在上面,因为人来人往太多,每次坚持的时间也就是十几秒钟,一共做了半个多小时。这次的做完后,双手手掌确实感觉明显加厚了很多,双手握拳的时候手心会有一种肉乎乎的感觉,脸上也会有些感觉,就是觉得脸部肌肉轻松了不少。每当有人过来的时候,我就背靠着树木玩手机,别人走了我再继续练习,我觉得还是可以坚持去做的。飞翔哥给我力量,我会坚持的,呵呵。

    \textbf{附评} 这个反馈案例是戒友元真的。他感觉到手掌变厚了,就是树疗后的一个感觉,这种变厚的感觉在树疗时就会感觉到,但最明显的感觉还是在树疗后的几个小时,特别是在晚上还有洗手之后,能感觉到手掌变厚很多,比刚做完树疗还要厚一些,大家细心体会就能感觉到的。
\end{case}

\begin{case}[树疗反馈]
    你说的那个树疗方法,我去试了,我学校有很多树,我选择了一棵大叶榕,我用手放在树干上,过了几分钟,我一握拳头,力量居然大了很多,然后握了左手的拳头,左手没摸,力量明显不如右手,换左手去摸,过了几分钟,左手居然也变得很大力,我想说,大自然的力量真是无穷的。

    \textbf{附评} 这个戒友很有意思,他这个感觉很细腻。其实就是补到了木气,中医:肝在力为握。而肝恰恰就是对应木,为肝木。所以手掌摸树一段时间,是可以增加握力的,这是很微妙的感受。
\end{case}

\subparagraph*{总结} 做树疗对改善容貌气色是很有效果的,不少戒友都有反馈了,希望更多的人能去尝试体验。找干净的树,选择人少安静处,这样有利于练习。平时也要注意养生之道,遗精后、熬夜久坐后、劳累后、久视电脑后,皮肤气色都可能会出现下滑的,所以树疗应该经常去做做,就像去充电一样。树疗对容貌气色的恢复是具有神奇效果的,这种改变微妙且显而易见。当你真正用脸去亲近树木了,坚持一段时间,恢复的奇迹就会出现。另外,尽量不要太早或者晚上去做树疗,因为阴气很重,最好是等太阳出来,在树的阳面做树疗,也就是树干被太阳光直射的那一面。

下面分享两个答疑案例。

\begin{case}
    飞翔哥,求救!我手淫史有六七年了,去年参加工作前感觉还能勃起,参加工作后突然不能正常做爱了,是不是我真的有病啊?我担心是阳痿!

    \textbf{答} 透支到一定程度,恶果就会显现出来,有一个量变到质变的过程。工作后,可能你锻炼也少了,或者总是久坐,久坐伤肾,容易伤性功能。我建议你坚持戒色养生,积极锻炼,这样性功能是会慢慢改善的。另外建议避免婚前性,婚前属于邪淫,加油!

    \textbf{分析} 撸管的恶果有积累和滞后性,就像水要煮到 100 \unit{\degreeCelsius} 才会沸腾,当恶果积累到一定程度就会显现出来,过了临界点,你就要受苦了,症状一出来各种烦恼就来了。有的戒友去年还好好的,今年身体一下垮掉了,他这样垮掉不是一年导致的,是透支了几年乃至十年以上的恶果显现,量变产生质变。撸管的恶果显现得还是很快的,即使有的人体质好,热爱运动,在年轻时症状表现不明显,但是过了四十岁,各种症状就会变得很显现,所谓年轻时放纵,中年后就开始买单了。到了四十岁以后,恢复速度也会减慢很多,恢复难度更大。所以,戒色要趁早,早觉悟早戒掉,不要和心魔妥协,一定要清除大脑里的各种思想误区,建立戒色的正知正见。
\end{case}

\begin{case}
    飞翔老师,我这几次破戒都是不到一分钟就射了,有时甚至十几秒就射了,我想问一下您我算不算早泄,另外,我看见贴吧一个帖子,里面说 SY 和 ML 不同,他以亲身经历,说他 SY 也就十几秒就射,但是 ML 时间完全正常,说那些 ML 秒射的都是心理问题,我真的很困惑,难道 SY 秒射不等于 ML 秒射?
    \subparagraph{答} 这个问题在我以前文章讲到过的,SY 早泄和 ML 早泄是有一定区别,比如有人 SY 时追求强刺激,追求快射,那么他就可能只是 SY 早泄,而 ML 正常。但有的人 SY 早泄,ML 早泄则更严重,如果你 SY 时并未追求快射,只是稍微一刺激就射,那么 ML 也很可能会早泄的,而且 ML 时会容易紧张。ML 早泄不都是心理问题,中医有讲到肾亏会出现早泄和阳痿的,比如有的人刚开始不早泄,但是放纵到一定程度,他就出现早泄了。这种情况在已婚戒友中很常见,比如刚结婚几年还行,放纵几年后就出现早泄了,并不是心理问题。当然心理问题是可以导致早泄的,比如有的人很紧张,心理压力很大,这样也是会出现早泄的。你原来身体好时可以撸管多久?有人十分钟,有人半小时都可以。但放纵到一定时候,撸管时间就变短了,JJ 也变得很敏感,十几秒甚至一碰就会射,就是肾气不足了。关于 ML 早泄和 SY 早泄是否等同,那要看具体情况,有人是 SY 早泄,ML 早泄更严重,有人其实体质还行,只是撸管时追求强刺激和快射。加油!
    \subparagraph{分析} 当伤精到一定程度,勃起不坚、早泄和阳痿肯定会出现的,撸管伤性功能,这一点毫无疑问,无数的案例已经充分证实了这一点。你现在撸得起劲,其实是在透支以后的性功能,如果靠吃药勃起,那是在调用你最后一点肾气,将来身体垮掉得更快。担心性功能不行是戒友普遍的心理,所以我们应该好好坚持戒色养生,避免未婚先废,这一点很重要,婚前放纵太过,婚后就很尴尬了,因此离婚的人也有很多。
\end{case}

下面步入正题,这季讲讲关于戒色新人、情绪管理深谈。详细论述如下。

\subsubsection{戒色新人}

新人来到戒色吧,来一百个新人,基本是一百个都被适度无害论洗过脑。而且从小就生活在适度无害论的“熏陶”中,适度无害论已经在它脑子里根深蒂固了,所以很多新人来到戒色吧,他们的发帖还带着很多适度无害论的影子。要让新人一下转变思想,有难度,只能建议他们多学习戒色文章,不断纠正自己的认识和理解。否则思想转变不过来,那就无法彻底戒掉了。甚至有的老戒友还在想着适度无害,还没爬出那个适度无害的思想陷阱,直到现在还没戒掉手淫。

戒色第一步,就是改造思想意识,就像你去坐牢给你改造思想一样,有的人改造过后就变好了,有的人改造过后还是死性不改,出来继续犯罪。新人来到戒色吧,面对的就是这样一个问题,因为新人有一个特点,那就是满脑子的思想误区,看提问的问题就知道是不是新人,我看戒友发言的内容就知道他是否能戒掉,就知道他处在哪个戒色层次。

很多戒友连意淫是暗漏这一基本常识都不知道,还在想着意淫很美好,这让资深戒友看到,可能会直摇头。戒不掉,其实就是思想上存在误区,认识上存在不足。新人来到戒色吧,如果你真想彻底戒掉,那就应该好好静下心来学习戒色文章,一点点完善自己的戒色知识,一点点提升自己的戒色觉悟。否则还在想着适度无害,那就是戒一万年也戒不掉的,所谓适度根本就是自欺欺人,适度无害不知害惨了多少人,都以为自己能适度,但手淫的高度成瘾性你考虑过吗?意淫是暗漏你是否知道呢?意淫后容易出症状,这一常识你是否真的了解?

思想有误区的新人,只要他肯坚持学习戒色文章,思想就容易改造过来。学习戒色文章就是一个改造思想、提高觉悟的过程,如果你不好好改造自己的思想,还是过去那套适度无害论,那就别想戒掉了。新人最缺的就是学习!新人最需要的也是学习!否则你一脑子的思想误区,甚至还不听劝、不肯学,那任何前辈都是没办法的,只能让症状告诉你事实真相了。我想肯定有人一辈子都爬不出适度无害的思想陷阱了,一掉进去,就一辈子在里面了,这样的人实属可悲!他被一种错误的思想俘获了,成了这种思想的傀儡,但他并未觉得自己的想法有任何不对的地方,还很自以为是,这才是他最可悲之处。

戒色吧很多新人一开始发心就不对,根本不想彻底戒掉,还在和心魔妥协,还在想着搞适度,动机不对,注定只有失败。适度无害论不从大脑里清除掉,等待你的只有戒色失败。不认清适度无害论的真面目,那根本无法戒色成功。我们戒色一定要彻底,包括意淫也一样要戒掉,彻底戒色就是我飞翔选择的道路,我是一个彻底者,绝不是一个妥协者。希望新人们能早日认清适度无害论,不要将来身体垮掉了,再来埋怨适度无害论,到时埋怨也无任何意义了。多看看吧里前辈的真实案例吧,被适度无害论毁掉的人有多少?无法计算。

知道有的戒色平台为什么不允许新人发帖吗?因为新人满脑子思想误区,一上来不学习,马上就强戒,戒色思路完全错误,还在想着靠毅力戒掉,结果就是不断失败。新人的思想误区五花八门,被无害论洗脑严重,一时难以纠正过来,必须通过大量学习戒色文章来纠正。对戒色没有正确的理解和认识,那是无法戒除成功的。比如有的戒友喜欢说“食色性也”,抓住这句话不放,还以为自己很对,你知道一个“食色性也”,那我问你,你是否知道“吃完了要付账”,天下没有免费的午餐,别以为手淫像吃饭,然后就放纵自己,为放纵找到了借口,其实是大错特错,这样理解“食色性也”就会步入歧途,古人说食色性也,更多的是传宗接代的需求,是婚后的正淫,而不是叫你肆无忌惮地放纵,古人是坚决反对婚前性行为的!

另外,请新人不要把戒色想得太简单,我指的是彻底戒色,包括戒掉意淫,不是戒戒破破。想得过于简单,只会产生轻敌的思想,其实出现轻敌思想的戒友相当多,戒到后来甚至骄傲起来,所谓骄兵必败。当然,我们也不应该把戒色想得太难,只要我们肯脚踏实地通过不断学习提高觉悟,觉悟提升到一定高度,即可降伏心魔。有一些戒友可能通过学习获得了一定的觉悟,比之新人是高出许多,但是他的觉悟还没达到完美降伏,认识上还有缺陷,这类戒友还是会破戒的。大家戒色应该保持谦虚谨慎,避免骄傲自满,觉悟越高就要越警惕,而不是觉悟高了就开始骄傲自满,那样心魔就会钻你的空子,太过自信就会变成盲目自负,最终只会导致悲剧的结果。

前车之鉴后事之师,希望新人多吸取前辈的错误教训,不要重蹈前辈的人生悲剧了。这样和你说吧,几乎所有前辈当初都和你一个想法,就是想着适度无害而最终撸废掉的,现在的你,就是过去盲目无知的我,当你开始学习戒色文章,不断提高觉悟,你就会转变思想意识,从而变成现在的我。前辈苦口婆心的劝导,就是希望新人能早日觉醒,步入戒色正道,戒色吧是一方净土,希望大家好好珍惜自己的福报,请自觉远离适度无害论,切记!

\subsubsection{情绪管理}

下面谈下情绪管理。先上一个答疑案例。

\begin{case}
    飞翔老师,你说的戒色知识我每天都在学,可是破戒照旧,我都觉得不好意思了,我那天破戒前总觉得很压抑。破戒没有白破,我把自己房间整理了一遍又一遍,以前房间太乱了像个狗窝,现在有空就帮妈妈做家务,以前总在学戒色知识,懒懒地不愿做家务,现在房间干干净净的,有空就听听轻音乐,时时微笑。我想戒色知识只是治标不治本,本还是心态,心静才是戒色王道。

    \textbf{分析} 这是戒友 lefty 的留言,lefty 是资深戒友,已经具备了很高的觉悟,但请记住一句话:如果一个人还会破戒,那么他认识上肯定还存在不足,好比百密一疏。我们要做的就是通过破戒看到这个“不足之处”,然后补强自己的觉悟。戒色要成功,必须不断提高自己的觉悟,让觉悟接近完美的状态。什么是觉悟?所谓觉悟,就是对戒色有正确的理解和认识。认识越深刻越到位,戒掉的可能性就越大,否则满脑子的思想误区或者认识盲区,那肯定还会破戒的。

    lefty 的不足之处,就在于情绪管理的失败,我看过无数的戒友案例,败在情绪管理上的人极多,生活中总有矛盾和不顺,比如和父母吵架,比如和同事朋友关系紧张,比如做事不顺利,遇见很大的困扰,诸如此类都会影响到一个人的情绪,当负面情绪产生时,一定要学会管理情绪。就像钟表走快或者走慢了,一定要把它调整到正确时间,这个情绪调整的能力对于戒色是异常关键的,情绪是可以导致破戒的,特别是不良情绪,当然狂欢情绪也可导致破戒,所以最好能保持稳定的情绪状态。

    lefty 把戒色知识和心态调整分开,其实也是认识上存在误区,因为戒色知识就包括对心态的调整,我在以前文章也多次强调情绪管理的重要性,对于不良念头要马上干预,否则这些负面念头会喧宾夺主,进而控制你的行为而导致破戒的发生。不过 lefty 通过破戒,已经认识到了情绪管理的重要性,也就是心态调整的重要性,希望他能越戒越好。过了情绪关,戒色就会变得很稳定。
\end{case}

大家无须把觉悟和心态或者习惯等分开,因为戒色觉悟就包括养成良好的习惯,比如不熬夜不赖床等,觉悟也包括调整自己的心态。觉悟是一个总词,戒色成功的理论都可以涵盖在这个词里面,包括警惕意识也可以包含在内,\textbf{戒色成功 = 觉悟高 + 警惕强}。我之所以把警惕分开来,其实就是为了强调警惕意识的重要性,其实警惕意识也属于觉悟范畴。

意淫关,是戒友间的一道分水岭,其实情绪关也是一道分水岭,死在情绪关的戒友太多太多,过了情绪关,就会进入新的戒色层次,否则过不了情绪关,破戒的可能性还是很大的。有些戒友可能要破戒很多次才能意识到情绪管理的重要性,心态调整和情绪管理其实是一个意思,我写的经验帖都是我深刻体会思考的领悟,很多戒友因为体验不深刻,所以看我文章可能懵懵懂懂,当看到“情绪管理”这个四个字时,他的印象并不深刻,当被心魔虐过无数次后,他可能才突然意识到原来情绪管理这么重要,原来情绪真的可以导致破戒,这时候他才知道我写情绪管理的分量。

我们戒色一定要学会情绪管理,有的戒友可能会问,怎么管理情绪呢?如果详细说情绪管理,可以写一本书,但一本书的内容可以概括为四个字,那就是:心理暗示。比如你心里很烦躁,这时候你可以暗示自己保持心平气和,不要多想,也可以想一些让自己开心愉快的事情。我那时得神经症,每天活在压抑绝望的情绪中,甚至曾经出现过自杀的想法,后来我学会了情绪管理,特别是一有压抑的感觉,马上就想些让人轻松愉快的喜剧片段,然后心情就调整过来了,现在我也是这样调整自己情绪的。心理暗示有多种方式和技巧,你也可以观想无我,无我何来情绪何来烦恼,把自己化空,自然就可以平息不良情绪的烦恼,也可以这样想,用中医养生的理论来开导自己,因为怒伤肝,烦劳则阳气张,七情致病,不良情绪是可以导致疾病的,所以我们要学会保持心平气和,尽量符合养生之道。我们不应该压抑不良情绪,强压是会损害身体健康的,我们可以转化情绪或者把情绪化空,这样就可以平安度过情绪关。情绪管理还包括激励自己,让自己产生向上奋进的情绪,激励自己可以有效克服戒色懈怠情绪,让自己保持在勇猛精进的状态。

另外,我要强调一点的是,尽量不要大笑,因为大喜伤心。我看过有的戒友说要每天大笑,那样也是不符合养生之道的,你可以会心一笑,有节制地笑,不需要大笑。大笑会乐极生悲,是容易导致猝死的,我身边就有两个真实的案例,一个是小区棋牌室,一个老头打麻将,抓到一张牌,构成了一手绝世好牌,然后他哈哈大笑,突然就猝死当场。还有一个案例更惨,一个 25 岁刚结婚的年轻人,吃完饭后和老婆打闹嬉戏,处在大笑的状态,也是突然猝死,救护车来了也回天无力了。这两个真实案例就是在提醒我们,尽量不要大笑,要有节制地笑,以免乐极生悲。

有中医常识的戒友应该知道,纵欲的人是很可能出现心脏问题的,因为肾是五脏之根,肾虚到一定程度,就可能出现早搏心悸乃至心脏神经官能症等表现。有一些戒友来到戒色吧,很可能会有这样一种想法:怎么什么症状都往手淫上套啊!其实如果他深入了解或者他深有体会,那就不会这样认为了,中医:精者,身体之本也。肾乃五脏之根,精少则病,肾虚百病丛生,这些都是中医的真理,经过几千年的验证,所以大家把症状归咎于手淫,并没有错。如果你认识有局限,或者体验不深刻,那就很可能会产生误解。当然致病因素有很多,外感因素、饮食因素、情绪因素,不良的生活习惯,比如熬夜久坐久视,劳累过度等,都可造成症状的出现,还有先天体质不佳也是一个因素,症状的出现和手淫是密不可分的,不过也和其他各类致病因素有一定关系,如果是多种致病因素共同作用,那么伤精就比较严重了。伤精的方式不只是手淫一种,熬夜、暴怒、久坐久视等都会耗损肾精,所以我们一定要注重养生之道才是。

下面根据我的学习和认识,把情绪管理的具体做法和大家做个分享,我们要做好情绪管理,做情绪的主人。

\begin{description}
    \item [确认当下的情绪] 当你出现负面情绪了,比如愤怒、压抑、颓废、绝望等,你要认识到负面情绪出现了,第一步就是发现并且确认负面情绪,就像发现敌人一样。
    \item [转移不良情绪] 当你意识到你处在不良情绪中,就像被乌云笼罩时,你要学会转移不良情绪,你可以折一个纸团,想象不良情绪被包裹其中,然后扔进垃圾桶。
    \item [用好的情绪替代] 当意识到不良情绪出现时,马上想好的开心的事情,比如一次开心的体验,一次美好的回忆,或者一个笑话,让自己马上摆脱出负面情绪。
    \item [化空自己] 你可以观想无我,无我何来情绪何来烦恼,一切都是庸人自扰,化空自己,让负面情绪找不到靶子。
    \item [发泄负面情绪] 可以通过适量的锻炼来消除不良情绪,比如打会篮球,或者出去走走,或者看看电影,听听歌。
    \item [激励自己] 要肯定自己的进步,说激励自己的话,想激励自己的人物和事情,听激励的歌曲,看激励的电影,让自己斗志重新燃烧。
    \item [多做善事] 助人为乐乃快乐之本,心情不好时,你可以试着做些小善举,比如上网为戒友加油,布施一下正能量,这样也可以调整情绪。
    \item [养生意识] 意识到负面情绪可以致病,我们何必沉溺其中呢?就像停止用刀砍自己,马上停止负面情绪对自己的伤害。
    \item [保持微笑] 微笑是最美的表情之一,多暗示自己保持乐观,保持微笑,积极乐观过好每一天,没有过不去的坎,关键在于态度。
    \item [放松自己] 不要把自己绷得太紧,绷太紧会崩溃的,要学会放松自己,洗个热水澡,做做养生功法,什么也不要多想,从烦恼中抽离。
\end{description}

\paragraph*{总结}

情绪管理,意味着我们可以选择情绪,情绪就像超市的货品,我们是可以选择的,当负面情绪出现时,我们应该马上意识到,然后我们可以选择好的情绪,不要让负面的情绪控制自己。负面的情绪其实就是负面的念头和想法,对于那种念头和想法,我们也要做到念起即觉,不要被负面情绪所控制,一旦被负面情绪所控制,那就很容易破戒了。戒色说到底,就是要学会观察念头,识别念头,管理念头。我们学习戒色文章提高觉悟,最终的目的就是提高对念头的掌控能力。比如一个人一天有三百个念头,哪些念头容易导致破戒,你应该要知道,比如负面情绪的念头、意淫的念头、怂恿的念头、狂欢的念头、对戒色疑惑的念头等,总之,能导致破戒的念头还是很多的,有些念头伪装得很好,比如怂恿的念头。对于念头,我们要做一个观察者,并且识别哪些念头有问题,哪些念头容易导致破戒,一旦出现那种念头,马上要进行干预对治。打个比方,就像在路上设一道关卡,让念头一个个通过,有问题的念头则不能通过,一旦让有问题的念头通过关卡,那么它就会主导你的行为,因为念头导致行为,念头 $\to$ 行为!有问题的念头一旦通过,就会导致破戒的发生。让我们做一个念头观察者,让我们做一个念头掌控者!

这季推荐一本书:

\begin{book}[《中阴闻教得度》,莲花生大士]
    这本书我极力推荐,有缘之人得到这本书一定会有如获至宝的感觉。当然也有很多无缘之人即使遇到,也会当面错过。这本书有图文版,图文版比较直观些,我推荐的是元音老人的讲解版本,用的是孙景风的翻译。关于这本书有好几个版本,我比较推崇孙景风的这个版本,配上元音老人的讲解就很完美了。元音老人关于中阴教法的文档,百度可以搜到的,大家下载后可以打印出来详看,网上也有书买的,希望有缘之人不要错过这本珍贵的法本。
\end{book}
