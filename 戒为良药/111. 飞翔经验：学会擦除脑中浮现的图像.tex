\subsection{学会擦除脑中浮现的图像}

\paragraph*{前言}

有的人会说,为什么别人撸了好几年,看上去还很正常。很多问题是不能光看表面的,得性病的、得艾滋病的,看上去是不是很正常?有的人查出癌症了,然后他周围人都说看不出啊,不像啊!是不是这样?周围人都会感到惊讶,所以,很多事情不能光看表面。伤精患者的痛苦只有自己知道,比如脑力下降、腰痛、腿软、精力不济、尿频尿等待、头晕头痛等等,人家不告诉你,你怎么能知道?很多痛苦只有自己知道,旁人根本是看不出来的,除非是具有深厚经验的中医,精通望诊,这样有可能会从脸部的一些微妙的变化看出内在的问题。有时患病个体的主观感觉也不一定正确,看过一篇报道,那个人在体检时查出肝癌,而且还是晚期,而他自己根本想不到,旁人一点也看不出,他平时看上去好像很正常,其实内在已经有大问题了,只是他自己暂时感觉不到,有的病人在早期和中期时一点感觉都没有,还以为自己很健康,但是一查出来就是晚期。之前有个神衰患者大言不惭地说自己曾经撸了八年,几乎没什么症状,他质疑手淫的危害,我要说的是,那八年就是为后来神衰爆发埋下了伏笔!没有那八年的伤害积累,怎么会有之后神衰的爆发?冰冻三尺非一日之寒,伤害是具有累积性的,达到临界点就会爆发症状。他说撸了八年没什么症状,我也不大相信,一般二、三年就会出症状,快的话三个月不到就会得上前列腺炎。之前有戒友是专业散打队的,还有专业练拳击的,还有国家一级运动员,他们的撸龄也不长,但也症状缠身了,腿无力、人变丑、体能下降、腰酸、四肢发冷、神疲乏力、精神不振、烦躁易怒、易疲劳、多汗、记忆力减退、耳鸣、脱发白发、早泄、勃起不坚等,这些都算伤精的症状表现。别以为撸管没事,别以为自己的身体还行,其实那都是表象!你的内在已经开始衰败了,只是暂时感觉不出来。扁鹊三次劝蔡桓公,但是蔡桓公说自己没病,最后一发病就要命!

最近看到有戒友提到了恢复不稳定的问题,有段时间恢复得非常好,但过一段时间就不行了。我们戒色之后一定要加强养生恢复,多学习养生的知识提高觉悟,戒色文章里也有提到养生的内容,光戒是不行的,戒色是系统工程,恢复也是系统工程,就像一只鸟的两个翅膀一样。有的戒友虽然戒色了,但是还在熬夜久坐,或者过于劳累,又或者控遗做得不好,天气转凉时又不注意,这样恢复就会时好时坏,不是很稳定。在恢复过程中一般都会出现症状反复,自己要注意休养和调整,加强养生之道,这样恢复就会越来越稳定。上季一位戒友说他破戒了,天数是否可以继续算下去,因为天数归零让他感觉很气馁,我的回答是可以继续算下去,可以按照年度破戒次数来算,比如半年不慎破戒一到两次,或者一年不慎破戒几次。这种算法也是很不错的,之前很多戒友都是用这种算法,天数归零的确会影响戒色的信心,所以在破戒后可以考虑采用这种算法。对于天数也不必过于纠结,关键是在破戒后要好好反省和总结,争取以后戒得越来越好。上季也有戒友问:当浴火焚身时,如何立即跳出来而不破戒?当欲火中烧时,其实已经晚了,就像从小火星演变成森林大火一样,最佳的扑灭时机已经错过了。但这时也应该尽量避免手淫,建议出去走走,跑跑步,做做运动,尽量避免独处,并且猛烈思维邪淫的危害,坚决把邪念赶出脑袋。有的戒友在快破戒时,他就来戒色吧,然后冲动就渐渐平息下去了,这样也可以,但戒色贵在平时,平时就要认真系统地学习戒色文章,好好练习观心断念,不要总是等到欲火中烧才干预,必须在小火星的阶段就坚决果断地掐灭。

下面分享一些案例。

\begin{case}
    十三年了,少年变了样!不知不觉邪淫已十三年,智力从当年的全班第一到现在什么事情都记不住,从小时候的认真坚持、满怀壮志到现在的碌碌无为。我不是一个善于交流的人,小时候话不多,而长期的邪淫更导致现在的社恐,高中的时候还意气风发,能快乐地与人交流,而现在完全自闭了。重要的是容貌丑了,猥琐了,伴随着严重的脱发,我有些不敢面对社会了。希望小朋友们看到后能引以为戒,这个恶习一沾染上之后是非常难戒掉的,越早戒掉越好!

    \textbf{附评} 这位戒友原来智力全班第一,撸了十三年之后,什么事情都记不住,记忆力已经大幅下降了。脑力不行,干什么都费劲,很多原来能够干好的事情,在脑力下降后就干不好了,总是容易出错,很容易犯低级错误。有的人虽然其他伤精症状不是很明显,但是脑力下降、容貌变丑、社恐强迫等对他的影响也颇为巨大,去医院检查也基本正常,但只有自己知道身体和容貌气质已经大不如前了,一种江河日下、衰败无力的感觉已经将他整个笼罩。有的孩子小时候虽然内向,但还能和同学快乐自然地交流,但在沉迷手淫后,身心出现了严重的失调,和人交往就会出现诸多障碍,别人说的话,很难听懂,而且答非所问,脑力一下降,再一出现社恐,那生活真的是一塌糊涂。之前有个撸者去面试,都被面试官赶出来,面试官最讨厌的就是答非所问,看到一脸晦气的撸者更是心生厌烦。撸者说话往往显得不够自信,眼神闪躲,不敢直视,缺少应有的底气和正气。一个充满正气的人往面试官面前一坐,面试官就会被那股强大的气场所深深折服,面试官看到你充满正气的眼神都会心头一惊,暗自叹服,而你表达起来落落大方,潇洒自如,言谈举止充满着自信,头脑也异常清晰,回答干脆利落,最后结束时,面试官肯定在心里极为认可你,你就是他们公司要找的那类人才。公司需要的是充满正能量的人才,一个邪淫之人进了公司,都会给那个公司带来霉运。这位戒友十三年的撸龄,从少年变成了严重脱发的猥琐大叔,可以说是严重变了样,把朝气和正气都给撸没了,全身充满了邪淫的晦气和戾气,这种感觉真的是糟透了。记得十几年前我所居住的小区,里面有一个十几岁的少年,人长得很清秀,但是过了好几年再看到他,几乎认不出来了,可以说是严重变形,原因就是进入了社会,接触了烟酒色,整个人从脸到身材都完全浮肿扭曲起来,再也不复当年的清秀和帅气,简直从天堂到地狱。黄赌毒,黄排在第一个,国家也在打击网络淫秽色情,色情对于人的腐蚀是极大的,特别是对于青少年,他们还处在长身体的年龄段,疯狂手淫很可能会影响他们的生长发育。这位戒友最后的告诫很恳切,手淫就是一个害人的恶习,具有高度成瘾性,必须彻底戒掉,不要自欺欺人,不要心存侥幸,越早戒掉越好,否则等到症状爆发,那就悔之晚矣!
\end{case}

\begin{case}
    问好飞翔哥,戒了这么久,感受最深的还是要坚持学习,前几个月工作忙,很少看戒色文章,笔记也没复习,由于在稳定期心魔基本上很少出现,直到半个月前心魔突然搞袭击了,进攻得特别疯狂,全是以前我破戒的老套路,意淫、怂恿、回忆邪淫经历。我意识到不对,戒色状态不对了,当天回去后赶紧把笔记本翻出来复习,念口诀半个小时,终于止住了这种心痒痒的状态,那时候我有点蠢蠢欲动了,如果没有复习笔记念口诀的话,那么我就沦陷了,从那天开始我每天都复习笔记,看戒色文章,不敢不学习了,无论戒多久都要保持警惕,无论戒多久还是要坚持学习,学习不仅能精进,还能保持良好的戒色状态不破戒!

    \textbf{附评} 这个案例出自“我们的布丁”,他是一位老戒友,之前分享过他的案例,他从躺床废人逆袭成公司白领,是非常励志的榜样。他的这条反馈很好,很能说明问题,在进入戒色稳定期后,很多人都放松了警惕,也不看戒色文章了,但是请不要忘记,心魔正在等待你放松警惕,然后它就会大举进攻。不管戒多久,都不可放松警惕,千万不要忘记心魔的存在,就像在和平年代,国家也在不断练兵和实战演练,随时都要做好战斗的准备。很多戒友看到前辈的文章里一直在强调警惕,而他没有实战的体会,直到被心魔虐得遍体鳞伤后,他才真正明白“警惕”这两个字的分量。进入戒色稳定期后,邪念是相对比较少,有时十几天乃至几十天都没一个邪念,似乎都把戒色这件事给忘了,但魔考还是会来的,而且会疯狂地来、猛烈地来,就像暴风骤雨般,心魔会疯狂进攻,企图再次把你拿下!布丁的警惕性还是很高的,他意识到不对了,当天回家就加强了学习并且练习了断念,这样才挺过了这次破戒危机。他没有被心魔攻陷,但已经尝到了心魔疯狂反扑的厉害,他这次也算长记性了。不管生活多忙、多累,你每天都会吃饭,你每天都会刷牙和洗脸,我们戒色之后,每天也应该抽出一定的时间来学习戒色文章,切记不要中断,一旦中断,再想找到好状态就比较难。每天抽十分钟复习一下戒色笔记,或者听听戒色录音,这样并不会影响到你的工作和学习,反而是一种调剂,看看戒色文章也是一种享受,而且是高层次的享受,我们应该学会享受戒色的生活。时间宽裕的话,可以多学习一些时间,半小时或者一小时以上,可以自己灵活安排。时间管理方面就有利用零碎时间,我们一天当中有很多零碎时间都被浪费了,其实零碎时间是可以利用起来的,比如五分钟的零碎时间就可以看几十条的戒色笔记,坐公交车时可以听听戒色录音,下个读书软件来听戒色文章,这也是一种很好的学习方式。黄念祖老居士最提倡废时利用。\textit{废时利用的天地非常广阔。(黄念祖老居士)} 戒色应该是君子终生的修为,不是说戒了一两年就不看戒色文章,也不学习圣贤教育了,如果这样戒,迟早还会回到老路上去,学习必须持之以恒,这样你的悟境才会日新月异,不断向上突破。
\end{case}

\begin{case}
    谢谢飞翔老师,现在已经戒色整整二十个月啦,感觉整个人充满自信,心态乐观,做事条理分明,说话底气十足声音洪亮。我认为戒色是青年男子的头等大事,身体好、思想正才能聚集各种有利条件从而在学业、事业、生活上齐头并进!再次感谢飞翔老师!

    \textbf{附评} 能够戒色二十个月很不容易,但这位戒友做到了,在某些新人眼里,别说二十个月,就是二十天都是一个难以完成的任务。这条反馈也充分证明了戒色并不是压抑痛苦的事情,戒色是一件非常快乐的事情,在人生的某个阶段应该要好好享受戒色所带来的快乐。很多人在婚后也想戒色,因为身体垮了,严重早泄阳痿,而且有很多其他的伤精症状,苦不堪言。但婚后戒色,难度就增加很多,因为需要老婆的理解和配合,也需要自己具备更高的定力。以前一位已婚戒友戒色四年多,阳痿成功恢复,他老婆很理解他,给了他四年的时间恢复身体,四年多的坚持实属不易,但是他做到了,用坚持换来了身体的恢复。很多人伤精了十几年,真正的恢复时间在三年以上,一般戒色半年可以有所缓解,但是恢复不太稳定,需要继续坚持戒色养生,不断累积恢复的能量,最后才能达到真正的恢复。一个人可以很有钱,可以享受到各种物质,或者不停地换女友,但是这种追求物质和沉迷邪淫的生活只会让他的灵魂更空虚,无法让他体验到更高层次的快乐与充实,他被欲望压缩在一个非常低级的层次,他甚至不知道还有更高快乐的存在,他以为人生就是物质和性,其实人生有更高的层面,在更高的层面就可以体验到纯粹的大快乐,在那种状态才能发自内心地快乐,一种真正持久的满足感油然而生,戒色是多么快乐幸福的事情,\textit{离欲清净,是最为胜。(《四十二章经》)} 砖家说禁欲是压抑是痛苦的。恰恰相反,真正的禁欲是充满快乐与喜悦的,因为你超脱了欲望的束缚,你获得了更高层次的快乐。戒色是能量管理,能量养足了,你的身心感觉就会焕然一新,充满了自信与底气,脑子也灵光了,思维也敏捷了,做事也变得高效起来,人际关系也改善了。戒色真的是人生的头等大事,必须纠正关于禁欲的错误观点,真正认清戒色所带来的好处,坚定地戒下去,戒出最自信、最帅气、最富正能量的自己,在学业、事业和生活上势如破竹、势不可挡。\textit{长风破浪会有时,直挂云帆济沧海。(李白)} 戒到一定程度,一种豪迈的气势就出来了,一切都变得明朗起来,人生有了新的方向、新的动力,一个朝气蓬勃的你焕然新生。
\end{case}

\begin{case}
    是啊,我记忆深处有童年的味道,简单快乐,慢慢上了高中逐渐没有了,自从接触戒色吧,才开始戒,才逐渐有了那种非常纯净的感觉,说不出来,反正感觉很好,只有戒了才能感觉到,那种没有邪念,单纯的感觉。今天看了飞翔哥的文章,受益匪浅啊!让我明白了要有坚决的实战意识,要深刻地学习,多看文章,深刻领悟,实战时当念头上来要立马断掉,实战意识要特别坚定,头脑一定保持清醒,要知道自己心里想的是什么,看住心。细微的念头一上来立马断念,我今天又深刻地领悟了实战意识,太重要了。

    \textbf{附评} 这位戒友的反馈挺不错的,我们曾经都是纯净无比的孩子,活在纯净美好的感觉里,然而随着染上手淫恶习,那种纯净美好就渐渐消失了,内心变得越来越不快乐,也变得越来越禽兽,进入了鬼畜的状态,脑子里充满着各种龌龊的邪念,就像一个发臭的垃圾场。失去了纯净的灵魂,就是最悲剧的事情,纯净是一个人最宝贵的财富,纯净本身就是一种伟大的力量。只有降伏心魔,戒掉手淫恶习,才能拿回属于你的力量,国外戒色网站也说:“Take Back Your Power from Porn(从色情里拿回你的力量)”,邪淫剥夺了你的力量,必须戒掉邪淫,再次获取纯净的力量,一个心灵纯净的人胜过一群大力士,大力士强的是体魄,而纯净的人强的是内心!内心的强大才是最具威慑力的。能戒邪淫,是正器,沉迷邪淫,是败器!败于心魔,败于邪淫,最终人生一败涂地。我们生下来就是要战胜心魔的,这是我们的使命!必须坚决完成自己的使命!不断学习戒色文章提高觉悟,强化自己的实战意识,努力提升自己的实战表现,看住自己的心,主宰自己的心!必须严格做到念起即断!戒色要具备金刚意志,无坚不摧,战无不胜,摧枯拉朽,以必死的决心去戒!拼死戒,戒到死!往死里戒,干死心魔!必须够狠、够决!这是你死我活的战斗,杀出一条血路!!!做戒色的特种兵!做戒色的硬汉,与心魔死磕到底!与心魔决一死战!破釜沉舟,杀出生天!!!
\end{case}

\begin{case}
    戒色八十多天了,离一百天还有些日子,最近喜感简直爆棚啊!逝去的美好仿佛都回来了,戒下去吧,不抛弃不放弃,有付出就会有回报,遇见一个不一样的自己,为明天加油!邪淫曾经毁了我,为图那一时快感,我也付出了惨痛的代价,如今我只会痛恨它,继续努力学习,尽快走出阴影。

    \textbf{附评} 有句话是这样说的:“别看你现在沉迷邪淫,你迟早会痛恨邪淫,因为邪淫迟早会毁了你!”邪淫不是一种正常的状态,是一种被心魔奴役的傀儡状态,谈不上什么自由,完全就是奴役般的生活,很多人撸完后突然惊醒:“我这是做了什么啊?!”射完后感觉太没意思了,一点意思都没有,把花样年华直接撸成了屎样年华,浑身散发着邪淫的恶臭!这位戒友戒色八十多天,逝去的美好再次回来了,他再次感受到了纯净的大快乐,那种大快乐就是内心喜悦的大爆炸,在内心一层层荡漾,一层层开花,一层层洋溢,特别幸福的感觉,特别纯粹美好。这位戒友说“喜感简直爆棚啊!”,仿佛像纯净的孩子一样开心快乐,那是一种“开心极了”的感觉,没有任何理由,不需要外在任何条件,只要你的心灵足够干净,就会迎来内心喜悦的大爆炸!这是一个秘密,有深厚经验的修道者都知道这个秘密。几乎所有人都在寻找快乐的体验,但世人不知道真正的快乐在哪,他们以为快乐在外面,其实真正的快乐就在内心深处,移除那些心垢,你自然就是快乐的。“别问我为何这样快乐,好好净化你的心吧!到时你就知道了,简直妙不可言!”这是一个秘密,知道这个秘密的人真的有福了。无边的海,蓝色的天,白色的云,纯净的你,两三只海鸥飞过,你闭上双眼,享受此刻的纯净美好,一种超然的大爽震撼着你每一个细胞每一根神经,这种感觉太久违、太美妙,宛如奇迹降临。童年的味道,记忆中的美好,在戒色后就能再次体会到那种纯粹,那种让你泪流满面的纯粹。做一个纯粹的人,一个脱离撸管低级趣味的纯粹的人,用心感受纯净的美好。
\end{case}

\begin{case}
    我于高考前四个月开始戒色,一开始并不是很顺利,第一次戒了五十五天,突破了我的最长纪录,当时破戒真的是特别难受。每次破戒,我都会感觉到以前那生不如死的生活在向我招手,于是每次破戒我都会诚心忏悔并且吸取经验,防止下一次破戒。就这样我断断续续地戒了四个月,虽然不是很理想,然而跟以前每天都要邪淫的自己相比已经算是天壤之别了。加上一些运动,戒色以后真的犹如换了一个人,我的体重在这几个月增加了二十斤,以前羸弱的我变得有些壮实了,精神状态也比以前好了,以前所患的重度抑郁症也开始慢慢缓解(当时在医院测试出的就是重度抑郁),人也有正气了。我记得我当时给我朋友说我感觉终于像个人了,以前的自己像个鬼。正因为有了戒色的基础,好运终于向我招手了!我终于在六月份成功考上了一所二本院校!真的是很开心,我记得当时二本分数线出来的时候我妈抱着我哭了……从此以后,我也更加坚定地要去戒色。到今天已经有四百天,现在回想起以前的那些往事,真的是有一种恍如隔世的感觉。

    \textbf{附评} 这位戒友之前自杀未遂过,可谓是不堪回首,他写的帖子被加精了,这是他帖子的一段摘录。戒色真的拯救了他,以前的他每天都要邪淫,后来他坚持戒了四个月,虽然不是很理想,但已经取得了很大的成果。重度抑郁症的确很折磨人,很多人都因为这个病自杀了,抑郁症的中医病因归纳为:肝郁气滞、痰浊内蕴、肝郁脾虚、肾精不足、心肾不交等,手淫伤精是导致抑郁症的一个潜在的因素,抑郁症的发病和生理的失调是密切相关的,不仅仅是情志的问题,其发病是有一定生理失调的基础的。中医讲肾为肝之母,肝肾同源,你疯狂手淫耗泄肾精,自然就会影响到肝,肝气一郁结,精神就容易抑郁,情绪也会变得低落。这位戒友戒色之后犹如换了一个人,原来瘦削,后来增加了二十斤,变得壮实了,以前的他像个鬼,戒色后总算有个人样了,而且还考上了二本。经历过自杀的风波,他的家庭肯定在崩溃的边缘,如今完成如此巨大的蜕变,真的是喜极而泣,他妈妈都抱着他哭,那一刻把之前压抑的情绪都宣泄出来了。邪淫对家庭的伤害实在太大了,自己邪淫,还连累自己的父母。这位戒友现在已经戒色四百天了,到了大学当上了班长,还获得了奖学金,家庭也变得和睦了。他用到了“恍如隔世”这个词,曾经我也用过这个词,戒色后会有一种重生的感觉,生活又回到了正轨,之前在邪轨上错得太离谱,过去邪淫的自己太不懂事、太无知、太荒唐了,现在终于清醒了。古德云:“何耽刹那之微乐,应受永劫之大苦。”邪淫之微乐会导致巨大的苦报,后果实在太可怕了,真正有觉悟的人是绝对不敢犯邪淫的。
\end{case}

\begin{case}
    飞翔哥,我来戒色吧一年十个月了,曾经最长戒过 171 天和一个 102 天,我以为自己能摆脱手淫了,同时也对戒色文章有了厌倦情绪,每天都不愿意看戒色文章了,也不愿意来戒色吧看精品帖了,结果最近我就在频繁破戒。

    \textbf{附评} 很多前辈都倒在了戒色厌倦期,之前不少人还是戒得不错的,过百天乃至过一年,但是一旦进入厌倦期,疏远了戒色文章,开始厌倦戒色了,这时候就很容易破戒。有的人不仅厌倦戒色,而且还骄傲自满,这样百分百会破戒。在我很早的文章里我就讲到了要克服戒色厌倦期,因为在我自身戒色的过程中,我也经历了戒色厌倦期,人是很容易厌倦的,爱情和婚姻到了一定阶段也会厌倦,关键就是要学会克服厌倦期,我给出的方法就是要养成良好的学习习惯,学会享受戒色,从戒色中获取无穷的乐趣,把戒色作为自己的兴趣爱好来培养,不仅戒色而且还要多帮助别人,学会行善积德。学习戒色文章千万不可中断,很多人都中了“中断魔”,梦参老和尚说过:“无论修行哪一个法门,最大的障碍就是中断魔。为什么中断叫魔呢?不能相续,修行力量中断了。再来又得从头做起了,不是那个力度了,修行也如是。过了这个机会,再想重复,恐怕很难了。”你一旦中断了,就会失去那种状态和感觉了,再想重新找回那种势头就比较难了。在行为心理学中,一个人的新习惯或理念的形成并得以巩固至少需要 21 天,称之为 21 天效应。我们一定要学会坚持,学会巩固这种学习的习惯,每一天都不要中断,我戒到现在五年以上,几乎没有一天中断过学习,我每天都有定课,我很注重积累,也很明白水滴石穿的道理,一时的冲劲很多人都有,但是最关键的还是恒劲,所谓恒则成!你看那些奥运冠军,他们都是经过多年的专业训练,最后才成功的,一旦中断训练,他们的运动水平就会出现下降,他们全年有着很系统的训练计划和安排,包括春训、夏训、秋训、冬训,非常系统和专业。运动员也会出现厌倦情绪,但是通过调整就克服了。一旦你出现了厌倦情绪,要像发现毒蛇爬上你的膝盖一样,一定要马上消除厌倦情绪,自己要学会调整和克服,不能让厌倦情绪主导你。戒色吧经常有这样的帖子,说的就是“不应该离开戒色吧啊”,离开戒色吧后他就破戒了。厌倦戒色,不再看戒色文章,也不帮助新人了,最后又回到了破戒的怪圈中。这位戒友之前的两次战绩还是不错的:171 天和 102 天。但是后来他掉以轻心了,并且不愿意看戒色文章了,这样肯定会再次破戒。一定要认识到戒色是终生的持久战,学习戒色文章一定要持之以恒,必须坚决克服戒色厌倦期,必须注重积累,这样才能越戒越稳定,要学会保持自己的戒色状态,如果你能克服戒色厌倦期,你就会进入更高的戒色层次,很多人就倒在了戒色厌倦期,戒色厌倦期不可小觑,必须引起高度重视,必须及时克服。
\end{case}

\begin{case}
    27 岁一事无成,欠了一屁股债,本来工作稳定,工资也有六千左右,自从前年手淫后遗症爆发以来就没有工作过,今天给家里打电话我妈被我气哭了,其实我也好想争气,可是现在感觉力不从心了,好无助。现在一工作头上就冒虚汗,像下雨一样,每次找到工作做两三天就感觉吃不消,好后悔啊!这种无奈谁能懂?好怀念以前,亲朋好友个个都夸,现在都像躲瘟神一样。手淫无害论不知道害了多少人,真正发现厉害的时候已经无法自拔了。

    \textbf{附评} 这位戒友的工作是鞋厂组长,工资六千多,这份工资在中小城市还算可以的,但是症状爆发后,他的身体就很难胜任工作了,最后只能辞职,重新找工作也干不长,两三天就感觉身体吃不消了。身体差了,就会严重影响到工作状态,最后只能选择辞职休养,财运受到了很大的影响。之前一位戒友月薪两万多,已经做到主管了,但是身体因为长期 SY 而出现了很多不适,只能辞职休养,财运就此断了,还要不断看病花钱,踏上了寻医问药之路。因撸致病,因病返贫,这就是很多人的悲惨现状,人生陷入了很深的困境,一方面收入断了,另外还要拿钱出来看病,这样没多久就捉襟见肘了,有的撸者看了几年病,基本都成穷光蛋了,现在看病吃药也贵,几年下来,也是一笔高昂的费用,有收入还别说,就怕没收入,那样就更加难堪了,看病都看不起。这位戒友以前也风光过,亲朋友好友个个夸,最后身体垮了,一事无成,还欠下了很多债务,估计很多都是向亲朋好友借的,过去人人夸,现在大家见到他就像躲瘟神一样。这种落差实在太大了,这种糟糕的处境和邪淫是密不可分的,正是因为邪淫把身体搞垮了,身体垮了,工作也没了,财运就断了,一步错,步步错,就像引发了蝴蝶效应。一位贪官落马后的感言:“我只看到了里面的五光十色,没有看到里面的刀光剑影。”很多撸者也是如此,只看到撸管的快感而没有看到撸管所导致的潜在恶果,很多撸者完全无知,被无害论洗脑后非常盲目,脑子里全是邪知邪见,只知道撸啊撸,最后症状爆发就彻底傻眼了。很多人正值壮年,本该是三军用命挥斥方遒之际,却陡然夭折,病魔缠身。眼前的生活突然变得好无望,眼睁睁看着自己掉入一个很深的漩涡,却无能为力,无法脱身。撸到最后,每一个人都有一段不为外人所知的悲情故事,内心都有着难以言说的悲苦。有一位戒友是这样说的:“从小学六年级到如今将近十年的时间,过得迷迷糊糊,荒废了青春的大好年华,折损了命里的福报,消耗了生命能量,失去了健康的身体,增加了心里的阴暗面!”邪淫就是一种幻灭,会让美好的事物残酷地消失,前辈剖心沥血的劝告,每一句都发自肺腑,也可以说是字字泣血,希望那些无知的撸者早日回头,不要再执迷不悟地错下去了,不要再相信什么垃圾无害论,名为无害,实则害人,都是伪科学的理论,自欺欺人的陷阱。最新的科学研究证明,手淫的危害是非常大的,对身心的影响极为严重,刚开始可能感觉不明显,然而伤精时间长了,恶果就开始显现了,很多人只看到了爽,却没有看到里面的刀光剑影、腥风血雨……这是一场暗战!头脑里面的暗战!只有战胜了自己的心魔,才能真正主宰自己的命运。
\end{case}

\begin{case}
    戒色一年四个月,说一说我的心得吧。希望能对刚戒色的兄弟有些帮助,也希望所有兄弟们都能早日戒除这个恶习,蓦然回首现在的我跟一年多前的我对比了一下,简直就是两个不一样的人,以前的我浑浑噩噩没有理想,没有目标,现在的我浑身充满斗志,非常自信,正朝着我的目标前进着,以前的我因为社交恐惧症,变得怕人,不敢跟人接触,如今的我走路挺胸抬头,堂堂正正、目光如炬,发宣传单也很自然了,内心非常光明,以前的我心里阴暗,嫉妒、怨恨等等各种负面情绪占据了我的心,使我变成一个浑浑噩噩如同行尸走肉般的人,看见别人做好事就认为别人是做作,看见别人比我好就嫉妒,别人说出了我的缺点我就怨恨。感恩戒色吧,让我获得了重生,也感恩戒色吧让我接触到了传统文化,以及佛法,感恩吧主们的无私奉献,也祝所有深受手淫危害的兄弟们,都能够早日戒除。

    \textbf{附评} 很多戒友在戒除手淫恶习后都判若两人,从颓废灰暗的状态回到了正气爆棚的状态,内心光明坦然,目光也纯净坚定,一股强大的气势开始蔓延开来。记得之前有位戒友也因为手淫得了社恐,出现了对视障碍,后来他戒到一定程度,就敢于和别人对视了,别人看到他这双充满正气的眼睛,马上就产生了敬畏之心。人与人之间的对视很奇妙,你正气足,就会让别人感觉到敬畏,感觉到一种强大的精神力量。邪气重的人很难与别人对视,眼神总是飘忽闪躲游移不定,说话也缺少底气。当你正气足了,你就不怕了,腰杆子也硬了,身材也挺拔了,一副气宇轩昂的样子,两只眼睛射出的神采让别人心生敬佩,英雄气概也开始出来了。以前把自己撸得像阴沟里见不得光的老鼠,戒色后撕掉鼠皮,完全是猛虎出笼,英姿焕发,顶天立地,昂首阔步,光明磊落,堂堂正正。邪淫会增加内心的“阴暗面积”,不断蚕食你的良知,一个孩子本来是纯真无邪的,染上邪淫后,心里充满了各种邪念,不仅是邪淫的念头,还有嫉妒、嗔怒、怨恨、傲慢等等负面的念头,发出的念头大多都是负能量的,一点也不懂得宽容和感恩,变得越来越自私自利、狭隘和偏执。每个人八识田里都有善恶种子,恶的种子不要去浇灌,不要让它开花结果,不要做邪事,要“闲邪”,然后要多做善事,多孝顺父母,多浇灌善的种子,让善开花结果。一个人在邪淫后,很容易滋生各种负面的念头,\textit{淫念一生,诸念皆起,邪缘未凑,生幻妄心;勾引无计,生机械心;少有阻碍,生嗔恨心;欲情颠倒,生贪著心;羡人之有,生妒毒心;夺人之爱,生杀害心;种种恶业,从此而起,故曰:“万恶淫为首。”今欲断除此病,当自起念,始截断病根。(印光大师)} 要戒除邪淫,那就必须在念头上下功夫,化解欲望就在念头上化解,要懂得修心,要懂得对治自己的邪念,不要让负面的念头占据你的内心。戒色不仅仅是对治意淫、怂恿之念,还要对治其他的负面念头,嫉妒、嗔怒、怨恨、傲慢一个都不能有,要把念头全部转为善的,存好心,说好话,行好事!善事要尽力去做,但不要执著于福报,君子只问耕耘,不问收获,尽人事,听天命,素位而行,随适而安。有的戒友就是太执著于福报,搞得自己患得患失,心理失去平衡。还有的戒友遇见一点挫折就变得不坚定,起了疑心和退心,开始怨天尤人。这样是不对的,挫折正是在考验你和磨炼你,挫折越多,你要越坚定,这样才对。之前一位戒友虽然努力行善,但生活中还是遇见了很多不顺和挫折,但他并没有怀疑和不坚定,虽然遭遇了不顺和挫折,但他依然继续坚持行善,结果好运终于来了,一笔生意赚了几十万!!!他的人生一下就逆转了!之前的不顺和挫折正是上天的考验,看你是不是真的有德行、有忍耐。这点太关键了,很多人遇见一点挫折就开始怨天尤人,这种人是不会有大发展的。\textit{孟子曰:“故天将降大任于斯人也,必先苦其心志,劳其筋骨,饿其体肤,空乏其身,行拂乱其所为,所以动心忍性,曾益其所不能。”} 我们必须要经得起上天的考验!
\end{case}

\begin{case}
    我每天坚持学习的积累这时候终于换来了一次顿悟,当然,这顿悟也不是学会了直接的观心断念,而是找到了我自己的一个斩断邪念的方法——思维对治。当时是这样的,我正在学习《戒为良药》,当时读到了这么一句话——“短暂的快感过后就是巨大的空虚,乃至无尽的悔恨。”那是读到这句话,我就有很深的触动,它唤醒了我那么多次戒色失败时的不甘心以及悔恨!我反复读了许多遍,有一种想要哭出来的感觉,然后我背了这句话,每次有强烈的冲动上来的时候,我就会拿出这句话,然后问问自己:“你还想悔恨吗?”就这样,我获得了我第一次的顿悟,也是这么多次失败与坚持每天学习半个小时积累下来的结果,就看了这一句话,我马上突破了戒色的最大天数,突破了一百天!我第一次感觉,自己可能真的开始重生了!从这次顿悟后,一直到后来突破一百天,我都靠着这句“法宝”给我斩断冲动的力量,所以当时我在贴吧鼓励破戒的兄弟,都会把这句话分享给他们。然后再谈谈渐悟吧,也就是我如何学会观心断念。如果你问我什么时候学会的观心断念,我还真回答不上来,我其实大概在八十天的时候就一直在练习断念口诀,每天五百遍不放松,但是成效不大,念着念着就起了念头跟着跑,然后就开始散心念,等到回过神来,才继续练习。其实回过头来看看,其实这也没什么,当回过神来意识到自己跟着杂念跑了,那一刹那就是观心的感觉,也就是觉察。不过我当时根本不识货,不知道那一刹那的价值,就这样一直练习。所以我当时一直不理解飞翔老师说的“不需要你思考,下意识地就断掉了。”当时我还傻乎乎地以为真是邪念上来,默念口诀就断掉了。后来,渐渐的,大概是在两百多天之后了,我忽然发现我不需要在对治邪念的时候搬出那句思维对治的话了,我一觉察到邪念,邪念马上就消失了,这个时候,我才渐渐明白飞翔老师说的“不需要你思考,自动就断掉”的内涵。其实背诵口诀五百遍的价值就在于尽量不要散心,学会那一刹那“念起即觉,觉之即无”的觉察力,念头一起,凛然一觉,马上就把邪念斩杀于无形,这才是这句口诀的真意,并不是说邪念来了,我这里念一声就斩杀了,而是直接一觉察,不需要任何思维。

    \textbf{附评} 这是戒友“开始行动吧196”的反馈,他戒色一年零一月,发了个帖子被加精了。他在帖子开头说:“戒色很难,因为你是一正念与千万邪念作战,敌众我寡;戒色又很简单,因为只需要时刻干掉当下一念,即可戒出生天。”他这句话说得非常好,是深有实战体会的断念行者才能说出的话,可以说是断念实战的高度总结。刚开始他通过每天坚持学习换来了一次顿悟,是对思维对治的顿悟,这让他突破了一百天,也算小有成就。后来他又逐渐领悟了观心断念,其实思维对治也需要观心,也需要看住念头,只不过是通过思维对治来消灭邪念,而直接的观心断念则不需要思考,直接觉察消灭即可。从熟背断 YY 口诀渐渐过渡到直接觉察消灭,这是需要一个不断练习的过程的,也需要对断念的理论有深刻透彻的理解与认识,熟极之后就不用背口诀了,直接觉察消灭即可。这就像学开车背诵路考口诀一样,比如一踩(踩离合)、二挂(挂一档)、三看(看倒车镜)、四转(转向灯)、五按(按喇叭)、六手刹、七走。路考口诀还有很多,刚开始通过背诵来熟悉,然后过渡到最后,就不用背口诀了,直接按照口诀的意思去做即可。小学时我们背过乘法口诀,做题时,心里还会默念口诀,到了初中就直接写出答案了,已经不用背口诀了。刚开始的戒色新人应该熟背断 YY 口诀,背到条件反射的程度,同时通过学习断念的理论来加深对口诀的理解,再通过不断实战来进一步体会和认识,这样断念实战的表现就会逐步提升。刚开始应该背十六字,也有戒友提倡背前四个字,也就是“念起即断”,或者背后八个字“念起即觉,觉之即无”。在最开始的阶段,最好还是背十六个字,这十六个字是一个整体,背十六个字虽然慢些,但有助于对断念整体的认识与理解,背到一定程度可以简化到四个字或者八个字,进阶到最后阶段,从有声化为无声,直接觉察消灭即可。根器好的、悟性高的戒友,他很快就能从背诵口诀过渡到直接觉察消灭,这个过程可长可短,因人而异,有的人可能需要一年后才能过渡到直接觉察消灭,和练习的勤奋程度和个人的悟性密切相关。这位戒友每天五百遍不放松,他说成效不大,其实已经有相当的成效了,刚开始练习都会经历一个“分心 - 拉回”的过程,不断分心,不断拉回,慢慢觉察力就变强了,分心的情况就减少了。他那时候虽然练得很勤,但是对断念的理论一直不理解,后来戒到两百多天之后,他忽然发现自己会了!也就是“一觉察到邪念,邪念马上就消失了。”这就是觉察即消灭!!!这位戒友的悟性很不错,但也用了两百多天才真正学会,坚持学习和练习,就会由量变产生质变,到时就会迎来质的飞跃。当邪念上来时,念一声口诀也有一定的断念效果,但不如直接一觉察来得痛快,来得迅速。断念的几种方法:思维对治、觉察消灭、念佛号等都是不错的,关键要熟练、要管用!觉察消灭是其中最快的一种,思维对治的威力也很大,佛号的殊胜在于佛力的加持,这几种断念的方法可谓各有千秋,自己一定要勤加练习,没有人生下来就是神枪手,都是练出来的,所谓久练自化,熟极自神,到时自然可以降伏心魔。\textit{妄想刚强,久战自服,必无疑也。(莲池大师)} \textit{孔子曰:“我战则克!”} 克念作圣,能够克服妄念、邪念,就可以做圣贤。孔子一辈子没掌过兵权没打过仗,但是他说“我战则克!”这四个字力道太大了,简直是字字千钧,对待邪念入侵,必须要力战!奋战!血战到底!决一死战!必须要战胜邪念!!!
\end{case}

下面步入正文。

这季是关于念头实战的,而且是关于念头实战中比较有针对性的部分,那就是要学会擦除脑中的图像,也可以说是清除。心魔的表现就是邪念袭脑,而心魔的三种表现形式为:

\begin{multicols}{3}
    \begin{itemize}
        \item 念头
        \item 图像
        \item 微妙的感觉
    \end{itemize}
\end{multicols}

第一种念头,是比较粗的念头,属于概念思维,自己可以清楚地感受到某个念头上来了;第二种是图像,图像是念头的另外一种表现形式;第三种是微妙的感觉,这属于比较细微的念头。有的戒友说不是还有头脑里的声音吗?其实声音指的就是第一种,也就是概念思维,就像有个人在头脑向你讲话一样,心魔怂恿往往就是这样,你会感觉到头脑里有一个声音在劝你破戒,为你破戒找各种借口,总之各种劝,就是为了动摇你的戒色立场而让你破戒。有时看过的不良小说的片段或者擦边文字也会从头脑里跳出来,这都属于第一种,也就是比较粗的念头。我这季主要讲的就是第二种:脑海中浮现的图像或者影像。大家可以回顾下以前破戒的经历,很多时候都是一幅图像浮现在脑海中,然后牢牢占据你的头脑,这就是心魔入侵了,以图像的方式进攻了你的头脑,如果不及时断掉,图像就会一帧帧播放,就像在头脑里播放短片一样。

有位戒友是这样说的:“才坚持三天就忍不住了,满脑子都是 AV 画面。”浮现在脑中的图像画面按照类型来分一般有三种:

\begin{multicols}{3}
    \begin{itemize}
        \item 黄片里的画面场景
        \item 自己意淫幻想的画面场景
        \item 放纵回忆的画面场景
    \end{itemize}
\end{multicols}

之前有位戒友一想到过去嫖娼放纵时的情形,他就会完全失控,那种图像面面一上脑,他就会彻底沦陷。画面图像的刺激是非常强烈的,就那么一闪的刹那就攻进了你的头脑,是非常强势、非常快的进攻。而且这种浮现几乎是没有任何征兆,走路时、坐着时、躺着时,或者干其他事情时,图像都可能会突然浮现出来,就像不速之客。小时候看过动画片《狼来了》,而图像来了,不亚于狼来了,图像的入侵是那么快、那么微妙。心魔的第三种表现形式就是微妙的感觉,其实图像也非常之微妙,只不过第三种表现形式比图像还要更加微妙。战场在内心,脑海即是内心,战场就在两耳之间,这是一个看不见硝烟的战场,上演的都是暗战!表面上看那个人坐着什么都没发生,但是他的脑海里正在进行激烈的战斗,两种念头在他头脑里打架。一种不想破戒,另外一种劝他破戒。这种暗战就发生在每个人的头脑里,这种战斗虽然看不到硝烟,但是废掉却是实实在在的。那些邪淫的图像冲上你的头脑高地,牢牢占据你的头脑,这就是一场侵略,图像入侵了你的头脑。本来你戒得好好的,但是图像入侵后,你就开始找黄看黄了,又掉进了破戒的怪圈。

回顾我十多年的手淫史,大多数都是以图像入侵为主,因为图像往往是最具诱惑性的,一意淫就是图像呈现,然后就开始找黄看黄,那个年代找黄也不是很容易,有时就自己一个人幻想龌龊的场景。那时我并不知道自己被入侵了,心魔在我头脑里植入了图像,我却浑然不知,只要脑海中的图像一浮现,我就开始故态重萌。记得做学生党时,有时上课都会沉迷在那种想象或者回忆里,几乎不可自拔,表面上看好像在听课,其实脑子已经不知道跑哪里去了。我现在已经学会了及时“擦除”脑中浮现的图像,即使戒到现在,那些回忆、那些黄片里的场景还时不时会浮现出来,对现在的我而言,它们已经构不成威胁了,因为我可以做到及时断除,所以它们就无法占领我的头脑。而很多人无法断除,结果就会被图像带走,深陷其中,从而欲火中烧,不得不破。袭脑的图像往往是你印象中比较深刻的内容,那种图像的威力非常大,很容易让你产生冲动。曾经的我一次次被心魔攻破,但现在的我已经不是过去那个被心魔随便虐的戒色菜鸟了,我对心魔的进攻套路已经有非常深的认识,并且一直在练习观心断念,所以图像一上来,我就能立刻消灭。心魔发动一次次攻势,而我自岿然不动,在刹那间就化解了心魔的进攻。记忆中极具诱惑的图像依然在攻击我,但是我一次都没让心魔得逞。

你的意淫、你的回忆会以图像的方式浮现出来,你看到的诱惑内容也会以图像的方式再次浮现出来,一般最近看到的内容会很频繁地骚扰你,比如昨天你不慎看到的图片,今天就可能不断地浮现在脑海里,企图把你拉下水。图像浮现的规律一般有两条:

\begin{multicols}{2}
    \begin{itemize}
        \item 过去印象深刻的会反复出现
        \item 最近看到的会反复出现
    \end{itemize}
\end{multicols}

另外图像的出现是有半衰期的,达到一定时间它的强度就会自动衰弱下去。如果你一直在想那个图像,它的强度就很难衰弱下去,因为你在喂养它,只要你不去跟、不去想,它的强度自动就会衰弱下去。原来它是一小时内浮现很多次,最后是几天或者十几天浮现一次,后来可能几个月浮现一次。不少戒友都用到了一个词:挥之不去。图像袭脑后,他无法使之消灭,结果那幅场景就牢牢占据他的头脑,像电影一样在他脑海里播放。这就是断念实战差,断念水平不行,有的戒友会使劲晃脑袋,希望把邪淫的图像晃掉,这种方法有时是有效的,但如果欲火中烧了,就难以奏效了,关键还是要把握好断念的黄金时机,也就是刚起念的刹那,刚起念时最弱,随着时间的延长,会变得越来越强,最后就会欲火中烧。图像上脑后没有及时断除,那就像被安装了一个木马病毒一样,接下去发生的事情完全是身不由己,一种强大的冲动会让你疯狂找黄看黄和手淫,这种冲动非常强大,有的人甚至可以连续看一下午、一通宵的黄,就像打了鸡血一样,达到了废寝忘食、丧心病狂的地步,像一个饿鬼一样疯狂下载、疯狂看黄,根本停不下来,在那种状态下时间过得飞快,感觉一会功夫几小时就过去了,还没怎么样,天就黑了。

当你的头脑处于无念的状态时,就像一块没有任何字和图像的黑板,是一块空白的黑板,当图像浮现时,你就要马上擦掉它,刚开始你也许做不到,毕竟断念水平很差,但只要勤加练习,慢慢就能做到了,当你真正能做到了,就会发现擦除的过程其实是很简单、很轻松的,就在那么一瞬间,实战就结束了,高手不会陷入拖沓与挣扎,高手断念都在一瞬间解决战斗,你甚至看不到高手的脸上有任何的表情变化,还是那么神情自若,仿佛什么也没有发生,他只是那么一觉察,一切就都结束了。关键就是要及时擦除图像,就像擦黑板一样,你的黑板擦就是“觉察力”,只要觉察力足够强大,就能在图像浮现的刹那擦除它,不费吹灰之力。

\subsubsection{首先必须完成认同的转变}

刚开始需要一个认同的彻底转变,这个转变异常关键,从把念头认作自己,转变为认出纯粹的觉知是自己,你可以观察自己的念头,你就是那份纯粹的观察,当念头出现时,你可以作为一个旁观的看客,你可以不跟着念头跑。这个转变是最最关键的,有了这个顿悟后,你就和念头或者图像之间拉开了一个微妙的距离,你开始学会控制自己的念头了。图像只不过是念头的一种表现形式,最根本的控制就是不要跟随,一跟随就是喂养,一喂养就会导致念头强壮,强到最后你就完全任其摆布了,强到最后念头就是一头怪兽。\textit{勿随妄想,善观自心。(顶果钦哲仁波切)} 不要跟随念头,不要给它注入能量,你不跟随,它就会失去动下去的动能。这点认识很微妙,过去的你一直在跟随,因为你认为念头就是你自己,当念头一出现,你就当做自己,从而跟着跑,学习戒色文章后,你突然意识到自己是纯粹的觉知,你仅仅是一个观察者,看着自己的念头。这份转变会带来全新且根本的蜕变,你开始认清真正的自己到底是什么了。

\begin{case}
    我总是会无意中想起自己三年前看过的片,并且里面的剧情在我脑海里一清二楚,我几乎都能把整集回忆出来,实在是太痛苦了。每次一回忆起,自己就感觉十分难受,以至于我要使劲地捶自己的脑袋,迫使自己不去 YY。每次 YY,我就感觉精走了,就有一种想上厕所的感觉。我知道,我破戒了。但我毫无办法,纵使我通过跑步、做俯卧撑之类的来断淫念,但是都只能适用一时,没过多久淫念又起了,对此我真的很痛苦。

    \textbf{解析} 这位戒友就是被回忆搞得焦头烂额,搞得异常痛苦。那种回忆的图像画面一袭脑,他就十分难受,因为他正在戒色,不想 YY,但是架不住回忆画面的强大,在 YY 后感觉精走了。沉迷 YY,下面就漏了,漏了之后身体就容易出现症状反应,气色也可能会瞬间变差。严格来说,沉迷 YY 就算破戒,所以必须要克服 YY,特别是回忆式的图像场景。这位戒友还没有真正学会断念,他说自己毫无办法,跑步、俯卧撑只能适用一时,你不可能总是在运动,而图像回忆会在你独处时对你疯狂轰炸,那种图像一入侵,自己又断不掉,那真的很难受、很无助。一次次被图像带走,一次次破戒,眼睁睁看着心魔一次次得逞,却无能无力。对付图像,可以选择多种处理方式,最直接的就是觉察消灭,思维对治和念佛也很不错。\textit{无上武器,便是一句佛号。(黄念祖老居士)} 一句佛号,破一切魔网,一句佛号,斩断一切邪念。图像一上来,马上祭出大招,打退图像。不管你用何种方式,总之要让入侵的图像立刻被消灭,不能让图像在你头脑里挥之不去。
\end{case}

\subsubsection{立断:一声好似轰雷震}

《三国演义》里有一段:

\begin{quote}\it
    却说文聘引军追赵云至长坂桥,只见张飞倒竖虎须,圆睁环眼,手绰蛇矛,立马桥上。飞乃厉声大喝曰:“我乃燕人张翼德也!谁敢与我决一死战?”声如巨雷。曹军闻之,尽皆股栗。曹操身边夏侯杰惊得肝胆碎裂,倒撞于马下。操便回马而走。于是诸军众将一齐望西奔走。正是:黄口孺子,怎闻霹雳之声;病体樵夫,难听虎豹之吼。一时弃枪落盔者,不计其数,人如潮涌,马似山崩,自相践踏。后人有诗赞曰:“长坂桥头杀气生,横枪立马眼圆睁。一声好似轰雷震,独退曹家百万兵。”
\end{quote}

邪念来犯时,你也可以像张飞一样勇猛,大喝一声“断”,这一声干脆狠决,大喝过后,邪念消失得无影无踪,什么图像场景,瞬间清空!在修道方面专门有这类修法,就是大喝一声“呸”或者大喝一声“断”,猛烈短促,瞬间杀灭入侵的邪念。我们戒色应该拿出张飞之勇魄,一夫当关万夫莫开。振威一喝,邪念粉碎!踞地狮吼,心魔胆裂!很多戒友一次次被心魔撕开防线,有时甚至是不堪一击,原因就是面对心魔时不够强硬,不够狠!诱惑的图像袭脑了,你要提起全副精神,迎战心魔,杀灭入侵的图像,千万不要犹豫,要果断坚决,必须拿出你的狠。“立断”这两个字是非常给力、非常狠的两个字,我最早是从大圆满法里看到的这两个字,觉得很殊胜,当下立断,立断一切妄念而清净自心,非常直接,非常干脆,非常有魄力,一点都不拖泥带水。立断坚决而果断,执行力极强,甚至带有一点强制性的意味,就像禅宗的棒喝一样,把你妄念瞬间打掉,是极具威力的霹雳手段。不管什么邪淫的图像,上来后立刻大喝一声“断”,瞬间就清净了,这个方法很不错,很长气势,但有地点限制,人多的地方不适用。

\begin{case}
    由于高中没有私心杂念,学习一直很好,我顺利地考上了重点大学,成了全家人的骄傲。大学的生活很轻松,也很自由,一个下午打完球很累,躺在床上,忽然第一次看黄片的画面浮现在我的脑海里,手不由自主地 SY 了,这是我第一次 SY,由于是住在宿舍里,所以每次我都要刻意地避开同学。开始 SY 还不是特别勤,但是过了些天,就不由自主地老想 SY。我开始了大学里的第一次逃课,回宿舍里一个人,关上门疯狂地看着黄片 SY,一次又一次地摧残着自己,就是为了获得那片刻的快感,其实是在摧残自己的生命。

    \textbf{解析} 这位戒友 SY 相对比较晚,一般都是初高中就开始了,而他大学才开始。他第一次 SY,就是黄片的画面浮现在脑海里,而他不知道去断除,结果就不由自主地开始 SY。图像浮现是那么微妙,也是那么突然,在一瞬间就插入你的头脑了,而且往往是最具诱惑性的情景,很容易让人失控。一旦打开了潘多拉的魔盒,这个人就开始过着两面的生活了,已经彻底被那短暂的快感俘获了,就像吸毒一样疯狂摄取,有的人隐约觉得这样不好,开始还有负罪感,看了砖家的无害论,就心安理得了,最后症状爆发才知道砖家说的都是鬼话。一旦尝到了那种快感,这个人肯定会进入隐秘的疯狂状态,在实验室中,小白鼠为了获得持续的多巴胺刺激,就会疯狂地按压控制器,达到了极端疯狂的地步。想象一下你自己在疯狂撸管时,那个节奏也像小白鼠一样。这位戒友考上的是重点大学,本来是好学生,后来竟然逃课看黄 SY,一次次摧残自己,一个好学生就这样堕落了。如果他深入学习过戒色文章,就应该知道在图像浮现时必须立刻断掉,这样就不会被心魔附体,从而走火入魔。
\end{case}

\subsubsection{断念的功夫是一层层提高的}

很多新人在刚开始戒色都不懂得断念的重要性,断念就是戒色最实战的部分,一切戒色理论都是围绕戒色实战展开的,脱离断念实战,就是纸上谈兵。一个人即使懂得再多,说得再对,如果实战表现一塌糊涂,那就和戒色菜鸟没有多大区别,关键还是看实战。之前有的戒友也很能谈理论,而且是关于断念的理论,说得还很对,我之前见过几个,他们对断念有一定的领悟,但是执行力却不行,缺少精深的练习,结果还是遇心魔即溃,没有任何战斗力。理论一定要落实到练习上,记得高中有本教材叫《天天练》,就是天天做题,天天练,强化实战能力。我们断念也要天天练,甚至分分练,秒秒练,每一秒都要看住自己的念头,不可放松警惕。这种看住并不是过于紧张,而是保持适当的警惕、警觉,是一种临战的状态,随时都准备着投入战斗。刚开始新人通过学习戒色文章,对断念初窥门径,初学乍练,然后学练结合,领悟加深,练习更勤,这样就登堂入室,渐入佳境,从略有小成到出神入化、登峰造极。断念的功夫是一层层提高的,一层比一层厉害,一层比一层狠,一层比一层强,一层比一层快!需要一个精进练习的过程,就像练习一门绝世武功一样,练到最后断念的功夫就会极为深厚,达到精熟化境的地步。断念的功夫是很高深的,但上手并不是很难,从熟背断 YY 口诀、学会观心开始,慢慢一步步深化和熟练,最后断念之刃就会极快、极狠,斩杀心魔于无形之中。图像上脑虽然很具有诱惑性,但只要你的觉察力够强大,依然可以瞬间杀灭之,犹如快刀切图,图像上脑,觉察之刃马上切掉它,就像切水果般干脆利落。

\begin{case}
    飞翔老师你好!我染上 SY 有好几年了,一直都在努力戒可就是戒不掉。每次一到第五天的时候,脑子里就不自觉地浮现以前看过的那些不健康的图片和视频,晚上做梦也是!我真的好烦恼感觉自己好没用。请老师帮帮我!

    \textbf{解析} 这位戒友虽然一直都在努力戒,但不得其法,还不懂得观心断念,这样当图像浮现出来时,他就会再次破戒,图像就是这么厉害,很多人在图像浮现后,很快就会失控,完全把持不住自己,又开始找黄看黄的老套路,那个猥琐的身影又开始忙碌起来。不少人也知道那些图像袭脑后自己就会失去控制,但他们却没有想到去擦除脑中浮现的图像,他们没有意识到这正是心魔的进攻。要想戒掉 SY 恶习,必须学会擦除脑中浮现的图像,这种浮现在很多时候都是自动的,并不是你主动去想,而是它自动浮现,真可谓防不胜防,关键还是要时时警觉,每时每刻都开启你的“杀毒软件”,图像一入侵,立刻杀灭,不能跟随。这位戒友说自己感觉好烦恼好没用,意识到自己没用,就应该发愤图强,天行健,君子以自强不息!一定要让自己的断念实战能力强大起来,必须强过心魔,必须比心魔快,必须比心魔狠,必须摆脱被心魔蹂躏的命运!要真正强大起来,就必须要勤奋练习断念,弹钢琴的天天练习弹琴,唱歌的天天练习唱歌,专业运动员几乎每天都在训练,为什么要训练?就是为了强化实战水平!在实战中一次次被心魔打败,感觉自己好没用,一直处于菜鸟的水平,难道你甘心吗?一辈子过这种被心魔奴役的生活?你难道就不能像个男人一样站起来战斗吗?必须抗争心魔的入侵,必须强大自己的断念能力,图像浮现,必须立刻断掉,必须够狠!拒绝再懦弱下去,拍案而起,拼命练习断念,做戒色的常胜将军!做戒色的烈丈夫!勇猛精进地戒!
\end{case}

\subsubsection{保持不被占据}

如果一个木马病毒占据了你的电脑,并且把所有的文件都变成了木马病毒的头像,你肯定会想方设法把它删除或者清除掉!大家应该都知道熊猫烧香病毒,2007 年肆虐网络,还有灰鸽子病毒,自 2001 年,灰鸽子诞生之日起,就被反病毒专业人士判定为最具危险性的后门程序,并引发了安全领域的高度关注。大家都在关注自己的电脑、手机是否中毒了,是否被病毒入侵了,但少有人真正关注自己的头脑是否被心魔入侵了,心魔才是真正的黑客!大家都没有这方面的知识与意识,那一幅幅图像入侵你的头脑,就是一个个木马病毒企图劫持你的头脑,必须建立“专杀工具”来对付这类病毒,人们因为没有意识到这是一种病毒,所以就任其肆虐自己的头脑,一旦真正认清了,就会想要清除它,以免被图像感染。必须强力清除霸占头脑的意象,意象就是脑中浮现的图像或者影像。解除占用、强力删除顽固木马,对于特别顽固的图像要强力绞杀!当你处于无念状态,你的头脑并没有被任何念头占据,处于“无占用”状态,当你开始思考了,就处于占用状态,我们可以利用好的念头去做事,包括学习工作都需要用到念头,但对于不良的念头病毒就要学会杀灭!就像电脑方面有正常的程序和恶意病毒一样,对于坏的,必须要及时杀灭!我们必须知道我们的敌人是谁,我们的敌人如何展开进攻,我们如何进行防御,保持自己不被恶意图像占据,必须学会杀灭恶意图像。很多人在图像入侵时,他不杀,他跟!他没意识到图像就是恶意病毒,会劫持他进行看黄手淫的行为,图像的入侵就是为了“实施控制”!把他变成撸管的肉机!一旦有了相当的觉悟后,就懂得要及时杀灭图像,不能被图像占据,对你的头脑进行反病毒安全保护。在电脑方面,专业人士对于病毒的出现用到的描述是“不断弹出”,而在戒色方面则是“不断冒出或者浮现”,对于图像的入侵用“浮现”来描述更为贴切一些,图像就是为了占据你的头脑,对你的行为实施控制。

\begin{case}
    老师,我真的很想戒掉这个“鸦片”,每次完事后就对自己说下次再也不玩了,可是过了两三天,最多三天,脑子里就慢慢浮现出那些色情。我现在还得了前列腺炎,可能很严重了,但没钱去看,也不敢告诉父母,前年我就得了,不过看好了,然后又控制不住玩了起来,到现在感觉屁股后面有时针刺一样痛,反正很严重了,我恳求飞翔老师能帮助我,如果能戒掉这个,我的日子可以过得很充实。

    \textbf{解析} 这位戒友用到了“鸦片”这个词,可以说一针见血。很多人从初中发育起就开始撸管了,从那时起就沦为了瘾君子,色情就是一种毒品,国外是这样定义的——new drug(新型毒品)!色情的危害是非常之大的,腐蚀人的灵魂,让人症状缠身,让人的思想变得肮脏龌龊,想想撸管前,你是多么纯真无邪,撸了几年后,你的内心已然阴暗了很多,很多负面的念头已经开始占据你的心灵。这种心灵纯度的下降,对于一个人而言是非常悲剧的事情,就像从天堂跌落到邪淫的粪坑里,做了邪淫的蛆虫。不少人也知道这样不好,那种被心魔控制的感觉真的很可怕,那完全不是正常的自己,心魔的进攻就是为了占据、占领你的头脑,你的头脑就是战略高地,是“兵家必争之地”!因为你不是心魔的对手,所以也谈不上争夺,简直就是拱手送给心魔,孔子是“我战则克”,而你是“一触即溃”,还不知道怎么回事,已经阵亡了,当脑子一次次浮现那些色情的图像场景,你就一次次败下阵来,毫无抵抗力与战斗力。首先你必须清楚地认识到图像上脑就是心魔的进攻方式,一上脑就企图占据你的头脑,而你要做的就是“不被占据”,必须把图像赶出你的头脑,强力清除霸占头脑的图像,彻底粉碎和清除。这位戒友已经得上了慢前了,还是学生党,年纪还小,但已经是慢前患者了,他想戒但控制不住,我很理解他的处境,我做学生党时几乎和他一样,也已经是慢前患者了,十七八岁时我就已经像具行尸走肉一样了,手淫严重摧残了我的青春。那时的我根本不是心魔对手,就像小学生和职业拳王打比赛,结果可想而知,根本就是一边倒,后来我通过学习戒色文章和练习断念,终于让自己强大起来了,最终把心魔给 KO 了。其实很多孩子都想戒手淫,但是图像一占脑,就马上开始找黄看黄了,在那种状态下,他是被心魔附体的,根本身不由己,就像一个恶意程序已经开始运行了,看黄手淫的行为就是那个恶意程序运行的结果。心魔是最黑的黑客,一次次入侵你的头脑,一次次得逞,一次次让你看黄手淫,如果你没意识到这个黑客的存在,那也就谈不上任何反抗和抵制,这个黑客一直在冒充你,而你却浑然不知,你一直在执行这个黑客的指令却以为是自己的想法,认识和看穿这个黑客,降伏它!保持头脑不被占据!你必须确保你头脑的安全,确保自己不被心魔攻破,心魔是 BOSS 级别的,攻破一个戒色菜鸟的头脑也许只需一刹那或者几秒,而心魔难以攻破戒色高手,戒色高手的安全级别相当高,防御体系非常完善,漏洞极少。心魔入侵,在头脑里植入图像,如果不立刻清除,那就会被心魔操控。当一个木马病毒入侵了你的电脑,电脑杀毒防御体系马上做出反应,第一时间拦截并杀灭,当我们的头脑被心魔入侵时,也要第一时间拦截并杀灭。一次次被心魔入侵控制,这就是发生在无数撸者身上的事情,当然很多人并不知道自己被入侵了,因为心魔在冒充他,让他以为是自己的想法。心魔入侵最根本的目的就是实施控制,控制你的肉身,控制你的行为,让你看黄手淫,让你邪淫堕落。大德说过,真正的敌人是自己的心魔,说得极是,必须随时开启你的“杀毒软件”,时时保持警惕,不要被心魔攻破!
\end{case}

\subsubsection{工欲善其事,必先利其器}

当你要去砍柴了,你会把自己的砍刀磨得很锋利,这样砍起来才爽快,才给力。我们戒色也是如此,必须要好好练习断念,这就是一个“磨刀”的过程,你平时就可以拿普通的念头来练习,比如一个念头在头脑里出现了,你是否能够立刻让它消失?如果普通的念头你都无法让它消失,那邪淫的念头就更难了,强迫的念头比普通的念头要强很多,恋癖的念头又比普通的意淫要强很多,就怕“意淫 + 强迫 + 恋癖”,这简直就是魔王级的“三头怪”。戒色就像进入了一个打怪系统,不同的怪,不同的难度,你必须一个个去打败它们,战胜一个就获取相应的经验值,真正的戒色高手都打败了成千上万的念头怪,当你的觉悟极高、断念极强时,你就可以做到“全服通杀”!没有你杀不掉的念头怪,任何念头、任何图像上来,都会被你立刻干掉!这个战场就在脑海里,你自从来到这个世界上,就注定要进入这个打怪系统。你必须要不断强化自己的断念能力,强化、强化、再强化,不怕千招会,就怕一招精,练就自己的超杀,做心魔的克星。你想想自己每次面对心魔时的表现有多么垃圾、多么菜鸟、多么不专业,每次都被心魔吊打、暴虐、蹂躏!毫无还手之力,这种憋屈的日子是该到头了,剧情应该反转了,心魔骑在你头上拉屎撒尿、作威作福的日子应该彻底被终结掉!必须要有一个决心和意识,那就是要赶紧强大起来,不顾一切地提升实战表现,要用强大的实战表现来征服心魔!我到现在回答了好几万的问题,其中关于戒色失败最终极的原因就是——断念实战不行!戒色的道理懂得再多、说得再对,但是却不练习断念,结果遇见心魔还是过去那个软弱无能的老样子,心魔入侵时表现极差,又掉进了破戒的怪圈。奥运冠军是练出来的,练习使人强大,理论知识是需要的,但更关键的是提升实战表现,不能仅仅停留在理论的层面,应该要极其重视断念的练习,只有坚持练习断念才能让你战力爆表,才能让你蜕变成戒色的战神!久练功自纯,勤悟理自通,要练惊人艺,须下苦功夫。必须花大力气打牢基础,练好断念的基本功,只有断念水平真正过硬了,才能戒得风生水起,否则就会越戒越差,一潭死水,自己也备受打击,乃至最后放弃。当你看到自己的断念功夫日益精进时,当你降伏心魔的把握越来越大时,到时就会越戒越好,自信心也会与日俱增。

\begin{case}
    飞翔哥,我今天下午,心魔在动摇我戒色立场与戒色决心,没有及时断除,结果真的就像是被心魔附体一样,太可怕了。我本以为自己戒了这么长时间了,一个念头没及时察觉断除,被心魔附体这么可怕。

    \textbf{解析} 心魔擅长怂恿,我之前专门写过怂恿专季,心魔的怂恿五花八门,比如“撸一次吧,最后一次”、“戒那么久了,该放松放松了”、“试试定力吧,不要撸就行”等等,对于心魔的怂恿一定要学会识别,必须认清心魔怂恿的套路,有时心魔的怂恿非常具有针对性,当你心里犹疑不决时,心魔就会专门针对那个问题疯狂怂恿你。这位戒友也知道心魔在动摇他的戒色立场,但是没能及时断除,结果就被附体了,身不由己,彻底沦陷。他戒的时间也不短了,但是就那么一个念头没处理好,结果就被心魔攻破了。不管戒多久,都要保持警惕,平时每天都要练习观心断念,绝不放松警惕,戒色文章也要反复深入学习,必须认清心魔所有的套路。有的人能识别能认清,但是缺少“断力”,这就需要靠平时不断练习来发展强大的断力,断不掉,就会被附体,这就么残酷。所有的战斗就在一念之间,当念头图像冲上来时,就看你的表现了,关键时刻不能掉链子,关键时刻绝对不能软弱和犹豫,有的戒友在断念时还在贪恋,舍不得断,这怎么行?!必须果断坚决,像劈砖一样,你一犹豫一贪恋,力量就不具足了。功在平时,台上一分钟,台下十年功,你实战的那一下子和你平时是否有深厚扎实的学习和训练是密不可分的,工欲善其事,必先利其器,磨刀不误砍柴工,不打无准备之仗,断念就要斩钉截铁,削铁如泥,断念是需要大魄力的,婆婆妈妈的不行!你平时成百上千小时的练习,最后就体现在那零点几秒,一切都在那零点几秒内分出了胜负,一个念头一幅图像上脑是极快的,刹那间就上来了,在这最关键、最紧要的关头,你却斩不断!太不给力了!你在那零点几秒内的实战表现应该体现出你受过非常专业的断念训练,是真正的练家子,而不是疏于训练的半吊子,你必须要坚决摧毁心魔的进攻!这是你的使命!在最关键的时刻怎能表现得像个菜鸟、懦夫和软蛋?!一触即溃,这是最大的耻辱!你必须强悍起来!你不干掉心魔,心魔就干掉你!
\end{case}

\subsubsection{做清理头脑的大师}

如果你家的客厅堆满了垃圾,你会放手不管吗?不会,你会立即清理。你的头脑就是一个内在的空间,你必须保持内在空间的干净,不能让邪念的垃圾堆满你的内在空间。修行方面叫净意、净心、洗心、清心,\textit{清静为天下正。(《道德经》)} 修行最关键的就是要不断净化自己的心灵,当你往内看了,就会发现原来自己的内在空间是那么脏,有那么多的负面念头,过去怎么一直没发现呢?只有当你真正往内看了,才发现自己的内心就像一个发臭的垃圾场,当你还是一个纯真的孩子时,你的内心是非常干净的,所以那时的你是那么开心,当你的内在空间被污染了,你就渐渐不开心了,不管撸多少次,你最后的感受就是没意思、不开心、不满足,更加的空虚和颓废,最后症状爆发那就相当惶恐和痛苦了。那一幅幅图像浮现在你的脑海里,企图占据你的头脑,你不能允许这种情况发生,戒色高手绝不允许诱惑图像占据自己的头脑,戒色高手立刻就清除了浮现在脑海中的图像,出手非常之快,图像一浮现,眨眼间就消灭了,这就是戒色高手的作略和境界。很多撸者不是清除图像,他是被图像吸引住了,然后展开进一步的回忆或想象,最后欲火中烧、欲罢不能而破戒。图像出现时,千万不能跟随,不要被图像带走,不要卷入图像之中,要时刻保持警惕、警觉、警醒!每一秒都是实战!必须保持强大的觉知,看住自己的念头。大家手机里应该都装有清理大师等应用,清除掉手机里的垃圾,加速手机的运行,这其实就是一种优化,我们也应该学会清理自己的头脑,任何负面的念头都要清除掉,保持头脑的干净。

\begin{case}
    请教飞翔老师,念头来了根本挡不住,或者说当念头一来,我就坐立不安,想看会书,想干点别的什么都无法静下来,就感觉内心蠢蠢欲动,难受得要命。

    \textbf{解析} \textit{心之可畏,甚于毒蛇、恶兽、怨贼,大火越逸未足喻也。……纵此心者,丧人善事;制之一处,无事不办。是故比丘,当勤精进,折伏汝心。(《佛遗教经》)} 邪念是非常可怕的,甚于毒蛇、恶兽、怨贼!没有任何人可以把你扔进地狱,是你自己的心魔把你驱赶进地狱的,你没意识到邪念的可怕,你一次次跟随邪念,跟随浮现出的图像,从来没想到要断除,最后邪念就会主导你的人生,让你彻底沦为心魔的奴隶,真正的奴隶社会就在人的头脑里,只有战胜了心魔,你才能解放自己,重新变为自由人。这位戒友说念头来了挡不住,为什么戒色高手能挡住?为什么戒色高手能灭掉?因为戒色高手专业而强大,觉悟高,断念强,心魔尽管放马过来,戒色猛将直接将心魔斩落马下。当你很弱小时,你是难以挡住心魔的进攻的,你的内心很快就会陷入混乱与被动,变得蠢蠢欲动,坐立不安,破戒一触即发。心魔的入侵会破坏你内心的清静、和谐与平衡,你的内心被心魔搅得翻江倒海,难以平静下来。回想我自己被心魔反复狂虐的日子,就是因为那时的我太弱小了,弱小注定挨打,即使外在的敌人不欺负你,你内在的敌人也会欺负你,心魔不会放过你!我们必须强大起来,只有强者才不会破戒,弱者肯定会被心魔攻破,然后被心魔奴役,做不得主。别看有的人一天撸七次,其实不是他厉害,而是他处在身不由己、被心魔奴役的状态,不断耗损宝贵的肾精,几年后就沦为了废人,各种伤精症状开始“伺候”他。心魔的入侵污染了你的内在空间,就像一盆清水里突然滴入了墨汁,我们要做的就是“及时清理”,晚一点都不行,必须要快,闪电般清除,否则等到扩散了,就难以收拾了。
\end{case}

\subsubsection{一块无形的肌肉}

人体全身的肌肉共约 639 块,还有一块肌肉少有人知,解剖也无法发现这块肌肉,因为这块肌肉是无形的,这是一块“觉察肌”,谁懂得开发这块“肌肉”,谁就能战胜自己的心魔。以前健身时会说,今天练什么肌肉,比如练二头、三头、练股四头肌、练背阔肌、练腓肠肌、练肩部三角肌等等,那时的我把自己练成了肌肉男,人是强壮起来了,但是强壮却加速了我的自毁,我变得更加纵欲,还以为身体强壮就没事,其实身体再强壮也没用,神经迟早要出事,一身肌肉疙瘩最后却得了神经症,生不如死。戒色这几年,我突然发现真正有价值、真正值得练习的就是这块“觉察肌”,大腿的股四头肌是人体力量最大的肌肉,深蹲可以超过一千两百磅,也就是五百多公斤,但是觉察肌的力量比股四头肌还要厉害,觉察肌的力量是无限的,就像数学上的正无穷大,而且觉察肌和高层次的修道是密切相关的,修道高人的觉察肌都特别发达。你去健身房练力量,慢慢力量指标就会增长,做的次数也会增加,本来你可以做三到五次,后来你进步了,可以做八到十二次,本来你可以举起五十公斤,后来你可以举起一百公斤,你一直练习弯举的动作,就会刺激到肱二头肌,让肱二头肌变得发达起来。同样地,只要你一次次练习观心,你的觉察肌就会慢慢变得发达起来,对内心的观照会变得越来越强,达到一定的程度,念头就再也无法主宰你了。\textit{念起即觉,觉之即无;修行妙门,唯在此也。(圭峰禅师)} 功夫深了,看到念头,念头就会消灭。观心的要点:看当下一念,妄念起时,凛然一觉,它就消失了,一会又起,又凛然一觉,它又消失了。如果你看到念头,念头并未消失,那就说明你的觉察力还不行,还需要不断练习以达到“念起即觉,觉之即无”的境界。平时应该养成随时随地观心的习惯,进而习惯成自然。之前有的戒友说自己工作忙,没时间练习观心,其实观心和工作并不冲突,你可以在工作中保持觉知,这样工作效率往往会更高,我们消灭的对象是邪念,对于学习工作的念头要善加利用,在觉察状态下工作,就不容易走神分心,工作效率会比之前高许多,也容易涌现灵感。

\begin{case}
    飞翔哥您好!我是高一学生党,因为 SY,感觉到学习变得很吃力,之前的我学习悟性很高,现在就是能听懂课但是不会做题,反应也慢,真的是有苦说不出,手会不自觉颤抖,月牙也基本消失殆尽,身体一日不如一日,才十六的我就精索和早泄……唉,更可怕的是,我现在感到很绝望,我知道我周日午休后很容易破,我特地午休前看了戒色文章和笔记。睡觉时,我迷糊醒来,睁着眼,我心里有个声音说:“撸一下,最后一下吧。”然后我断掉了这个怂恿,但是我继续赖床,然后邪念一次比一次猛,怂恿一次比一次厉害,它说:“看 H 吧,换个方法撸。”这时候 H 片中的场景止不住地袭来,完全控制不住,嗡的一下我就无意识地开始撸了,完后我懵了,我不知道刚刚发生了什么。

    \textbf{解析} 十六岁的孩子已经得上了精索和早泄,脑力也严重下降,他说“能听懂课但是不会做题”,其实还是没听懂,否则怎么可能不会做题,他的脑力下降了,所以听不懂,而且反应迟钝,脑力下降对于学生党简直是一种灾难,学习太需要脑力支持了,其实工作也一样,脑力一下降,工作也容易出错。周末是很容易破戒的,因为独处时间比较长,所以周末要格外警惕,必须加强观心断念。虽然他特地看了戒色文章和笔记,但是当心魔怂恿时,他断念不力,还是被心魔蛊惑了,当脑中那个怂恿的声音响起时,必须立刻断除。心魔先是怂恿,然后 H 片中的图像场景开始在他脑海中浮现,他很快败下阵来,他说“完全控制不住”,这就是断念实战不行啊!觉察力强悍的戒色高手,当怂恿一出现,马上凛然一觉,怂恿就被消灭了,当图像场景浮现时,也是凛然一觉,图像场景立刻就消散了。凛然一觉就像一道刀光,快刀切念,快刀切图,真正把觉察力练强大了,在实战中就不怕心魔了,否则心魔的暴戾恣睢会让你感觉自己不堪一击,简直弱爆了。心魔虐你就像玩儿一样,打败你是那么轻而易举。这位戒友直接被心魔打蒙了,嗡的一下就开撸了,心魔的“怂恿拳”和“图像拳”暴风雨般砸向他,根本招架不住,就像把一个人逼在角落暴揍一样,直接被打蒙了。破戒的原因就是太弱了,必须强壮觉察肌,当心魔再次进攻时,你一记觉察重拳,直接把心魔撂倒!面对心魔的入侵,必须给出最强硬的回应!在内心的八角笼里降伏自己的心魔,做不败的王者!
\end{case}

\paragraph*{总结}

这季的文章我用到了擦除、断除、解除、清除这几个词,最核心的一个字就是“除”,愚人除境不除心,智者除心不除境。心生,种种魔生;心灭,种种魔灭。心即是念头,心魔的表现就是邪念袭脑,图像只是邪念的一种表现形式,必须要学会擦除脑中浮现的图像,你可以用觉察力去擦,也可以用思维对治去擦,也可以用念佛号去擦,也可以大喝一声“断”,直接把头脑瞬间清空。无论何种方法,必须要练到出神入化,必须要达到纯熟之境,这样才有望击溃心魔的进攻。脑中浮现的图像往往很具有诱惑性,甚至可以让人瞬间丧失抵抗,从而乖乖就范,之前有的戒友也写过精品帖,但是后来还是败在了图像上,独处时邪淫的回忆以图像的方式在脑中浮现,没有及时断掉,结果就被附体了。在被心魔打败后,真的是灰溜溜的,之前通过戒色积累的自信和底气也因为连续破戒而丧失殆尽。破戒后一定要找出根本原因,当图像浮现时,你在干嘛?你为什么放任图像浮现而没有及时断除,你的警觉在哪里?你的实战意识在哪里?破戒后一定要好好反省和总结,一定要不断强化断念实战,不要做理论的巨人,实战的矮子,实战的那一下子必须够狠、够快,瞬间解决战斗,绝不拖泥带水,当断不断,反受其害,断力要足够强大,必须及时清除浮现在脑中的图像,不要让头脑被邪淫的图像占据。最近有一天我正在吃早饭,突然一幅图像就上来了,是过去邪淫的回忆画面,非常快,诱惑性很强,实战就发生在零点几秒内,唰!一记觉察就消灭了,有时一天上来五、六幅,有时上来十几幅,但都是唰唰唰,觉察的刀光切掉了所有的图像,也切掉了胜负的悬念。在实战时必须狠,不能懦弱,不能犹豫,不能贪恋,要拿出最狠、最强大的实战表现来抗击和抵御心魔的入侵,千万不能被心魔攻破,一旦沦陷,心魔就会为所欲为、烧杀屠城!

这季专门强调了清除图像,图像浮现导致的破戒数不胜数,希望大家对实战的体会能够细腻起来,图像的浮现是非常微妙的,\textit{吾人为学,当从心髓入微处用力,自然笃实光辉。(王阳明《传习录》)} 入微是指达到非常精细微妙的程度,要达到入微之境,必须多学习戒色文章,然后对实战要反复深入体会,以捕捉到那非常精细、非常微妙的部分,入微就是一种高层次的境界。图像来得那么微妙,然而一个觉察就可以让它瞬间消失,关键是要时刻警觉,不断练习和体会,最后你的感觉就会越来越敏感、越来越细腻。当说到某种轿车,那种轿车的样子就浮现在你的脑海里;当说到某个人物,那个人物的形象就浮现在你的脑海里;当说到某个著名的景点,那个景点的样貌就浮现在你的脑海里。你过去一直认为这是理所当然的,但是当你试着去仔细感觉它是怎么浮现的,它又是怎么消失的,就会发现那个过程是极其微妙的。当你跟随图像,它就会继续浮现,以你想象的方式浮现下去,而当你集中全部注意力去看它时,它突然就消失了,消失得无影无踪,这就是最奇妙之处,这就揭示了背后一个至为关键的原理,那就是当你处于极度警觉、注意力高度集中时,你的脑中是没有任何念头、图像的,什么也没有,就是一片空白,就像一张没有画画的白纸。当你不去跟,而只是单纯地“看”,随着这种“看”得到不断强化,到时就能消灭掉所有念头、图像。这种“看”就是对内心的觉察,而不是用肉眼去看,过去的你一直在跟念,却几乎从来没去“看”念头,这就导致你一次次被卷入念流之中。戒色的高手已经学会通过觉察、通过“看”来进入无念的状态,念头是被他们“看”掉的,他们的“看”威力十足,他们真正懂得如何切换进入无念状态,就像把手机切换成静音模式一样,只要一记觉察就可以切换进入。他们已经学会让图像立刻消失,这样就等于击溃了心魔的进攻。希望大家深入学习和体会这季的内容,这季的内容非常重要,如果你把这季真正悟透了,你的实战表现将会得到极大的提升。

下面分享四首戒色诗歌。

\begin{poem}[抗撸狙击手]
    \begin{multicols}{3}
        \centering~\\
        魔军入侵,兵临城下 \\ 狙击手临危不乱 \\ 在悄无声息中 \\ 沉着应战 \\ 予以心魔迎头痛击 \\ 冷静果断、闪电出手 \\ 我们没有退路 \\ 必须击溃心魔 \\ 不是心魔死 \\ 就是你亡 \\ 就这么残酷 \\ 每一秒都是实战 \\ 必须具备超乎常人的警觉 \\ 孤胆英雄,坚守戒色阵地 \\ 一个人干掉一支军队 \\ 对一切邪念采取零容忍的态度 \\ 只要出现,必将消灭 \\ 狙杀邪念,雷霆出击 \\ 来多少杀多少,一个不留 \\ 秒杀邪念,履行天职 \\ 每一颗子弹消灭一个敌人 \\ 邪念就是来送死的 \\ 强悍的抗撸狙击手 \\ 绝对坚毅的眼神 \\ 一个让心魔颤抖的 \\ 至强存在
    \end{multicols}
\end{poem}

\begin{poem}[在纯净中燃烧]
    \begin{multicols}{3}
        \centering~\\
        十年之前 \\ 我看到你的灵魂 \\ 美好而清澈 \\ 十年之后 \\ 我再次看到你的灵魂 \\ 丑陋而龌龊 \\ 我都快认不出你了 \\ 是邪淫把你变成了地狱中的丑鬼 \\ 你心里清楚得很 \\ 你一直在喂养自己的心魔 \\ 你就是一个傀儡 \\ 你就是一具撸管的肉机 \\ 在毁灭前一次次疯狂 \\ 在堕落中渐行渐远 \\ 那个纯真的孩子 \\ 已然沦为了猥琐大叔 \\ 为了再次获得纯净的力量 \\ 你必须彻底蜕变 \\ 与邪淫的过去彻底决裂 \\ 在纯净中燃烧 \\ 烧光所有的丑陋与龌龊 \\ 烧光一切邪淫罪恶的过去 \\ 燃起熊熊的烈火 \\ 在烈火中净化 \\ 在烈火中涅槃
    \end{multicols}
\end{poem}

\begin{poem}[尘世卑微的撸者]
    \begin{multicols}{3}
        \centering~\\
        最后他松开了手 \\ 躺在床上 \\ 呼出了最后一口气 \\ 眼皮缓缓垂下 \\ 一滴浊泪从眼角滑落 \\ 过完了这邪淫荒唐的一生 \\ 若干年前 \\ 他哭着来到这个尘世 \\ 一个纯净的婴儿 \\ 睁着大眼睛瞧着这个世界 \\ 那么的纯真无邪 \\ 小手在空中抓着什么 \\ 若干年后 \\ 一具猥琐的躯壳 \\ 被送进火葬场 \\ 焚成灰烬 \\ 白发人送黑发人 \\ 人间的悲剧 \\ 邪淫真是大不孝 \\ 浑浑噩噩地撸 \\ 耗尽生命所有的精华 \\ 最后手终于松开了 \\ 什么也没抓到 \\ 卑微的撸者 \\ 过完了悲惨的一生
    \end{multicols}
\end{poem}

\begin{poem}[因为不撸,所以美好]
    \begin{multicols}{3}
        \centering~\\
        戒色百日后 \\ 一种微妙的美 \\ 开始浮现在他的脸上 \\ 纯净美好的笑容又回来了 \\ 这个笑容太久违了 \\ 原本这个笑容已经封存在 \\ 儿时的相片里和记忆的深处 \\ 而现在这个笑容又出现了 \\ 这个笑容里有很深的满足感 \\ 他双膝跪地,泪流满面 \\ 终于从邪淫的地狱出来了 \\ 终于找回了久违的美好 \\ 纯净丰盛的喜悦 \\ 再次进入了他的生活 \\ 在欲海里沉沦了太久太久 \\ 最后才突然明白 \\ 做回单纯的孩子是那么幸福 \\ 一个美丽而纯净的灵魂 \\ 一个纯粹的存在 \\ 因为不撸,所以美好 \\ 纯净就是一种力量 \\ 戒除邪淫,拿回你的力量 \\ 掌中,堕落的岁月在静静燃烧 \\ 最后全部化为灰烬,随风飘散 \\ 纯净的时光再次回来了 \\ 你就是一个孩子 \\ 一个无比纯净快乐的孩子 \\ 活在不撸的奇迹里
    \end{multicols}
\end{poem}

下面推荐一本书。

\begin{book}[《生命的真相:慈诚罗珠堪布谈轮回》]
    作者以无懈可击的推理、强大有力的论证、铁一般的真实故事,为你揭示一个你或许从来都没有思考过的生命真相,以现代人容易接受的方式深入浅出地为读者呈现了一部罕见优秀的关于前世今生的力作。通过佛法与科学多方面的比较说明,一方面找到它们的相异之处,一方面又挖掘出二者可互为印证的地方,并最终从科学与佛法互不相违的角度,论证了前后世存在的合理性与必然性。《涅槃经》中有一句话:“不见后世,无恶不造。”所以,懂得前后世的道理,对当代社会的健康发展,现代人群的人格改造,皆具重大意义与现实作用。慈诚罗珠堪布,1962 年出生于四川省甘孜藏族自治州炉霍县。1984 年于藏区之佛法重镇——色达喇荣五明佛学院出家,并依止当代最伟大的心灵导师法王如意宝晋美彭措,潜心深入经藏,刻苦学习。经过多年努力,精通了显宗五部大论及密宗之续部,并获得堪布学位。近十年来,更着力于深入研究西方科学、哲学,期以更贴近现代人的方式弘扬佛教文化。关于转世轮回这个主题,我个人是比较感兴趣的,濒死体验和再生人现象也经常关注,著名歌手火风就经历了濒死体验,后来他开始学佛,还有猪跑到寺庙去下跪,网上可以搜到图片的,就是一头猪两个前爪跪着,肯定之前是人,还带有人的记忆,做了恶事投了猪胎,这时才后悔。新闻也报道过湖南发现一百多“再生人”轮回转世,震惊全国,上海教育电视台也专门报道过,其中一个真实的案例是这样的:坪阳乡谱头寨有个吴姓男孩,前世是一头白猪,转世投胎为人后,因尚能准确地认出曾经杀死它的屠夫容某而在当地轰动一时,屠夫容某因此发誓今生今世不再杀生。原来,吴姓男孩与屠夫容某是一个村子里的人,小男孩一岁多时,家人带他到村里去玩,每次只要碰见屠夫容某,小男孩就要拼命地哭叫、挣扎,每次都这样,家里人也不知道个所以然。小男孩长到两三岁时,每当看见有人在地里采猪菜,他都要告诫他们,哪种菜太苦,哪种菜太辣,采多了,吃不下等等一些话。弄得大人们直好笑,说他小小男孩能懂啥事。这个时候的小男孩在村里更加害怕见到屠夫容某。每每见到容某,他就老远都会拼命往家里跑去,每次都这样。久而久之,村里人感到这里肯定有蹊跷,便试着问小男孩是何原因。哪料,小男孩说出了一个惊人的大秘密。原来,他前世就是他外公家里养的一头大白猪。还说,那天,屠夫容某带着一个人来买猪,白猪见不妙,拼命地往外跑,一直跑到他家背后的山地上,还是被容某等人追上来抓住,抬去他们家给杀卖了。这可是个爆炸新闻。村里人一传十,十传百。小男孩是白猪转世的事就这样传开了。还有再生人“石爽人”叙述死后经历,那个案例也给我留下了深刻印象,大家有兴趣可以搜着看一下。网上还有一篇文章也挺有名,那就是《现代中国境内 52 个投胎转世案例》,是摘自阿秋法王的传记,看着挺震撼的,也更加坚定了改过迁善的决心。重庆著名的大足石刻已经向世人透露了天机,看过国外的一篇文章说 21 世纪灵魂转世轮回的观念将会被普遍接受,我觉得现在这种趋势已经很明显了,现在国外濒死体验的研究书籍和视频越来越多,国内也有这方面的报道和研究,在我学佛之前,我就对此类现象很感兴趣,这对于了解生命的真相是很有帮助的。某些现象已经向你透露了背后的天机,如果不是愚夫,应该会猛然醒悟,从而踏上断恶修善之路。
\end{book}
