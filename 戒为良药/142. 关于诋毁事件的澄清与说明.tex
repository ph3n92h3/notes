\subsection{关于诋毁事件的澄清与说明}

某文章说戒色吧丢失了戒色本质,其实戒色吧从未丢失戒色本质,说丢失本质完全是无稽之谈和贬低之论,相反,戒色吧一直牢牢把握住了戒色的本质,不仅把握住了,而且还把戒色的意义提升到了更高的层面。戒色是系统工程,修心、改过迁善、行善积德、培养健康的生活习惯、严格自律、奋斗人生、积极锻炼、学习圣贤教育、养生之道、戒沉迷游戏、戒手机瘾、戒赖床、戒懒、戒骄、戒怒、戒贪、戒酒、戒不良习惯、时间管理、情绪管理、学会做人、学会感恩、孝顺父母、远邪友、交善友、反省忏悔、素食为主、戒杀放生、培养德行等,这些戒色要点,戒色吧反复多次提到过,不仅我的文章反复强调过这些戒色要点,戒色吧其他戒色前辈也都有提及。戒色是全面的改造,整体的提升,是整个生命模式的彻底转变,戒色吧很早就认识到了这一点,在他们的误解和贬低里,戒色吧好像一群沉迷戒色的初级戒色者,只知戒色,其他一概不懂,然而事实完全不是这样,戒色吧有很多高阶的戒色者,他们的觉悟和认识水平是相当高的。戒色只是生活的一部分,戒色吧从来不提倡沉迷戒色,而是应该理智对待戒色,戒色不仅仅是戒手淫,而是全面改造和整体提升自己,积极奋斗自己的人生。

戒色的根本并非修身,而是修心,大德也多有提到修心是根本,修行的核心是修心,这点是毋庸置疑的!念头是行为的先导,必须在起心动念上下功夫,修身的确重要,但最根本的还是修心。一位戒友说:“身体上的行为就是内心导致的,先修身再修心就是本末倒置,修心都做不到如何修身?”他说得很有道理,修身是外在的,要修身,首先要内心有所改变才行,修身先正心,心正了,外在的行为才能正,所以修心才是根本。\textit{修身在正其心……欲修其身者,先正其心(《大学》)},修身一定要首先正其心。《四十二章经》中,佛陀谈到戒色问题,直接一针见血地指出“断心”!佛陀为什么不说充实生活、改正不良习惯、健身锻炼、转移注意力?因为这些都是辅助,是外围,真正的核心是修心,是断念!如果本末倒置,把修身当作核心而忽视修心的重要性,那迟早还是会失败的,也许能戒除一段时间,但最终肯定会失败。\textit{要修身必正其心,儒家讲修身根基是你的心正,心不正,光练外形没用。现在很多年轻人学瑜伽术,做各种功夫,修不好的,修身之道根本在静心,心正了以后再修身。(南怀瑾先生)} 修身必先修心,修心是根本,不可本末倒置。大家想一想,当初染上手淫恶习前是什么状态?那时年纪还小,也就十三四岁,初一初二的年纪,那时的生活健康有序,读书,锻炼,生活很充实,作息饮食也很规律,基本没有其他不良习惯。为什么在健康有序的生活方式下还会染上手淫恶习?就是因为不懂得修心,不懂得对治邪念!这就是根源!到了一定的年纪开始有邪念了,有意淫了,想看黄了,如果不懂得断念,就会跟随那种想法,最后欲火中烧,不得不破。所以,修心才是根本,外在的修身只能作为辅助。说戒色吧丢失本质,实则是他们自己产生了思想误区,误认修身为本质而丢失了修心的本质!有一定修行常识的人都知道修心是根本,大德一直在强调修心是修行的根本,说修身是根本,真的是犯了很低级的错误,明眼人一看就知道不对。

外在修身方面,估计没几个人可以超过曾国藩了,而曾国藩日记写道:“\textit{午初,人欲横炽,不复能制}”,做了不应该做的事,遂骂自己“\textit{真禽兽矣}”!修身已经达到如此程度的人,还是会因为一次修心没做好,导致进入禽兽状态。那些成功人士修身不比你强?生活不比你充实?都奋斗成亿万富翁了,最后还是栽在了“色”上,邪念一起,不知对治,立马沦为衣冠禽兽!修心才是真正的本质和核心,这点认识一定要正确,修心是最究竟的,大家都知道《心经》,还有《金刚经》的“降伏其心”,突出的就是修心,修心是最根本的,真正治本的是修心,仅仅修身治标不治本,我们要依圣言量和大德的开示,这样才不会走错路。我最近看了国外 NOFAP 的戒色教程,他们也很强调断意淫,可见断念是戒色的重中之重,是最核心的关键。大家可以回想一下过去每次破戒是什么情况?就是念头、图像、微妙感觉袭脑时,没立刻断掉,结果那种念头变得越来越强,最后欲罢不能而破戒。戒色实战的那一下,就是断念!充实生活只能作为辅助,根本不是戒色的核心和关键。只重修身,严重偏离修心的戒色方法,是戒色门外汉的认识水平。他们说不战而屈人之兵,这并不适用于戒色,因为邪念来了,你不战是不行的,并不是说你有了很多好习惯,邪念就不来了,也不是说你念了几百万的佛号,邪念就不来了,邪念肯定会来,佛言:“\textit{夫为道者,譬如一人与万人战。挂铠出门,意或怯弱,或半路而退,或格斗而死,或得胜而还。}”戒色是要“降伏其心”的,并不是建立好习惯就能防得住的,真正防得住的是你的断念能力。修身的确重要,但修心才是根本,戒色的根本是修心,要从起心动念上去下功夫!多看大德开示自然就懂得修心的重要,仅凭凡夫有限的智慧所得出的结论,根本经不起实战的考验。一位资深戒友说:“向外驰求的戒色只能成功一时,走不进极其稳定的戒色层次,甚至在失败破戒后再成功一时都做不到。”戒色的根本是修心,把握住了根本,才能越戒越好,才能进入极其稳定的戒色层次。

学习戒色文章也并没有错,大家都是学生党过来的,都心知肚明学习的重要性和做笔记的重要性,所谓学无止境,即使不看戒色文章,也要坚持学习圣贤教育,根据我的研究和体验,学习可以让人保持良好的戒色状态,也可以增长知识和阅历,我戒到现在每天还在学习,真的是受益匪浅。戒色吧不仅吸收了传统戒色的精华,也有自己科学深入的研究和总结,更有学习和借鉴国外戒色的最新研究成果和成功案例分享,包括国外戒色视频的学习和激励,戒色吧可谓与时俱进,海纳百川,戒色吧是集戒色大成之地,众多戒色前辈都在这里分享经验,他们的经验和境界已然很高,戒色天数也非常可观。戒色吧具有自己的科学而成熟的专业戒色体系,很多戒友已经戒得非常专业,戒色吧的戒色体系和研究成果是基于大数据的分析、对比、反馈和研读,这是千万级的帖子数据,很多前辈的答疑量都超过几千乃至几万,他们的经验和阅历极具借鉴和研究价值。大数据得出的经验和研究成果,会更科学、更深入、更全面、更完善、更具针对性与参考性,在细节方面的把握会更精确,大数据的架构可以让你从更多的层面和角度来看待戒色这件事,你的视野会变得更广阔。戒色是一门学问,也是一种修为,可供挖掘的深度是极深的,而要挖得深、看得广、钻得透,就必须有大量化、多样化的大数据案例作为支撑,这是戒色吧所具有的独特优势。

关于断念口诀,也并非高僧大德的专利,好像我们凡夫不可能掌握,其实这种误解是在固步自封,在《菜根谭》《了凡四训》《传习录》中都有提到断念口诀的原理,这个口诀本来就是面向普通读者的,在戒色吧很多戒友已经可以熟练应用这个口诀了,只要正确理解和坚持练习,这个断念口诀是完全可以掌握的。这个口诀是最普适的,不信佛也可以用,断意淫是戒色的重中之重,所以这个断念口诀就显得极其重要和关键,这是一个传承上千年的修心诀,不仅可以断意淫,也可以断其他妄念,其价值实在不可估量。用失败案例来反对戒色吧是完全错误的,每个戒色平台都有很多失败的案例,毕竟这个时代色情泛滥,外界诱惑实在太多,而且从入门到精通也需要一个过程,在这个过程中就可能会出现多次失败,失败者要从自身找原因,行有不得,反求诸己。戒色几年出现破戒也是有可能的,不管戒多久,只要放松警惕,没做好断念,就可能被心魔附体,没有哪种戒色方法能保证永不破戒,关键还是要看自己的修为,前辈只能告诉你方法,至于能修到什么程度,完全是在个人。个别戒友也会走极端或者误解戒色方法,方法本身没问题,但是到了他那里,就容易走极端或者产生误解,这类戒友需要正确引导和纠正思想误区,并不能怪方法不好。某些人用练习不得法、有思想误区、走极端的失败案例来反对断念反对学习,这是完全错误的,明眼人一看就知道是偏见和误导,戒色必须注重断念和学习,但并不是叫你沉迷戒色,我们应该合理规划自己的生活,做好时间管理,平时还是以学业和事业为重。戒色不是让你整天抱着戒色文章不放,戒色只是生活的一部分,每天分出一部分的时间学习戒色文章,不要影响正常的生活,戒色后还是应该积极奋斗自己的人生,戒色并不是让你与现实脱轨,戒色是为了让你更好地面对自己的生活和人生。

我很注重和谐与团结,我是从大局来考虑的,以和为贵,以大局为重,戒色界内部已经分裂成好几块了,就是因为没有修好“和敬”。我很怀念以前那段时间戒色前辈之间惺惺相惜的日子,那时没有争执,没有贬低,只有互相支持与欣赏,但后来就慢慢变味了。公益事业没必要搞专利、争第一,我从来没有标榜自己的戒色文章是最好,我衷心希望后来人超过我,看到别人超过我,我随喜赞叹。真正的第一永远属于圣贤教育,在圣贤教育面前,凡夫的那点智慧显得极其浅薄和渺小,根本不值得一提。我呼吁戒色前辈之间多一些尊重和包容,放大自己的心量,不要陷入狭隘和自私,不要争第一,争心一起,诋毁随之,争伤和气,和谐才能双赢。戒色前辈之间应该互相谦让,互相支持,共同合作,只有这样才能让中国的戒色公益事业更有凝聚力,更有感召力,才能更好地帮助戒友。如果戒色前辈之间互相攻击、诋毁和诽谤,那是不堪设想的,我不想看到这样的局面。不管他们怎样贬低和诋毁,我也已经宽恕他们了,他们是前辈,能坚持下来也是不容易的,我希望大家能够和谐相处,共同为中国的戒色公益事业尽好一份力。我相信广大戒友也希望看到前辈之间能够和谐相处,这对于他们来说也是一种激励,和谐友好的氛围会带来极大的动力。之前大悲论坛内斗,让其从巅峰走向衰落,一内斗,有的前辈就会心生退意,之前的干劲也消失了,如果能够注重和谐与团结,那么中国的戒色公益事业肯定会蒸蒸日上。
