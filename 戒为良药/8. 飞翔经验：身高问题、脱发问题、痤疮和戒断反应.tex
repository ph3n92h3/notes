\subsection{身高问题、脱发问题、痤疮和戒断反应}

这季就身高、脱发、痤疮、戒断反应这四个方面谈一下,前面三个是戒友比较关心的问题,而戒断反应则是戒色后比较常见的问题,详细论述如下。

\subsubsection{身高问题}

很多戒友都会问,SY 究竟会不会影响身高,我的回答是:SY 的确会影响到身高,会影响到骨骼的发育。因为中医:肾主骨。问身高问题比较多的是 90 后,因为他们这一代正处在发育期,80 后问得比较少,因为 80 后基本都长定型了。很多人会产生这样的疑问,大家都 SY,为何有的人还是长得很高,为何有的人就长不高,其实这个问题很好回答,因为影响身高的因素有很多,SY 只是其中一个因素,很多人虽然 SY 频繁,但其他影响身高的因素他都避免掉了,这样对身高的影响就不明显。就拿我来说吧,我爸妈都不高,我妈 160 \unit{\cm},我爸 170 \unit{\cm},而我 186 \unit{\cm},虽然我发育期在频繁 SY,而且我的遗传基因也不好,但我其他方面做得很好,所以我依然能长到 186 \unit{\cm},虽然我能长到 186 \unit{\cm},但是我明显感觉我的骨密度不行,有点骨质疏松,运动时容易崴脚和骨折。后来我学了医理才知道,肾气不足会导致骨质疏松,而且一旦受伤,也不容易好彻底。据我了解,很多人经过一段时间的精心调理,虽然外表并没有长胖的迹象,但是一测体重,的确增重了,哪里增重了呢?其实就是骨密度,原来骨质疏松,经过调理,骨密度上去了,体重就上去了,外表是看不出来的。

我总结的影响身高的因素如下:

\begin{multicols}{2}
    \begin{itemize}
        \item 遗传基因
        \item 是否积极锻炼
        \item 是否熬夜久坐
        \item 营养如何
        \item 睡眠质量如何
        \item 消化能力如何
        \item 是否 SY
        \item 心理是否健康
    \end{itemize}
\end{multicols}

影响身高的因素一般就这八个方面,很多文章只谈到五个方面,有的文章认为遗传占主要因素,有的文章则认为遗传只占 30\%,我比较倾向遗传只占 30\% 的论点,因为我就是活生生的例子,我发育期虽然频繁 SY,但我其他方面做得很好,第一项和第七项我得分低,也就是遗传和 SY 这两项我得分比较低,但是我其他六项得分都很高,我发育期经常打篮球,而且是户外阳光下的篮球,多晒太阳有助于增高,因为阳光的照射能促进人体合成维生素 D,有利于骨骼中钙质的积累、沉淀,使骨骼快速生长,机体快速长高。积极锻炼对骨骼的发育是非常有好处的,是会促进骨骼发育的,我也从来没有熬夜和久坐,我天性爱动,很少一坐几小时,一般九点半就睡觉了,睡眠质量很高,消化能力也很强,吃得下,睡得好,很多人能吃,但消化不行,吃下去也不吸收,所以消化能力非常重要。另外,心理健康也很重要,中医讲:七情致病,情绪会导致疾病,会导致内分泌紊乱,这样也是会影响到身高的。90 后的戒友如果你想长高最好在这八个方面好好下功夫,SY 当然应该要戒掉,这样更有利于长高,戒掉 SY,长出来的骨密度也高。

\subsubsection{脱发问题}

下面谈下脱发问题。

脱发问题一般是 SY 十年以上的戒友会遇到的问题,80 后的戒友遇到脱发问题可能性比较大,导致脱发的因素也是有很多的,SY 导致的肾虚型脱发就是比较普遍的原因。脱发问题是非常愁人的问题,因为头发对于一个人的外貌至关重要,出现秃顶倾向给人的感觉就是未老先衰,精气神不行,很多戒友每天掉一百根以上的头发,看到梳子上和脸盆里的头发,人会产生一种恐慌感,对于秃顶的恐慌,心乱如麻,对头发会变得异常敏感、异常在乎,掉一根头发都会让人神经质好几个小时,生活在一种担心、恐慌的情绪中,这样的心理又会加重脱发,因为中医:恐伤肾,发为肾之华。所以就会陷入恶性循环,到处找治疗脱发的药物,但不管何种药物效果都不会理想,于是心理压力会变得很大,很在意别人的眼光,也害怕照镜子,一看头发又少了,连自杀的心都会有,脱发问题就是让人处在一种心理困境之中,无法自拔,而很多人会把压力转变成 SY 行为,这样头发就会掉得更厉害,最后真有可能变成“地中海”。

很多戒友的发际线已经开始后移了,有人则是头顶先秃,我以前就是头顶先秃,有时一天能掉几百根头发,一梳头发,肩膀上就几百根,太恐怖了!然后这样掉了一个月,头顶就有点看见头皮了,好在当时我已经看清无害论的真面目,并且对中医医理有了比较深入的理解,于是我赶紧戒掉 SY,积极锻炼,按时作息,慢慢把肾气养足,差不多花了半年时间,我掉头发恢复到每天五根以内,现在梳头发,梳子上看不见有头发,枕头上也很少有头发了。

我现在更不敢 SY 了,因为我深知危害之惨烈。很多戒友处在无知状态,无知者无畏,对 SY 导致的恶果认识不深不全面,甚至不少戒友还迷在无害论的泥潭里,脱发了也不知道真正原因,继续 SY,把压力变成 SY 行为,这样只会越陷越深,陷入恶性循环,头发就更难恢复了,因为认识上有盲区,有误区,这样使用任何药物都不会有多大用处,永远无法治根,肾气不养足,头发问题就无法解决。就像一盆植物,外行看到叶子枯萎了,就会认为是叶子的问题,于是把叶子摘掉,但是没过多久,叶子全枯萎了,植物就死掉了,换做内行,一看就知道是根的问题,马上打开土壤,在根上用药,这样植物才会继续存活,头发也是这个道理,一定要认识到根上的原因,不能迷在表面,靠几个所谓的产品,吹得天花乱坠的产品,就能治愈脱发,那是不现实的。即使你去植发了,只要肾气不足,依然会继续脱发,除非你定期去植发,掉了就植,那样费用太大了,一般人用不起。

导致脱发的因素如下:

\begin{adjustwidth}{-1.5em}{-1.5em}
    \begin{multicols}{3}
        \begin{description}
            \item[遗传因素] 父母
            \item[雄性激素分泌过高] 雄秃
            \item[熬夜久坐] 伤精伤肾
            \item[纵欲过度] 肾虚导致的肾秃
            \item[饮食习惯] 吃得太咸容易脱发
            \item[心理因素] 压力过大脱发
            \item[疾病因素] 很多疾病导致脱发
            \item[清洁因素] 头皮屑堵塞毛孔
            \item[季节性脱发] 一般夏季容易脱发
            \item[营养性脱发] 严重营养不良导致脱发
            \item[物理化学脱发] 外在刺激影响脱发
        \end{description}
    \end{multicols}
\end{adjustwidth}

\subsubsection{痤疮}

再来谈下青春痘痤疮。

很多人认为过了青春期,痤疮就会自己好了,其实这种看法是错误的,我见过很多人三十多岁四十多岁,依然一脸粉刺痤疮青春痘,那其实已经不叫青春痘了,就是内分泌紊乱造成的,不是你过了青春期就能好的,一般的痤疮,只要注意休息和饮食清淡,很快就能好,但是 SY 导致的顽固性痤疮就很难好了,非常顽固,必须戒掉 SY,积极锻炼,养足肾气才有望痊愈,否则那种五脏失调的状态可能会一直持续下去,伴随终身,脸是毁容了,彻底废掉了,精气神更别谈了,自卑至极,一脸脓包粉刺,像个怪物,惨不忍睹。和发育前那种清爽肤质,一个天一个地,一 SY,就出油,去医院就说痤疮,溢脂性皮炎,开药调理,时好时坏,因为依然在 SY,只要还在 SY,吃药就无效,只能暂时缓解,无法去根。不少戒友都变丑了,变丑的占绝大多数,因为 SY 导致肾气亏损,肾气一走漏,五脏功能就容易紊乱,五脏一失调,就会表现在脸上,因为脸是五脏的镜子,脸的气色就会不行,感觉萎靡猥琐,得上痤疮的可能性就会很大。

总之,要彻底治愈顽固性痤疮,必须戒掉 SY,养成良好的作息饮食习惯,积极锻炼,这样才有望恢复健康肤质。否则,免谈。

另外,有的人虽然没有得上顽固性痤疮,但脸部的气色感觉像鬼一样,没有阳光的感觉,只有晦暗,甚至会出现凹陷,感觉就像骷髅包了一层皮,没有生气没有活力,未老先衰。年纪轻轻纵欲过度,把自己搞得人不人,鬼不鬼,就像泄了气的皮球。那种感觉就像旧社会吸鸦片上瘾的人,网瘾 + 烟瘾 + 性瘾,这样的身体能好吗?所以戒为王道,不戒就是在害自己。有的人身体好,报应出来就晚,但只要在纵欲,迟早会出来的,没有人可以逃得掉。不注重养生,身体就特别容易出问题,人的身体就像机器一样,需要保养,不保养就容易磨损折旧,好比一辆新的自行车,注意保养,一年后还可以九成新,但如果你不保养,一年后可能只有五成新了,或者已经坏掉了。这就是保养意识的重要性,不注重保养,不会长久。

\subsubsection{戒断反应}

最后再来谈谈戒断反应。

我研究过很多成瘾行为,包括酗酒、抽烟、吸毒、网瘾、性瘾、购物上瘾等,凡是成瘾的行为要戒掉几乎都会出现戒断反应,有的人表现严重,有的人表现轻微。而 SY 导致的性瘾,戒断后也会出现戒断反应,绝大多数戒友在戒掉后都会出症状,而 SY 时却没表现出来,从戒友的发言来看,绝大部分的戒断反应表现为泌尿系统的问题,以前列腺炎为主,其次为睡眠障碍和烦躁情绪等。

戒断反应可以这样理解,和中医的排病反应有相似之处,很多中药吃下去会有排病反应,就是反而严重了,这其实是好现象,是正邪在交战,只要继续吃药,正气就会占上风,这样疾病就会慢慢痊愈,很多人不懂得排病反应,感觉严重了,就以为吃错了,就不吃了,这其实就是耽误治疗了,古语有云:为人子弟不可不知医,因为你知道医理了,才能更好地配合医生治疗。

戒断反应的机制在西医的研究来讲就是:由于长期 SY 后,突然停止引起的适应性反跳。例如酒精戒断后出现的是兴奋、失眠,甚至癫痫发作等症状群。吸烟者在强制戒烟之后也可能会出现诸如焦躁不安、失眠、食欲增强、吐黑灰色痰、血压升高以及心律不齐等戒断反应,会产生极大的痛苦,但是这种反应大多数会随着体质的恢复而逐渐消失。一般 SY 戒断后,出现戒断反应的时间为一个月内,有的人戒了几天就会出现戒断反应,出现戒断反应不要害怕,积极锻炼,注意休养,戒断反应会自动消失的。
