\subsection{戒色德训篇}

\paragraph{前言}

这季前言分享一些案例。

\begin{case}
    我昨晚差点死掉!昨天几乎看了一天的黄,射了四次,第一次射的精有点黄,第二、第三次射的就像水一样透明的,这时感觉很累,有轻微窒息感了,然后晚上九点多又射了一次,这一次真的把我给吓死了,其实我当时已经看黄没啥感觉了,我必须射出心魔才肯罢休,我抛开一切后果在做最后的冲刺,在射出的一瞬间我感觉大量的能量从我身体里跑出去,射完后我的双腿在不停发抖,身体总是不自主地抽动,这一次射的精很黄,很厚。感觉把身体里所有的精华都射了出来,然后颤颤巍巍地去尿了尿回来睡觉,这时感觉呼吸困难,喘不上气,有一种随时都可能猝死的感觉,脑袋眩晕。我以前也好几次这样,要不是我一直在吃焦虑症的药,我估计还会更糟,我想我第一次惊恐发作最后被诊断为焦虑症也是因为像这样疯狂 SY 造成的,所谓毁灭之前必疯狂。第二天也就是今天感觉手脚冰凉,浑身乏力,呼吸不畅,关节经常响。
    \subparagraph{附评} 这个案例也让我想起了我过去疯狂手淫的经历,那种丧心病狂的状态,不要命地撸,撸不出,还在疲于奔命,做最后的冲刺,已经软下来了,还不死心,必须射出来才算完成任务。最后终于射出来了,我也快倒下了,终于完成任务了,长吁一口气,其实后来已经感受不到快感了,JJ 都是麻木的,只是为了完成“任务”,只是想快点结束。被心魔附体,不彻底掏空决不罢休,完全就是作死的节奏,非常可怕!真的撸到了形神俱灭,半只脚已经踏进棺材了。我记得那时我一天多次之后,人都站不稳了,走路都是颤抖的,下楼梯都要扶墙,生怕摔倒,脸上的气色暴差,出现蜡黄色,整个人病恹恹的。我记得那时必须找新的 H 片,才能一次又一次调动起我的冲动,看过的会差很多,兴奋不起来,必须是新的,那个年代很难找新的,而现在网络时代,新鲜的刺激太多了,所以很容易造成连续疯狂手淫的情况,现在有手机,找黄很方便,连小学生都在看黄了,色情已经开始荼毒小学生了……这方面家长一定要格外注意,不能让孩子接触黄源,色情的危害真的太大了。记得过去每次疯狂纵欲过后,就是症状的大爆发,尿频和腰痛把我折磨得够呛,那时我才十几岁,还在发育期,已经得上前列腺炎了,弯腰一会就腰疼得直不起来,撸后鼻炎也加重了,经常感冒,鼻子不通气,特难受。这个案例的戒友说自己差点死掉,一天射四次,几乎是看了一天的黄,这个耗损程度实在太厉害了,真的有猝死的可能。疯狂 SY 是可能导致惊恐发作的,之前分享过一些这样的案例,都是连续放纵,过了一个临界点,那种不详的感觉就来了……\textit{夫精者,身之本也。(《黄帝内经》)} 肾精是人体的宝贝,是身体的老本,老本是不能随便往外拿的,一定要好好珍惜它,守护它。“精足则神明,精夺则神衰”,懂得保精养精,才能内气充沛,生机不息,精神饱满,神采奕奕。古人强调的是“积精全神”,要懂得积攒肾精,不能随便耗泄,要知道射掉后,身体感觉就会差很多,严重的话会导致各种症状爆发,有的昆虫交配后是会死掉的,因为赖以养命的能量被消耗了。古人很有保精意识,知道肾精的宝贵,今人多被无害论洗脑,又身逢色情泛滥的时代,这样就很容易疯狂放纵,放纵之后身体就很容易垮掉,年纪轻轻,十几岁、二十多岁就垮掉了,得上神经症,生不如死,这类案例数不胜数。

    一位小戒友的反馈:\begin{quote}\it
        出来撸迟早还是要还的!最近我不是疯狂放纵每天一两发么?最近感冒也没有休息。前天打了两发,昨天吃完感冒药后就有点虚弱感,晚上八点我躺下休息了。虽然是躺着但是内心总有种焦躁不安的感觉,突然那种不适感爆发了,心里说不出的恐惧,我当时心律失常,呼吸急促严重缺氧,四肢发冷发麻,脸部没有知觉思维混乱,马上快死了的感觉,非常恐慌,如果不是我蹲下缓解一点,我当时就叫出来了。一年前的这个时候第一次爆发了这种症状,我知道,这是撸海中躯体症状最恐怖的濒死感,我才十五岁,就要受到这种症状的折磨!唉,一把辛酸泪,希望老师安慰一下我,也希望老师分享我的案例,警示各位师兄!我主要还是因为控制念头不行。
    \end{quote}

    容易导致惊恐发作或者猝死的情况:\begin{multicols}{2}
        \begin{itemize}
            \item 劳累时手淫或者同房
            \item 连续两次以上,耗损太大
            \item 生病时手淫或同房
            \item 酒后手淫或同房
            \item 吃壮阳药纵欲
            \item 熬夜手淫或意淫
            \item 持续时间过长
            \item 身体虚弱时纵欲
        \end{itemize}
    \end{multicols}

    看过一个案例,就是在外喝了不少酒,并且已经很累了,回家和妻子同房,结果猝死。我们一定要警惕,身体状态差时一定要慎重,因为此时纵欲的后果实在不堪设想。疯狂过后就是满目疮痍,一场大祸也许就要临头了,这不是闹着玩的,关乎性命安危,一定要吸取前人的惨痛教训,否则很可能会再走前人的痛苦老路。在这个色情泛滥的时代,真的要有戒色保精意识,诱惑实在太大,很容易进入打鸡血的放纵状态,到时身体垮掉就苦大了。
\end{case}

\begin{case}
    本人已戒色 230 天,遥望过去,看黄手淫,形容猥琐,身力憔悴,焦虑自卑,与家人吵架,大学挂科,尿频尿急,频繁遗精,辍学休养,在家躺床废人一个,到开始接触《戒为良药》,潜心学习,努力戒色养生,到现在脑力恢复,身体恢复,可以出来社会工作,充满信心,内心光明、平静而强大,倍感现今的脱胎换骨。感谢戒色吧,感谢飞翔哥,感谢所有无私奉献于戒色宣传的人!
    \subparagraph{附评} 这位戒友蜕变了,230 天,躺床废人崛起了,从一个猥琐憔悴的撸者蜕变为充满自信、内心光明强大的人,真的非常棒!躺床废人是看不到任何希望的,每天只有深深的绝望,还连累家人,家人每天也唉声叹气、愁眉苦脸。通过戒色养生,身心脑力都渐渐恢复了,又可以出来工作了,又有了奋斗的底气与勇气了,这是脱胎换骨的变化。躺床废人是多么糟糕的状态,230 天过去了,废人变新人,重生了,蜕变了,充满朝气与能量,又开始感受到生活的美好了,生活又有了憧憬与希望。看到大家蜕变,是我最高兴的事情,拯救一个人,无异于拯救一个家庭,宣传戒色,助人戒色,真的是一种很大的善举,希望大家一起来传递正能量,一起帮助更多的人更多的孩子,让他们认识到色情与手淫的危害,普及戒色的理念,这真的是在拯救他们。沉迷邪淫就是在积累负能量,到时身体、家庭、学业、事业都会受到很大的影响,能量场变得无序混乱富有攻击性,各种负面的念头,这样就会感召各种不顺的事情。戒色后懂得修心,断除负面的念头,多发善念,感恩的念头,孝顺的念头,助人的念头,这样场能就能重新变得有序,从负能量转变到正能量,生活就能顺很多,身体好了,也有本钱去奋斗了,否则一身的病,即使工资再高,也很难干得下来,因为身体扛不住。所以一个好身体实在太重要了,我看过很多戒友的文章,都感叹身体好太重要了,身体一垮,各方面都会陷入困境。纵欲会让栋梁之才变成朽木不可雕也,就像被色情的白蚁蛀空了一样,随时都会垮掉,而戒色是在让朽木重新变得坚实起来,变成真正的顶梁柱,撑起整个家庭,撑起你的人生。当你能量养足后,你会发现自己是可以撑起来的,当你被色情蛀空后,你会发现即使自己想撑起来也撑不起,有心无力,即使能勉强撑起一小会,也扛不了多久,压力一大就垮了。
\end{case}

\begin{case}
    飞翔哥我最近到深圳这边,想要在这边发展事业,刚开始没有什么资金,就到工厂里面打工,每天十二个小时,每天上班就听《戒为良药》,我发现这几天听着听着就哭了,或许我仅有的一些善根发芽了,以前的我就是麻木不仁的撸囚,从来不懂得感恩,经常跟家里人吵架,脾气不好,自从戒色后好几年没有见的同学都说我现在满满正能量,而且脾气也特别温和,而且我明显感觉,自己身边越来越多的贵人在帮助我,或许这就是戒色的福报吧!而且我现在每天五点半起床练功,每天只睡六个多小时,中午只休息十五分钟,工作十二个小时一点都不觉得累,而且每天精力旺盛,对未来充满了无限的向往以及规划,这些都是我以前不敢想的。飞翔哥,这一切都是您在默默帮助我,帮助一个年轻的生命没有凋零,如果没有飞翔哥我现在不知何去何从,应该还在地狱的轮回中受苦,飞翔哥您说过一灯燃千灯,我现在终于懂了,是您帮助我脱离苦海,我也应该去帮助更多人,让他们的生命也彻底亮起来。我去年见到我初中同学,他真的变得面目全非,脱发,虚胖,以前挺帅的,我那时只是引导了一下,没有说明白,但是我昨天想他还那么年轻就被手淫毁了,我晚上下班果断打电话告诉了他,也算种下了种子了,我瞬间感觉全身特别舒服,身心愉悦,这就是行善的力量吧!而且工厂里也有很多邪淫的人,我发现我现在都可以看面相了,我慢慢从身边人引导,飞翔哥我特别特别感恩遇到您,遇到戒色吧!让我生命亮起来,我也应该帮助更多的人。
    \subparagraph{附评} 一灯燃千灯,度人即度己。过去的我也是麻木不仁的撸囚,戾气很重,也和家人吵架,后来戒掉了,学会修心,学会行善积德了,我发现这样的生活才是我真正想过的。无私帮助别人也让我感到莫大的快乐,快乐不是超过别人,不是比别人强,而是帮助别人回归正轨,帮助别人解决困难,当你学会无私奉献、无私利他后,即使别人超过自己,也会由衷地感到高兴,感到快乐,我希望每一位戒友都能超过我,我垫底也没事,我觉得只要大家蜕变了,逆袭了,我垫底也是心甘情愿的,也是幸福无比的。一个无私的人是绝对幸福快乐的,只要有一点自私狭隘的想法,内心就会很痛苦。撸者脾气容易不好,这很普遍,肾精亏损,脾气就会变得暴躁,烦躁易怒,看人不顺眼,戒色修善后,人就变样了,内心祥和稳定了,脾气就变好了,看人也顺眼多了,懂得包容别人了,人际关系也变得好起来了,爱笑了,会开玩笑了,轻松了。一个烦躁的人就像一只刺猬,到处刺别人,一个祥和的人就像一缕春风,给别人带去美好愉悦的感受。这位戒友戒色养生后,精力恢复了,每天工作十二个小时都不觉得累,这种精力旺盛的状态是奋斗事业所必须的,不少撸者干不成事,为什么?脑力、精力暴差,干一会就累,干一会就坐不住,这样怎能成事?好的精力状态太重要了,精力这两个字很有意思,有精才有力!精亏损了,整个人的状态就会下降很多,很多事情就无力应对了,再大的雄心壮志也会半途而废,因为精力跟不上,事情做不好,坚持不下来。一位戒色 360 天的戒友说:“以前我也可以表现很好,就因为撸管,精力不济,导致错过大量的机会。这次我把握住了,有了雄厚的精力,就有了把握自己未来的能力。”有精力,才有未来,没精力,机会再多,也把握不住!其实我们身边有很多撸者,有些人也不愿意承认,毕竟不是什么光彩的事情,但在合适的时机,我们应该让他们知道戒色吧,知道戒色这件事。从外貌衰败的角度来劝是比较有效的,比如脱发问题,中医讲发为肾之华,手淫伤肾会导致脱发,把这个知识点一说,对方就听进去了。看到别人气色不好,双眼无神,黑眼圈眼袋,也可以半开玩笑地说:“是不是最近撸多了?”然后再点出撸管的危害。有的人听到了,虽然不马上开始戒,但到了一定时间,他就开始戒了,症状会突然敲醒他的,我们让他们知道有戒色这回事即可,我们负责播种,何时发芽,是他们自己的事。让我们一起帮助更多的人,点亮他们灰暗的人生,照亮他们的生命,成为别人的指路明灯。
\end{case}

\begin{case}
    飞翔大哥好!在遇到戒色吧之前,我听过不少开示,可能是我悟性不够吧,我那时也一直以为念佛是为了增加福报,甚至还以为念佛求阿弥陀佛来。直到一天无意中发现了戒色吧,看到了《戒为良药》,当我看了断念的部分,才明白念佛也是为了断念。文章中又提到了元音老人,我就去网上找了老人的视频,我随便点开了一集,没想到那集就是介绍念佛,老人说妄念一来,就用佛号在心头一转,心就空了,佛号就像一根拐杖,帮助我们空心,而不是求佛来。那一刻我更加认识到念佛的作用,老人在那张大家很熟悉的靠椅上缓缓道来……在看老人的视频之前,我不太清楚什么叫慈悲、慈祥,这些词,我以前只在书上看到,生活中没有太大体会,看到元音老人的视频后,我被他那种慈祥、慈悲而又正气浩然的强大气场震撼了,而且老人又是那样乐观、平易近人,无论来访者贫富贵贱,他都一视同仁,不卑不亢,而又宽和近人,值得我终身学习。感谢飞翔大哥一直以来的无私奉献,飞翔大哥很多文章,我感觉不仅适合于戒色,还可以适合生活中很多事。
    \subparagraph{附评} 这位戒友看的是元音老人上海寓所开示,那个系列有 28 集,应该是元音老人所有视频开示中最经典的一个系列了,真的太殊胜了,我当初第一次看元音老人的开示,也是看的这个系列,当时就被深深震撼到了,那个慈悲恳切的声音响起,内心莫名有一种很深的感动,而且有一种很强的直觉,那就是觉得元音老人说的是真理,我如饥似渴地把那 28 集给看完了,受益匪浅,百看不厌,其中很多开示我看了一遍又一遍,非常经典,非常震撼,老人的一言一行一个手势乃至一个表情都有很强的感染力,我被那种强大的气场给彻底折服了,老人在那张靠椅上进行开示,那张靠椅都有了某种特别的感觉,那个场景感人至深。这个系列是在老人的家里拍摄的,所以非常具有亲和力,有家的感觉和氛围,老人的慈悲、慈祥、和蔼、亲切、乐观、幽默、平易近人让我如沐春风,老人非常随和,就像邻家的老爷爷,充满智慧,充满慈悲,充满幽默,哈哈大笑。我对元音老人很有感觉,我也喜欢元音老人给我的感觉,那种感觉魂牵梦绕,我对元音老人有一种崇高的情感,一种深深的敬爱,一种泪流满面的感觉。能够得遇元音老人真的很有福报,绝对是泰斗级的大德,本焕长老也推荐大家学习元音老人的开示,丁愚仁老师也推崇元音老人,我的断念理论其实就是师承于元音老人。修行的核心是修心,不管何种法门,都是为了把自己的内心修空、修净,断念实战时可以一觉即空,也可以思维对治、念佛持咒等,最好的操作就是一觉即空,当然思维对治和念佛持咒也很殊胜,各有千秋。学佛不能停留在迷信的层次,普通人求平安、求福报,这类是比较多见的,而有的修行人求见光、求见佛,弄不好会走火入魔。修行是要从自己心地上用功的,要真正理解念佛的作用,把心地扫干净,这样修行才相应。
\end{case}

\begin{case}
    飞翔大哥,我之前看过的色情图片会出现在脑子里,那种图像一出现不容易断掉,生殖器官立马会出现不适(就感觉有东西堵住了,不知道是不是暗漏了?)我断念的方式:综合用的,断念口诀、念佛持咒和思维对治,都在用。但是有时候那种图片上脑极快,断得很慢,下体很不舒服。你说我该如何是好?
    \subparagraph{附评} 前段时间一位戒友在我帖子里也问了图像上脑的问题,看到的诱惑图像会在短期内反复出现,印象深刻的图像有时也会浮现出来。图像上脑极快,一瞬间就攻上来了,关键还是警惕,对于图像这类进攻方式要格外警觉,图像一上来,马上觉察消灭,也可以念佛持咒等,反应一定要快,慢一点图像就开始在脑海里变成小视频了。我也经常被图像攻击,有的图像还没完全成形,就被我断掉了,鉴于图像容易引起强烈冲动,一般都是及时觉察加上改变姿势,坐姿变站姿,然后来回走一下,这样就能很好地消除那种残留感和身体的不适,也可以配合做一下缩肛和固肾功。实战时,如果反应慢一点,睾丸就会松,再慢一点,不适感就出现了,再慢一点就可能漏液体了,漏液体后身体更难受。我戒到现在,经历了成千上万次的断念实战,睾丸松过,不适感也出现过,漏液体一次都没有过。即使高手也不是每次实战都能做得极好,就像世界级运动员也不可能每次发挥都是超一流水平,状态总有起伏的,关键是自己要善于总结,不断练习,这样下次的实战表现才能更好。图像的攻击方式是非常常见的,关键反应要快,要高度警觉,经历过一些实战后,慢慢就熟悉了这种攻击套路,到时就能沉着应对了,自己也会有一些自己的实战经验,下次再遇到这种情况,就能很好地处理了。来果禅师:“五根本中,唯有淫戒极为易犯,特将拈出,以明止法。如眼见淫色,耳听淫声,当时淫心兼身俱动;才识将动,急快反观,就处一觉,一觉不止,切须再觉。何以故?将起即觉,则觉力自胜,淫念易止,若起多时,已发身口,其淫力胜,故难觉除。由是之故,念起即觉。”大德都在说这个“觉”,念起即觉,觉之即无,一觉即空,戒色一定要学会观心,学会觉察,观心是基本功。要觉得快,觉力要不断强大,这需要一个坚持训练和培养的过程,刚开始念起不觉,跟着念头跑,跑了一段才发现自己在意淫,这时已经欲火中烧,欲罢不能了。后来通过不断学习和训练,就能做到念起即觉,觉察力变得越来越强,到时就可以主宰内心了。
\end{case}

\begin{case}
    飞翔哥,我体会到纯净的大快乐了!作为一名大三考研党,自己已经记不清自己多久没破戒了,自从每天坚持学习飞翔哥以及戒色前辈的精华帖 45 分钟,目前戒色稳定,内心干净。每天在图书馆学习,今天下午到了饭点就离开图书馆前往食堂,于是在去食堂路上不知所以地开心,包括吃饭、吃完饭后散步直到回到图书馆后都一直有一种莫名其妙的开心,天呐,那感觉太美好了,就是想笑,就是开心,没有任何理由的快乐。现在想想还很开心,原来这就是我孩童时期的快乐呀,多么美好,多么纯洁,原来前辈们说自从我们长大了,纯净消失了,于是纯净的快乐也不复存在,这些理论都是真正无疑的,因为当我体会到纯净的大快乐后我才体会到这种美好的美妙。感恩,我会再接再厉,也希望各位戒友齐心努力,与邪淫战斗到底,邪淫一点都不酷。
    \subparagraph{附评} 天然的喜悦不需要任何的理由,因为你就是喜悦本身,你整个就是一个开心果。小时候的神奇感觉,我记忆犹新,那时我内心很纯净,所以有一种很快乐的感觉,那种喜悦不是来自物质,不是来自玩具,不是来自性,多年以后,我才真正领悟到这是天然的喜悦,不需要任何外在的物质,不需要性,就是纯净的存在,纯粹的快乐。一个人一旦再次恢复纯净,他就能找回这种神奇而美好的感觉,的确很美妙。在手淫后,这种感觉就会消失,在多巴胺的升升降降中,整个人都迷失在色欲里了,完全被快感冲昏了头脑,这时会误以为快感就是快乐,不惜一切代价地疯狂追求快感,其实快感并不是真正的快乐,真正的快乐不需要理由,真正的快乐不需要外在的条件,真正的快乐是内心时刻泉涌的新鲜喜悦,它来自于内在,而不是外在的刺激。很多人在手淫几年后,彻底没了笑容,一张僵尸脸,怎么也挤不出一个笑容,即使勉强笑起来,也显得特别猥琐,曾经的纯真笑容完全消失了,相由心生,内心是龌龊的,不可能有纯真美好的笑容。邪淫的确一点都不酷,猥琐得很!那个猥琐的身影在疯狂,在抽搐,在空虚,在悔恨!在痛苦!在惶恐!在挣扎!在绝望!在哭泣!真正酷的是戒色,极具智慧的酷!酷到骨子里了!表面的酷是肤浅的酷,做作的酷,装腔作势的酷,缺少智慧,缺少内涵,缺少深度!你不可能一直保持喜悦的状态,除非你变成喜悦本身,那就时时刻刻在泉涌着喜悦,变成一口喜悦的活泉。到那时,你就不需要从外在寻找喜悦了,因为你本身就是喜悦,你时刻散发着喜悦,你把喜悦的感觉带给身边的每一个人,别人也会被你喜悦的感觉所感染,也会跟着一起喜悦,就是这么奇妙。就是想笑,就是开心,没有为什么,不需要任何理由,因为你就是喜悦本身!你已经领悟了快乐的真谛!一位戒友说:“现在感觉到,其实接触大自然比接触黄更快乐,那种太阳生起的信心、站在高山上的豪迈、看到树木的宁静、柳树垂条的摆动、湖波荡漾的感觉、嫩绿的小草、奇花异草的绽放等等,有时候一阵清风扑面而来,都能特别欢喜和满足,这在接触黄的过程中是无法体会的,有时候我们真的要多学习孩子,他们那个时候先天意识还比较强,是那么纯真善良美好!”戒掉之后,就会发现之前沉迷邪淫时所忽略的美好,邪淫时对这种美好是麻木的,是缺少感知的,戒掉之后这种对美好的感知又会回来,看着花花草草都感觉特别美好,特别幸福,看见鸟儿飞过,都会感到特别喜悦和自由,情不自禁地开心。这种感觉真的是看黄手淫时无法体会的,这才是真正值得拥有的感觉。
\end{case}

\begin{case}
    飞翔哥您好,今天我在看您写的《戒色十种力量》时突然有一种感觉,就是特别开心,特别欢喜,感觉看到的、想到的一切都和山泉水一样清澈。以前看到前辈们说戒色是飞龙在天的大爽,总觉得未免言过其实,但我今天真的感觉到了,那一瞬间,真的是发自内心地觉得幸福而快乐,戒色原来真的可以这么爽!感谢飞翔哥,感谢各位前辈,感谢当年不怕被辱骂嘲讽坚持宣传的英雄们!你们都是活菩萨!没有你们我也许永远也不会知道这世界居然如此美丽!相比而言邪淫就好像垃圾一样!我会一直戒下去的!拼死戒,戒到死!!
    \subparagraph{附评} 这位戒友也感受到了,戒到一定时间,会有一瞬间进入那种感觉,感受到飞龙在天的大爽,这位戒友说“戒色原来真的可以这么爽!”其实前辈的文章都是过来人的深刻体会,戒色的确可以这么爽。蛆拱屎是一种爽,低级的爽,飞龙在天是另外一种爽,高级的爽。有低级的爽,肯定有高级的爽,低级趣味带来的是低级的爽,净化心灵带来的是高级的爽,世人多钻在低级的爽里头了,所谓:“大可笑,大可笑,好汉多迷屎尿窍。”他们不知道还有高级的爽。喜悦到爆,那是怎样的感觉呢?整个人充满喜悦开心的感觉,好像爆炸一般,这种狂喜没来由,不是因为外在的刺激导致的,就是内心泉涌的感觉,就是开心,就是爽!看看小孩,他们很多时候都会笑,你不知道他们为什么笑,他们自己也不知道,就是喜欢笑,内心就是开心快乐,这就是天然的喜悦,非常珍贵的体验,成年后基本都丧失了,因为大多数成年人都掉进了欲望的深坑,开始丧失天然的喜悦,转而追逐外在的东西,他们误以为外在的东西能让自己开心快乐起来,然而最后体验到的只是沮丧和失望,任何物质在得到的一小会也许会开心一下,但转头就消失了。很多有钱人、开豪车的人,你看看他们的表情,基本没有什么笑容,板着一张脸,显然不是开心快乐的状态。恢复纯净,才能真正快乐起来,钱、性、物质都无法让一个人真正快乐起来,只有恢复纯净心灵,行善积德,才可以。

    一位资深戒友在帖子里写道:“自从我接触到戒色以后,我翻出以前的照片看了看以前的样子,几乎每一张照片都没有笑容,冰冷着一张脸,非常难看,每一张照片的眼睛都非常小,眼皮耷拉着双眼无神。我记得当时女朋友对我说,让我和父母照相的时候笑一笑,我这个样子他们看了难受,我当时真的感觉笑不出来,心里很沉重的样子,每天都不快乐,我现在才明白是因为我沉迷于邪淫,整个人陷入巨大负能量,处于低频率振动,低频率振动是沉重压抑的,因此我的内心感到压抑沉重,感觉不到快乐,需要用刺激来获得快感。就像是现在街上的那些所谓的美女,整天冰着一张脸一副高冷的样子,自以为像是女神一样,其实这些人大多数都是因为过于追求物质、虚荣,丢掉了纯真,整个人处于低频率振动,内心沉重压抑根本快乐不起来,需要衣服首饰各种消费等这些外物的刺激获取快乐,很可悲啊!相反真正内心纯净的人,不管走到哪里整天都是笑嘻嘻的像小孩子一样,内心纯净的大快乐如泉涌一般,根本不需要外物的刺激。而我现在有时感觉到莫名的开心,像是小孩子一样,照相的时候笑容也灿烂,面相和善,而且眼睛变得很大,以前努力睁,都没有这么大,像是开了眼角,变长变大,有了精神,眼白以前布满血丝,现在渐渐地清亮了起来。我遗传继承了父母外貌,自小相貌清秀,后来有些长残,戒到现在,重新恢复了神采,常常会听到关于外表的赞扬。毕竟我不再年轻,加上学习传统文化以后,我不再追求外表,但是这是对我戒色成果的肯定,我觉得大家表扬的更多的不在我的外表,而是我的精气神!戒色真的可以让你提高颜值,充满精气神!”

    充满正能量、有精气神、内心纯真的人,自然就是快乐的,因为他们处于高频振动的模式,变成了快乐的源泉。外在的刺激所带来的快乐是短暂的,而且还可能会让欲望不断增大,越来越难以满足。一个欲望的满足往往会导致更多欲望的产生,于是疲于奔命,不断追求虚假的快乐、虚无缥缈的快感,把快乐建立在性与物质之上的人,最后都迷失了自己,肯定会进入低频模式,变得越来越不快乐。真正快乐来自于净化心灵,培养自己的正能量!
\end{case}

\begin{case}
    看了很多飞翔老师的文章,也按照飞翔老师所说的去做,发现很多东西真的只有去亲身去做,你才会发现老师话语的含义。就比如我最近不断地帮助戒友答疑(挑我会的回答),鼓励戒友,我的心境也发生了一些细微的的变化,尤其是此时此刻它还是充满欣喜与雀跃,根本停不下来,这种久违的感觉就像小时候得到自己喜欢的玩具那样的开心满足。说真的我还不能完全地形容出来,只能去慢慢地体会这种感觉给我带来的回馈,真是一种大快乐,大爽,原来戒色的生活是这么的美好。我在帮助别人的同时,我自己也很开心、满足,有时候别人仅仅是一句简单的“谢谢”,我觉得没有比这个更好的馈赠了。所以我决定按照飞翔老师所说的尝试不断地去发一个个善愿,去帮助更多的人,让他们能有更加美好的人生,而不只是邪淫所带来的灰暗人生,让他们知道世界的美好,让他们热爱自己的生活,热爱身边的事物,珍惜身边的每一个人,去帮助他们,将善良这颗种子播种到更多人的身上,让这种美德延续下去,让生活更加的美好。得遇老师无比感动,思之泪目,无限福报,祝老师身体健康,事业更加顺利,生活更加美满。也祝戒色吧的各位师兄师弟们早日摆脱邪淫的束缚,找到属于自己的美好的人生。
    \subparagraph{附评} 有四个字叫“为善最乐”!这四个字很好,我自己的真实感受就是净化心灵可以带来快乐,安住真我会带来不可思议的美妙快乐,行善利他也可以带来很大的快乐。我再说两句,“邪淫最苦”、“自私最苦”。邪淫其实就是一种自私,要自己爽,非常自私的贪心。自私是在强化小我(ego),最后肯定感召苦果。真正有智慧的人懂得无私奉献,无私利他,无私是真我的品质,智者的核心是无私的,这点很重要,有的人虽然也行善,但他的核心还是自私的,要求福报,想怎么样,都是围绕那个自私的核心。这样行善就大打折扣了,只有核心是真正无私的,这样行善才能带来真正的大快乐,自私行善之人肯定患得患失,看不到效果也会怨天尤人,甚至会后悔。

    前辈的很多话,只有亲身去做了,才能真正明白前辈的意思,就像断念口诀,只有自己练到一定程度,才能真正深刻地理解前辈强调的觉察是什么意思,是什么实战感觉,只有反复训练,才能真正彻底地把握。我最近有了一次顿悟,这次顿悟让我欢喜雀跃,其实就是对一个看过无数遍的简单知识点的深刻理解,过去看了很多遍,都没有很特别的感受,当我不断练习观心,不断练习安住真我,几万次、几十万次地安住,终于有了特殊的领悟,这个领悟就是:念头遮蔽了真我,我的工作就是清除念头的遮蔽,让真我时刻显现。其实关于这个开示,我看过很多大德都开示过,但并没有特别的感觉,就是因为我没有练到那个程度,练到了自然对这句开示有了入木三分的体会、理解和感受,一定要练到,才能真懂,才能真正明白是怎么一回事。有了更深的领悟后,练习的感觉和质量又会上一个全新的层次。有的戒友说自己总是突破不了半个月,虽然看了很多戒色文章,也做了笔记,为什么突破不了怪圈?原因就是没有正确深入的理解和精深持续的练习,戒色不在于学习了多少,记了多少笔记,而在于吸收了多少,领悟了多少,真正落实了多少!当然学习和记笔记是非常重要的,但更重要的是看你的吸收、领悟和落实,有的人看了很多戒色文章,也记了不少笔记,但吸收率有多少?有没有转化成自己的实战意识?有没有坚持去练习?实战的表现是否有了质的提升?有的戒友总抱怨自己破戒,看看他的实战表现就明白了,看到诱惑不知避开,还是去点击,去聚焦,去盯着看,断念也不行,甚至还贪恋邪念,不肯断,这样怎么能行呢?一定要严格落实戒色十规,十规是围绕实战的戒色体系,实战强,才是真的强!实战狠,才能真正做到不破!才能真正守得住!实战就看对境和断念,一外一内,外避内断,任何戒色体系都绕不开这两点,任何戒色方法脱离了这两点,必败无疑!脱离实战,是不可能成功的!有的戒色方法可以成功一时,但要成功一世,必定要真正重视实战,强化实战!

    这个案例的戒友体会到了助人的快乐,他描述得很详细,也说得很好,一个人要懂得行善积德,懂得积累正能量,这样气场才能真正强大起来,气场强了,才能做成大事,你见过一个猥琐的人做成一番伟业吗?几乎没有,能做成大事的人必定要有正能量、德行的支持,要有魄力、精力去支撑他奋斗,被邪淫掏空身子的人,走路都费劲,何谈闯事业呢?帮助别人就是在强化自己的正能量,这点在很久之前我就反复强调过了,度人即度己,多发善愿、多起善念、多帮助别人,弯腰捡垃圾、让座的小善也不要放过,积小善才能成大善!告别邪淫负能量的状态,找回积极向上、充满正能量的自己,戒色最终就是要回归正能量的状态,拥有了正能量,才能拥有真正美好幸福的人生。反之,沉迷邪淫,沉迷色情,必将多灾多难,各种不顺,一个负能量的人感召的必然是负能量的人事物,这是必然的,完全符合科学规律,当你充满正能量了,你的处境和心境自然就转变了,你会开始喜欢正能量的感觉,而厌恶负能量的言行,厌恶负能量的事物,厌恶负能量的人。我戒色前也总说脏话,脏话是负面低频的词汇,戒色修善后,我不喜欢说了,一句都不喜欢,因为我知道,说一句脏话就会给我带来负能量,我不喜欢负能量的感觉,我喜欢正能量充满凛然正气的感觉。一个邪淫的人必定是充满负能量的,正能量已经被邪淫压制在非常小的范围,邪淫之人必定自私,戾气超重,脾气暴躁,这是必然的,因为负能量对应的就是那种状态。有了正能量就完全不同了,我蜕变了,也开始喜欢正能量的生活,以前沉迷邪淫,我以为那是正常的,其实那种生活危机四伏,有了正能量,我的心真正安稳了,祥和了,学习圣贤教育后,也知道要行善积德,我对我目前的生活状态是比较满意的,也希望大家都能懂得戒邪淫,懂得行善积德,学会对治负面的念头,为道日损,把负面的念头损之又损,最后变得极少,而同时,善念善愿又在每天不断地发,这样正能量就能迅速积累起来。当我发感恩的念头时,我很享受那种感恩的感觉,那是高频正能量的念头的一次洗礼,你会发现当你感恩时,你的心情是很愉悦的,当你做善事时,听到一声谢谢,你会很开心,会很快乐,很有成就感,也感到生活很有意义,很充实,这是行善的真正价值所在。你在无私奉献,其实帮到的正是自己,我希望大家共同来传递正能量。有句话说得好:“外在的快乐所带来的是一时短暂的刺激,感官上虽有享受之乐,但未必能心安理得、宁静祥和。”真正的快乐不在于获取,而在于付出,在于无私奉献,这是人生快乐的源泉。
\end{case}

下面步入正文。

这季是关于德行培养的,德行是戒色的基石,戒色后一定要学会培养自己的德行,德有多高,戒有多稳!戒到后来对德行的要求越高,德行有亏,就很容易发生破戒,记得之前一位老戒友戒了两年多破戒了,原因就是嗔恨心。嗔恨心是负能量,会导致情绪失控,这样就容易出现情绪破戒。培养德行一定要学会克服嗔恨心,克服一切负面的心态,这点异常重要。培养德行也要多行善积德,不断强化自己的正能量。至戒要靠德!德行出现问题,很可能会栽大跟头,爬得越高,摔得越惨!德行好,就会很稳定,如果还是自私狭隘,嫉妒心、嗔恨心重,那肯定会破戒的。内心负能量一重,迟早会破戒,这是必然的规律,内心正能量强,就会戒得比较稳定。进入戒色的高层次,必然要无私,如果还是围绕着自私,贪名贪利,肯定会出问题,自私就像癌细胞,一个人如果以自私为核心,他做得越多,这个癌细胞就长得越快、越大,不断侵蚀自己的健康,最终导致毁灭。如果以无私为核心,懂得培养德行,那就是一个健康的细胞,你做得越多,身心的感觉就会越好,精神境界也会越高。这是自私与无私的天壤之别,我戒到现在,经历了很多事,也看到过很多戒色前辈的经历,我发现无私真的很关键,一旦自私,就会争名争利,心态就会变坏,心态一坏,就很容易出现破戒。

武德风范四句诀:武德高尚、武风正派、武礼谦和、武技精湛。学武是很重视武德的,那些武学大家都是道德高深人士,否则是无法进入高层次的,武学大家很重视习武者的武德,人品不端者不传,不忠不孝者不传,人无恒心者不传,不知珍重者不传,习武是有很多规矩的,对德行要求很高。“习武先习德”是中国传统文化在武术中的体现,是武术最核心的指示,德行上去了,练武的境界自然就上去了。我们戒者也要注重德行的培养和提升,德行好,才能戒得好,德行差,戒色之路也会充满坎坷。将来出来工作,领导不仅会看你的才华,也会观察你的德行,有德行,会做人,就能得到领导的赏识,德行差,不会做人,就会越混越差。有才华的人不如有德行的人,有德行的人可以统御一群有才华的人,有的老总本身学历不高,但德行却非常好,他的员工基本都是本科、硕士乃至博士生。刘备,字玄德,玄德是很深的德性,是非常强大的德性,具备如此德性的人可以做领袖,一群武将、军师都围绕着有德性的人,都尊崇他。大家可以观察很多老总,很多 CEO,都是德性极强之人,才能坐到那个位子,德性差的人,即使坐了,也稳不住,一定要有强大的正能量和德性才能稳得住。为什么德行好,德性强的人可以做老大?因为他的振动频率高!这是我反复研究发现的,宇宙的一切都是振动的能量,一切都可以用振动频率来解释,有才华的人不一定振动频率高,有才无德是小人,有德之人振动频率才高,振动频率高就能做老大,才能服众!而影响振动频率的是什么?是一个人的念头!断除恶念,多发善念,多做善事,多做仁义之事,这样振动频率就会越来越高,能量场就会越来越强,真正的力量就是振动频率高!不是肌肉的力量。之前看过一个大力士硬拉 500 公斤,破了世界纪录,他有没有力量,肯定有,但他的力量属于肌肉的层次,这还不是最高层次的力量,最高层次的力量是修心的力量,是内在的德行,是真我的力量!

\begin{case}
    飞翔老师,您好,我戒色还差一个月就三年了。可是我今天破戒了,我想把原因反馈给您,希望您可以收录进《戒为良药》以警醒广大戒色吧戒友。我没有重视养生,从一开始就没重视。我总觉得只要不破戒就是戒色成功,只要戒下去身体就会自动恢复。我自认为养生意识不错,生活习惯健康。就我的感受而言,养生不容易,养生需要在戒色的基础上更加精微地控制。而最影响身体恢复的是名利心,我不熬夜,不抽烟也基本不喝酒,一年也超不过三次的应酬。这些都可以忽略不计,每天有运动习惯,由于武术世家,每天都练习太极拳、形意拳和打坐,这些对我的恢复都起到了很重要的作用。然而强烈的名利心大大地消耗了我的精气神,它使我思虑过度,它使我压抑焦虑,它使我情绪不佳,它使我时常不顾身体是否能承受而过度劳累。它无时无刻不消耗着我,就像一个蛀虫,三年来身体的恢复时好时坏,总是在筑基还未稳固就被我消耗一空。我在北京闯荡,在这座个人欲望极度膨胀的城市里我也没能逃出去。就这样慢慢拖到现在,身体的恢复长期不理想慢慢地磨垮了我的戒色意志,我是在抵抗中被攻破的。三年了,心魔从未攻破过我,今天我遭遇了滑铁卢。因为我再也无法调动意志了,就像悬在半空失去了支点。在长期戒色的路上最重要的就是保住初心,只有它稳固了才能保证戒色的态度和面对心魔进攻时的坚决程度。可是随着时间的延长初心必然会慢慢消退,每到这时我就会通过看受害者案例和背诵戒色语录来坚定信心。每次都成功了,可是我没有认识到还有一个重要因素就是身体的恢复情况,它直接左右戒色的意志。古圣先贤关于养生的论述里第一条都不约而同地提到了薄名利,而很多时候不跌一跤不会认识到问题的严重性。希望您可以通过《戒为良药》让大家看到,让大家充分重视养生,认识到名利心对养生的影响。在心底去除名利之欲,放下执着心,生而不有,为而不恃,努力生发而无欲。人为财死鸟为食亡,最害人的还是名利之心,最影响养生的根苗是名利之心。
    \subparagraph{分析} 这位戒友能戒三年很不容易,可惜他破戒了,一方面身体恢复情况不理想磨垮了他的戒色意志,另外就是名利心太重,导致心理失衡,他之前戒得还是挺不错的,养生方面也做得还行,就是因为名利心太重,思虑过度,这样能量耗损也很严重,一直想一直想,就像在脑袋里跑马拉松一样,是会很累的,消耗是很大的。有句话是这样说的:“学会看淡名利是对自己最好的保护。”追求名利,本质是一种私心,越想追求,却越难得到,心里会变得压抑而焦虑,那些好的名人,他们做了多少善事,捐了多少款,这是我们很难想象的,他们的不断付出才有了他们好的福报。一个人应该看淡名利,而专注于行善积德,多帮助别人,学会慢慢培植自己的福报,到时虽然不求,而名利自得,得到名利也不会执着,而会看得很淡,很平常。在一个大城市打拼是很不容易的,要面对各方面的压力,自己一定要注重养生,不仅是养身体,更要注重养心,让内心祥和、稳定,充满正念,充满正能量。这样有了一个健康的身心,一个良好的精气神,一个充足的精力储备,才能更好地拼搏自己的事业。不要羡慕别人的物质条件,不要嫉妒别人的名利,那是别人行善积德得来的福报,要深深明白这一点,就知道该往哪里使劲了。嫉妒别人只会导致内心失衡,内心会很痛苦,这其实是完全没必要的,关键要自己坚持行善,积累起正能量,到时候振动频率上去了,自然会匹配好的人事物。在欲望极度膨胀的都市要保持一颗清净心是很不容易的,需要很高的修为,要闯荡出一片属于自己的天地,必定要有深厚的德行作为根基,有德行,有拼劲,领导就会赏识,有德行才能真正得人心,你会做人,有德行,很多机会都会留给你,老天也会在关键时刻帮你一把,这样你就飞黄腾达了,年薪十万、几十万,慢慢就能在大都市站稳脚跟了,否则缺少智慧,缺少德行,只在那空想名利,并且还产生很多负面的念头,这样只会拉低自己的振动频率,境遇只会越来越差。这位戒友的善根还是不错的,已经认识到了自己的问题,相信他会做出相应的调整。把名利看得太重,就会活得很痛苦,看淡名利,注重养生,养好精气神,就能好好奋斗自己的事业。一个好的心态真的很重要,好的心态对应的就是好的德行,戒到最后拼的就是德,德行有亏,就可能间接导致破戒。秦东魁老师说过,上等风水在自己身上,在自己的内心,思人恩德,想人好处,这叫聚光。光向上走,表现在脸上,就是微笑。微笑的脸是元宝形,嘴像莲花一样,肯定发财。想人不好,抱怨人、嫉妒人、憎恨人,这叫聚阴。气阴则下沉,表现在脸上,就是冬瓜脸、苦瓜相,肯定倒霉。学了圣贤教育,就知道自己问题出在哪了,一定要把自己的德行提上去,这样才能进入更稳定的戒色层次。
\end{case}

这季我特别总结了 256 个字的戒色德训,最好能每天读一遍,经常看看,感受来自正能量的加持,读一遍就能感受到一股力量在升起,在强化,在把你导向强大而正确的方向。这 256 个字就像一块异常强大的能量块,读一遍就像服用了一次能量块,瞬间会感觉到振动频率的提升,这种感觉十分微妙,还会感觉到某种崇高的力量,你的眼神和举止也会变得崇高起来。经常读诵自然可以让自己按照这 256 个字去做,去落实,让自己的言行去符合这 256 个字。

\begin{center}\it
    崇德弘毅,使命担当,正己化人,立场坚定;\\ 劳谦无怨,中道平和,正直仗义,孝顺感恩;\\ 谦让团结,不争第一,谦卑自牧,虚怀若谷;\\ 谦和宽厚,刚烈忠义,温良忠厚,不慕虚荣;\\ 有惭有愧,不骄不炫,无私奉献,不求回报;\\ 宠辱不惊,低调内敛,宽恕仁爱,和蔼可亲;\\ 真诚友善,以和为贵,心地放宽,和气致祥;\\ 修己以敬,涵养德性,品雅行优,乐我天真;\\ 待人诚恳,与人为善,稳重做事,坦荡做人;\\ 诚信知礼,光明磊落,持身谨严,方有刚毅;\\ 守正待时,克服嗔嫉,心胸豁达,乐观豪爽;\\ 以德御才,修身种德,养德远害,修心为上;\\ 非德勿行,断恶修善,德义为重,谨慎对待;\\ 积功累德,有德自安,德为人先,行为世范;\\ 尊师重教,恭敬老师,善体师心,敬顺无违;\\ 常念师恩,让功于师,功成弗居,抱朴守拙。
\end{center}

下面我逐条解析。

\subsubsection{崇德弘毅,使命担当,正己化人,立场坚定}

崇德:崇尚美德。弘毅:宽宏坚毅;刚强、勇毅,谓抱负远大,意志坚强。\textit{士不可以不弘毅,任重而道远。(《论语》)} 戒色到一定程度要有使命感,这种使命感就是唤醒更多的人认识色情与邪淫的危害,去帮助他们,要有一种神圣的使命感,这是一件绝对正确的事,虽然在邪淫泛滥的时代,不少人刚开始会误解我们,诋毁我们,但只要我们坚持做下去,自然会得到越来越多人的认可与支持,他们是会慢慢醒悟的,特别是在伤精症状爆发后,他们会开始认识到戒色的正确性。要勇于担当这一伟大的使命,要把这个公益事业坚持做下去,帮助一个是一个,有时候你的一句劝导的话就可以改变一个人的命运走向,真的有这么强大的力量,他会深深感谢你一辈子,我见过很多这样的案例,最后在文章里都表达了对宣传人员的高度敬意和感谢。一句宣传,照亮了一个邪淫者灰暗的人生,突然他的人生就开始转变了,就因为你的一句宣传,他的人生轨迹就开始向着好的方向转变了。

有次和一位小吧聊天,他提到了最正确的戒色动机应该是无私奉献、无条件的爱与付出,我对他的观点深表认同,因为这个动机是无私而崇高的,的确是最正确的动机。刚开始戒色,往往是一些个人的动机,是完全围绕自身的一些动机,比如恢复身体健康,恢复容貌气质,恢复帅气、恢复脑力,为了自己的学业或事业等等,刚开始都是这些个人的动机,当然一开始我也是这样,毕竟当时身心垮了,急需恢复健康,是会出于个人的动机来戒色,后来身心恢复了,这时就要升华自己的戒色动机了,否则身心容貌恢复后,就会丧失动力,不知道戒色是为了什么,会感觉人生变得无意义。古圣先贤提出了“正己化人”,这四个字真的太好了,不仅正己,还要化人!要帮助别人,要无私奉献,这样的动机就相当崇高了,也是我最提倡的戒色动机。一开始很难有这个觉悟,慢慢通过学习,就有了正己化人的觉悟了,这时戒色动机就完全升华了,不再局限于自身,而是为了大家的福祉而努力奋斗。真正的戒者应该是立场坚定之人,绝不会动摇,立场、信心、决心和执行力,具备这四者的戒友很容易成功。

\subsubsection{劳谦无怨,中道平和,正直仗义,孝顺感恩}

\textit{劳谦君子,有终,吉。(《易经》)} 劳谦君子,万民服也。勤劳而谦虚的人,真的很得民心,不怨天尤人,也彰显一个人的德行。人生处世,一个谦字便可以安身立命,谦为德之柄,谦举则德张。在《易经》六十四卦中,谦卦是唯一的全吉卦。我很强调谦虚,也很看重谦德,因为谦虚是真我的一种品质,而骄傲则是在强化小我(ego)。谦虚的念头是高频振动的念头,而骄傲的念头是低频负面的念头。前段时间一位资深戒友发的文章,他指出戒不掉的原因之一就是懒!说得一针见血,很多新人都懒得学习戒色文章,破戒了就说自己定力不行,毅力不行,殊不知光靠毅力是不行的,是在蛮干,定力也是通过学习戒色文章提升的,也要通过实战的磨练而变得稳固。观察一个人是否能戒掉,首先看他的决心和学习热情,这两点异常重要,能戒掉的人都有认真学习戒色文章,而且是养成学习习惯,坚持日课不中断,每天坚持学习,做笔记,复习笔记,严格落实前辈反复强调的内容,就像备战中考、高考和考研一样。戒色觉悟的提升是一个不断深入、不断领悟的过程,学习并不是枯燥的,当一个个顿悟爆发的时候,那种喜悦是极度兴奋的,到时一下就明白了前辈的真意,喜悦之情,无以言表,真的是拍案叫绝。对戒色文章有强烈的兴趣,强烈渴望学习戒色文章,强烈渴望戒除恶习,这样的人提升觉悟是突飞猛进的。你想戒掉,你是否够强烈?当你疯狂找黄,迫不及待打开黄片,把那种强烈的冲动用在学习戒色文章上,你早就戒色成功了。

中道,即中正之道。公正、正义、不偏不倚,不会有偏激的想法,整个人的气度也比较平和,这样的人是具有大智慧的人。一偏激就不符合中道,中道是绝对的理性,为人处事也很有分寸,邪淫之人往往思想偏激、偏执,脾气暴躁,这是负能量的一种表现,当戒掉恶习,多学习圣贤教育,慢慢就能回归中道平和,这时候会发现人际关系也自然改善了,一个偏激的人是很难与人相处的。平和养无限天机,真我的状态就是一种平和、平淡的状态,内心平和,能有作为,平和之人必定很有耐心,做事很有章法,能做到平和,必定具备很深厚的德行,大德都有一种平和的气度。正直仗义也是我很看重的品德,正直之人也是我所钦佩的,要做到表里如一的正直是很不容易的,还能仗义,那就更难了。孝顺感恩是最根本的德行,万恶淫为首,百善孝为先,非常孝顺父母师长的人,一定是一个有福之人。孝顺父母——世间第一福田!要种福、修福,第一就是要孝顺父母,不能抵触顶撞父母,有事好好沟通,尽量避免吵架,要多孝顺父母,一个孝子的能量场是很强的,到了社会上也懂得尊敬领导,尊重长辈,很得人心。第一福田,如果不懂得去种,那真的很愚痴,很多人不仅不懂得去种这块福田,还经常去糟蹋这块福田,和父母吵架,家庭很不和谐,这样的人将来肯定到处碰壁。一旦你懂得孝顺感恩了,你的能量场一下就好了,走到哪里都容易受到欢迎。《论语》中也讲“\textit{事父母,能竭其力}”,竭尽自己的能力去孝敬父母,陪伴父母,回报父母,这是人生最重要的一条。\textit{复有十业,能令众生得端正报。一者不嗔;二者施衣;三者爱敬父母;四者尊重贤圣;五者涂饰佛塔;六者扫洒堂宇;七者扫洒僧地;八者扫洒佛塔;九者见丑陋者,不生轻贱,起恭敬心;十者见端正者,晓悟宿因。以是十业得端正报。(《佛说业报差别经》)} 很多佛经都强调了孝顺的重要性,对父母、对长辈、对领导、对老师,你孝了,你就顺了,反之,忤逆不孝,真的是到处倒霉,各种不顺,真的无法做人了。

\subsubsection{谦让团结,不争第一,谦卑自牧,虚怀若谷}

《礼记》中有“\textit{让,礼之主也}”、“\textit{让,德之主也}”、“\textit{卑让,德之基也}”等话,都强调了“让”对于礼义道德的重要。谦让团结是非常重要的,不管是戒友之间,还是戒色前辈之间,都应该注重谦让团结,不要争第一,公益事业没必要争第一,不要有超过别人、比别人强的想法,要互相学习,共同进步。上善若水,水善利万物而不争,争第一,想比别人强,这其实是一种自私的想法,也是在强化 ego。谦让团结,才能和谐,勿争高下,互相支持,这才是正确的做法。儒家认为,谦让是一种高尚的道德行为,也是君子所应当具备的道德品质。《礼记·聘义》中说“\textit{敬让也者,君子之所以相接也}”,孔子也说“\textit{君子无所争}”。记得小学时,老师就专门讲过“团结友爱、互相谦让”。谦让是一种胸怀,一种美德,一种风度,一种智慧,更是一种修养。争第一其实是名利心,贪名的念头在作怪,有了争第一的心,心理就会失衡,就会自赞毁他,争心一起,诋毁随之,让自己陷入负能量。

\textit{谦谦君子,卑以自牧。(《周易·谦》)} 印光大师在一封回复的手书里写道:“\textit{汝年二十一,能诗能文,乃宿有善根者。然须谦卑自牧,勿以聪明骄人,愈学问广博,愈觉不足,则后来成就,难可测量。}”学习了很多戒色文章,懂了很多戒色养生知识,修心方面有了长足的进步,这时一定要懂得谦卑自牧,仍然觉得自己很浅薄,还需继续努力,有这样的心态,才能进入更高的层次与境界,否则得少为足,觉得自己很了不起,很骄傲,这类人很快就会破戒。我之前看过很多案例,不少人戒了几十天或者上百天,就开始骄傲了,一骄傲自满,不久就破戒了,十几岁的少年很容易犯这个毛病,吃过几次亏,再学习前辈的文章,就知道谦卑自牧的重要性了,再也不敢骄傲了,骄傲的念头其实是很狡猾的,也是心魔的一种攻击方式,往往在进步的时候就潜藏着骄傲的危险,一骄傲,就要倒霉,因为骄傲是负能量的低频念头。有的戒色前辈说自己发表了戒色文章,得到了大家的支持与赞许,这时候微妙狡猾的骄傲念头就开始出现了,这时要提高警惕,坚决断除,然后多思维骄傲的坏处、谦虚的好处,切不可有半点骄傲得意的念头,时刻警惕自己不要犯这个错误。孙子兵法里面就有使敌人骄纵这一条,所谓骄兵必败,骄傲的人肯定会放松警惕,而心魔这时就容易得逞了。

悟达禅师的法缘日盛,唐懿宗非常欣仰其德风,备极礼遇,特尊他为国师,并钦赐檀香法座,禅师亦自觉尊荣,集朝野礼敬于一身,起了贡高我慢之心,于是膝上忽然长了个人面疮,就是因为贡高我慢让宿世的冤家债主得便。这个公案很有名,给我的印象很深刻,大禅师也可能犯这类低级错误,狡猾的骄傲念头实在太可怕了,很容易导致阴沟里翻船。一骄傲,整个能量场就坏了,到时就要失败。谛闲法师有一徒弟,名显荫,人极聪明,十七八岁出家。但气量太小,一点屈不肯受,傲性日增月盛,后来谛闲法师说他声名很大,惜未真实用功,当闭三年关,用用功方好,显荫一听就气病了,年余而死。曾国藩说过“\textit{败人两字,非傲即惰}”,曾国藩在家书中展开来说:“\textit{天下古今之庸人,皆以一惰字致败;天下古今之才人,皆以一傲字致败。吾因军事而推之,莫不皆然。}”不管戒多久,学了多少戒色文章和圣贤教育,都要虚怀若谷,始终保持谦虚谨慎,这点特别重要,我看多了因为骄傲自满而破戒的案例,所以对骄傲的念头格外警惕。稍微有一点骄傲的微妙感觉,就要警觉,就要立刻断除!绝不姑息!骄傲的念头绝对是大敌!

\subsubsection{谦和宽厚,刚烈忠义,温良忠厚,不慕虚荣}

二、三、四条都有谦字,我极度重视谦德!满招损,谦受益,天道亏盈而益谦,地道变盈而流谦,鬼神害盈而福谦,人道恶盈而好谦。谦,亨,君子有终。\textit{谦者,屈躬下物,先人后己,以此待物,则所在皆通,故曰亨。尊者有谦而更光明盛大,卑者有谦而不逾越。君子能终其谦之善,而又获谦之福。(孔颖达)} \textit{以崇高之德而处于卑下,谦之意也。惟君子能谦,惟谦终成其为君子也。圣人以谦望人,实乃天地之德,本如此也。(刘沅)} 谦和宽厚这四个字非常好,是很深厚的德行,不仅能谦,还很注重和谐,不仅注重和谐,而且还很宽厚,宽者容人,厚者载物,宽厚有如大地之势,厚实和顺,有如明月之皎皎,光罩万物。宽厚是一种境界、一种品质、一种高度,能做到宽厚真的很不容易,我观察过很多领导者,很多领袖,都有宽厚之相,能做到宽厚,是做人极高的境界,宽厚的长者往往德高望重,很能服人。

刚烈忠义也是我极为钦敬的德行,能做到刚烈忠义之人,在我心中是真正的英雄豪杰。三国英雄大多都是刚烈忠义之士,他们很有气节,也懂得戒邪淫,三国里面我个人比较喜欢关羽,刚烈忠义,义薄云天。张飞、赵云、刘备、诸葛亮,我都挺喜欢的。我喜欢张飞的悍勇、赵云的儒雅、刘备的德行、诸葛亮的风度和睿智。男人就应该有刚烈忠义的一面,堂堂正正,不苟且偷安,原则和立场极度鲜明,极度坚定。如果说谦和是柔的一面,刚烈忠义就是男人刚的一面,要刚柔并济,刚而不柔,脆也;柔而不刚,弱也;柔而刚,韧也!最近我又学习了《关圣帝君戒淫经》,很经典的传统戒色文章,里面有一段话我记得很深,那就是:“\textit{情窦初开,天良渐失,本是玉堂人物,弄成邪僻儿郎,无论磊落奇才,一定功名偃蹇,庸师之过,害人不浅。}”在邪淫前,很多人都是玉堂人物,一表人才,邪淫几年后,真的猥琐不堪,弄成了邪僻儿郎,功名也被削掉了,本来是考名牌大学的料,最后去了不理想的学校。庸师之过,害人不浅!砖家横行的年代,很多孩子都被误导,看了无害论更加肆无忌惮,放纵自己,这都是庸师之过!

再提一位人物,颜真卿,我很钦佩他的人品,其次才是书品。颜真卿的《与郭仆射书》,系颜真卿给右仆射郭英乂的书信手稿(右仆射为官名),信中直言指谪郭英乂在两次隆重的集会上藐视礼仪,谄媚宦官鱼朝恩,这是颜真卿不满权奸骄横跋扈而奋笔直书的作品,刚烈之气跃然纸上,忠义之气充之于心、赋之于文、形之于书,全篇理正、词严、文厉、书愤,洋洋千文,如长水蹈海,无可阻挡,遂使历代书家无不为之服膺倾倒,迄今一千余年,读之莫不令人肃然起敬。显示了颜真卿刚强耿直、朴实敦厚的性格,颜真卿秉性正直,笃实纯厚,有强烈正义感,从不阿于权贵,屈意媚上,以义烈名于时。\textit{以崇高之德而处于卑下,谦之意也。惟君子能谦,惟谦终成其为君子也。圣人以谦望人,实乃天地之德,本如此也。(欧阳修)} 刚烈忠义,真英雄,真烈士,戒色也要有这股劲!虽千万人,吾往矣!一往无前,势不可挡!千古刚烈忠义之气,那个表情,那个眼神,那个气概,那个立场,让人印象太深刻了,也太敬佩了,那个感动无以言表。

温良忠厚也是很好的品德,温,不暴躁,说话和气,懂得和顺之道;良,心地善良,人很贤良。忠厚这两个字,也很了不得,能做到忠厚也很不简单,忠厚传家久,诗书继世长,具有忠诚、厚道的道德品质的家庭,能够长久地绵延下去。不慕虚荣也很重要,一些戒友说自己爱慕虚荣,他们也意识到这样不好,虚荣心会使一个人失去心灵的自由,常常使人觉得没有安全感,不满足,会很在意别人对自己的看法,往往会因为贪慕一时的虚荣而丧失自我。虚荣是对名的一种变态追求,一个爱慕虚荣的人只会去伤害别人、贬低别人,抬高自己,虚荣也会使人形成不务实的浮夸之风。不慕虚荣,脚踏实地,人生才能过得安稳而实在。

\subsubsection{有惭有愧,不骄不炫,无私奉献,不求回报}

《华严经》里面讲,有惭有愧,即是菩提。惭愧二心,是菩萨藏。培养恭敬心、惭愧心可以有效对治傲慢心,别人赞扬你几句,你应该感到惭愧,觉得自己做得还不够,这样就不会起傲慢心。很多戒友在我帖子里留下了支持,也有赞扬,我感到很惭愧,也无比感恩大家的支持。每次发惭愧心,自己就会进入一种很微妙的心态,这种心态带有反省、自我鞭策的意味,所以我经常发惭愧心。如果没这种修为,那么面对赞扬是很危险的,因为很容易助长一个人的傲慢心,会强化他的 ego,让他自以为是,觉得自己了不起,这样就毁了。我很早就明白这一点了,所以面对赞扬都是很小心的,以惭愧来应对,这样内心就很平静,也很理智,不会出现傲慢心。我们要做到不骄不炫,年轻人最大的毛病就是骄傲和炫耀,这在年轻人当中是很普遍的,买个新手机或者新鞋子都要炫耀一下,比别人考得好,也会骄傲,十几岁的孩子很容易出现这两个毛病,长大成熟后会慢慢好些,但很多成年人也有这两个问题,不骄不炫才能有深厚的德行,一骄傲一炫耀就显得格外肤浅了。

上面已经讲到了无私奉献,最近看了一篇戒色文章,作者讲自己的戒色动机,就是为了恢复脑力,提升学习成绩,然后就是为了考研,这都无可厚非,当然他都做到了,成绩上升,拿到国家奖学金,并且考研成功,后来他又有了新的动机,那就是要配得上自己心目中的女神——他暗恋的女孩。他的这些动机全部是围绕个人的,大家刚开始戒色基本都是这样,因为还没有无私奉献的觉悟,基本都是为了自己。等到学习圣贤教育,觉悟一步步提升后,这时候就要建立崇高的戒色动机了,那就是正己化人,无私奉献,无条件的爱与付出,这是最终的戒色动机。如果仅仅是为了自己,一旦目标达到了,就很容易松懈,到时也会失去动机,所以必须要学会建立崇高的动机,要多帮助别人,给别人带去正能量,给别人心里也种下善的种子,这颗种子一旦发芽成长,将来也能长成参天大树,正能量会不断传递下去,影响越来越多的人,一盏灯可以点燃成千上万盏灯!你帮助了别人,你自己积累了正能量,你的振动频率得到了提升,其实利人就是在利己,而你没有一点自私的心态,不求任何回报,这就是真正的无私奉献,想求回报还是自私的表现,真正无私绝对不会求任何回报,看到别人超过自己,也会真诚祝福对方,希望对方生活幸福。佛法方面讲三轮体空不住相,这才是布施的最高境界!\textit{菩萨不住相布施,其福德不可思量。(《金刚经》)} 做的时候努力去做,做完心里就不要去想了,就像没做过一样,内心十分清净。千万不要去贪功德,这是很多人常犯的毛病,贪功德也是贪心的一种表现,贪心就是自私,这与修行是背道而驰的,仍属于贪嗔痴三毒的范畴,一定要学会无私奉献,做到三轮体空,这才是真正有智慧的表现。

\subsubsection{宠辱不惊,低调内敛,宽恕仁爱,和蔼可亲}

\textit{宠辱不惊,闲看庭前花开花落;去留无意,漫随天外云卷云舒。(《菜根谭》)} 这段话道出了对名利应有的态度:得之不喜、失之不忧、宠辱不惊、去留无意。这样才可能心境平和、淡泊自然。不要过分在意得失,不要过分看重成败,不要过分在乎别人对你的看法,做好自己即可。心态平和,恬然自得,方能达观进取,笑看人生,这是一种极为淡然平和的心态,这种心态非常难得,有大智慧的人才能做到,普通人有点宠辱毁誉肯定要动心。苏东坡先生曾经有过这样一个故事,他是一个在家学佛的居士,有一次,他写了一首诗:“\textit{稽首天中天,毫光照大千。八风吹不动,端坐紫金莲。}”八风是指“利衰毁誉,称讥苦乐”,也就是无论别人称赞我还是讥讽我,无论是别人毁谤我,还是给我很好的名声,无论是在苦中,还是乐中,都不为这些外境所动,保持着一种如如不动的平和心态。苏东坡写了这首诗之后,越看越觉得境界好,越掂量越得意,于是就派自己的童子把这首诗送给庐山归宗寺的佛印禅师,让他给看一下。可是禅师看了之后,没有说什么话,只是在上面写了两个字,然后交给童子让他拿回去。苏东坡左等右等,心里非常的焦急,看到童子之后就迫不及待地问:“禅师看了我的诗之后,对我的诗有什么好的评价吗?”结果这个童子说:“禅师什么话都没说,只是在信纸上批了两个字,让我给你带回来。”苏东坡马上把信封打开,看到上面写了两个字“放屁”。苏东坡一看怒火中烧,他觉得这个禅师理解不了我的境界也就罢了,为什么还这样糟蹋我的诗呢!于是他不等天亮,连夜坐船渡过江去,要找禅师评理。最后禅师送他两行字:“八风吹不动,一屁打过江。”苏东坡一看觉得很惭愧,自己是能说不能行,也就是说能够体会到这个境界,但是自己却不能够在生活中去实行。苏东坡这个故事很有名,相信很多人都看过,境界一来,就验出来了。能做到宠辱不惊是很不容易的,需要有很深的定力和很高的智慧。平时做任何事都要注意低调,不要过于高调张扬,不要到处炫耀,不要得意自大,要低调内敛,始终谦虚谨慎,这才是真正有智慧的表现。记得有一位名人做善事很高调,那位名人前几年挺火,后来他也知道要低调了,不再张扬了。往往在高调张扬的过程中会说出很多大话,骄傲的话,这样很容易招致批评。

宽恕仁爱,和蔼可亲,这八个字的气场很强大,不仅懂得宽恕别人,也要懂得宽恕自己,原谅自己,这点很重要,过去很多做错的事时不时会冒出来,很容易让人陷入痛苦之中,这时要懂得原谅自己,人非圣贤孰能无过,过而能改,善莫大焉!过去的错事就不要去想了,冒出来就断掉,犯错当然要反省和忏悔,但不要一直去想,否则很容易陷入痛苦而不能自拔,一回忆,内心就很痛苦,很难受,什么事也干不了,甚至要捶桌子、抓头发,脸上的表情那叫一个痛苦啊!很多邪淫者过去的确干了很多禽兽不如的事,好好忏悔后就不要去想了,让心态保持平和才是关键。对于别人伤害自己,也不要去记仇了,宽恕别人,也就放过了自己,一直记仇,就会产生很多负面的念头,增加了自己的负能量。有句话是这样说的:爱你的敌人,才是一个真正的强者。《圣经》中也说过:要爱你的敌人。我的理解是爱是包容的,包容一切人,当然也包括你的敌人,但并不是叫你丧失立场,是非不分,爱你的敌人,是要懂得原谅对方,在适当的时候帮助对方,这样才能化干戈为玉帛。要懂得运用宽恕的力量来化解问题,这是很重要的技巧,因为宽恕了,所以心态就平和了,心态平和了就可以契入真我了。

\textit{仁爱士卒,士卒皆争为死。(《史记》)} \textit{尧立孝慈仁爱,使民如子弟。(《淮南子》)} \textit{君子所以异于人者,以其存心也。君子以仁存心,以礼存心。仁者爱人,有礼者敬人。爱人者,人恒爱之;敬人者,人恒敬之。(孟子)} 做人要懂得仁爱之道,这样才能真正得人心,靠物质利诱来笼络人心是不可靠的,要用德行来感化对方,这样才能得到对方真正的敬重和由衷的敬佩。物质利诱是一种暂时的利用关系,虽然有一定效果,但不如德行来得长久。仁爱对方,对方才能发自真心来为你做事。这里讲的就是做人之道了。

\textit{子张问仁于孔子。孔子曰:“能行五者于天下为仁矣。”“请问之。”曰:“恭、宽、信、敏、惠。恭则不侮,宽则得众,信则人任焉,敏则有功,惠则足以使人。”(《论语》)}(子张向孔子请教什么是仁。孔子说:“能够推行五种品德于天下的人,可以说就是仁了。”子张说:“请问是哪五种品德。”孔子说:“恭敬、宽厚、诚信、勤敏、恩惠。恭敬就不致遭受侮辱,宽厚就能得到众人的拥护,诚信就能得到别人的信任,勤敏则做事易见功效,施人恩惠别人就心甘情愿为你效劳。”)这五种品德很重要,谦则人和,宽则得众,要得到众人的拥护,必须懂得宽厚之道,领导者要具备宽厚的品质。宽则得众,这四个字我很欣赏,实在佩服至圣先师孔子的为人处世的智慧。我们真的要好好学习《论语》,要不断挖掘里面的智慧,就像挖掘金矿一样。学到一些道理,对于为人处世真的很有帮助,因为你知道该怎么做了!施人恩惠也是在德行的感召之下,这样别人就心甘情愿为你效劳了,如果仅仅是物质利诱,那只是一时的雇佣关系。有些戒友将来可能会走上领导岗位,如何做一名宽厚有德行的领导,是要好好研究的。

和蔼可亲,这四个字非常了得,能做到和蔼可亲实属不易,一,要没有嗔心;二,要具备和气;三,要具备慈悲心;四,要具备很深的善心;五,要具备很强的喜悦感。六,要具备很强的亲和力;七,要具备很高的智慧。和蔼可亲的本意是指人态度温和,性格善良容易接近。这是普通的解释,我对和蔼可亲的定义更高一些,因为我觉得这四个字实在不简单,能做到这四个字的人真的不得了。有的老人可以做到和蔼可亲,因为他们已经进入无欲无求的状态,而且有很高的智慧和修养,但也属于少数,观察生活中的老人,真的和蔼可亲的还是少数,很多老人也是精神不振,脸上也没有什么笑容,毕竟风烛残年了。我在元音老人身上看到了和蔼可亲这一特质,上面七点就是按照他的示现所总结出来的。真正的和蔼可亲是很不容易做到的,自身必须具备很强的正能量,特别是深厚的慈悲心,没有一点嗔心,一团和气,脸上也有喜悦的神情,这样就具备很强的亲和力。和蔼可亲是比较高的境界,也许现在我们无法做到,但我们至少可以向着这个方向努力,一点点去接近,也许将来某一天我们就能做到了。

\subsubsection{真诚友善,以和为贵,心地放宽,和气致祥}

这条也是讲做人的,做一个真诚友善的人,这点很重要,要善于经营自己的人脉,交到真正高质量的朋友,特别是善友,邪淫之人往往交到的是狐朋狗友,所谓物以类聚,人以群分,戒色之后会发现自己与过去的邪淫圈子格格不入了,很难交流了,也不愿与他们多说话了。虽然看似自己被孤立了,但更高振动频率的朋友圈正在向你敞开,慢慢就能遇见志同道合的朋友,和你振动频率接近的朋友。我个人的体会就是,以邪淫为乐、经常开黄色笑话、谈自己放纵经历的朋友,是绝对靠不住的,因为他们的振动频率太低,思想也多是负面的,自己也很不稳定,当你发生什么事了,想指望他们帮你,是很有难度的。处在邪淫状态交到的朋友,也不会维持很久,要交到善友,你必须具备和他们类似的想法,善友想着戒邪淫,行善积德,当你有了这样的想法后,你们就有了共识,就很好沟通了,你一下就进入了那个圈子,交到了一群具有正能量的朋友,这类朋友在你需要帮助的时候,是会想方设法帮你的,他们也有很强的责任感和行善意识,这类朋友很有质量,值得一辈子去交往。朋友之间,同学之间,同事之间,家人之间,都要突出一个“和”字,以和为贵,一定要注重和谐,家和万事兴,与同学和,与领导和,与父母和,到处都能和得来,这样人生就会顺利很多。对于他们的一些缺点和错误,也不要放在心上,心地要放宽一些,懂得宽容和包容,你的内心就会一片祥和。和气致祥,喜神多瑞,邪淫者的家庭往往经常争吵,为什么会这样?因为邪淫的人一直在起负面的念头,会影响到家人,邪淫的人脾气也会变得暴躁,处于到处失和的状态,这样人生就会坎坷很多,会到处倒霉。

“和”是中国文化和人生智慧的重要特征,其内涵十分丰富,充满了大智大慧的深刻哲理。和谐融洽的气氛可致吉祥,不和谐的气氛则招致灾祸。古人很早就认识到这一点了,古人交往很注重礼节,很有君子的风范,也懂得和之道,\textit{和气迎人,平情应物。抗心希古,藏器待时。(《围炉夜话》)} 简单的十六个字,就将待人接物、进德修业的道理阐述得明白亲切。(抗心希古,指以古代的贤人为榜样。)用和气的态度与人交往。古人讲:“和气致祥,乖气致戾。”和气能带来吉祥,乖张会导致祸殃。以和为贵,则万事皆兴,和气方能生财,平易才能近人。平情应物,以平常之心对待事物,处理事情。说话和气,待人和气,注重和谐,不乱发脾气,懂得尊重别人,体谅别人,包容别人,这样的人是具有大智慧的人。一个成功者要充满祥和之气,这样就能稳住自己的事业,我们戒色也是如此,内心要祥和,尽量不要争吵,争吵后情绪会失控,很容易破戒。养祥和之气,就是在做情绪管理,对自己各方面都很有好处。大家肯定都经历过与别人失和的情况,一旦失和,彼此内心都不好受,所以我们平时要注重和谐,尽量以和为贵,这样走到哪里都能带去祥和友好的交流氛围。

\subsubsection{修己以敬,涵养德性,品雅行优,乐我天真}

修己以敬这四个字曾经让我眼前一亮,顿感说得很有道理,做人首先要懂得恭敬,\textit{诚与恭敬实为超凡入圣了生脱死之极妙秘诀。(印光大师)} 能做到至诚恭敬,老天都会感动,一分恭敬,一分利益;十分恭敬,十分利益。\textit{敬,身之基也。……敬,礼之舆也,不敬则礼不行。……敬,德之聚也,能敬必有德。(《左传》)} \textit{凡百事之成也必在敬之,其败也必在慢之,故敬胜怠则吉,怠胜敬则灭。(荀子)} \textit{君子敬而无失,与人恭而有礼。(孔子)} 敬是一项普遍的道德规范,也是一种高尚的道德品质。一个人如果具备这一道德品质,在人际交往中时时处处遵循敬的要求,就可以得到他人的爱戴和尊重,也有利于形成良好的人际关系。恭敬也是获取一切智慧的珍宝,在学生时代,我们懂得尊重与恭敬老师,老师就愿意教我们,师生之间就有很好的互动,就能学到老师传授的知识。如果以轻慢心去学习,那根本学不进去,老师看到这类学生也不喜欢。\textit{子曰:“修己以敬。”} 之后还提到了:修己以安人、修己以安百姓。安人:使自己身边的人安乐。安百姓:使老百姓安乐。所以修己不仅仅是自己的提升,而且还会给身边的人带去安乐,比如你邪淫的时候,脾气暴躁,就会使家人遭殃,一人不好,会影响到整个家庭的氛围,如果你戒色了,心情祥和了,懂得孝顺了,那真的是“修己以安人”。父母也跟着安乐,家庭也和谐了,这样多好。懂得恭敬,就是在涵养自己的德性,看看古人对人的恭敬态度,真的令我们感到惭愧,古人互相见面,拱手行礼,是为揖,这是古代宾主相见最常见的礼节,很有君子的风度,讲话也很有分寸感。我发现注重恭敬,注重礼节,也能有效对治邪淫的念头,一旦恭敬起来,自然就不敢去起邪念,自然就不敢去看诱惑的内容,因为这和恭敬是相违背的。修己以敬就是让自己时刻保持在恭敬的状态,不敢越雷池一步。对人恭敬,不敢有非分之想,不敢萌邪思邪念。对老师恭敬,就能真正从老师的教导中获益,因为恭敬,你就会极其重视老师所教的内容,会虚心去学习,恭敬心真的太重要了,我们一定要培养恭敬的态度。孔子那个手势很有含义,身体微微鞠躬,把双手交叉前伸,放在全身之最前,男性左手在前,女性右手在前,表示虔诚恭谦之意。孔子身着宽大的儒服表情慈祥而端庄,双手交叉并在胸前,不露大拇指象征谦恭,四指并拢代表四海之内皆兄弟天下大同。孔子提倡的就是克己复礼,修己以敬,孔子雕像就含有这个深意。

养我灵台,时时清净,品雅行优,乐我天真,灵台指的是心灵。古人的认识水平很高,知道恢复纯净的心灵,去感受真正的大快乐,灵台肮脏龌龊,整个人的能量场就很差,内心里负面的念头就会很多,到时就会感召很多不好的事情。存在的最终目的和意义就是认识真我,活出真我,再次回到天真无邪的状态,一旦开始邪淫就会严重偏离这个状态,邪淫后变得越来越不快乐,脸上的笑容越来越少。古德强调“思无邪”,让自己的内心恢复纯净纯善,到时你就会发现纯净的状态是如此的快乐、如此的幸福、如此的美好,当你再次回到那个久违的频率,就会发现自己在邪淫时失去了什么!\textit{上古之人,心正无邪,性纯不杂,所以成真证圣者多。无如末世人心,习恶多端,逞强斗智,昧却天良,迷了世俗。失本来之面目,丧固有之精神;视荣华如性命,看道德若泥涂;不识寻真放假,焉知返本还原。日失此心,时迷斯性,朝来醉死,暮到偷生。孽海茫茫,无时得度。……一举一动,不生邪念;养我灵台,时时清净;与月同明,与水同清。……立世效圣贤之道,出尘学仙佛之风;作堂堂之丈夫,为谦谦之君子。清夜问心,仰不愧天;闲日省身,俯不怍人。品雅行优,乐我天真;身安命立,超人性德。(《中华始祖伏羲老祖慈训》)} 《中华始祖伏羲老祖慈训》是篇很好的文章,大家可以学习下,强调的就是孝顺和净化心灵,多修善德,流芳千古,始不枉为人也。

\subsubsection{待人诚恳,与人为善,稳重做事,坦荡做人}

待人诚恳,会被人信任,别人对你会有好感,与人为善,会活得心安,不会到处树敌,能广结知己,广结善缘,能恕人之过,可远离争斗祸端。待人不诚恳,说话虚假敷衍,别人也会远离,我在生活中也曾经遇见过这类人,不诚恳的人得不到人心,谁也不喜欢虚伪的人。常常对别人不诚恳的人,最后对自己也不会诚恳,只能自欺欺人,丧失人心。古语云:“得人心者得天下!”做人一定要真诚,待人要诚恳,要有诚意,真正尊重别人,无论是亲朋好友还是素不相识的陌生人,都要真诚相待,用己真诚换取他人真心。做人要友善,与人为善,自己也将变得快乐。中国文化十分重视人与人和睦相处,待人诚恳、互相关心、与人为善、推己及人,学会换位思考,多体谅别人,原谅别人,以达到人际关系的和谐。从某种角度而言,人生实际就四个字——做人、做事,先学会把人做好了,做事也就简单了。进入社会工作,特别强调做人,会做人,有德行,前途不可限量,能得到领导器重,领导最终是被你的德行所感化的,才华虽然可以让人惊喜一时,但最终让人钦佩和感动的却是德行。你有诚意,别人也对你有诚意,如果玩虚假,别人也会疏远你,踏踏实实做事,坦坦荡荡做人,浮夸吹捧的话不说,利欲熏心的事不做。做人要问心无愧、表里如一,做事要稳重可靠、尽心尽力。

\begin{quote}\it
    1920 年,著名画家齐白石应邀参加一个活动,因为穿着土气,加之不善言辞,结果无人理睬。这让他很尴尬,坐也不是站也不是,走又走不掉。恰在此时,梅兰芳走了进来。彼时,梅兰芳已是京剧名家,衣着光鲜,捧他的客人极多。但梅兰芳还是在众多宾客里,发现了齐白石。此前他们只是相识,并无深交。但梅兰芳还是很恭敬地走过去,与齐白石寒暄。此举让在座宾客很惊讶,纷纷询问梅兰芳主动招呼的客人是哪位,当得知眼前那位衣着土气之人竟是画家齐白石时,宾客也过来寒暄,这才打破僵局,让齐白石摆脱尴尬。事后,齐白石很用心地为梅兰芳画了一幅《雪中送炭图》,题句云:“而今沦落长安市,幸有梅郎识姓名。”后来,梅兰芳想学习画草虫,齐白石欣然收他为徒。齐白石对他说:“你这样有名,叫我一声师傅也就是抬举老夫了。就别提什么拜师不拜师的啦……”可梅兰芳坚持一定要举行拜师仪式,行跪拜大礼。学画也特别认真,只要不排练不演出,不管风天雨天,他都按时坐黄包车到齐宅学画,进门先向老师鞠躬问好,谦恭的样子像个小学生。
\end{quote}

我们一定要学会做人、做事,这决定你最终的境界和人生的高度。德行实在太重要了,最感动人、感动天地的就是德行。\textit{读书不一定重要,但做人一定要做好,一定要重视人文思想、道德教育。(星云法师)} 很多人会读书,能考高分,但是进入社会后发展却不太理想,就是不会做人,反观很多人在学校时成绩一般,但很懂得做人的道理,很注意培养自己的德行,后来进入社会,事业做得风生水起,做人的确比读书更重要,做人的学问也真的很深,学习做人是一辈子的事,需要自己不断去学习和领悟。\textit{你可以没有学问,但不能不会做人。人难做,做人难。在现今的社会,人要有表情、音声、笑容,才会有人情味。懂得感恩者,才会富贵。一点头、一微笑、主动助人,都是无限恩典。我们面带笑容,看在对方眼中,那朵微笑是发光的;当我们口出赞叹,听在对方心底,那句赞美是发光的;当我们伸手扶持,受在对方身上,那温暖的一握是发光的;当我们静心倾听,在对方的感觉里,那对耳朵是发光的。因为发心,凡夫众生也可以有一个发光的人生。贪爱愚痴的人,永远不懂得利用时空,甚至错过了时空。只有懂得利他利众的人,才能把握无限时空。(星云法师)}

星云法师还有一段话我也很欣赏:“\textit{回想我这一生中,不也常被人拒绝,被人挖苦,甚至被人毁谤,被人诬蔑吗?我之所以能安然度过每个惊涛骇浪,首先应该感谢经典文籍里的嘉句和古德先贤的名言,其中史传描述玄奘大师的‘言无名利,行绝虚浮’,是我自年少以来日日自我勉励的座右铭,多年来自觉从中获益甚深;地藏菩萨的‘我不入地狱,谁入地狱’的精神,总是在我横逆迭起的时候,掀起我无限的勇气;每当险象环生的时候,想到鉴真大师所说的‘为大事也,何惜生命’,强烈的使命感不禁油然而生,增添我心中无限的力量。}”多学习大德开示可以让我们更具智慧,不仅学到修行方面的道理,如何为人处世,如何面对生活中的挫折,都可以从大德的开示中学到。

\subsubsection{诚信知礼,光明磊落,持身谨严,方有刚毅}

立身诚为本,处世信为基。诚信,即待人处事真诚、诚恳、讲信誉,言必信、行必果,一言九鼎,一诺千金。“诚”是一种真实不欺的美德,修德做事,必须效法天道,做到真实可信,说真话,做实事,反对虚伪。“信”,就是对人不仅要求说话诚实可靠,切忌大话、空话、假话,而且要求做事也要诚实可靠,脚踏实地。\textit{诚者,天之道也。(孟子)} \textit{至诚如神!(《中庸》)} 发心正,起心动念至诚,表里如一,内外澄澈,这样的人做起事来很有号召力。诚则明,做人要讲究诚信、诚恳、宽容、忍让。这样就能得人心,就能闯出一番真正的大事业,大事业必须要有深厚的德行作为基础,这样才能感召一批真正的人才来帮助你。种下梧桐树,引得凤凰来,德行就像梧桐树,凤凰就像人才。你若盛开,蝴蝶自来,你若邪淫,蛆蝇聚集!邪淫之人聚集的是一群以纵欲为乐的损友、邪友,同流合污,一起堕落,邪淫之人绝难与正人君子为伍。邪淫之人很难做到诚恳,因为他在做着见不得人、见不得光的龌龊事情,邪淫之人必然充满虚假,诚恳是光明磊落的君子才能做到的。\textit{不学礼无以立。(《论语》)} 人无礼则不立,事无礼则不成,国无礼则无宁。学礼则品节详明,而德性坚定,故能立。有礼则安,无礼则危。故不学礼,无以立身。古人很注重礼,\textit{子曰:“君子博学于文,约之以礼。”} 礼是有对治邪淫的功能的,知礼守礼则不敢犯邪淫。\textit{憨山大师:“不邪淫,礼也。”}我接一句:“邪淫,禽兽也。”古人很懂得用礼来规范约束自己的言行,这点我们应该多向古人学习。

有的戒友会说戒色后看到异性会紧张,我们要克服这种紧张,戒色不是让你见到异性就紧张,而是要淡定和坦然,要光明磊落,正气十足。对境时要警惕,但不要过于紧张,要保持自然,这样才能正常交往,否则过于紧张,别人会觉得你很怪异,不自然,这样就会影响人际关系。君子要有光明磊落的气场,眼神特别坚定,特别有正能量,不能有一丝一毫的邪念,这样异性也会对你留下好的印象。光明磊落是一种心底无私的气魄,古语有云:“大丈夫行事,当磊磊落落,如日月皎然。”胸怀坦荡,正大光明,令人敬仰,只有光明磊落、不匿私心,浩然正气才能油然而生。\textit{盖先生之治人,尤重品节。(曾国藩)} \textit{此心光明,亦复何言。(玩阳明)} 坦荡必光明,光明必坦荡,用尽一生,去做一个光明磊落的人。只有拥有了光明磊落的品格与气魄,才能拥有真正的底气与自信。一旦邪淫,肯定无法光明磊落,所以要做到真正的光明磊落,必须戒掉邪淫,只有这样正气才能刚起来!无欲则刚!只有持身谨严,方有刚毅,才能刚得起来,如果肆意放纵自己,肯定会软下去!宝贵的肾精一泄,必定腰膝酸软,身心都会软下去。\textit{心术不可得罪于天地,言行皆当无愧于圣贤。……庙堂之上,以养正气为先;海宇之内,以养元气为本。(《钱氏家训》)} 有了很强的正气、正能量,才能拥有幸福的人生。

\subsubsection{守正待时,克服嗔嫉,心胸豁达,乐观豪爽}

1496 年,王阳明在会试中再度名落孙山。有人在发榜现场未见到自己的名字而嚎啕大哭,王阳明却无动于衷。王阳明说了一句很经典的话:“\textit{你们都以落第为耻,我却以落第动心为耻。}”人生中会遇到很多的艰难困苦,越是在这种时候越能体现人的心性修养。寻常人往往慌乱悲戚,唯有修养深厚者能做到泰然处之。王阳明还有一句话:“\textit{人须在事上磨,方能立得住;方能静亦定,动亦定。}”艰难困苦,正是对心性的最好磨砺。很多成功人士都经历过失败,甚至是反复的失败、多年的失败,不断总结经验,不断奋斗,最后才取得了成功,失败时他们不会气馁,而是会继续积累正能量,真正做到了“守正待时”,到了一定时候,能量积累足够了,就可以一飞冲天、一鸣惊人了。一定要懂得守正,生活中肯定会遭遇各种困难和挫折,守正不移,等到时机成熟,自然有翻身的一天。如果自暴自弃,自甘堕落,这样只会陷入恶性循环。克服嗔嫉也是非常重要的,嗔恨和嫉妒是两种很坏的念头,是非常强的负能量,有些戒友戒色后还在嗔恨和嫉妒,这样对戒色很不利,嗔心不改,嫉妒心重,戒色易破!戒色不仅是断除与邪淫有关的念头,而是所有负面的念头都要一并断除!因为只要是负面念头,都可能导致间接破戒,让你心理失衡,最后情绪失控而破戒。

心胸要豁达一些,心胸豁达,足能涵容万物,心胸狭隘,不能涵容一沙。心胸狭隘的人是永远看不见天堂的,他活在自己制造的自私的地狱里。心胸开阔,气量宏大,不计较个人的利害得失,这样的人才是真正的智者。智者都是心胸豁达之人,豁达是一种大度和宽容,豁达是一种品格和美德,豁达是一种乐观的豪爽,豁达是一种博大的胸怀、洒脱的态度,也是人生中最高的境界之一。沉舟侧畔千帆过,病树前头万木春,要做到乐观豪爽,恢宏大度,多积极思考,不要消极思考。我在答疑的过程中就发现很多破戒的戒友都在消极思考,对待失败,应该积极思考,多鼓励自己,从失败中吸取教训,不要再犯同样的错误,如果消极思考,认为自己怎么这么差,这么没用,那只会让自己越戒越差,要相信自己能行的!真正能戒掉的戒友,就在于他对待破戒的态度。他不会被破戒打倒,而是在破戒后认真反省,更加精进地学习戒色文章,从前辈的经验中找到自己的问题,然后就能越戒越好。面对暂时的失败,要乐观一些,人总会经历一些失败的,我之前也失败过很多次,也记不清多少次了,后来就是通过学习提高觉悟、练习观心断念才突破了怪圈。前辈的经验和方法已经摆在那里了,关键就是执行的力度,还有自己的悟性,这完全在每个戒友自己。一开始大家的水平都差不多,过几个月再看,有的戒友已经具备一定的觉悟了,而有的人连基本的思想误区都没有纠正过来,差距就是这么明显。就像跑步一样,过了一段时间,每个人之间的差距就出来了,这全在每个人自己的努力。

\subsubsection{以德御才,修身种德,养德远害,修心为上}

一位企业家说了这样一段话:“有德有才,破格重用;有德无才,培养使用;有才无德,限制录用;无徳无才,坚决不用。”领导最喜欢的是有德有才的人,这当然是最好的,最理想的,如果有德无才,领导也会尽量培养使用,因为有德行,所以领导自然会给机会。有才无德,就要限制了,德行差,就不会受领导的欢迎,这类人会给单位制造矛盾和混乱,所以要限制录用。无德无才,肯定不会用了,两个都不行,领导肯定不喜欢。有人是这样总结的:“德才兼备是正品,有德无才是次品,有才无德是危险品,无德无才是废品。”\textit{才德全尽谓之圣人,才德兼亡谓之愚人,德胜才谓之君子,才胜德谓之小人。(《资治通鉴》)} 有德无才领导还会培养一下,会用心教你,培训你,只要你态度好,还是有望胜任工作的。有才无德是危险品,这一点说得很对,很多人有学历有技术,但是没走正道,利用所学去犯罪了,是危险品。即使不去犯罪,这类无德之人到了单位,肯定是害群之马,会给单位带去负能量,制造不和谐的现象。领导其实最怕无德之人,无才还好一点,无才照样可以做一个好人,但无德就不行了,品行上直接坏了,充满负能量,这类人会到处树敌,到处诋毁别人,所以是危险品。

我们当然要争取做一个德才兼备之人,要以德御才,德是做人的根本。好好修身,改正不良习惯和不良习气,多行善积德,\textit{不责人小过,不发人阴私,不念人旧恶。三者可以养德,亦可以远害。(《菜根谭》)} 忠恕待人,养德远害,诗曰:造功积德培加深,忏悔自新福降临;寸尺进升天理在,根枝旺盛靠修心。秦东魁老师的《遇见智慧》里面讲到“人生祸福的核心:了解念头,观察念头,控制念头”,学会观心断念、控制自己的念头真的很重要。修心为上,修心为核心,修身为辅助,念头是行为的先导,修心是最根本的。光修身而不注重修心,是治标不治本的,就像治疗痘痘问题,如果内分泌不调整过来,光在皮肤外面涂药,是很难好的,也容易复发。戒色一定要注重修心,俞净意公外在的修身、行善等也没少做,但是潦倒终年,贫窘益甚,原因就是“意恶太重,专务虚名”。不懂得修心,还在争名利,这就是俞净意公穷困潦倒的原因。所以修心才是核心和关键,大德反复强调的核心就是修心,你把握住了核心,就能越戒越好,修行方面也会突飞猛进,不断取得突破。

\subsubsection{非德勿行,断恶修善,德义为重,谨慎对待}

不符合德行的事情不要去做,做事一定要符合德、符合义,符合天道、天理,这样才能问心无愧。当人非义而动、背理而行时,就会“大则夺纪、小则夺算”,而算减则贫耗,多逢忧患,最终算尽则死。我所了解的罪过中,邪淫和杀生可以说是非常严重的两个,果报都非常之惨。这两个一定要戒掉,否则人生真的危机四伏。不仅自己戒邪淫,还要告诉别人邪淫的危害,劝别人戒邪淫,不仅不杀生,还力所能及地放生,放生可以让心情舒畅,放生合天心,对于身心的恢复很有帮助。改造命运,断恶修善,一定要从意恶下手,意恶不除,做多少善事都是虚伪的,断恶要下大决心,要有力度!修善也是如此,要大力行善,恒久力行,不断对治意恶,不断坚持行善,这样一方面恶减少了,而善却在大幅增加,这样人生就会越来越好。中华传统美德很强调德和义,记得一位练武的师傅,他收徒的标准就是考察徒弟的德和义,重德贵义是很多师傅的收徒标准,就是考察一个人的德行、品行。我们做任何事都应该做到德义为重,强调德,强调义,做人做事都要突出这两点,不可轻浮草率,要谨慎对待,有礼有节,为人处世有原则、守规矩。

再讲下口德,口德好才能运势好,说高频的话,正能量的话,不说负能量的话,邪淫的话,伤人的话,这是我们应该具有的修养。

\begin{multicols}{2}
    \begin{description}
        \item[戒多言] 病从口入,祸从口出。说话不要太多,言多必失。\textit{大处着眼,小处着手,群聚守口,独居守心。……行事不可任心,说话不可任口。……禁大言以务实。(曾国藩)}
        \item[戒轻言] 不要轻率地讲话,轻言的人会召来责怪和羞辱。不要轻易向人许诺,轻易许诺而做不到,会丧失信用。
        \item[戒狂言] 不要不知轻重,口出狂言,胡侃乱说,这样往往后悔。
        \item[戒恶言] 不说恶语伤人,古语说,刀疮易没,恶语难消。恶言恶语给人心理上造成的伤害远甚于身体上的伤害。
        \item[戒矜言] 矜就是自大、自以为是。\textit{自伐者无功,自矜者不长。(老子)} 伐这里指自夸,自我夸耀的人抵消了功劳,妄自尊大的人不会长进,也会招致别人的厌恶。古人云:“自谦则人愈服,自夸则人必疑。”自矜自夸是涵养不够的表现。
        \item[戒谗言] 谗言就是背后说坏话,挑拨离间或者恶意诽谤、贬低和侮辱别人,谗言之人,皆是小人。
        \item[戒怒言] 生气的时候容易丧失理智,这时候说出来的话伤人伤己,生气的时候要学会冷静下来,不要乱说话,要三思而后行。
        \item[戒污言] 不聊性,不谈黄,不说放纵经历,不说黄色笑话,不说脏话。
    \end{description}
\end{multicols}

\subsubsection{积功累德,有德自安,德为人先,行为世范}

《太上感应篇》中很重要的一个教导就是强调积功累德,当立一千三百善,了凡先生发愿做三千善事,是道则进,非道则退,不履邪径,不欺暗室。做各种善事积功累德,古人的积善意识之强,真的达到了登峰造极的地步了,古德一直在强调改过积善,强调了几千年。我们戒色后一定也要学会积善,多做善事,你每天答疑或者宣传戒色,都是在积善,你写戒色文章分享戒色经验心得,也是在积善,积善有很多种方式,自己要培养积善意识,坚持去做,一点点积累,积累个几年就相当可观了,贵在坚持,然后做过后也不要放在心上,就像没做过一样,做好每一天,做好每一件小善。

劝世贤文:为善必昌,为善不昌,祖宗必有余殃,殃尽必昌。作恶必亡,作恶不亡,祖宗必有余德,德尽必亡。积金遗子孙,子孙未必能守。不如积阴德,以为子孙长久之计。人修一分德,便造一分福。诗曰:“自求多福是也”。人积一分功,便积一分禄。谚云:“无功不受禄是也”。人生福禄,生来注定,要想转移,除非积善无涯。世人未能修德以造福,积功以增禄,安望百福骈臻,百禄是荷?(百福骈臻:很多的福分、福祉一起到来;百禄是荷:承受各种福禄,“荷”在这里是“承受”的意思。)

有德自安这四个字是我由衷喜欢的,智者的气度是那么安定、从容,俯仰不愧,堂堂正正,光明磊落,而邪淫者给人的感觉就是急躁、暴躁,很容易生气,充满戾气和负能量,内心很不稳定。德行高深之士内心是安定祥和的,是非常稳定的,即使外在遇见不顺和挫折,也能有一个良好的心态去从容应对。诸葛亮在大军压境时,还能泰然自若地大笑,还若无其事地和别人交谈,然后胸有成竹地说出退兵之计,这种智慧和气度很值得我辈学习。人生在世,每临大事有静气,春风得意时不张狂,身处危难时不惊惧,就不愁事业不成功,面对困境要有一种处变不惊的大将风度,这种风度养之在平时,求之在德行,成就大事者必定成不骄、败不馁、顺不喜、困不惧,碰上顺境也好,陷入逆境也罢,总能沉心静气,理智处事,不慌不乱,从容应对。人生要进入更高的境界,必定要进入德的修炼,记得一位武学大家说过:“最高的武功是德行!”把对方打败了,别人可能还想报仇,还记恨你,如果在比武的时候很注重德行,既胜了对方又能表现出谦让、尊敬、友爱和团结,对方就心服口服了,甚至甘拜下风,想拜师了。所以,最厉害的还是德行。君子有德而自安,胸襟气度不同凡响,无论遇到好事还是坏事,都能拿捏得住,不会将情绪随随便便地写在脸上,总能给人一种镇定自若的感觉。

德为人先,要努力培养自己的德行,“先”这个字很好,就是不要在德行方面落后了。行为世范,要努力成为正能量的榜样,帮助和影响更多的人,给天地之间增加正能量。来到这个世界上我们都是有使命的,使命就是正己化人,行善积德,帮助沉沦色情与邪淫的众生重回正轨,这是我们的神圣使命,我们要有责任感与担当,坚持做下去,帮助一个是一个,你帮助了一个戒友,也许那个戒友能力很强,他将来也许可以帮助成百上千乃至上万的人,而他最初感谢的就是你。我们宣传戒色,帮助别的戒友,真的产生了巨大的影响,刚开始我体会不深,因为看到的反馈还不是很多,后来看的反馈案例越来越多,才真正感受到原来助人戒色能产生这么巨大的影响,可以真正拯救一个人,扭转他的命运,甚至他的祖先都要对你感恩戴德,因为你救了他们的子孙。戒色公益事业值得我们用一辈子的时间去坚守,我们是在做一件绝对有意义的事情,要以高度的责任感和强烈的使命感,坚持做下去。

\subsubsection{尊师重教,恭敬老师,善体师心,敬顺无违}

从一个人对老师的态度,就可以看出他最终能达到的成就。对待老师要恭敬,要有礼貌,要让老师欢喜,让老师受到尊重,要孝敬老师。古人云:“三教圣人,莫不有师;千古帝王,莫不有师”,人不敬师是为忘恩,何能成道?真正有德行的老师,值得我们用生命去尊重,有的老师即使有一些缺点,我们也应该学会包容,要看整体。当然对于邪师,我们是要远离的,老师之中也有败类,这要区别对待。对于品德高尚的老师,我们要恭敬对待,特别敬重,特别爱戴,特别拥护,信心和立场要绝对坚定。古往今来,尊师重教已成传统,代代相传,师恩当永远铭记!\textit{为学莫重于尊师(谭嗣同)},学习最重要的就是尊重老师。做人首先要从尊重父母、尊重老师开始,尊重老师才能真正学到老师传授的知识,尊重老师才能和老师和谐相处。要善体师心,不要让老师难堪,不要让老师生气,敬顺无违,注重和谐之道,要像重视生命那样重视师道。

\begin{quotation}\it
    “程门立雪”这一成语家喻户晓。它出自北宋著名理学家杨时求学的故事。杨时,将乐县人,天资聪慧,四岁会读诗经,七岁就能写诗,八岁就能作赋,人称神童。他十五岁时攻读经史,宋熙宁九年登进士榜。有一年,杨时赴浏阳县令途中,不辞劳苦,绕道洛阳,拜著名理学家、教育家程颐为师。时值冬季的一天,杨时因与学友游酢在对某问题有不同看法,为求正解而一起到老师家请教。他们顶着凛冽寒风来到程颐家时,适逢先生坐在炉旁打坐养神。杨时二人不敢惊动打扰老师,就恭恭敬敬侍立在门外,等候先生醒来。过了良久,程颐一觉醒来,从窗口发现侍立在风雪中的杨时和游酢,只见他们通身披雪,脚下的积雪已一尺多厚了,赶忙起身迎他俩进屋。此后,“程门立雪”的故事就成为尊师重教的千古美谈。

    子贡——尊师至诚孝道楷模,子贡,孔子杰出弟子。后弃官从商,成为孔子弟子中最富有者,商界历来公认他为“儒商始祖”。公元前 479 年,中国古代伟大的思想家、教育家——圣人孔子溘然长逝。孔子死后,众弟子皆服丧三年,相诀而去,独有子贡结庐墓旁,守墓六年,足见师徒情深,尊师之诚,实属中华尊师孝道楷模第一人。后人感念此事,建屋三间,立碑一座,题为“子贡庐墓处”。因子贡为孔墓所植为楷树,后世便以“楷模”一词来纪念这位圣徒。子贡尊师达到这种程度,真的让人感慨万千,感动无比。

    汉明帝刘庄放下九尊之躯尊师,汉明帝刘庄,东汉第二位皇帝。明帝在位期间,吏治非常清明,境内安定团结。博士桓荣是汉明帝做太子时的老师,而明帝对老师一向非常的尊敬,后来他继位作了皇帝“犹尊桓荣以师礼”。有一次,明帝到太常府去,在那里放了老师的桌椅,就请老师桓荣坐在东边的方位,又将文武百官都叫来,当场行师生之礼。桓荣生病,明帝就派人专程慰问,甚至亲自登门看望,每次探望老师,明帝都是一进街口便下车步行前往,以表尊敬。桓荣去世时,明帝还换了衣服,亲自临丧送葬,并将其子女作了妥善安排。明帝能放下自己九尊之躯的至高身份来恭敬老师,可见他的用心与风范,值得大家学习。

    岳飞敬师如父,岳飞幼年家境贫寒,无钱上学。但他非常好学,常在私塾窗外听课,无钱买纸笔,就以树枝为笔,大地为纸。私塾老师周侗免费收岳飞为学生,教育他做人的道理,每逢单日习文,双日习武。岳飞不负师教,勤学苦练,文武双全。学得射箭绝技,能左右开弓,百发百中。周侗去世后,岳飞披麻衣,驾灵车,执孝子之礼,以父礼安葬他。且在朔望(初一、十五)之日,无论在外行军打仗,还是驻扎营中,他都要祭拜自己的恩师,每次痛哭之后,必定会拿起老师所赠的“神臂弓”,射出三支箭。岳飞说:“老师教我立身处世精忠报国的道理,还把他一生摸索的箭法和武艺都传授给我,师恩是我一生都不能忘怀的。”
\end{quotation}

《华严经》讲的“敬师九心”:\begin{multicols}{2}
    \begin{itemize}
        \item 视师如父母的“孝子心”,
        \item 诚心不变的“金刚心”,
        \item 负载重任的“大地心”,
        \item 风雨不动的“山岳心”,
        \item 忠诚服务的“仆人心”,
        \item 谦虚恭敬的“下人心”,
        \item 承受上师负担的“车乘心”,
        \item 能忍辱而不背叛的“义犬心”,
        \item 为上师而风里来、浪里去,从不厌烦的“航船心”。
    \end{itemize}
\end{multicols}

能做到敬师九心,在修行方面一定会取得很大的进步或成就。修行方面也特别看重一个人的德行,贤德之人很容易接受教诲,也愿意按照教诲的要求去做,这样就能不断地进步。如果对老师不尊重,那肯定无法获得更深的法益,\textit{不敬勿说法(《毗奈耶经》)},老师很注重考查学生的德行,对于德行差的人,肯定会有所保留,不愿意倾囊相授,而对于德行好的学生,肯定愿意把所有的知识、窍诀全部传授给他。首先自己必须是一个合格的法器,这样才有资格接受佛法的甘露。\textit{一切罪从忏悔灭,一切福从恭敬生。(黄念祖老居士)}尊重老师,恭敬老师,显得尤为重要,能尊重能恭敬,就能得到老师的真东西,不管是世间的教育还是出世间的教育,都把尊师重教放在极其重要的位置。

\subsubsection{常念师恩,让功于师,功成弗居,抱朴守拙}

滴水之恩,当涌泉相报,对于老师的恩德要铭记在心,不能忘记,要常念师恩。感恩老师是一种流传千古的美德,古人从很早就已经懂得这个道理,每一个朝代的文人志士都会把感恩老师、感恩师德,作为对自己道德品质的一个基本要求。不管自己取得了多大成就,都不能忘记老师,都要把老师放在最上面,让功于师!让功于圣贤!这是规矩!懂得让功,是一种很深的大智慧。面对领导的赞扬,要让功于领导,“都是领导栽培的。”让功于同事,“都是大家一起努力的结果”,这样就能得人心。前段时间看到一张海报,说国家这些年的发展全部归功于人民,领导者不居功,而是归功于人民,这是得民心的做法。智者不居功,而是让功,居功者往往自傲,是在强化自私的小我,而让功是真我的品质,是无私的,是最得人心的一种做法。一,不可居功,居功易自傲;二,不可摆功,摆功就是炫耀功劳。要懂得让功,对于取得的一点成绩、成就,让功于别人,也不要放在心上,不要去想它,就像没做过一样。《道德经》里面讲:“\textit{功成而弗居。}”这里面有着非常深的智慧,在功成之时,往往是一个人最容易得意忘形、最容易骄傲膨胀的时候,这个时候如果处理不当,就可能引来祸患,甚至是杀身之祸。历史上就有这样一位战功赫赫的将军不懂“功成弗居”而白白丢掉了性命,他就是年羹尧,立了大战功,志得意满,完全处于一种被奉承被恩宠的自我陶醉中,进而做出了许多超越本分的事情,他在皇帝面前,态度竟也十分骄横,“无人臣礼”,狂妄、嚣张、自大、目中无人,连皇帝都不放在眼里,最后被削官夺爵,列大罪 92 条,赐自尽。曾经叱咤风云的年大将军最终落此下场,实在令人扼腕叹息。德不配位,必有灾殃,爬得越高越考验一个人的德行,德行出问题,迟早要栽大跟头。

做人的最高境界莫过于抱朴守拙,\textit{抱朴守拙,涉世之道。(《菜根谭》)} 大智若愚,大巧若拙,老拙是旧时老年人自称的谦词,这个词汇我很喜欢,老拙,一个老字,有深厚的味道在里面,智慧够老,够深;一个拙字,超级质朴,超级厚道,也很耐人寻味。\textit{老拙穿衲袄,淡饭腹中饱,补破好遮寒,万事随缘了。有人骂老拙,老拙只说好;有人打老拙,老拙自睡倒;涕唾在面上,随他自干了,我也省力气,他也无烦恼,这样波罗蜜,便是妙中宝。若知这消息,何愁道不了。(弥勒菩萨偈)} 弥勒菩萨也自称老拙,一个拙字还有童真的趣味在里面,小孩子刚开始学走路就有点笨拙,写字画画都显得笨拙,这是一种天真质朴的境界。抱朴守拙,不耍小聪明,回归最原始的本真,保持朴实的本性。抱朴守拙的状态显得那么平凡、普通、简单和低调,这是最符合道的一种状态。在印光大师身上我看到了这种特质——抱朴守拙,我个人认为这是最高的一种特质,也是最感动人的一种特质,就是超级质朴,质朴到极致了,诚恳到极致了,慈悲到极致了。

纪录片《三十二》和《二十二》感动了我,是非常好的纪录片,这是两部片子,数字从《三十二》变成了《二十二》,是因为短短两年,幸存者中有10位老人相继辞世。里面一位叫“韦绍兰”的老人感动了我,虽然她不是修行人,但她也有抱朴守拙的特质,让我想起了我的奶奶,说话,走路,干活,坐在那里,都能体现出这种纯真质朴的特质,非常感人。抱朴守拙真的是非常高的境界,这是最高也是最后的品质,所有好的品质都包含在抱朴守拙里面了,是最终极最感人的一个品质。前几天看新闻,韦绍兰老人也走了,享年 99 岁,一生饱经沧桑,愿她一路走好。

\paragraph{总结}

学高为师,德高为范,人无德不立,国无德不强,一身正气冲天地,两袖清风鉴古今。戒色必须要注重培养自己的德行,我看到不少人戒到一定天数出现破戒,就是因为德行有亏,德行没跟上。戒到一定程度就开始骄傲自满,或者嗔恨心、嫉妒心、傲慢心、贪心、名利心重,导致内心失衡、情绪失控而破戒。以自我为中心的动机,戒到一定天数也容易出现破戒,一定要以正己化人、无私奉献为动机,这样才能戒出责任感、戒出使命感、戒出崇高感,才能戒得稳定长久。德行就像地基,造高楼对地基的要求很高,楼越高,对地基的要求就越高。戒色大厦也需要稳固的地基,我们一定要夯实德行的地基,这样戒色大厦越来越高的时候,就有坚实的保障。否则戒到一定天数,真的要成危楼!一旦垮塌下来,会很惨烈,一些戒色前辈就是德行有亏,最后出现破戒,甚至充满负能量。德行实在太重要了,要不断加固这个地基,用钢筋水泥去浇灌,这里的“钢筋水泥”指的是对治负面心态,多发善心,多做善事,多培养自己的德行,不断增强正能量。人生存于世间,平时能够培养自己的德行,能够自我反省,自我约束身、口、意,忏悔自己的过错,勇敢改过,规范自己的行为,做个安份守己的人。在家孝顺父母,敬重五伦,心存慈悲,在外敬老尊贤,尊师重教,谦虚待人。这才是一个堂堂正正的人,走到哪里都能立得住,站得稳!

\textit{子曰:“吾未见好德如好色者也。”} 好色是很容易犯的毛病,自古以来就是如此,而君子则更注重德行的培养,历史上也有很多君子的典范,面对诱惑能够做到不动心,能够把持得住,能够以凛然正气来回应。我们当然要学习好的榜样,这样才能有一个正能量的人生。如果真能把好色的这股劲头,这种疯狂的投入、极度专注的精神、不知疲倦的状态,用在学习、工作或修道上,那肯定会有一番大成就。天道福善祸淫,好色必然会产生很多负面的念头,最终其实害了自己。戒色后一定要学习圣贤教育,做一个贤德之人、孝顺之人、谦虚之人、感恩之人。学习最重要的是端正学习态度,制定自己的日课,坚持不能中断,能成功戒掉的人基本都有自己的日课,都在一天天地坚持,坚持到一定天数就会有顿悟,就会有更深入的认识和理解。正如荀子所说:“\textit{积土成山,风雨兴焉;积水成渊,蛟龙生焉;积善成德,而神明自得,圣心备焉。故不积跬步,无以至千里;不积小流,无以成江海。骐骥一跃,不能十步;驽马十驾,功在不舍。锲而舍之,朽木不折;锲而不舍,金石可镂。}”荀子在《劝学》的开篇就说:“\textit{学不可以已。}”读书是一个需要长期坚持的过程,能坚持、注重积累、注重复习的人必定会戒色成功。不在学习了多少,关键是吸收了多少,真正落实了多少!记得一位戒友虽然只看了不到三十季的《戒为良药》,但他把有的文章看了十遍、二十遍乃至三十遍,真正吃透,真正落实,真正内化成自己的戒色意识和实战意识,这样他戒了一年多也没破戒,他做到了,之前也是屡戒屡败。后来他明白了学习的重要性,也顿悟了观心断念的重要性,于是他突破了怪圈。他很重视吸收率,看再多的文章,吸收率低下,看完就忘,实战还是和过去一样,那怎么能行?真正吸收和落实,才能有质的改变和飞跃。

之前有戒友建议我讲一下为人处世,这季已经专门讲到了。做人是一门很深的学问,值得终生去学习和领悟。孔子在做人上最大的主张,就是要做君子。在短短两万多字的《论语》中,“君子”这个词竟出现了一百多次。孔子在为我们勾勒出人们心目中理想的君子形象的同时,也提出了君子最基本的人格标准:做一个有德行、有正气、有担当的人。不要做小人,\textit{子曰:“君子泰而不骄,小人骄而不泰。”} 君子宽宏大量,胸襟开阔,光明磊落,舒泰自如,绝不骄傲。小人虽然表面骄傲,但是内心却是自卑的,所以心境就不泰然了。君子坦荡荡,小人常戚戚,君子心胸开阔,神安气定,不忧不惧,胸襟永远是光风霁月,像春风吹拂,清爽和畅;如秋月挥洒,皎洁光华。所以“坦荡荡”。小人则是斤斤计较,患得患失,不是觉得别人对不起自己,就是因为某件事对自己不利而忙于算计,受各种利欲所驱使,经常陷入忧惧之中,所以总是“长戚戚”。君子为何能“泰”?因为君子有无私和宽容的心态,不以物喜,不以己悲,先天下之忧而忧,后天下之乐而乐。君子有所为但无所求,一切尽人事以听天命,只问耕耘,不问收获,无私而纯粹,但求无愧于心,无愧于天,一切随缘,根本就不把结果放在心上,如此的豁达,还怎么会骄傲呢?小人为什么“骄”?因为小人心中惦记的总是个人的利益得失,常常处于焦躁不安中,外表就少了一份气定神闲。他们在外表现的自负和高傲,不过是在掩饰内心的空虚和不自信。小人最怕的就是别人不认同自己,看不起自己,所以才表现出一种盛气凌人、毫不在乎的态度,来掩饰自己内心的不安。正因为心态不同,君子和小人所流露出来的外在气质和气度也就不同,他们的为人处世方式也不一样,人们的评价也当然就不一样了。君子无私,小人自私,君子喻于义,小人喻于利,一个真正的君子,必然是具有高尚精神追求的人。孔子说:“\textit{朝闻道,夕死可矣}”,“\textit{君子谋道不谋食}”,“\textit{忧道不忧贫}”,他称赞学生颜回:“\textit{贤哉,回也!一箪食,一瓢饮,在陋巷。人不堪其忧,回也不改其乐。}”孔子自己也是这样,“\textit{饭疏食饮水,曲肱而枕之,乐亦在其中矣}”。真乐不在邪淫中,而在平淡中,心灵净化、品德高尚之人,能体会到真正的大快乐。对于品德高尚的君子来说,快乐不在于物质追求,不在于沉迷色情,而在于心灵的纯净,返璞归真,符合大道。君子能够安贫乐道,孔子说:“芝兰生于深林,不以无人而不芳。君子修道立德,不为穷困而改节。”我们要做戒色的君子,君子必然要戒色,戒色是君子第一修为,君子内心非常庄重、恭敬,邪淫之人内心充满了龌龊、肮脏、不可告人的邪念。孔子专门强调了戒色,从少年时代就要懂得戒色,邪淫时那么猥琐,那么充满负能量,根本称不上君子,君子必然要拒邪淫于千里之外,一派凛然刚正的气象。

君子有四大修养:\begin{adjustwidth}{-1em}{-1em}
    \begin{description}
        \item[自省] “\textit{君子求诸己,小人求诸人}”,遇到问题先找自己的原因,而不是只知责备别人。
        \item[克己] “\textit{克己复礼为仁}”,克制自己的欲望和不正确的言行,自觉遵守操守和规矩。
        \item[慎独] “\textit{君子慎其独也}”,在别人看不到的时候和地方,也能保持表里如一的操守、品格和仪态。
        \item[宽人] “\textit{躬自厚,而薄责于人}”,“\textit{己所不欲,勿施与人}”,凡事要推己及人,将心比心,设身处地为他人着想。
    \end{description}
\end{adjustwidth}

曾国藩在日记中写道人有四种福相:\textit{端庄厚重是贵相,谦卑含容是贵相。事有归着是富相,心存济物是富相。} \begin{description}
    \item[第一福相:端庄厚重] 做人要有厚的效果。\textit{君子以厚德载物。(《易经》)} \textit{大丈夫处其厚不处其薄。(《道德经》)} \textit{君子不重则不威,学则不固。(《论语》)} 做人要稳重、厚重、庄重,一个人不庄重,就会轻佻,没有威严,流于散漫,心不能自守,所学的东西也不会牢固。端庄厚重的人,表明他懂得敬畏,一个有敬畏感的人就不至于放肆无忌,思虑就会深远,处事不至于鲁莽,说话就会谨慎,交际不至于随便。自己有敬畏感的人,往往别人也会敬畏他。曾国藩多次申斥儿子力戒轻佻,多多修炼些举止。据后人记载,曾国藩“行步极厚重,言语迟缓”。他走起路来脚步很沉稳,说话很慢,但一句是一句,每一字都有一种打动人心的力量。明代大思想家吕坤说:“\textit{深沉厚重是第一等资质,磊落豪雄是第二等资质,聪明才辩是第三等资质。}”
    \item[第二福相:谦卑含容] 为人谦卑,待人宽容,从做事的角度讲,这样的人必然更容易成功。
    \item[第三福相:事有归着] 所谓事有归着,就是办事沉稳有着落。言下之意也是脚踏实地。用现在的话说,就是要抓落实。
    \item[第四福相:心存济物] 心存济物包括关心他人、关心社会、关心天下。有一颗慈善之心,懂得帮助他人其实就是帮助自己,这样的人即便物质上不富有,精神上也是大富大贵者。心存济物,就是“达则兼济天下”,一个人心存济物,心量就大了,格局就大了。
\end{description}

君子六德:\begin{multicols}{2}
    \begin{description}
        \item[做人] 对上恭敬,对下不傲,是为礼;
        \item[做事] 大不糊涂,小不计较,是为智;
        \item[对利] 能拿六分,只拿四分,是为义;
        \item[恪律] 守身如莲,香远益清,是为廉;
        \item[对人] 表里如一,真诚以待,是为信;
        \item[修心] 优为聚灵,敬天爱人,是为仁。(优为聚灵,我的解释是优为:优秀的行为,合乎礼仪,合乎道德规范;聚灵:聚集灵气,聚集能量。)
    \end{description}
\end{multicols}

\begin{quotation}\it
    子曰:“君子有三德:仁而无忧、智而不惑、勇而不惧。”

    子曰:“君子有三畏:畏天命,畏大人,畏圣人之言。”
\end{quotation}

君子,德才兼备,有所为有所不为,穷则独善其身,达则兼济天下。君子处世,应像天一样,刚毅坚卓,发愤图强,永不停息;君子为人应如大地一般,厚实和顺,仁义道德,容载万物。君子有四不:第一、君子不妄动,动必有道;第二、君子不徒语,语必有理;第三、君子不苟求,求必有义;第四、君子不虚行,行必有正。君子说话必定有其道理,会要求自己谨言慎行,凡事讲求合乎礼仪、不随随便便,每当有所行动,必定有其用意,此即所谓不妄动,动必有道;君子非礼勿言,守口如瓶,不说空话,不讲不实在的话。但在该说的时候也必定会说,因为应说而不说,有失于人;不应说而说,则是失言。要做君子,必须不失人也不失言。君子所言,都是有意义的话,慈悲的话,正义的话。所以君子不徒用语言,说话必定有理;君子不会以苟且之心妄想获利,更不会落井下石,谋求个人私利,豪取强夺。君子假如有所求,一定是为了国家、为了社会、为了正义;君子的一言一行都不会随便,凡事他都会经过再三的考虑,想清楚了,他才会有所行动,所以君子的行为必定合乎正道。讲真实的语言,起正直的念头,说正直的话、做正直的事。

这季分享了 256 个字的戒色德训,希望大家能够背诵下来,作为日课每天看看,读一遍后的感觉是非常微妙的,仔细体会那种崇高的感觉,充满高频的正能量的振动。这篇文章我分享的内容很多,也把一些摘抄的笔记整合在一起分享给大家了,希望大家能够注重善行、德行的培养,善日加修,德日加厚,在德行方面能够更加扎实和稳固,这对于戒色、对于为人处世都是很有好处的。当然我也实话实说,很多时候我自己也做得不够好,我也要认真反省、忏悔和改进,我自己也需要不断地完善和提升德行,这是一辈子的修为。希望我们一起努力,一起来完善自己,帮助别人,做戒色的君子!活出圣贤的教诲,无愧于祖先,无愧于天地,做一个堂堂正正的炎黄子孙。不为自己求安乐,但愿众生得离苦,正己化人,无私奉献,让我们坚持到底!

最后再分享一个最近看到的案例,用在这季正好说明问题,真的很发人深省。

\begin{case}[戒色三年福报现前,迷惑放纵又失去一切:戒到最后拼什么?]
    大家好,以下是我真实的经历,成功戒色三年终得福报现前,然而被福报迷失了自我又失去了一切。我是小学三年级在一个偶然的机会偷看了父亲藏在家里的 A 片而学会了手淫,从此以后就一发不可收拾,开始了疯狂的手淫,更严重的是我还传播手淫,教会了很多小伙伴们一起手淫,现在想想真是害了他们,我罪大恶极。手淫所有的症状都在我身上淋漓尽致地呈现了:性格内向自卑、社恐、胆小、抑郁等等。身体方面更是非常的衰败:记忆力减退、右肾疼痛、心脏早搏、早泄、精索静脉曲张等等。人际关系更是差到了极点,和我一起的所有人都讨厌我,内心很痛苦。技校毕业后,一个偶然的机会学习了佛法,认识到了邪淫的危害,也知道自己为什么这么倒霉和痛苦了,就开始戒除邪淫,并积极行善修福,就这样做了整整三年,没有破过一次戒,我是真正地做到了。期间工作虽然一直都很差,做过饭店传菜员、搬运工、电子厂流水线、保安等。但是手淫后遗症的症状有了很大的改善,人变得很自信阳光,每天都很快乐,气场也很足,也不那么惹人厌,身体方面也恢复得不错。也许是我这三年努力戒色修善积福,结果有了回报,在我上海当保安的时候,当时感觉有种莫名其妙的力量促使着我要去网上发份简历,我就在网上随便发了一份简历,还不是招聘网站上发的,我当时心想,就我这种低能力的人,有人用我才怪呢。结果第二天真的有人打电话让我去面试,还是一家特别大的台企公司,去了之后人事先面试我,之后副总面试,最后老板亲自面试我,一路绿灯就这么神奇通过了。当时老板面试我的时候,我说这个岗位我不会,老板说不会没关系,我可以亲自教你。就这样我顺利地成为了这家公司的物流高管,并且由老板亲自带领。幸福来得太猛,一下从最低层的保安成为了一家公司的高管,物流高管这个岗位也是公司的肥差,事少、权利大、自由、油水多。我也是在这期间又开始堕落了,我管理着四家物流供应商,他们每周都会请我吃饭按摩唱歌,刚开始不习惯,到后来已经自然麻木了,逢年过节送红包更不是少数。福报来得太大,自己又飘然了,忘了之前的教训,饱暖开始思淫欲了,开始找女朋友,同时谈两个,最多的时候三个,不断在女朋友们身上花钱,经常性地开房,最后又不断地伤害她们的感情。就这样浑浑噩噩地过了三年,期间工作上经常出问题,老板那边我已经得不到重视,还差一点被开除,性格上又开始变很自卑抑郁起来,身体也变差了,每天困顿无精打采。由于不珍惜自己修来的福报,这个时候北京有一个朋友要开公司,让我去帮忙,当时也鬼迷心窍了,以为自己能力很牛似的,毅然辞了这么好的工作去了北京。其结果就是在北京经历了两年的非人生活,住着没有窗口的房子,拿着最低的工资,干着最低端的工作,还天天加班到晚上十点。由于我能力太差,让我去帮忙的朋友后悔让我去了,又因是他叫我来的,又不好意思开除我,整天骂我没能力,排挤我,我自己又胆小,习惯了坐办公室工作的我,如果真不干这份工作,出去不知道能干嘛,所以挨骂的我又厚脸皮地不走,内心是多么的痛苦和煎熬啊!每天上班的心情特别的痛苦。真是后悔在上海的时候不知道珍惜自己的福报。现在北京的公司已经开不下去了,我那个朋友还欠我三个月工资没给我。我已经在家待了半年没工作了。回想一切都是自己自作自受啊,但是我不气馁,因为我成功戒了三年,福报现前,自己不珍惜又失去了一切,这么大的教训,我会拼命努力三年,去恢复我的福报。望大家在戒色的路上一直坚持,即使你成功了,做人做事也要时刻小心,不忘却那善的根本,做人每天都要战战兢兢,如履薄冰,加油!
    \subparagraph{附评} 这位戒友的帖子标题就有“戒到最后拼什么?”其实我之前的文章反复说过,拼的就是德!德不配位,必然倒霉,脑子一起负面的念头,能量场就变差了,各种不顺和倒霉的事情就要来收拾自己了。印光大师说过八个字,我记忆很深,那就是“富贵迷人,可畏之至”。没钱的时候还好,一有钱就开始变坏了,又开始堕落了,富贵之后是很危险的,如果德行跟不上,没有很强的原则和底线,肯定会再次堕落,因为是明知故犯,所以更加倒霉。这位戒友之前戒色修善三年,福报起来了,人也变得很自信阳光,每天都很快乐,气场也很足,老天给机会了,一个好的职位在向他招手了。有德无才,培养使用,他当时的善行和能量场,注定那个职位是他的,即使没才,老板也亲自来教,慢慢就学会了。一个人的善行和正能量是多么重要,直接会匹配好的职位给自己,从底层的保安到公司的高管,真的是鲤鱼跳龙门了,估计年薪应该有几十万了。可惜他没把持住,忘记了过去邪淫的教训,没有在德行方面继续培养和提升,也可以说是忘本了,又开始邪淫了,后来的遭遇可想而知,只会越来越差。负能量一多,领导就看不顺眼了,很容易被领导骂,他甚至差一点被开除。邪淫后不管脑力、精力、精气神都大幅下降,每天无精打采,工作方面的效率肯定会大幅下降,领导自然就有意见了。这么好的福报被他糟蹋了,实在太可惜了,如果他能在飞黄腾达之时,牢记过去邪淫的教训,并且继续加强自己的德行,继续行善积德,也许将来副总的位子就是他的,也许到时年薪就有百万了,可惜他没把持住。厚德载物,德行是地基,也是事业的基础,邪淫伤身败德,直接把地基给拆了,这样人生肯定会垮塌下来。

    这位戒友后来去了北京,住着没有窗口的房子,拿着最低的工资,干着最低端的工作,还天天加班到晚上十点,还整天被骂,被排挤,最后失业在家。和他当公司高管的时候,简直一个天一个地,落差太大了。上天看他这么堕落了,又剥夺了他的一切,其实也是他自作自受,咎由自取,起负面的念头,做负面的事情,自然会感召一个负面的结果。在人生成功之时,一定要加倍小心,因为这时很容易迷失自己,很容易再次放纵,一定要小心谨慎,继续培养德行,稳固地基。戒到最后拼的就是德,小戒靠忍,大戒靠悟,至戒靠德。德不行,就会垮塌下来,德跟不上,就很危险,爬得越高,摔得越惨!这个案例的戒友也算大起大落了,之前他成功戒了三年,当正能量充足时,人生真的会风生水起,当负能量爆棚时,人生就会陷入低谷。在成功时能把持得住,需要很高的修为和定力,也需要很高的觉悟和警惕,知道这个时候不能乱来,否则后果不堪设想。果卿居士在开示中说过这么一句话:“贪色漏得最快!”这句话我当时做了笔记,后来复习到这一条,一下就记住了,贪色挖身最猛,漏福报最快,即使你有几千万的福报,都能给你漏光!败光!让你的人生彻底陷入困境。我发现一个现象,就是很多公司或者企事业单位的某些职位是会经常换人的,职位很高,收入很丰厚,但这种位子不是随随便便可以坐上的,需要在过去培植很深的福报,做了很多善事,有相当的正能量和德行的人,才能坐得上、坐得稳的。没有善行,没有德行,没有正能量,是坐不住的,迟早会下来。犯了邪淫或者做了其他坏事,就很难坐稳了,甚至会被领导开除,生活的境遇会越来越差,人际关系也会越来越差,一个人的能量场变成负面了,走到哪里都会倒霉。这个案例的教训实在很深刻,分享给大家,可以起到很好的警醒作用,希望大家都能加强德行的培养,好好夯实德行的地基,不管对于戒色还是对于以后的人生、事业和家庭,都是极端重要的。

    \textit{故吉人语善、视善、行善,一日有三善,三年天必降之福。凶人语恶、视恶、行恶,一日有三恶,三年天必降之祸,胡不勉而行之?(《太上感应篇》)} 这位戒友之前戒色三年,没破过一次戒,行善修福也做了整整三年,期间工作虽然一直很差,但他还是坚持在做,结果上天赐福了,面试一路绿灯,当上了公司高管,低能力的人怎么可能当上公司高管?这非常不可思议,因为他戒色修善了三年,天必降之福!如果做了三年,没降福,可能的原因就是:\begin{multicols}{2}
        \begin{itemize}
            \item 出现破戒
            \item 贪求福报,一不如意就怨天尤人
            \item 内心其他负面念头没断除,或者做了其他恶事
            \item 不孝顺父母,不懂得感恩
            \item 不谦虚,骄傲自满
            \item 行善的力度不够大,量不够多。
        \end{itemize}
    \end{multicols} 做善事要本着无私奉献、不求回报的心态去做,只问耕耘,不问收获,要好好孝顺父母,谦卑自牧,多行善积德。这样戒色修善三年,天必降之福!\textit{救人之难,济人之急,悯人之孤,容人之过。广行阴骘,上格苍穹。人能如我存心,天必赐汝以福。(《文昌帝君阴骘文》)} 《太上感应篇》和《文昌帝君阴骘文》实在太经典了,也太重要了,对这两篇文章要高度重视起来。
\end{case}

下面分享一首诗歌。

\begin{poem}[从猥琐到崇高]
    \begin{multicols}{2}
        \begin{center}~\\
            还记得小学早读时的“子曰” \\ 朗朗上口,读起来很有古韵也很优美 \\ 纯真的孩子们一起读着圣贤教育 \\ 那种氛围和感觉真的很好 \\ 子曰,这两个字深深地印进了灵魂深处 \\ 长大了,我远离了子曰 \\ 开始沉迷于色情,沉迷于手淫恶习 \\ 沉迷于放纵堕落的生活 \\ 满脑子的龌龊邪念 \\ 圣贤教育离我十万八千里 \\ 我完全迷失了 \\ 曾经清澈明亮的眼睛暗淡了、混浊了 \\ 曾经纯真美好的笑容消失了 \\ 活得浑浑噩噩,宛如行尸走肉 \\ 直到神经症彻底爆发 \\ 才把我猛然敲醒 \\ 邪淫时我是那么猥琐、阴暗和充满戾气 \\ 也是那么无知、愚蠢和荒唐 \\ 当我再次下决心戒色时 \\ 我开始学习圣贤教育和中医养生知识 \\ 开始发善愿,起善念,开始行善积德 \\ 我彻底告别了负面低频率的状态 \\ 我重生了,蜕变了,逆袭了 \\ 一种崇高的表情出现在我脸上 \\ 替代过去猥琐不堪的感觉 \\ 一股凛然正气把我撑起来了 \\ 从猥琐到崇高,从自私到无私 \\ 我完成了不可思议的转变 \\ 我无比感恩圣贤教育 \\ 圣贤教育就像一位充满智慧的老人 \\ 是那么慈祥、亲切、和蔼和平易近人 \\ 他在等待我回头,等待我觉醒 \\ 并且帮助更多的人完成逆袭 \\ 正气一生难可贵,清风半世不凄凉 \\ 不为自己戒邪淫,但愿众生归正途
        \end{center}
    \end{multicols}
\end{poem}

下面推荐一本书。

\begin{book}[《星云大师谈处世》]
    星云大师,江苏江都人,1927 年生,为禅门临济宗第 48 代传人。12 岁于南京栖霞山寺出家,1967 年创建佛光山,弘扬“人间佛教”,树立“以文化弘扬佛法,以教育培养人才,以慈善福利社会,以共修净化人心”的宗旨。星云大师是我比较喜欢的一位大德,在为人处世方面给出了很多建议和指导,很值得我们学习。星云大师的那一口方言让我感觉很朴实和亲切,大师也很和蔼。那个跳探戈的开示给我留下了很深的印象:\begin{quote}
        人我之间的探戈要跳得好,必须细心留意彼此的步调,知进知退,通权达变,不但不能踩到对方的脚,而且也不能让对方踩到自己的脚。人生的道路本来就是有来有去,有进有退,像待客的妙方是送往迎来;周全的礼貌是礼尚往来;而座谈会议若能问者、答者环环相扣,才会趣味横生;彼此闲聊必须说者、听者应对如流,才能宾主尽欢。大自然的事物也是来往复始,循环不已,像严冬一过,春天跟着来临;太阳下山,星月接着高挂天空;旧的一年逝去,新的一年接踵而至;枯叶落尽,枝桠继续抽出嫩叶。宇宙人生就在这一来一往,一进一退之间,处处显得生机盎然。凡有去无回者,大多不是好事,像射出的箭一去无回,必定会有死伤;攀登高山一去无回,往往凶多吉少。
    \end{quote} 为人处世的确要知进知退,通权达变,这样才能生机无限,才能在生命的舞台上随缘放旷,挥洒自如。李开复去见星云大师,大师直接点出了问题的核心——追求名利!名利心太重。李开复:“为了追求更大的影响力,我像机器一样盲目地快速运转,我心中那只贪婪的野兽霸占了我的灵魂,各种堂而皇之的借口,遮蔽了心中的明灯,让我失去准确的判断力。……身体病了,我才发现,其实我的心病得更严重!当我被迫将不停运转的机器停下来,不必再依赖咖啡提神,我的头脑才终于可以保持清醒,并清楚地看到,追逐名利的人生是肤浅的,为了改变世界的人生是充满压力的。珍贵的生命旅程,应该抱着初学者的心态,对世界保持儿童般的好奇心,好好体验人生;让自己每天都比前一天有进步、有成长,不必改变别人,只要做事问心无愧、对人真诚平等,这就足够了。如果世界上每个人都能如此,世界就会更美好,不必等待任何一个救世主来拯救。”李开复顿悟了,过去太执著于名利,其实是一种自私的心态,是一种贪心。自私的心态迟早会让自己陷入深坑,陷入恶性循环,贪心是无止境的,而身体健康是有限的,不顾身体健康去追求名利,是愚蠢也是肤浅的。看淡名利,转变自私的心态,多行善积德,学会无私奉献,积累正能量,这样的人生才能越走越好。星云大师还有一句话给我的印象也很深刻,那就是“要产生正能量,不要产生负能量。”也就是要发善念,做善事,不要发恶念,做恶事。这是人生最根本的指导,诸恶莫作,众善奉行。希望大家好好践行之。
\end{book}
