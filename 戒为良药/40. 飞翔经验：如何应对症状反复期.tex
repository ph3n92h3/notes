\subsection{如何应对症状反复期}

\paragraph*{前言}

这季前言分享两个案例。

\begin{case}
    本人手淫十二年了,现在全身都疼,全身是病,连重活都干不了。长时间手淫导致现在 JJ 变得很小很小,跟老婆过性生活的时候根本勃不起来,老婆马上要跟我离婚了!因为手淫我变得很自私,相貌也变得很丑陋,24 岁像 42 岁的样子,脸上全是胡须、雀斑。我真的不行了,生活事业一团糟,干啥都不成。现在腰疼得像有人用刀在割我的肉,好难受啊!现在我还有手淫,昨天晚上还撸了两次,今天头都炸了,我感觉快死了,谁来救救我啊!

    \textbf{附评} 这位戒友已经伤得很重了,但是他的心瘾实在很强,腰疼到这种程度,还在疯狂自残,简直就是不要命了!手淫十二年的报应实在太惨烈了,全身是病,彻底阳痿,还要面对离婚的打击。邪淫的人会变丑变猥琐,这已经是公开的秘密了,相由心生,当一个人经常沉迷于邪淫的内容,最后必将变得丑陋不堪,容貌也会迅速老化,一副邋遢龌龊的糟男形象。很多人已经把自己撸成了行尸走肉,常言道,不见棺材不掉泪,但有些人见了棺材还不知悔悟,还在麻木不仁地摧残自己,这种疯狂自毁的状态,实在太可怕了!这位戒友的 JJ 也萎缩了,之前就有好几位戒友反馈过这个问题,说以前自己的 JJ 完全正常,后来肾气亏损后,长度就缩水了,就像泄了气的气球一样,不仅 JJ 会萎缩,睾丸也可能出现萎缩或者下垂。很多人在泄精后,都会出现后脑勺疼痛的表现,中医讲到肾上通于脑,手淫挖身掏脑特别厉害,时间长了肯定会出现头痛的表现。这个案例的戒友应该出现神衰了,出现神经症之后,一般都会要死要活的,真的感觉生不如死,到时真的要喊救命!24 岁的人撸成了 42 岁的样子,全身是病!喊救命!大家应该知道手淫的危害有多大了吧!!
\end{case}

\begin{quote}\it
    现在后生,已知人事,即当为彼说葆精保身之道。若知好歹,自不至以手淫为乐,以致或送性命,或成残废,并永贻弱种等诸祸。未省人事不可说,已省人事,若不说,则十有九犯此病,可怕之至。孟武伯问孝,子曰,父母唯其疾之忧。他疾,均无甚关系,冶游,手淫,贪房事,实最关紧要之事,故孔子以此告之。今之少年,多半犯手淫病,此真杀身之一大利刃也,宜痛戒之。多有少年情欲念起,遂致手淫,此事伤身极大,切不可犯。犯则戕贼自身,污浊自心。将有用之身体,作少亡,或孱弱无所树立之废人。(印光大师)
\end{quote}

\begin{case}
    我一直都有侥幸心理,直到昨天下午连撸三次,第二天出现头晕头痛,没胃口吃饭,吃的饭都吐出来了,老是作呕,现在生不如死,真想一死了之,但是又不甘心,我才二十岁,人没堕落之前也算一表人才,现在我真的怕了!不敢再有侥幸心理了,心魔太可怕了!我一定好好学习,不敢再撸了。我可能得上神经衰弱了!撸了六年了!又熬夜又连续几次三番地撸,我要死了!好痛苦!容貌变丑算什么?比起神经症,变丑简直就不叫事,神经症,头晕头痛就像紧箍咒一样,天天都是折磨,症状发作的时候我想死了,真的好痛苦,谁能告诉我,我会不会死,我真的不想死啊!谁能救救我啊!!

    \textbf{附评} 不少手淫的人都会有侥幸心理,他们也知道一些手淫的危害,但是他们觉得这种事情不会降临到自己头上,当报应真的现前了,到时候悔恨就晚了。侥幸心理完全无视事物本身的性质,彻底违背了事物发展的本质规律,很多人以为手淫没事,其实只要开始手淫,坏变化已经悄悄发生了,千里之堤溃于蚁穴,量变产生质变,沉迷手淫的结果必然就是症状爆发,这是必然的,是成千上万的受害者案例所揭示的一个真理,精少则病就是铁律!一次次撸,就是在一次次积累负能量,最后整个人就会被负能量撑爆!就像给车胎打气,一直打一直打,最后就会突然爆裂,你自以为不会爆掉,其实在不知不觉间就会越过那个爆掉的临界点。这位戒友已经得上神衰了,头上像戴了个紧箍咒,那种感觉是相当折磨人的,非常痛苦。侥幸心理真的要不得,只要撸,症状迟早会爆发出来的,这个案例也是以喊救命结束的,相当触目惊心!手淫终究会引爆人生的大苦,你以为自己爽到了,其实你亏大了,每一次撸都是在亏损,积小亏为大亏,最后积大亏为巨亏!到时候的下场就相当惨烈了!留得一分精液,便有一分生机,滥撸滥泄之人,是在自取灭亡!!!
\end{case}

下面步入正文。

很多新人在戒色后身体反而差了,然后他就会产生疑惑,以为禁欲不好,其实戒色后一个月内会有戒断反应,就像戒烟会有戒断反应,其实戒色也是如此,一般戒色的戒断反应在三周内即可自行缓解乃至消失。戒色后还有症状反复期,症状反复期在戒色后的任何时间段都是可能出现的,为了区别于戒断反应,如果是戒色后几个月出现的症状表现,那就考虑为症状反复期。如果你是戒色后一个多月出现的症状表现,那也可以考虑为症状反复期。

几乎每个人都会经历症状反复期,因为身体痊愈的过程肯定不是一帆风顺的,导致症状反复的因素也有很多,比如沉迷意淫、熬夜久坐、劳累、生气、受凉、遗精、饮食不注意等。戒色是系统工程,恢复也是系统工程,我们应该要加强养生之道,各方面都要做好,这样症状反复的次数就会大大减少,最后就会趋于稳定。光戒是不行的,一定要在养生恢复方面多下功夫。如果你的症状比较严重,那我建议你可以配合中药调理。请记住:戒色是不会错的,孔圣人也提倡过戒色,很多高僧大德都活过八十岁乃至一百岁。

很多戒色新人的信心往往不够坚定,以前也一直被无害论洗脑,所以一旦身体出现症状反复,他们马上就可能会发生戒色立场的动摇,甚至会怀疑戒色。其实痊愈的过程,基本都会经历症状反复的,很多人伤精史很长,伤精程度也比较严重,这就注定恢复的过程是曲折的,但只要好好坚持下去,最后的胜利必然是属于我们的,这点信心一定要有。戒色和恢复都需要三心:信心、决心和恒心,三心稳固,一切都会水到渠成。

痊愈的规律表明,症状的消失会经历一个不稳定到稳定的过程。就像练习射击,刚开始很少能打到十环,然后坚持练习,慢慢就能稳定在较高的水平,打到十环的次数就会越来越多,身体恢复的过程也会经历反复,很多戒友可能不会出现明显的戒断反应,但几乎每个戒友都会经历症状反复期,症状反复期可以说是必经的,关键就是自己要学会保养,这样症状反复的次数就会越来越少,然后就会趋于相对稳定的痊愈状态。

很多戒友一出现症状反复,他就慌了,其实出现症状反复很正常,毕竟导致反复的因素有很多,即使正常人也有生病的时候,何况不少戒友都有好几年乃至十几年的伤精史。戒除手淫恶习之后,身体是会慢慢恢复的,但恢复的过程也是曲折的,自己一定要在养生方面多下功夫。戒色之后如果恢复不理想,这样也是容易导致破戒的,之前就有戒色两年左右破戒的,原因就是恢复不理想,自己感到灰心丧气而破罐破摔,虽然有在坚持戒色,但是在养生方面做得实在太差了,不仅还在熬夜久坐,而且饮食方面也不规律。很多人忙起来饭也不按时吃,有的早饭都不吃,一睡就睡到十一点,中医有讲到久卧伤气,起得太晚也会对身体的恢复产生很不利的影响。传统中医有讲到不妄作劳,即不要过度劳作。\textit{久视伤血、久卧伤气、久坐伤肉、久立伤骨、久行伤筋,是谓五劳所伤。(《素问》)} 视、卧、坐、立、行都是生活中必不可少的行为,但过度则会导致伤血、伤气、伤肉、伤骨、伤筋等。所以我们戒色后在养生方面一定要格外注意,要多学习养生文章提升自己的养生意识,这样就可以大大减少症状反复的次数。

在戒色之后,要对症状反复期有一个心理准备,这样当症状开始反复时,心里也就不会慌了,可以做到从容应对,到时候自然可以顺利过关。把握了痊愈规律,心里就有底了,即使症状反复了,也不会感到困扰,因为你知道这是暂时的,只要注意休养,适量锻炼,反复的症状很快就会缓解的。想当初我的痊愈过程也是经历过多次反复的,但每次症状反复我都不会慌张,因为一切都在预料之中,任凭风波起,稳坐钓鱼船,泰山崩前色不变,猛虎在后步不乱,戒色之后的心态一定要稳!不管发生什么情况,一定要从容镇定地应对,不可有丝毫的慌乱,你心里一慌,心魔就有可乘之机了,心魔在你心态不稳的时候,会给你来一通怂恿,让你本已动摇的立场瞬间塌陷。其实如果你真正把握了痊愈规律,也就淡定了,这种淡定就像你打牌时已经知道了对方的底牌,一切都在你的预料和掌控之中。

我那时经历的症状反复主要有慢前的症状反复、脱发的症状反复、荨麻疹的症状反复、神经症的症状反复等,一般在天冷时或者遗精后,慢前是容易反复的,特别是在连续遗精后,这时出现症状反复的可能性就会很大,还有熬夜久坐也很伤,也易于出现慢前的症状反复,可以说慢前是最容易反复的症状。脱发的恢复也容易经历一个时好时坏的过程,如果保养得当,那么慢慢就会趋于稳定,脱发要恢复,一定要有耐心,然后各方面都要做好,特别是在养生方面要下大功夫。我的胆碱荨麻疹在戒色的第一年也经历过多次反复,后来反复次数就越来越少了,到现在已经很久没有反复了,可以说基本痊愈了。另外在我戒色的过程中,神经症也经历过多次反复,但因为我平时比较注意养生,也不曾破戒过,包括意淫也是严格做到念起即断,所以神经症的反复势头就不会很猛烈,只要注意休养一两天,身体就会大大缓解,我戒到现在,神经症已经基本痊愈,也是很久没有反复了,处于相对比较稳定的痊愈状态。在戒色吧也有不少戒友反映痘痘经常反复,痘痘是比较容易反复的,光戒是不行的,一定要加强养生,熬夜、饮食不注意、遗精等因素皆可导致痘痘的反复,自己各方面一定要多注意才是。

恢复的过程中出现症状反复是非常多见的,几乎每位戒友都会经历反复,这很正常,不必慌张,可以从以下五个方面来调整:

\subsubsection{通过心理暗示稳定军心}

当出现症状反复了,你要告诉自己,这是暂时的,一定要明白痊愈的规律,痊愈的规律就是:反复次数较多 $\to$ 反复次数减少 $\to$ 逐渐进入稳定的痊愈状态。不管风吹浪打,胜似闲庭信步,当你掌握了痊愈规律,你就不慌了,否则症状一反复,很多人的心态马上就会发生微妙的坏变化,军心立马就不稳了。你应该告诉自己,出现症状反复是很正常的,症状的反复完全符合痊愈的规律,既然如此,自己也就没必要慌张了,到时候对于症状的反复,你就可以泰然处之,从而可以做到处变不惊,安之若素。

\subsubsection{保证充足而良好的睡眠}

症状出现反复了,自己一定要注意休养,一定要保证良好的睡眠,千万不要熬夜久坐,否则就可能加重症状的表现,一般伤精的症状出现反复后,自己注意休养,很快就能有所缓解。有的人虽然戒色了,但是熬夜的习惯还是在持续着,这样他的恢复就不会很快,也容易出现症状反复。熬夜的危害非常之大,比如会导致肌肤状态越来越差,痘痘问题也会变得更加严重,总之,熬夜非常伤肤质。保养肌肤的最佳时间就是晚上十点到凌晨两点,如果这个时间段没有睡好觉,那么身体的内分泌和神经系统都会受到不良的影响,长期熬夜容易导致过敏、抵抗力下降、神经症等,而一个良好的睡眠则可以促进身体的修复。学生党很容易出现熬夜的情况,这样对于身体的恢复就很不利,所以,必须建立一个良好的作息习惯,这样身体才能更好地恢复。症状出现反复后,更须保证充足的睡眠,必须保证生活作息的规律。在睡眠的过程中,人体就是在充电和修复了,能够保证良好的睡眠,一般的症状反复很快就会自动缓解的。

生病后人总是特别嗜睡,因为机体通过睡眠来抑制其他生理功能,从而突出免疫功能,以便帮助人体早日恢复健康,由此可见充足的睡眠能增强人体免疫力。医学实验也发现,人如果减少四小时的睡眠,体内免疫细胞活力就减弱 28\%;而获得充足睡眠后便可恢复正常。睡眠状态下,规律分泌的各种激素积极发挥着作用,以生长激素为例,当你进入深睡状态一小时后,其分泌进入高峰,是白天的三倍多,该激素除了促进生长,还能加速体内脂肪燃烧。相反,若睡眠不足,内分泌就会紊乱,激素分泌就会丧失规律,不仅情绪容易变得激动,还可能会影响到生育能力。另外,人体大多数内脏器官如心脏、肠胃等都受植物神经支配,植物神经分为交感神经和副交感神经。白天,交感神经活跃,心跳及肠胃蠕动都加快;当你睡着时,交感神经变得抑制,副交感神经呈现活跃状态,内脏器官得到休息和放松。如果疲倦时不睡觉,不仅内脏器官得不到休息,植物神经也容易变得紊乱,从而为神经症的爆发埋下了隐患。很多戒友都是熬夜加手淫,最终把自己搞废了,神经症一旦爆发,那痛苦就呈几何级数增长了,到时候真可谓惶惶不可终日,活得异常痛苦和折磨。

\subsubsection{建立良好的运动习惯}

\textit{流水不腐,户枢不蠹(《吕氏春秋》)},\textit{唐代名医孙思邈} 就体会到运动能够使“\textit{百病除行,补益延年,眼明轻健,不复疲乏}”。适量的运动对保持人体健康、祛病延年可以起到积极的作用,但过量的运动则很可能会适得其反,甚至会导致免疫力的下降。当症状出现反复时,我们可以做一些适量的有氧运动,比如慢跑快走等,这样可以促进身体的恢复,但也不能搞得太累,也要避免出大汗,因为大汗伤阳,大汗过后,身体往往容易虚掉。

冬天动一动,少闹一场病;冬天懒一懒,多喝药一碗。其实不仅冬季如此,其他季节也是如此,积极锻炼可以让你有一个更棒的能量状态,可以让你以更好的精力去面对每天的学习和生活。反之,锻炼不足与饮食过量则会引起体重的增加,当肥胖发展到一定程度时,患上心脏病、高血压和糖尿病的可能性就会大大增加。有氧运动加上适当的饮食控制,可以最有效地去除体内多余的脂肪,可以让一个人恢复满满的精力与活力,可以让自己活出年轻态。当一个人缺少运动时,常常会感到疲劳、情绪抑郁、记忆力减退,甚至丧失工作兴趣,而有氧运动可以奇迹般地改变这种状态,使人情绪饱满,精神放松。当我们出现症状反复时,可以通过适量的有氧运动来得到改善和缓解,我以前经常去公园散步,走一圈回来就感觉神清气爽了,我们应该多感受大自然的气场与频率,让身心灵恢复到单纯而和谐的状态。

\subsubsection{养生功法以及食疗}

在症状反复出现时,我们也可以做做养生功法,比如八段锦、六字诀和站桩等,这样也可以起到缓解症状的作用,养生功法不仅仅是一种保养,其实也是一种治疗手段,有时并不比药物的效果差。我那时得神经症时,吃过非常多的药物,刚开始吃是有效果,但时间长了就耐药了,再吃效果就不好了。后来我就自己钻研养生功法,看了很多的养生书籍和讲座,然后自己也开始尝试艾灸、站桩、打坐、六字诀等,这样经过一年多的戒色养生,我的身体症状基本就消失了。所以,会养生的确非常关键,很多戒友为何那么容易症状反复,为何戒了很久身体都没有明显的改善,关键就是他不懂得养生之道,在戒色的同时还在做着其他伤身体的事情,而他因为缺少养生意识而浑然不知。

古人云:“饮食进则谷气充,谷气充则血气盛,血气盛则筋力强。脾胃者,五脏之宗,四脏之气皆禀于脾,四时以胃气为本。”症状出现反复后,也应该加强食疗方面的调养,营养要跟上,我以前推荐过粥疗,养生粥具有较好的健脾胃、益五脏作用,尤其适宜于少儿、老年及久病体弱者食用。高濂在“饮撰服食笺”中录有 38 种粥食方,诸如芡实粥、莲子粥,食之可益精气、强智力、聪耳目;山药粥可治虚劳骨蒸等。饮食的调养也要注意保持清淡,中国自古就有“大味必淡”之说。饮食清淡是食养的重要原则之一,高濂认为“\textit{人食多以五味杂之,未有知正味者,若淡食,则本自甘美,初不假外味也。}”名医朱丹溪也极其注重饮食清淡,他活到了 87 岁,老年时依然形体矫健、精力充沛、思维敏捷。他指出:“\textit{人之饮食不出五味,味有出于天赋者,有成于人为者。天之所赋者,谷蔬菜果,自然冲和之味,有食之补阴之功,此《内经》之所谓味也。}”

\textit{夫精者,生之本也。(《素问》)} 肾所藏的精包括“先天之精”和“后天之精”。先天之精禀受于父母的生殖之精,它是与生俱来的,是构成胚胎发育的原始物质。后天之精是指人出生以后,摄入的食物通过脾胃运化功能生成的水谷精气,藏之于肾。肾为封藏之本,\textit{肾者,主蛰,封藏之本,精之处也。(《素问》)} 人的一生必定会经历生、老、病、死这些阶段,而每一阶段机体的生长发育或衰退情况,都取决于肾精及肾气的盛衰,现在很多无害论的观点真可谓荒谬透顶,完全和中医科学相违背,真是害人不浅的歪理邪说。\textit{善养生者必宝其精,精盈乃气盛,气盛则神全,神全则身健,身健则病少。(《景岳全书》)} 我们戒色之后一定要学会保精之道,精之于人,犹如油之于灯,而伤精的方式有很多,诸如熬夜久坐、暴怒、吃冷饮、用力过度、频遗等因素,都会导致肾气的损伤,所以戒色后一定要多学习养生的知识,这样出现症状反复的次数就会随着养生意识的提高而大大减少。

\subsubsection{积极治疗,药物调理}

如果症状反复比较严重,那也可以去看医生,通过药物调理也是可以使症状缓解乃至消失的。戒色后只要不出现破戒和频遗,那么症状反复一般不会很严重,休养一到两天即可自行缓解,但有的戒友伤精史比较长,而且养生方面做得很不好,所以他出现症状反复就可能比较严重,这时候是应该寻求积极治疗的。三分治疗,七分养生,自己一定要在养生方面多下功夫多研究。养生意识上去了,出现症状反复的次数就会大大减少,因为你知道哪些因素容易导致症状反复,这样就可以学会避免那些因素,从而把症状反复的可能性控制在比较低的程度。

吃药调理一定要注意对症,对症才能取得比较好的效果,另外,也要注意不能依赖上药物,特别是滥用药物。很多戒友在出现症状后,会选择在网上买药吃,这样很可能不对症,所以最好去看中医调理,具体望闻问切后再拿药,这样才比较稳妥。不少人通过网络买的药,完全就是壮阳药,这样吃下去很可能会适得其反。药物壮阳,古已有之,于今尤烈,在这个邪淫泛滥的时代,各种壮阳药的广告真可谓满天飞,这类药物刚吃时生龙活虎,像打了兴奋剂,泄精之后就完全萎靡了,很多人都彻底蔫了,打不起精神。所以,壮阳药要慎吃,如果靠吃壮阳药来纵欲,那真的就是在自取灭亡了,之前就有滥服壮阳药导致肾衰竭的报道,甚至还有吃壮阳药猝死的案例,可不畏哉!

\paragraph*{总结}

出现症状反复期,不必慌张,应该要学会从容应对,恢复的过程肯定是曲折的,并不是一帆风顺的,我们一定要学会养生之道,这样就可以减少反复的次数,最后就会趋于稳定。现代人们的养生观念大多是建立在“补”的基础之上的,很多人根本没有“守”的观念,身体出现亏损了才想到补,这时候已经有点被动了,如果能够避开那些导致肾气亏损的因素,真正能够守得住,那么就可以完全掌握主动权。对于能够导致症状反复的因素一定要学会避开,\textit{夫上古圣人之教下也,皆谓之虚邪贼风,避之有时(《黄帝内经》)},能够避开导致症状反复的因素,就能最大程度地减少反复的次数。《内经》还特别强调了养神的重要性,“\textit{恬惔虚无,真气从之,精神内守,病安从来}”指出思想清静纯正,心无杂念,这样可保正气调和,百病不生。《灵枢》还提出了情绪养生的要求:“\textit{智者之养生也,必顺四时而适寒暑,和喜怒而安居处,节阴阳而调刚柔。如是则僻邪不至,长生久视。}”我们在出现症状反复后,也要格外注意情绪管理,保持心平气和就能加快身体的恢复。

\textit{富润屋,德润身。(《礼记》)} \textit{性既自善,内外百病悉不自生,祸乱灾害亦无由作。(孙思邈《千金要方》)} \textit{积善有功,常存阴德,可以延年。(明代《寿世保元》)} \textit{唯乐可以养生,欲乐者莫如为善。(张景岳《先后天论》)} 简明地道出了为善乃快乐之道。\textit{善养生者,当以德行为主,而以调养为佐(清代养生家石天基)},提出了常存安静心、常存正觉心、常存欢喜心、常存善良心、常存和悦心、常存安乐心等。养生贵在养德,我们戒色后一定要注意行善积德,多帮助别人,多做好事,这样症状反复的次数就会大大减少,养生的最高境界就是养心,心地一定要善良、豁达、宽容。正所谓:天地逍遥一戒客,清风随形德逸香!
