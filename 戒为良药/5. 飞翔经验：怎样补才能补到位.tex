\subsection{怎样补才能补到位}

今天专门来谈如何“补”的问题,这个问题是戒友们关注度非常高的一个问题,和戒一样,补也存在非常多的思想误区。这篇文章就来详细谈一下,希望能帮到大家。

\subsubsection{补的误区之一:只知补,不知修心}

补是门大学问,里面的水很深,不是想象的那么简单。很多戒友因为长年 SY 恶习,身体严重亏损,自然第一反应就是补,总想吃点什么补一下,自己不敢乱补的就会去看中医,吃中药来补肾气。其实补是应该的,但也要学会如何高效率地补,而不是低效的补,里面有很多讲究。不管是自己吃补药或者食补,还是找中医开药方,补的第一前提就是:修心功夫要到位。因为你吃补药后,肾气会有所恢复,肾气一恢复,欲望就会起来,这时候非常容易破戒,一破戒等于在泄漏肾气,这样补的效果就会大打折扣了,得不偿失。所以,如果你意识不到修心的重要性,而去一味地补肾气,结果就是时好时坏,身体还是没有多大的恢复,甚至会出现这样的情况,因为吃太多补药,反而出现了一些副作用,身体更加不行。而且补药也是会耐药的,吃多了你就会发现,效果没刚吃时好了。这其实就是补的误区之一:只知补,不知修心。所以大家一定要注重修心,多学习戒色文章,把修心功夫提起来,只有通过学习才能增加戒色的定力值,有了定力,YY 自然会少很多,甚至可以做到没有 YY,这样的心理状态再去补,效果就会加倍,否则上补下漏,肾气始终满不了。

\subsubsection{补的误区之二:忽视吃饭的重要性}

其实最补益精气的不是人参,不是鹿茸,不是山药,不是黑豆,不是任何一种补药,而是大米!假如补精是在造房子,大米的作用相当于地基,如果你不好好吃饭,那么你吃再多其他的补肾食物,也不会收到多大的效果,因为你补的地基不稳。

大米有很多做法,我比较认可的是吃粥养生。

\textit{每日空腹,食淡粥一瓯,能推陈致新,生津快胃,所益非细。(养生名著《老老恒言》)} 宋代大诗人陆游专作《食粥》,其诗写道:“\textit{世人个个学长年,不悟长年在目前,我得宛丘平易法,只将食粥致神仙。}”陆游寿逾八秩,可见其食粥的补养之效。传统中医认为,食粥能滋生精液,培养胃气,助消化、且营养俱存。

唐朝医学家孙思邈,因少年多病而学医,并以佛家与道家的智慧来养生,活到一百多岁,他亦主张清晨食白粥。另外,又将中药煮粥,利用“米气”与水分作“药引”,根据五脏六腑的“生物时钟”,去调理身体,治疗疾病。

所以,大家不要舍本求末,把最重要的,也是最补益精气的大米给轻视或者忽略掉,好好吃饭比什么都重要。

\subsubsection{补的误区之三:不看中医瞎补,乱补}

这种误区在戒友中相当普遍,有人不想去看医生,嫌麻烦,有人则怕难为情,原因很多,最后就是自己网上查补药,看自己的症状更适合哪种补药,然后去药房买,其实这种做法是有失偏颇的,肾虚可不像感冒,到药店买个感冒药就 OK 了,肾虚是需要对症治疗的,中医讲究:同病异治,异病同治。即使相同的症状,因为个体差异,药方也不尽相同,具体要望闻问切之后才有明确的判断。所以,最好是去正规的大医院去看,找有经验的老中医看比较保险。另外,也不要把全部希望压在医生和药上面,三分治,七分养,如果你不学习养生知识,没学会养生之道,那么药的作用非常有限,比如你在吃补药,同时又在熬夜久坐,这样补的效果就很有限了。另外,很多人身体很虚,是不适合大补的,因为身体“虚不受补”,脾肾阳虚,吃下去也吸收不了,反而成了胃肠的负担,所以最好找有经验的医生把脉看一下,不要擅自瞎补。

\subsubsection{补的误区之四:不运动}

大家一说到补,第一反应就是吃。其实吃并不是最高明的补,最高明的补是运动!药补不如食补,食补不如动补。动补这个词太好了,运动就能帮助你身体恢复,比药的疗效还好。

中医讲三阳开泰,善则升阳,喜则升阳,动则升阳。又讲阳强则寿,“阳气者,若天与日,失其所,则折寿而不彰”,所以运动这种补药效果非常好,只要运动不过量,对于身体的恢复是很有帮助的。大家戒色的误区就是不运动,抱怨戒色后怎么身体恢复情况不佳,其实你扪心自问一下,你运动了吗,你学会养生之道了吗?光戒是远远不够的,必须动起来,必须学会养生之道,这样恢复才快。三阳开泰里面还有个“善则升阳”,这里面有很深刻的养生哲学,多做善事,包括佛教的放生,其实对你身体的恢复是有很大帮助的,做善事发慈悲心,你身体的阳气自然就补足了。

我推荐的最佳补法:就是打坐和站桩。

我曾经试过很多补药,都不如打坐和站桩好!打坐和站桩既可以补元气,又有利于修心,一举两得,不用花钱,不必担心副作用。很多人也许会认为修炼气功会出偏,气功的确有出偏的情况,但我说的打坐和站桩不是那种容易出偏的类型,我说的是养生类的打坐和养生桩,是不会出偏的。

\bigskip

最大的补药:不泄为补!可以说这四个字就是补法的王道,否则你大吃大补几十天,连续两天遗精就会让你“一夜回到解放前”,所以,要补到位,必须在如何减少遗精次数上狠下功夫,把遗精次数减少到最少,杜绝各种容易引起遗精的原因,再配合上积极锻炼,这样身体恢复才比较快。特别是对于长期 SY,肾气透支严重,身体症状繁多的戒友来说,如何减少遗精更是重中之重,是必须思考和研究的一个问题。

最后补充说一下前列腺炎的问题,这个问题是 SY 戒友最容易遇见的问题之一,非常普遍,基本人人都遇见过。很多戒友思想上存在误区,认为前列腺炎和感冒一样,看医生吃下药就会彻底好了,其实这是完全错误的,好多帖子里都有这样的戒友,看了很多次医生,花了上万的医疗费,但还是看不好,原因何在?其实如果懂点中医常识就知道了,SY 伤了肾气,肾气损伤后就会出现前列腺炎,医生给你治病,帮你恢复肾气,前列腺炎暂时好了,但是你又 SY,肾气又伤,所以前列腺炎特别容易复发,复发率在 90\% 以上,原因就是没有明白医理,思想认识上有问题。只有彻底戒掉 SY,前列腺炎才有望真正康复。这种康复其实就是思想认识上的飞跃,就是认识到了根本原因所在,注重养生了,节约使用肾气,这样前列腺炎才不容易复发。否则在这个问题上认识不清,你有得好跑医院了,花在检查和药费上的钱就会越来越多。
