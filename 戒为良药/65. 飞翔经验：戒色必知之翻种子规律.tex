\subsection{戒色必知之翻种子规律}

\paragraph*{前言}

关于戒色多久算成功,这季前言再说说,戒色 180 天(也就是半年)破戒的,还是很多的,戒色一年破戒的,我也看到过不少,戒色二年以上破戒的,我也看到过一些。我一般按进入戒色稳定期来判断戒色是否成功,天赋好的戒友在半年左右即可进入戒色稳定期,戒色稳定期的标志就是意淫极少,煎熬感消失。当然刚开始戒色,会进入欲望休眠期,这有点像戒色稳定期,但只是欲望休眠期而已,过了欲望休眠期就是破戒高峰期。戒色半年,时间还是有点短,稳定性还是较差,某戒色平台曾经说过戒色三年才算戒色成功,这个标准我还是很认同的,三年的确是个坎,一般戒色二年可以进入比较稳定的层次,戒色三年则可以进入更稳定的层次,戒色三年的体验和两年是完全不同的,戒色三年的人是久经魔考的人,魔考来自两方面,一方面就是自己的邪念,邪念来犯,是否能够做到念起即断;另一方面,就是来自外界的诱惑,我们生活在邪淫泛滥的时代,外界的诱惑实在猛烈。一个戒色三年的人所经历过的魔考,少则几百次,多则上千次,有时一天就要经历十几次,因为一打开网站,诱惑可谓无处不在,面对诱惑,是否能够做到视而不见,心如止水?戒色三年的人就像一个老兵,身经百战乃至千战,久经考验,每次都能顺利过关,这种人的戒色稳定性是极强的,坚持到三年,这种稳定性基本就固化了,一般不会轻易退转的。戒色一年,或者二年,发生破戒的可能还是很多的。戒色三年的风景和戒色一年二年是迥然不同的,值得注意的是,我这里指的三年,不是戒戒破破的三年,而是真正彻底戒色的三年。

上季轩辕七郎反映了几个问题,这季前言讲一讲。

轩辕正好点中了几个比较常见的问题,其实这几个问题在一年前就曾经出现过,戒色吧的发展就像滚车轮,不断有大量新人涌入,很多问题又开始重复出现,现在戒色吧的人数较之过去有了很大的增加,所以过去的一些问题和矛盾在现在看来一下变得很突出,一个帖子会引来大量的回复。我在《戒为良药》里也说过,戒色吧管理层不会把戒色吧变成佛教吧,戒色吧主要走的还是专业戒色的路线,研究的是戒色的原理与规律,分享的是前辈的戒色经验和方法。专业戒色的覆盖面是最广的,新人也比较容易接受。如果让新人都接受佛教戒色,也不大现实。很多新人对于宗教也存在误解和偏见,一看到这类内容容易产生反感。所以,我们对于新人的引导,还是应该以专业戒色的内容为主。佛教戒邪淫的内容,戒色吧也不排斥,毕竟儒释道都属于传统文化,在戒邪淫方面,佛力的加持的确不可思议,佛教文章的存在可以接引有佛缘的戒友,戒色吧关于佛教的帖子也不是很多,5\% 左右,可能现在帖子的回复有很多信佛的师兄在大量粘贴复制,所以感觉佛教的内容多了起来。其实在土豆当吧主时,佛教的帖子比现在还多一些呢。中国信仰自由,戒色吧也是如此,佛教戒色也是一条很好的戒色路线,大家应该多包容些。关于信仰,我也主张随缘,不要勉强新人接受,以免引起新人反感而退失戒色的初心。

戒色吧的戒色理论的确很多,大部分理论都是好的,百家争鸣的局面有利于戒色吧繁荣发展。不过,有些戒色理论的确存在思想误区,也的确可能对新人产生误导,有些戒友自己并未戒色成功,自己都在摸着石头过河,理论还不是很成熟很完善,这样的戒色理论是可能让新人多走弯路。还有一些戒友的思想里还残留着适度无害论,自己还想着搞适度,这样他写的戒色文章就会对新人产生严重的误导。从中医角度来认识撸管危害是极好的,但是中医和戒色也是两个不同的领域,学中医的人不一定能戒色成功,我们还是要掌握戒色的原理和规律,掌握专业的戒色方法。我是很主张学习中医养生知识的,我把中医的内容作为专业戒色的一部分,在认识撸管危害方面有其独到之处,这是国外戒色无法比拟的优势。目前戒色吧中医的内容是不多,偶尔有一些转帖,还有一些中医爱好者在帮助解答,希望将来能够有更多精通中医的人才入驻戒色吧。

吧务删帖的标准有一些个人的差别,这个问题吧务群也专门沟通过,比如如何让被删者知道被删的原因等,很多新人不懂规矩,或者思想存在误区,又或者帖子出现偏激谩骂等,在新人的角度,他会觉得吧务无故删帖,在吧务的角度,则有其正当的理由。吧务删帖的尺度略有差别,当然偶尔也会有手滑误删的情况。大家对吧务有意见可以提,吧务群也经常开会来完善管理方案,戒色吧的摊子越来越大,人也越来越多,管理的难度也在不断增加,人多嘴杂,新人思想误区多,如何引导新人,如何让他们快速进入戒色正轨也的确是个大问题。

之前也有不少戒友去重灾区宣传,结果就是两种情况:一,自己被拉下水而破戒;二,被骂被诽谤而心理失衡,宣传不成反而坏了自己的心态。更有甚者,引起了自己的争斗之心,开始爆粗口,挑起骂战,宣传最后变成了互相谩骂。我不希望看见此类情况出现,而现在很多新人正是如此,我们戒色吧的人应该是一群有素质之人,不要强迫别人戒色,应该随缘度人,出去宣传也要注意宣传方式和宣传策略,一定要做好被删被骂的心理准备。另外,重灾区少去,很多人喜欢去 H 吧宣传,这种宣传的想法很可能是心魔怂恿他去的,名义上是去宣传,其实是去破戒。这点应该高度警惕,之前很多戒友就是喜欢去重灾区宣传,结果宣传不成,自己反而被拉下水。所以,没有很高的觉悟和定力,尽量避免去重灾区,宣传的途径有很多种,不一定要去重灾区,重灾区的人往往更加执迷不悟,也更难度化。到处爆粗口,我是很反对的,人家一看,戒色吧的人怎么是这样的,你出去宣传代表戒色吧的形象,如果你这样谩骂,人家自然对戒色吧的印象很不好,觉得戒色吧的人很极端很偏执,而且素质还那么低。

关于 SY 万病论,《戒为良药》和《以戒为师》里面都讲到过的,SY 的确可以导致很多疾病,肾虚百病丛生绝非虚言!但也不是所有疾病都是 SY 造成的,有些戒友在 SY 前就有病,只是在 SY 后身体变得更差了。致病因素也是非常多的,熬夜久坐也伤身体,其他很多恶习也会损害身体健康,如果你是 SY 后明显感觉身体不行了,那很可能和 SY 有关。戒色养生是很重要的,但如果你症状很严重或者持续,则应该积极就医治疗,该治疗的还是需要积极治疗,三分治疗,七分养生。

不断有新人大量涌入,不断有重复的问题出现,新人是该多看看前辈写的文章,因为你的大部分疑问在前辈文章里都有讲到过,新人应该先静下心来好好学习一番,你想问的问题,你心中的疑问,基本都有戒友问过了,戒色吧在不断重复着过去,历史有着惊人的相似。去年的问题在今年又反复出现了,今年又是一批新的戒友,每个月都有大量新人涌入,重复着过去的故事。

下面分享几个答疑案例。

\begin{case}
    飞翔哥在之前的文章里讲“我一天也不会有一个 YY 念头”,在 \ref{63} 里讲“我也只是降伏,而没有彻断”。这不矛盾吗?曾经说实话,我真以为飞翔哥没有性的念头了,我自己理解性的念头就是意淫。看了 \ref{63},才觉得不可能不起性念,关键在于念起即断而已。所以戒色的人要警惕,手淫可以戒掉,意淫我觉得不行,戒掉就是能不起念,一个凡人真能不起性念吗?

    \textbf{答} 这其实不矛盾,进入戒色稳定期后,YY 极少,有时一天都没一个邪念,并不是说再也不会起邪念了。进入戒色稳定期后,有时十几天都没一个邪念。不怕念起,就怕觉迟。不要跟着邪念跑,一定要保持高度警惕!完全做到不起一个邪念,那是圣人的境界,我们凡夫能做到降伏就算成功了。加油!

    \textbf{附评} 我们凡夫都有淫欲种子,要彻断淫欲种子,极难,能够降伏已属不易。意淫是可以戒掉的,但戒掉的标准是念起即断,而不是再也不起意淫。警惕是永远的主题,除非你能够做到再也不起一个邪念,那也不是凡夫的境界了。有时候很多邪念,并不是你主观想起,而是它自动跳出来的,这就是翻种子!我们凡夫真能把淫欲的种子彻底消除吗?难度是极大的。你这辈子看了多少 AV,看了多少黄图,又进行了多少邪淫的行为?这些都印在你八识田里了,有时候它自动就会跳出来骚扰你。如果你能做到念起即断,它也不会对你造成伤害,如果你沉迷意淫乃至撸管,那对身体是会造成很大伤害的,伤害到临界点就会症状缠身。很多小孩还没有发育,但是已经有那种念头了,淫欲心是与生俱来的,如果我们能够降伏其心,已经算戒色成功了。

    从非撸者变成撸者,那太容易了,我们都是从非撸者过来的;然而从撸者变回非撸者,难度就大了很多。由俭入奢易,由奢入俭难。有过专门的统计,90\% 的男生都有过手淫的经历,其实我们身边到处是撸者,你以为人家是非撸者,其实这都属于个人隐私,很多人也不愿意透露自己在手淫,毕竟这并不是什么光彩的事。有的人手淫年限少,又热爱运动,外表暂且看不出来,但千万别以为人家是非撸者。另外,非吸烟者比较好鉴别,我们周围的确有很多人不抽烟,特别是学生党,不抽烟的占大多数。但是,撸管就很不同了,撸管是一种很私密的行为,隐秘性极高,你自己不说,别人很难知道。大家应该都有这个体会,在父母眼中你还是过去那个纯洁的孩子,但是只有你自己知道,你已经沉迷撸管很久了,父母看不出你是撸者,就像你看不出同学是撸者一样。你以为你周围都是非撸者,弄不好你周围正潜伏着一群撸者,弄不好有的人比你撸得还凶!我们都是带着淫欲心来到这个世界上的,你以为你的同学生来觉悟和定力就比你高吗?现代社会结婚相对比较晚,邪淫文化又非常泛滥和猛烈,加之无害论的毒害,你以为他们可以幸免吗?

    有些人撸龄短,身体底子厚,又很年轻,也就十几二十出头,虽然撸管,但是外表几乎看不出什么;还有的戒友在学校里体育拔尖,校运会上可以拿名次,你觉得他的同学会认为他是撸者吗?有些人表面上貌似正义凛然,其实早已经深陷手淫恶习;还有些人口口声声说不懂撸管,其实他早就是撸油子了!撸管是背地里见不得人的恶习,大家都希望在别人眼中是好印象,除非这个人思想很开放或者出于炫耀的目的,他会说自己一天多少次,而大多数人则会隐秘自己的撸管行为,不会让别人知道。我做健身教练时的一个同事,他外表真的很阳光很有精神,也很帅,皮肤也好,但是他告诉我他也有撸管史,并且说自己身体没以前好了,以前可以射得很远,现在明显不行了,是流淌出来的,没有冲劲了。还有一个同事也很帅很魁梧,外表也几乎看不出什么,但是他不仅是撸者,而且是嫖者,是不是大跌眼镜呀。所以,大量的撸者就潜伏在我们周围,不要以为人家都是非撸者,大家其实都差不多的,90\% 一点不为过。如果真正坦白来讲,我相信几乎每个人都会有撸管经历,不撸管也会磨床,不磨床也会夹腿,总之,他会通过某种方式来获取快感,这是一个极易犯的恶习,谁叫我们都有淫欲种子呢?每年都有无数的人从非撸者变成撸者,保守估计也应该有几千万,很多人一发育就开始撸管了,之前他们之所以是非撸者,并不是他们的觉悟有多高,而是他们尚未掉入撸管陷阱!就像小学生基本都是非撸者,但他们有任何的戒色觉悟吗?当然没有,他们只是未发育,尚未掉入撸管陷阱而已。要从撸者变回非撸者,必须通过学习达到一定的觉悟水平,这样才可以降伏心魔,彻底戒掉撸管!

    我接触的大量案例中,的确有一些戒友第一次撸管比较晚,大多数人在初中就会撸管,甚至还没发育就会在床上摩擦,而有的人是大学才开始第一次撸管,然后就是一发不可收拾。从非撸者变成撸者,不需要任何觉悟,一次好奇心就可以让你堕入手淫魔网;而从撸者变回非撸者,则需要通过不断学习提高觉悟,觉悟达到了,才可以降伏心魔,才可以变回非撸者。有的戒友看完一篇文章,就觉得自己是非撸者了,甚至觉得自己不需要再学习了,这其实是在自欺欺人。我前面的文章也说过,戒色和戒烟有相同点,但也有很大的区别,把快乐戒烟法完全套用过来,很可能会导致失败,事实也的确如此。不管什么戒色文章,专业的、传统的、宗教的、外国的、中国的,最后都万法归一,都要回到如何控制念头上来,也就是修心。不是说看了一篇文章就变成非撸者,就不需要修心了。修心是永恒的主题,除非你能够做到不再起一个邪念。孟子四十不动心,这是相当高的境界,你觉得你看了一篇文章就能做到不动心吗?一篇戒色文章可以让你进入欲望休眠期,但无法彻底消除淫欲种子,过了欲望休眠期就是破戒高峰期,到时候就见分晓了。

    我们戒色还是要不断学习提高觉悟,扎扎实实地提高觉悟,不要贪捷径,毕竟戒烟不同于戒色,戒烟的理论完全套用过来,就像两双不同尺码的鞋子,相同点都是鞋子,不同点就是尺码不同,适合戒烟的理论不一定就适合戒色,可以借鉴学习和参考,但千万不要完全照搬和套用,否则很可能会失败。快乐戒烟法的讲座我也专门看过,的确有可取之处,但快乐戒烟法的成功率也不像宣传的那么高,戒烟的讲座可以消除你吸烟的心瘾,这是有可能做到的,但成功率能够达到一半已经非常不错了。这个世界上没有哪篇戒色文章能够彻底消除你的淫欲种子,只要淫欲种子还在,那就会起邪念,还是要做好断念的思想工作!并不是看了一篇戒色文章你的心瘾就彻底消除了,念头肯定还是会起来的,关键就是要做好断念!

    色来自先天,烟来自后天,一个在你身上,一个是身外之物。戒色是需要不断系统学习戒色文章的,并不是看了一篇戒色文章就能彻底成功的,快乐戒撸法的作者自己都在学习其他的戒色文章,各类好的戒色文章我们都应该多学习,多做笔记,多思考,多领悟,前辈好的经验我们应该深入学习和借鉴,这样就可以少走很多戒色的弯路。我到现在没见过哪位戒友只看一篇戒色文章就成功的,很多戒色一年以上的戒友,都看了非常多的戒色文章,还包括中医养生类的文章,还有传统文化的讲座和文章,还有无数的受害者案例,只有当综合觉悟达到了一定的高度,才有望做到戒色一年以上。当然,这句话我也不敢说绝对,也许你就是万里挑一的戒色奇才,只要看一篇戒色文章就能戒三年,甚至不需要看戒色文章,只要想戒,就能戒三年,这种特例,这种天赋异禀的戒色奇才,我想还是应该存在的,我到现在阅戒友几万,遇见过一个戒友,他就是天生不想撸,但他的意淫非常严重,也算是个奇人。
\end{case}

\begin{case}
    我的撸龄有 15 年了,今年 29 岁,已婚,有一子,两岁!年轻的时候一天撸两三次都没事,现在发现身体十分虚弱,坐在那里就不想动,整天累得很,腰也酸软无力,悲了个催啊,以前身体很好,前一段时间我妈去中医那里看病,我顺便让大夫给把了下脉,大夫说我这年纪不应该是这样的脉,太弱了。

    \textbf{附评} 这个案例很典型,年轻时一天撸两三次都没事,而他现在也就 29 岁,也还算年轻,但是身体已经十分虚弱了。易疲劳,不想动,腰酸无力,这些表现都很典型,让中医一把脉,就知道身体已经很差很弱了,记得之前有位戒友去中医那把脉,二十岁的人六十岁的脉象,你说恐怖吗?还有一位戒友自感身体强壮,然后让中医一把脉,也发现很多问题,强壮并不代表健康,有些人的强壮只是外强中干,强壮者也有不少猝死的案例,所以千万不要以为强壮了就可以乱来。手淫的恶果有一个累积效应,是一个温水煮青蛙的过程,是在一点点废你,不知不觉就把你废了。等你突然惊醒,才发现自己已经症状缠身了,这个案例也给出了一条真理:年轻时千万不可乱来,年轻时透支,到三十岁就给你颜色看!到四十岁,身体将会进入全面的衰败,到时就苦大了,药罐子无疑!
\end{case}

\begin{case}
    飞翔哥,我想知道 JJ 上面血管变粗了,是不是特别严重的症状,特别明显,还有精索是什么,粗看了一遍戒为良药,记得上面没十分相似的案例。

    \textbf{答} 如果 JJ 血管异常变粗,有可能是阴茎静脉曲张,具体还要去检查才能知道。加油!

    \textbf{附评} 静脉曲张是人体静脉系统的常见与多发疾病,可病发于人体的许多部位,按发生部位不同,可分为六种类型:上肢静脉曲张、下肢静脉曲张、阴茎静脉曲张、精索静脉曲张、胃底静脉曲张、盆腔静脉曲张。戒色吧最常见的就是精索静脉曲张,因为 SY 得这个疾病的非常多,不少戒友也进行过手术结扎治疗,西医也证实手淫是导致精索静脉曲张的一大诱因。除了精索静脉曲张,有一些戒友也会得阴茎静脉曲张,阴茎上有血管异常突出,和正常的血管突起不同,可以明显感觉到差别。之前也有戒友反映过这种症状表现,但较之于精索静脉曲张,相对少见些。
\end{case}

下面步入正文,这季就戒色必知之翻种子规律和大家做一个分享,具体如下。

我们都是欲界众生,欲界众生最重的烦恼就是淫欲心。欲界众生皆由淫欲而生,淫欲由爱而生。一切众生由于淫欲而生起烦恼,从而成为流转生死的根本。这是最粗、最重、最难断的烦恼。

\begin{quote}\it
    若诸世界,六道众生,其心不淫,则不随其生死相续,汝修三昧,本出尘劳,淫心不除,尘不可出。(《楞严经》)
\end{quote}

不管什么戒色文章,最核心的内容都离不开断意淫,我们都是带着淫欲种子来到这个世界上的,淫欲心来自先天,一篇好的戒色文章可以让你进入欲望休眠期,欲望只是暂时休眠了,邪念迟早会再次冒出来的,这是百分百肯定的,因为任何一篇戒色文章都无法彻底清除掉八识田里的淫欲种子,只要种子在,邪念就会出来,就像一棵草的根在,迟早就会长出来。过了欲望休眠期就是破戒高峰期,到时候的魔考就能验出你觉悟的高下,觉悟不过关,肯定会破戒!基本每位戒友都会经历欲望休眠期,不少戒友都以为自己成功了,滋生了骄傲自满的情绪,这类戒友在戒色吧经常能看到,之所以会这样,就是他不明白戒色的原理与规律,不知道欲望休眠期过后就会有魔考。

佛家所说八识:

\begin{multicols}{4}
    \begin{enumerate}
        \item 眼识
        \item 耳识
        \item 鼻识
        \item 舌识
        \item 身识
        \item 意识
        \item 末那识
        \item 阿赖耶识
    \end{enumerate}
\end{multicols}

第八阿赖耶识的种子多得很,不单是今世的种子,无始劫以来所有善恶种子均由第八阿赖耶识储藏起来。所有世间法和出世间法的一切种子,都收藏在第八识里,遇到缘就会起现行,就像在田里放下了种子就会生出果实一样,所以叫做八识田。

\begin{quotation}\it
    问:最近淫欲心动得厉害,盼师父指示如何能将此心迅速断除?

    元音老人答:不要紧,这是翻种子,这深藏八识田中的淫欲种子不经过多次的反复是消不尽的。过去的大祖师没一个不翻种子,所以他们都秘密加持“楞严咒心”。
\end{quotation}

我们戒色也是如此,刚开始戒色,下了决心或者得到了一次顿悟觉醒,欲望就会暂时休眠,这时候千万不要以为自己戒色成功了,这只是休眠状态,淫欲的种子并没有彻底清除,过了欲望休眠期就是破戒高峰期,到时候就会翻种子,各种邪念都会跑出来,到时候就极有可能会破戒。戒色说到底,就是念头之间的战争,一定要学会看住自己的念头,严格做好断意淫的思想工作,一定要保持高度警惕,警惕外在的诱惑,警惕内在的邪念,要做到:内不随念转,外不为境迁。这是一内一外两层功夫,不管你看了什么戒色文章,也不管你获得了怎样的顿悟,最终都要回到念头的实战上来。我们凡夫能做到降伏其心,已经实属不易了。真正能把八识田里的淫欲种子彻底消除,那是圣人的境界了。

翻种子的规律:刚开始戒色进入欲望休眠期,淫欲种子基本不动,戒得很轻松,很多人以为戒色很简单嘛,以为自己成功了。慢慢进入破戒高峰期,翻种子就很厉害了,这时候极有可能会出现破戒,如果觉悟过关,那么就不会破戒。一般在戒色后一个月左右,就会经历一次魔考,如果一个人经历了大大小小几十次乃至上百次魔考,都能做到不破戒,那么淫欲种子又会慢慢进入休眠态,这时候就会进入戒色稳定期,戒色稳定期的标志就是意淫极少,煎熬感消失,有点像无限延长的欲望休眠期。不过,即使进入了戒色稳定期,也不能放松警惕,因为淫欲种子还在,心魔时不时还会来考验你,不管戒到何种程度,千万不可放松警惕!切记!

戒色就是一个不断翻种子的过程,每次魔考都能过关的话,种子就会慢慢进入休眠态,但并未彻底消除,还需保持高度警惕,做好念起即断的思想工作。

{\it 有的人戒色半年,以为自己成功了,一骄傲一放松警惕,马上破戒;

有的人戒色一年,因为一次吵架生气,情绪没有管理好,继而破戒;

还有的人戒色二年,身体恢复不理想,又或者想试定力,结果破戒。}

{\it 只要淫欲的种子在,

你就可能破戒!

这是毫无疑问的,

即使戒了十年,

也不可放松警惕!

除非你已经达到了圣人的境界

能够做到彻底消除淫欲种子。

我们一定要学会控制自己的念头!

主宰你的念头!看住你的念头!

严格做到念起即断,念起不随,念起即觉,觉之即无!}

翻种子一般有两种类型:

\begin{multicols}{2}
    \begin{itemize}
        \item 主动翻种子,主动去意淫
        \item 自动翻种子,邪念自动跳出来
    \end{itemize}
\end{multicols}

主动翻种子,不必多说了,很多人看到黄图,就主动在那意淫了,或者看到某个美女,就马上意淫别人,这是一种主观上的意淫。自动翻种子就由不得你了,不少戒友都反映,并不是自己想去意淫,而是脑中突然自动跳出一个念头,又或者突然浮现一幅邪淫的画面,曾经让自己很兴奋而且记忆很深刻的邪淫画面,以一种回忆的方式跳出来。自动翻种子其实是非常常见的,一般进入戒色稳定期,主动翻种子的情况几乎没了,因为已经具备很高的觉悟和定力,主观上不会去意淫了,但是因为淫欲种子并未彻底消除,自动翻种子的情况还是时有出现的,我戒到现在,还是偶尔会自动翻种子,但我能够马上做到念起即断,这样对我也不会造成什么不良影响。自动翻种子如果不及时断掉,它就会变成主动翻种子,邪念会牢牢控制你,就像你的身体被植入了一个木马病毒,然后你就会变成撸管肉机,身不由己。

饮食方面一定要多注意,如果补的东西吃得太多,那就很容易导致翻种子,记得有一次我吃了韭菜馄饨,刚吃完没半小时,自动翻种子就开始了,还有肉类也少吃,吃得清淡对于戒色是很有利的,可以很好地控制自己的欲望,可以减少很多翻种子的情况。遗精后也容易翻种子,也要格外保持警惕,牢牢看住自己的念头。不怕念起,念起了就干掉它,你不干掉它,它就干掉你,念头起来不要跟着跑,你跟着念头跑,就是被邪念控制了。

\paragraph*{总结}

我前几季的文章也说到,我只是降伏,并未彻断,种子彻断是圣人的境界,我现在还是凡夫,我现在每次自动翻种子,都能立刻降伏之!我们普通凡夫能够达到降伏已经很不容易了,需要达到相当的觉悟才能反转心魔的控制。我写这季,可以让大家从翻种子这个角度来理解戒色,这个角度也更清晰。有几位戒友提倡转移注意力来戒色,他们的这种戒色思路是有一定道理的,就是让自己忙起来,让自己充实起来,把自己的生活填满,不给撸管留空隙。但是这种戒色思路也存在缺陷,那就是一个人总有休息的时候,也总有无聊的时刻,到时候就容易翻种子,心魔就会跑出来考验你,总之,不管何种戒色思路,都会迎来魔考的,魔考是躲不掉的。很多学生党周一到周五,戒色无压力,一到周末,一到寒暑假,就疯狂破戒。上次一个戒友号称和心魔大战五天五夜,最终还是败下阵来而破戒。还是那句话,戒色就是一场念头之间的战争,向内看,看住你的念头,保持足够的警惕,严格做到念起即断。这样的戒色思路才是最根本的,戒色就是修心,就是修念头。一切戒色方法,一切戒色思路,一切戒色文章,万流归海,都要回到念头的实战上来,邪念出来看你怎么办,遇见诱惑看你是否挡得住?佛言:修道如一人与万人战!这场战争就发生在你的脑海里!\textit{胜人者有力,自胜者强!(《道德经》)}真正的强者就是能够战胜自己的人,希望我们都能够做强者!共勉!
