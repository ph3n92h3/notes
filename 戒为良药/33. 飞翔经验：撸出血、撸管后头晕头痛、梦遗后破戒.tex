\subsection{撸出血、撸管后头晕头痛、梦遗后破戒}

\paragraph*{前言}

最近天气转凉,反映症状反复的戒友多起来了,以慢前的症状反复比较常见,比如夜尿频多、尿急尿等待,还有些戒友会出现睡眠障碍、遗精增多等现象。这些问题和气温下降是有密切关系的,气温降低,各类疾病出现反复的可能性会增大很多。由于“天人相应”,因此对于自然界的变化,人也要适当调整起居饮食,从而保存体力,达到阴阳气血的平衡,保持健康。所以我建议大家一定要注意御寒保暖,注意养生之道,养成良好的作息饮食习惯,不要让自己太劳累,让自己的身体尽快地适应天气的变化。

天气转凉出现症状反复,如果不是很严重,也不一定要去就医。一般休养几天,营养跟上,做些适量的锻炼,症状是会慢慢缓解的,不用太担心。

大家都知道,冬季天冷易生病,却不知冬天不养好身体,春天更易得病。\textit{冬不藏精、春必病温(《黄帝内经》)} 说的就是这个道理。中医认为,寒邪最易中伤肾阳,所以,要抵御寒邪,首先就要注重养肾。否则到了春天就会因肾亏而影响机体的免疫力,容易导致生病。从中医角度来讲,精、气、神是人最重要的生命动力,尤以精为基础,所以“善保精者多高寿,过损精者必早衰”。到了冬季,我建议各位戒友一定要注意养肾,尽量不要去破戒,否则极有可能会影响到你来年的身体健康。

\begin{quote}\it
    冬三月,此为闭藏。水冰地坼,勿扰乎阳,早卧晚起,必待日光,使志若伏若匿,若有私意,若已有得,去寒就温,无泄皮肤,使气极夺。此冬气之应,养藏之道也;逆之则伤肾,春为痿厥,奉生者少。(这段话出自,白话翻译过来的意思是:冬天是阳气闭藏的季节。草木凋谢,种子埋藏在冰雪之下,植物凋谢,动物冬眠,地面的一切生机都看不到了,水面也结冰了,这就是“藏”。人要早点睡觉,晚点起来,比如等到太阳升起来了才起来,人的意志也要像冬眠那样,好像伏匿起来,注意保温,不要外露皮肤,把阳气泻出去。这就是是顺应冬天,养藏之道。逆此道就会伤肾,肾水伤则肝木失其所主,肝主筋,肝木不长,筋失其养,春天手足则容易痿弱,阳气的生长会受到影响。)(《黄帝内经四气调神大论》)
\end{quote}

五脏中肾主水,对应的季节是冬季,肾为先天之本,生命之根,藏真阴而寓元阳,肾藏精,精宜藏而不宜泄,具有蛰伏闭藏的特性。

所以说,到了冬天我们更加不能破戒,一定要把欲望冰封,让自己处在蛰伏不用的状态,把肾精封存好,“善养生者,必奉于藏”,而冬天更要注重“养藏之道”。如果你在冬天反复破戒,那么到了来年春天,你的精神状态就会很萎靡,对疾病的抵抗力就会出现下降,非常容易生病。

最近有戒友说起了皮肤反复的问题,脸是五脏的镜子,而皮肤出现反复的情况是比较常见的,特别是那种不爱运动,长时间面对电脑的戒友,长时间面对电脑会得“屏幕脸”,即使你不破戒,皮肤也会变差的,因为有电脑辐射,然后你处在久坐久视的状态,所以是不利于内分泌调整的,肤质也难以改善。另外,导致皮肤反复的因素,还有遗精、劳累、暴晒、吃发物、熬夜、季节等因素。饮食对于皮肤是有直接影响的,吃了发物或者油腻的辣的东西,都有可能出现皮肤问题。所以,我们戒色之后要尽量保持饮食清淡,然后多做适量的有氧运动,这样出现皮肤反复的情况会减少。我那时坚持一段时间有氧运动,就能感觉到脸部肤质的明显改善,脸摸上去光滑了,以前久坐不动,面对电脑,脸摸上去是粗糙的,气色晦暗,也容易出油出痘。所以,适量的有氧运动是有助于改善肤质的,请大家一定要注重“动养”,这样对于肤质的改善是比较有利的。

再讲讲固肾功的拉紧感问题,如果你找不对大腿后侧的拉紧感,那么固肾功的效果是很有限的。做到位,找对拉紧感,固肾功才能发挥效力。不少戒友都提问拉紧感怎么找,拉紧感到底在哪里?其实固肾功的拉紧感,最主要是在大腿后侧,其次是膝盖后侧,最后才是小腿,膝盖后侧和小腿比较容易感觉到,大腿后侧相对难找些。大腿后侧的拉紧感,可以通过两种方式去找,一种就是感觉,是否感受到大腿后侧有拉紧的感觉;其次就是靠摸,是否处在紧张状态是可以摸出来的,有时候可能主观感觉不是很强烈,但摸上去却感觉很紧,有拉长拉紧的感觉,那也是可以的。找对拉紧感,强化拉紧感,这样固肾功就算做到位了。戒色必过频遗关,这样身体的恢复才有保障,而导致遗精的其他因素也有很多,我总结过有几十种,以前的文章有专门写到,大家可以看看,避免其他导致遗精的因素也异常重要。

下面进入正文。这季就撸出血、撸管后头晕头痛、梦遗后破戒这三个问题详细论述一下,具体如下。

\subsubsection{撸出血}

常驻戒色吧的资深戒友应该会知道,撸出血的帖子是经常出现的,每隔几天就有这类帖子,有时甚至会连续出现。撸出血是容易造成心理恐慌的,有的戒友说“出大事了”,有的则说“真的怕了”,撸出血应该是比较常见的现象,很多戒友处在盲目无知的状态,被瘾控制,疯狂撸管,这种疯狂放纵的撸管其实已经给身体造成了潜在的伤害,症状迟早会表现出来的,只不过撸出血这种情况,给患者造成的心理冲击比较大,这其实也不是什么坏事,能让他快速觉醒,发出戒色之心。经济学界有句话叫:危机导致收敛,属于强制性的收敛。如果不撸出血,也许他还会继续执迷不悟,在通往自废的道路上一路狂奔。

下面看几个撸出血的案例:

\begin{case}[撸出血]
    昨天破戒撸了三次,第一次射精很正常,第二次射精很少,带点血丝,结果好害怕,今天想看看精液里有没有血丝,可是撸了下,感觉不射精,我也不敢太强撸,这是怎么回事啊?急求!现在好痛苦啊!真心后悔了!
\end{case}

\begin{case}[撸出血]
    SY 射的时候,前面射出来的是乳白色的,后面有三滴却是淡红色的。前天也 SY 了!前阵子有一天小腹有暗痛感,十几岁的时候有过一次附睾炎,吃药治好的。
\end{case}

\begin{case}[撸出血]
    我的妈!撸出血了,淡红色的,求救!怎么办?我不痛但有血!
\end{case}

\begin{case}[撸出血]
    四天前 SY 伴有很多血,前两天第二次射精,精液有一点点红色,我今年十九岁,平时 SY 较频繁(现正在戒除),遇到这样的情况很害怕。
\end{case}

\begin{case}[撸出血]
    真的怕了!再也不敢撸了,精液里又一个血块。要断子绝孙了吗?!真的是撸出血了,真心怕了,那个血块开始好像是凝固的,我以为不是血,用手一捏,就散来了,是血。再也不敢撸了,各位好自为之!
\end{case}

\begin{case}[撸出血]
    我总是强忍着射精,以前是透明的,现在出来的却是血精,这是怎么回事?急!
\end{case}

\begin{case}[撸出血]
    我 SY 很多年了,一直都很频繁,可是最近一次 SY 有血丝,不知道是不是因为我 SY 而得了什么病啊,怎么办啊!这是什么原因引起的精液有血丝啊?我该怎么办啊?
\end{case}

据我研究,一般连续几天撸管,或者一天内连续几次撸管,是容易出现这种情况的,就是撸狠了。连续撸管,出症状的概率是非常大的,即使不是撸出血,出现其他症状的可能性也是极大的。有多少肾精给你撸呢?即使你身体底子再厚,长期沉迷撸管,也是会出症状的。精液中有血称为血精,出现血精的原因一般考虑为炎症或者结石问题,比如前列腺炎、精囊炎、尿道炎、附睾炎、精索、前列腺结石、尿道膀胱结石。当然还有其他原因可导致血精,鉴于大家都比较年轻,一般就考虑炎症和结石的可能性,炎症的可能性比较大些。如果实在不放心,可以去医院做个检查。出现血精,也不要太恐慌,恐伤肾,血精是身体在给你信号了,告诉你不能再撸了,该收手了。一般出现血精,是个人都应该知道该收敛了,不能再搞了。如果你继续顶风撸管,那真是活腻了。

\subsubsection{撸管后头晕头痛}

下面谈下撸管后头晕头痛的问题。

中医:肾上通于脑。

人体头面颈项处穴位最多,共有 76 穴,占全身 361 穴的 21\%。头藏脑髓,髓为肾精所化,为肾所主。脑是神经中枢,是管理全身运动、感觉、语言和内脏活动的最高司令部。面部内应脏腑,为经脉之所会,气化之所通。人体十四正经,有手三阳经、足三阳经和任督二脉八条经脉循行起或止于头面,故中医认为:“头为诸阳之会,脑为精明之府,又为髓海之所在,凡五脏精华之血,六腑诸阳之气,皆上注于头。”

一个非常明显的事实就是:SY 是伤脑的,SY 是伤头的!这季就讲讲撸管后头晕头痛的问题。

下面分享几个案例:

\begin{case}[撸管后头晕头痛]
    飞翔大哥,您好,我应该说已经 SY 二十年了,我记得我懂事的时候就喜欢玩弄小 JJ,我在 2009 年到 2010 年的时候和一个女孩子天天发淫秽的短信,也就是不停地 YY,并在 2010 年 11 月身体开始出现了严重的头晕情况,一直到今天头晕还是很严重。一直在吃中药,同时和你说的一样,一边吃一边泄。还有一直到今年的五月份还在 SY,虽然知道是 SY 或 YY 搞出来的毛病。之后我也下定决心不再 SY。同时也真做到了,但 YY 怎么都搞不定。我发现我只要想起我的小 JJ,它就流液体,想到 SY 这两个字或 YY 这两个字都会流。感觉已经到了走火入魔的情况了。 我现在眼睛非常昏暗,脑袋晕,我的头晕一直没有改善,我的眼睛一直很昏沉。
\end{case}

\begin{case}[撸管后头晕头痛]
    以前 SY 过后经常头痛,头晕晕的,整个人像轻飘飘的,像喝醉酒一样,而且记忆一点不好,忘事,理解能力差,大脑反应慢。因为这样,我前段时间也去过医院,做了脑电图,说我是神经衰弱,还有植物神经紊乱,我也吃过六味地黄丸,感觉吃过也没有效果,我害怕,现在我都不敢 SY 了。
\end{case}

\begin{case}[撸管后头晕头痛]
    我长期 SY,最近一次 SY 后头晕头痛、眼花、嗜睡,尿呈黄色,味重。头痛是太阳穴和后脑勺转移性痛,一会这痛一会另一个地方痛,眼睛胀痛不能看一个地方太久,上网就难受,觉得睁不开想闭眼。
\end{case}

\begin{case}[撸管后头晕头痛]
    从小学开始 SY,现在上大学了,头脑总是昏沉,感觉脑子里有个大铁球(不是夸张)。肾脏半夜老跳动,四肢无力,打篮球都不行了,哎!影响学习,也难受,哪位来帮帮我啊!
\end{case}

\begin{case}[撸管后头晕头痛]
    我今年 22,SY 有几年了,比较频繁,这几天 SY 后马上感觉口干口渴,腰酸得要命有点疼,好几天都没退掉。后脑勺晕得要命,有一个星期了,特别晕的时候我跑去躺一会起来后,人就比较清醒,过了几个小时后又开始晕了,上火还很厉害,晚上睡不着。现在天天后脑勺晕,没心情没力气,都不能工作了。有看医生和吃药,都没用啊,救命啊!!再这样下去我不活了,总感觉后脑被什么压住。吃六味地黄丸和安神补脑的药物,没见效果啊!还是晕呀,医生给我刮痧,身体全部是黑红黑红的,上火好厉害啊,我还是晕得要命啊,刮完痧都没好,我怕告诉医生是 SY 引起的,我真的好痛苦啊,要怎样才可以好啊,好难受呀,救命啊!
\end{case}

我手里掌握的案例,SY 后出现头晕头痛的非常多,SY 很伤头,开始是伤脑力,伤到一定程度就可能会出现头晕头痛的情况了。如果你这种头晕头痛的情况一直持续,去看医生,一般就诊断为神经衰弱了,或者植物神经紊乱,总之神经出现问题了。我 \ref{30} 的文章也专门讲到神经症,SY 的确是会伤到神经的,但要伤到一定程度才会出现神经症,如果你有熬夜和久坐,再加上疯狂 SY,这样出现神经症的概率就比较大了。还有一些戒友先天体质不行,更没有资本给他这样撸,那真是伤不起啊。

出现头晕头痛,一定要彻底戒色,然后积极治疗,这样慢慢是能恢复正常的。我原来头晕头痛都出现过,大概戒色养生半年,自己就慢慢好了。那段时间我经常站桩、做六字诀和艾灸,然后积极锻炼,刚开始因为头晕无法多运动,就以散步快走为主,慢慢地,身体就恢复过来了,的确是病去如抽丝,是一丝一丝恢复过来的。恢复正常的感觉,就像从地狱爬回人间一样,从此倍加珍惜这来之不易的健康感觉。

健康是修回来的,修德,修道,修养生,然后再加上积极治疗。其实积极治疗是小头,修德修道修养生才是大头。三分治,七分养。我们自己要学会做一个合格的病人,否则完全靠医生靠药,想要痊愈,那是不现实的,毕竟神经症不是小感冒。一边泄,一边治,敢问痊愈的路在何方?

\subsubsection{梦遗后破戒}

最后讲下梦遗后破戒这个问题。

先上个案例:

\begin{case}[梦遗后破戒]
    我戒色一年了,最近戒的效果比较好,可是遗精后欲望很强很强,而且心里很苦恼,就会 SY,过会死的心都有了,怎样才能在遗精后不破戒?
\end{case}

遗精后出现破戒的情况是比较多的,在破戒类型里我也总结过的,遗精后容易出现以下几种表现:

\begin{itemize}
    \item 心情变差,肾精丢失后,情绪是会发生变化的,会有急躁易怒的表现,还有心灰意懒的挫败感,当然遗精不算破戒,挫败感是不需要的。只要不是频遗,就不用太担心。
    \item 重新回到“虚则亢”的状态,遗精后,欲望会被唤醒,会变得“蠢蠢欲动”,这时候如果不提高警惕,是极容易破戒的。
    \item 症状反复,遗精也伤身体,频遗伤害更大,一般遗精后会出现症状的暂时反复,休养几天即可缓解乃至消失。
\end{itemize}

对于遗精后的破戒现象,我们一定要认识到,正确对待遗精问题,在遗精后一定要注意情绪管理和提高警惕,以免出现破戒的情况。我每次遗精后都会及时忏悔的,因为每次遗精后我都能感觉到欲望在萌动,这时候发忏悔心,可以起到消除欲望的效果。每次遗精后,我的警惕性都会更高,因为这时候是非常容易出现破戒的。情绪管理的重要性我已经反复强调过多次,这个案例的戒友,就是在情绪管理方面做得不够好,他在破戒后出现了苦恼情绪,这时候一定要及时调整情绪,多给自己积极正面的暗示,让自己保持在心平气和的淡定状态。另外,就是要提起自己的警惕意识,这样遗精后破戒的情况就可以避免了。

这季继续推荐五本书:

\begin{book}[《楞严经》]
    由于《楞严经》的内容助人智解宇宙真相,古人曾有:“自从一读楞严后,不看人间糟粕书!”的诗句。所谓成佛的法华,开慧的楞严,要深入佛陀智慧之海,楞严经是必读的经典之一。楞严经我看了很多遍,每次看都有新的领悟。里面也专门讲到了戒淫,还有四种清净明诲。有佛缘的戒友不容错过,可以看文白对照本。
\end{book}

\begin{book}[《黄帝内经二十四节气养生法》,王彤]
    这是一本引导我们顺应自然,掌握二十四节气养生精髓的保健图书。本书在《黄帝内经》的基础上,结合二十四节气不同的气候特点,进行深入和详尽的分析,提出了每年二十四节气中需要注意的养生要点和养生方法。这本书我非常喜欢,从这本书里我学到了不少养生知识,也算开了养生眼界。
\end{book}

\begin{book}[《王晨霞掌纹诊病》]
    掌纹里藏着健康的秘密。你不知道的十四条神奇线,你不了解的八种异样纹透露你身体秘密的源泉。大家想到掌纹,肯定会想到算命。其实掌纹诊病和命理有相通之处,只不过现代的掌纹诊病学更科学,西方科学界其实早已开始研究掌纹和身体的健康联系。所以现代掌纹诊病是相当科学的,大家可以看下这本书,这是一本奇书。优酷有 32 集王晨霞的视频,大家可以看看,也可以买书看。我视频和书都有研究过,非常不错,所以把掌纹诊病推荐给大家。很多戒友把月牙作为身体健康的一个标志,其实掌纹里的四线也可以作为是否健康的一个标志,因为四线是健康线。大家去学了,就知道了。
\end{book}

\begin{book}[《人体的春夏秋冬》,史赞华]
    史赞华,齐鲁名医,出身中医世家,博览群书,熟谙医家经典,素以中西医结合见长,在保健养生方面造诣颇深,见解独到。肝为春,重养“生”;心为夏,重养“长”;肺为秋,重养“收”;肾为冬,重养藏!\textit{善养生者,必奉于藏。(《黄帝内经》)} 养生不如养藏。老中医倾囊传授中医养藏的秘密,帮助你唤起人体的自愈能力。这本书从四季的角度去诠释养生,非常好,我也经常复习这本书。
\end{book}

\begin{book}[《志愿军老兵回忆录》,袁永生]
    这本书可以大大激励你,可以给人以强大的精神动力。它就像一个饱经沧桑的退伍老兵,和你讲着当年战火硝烟的往事,我平时是一个低调淡定的人,但是看这本书时还是非常激动,这本书里的故事是可以穿透灵魂的,看完这本书,我觉得自己充满了力量,它是我的强大精神补剂,在我心中有很重的分量。
\end{book}
