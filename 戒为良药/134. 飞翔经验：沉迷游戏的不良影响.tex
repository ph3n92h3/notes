\subsection{沉迷游戏的不良影响}\label{134}

\paragraph*{前言}

这季前言集中谈一些问题。

\begin{itemize}
    \item 梦撸问题,就是半梦半醒在撸,这种情况一般坚持学习戒色文章净化思想意识,平时加强修心,坚持一段时间梦撸就会自动消失,如果是比较顽固的梦撸,可以买个手指分离板,睡觉时戴着,不要趴着睡,养成右侧卧的睡姿。
    \item 打坐出偏的问题,前段时间一位戒友因为强压呼吸,导致身体气机紊乱,这是走入误区了,呼吸要深、长、细、匀,不可强压呼吸,要保持自然,可以看下南怀瑾静坐的文章。一般养生桩和静坐是不会出偏的,但如果自己有误解或者操作不当,那也可能出现一些问题。如果身体出现失调了,先停止练习,然后注意休养,如果没有明显的缓解,则可以去看中医调理。
    \item 中考、高考完了出现破戒,考试结束后容易过度放松,这一放松,警惕性也下降了,考试完了还有狂欢的心理,这时候也容易出现情绪破戒。所以考试完了要注意调整心态,不可过于高兴,也不可过于放松,要学会保持稳定的情绪,充实自己的生活,每天看看圣贤教育,适量锻炼,帮家里做做事。
    \item 关于五毒月九毒日出现破戒,不要过于恐慌和担心,应该注意休养,尽量避免连续破戒。在有些日子或者时辰出现破戒,伤害是加倍的,破一次相当于其他时间破两三次。前段时间一位戒友在夏至日下午五点多在厕所犯邪淫一次,然后晚上七点以后越来越感觉浑身无力,腿软站也站不住,蹲也蹲不住,后脚跟已经支撑不住身体了,上半身也无力,刷个牙都得蹲着刷,已经站不起来了,感觉就是特别虚软无力,特别累。夏至日阴气生而阳气始衰,一天之中也有阴阳之分,从午夜开始,阳气逐渐上升,至中午阳气最旺。午后阳气逐渐衰弱,直至午夜阳气最弱而阴气最盛,午夜过后,阳气逐渐恢复。有两个时段,第一个时段,子时一阳生,阳气始生,处于微弱状态,熬夜破戒危害非常大,第二个时段就是午后阳气开始衰弱了,在阳气衰弱时破戒,危害会更大。了解一下阳气走势图很有必要,在阳气微弱和衰弱时破戒,对身体的伤害会更大。
    \item 夏季空调电扇少吹,运动出汗、洗澡出汗后不可立刻吹风扇空调,这时体温比较高,身上毛孔是打开的,立刻吹风扇空调,寒气会进入体内,对健康大为不利,会埋下很大的隐患,也许当时感觉没问题,但到了冬季就开始爆发疾病了。身体虚弱时更要注意这方面,避免自己着凉受寒,要有较强的养生意识。
    \item 上季一位戒友给我的反馈:“您认为觉知是无念状态,是观察者,应该安住于觉知。元音老人认为无念是不停留,两者说法是否矛盾?”我和元音老人的观点并不矛盾,看开示一定要看全面,而且要看是从哪个角度来讲的,是从体的角度,还是从用的角度来讲,这要分清楚。无念有两层含义,一、就是一念不生,了了分明;二、就是念起不住,要起妙用。很多人只知其一,其实这两层含义都要懂的,否则就会产生误解,该起用时起用,起用完了要懂得摄归真心。对于大德的开示,一定要正确全面理解,否则就会产生偏见。刚开始看大德开示,肯定会有不懂的地方,甚至觉得前后有矛盾之处,等到智慧渐开,就会明白所谓矛盾之处其实是从两个不同的角度来讲的,看似矛盾实则统一。我们要扎根于纯粹的觉知,然后该起用时就起用,不能死在那里不动,学习、生活和事业都需要用到念头,用完念头要记得回归纯粹的觉知,安住于纯粹的觉知。安住的原则是短时、多次,慢慢安住的时间会自动延长,安住的深度也会不断加深,到时觉察力也会变得更强。
\end{itemize}

下面分享一些案例。

\begin{case}
    戒色 277 天谈一下感受,身体恢复差不多了,尿频尿急基本没有了,性格变得活泼开朗了,主动和陌生人说话了,运气也变好了,家庭也和睦了,人有精神,帅气了,最主要心里快乐敞亮了。我很感谢 277 天前自己下定决心戒色,我为什么能戒这么久一次未破,一次没看不健康的东西,因为有坚定的信念,要么戒,要么废,认真落实飞翔老师的方法,像他说的这么去做。我知道戒色道路一点不能忽悠,不是戒一天两天,而是戒一辈子,当成一生的事业,平时学习戒色文章只是考试前的复习,真正的考试是当你脑海起邪念心魔浮现时候,看你怎么应对了,就像战场上的士兵还没拿起枪就投降了,这怎么能戒色成功?必须具备坚定不移的信念,你强心魔就弱,你弱心魔就强。戒色最重要的是观心断念,时刻关注自己想法,一有不正当想法,及时断除,可以念口诀,念起即断,念起不随,念起即觉,觉之即无,每天没事就念,坚持念十天,自然有念头冒出来就会察觉。在戒色道路上必须时刻提高警惕,不能有一点松懈,心魔就像幽灵一样冒出来,管理好自己的视线,看见诱惑千万不能看第二眼。看戒色文章你觉得一针见血的句子就写下来,每次复习都会有新的感悟。几秒快感背后只有无尽的痛苦和忏悔,没有快乐,只有一身症状,为什么去贪恋几秒快感,用一辈子幸福抵押呢?一个大男人玩自己下体很有本事吗?肾精是人体最宝贵的资源,你浪费了多少!我们一起努力早日恢复身体健康,离苦得乐!

    \textbf{附评} 这位戒友的感悟很好,看到他的蜕变我很高兴,随着坚持戒色养生,身体逐步恢复,心理也恢复健康了,变得活泼开朗有精神有自信,身心一旦恢复了,人生就会顺利很多,如果身心都失调,人生就会变得坎坷和不顺,甚至处处倒霉,深陷困境。他能戒 277 天一次未破,首先他有坚定的信念,决心很大,其次认真落实了专业戒色的方法,真正去严格执行了,执行力太重要了,再好的方法不去执行就等于零,那几个基础要点必须严格落实和执行,否则是很容易破戒的,把戒色十规真正吃透,坚决执行,肯定会突破怪圈的。狠抓落实见行动,切忌空谈不练习,要真干!实干!严字当头、实字着力,抓落实,抓执行!不松劲,不懈怠!戒色十规最好每天都要看几遍,上午一遍,中午一遍,晚上一遍,反复提醒自己落实这些关键的要点,有些要点看似平常,但却极端重要,直接影响着戒色的成败。一位冠军说过:“如果你全身心投入去做一些事情,并为之付出和努力,你就会成功的。”这句话说得很好,掌握方法后,就要全身心投入,真正吸收戒色文章的精华要点,真正去落实这些要点,反复强化实战意识,在实战时做对的事,对境避开要快,坚决不看不贪恋,断念实战更果断些,如果感觉做得不好,就要及时反省,不断优化实战表现。从这位戒友所总结的经验中可以看出,他在断念和对境实战方面已经具备很高的觉悟,实战意识已经很强了,他之所以能做到一次未破,从他的这段总结中就可以看出他已经把握了戒色的核心与精髓,相信他会越戒越好的。戒色是系统工程,是生命的整体改造与重建,修身的确很重要,但修心才是最核心的内容,以修心为核心的戒色方法才是最正宗、最正确的,你的念头直接导致你的行为,如果能够学会控制自己的念头,你就可以控制自己的行为。偏离修心、偏离实战的戒色方法迟早会失败的,就像打仗一样,如果脱离实战,最后的结果肯定是失败。戒色吧并不是只是为了戒色而戒色,戒色是系统工程,是整体的改造,戒色吧各方面都已经讲到了,而且也真正把握了戒色的本质与核心——修心为主,修身为辅。这个戒色理念和大德的开示是一致的,心为根本,修身先正心,如果把修身放在核心,有本末倒置之嫌,治标不治本。大家应该都看过十界图,当中一个心字,为什么不是一个身字?因为心是最根本的,修行的核心是修心,这个理念是颠扑不破的真理,是任何人诽谤不了的。这位戒友最后几句话说得很好,几秒快感的背后是多少悔恨、空虚和痛苦的挣扎啊!那种内心的无力感深入骨髓,让人看不到希望,只有戒掉恶习,做回阳光纯净的自己,未来才会一片光明。

    金元医学四大家之一的朱丹溪专作《色欲箴》以警戒世人:“\textit{彼昧者,徇情纵欲,惟恐不及,济以燥毒。士之耽兮,其家自废,既丧厥德,此身亦瘁。远彼帷薄,放心乃收,饮食甘美,身安病瘳。}(大致翻译:不懂的人,沉迷色欲,还觉得不够,吃壮阳药助兴。男人沉迷于此,其家庭衰败,道德沦丧,身体也会染上疾病。应该远离房事,收心戒色,营养跟上,到时自然身体轻安,疾病痊愈。)”丹溪先生医术高明,是一代名医,专门写文章告诫世人要戒色,古人很懂得保精之道。前段时间一位戒友分享的帖子,关于日本人也戒色,在日本戒色也很流行,让我大吃一惊的是,原来日本有那么多戒色书籍,而中国上架的戒色书籍几乎没有,中国还是无害论当道,大多数专家还在沿用上世纪九十年代过时的错误理论。一位戒友去看医生,医生建议定期排精,一周三到五次,结果彻底阳痿了,那个医生很权威,如此权威的人居然还在推荐定期排精的谬论,这是对患者健康的不负责任。不过我也看见还是有不少医生是反对无害论的,是坚决要求患者戒除手淫恶习的,这类医生的建议值得我们参考。一位戒友说:“之前看过一位大夫,把脉后就说我是不是有手淫遗精,让我千万别再手淫了,给我开的补肾补肝的药。”一把脉就知道肝肾大亏,身体要垮啊!所以告知患者千万别再手淫了,这种医生就是有正知见的医生。
\end{case}

\begin{case}
    问好飞翔大哥,最近我戒色有一些非常好的体会,和大哥分享一下。

    \begin{itemize}
        \item “依报随着正报转”,我发现当我视线管理做得不好时,黑眼圈就很重,或者眼睛没神(哪怕我睡眠很足)。最近戒色的十四天,视线管理做得很好(比如地铁上比较挤,有女孩子靠近,自己尽量避免接触,眼神也避开),昨晚回家我大哥就说我眼睛发白发亮,和家里的小朋友很像。我笑了笑,但是我心里明白,这是我彻底戒色的结果,心干净了,眼神自然也干净了,现在我就要往这个方面多练习,我牢牢把“不聚焦、不停留、不回看”的三不原则记在心里,严格落实,真的是太好了!太精辟了!无论在线上线下,对境时很给力!
        \item 最近在看您写的《当下的力量》那一季,太好了!我最近就在加大力度观心断念,我读研,最近做项目很头疼,遇到坎了,没有进展。我今天就试了试,在学习之前先观心断念,专注于观察,并至诚念佛(我信佛),然后以平静的心态去学习。不可思议的事发生了,所有的难题今天很快地解决了。我突然明白为什么说“我们每个人都有宝藏,只是我们没有发现它”!我感觉我的生命又进入到一个新的阶段,心里无限地感恩!加油!共勉!
    \end{itemize}

    \textbf{附评} 正报指的是身心,依报指的是我们的衣食住行等生活环境,类似境随心转的道理。这位戒友提到了对境的视线管理,这是戒色十规的第五规,如果发生破戒,可以去看戒色十规(\ref{126}),看一遍就知道自己哪里没做好了,也知道该在哪方面进行强化。一位戒友说:“戒了一年了,不断学习戒色文章,慢慢发现,戒色十规才是戒色的精华,只要严格遵守戒色十规去做,一定可以戒掉的。”前段时间一位戒了几年的资深戒友也专门强调了戒色十规,他真正发现了戒色十规的价值,戒到一定时间回头看戒色十规,突然就会发现原来戒色十规已经涵盖了戒色成功的全部精华要点,而且是最关键的要点,都在里面了,能够真正落实,就能少走很多戒色弯路,一下就能紧紧抓住戒色的重点。现在夏季,生活中对境的考验太多了,为何进入夏季时破戒会增多,一方面就是夏天的太阳,晒了之后身体会有燥热的感觉,容易产生微妙的冲动,另外一方面就是夏季穿得少,街上诱惑多,面临的对境考验会比其他季节多很多。这是一个容易堕落的季节,记得我做学生党时,发育后的每个暑假基本都会狂破戒,有时一天好几次,夏季出汗多,无病三分虚,加上狂破戒,身体很容易爆发症状。夏季戒色是有相当难度的,但只要掌握戒色方法,严格落实戒色十规,还是可以顺利过关的。我最近遭遇了几次邪淫回忆的攻击,图像袭脑极快,还带着意淫的幻想,被我及时发现断除了,我一直有着很强的实战把握,所以实战时比较镇定,不会慌乱,能够有效击退心魔的夏季攻势。《当下的力量》我之前专门分享过笔记与解析,是在 \ref{128},戒友提得比较多的就是 \ref{90}、\ref{99}、\ref{128}、\ref{130}、\ref{132}等,其中 \ref{99} 和 \ref{128} 提得最多,是顿悟大爆发的两季,值得反复研读。《当下的力量》是戒色后必读的灵性书籍,很具有启发意义,可以让人领悟到自己的真我,这个领悟价值无量,犹如鲤鱼跳龙门!真我就是源头智能、无限智能,大家应该都知道智能的力量要大过念头,一些科学家和艺术家都在有意无意地安住真我,只需安住一小会,灵感就会自动冒出来,因为他们连接上了无限智能。我的戒色文章和其他一些戒色前辈的文章有共同之处,但也有独特的殊胜之处,通过我的分享,我希望大家都能领悟真我,学会安住纯粹的觉知,做真正的自己!发现每个人本有的宝藏,这个宝藏就是真我——源头智能!通过不断安住真我,观心断念能力也会得到极大的提高,断念水平会变得更稳固,这是我的实践体会。
\end{case}

\begin{case}
    飞翔哥!刚刚躺床上玩了下手机,邪念就开始出来了!我跟着跑了一会,但它还没成气候就被我发现了,一知一觉邪念就没了,消失了!过了一会又冒出来,我这次觉察明显比上次要快很多,邪念又消失了!过了一会又上来,我又比上一次觉察快了,所以邪念又被我打败了,这次过后邪念就不敢冒出来了!觉察消灭念头这种感觉真的太好了!跟着念头跑就是束缚,就是身不由己,就很难受!

    \textbf{附评} 这位戒友的实战表现还算不错,不过他不应该躺着玩手机,这个姿势容易放松警惕,他跟着念头跑了一会,好在后来的实战表现很给力。有的戒友就在抱怨邪念消灭一波,又上来一波,感觉根本顶不住,搞得自己很烦恼,其实这就是实力不济的缘故,如果实力强劲,完全可以断出威慑力,就像这位戒友的表现一样,一次比一次快,一次比一次狠,觉察三次过后,邪念就不敢上来了,我的实战体验也是如此,只要够快够狠,就能战出强大的威慑力,到时心魔就不敢进攻了,因为它知道你不是好惹的,知道进攻就是来送死,所以就不敢冒出来了。这种战胜的感觉的确很棒,有点像打篮球时,一连盖了对方两记血帽,之后对方见了你就躲开了,不敢进攻了。断念时的确要狠一点,看到一张海报上写着:“出重拳、下狠手,扫黑除恶!”我们断念何尝不是如此呢?必须斩尽萌芽,打出统治级表现,一拳躺尸!彻底打瘫!纯粹的力量碾压!在绝对实力面前,一切都是笑话!实力是平时练习和实战磨练出来的,高手对待训练的认真程度都是一流的。实力强,就能打出 KO 的表现,KO 是擂台上最受欢迎的终结方式,它往往能引爆全场,并让观众为之疯狂。实力、经验、决心都要非常强!强势 KO!绝对主宰!邪念、图像、微妙感觉一上脑,瞬间霸气清屏!我一般不用霸气这个词,因为这个词有点负面的感觉,不够谦虚,不过这个词也可以用来形容实战表现的无比坚决和强悍,面对心魔的入侵和骚扰,实战表现就应该霸气一些,凶悍一些,出狠招,下狠手,打出那种炸场的强势 KO,KO 完了之后龙行虎步,尽显王者风范!那种力量感、主宰感和终结能力,有一股威严的气势!这就是级别!最让人血脉喷张的不是黄片里的场景,而是你一记觉察重拳把心魔打成躺尸!!!那种犀利的眼神和强悍的操作!够狠,够强硬!要练快的不是你在 JJ 上冲刺的手速,而是你的断速!!!最爽的不是手淫的快感,而是断念的快感!!!你要彻底主宰自己的内心,断念要凶悍,形成强大的威慑力!成为让心魔闻风丧胆的存在,宣布你对身体的主宰权!练出百里挑一的觉察力!就像百里挑一的后手重炮,直接把心魔打晕。我喜欢这句话:“唯有辛勤地训练才是成功最坚固的磐石!”百遍不如千遍熟,千遍巧后万遍精。接受了大量训练,并且具备勇气、决心、实力和实战经验,这样才能成为高手,高手都达到了超强的水平,只要你实力够强,终将迎来闪耀全场的那天!坚持练习,断念就会越来越老道,越来越像顶级高手的做法,一阵猛如虎的操作后,心魔就退怯了。从初窥门径到顶级高手,是一条练级之路,断念非压念,而是学会觉察,压念是误区,觉察是正道,学会觉察,强化觉察,就能主宰两耳间!

    一位戒色一年的戒友:“我想说要想戒色成功,真的就是断念,只有真正学会了断念,你才能戒色成功。我现在就是会观察自己的念头,如有不好的念头,我会告诉自己这是不好的念头,赶紧打断,如果不能及时打断,你会被这个念头带着走,可能会导致破戒。”一位戒色 709 天的戒友说:“戒色如一人与万人敌,只要将猛,不怕贼强!”一位资深戒友在帖子里写道:“断念口诀是一个宝藏!很多兄弟抱怨说口诀无用,欲望上脑了再念起来根本很无力。对于这一点,飞翔大哥的观点是宝剑锋从磨砺出,当时我戒了二十多天,根本控制不好自己的念头,已经觉得自己走投无路了,于是带着半信半疑我也开始把断念口诀的练习加入到每日的戒色任务中。从开始的每天练习 500 次到提高到每天 750 次,一开始前几天效果确实不明显,但是坚持下来,到了四十多天的时候我就感觉不一样了,说说我的感觉:练习断念口诀一是可以做到迅速斩断邪念,这是最直观的效果,而随着练习的增多,保持警惕性和避色能力都有了不同程度的提高,也渐渐学会了观心。现在一个不好的念头从我心里飘过,我都能迅速察觉,判断出是否是心魔的怂恿或者迷惑。而且随着觉悟的提高,现在不仅能立刻断除直接指向看 H、破戒的念头,还逐渐扩大了范围,贪嗔痴慢疑的念头也能及时反应。念头一上脑,口诀即出,很多次戒色实战就可以轻易化解,这对于戒色初期念头纷飞的戒友来说实在是太重要了。记住,宝剑锋从磨砺出!”直到真正开始练习断念口诀,你开始变得强大,开始赢得战斗,你不再胆怯,不再惧怕心魔,你开始胸有成竹、沉着应战!

    前几天一位戒友在我帖子里反馈:“飞翔哥,现在就是吧,邪念、yy、图像一来,我一觉察到,我知道它是邪念,它来了,它就停下来了,不继续发展了,然后我看着自己的内心,安住内心一会儿,念头续流消失了。”他已经学会了观心断念,也学会了安住,实战感觉找到了,只要继续强化,就能越来越强了,其实所谓觉察,就是当念头上来时能知道、发现它,这本身就是在觉察了,一知一觉,念头就停下来了,不继续发展了,就是这么奇妙,绝对不是去压念,压念会导致烦恼和挫败,而觉察很轻松,一觉即空,轻松得很!很多戒友都学会觉察了,他们开始真正入门了。看大德开示,看到最后就是在讲观心断念,这是最高的内容,也是最基础的入手处,观心断念是基本功,也是最奥妙的功夫。
\end{case}

\begin{case}
    飞翔哥,首先谢谢您!如果有空,请您回复一下,即使只有一两句话,也会很感激您的。您的著作《戒为良药》和《SY 受害者 5000 例》伴我走了将近四个冬夏,如果没有它们,我现在已经无药可救了。特别是最近一年多,戒得很彻底,没有再犯过,以后应该也不会再去犯了。我是一名大学老师,也是硕导,虽然我工作到今年暑假才三年,该有的基金项目(国家的、省级的)奖励,我都得到了。可是我还是有个严重的心结就是还是很自卑,有时候情绪来的时候觉得自己就像条狗,恩师和前辈们都不喜欢,虽然他们没有说过,但是我就有这种强烈的感觉,自卑得都想把自己封闭起来。有时候这种感觉让我很失败,想发疯,我知道跟那些 985 学校的学者来比我很弱,可是内心还是希望能跟他们平起平坐,有时又觉得无能为力,不知道路在哪里,也不知道怎么才能让他们觉得我并不差。除了这些,想想自己家庭也不富裕,就觉得不谈学术,谈生活也是一个 loser,觉得对自己很失望。想拿更大的项目,但是我们学校平台不高,觉得根本没有希望,但是内心又有种强烈的感觉,觉得我应该可以突破。说得很乱,总之内心很自卑,但有时又觉得自己不差,就是这么矛盾!我这样是不是因为身体还没有完全恢复健康?截止到去年 SY 有十年了,这次一年多没有再犯邪淫了。我觉得身体恢复得挺好的,就是心理极其矛盾,既自卑又自负,以前也没有这么严重。是不是也因为我知道我这十年都没有尽全力做科研,所以错过很多好的事业上的机会,所以内心自责导致的?有点迷茫。这是心魔在作祟吗?这种感觉会让我一两天做什么都提不起劲。

    \textbf{附评} 《SY 受害者 5000例》是 2014 年的,2015 年完善至《SY 受害者 10000 例》,时常看看受害者案例警醒自己,对于伤精症状可以加深认识,也有利于保持警惕。这位戒友的问题就是自卑和自负,问题就是出在这两个上,没人喜欢自卑和自负的人,特别是自负的人就像刺猬一样,到处伤害别人。行有不得,反求诸己,遇到了挫折和困难,或者人际关系处得不好,就要自我反省,一切从自己身上找原因。有了问题,不要怨天尤人,而是要反躬自省。古人云:“劳谦虚己,则附之者众;骄慢倨傲,则去之者多!”要看淡名利,不要去攀比,否则人比人,那真是气死人,虽然拿到了国家的基金项目,已经挺不错了,但还是自卑,因为别人不喜欢自己,和 985 的学者相比,感觉自己很弱,家庭也不富裕,这就是他自卑的原因,对自己很失望。首先要转变自己的心态,德行要跟上,要懂得知足,懂得感恩,对于自己的评价不要过低,要经常鼓励和激励自己。自信应该建立在正能量上,不应该建立在外在的攀比上,否则心态是很容易失衡的,因为总有人比你强,一味攀比,只会让自己受挫。其次,一定要懂得行善积德,多积阴德,这样命运自然会改善的,到时候事业上会有更好的机会出现,到时一下就飞跃了。对于过去的一些错误,也不要太自责和纠结,过去的就过去了,一切向前看,要积极开朗起来,如果陷入自责,当然会感觉压抑和无力,做什么都提不起劲,感觉自己是一个失败者。\textit{善的人,将来准富,有洪福享。因为天不亏人,天理是明中施舍,暗里天还。要是为了义举,尽心竭力,任劳任怨,刻苦完成,那叫做功。有功于世的人,准出贵,有权柄。因为舍己为人是出贵的根。人借行善立功,把性情练得炉火纯青,遇逆境能和颜悦色地忍受,死心化性,才叫做德。有德性的人,才能立住万古。(王凤仪先生)}真正有智慧的人不追求名利而名利自来,因为他们有德行,懂得行善积德,懂得无私奉献,生活不再是以自我为中心,而是以无私利他为中心,这样的人生才是真正有价值的人生,很容易得到大快乐,否则经常攀比,活得很自私很狭隘,那内心肯定会很痛苦的,别人比自己强,感到嫉妒和痛苦,别人不喜欢自己,感到自卑和痛苦。如果懂得转变心态,多利他,多感恩,多无私奉献,内心真正阳光了、开朗了,充满正能量了,有德行了,别人自然会喜欢自己了,人缘自然会变好,看到别人比自己强,也能坦然接受,不会产生嫉妒等负面心态。自负只是表象,而自卑才是真正的原因,极度自负的人往往极度自卑,都是攀比心、名利心、贪心在作怪,放下负面的心态,多发正面高频的念头,处境自然就会改善,你给出什么样的振频,你就会收到什么样的境遇,你拥有改变一切的力量!时刻觉察内心是我们要做的关键。\ref{133} 德训篇要反复研读,戒到一定程度,德行必须要跟上,要从自私到无私,自私是低频,无私是高频,自私肯定痛苦,感到压抑和沉重,因为这是低频的特征,高频就是轻松、愉悦、发自内心的快乐。正确的人生观是行善积德,有无私利他的心,为大众服务、回馈社会的心,如果仅仅是为了自己,一味攀比,肯定会陷入负能量。养性贵在平易恬淡,\textit{平易恬淡则忧患不能入,邪气不能袭,故其德全而神不亏。(《庄子》)} 平易恬淡这四个字真好,心地坦荡,以平常心对待人生,知足常乐,追求内心的那一份恬淡平和,淡泊名利,寻求简单、自由的生活。《黄帝内经》里也提到了“德全”——德全不危也!君子要有自己崇高的精神追求,不断完善自己的德行,这样才能以更高的境界生活在这个世界上,不为名缰利锁所牵绊。
\end{case}

\begin{case}
    请问飞翔老师,过去曾经看到说念佛号一百万遍就能戒邪淫。我一段时间都非常努力念,争取到一百万遍,但期间还是破戒。后来因为念佛方法不当,念得伤了心脏,现在一念就心脏不舒服,请问飞翔老师,有什么方法能恢复心脏,念佛心脏不难受吗?

    \textbf{附评} 有一些文章会说念佛号多少遍或者念经、抄经就能戒除邪淫,这类文章是为了给你信心,但最终还是要学会修心的,念佛号抄经等是有加持力的,但最终还是要看断念实战,修行的最终目的是为了“降伏其心”,\textit{降伏自心才是佛法。(大宝法王)} 这位戒友念到了一百万遍,但还是破戒,这就是念佛和实战脱节,念佛是要用于断念实战的,这方面元音老人也专门强调过的,就是妄念一来,马上就要转成佛号,这个反应速度一定要快,就像条件反射一样。我修净土宗,念到现在几千万遍的佛号了,但现在邪念、图像等还是会上来的,如果我只注重数量,不注重断念实战,那我肯定还会破戒的,关键还是看实战,否则就停留于表面形式了。这位戒友还有一个问题,就是念佛方法不当,最好是默念,也不要念得太快,以免用力过猛伤到心脏,\textit{然只可听其自然,不可过为大声。过为大声,或致伤气受病。固宜小声念,金刚念,默念。以朗声常念,必至于伤气。(印光大师)} 如果身体出现问题了,就先停一段时间,好好休养,或者减少日课,采用默念或者金刚持。另外心脏不舒服,可以按揉掌心的劳宫穴,手握拳时,中指指尖所指区域为劳宫穴。心脏不适时,按揉劳宫穴有一定的缓解作用。过去有时我心脏不适时,就会按揉劳宫穴,一会就大大缓解了。心脏不适要注意静养,避免劳累和用力,保证睡眠,如果不适感持续,则应该积极就医治疗,可以看中医调理。
\end{case}

\begin{case}
    记得是去年四月八日开始,发大愿戒掉邪淫,如今一晃一年多的时间过去了,先看看我眼睛的变化,戒色前,肾气不足,上眼皮下塌!戒色后,我的左眼从单眼皮变成双眼皮!戒色前有早泄,戒色后恢复。刚开始时戒邪淫时,觉得小戒靠忍,大戒靠悟,至戒靠德!于是开始行善积德,最后领悟到,戒色的核心是修心断念,要做自己念头的观察者,觉者,取得新的突破。平时,我喜欢打坐,刚开始打坐时,各种杂念纷飞,最后,把自己的念头集中到自己的呼吸上,让心慢慢静下来。很多人破戒的根本原因是自己的意淫念头开始的,然后搜索黄片,然后破戒!所以管理自己的念头很关键,戒色的本质是念头的战争!要做到“念起即断,念起不随,念起即觉,觉之即无”,淫念刚起时,眼睛看到美女时,一定要立马断掉淫念!否则白天意淫很多会造成暗漏,而且晚上还很容易遗精。当你做到“念起即断”时,做到快速反应时,戒邪淫就会到达新的境界。戒色前,自己感情之路非常不顺,非常坎坷,戒色后会有很多好事发生,比如快要与女友领证结婚了。在结婚时,有两个绝佳戒邪淫时机,一个假如是异地恋,可以好好戒邪淫,恢复身体,另一个是老婆怀孕时,不是去出轨邪淫败德,而是好好戒色养生。我最大的理想与愿力是:度尽三恶道的众生,度尽被疾病折磨的苦海众生!戒色吧的一些戒友,每天被症状折磨,仿佛活在人间地狱一般!不过症状是最好的老师,否则也不会来戒色吧戒邪淫。希望各位戒友顿悟觉醒,脱离苦海,我们的自性本来是清净无染的,自性若悟,凡夫即是佛,自性若迷,佛即凡夫!佛的本质是指开悟的人!一切的解救之道在各位戒友的心上,念头上!(断掉意淫念头!断掉邪淫念头!)这是本质,大家一起努力!加油!境随心转,开创新的美好的幸福快乐人生。

    \textbf{附评} \textit{威力在于发心上!(黄念祖老居士)} 这句开示给我留下很深的印象,要有度众生的大心,不能自私自利,有了大心大愿,才有大行,才有更高的使命感和责任感。戒色后的好变化是显而易见的,眼睛是五脏六腑状态的直接反映,肾气足,眼睛自然睁到位,不会下塌,有的人本来是双眼皮,一下塌就变成单眼皮了,而且下塌后感觉眼睛变小了,变难看了。肾气养足,眼睛会变得明亮有神采,敢于直视对方,有底气和自信,心地光明,目光纯净。古德云:“天律于淫最严,人祸于淫最惨。小则戕生,大则绝嗣。近则削其福寿,远则灾其子孙。阳则受国宪之诛,阴则干神明之谴。鉴无或爽,数有难逃。况乎天道好还,淫人妻女,妻女必被人淫;坏人名节,名节必被人坏。理所必至,岂妄言耶。故欲念萌动之初,视如毒矢着身,恶蛇螫手。急须刮骨断腕,始免裂肝腐肠。”我看过很多传统的戒色书籍,里面都提到了断欲念、断邪念,这是戒色实战的根本,也是大德反复强调的重点,一个微小的念头会导致疯狂堕落的行为,就是一个念头没断掉,后果不堪设想,进入了极度疯狂丧失理智的丧心病狂的状态,那种为了放纵不要命的状态真的很可怕!一位戒友说:“在放纵后一天都是压抑,撸管时只有快感没有快乐。”那是一种被邪念劫持的状态,身不由己,快感不是真正的快乐,撸时往往板着一张脸,严肃中带着猥琐的感觉,眼睛死死盯住屏幕,放射出贪婪邪恶的目光,最后射掉后,感觉整个人都要衰竭了,站都站不稳了!射掉后就感觉一切没意思了,对未来也无望了,眼珠都懒得转了,像行尸走肉一样,失去了奋斗人生的斗志与勇气,虽然感受到了短暂的快感,实则潜藏着无量的痛苦和折磨。

    看到戒友的生活回归正轨,工作稳定,找到女友,结婚生子,过上幸福的生活,我由衷感到欣慰,真诚祝福他们阖家幸福美满,再分享两个反馈案例:\begin{itemize}
        \item 飞翔老师,终于找到您了,看到您还在戒色吧带领大家戒色,真是功德无量,感恩您!我是 2013 年的老戒友,当初是一个屌丝,但是来到戒色吧后发心戒色养生,虽然中途有过反复,但是始终坚信这条路是对的,所以越戒越好,2016 年找到女朋友,2017 年结婚,2018 年有了小孩,真是非常感谢您和戒色吧的兄弟!也祝愿戒色吧的其他师兄越戒越好,早日步入人生正轨!
        \item 无比感恩飞翔哥,戒色吧有的人走了,有的人来了,而你一直都在。谢谢你无私的付出,你就是很多人的戒色主心骨。因为戒色,我工作稳定了,领导同事关系都好,家庭和谐,身体也很健康。感召善缘(妻子对这方面要求不高,也支持我戒色),同时我们也有了一个可爱的宝宝,非常幸福。我要以身作则,好好戒色,做孩子的榜样,希望他成长之路,不受色情之毒!再次感恩飞翔哥,我会努力学习,努力练习,学习你的正己化人!
    \end{itemize}
\end{case}

\begin{case}
    不知不觉戒色六百天了,一个人,走过了春夏秋冬,走过了严寒酷暑,六百天,一个曾经想都不敢想的数字,我做到了,我真的做到了。感谢飞翔老师,感谢戒色吧相互鼓励的兄弟们,没有你们的支持和力量,我不会有自己的今天。昨天面试完,在回来的路上我坐到公园里,看着明媚的阳光,树梢上闪闪发光的影子,湛蓝的天空,不时还有鸟儿飞过,泪水又打湿了我的眼眶,感恩。这次反省,我想向大家说说自己在外经历的一次违缘以及由此引发的魔考的思考及再认识。在 5 月 17 日的晚上我因为第二天要面试,就去了外边的宾馆住了一晚上,我住的地方离我面试的地方很远,所以我特意早早在宾馆入睡了。可是,因为宾馆的房间比较小,而且房间里靠近烟道,有很多烟气,很呛。我在凌晨醒来了,然后听到了隔壁很不好的声音,大家应该都懂在干什么,这个时代邪淫的人太多了。然后我用卫生纸沾湿了水将耳朵堵得死死的,当机立断坐起来开始念断念口诀和佛号。这次违缘,我印象深刻,魔考,不仅有内心的,更有外部的。如果我们不小心,很可能结果只有一个——破戒。实战,我认为不仅仅是对峙念头,更重要的是当机立断的勇气和那股狠劲。尽管很痒,但是必须狠断!根除所有的舍不得,根除自己的邪念。同时戒色后期,难免会有魔考,外缘的恶自然也会不少,很多邪淫的事情就发生在我们身边,作为修行持戒的人,应该立马远离,也就是学会保护自己,学会持戒,学会实战的那狠狠一下。我们必须拥有断念的实力,必须远离邪恶的外缘。就像剑客要有剑,士兵要有枪一样,不能只靠福报,福报得有,但还是得有自己的实力。光有福报没有保护自己的功力,迟早遇到强大的欲望而破戒。

    \textbf{附评} 这是一位资深戒友的反馈,他戒得还是挺不错的,这次在宾馆的对境考验的确很严峻,不过他的处理还是不错的,警惕性很高。这个时代色情泛滥,邪淫的事情也很多,特别是宾馆这种地方,更是邪淫的场所,所以去这种地方,一定要提高警惕,加强修心。当然不是每次住宾馆都会遇见这种情况,但还是要提高警惕,一个人住宾馆时也要做好慎独,慎独永远是强调的重点,古人已经强调了几千年,因为独处时心魔容易入侵,独处缺少潜在的监督,很容易放纵,所以必须做好慎独。慎独是戒色十规的第四规,是绝对要注意的重点。实战需要当机立断的勇气和狠劲,不能有一丝一毫的犹豫和贪恋,实战过后的一两天心魔还会以回忆的方式入侵,到时还需提高警惕。最近看到两个案例,一个就是遗精后出现破戒,遗精后身体状态会变差,邪念会变得活跃,这时候必须提高警惕,注意休养和调整。另外一个案例就是酒后破戒,是一位戒了好几年的老戒友了,因为戒色,他的人生成功逆袭,但也许是戒久了,放松了警惕,在喝酒方面没慎重,结果导致破戒。一个遗精后,一个喝酒后,这两个点都要严格注意,尽量避免喝酒,即使应酬,也要尽量少喝。早晨醒来也是一个点,很多人都是醒了之后破戒的,这时候应该避免赖床,马上起床。戒色戒到后来,所要强化的就是那些看似基础的点,基础的点做扎实了,戒色大厦才能稳固,有时真的一着不慎,满盘皆输,前人的教训真的很深刻,如果你不注意,也许某一天也会发生在你身上,所以不能盲目自信,还是小心谨慎为妙,小心驶得万年船。一位戒友说:“今天是戒色第 103 天,但我今天才明白,一定要把戒色的每天都当做是戒色第一天,要有戒色第一天的心态、斗志、决心。就算戒成功十年,也要在内心中告诉自己,今天是我戒色第一天,并永远保持如履薄冰的警惕之心。只有把戒色的每一天当做是戒色第一天,才能永久戒色,与诸位戒友共勉!”戒色后要面对的考验有很多,网络上,生活中,脑海里冒出的邪念图像等,要反复强化实战意识——外避内断!在实战中真正做到位,强化基础要点的认识与执行,这样才能立于不败之地。光有福报当然是不行的,大家看新闻都知道,很多有福报的人因为犯了邪淫栽了大跟头,身败名裂,一定要具备断念保护的能力,其他说得再多,做得再多,如果断念不力,那依然还是会失败的,有福报却不懂得修心,那是很危险的。
\end{case}

\begin{case}
    戒色以来终于感受到了纯真的大快乐,断断续续一直有破戒,但是也不敢放弃,更不能放弃。今天吃完饭想到回宿舍一个人有点无聊,就随便走走。走到上班经常经过的那条不通车的铁路上,很多人,老老少少都有,看着旁边玩耍的小孩真的是太可爱了,我直接在铁轨上躺了一会,躺上去一天的疲劳感觉消失了,腰腿都放松了。正好看见天空蓝得可爱,旁边的树也是那么美,和蓝天一起,映入我的眼帘,这一刻我感觉真的太幸福了。虽然这里的人我都不认识,但是我能真切地感受到每个人的气息,世间真的有比一个人窝在房子里偷偷看黄片爽一百倍的事情。慢慢学会生活,找到曾经无邪的少年,一起加油!

    \textbf{附评} 跳动的不止是心脏,还有纯净的灵魂,恢复纯净的自己,才能真正快乐起来,再次回到高频美好的神奇世界,为的就是那一份久违的纯净美好的内心感受。多少纯真的孩子在邪淫后眼睛暗淡了下来,脸上再也没有了纯真无邪的笑容,这是非常可悲的事情,他们进入了灰暗的邪淫世界,人生的灾难在等着他们去承受。当你不再邪淫了,你会发现因为沉迷色情与邪淫所忽略的一切:纯真的孩子、慈祥的老人、美丽的自然风景、内心的纯粹美好、爆棚的幸福感,一种高频振动的灵魂大爽再次眷顾了你,仿佛回到了童年的神奇时刻,那种感觉太久违了。告别邪淫的龌龊、肮脏与不堪,归来仍是纯真美好、充满活力的少年。赶走油腻沉重的感觉,浑身清爽而轻松,让你重新像个少年一样,迎风而行,千树花开。云淡风轻的日子,纯白的灵魂飘扬在清风里,无限惬意,满满的少年感,永远的向往。

    戒者涤生:“戒色第 159 天晨跑,戒色的纯净大快乐使我如生活在天堂!真好!今天继续早起晨跑,这纯净的大快乐让我身心愉快,犹如在天堂漫步,戒色太快乐了。……戒色太爽了,我感觉自己越戒色越快乐。这种快乐是以前邪淫时从未有过的自由的纯净的大快乐!现在想想,自己以前邪淫真是太蠢了,蠢得跟猪一样的。恶心的邪淫快感根本就无法与戒色后的纯净大快乐相比,根本就不在一个等级,一个层次。生命不息,戒色不止!戒色的每一天,真好!……今天来公园逛逛,这纯净的快乐感觉太爽了!戒色后的每一天感觉都生活在天堂!我爱上了这戒色的感觉,我觉得戒色就是要爱上戒色的感觉,才能永久戒色!反正我是爱上了戒色的感觉。戒色是一辈子的事情!加油吧戒友们,只要标准戒色一百天后,你也会体会到戒色后这纯净的大快乐,也会体会到每一天如在天堂漫步般纯净的大快乐!”(戒者涤生这段文字很好,他爱上了戒色的感觉,美丽的风景和纯净的自己,这是绝配!突然就会感觉到那种大爽,这种纯净快乐的感觉真的太爽了,仅仅是做回纯净的自己,就能释放出如此大的快乐,实在不可思议,只有亲身体会过的人才知道。)

    “小学到初二这段时间是我人生中最快乐,最单纯的时光。那段时间,心中没有任何邪恶,无比的天真和快乐,没有烦恼没有忧愁。可是,自从初二下学期,家里买了电脑,接触了网络以后,我的人生开始向另一条路发展,那年我十五岁,正直青春期,对异性有着很强的好奇心,开始上网搜不良图片,偶然一次,进入了一个 H 网,从那以后,我永远告别了那个天真无邪的我。”(这位戒友的经历,相信大家感同身受,心中没有邪淫的念头,单纯而美好的时光,那段时光真的很快乐,后来发育了,开始搜黄看黄,进入了另一个世界,一开始以为是天上掉馅饼的好事,后来才发现这是一个废人的陷阱,想挣扎着跳出来,却一次次失败。)

    “戒色养生十个月我整个人都感觉脱胎换骨了一般,整个人有气质了,状态好了,干净了,没痘痘了,皮肤不灰暗不油了,眼睛真的变大了,眼神犀利有光了,而且我发现人际关系好了很多,社交恐惧症基本消失了,心态也好多了,不失眠了,免疫力全面提升,很久都没生病了,纯净的健康快乐真的太好了,我不想回到以前那种地狱般的生活了,继续坚持会更好的。”(这是戒友“棋少”的反馈,脱胎换骨,焕然新生,这种感觉太棒了,告别油腻灰暗的感觉,变得清透明亮,社恐也基本消失了,人际关系改善,身体变好。)

    “昨晚梦见自己的小学时代,自己背着小书包告别自己的父母,和自己的几个发小一路走到学校,进了教室看着熟悉的一张张面孔,心底里很高兴,和男女同学都无忧无虑地嬉戏,每一个笑脸都是那么阳光。那时候的我是多么纯粹呀,自从初中被一个比我大一岁的孩子带着看黄碟之后,就一发不可收拾了,从来没有做过这么美好的梦境,心底更是有信心告别邪淫的自己。”(这位戒友的梦境很美好,纯净的自己是如此魂牵梦绕,回归纯净的状态是一个人最深的渴望。)
\end{case}

下面步入正文。

这季是关于沉迷游戏的不良影响,关于这个主题前几年就有过想写的想法,只不过时机还不够成熟,到今年的现在,感觉时机已经成熟了,关于这个主题是该好好谈一下了。色情与游戏是很容易让人沉迷的两个东西,记得在接触色情之前,我就对游戏沉迷了,想方设法去玩游戏,拿家里钱、卖铁,去打街机,我到现在还记得好几家街机房的位置,有的已经拆了,有的则变成了其他的门面。那是上世纪九十年代的时候,当时我还是小学生,小孩子天性爱玩游戏,什么新鲜的游戏都爱接触一下,小时候玩过打弹子、打洋牌、滚铁箍、跳方格、打陀螺,也跳皮筋,那个年代男生女生都跳皮筋,还有跨大步等,我现在回忆那个年代,感觉那时候脑子特别单纯,那时的我是一个腼腆的男孩,不善言辞,但很顽皮。记得某个暑假,电子游戏出现了,黄颜色的游戏卡,我到现在还记得那种质感和气味,八零后和九零后应该都有印象,刚开始也玩过俄罗斯方块的掌机,还记得我的第一个游戏机是我爸爸买的,叫黑金刚!游戏机是黑色的,后来还买了小霸王学习机,基本没怎么用于学习,就光玩游戏了,还记得小霸王学习机可以练习五笔打字,可是那时却没有去练,只是一味玩游戏了。

我那时很喜欢玩街机,从小学一直玩到了初中,上了高中基本就没玩了,渐渐告别了街机房,还记得最后一次去街机房玩的是大家来找茬,和几个儿时伙伴一起玩的,那是我最后一次去街机房,几个儿时的伙伴也都长大了,大家一起玩街机,似乎在重温当年的感觉。网吧出现后,街机就没落了,渐渐退出了历史的舞台,不过现在有的地方还有,我前段时间看见几个小学生在超市门口的几台街机上玩格斗游戏,发连招和大招,我就想起了我小学的时候了,也是那张稚嫩的脸,那双稚嫩的手,还有一脸的纯真,在那摇街机的摇杆。我个人比较喜欢玩闯关游戏,也喜欢玩格斗游戏,只是那时我并不知道人生的一大关口正等着我去闯,这道关就是色关!一道横在我面前的大关!古今中外,多少人都是倒在了这个关口,上至帝王总统,下至黎民百姓,都要闯这个色关,我们生下来似乎就是来闯关的。\textit{色,少年第一关,此关打不过,任他高才绝学,都无受用。(《寿康宝鉴》)} 也有一个大 BOSS 在镇守色关,它就是心魔大 BOSS,那时的我并不知道什么是心魔,根本没概念,也不知道心魔的出招表,自然无法战胜心魔,那个阶段的我处于非常无知的状态,把念头当做自己,心魔稍微一怂恿,我就会失控,心魔那时对我进行最多的怂恿就是:“最后一次,下次再戒”,还有“只看看,不撸”,心魔鬼鬼祟祟在我脑海里说话,我每次都以为是自己的想法,真的就是认贼作子,敌我不分!我那时真的被心魔虐得遍体鳞伤,心魔大 BOSS 太强大、太狡猾,也太阴险了,而我太无知、太菜鸟、太弱小,注定被虐,虐了我十几年,我才翻身,在邪念附体的情况下,我做过很多荒唐错误的事,真的感到很惭愧。

我那时闯关游戏和格斗游戏玩得都很好,水平很高,在街机房属于高手之列,这和我的大量投入是分不开的,水平是越玩越高的。我到现在还记得初中时在街机房和人决斗的场景,是玩格斗游戏,有时对决的场面真的是惊心动魄,胜负往往就在一招之间。这段经历也给了我一些启示,要成为戒色高手,必须要坚持学习和练习,即使暂时失败也不气馁,继续坚持下去,向高手学习,看高手怎么操作,跟着学,这样终有一天自己也会成为高手,一定要坚持投入,在实战中不断摸索不断总结经验,强化实战意识,把断念练熟。每一位高手都有很多失败的经历,他们都是从失败中崛起的,是一步步变强的,只要坚持下去,不断完善和提升,就会变得越来越强,过一段时间,就和别人拉开了差距,水平超过了很多人。关键还是持之以恒地投入,对于失败有一个正确的态度,不气馁,不放弃,不断反省和总结,吸取失败的经验教训,坚持学习和练习,就能变得强大起来。记得以前看过一部影片,反派大 BOSS 在虐主角的时候说了一句话:“你还不够强大!”这句话我记得很深,感触也很深,当我不够强大时,真的被心魔随便虐,一触即溃,后来我强大后,局面反转了,我开始击溃心魔的进攻,开始真正主宰自己了。《一万小时天才理论》里有四个字很好——“精深练习”,技能线路锻炼得越多,使用就越自如,就像学驾驶,一个动作又一个动作地反复练习,最后熟能生巧,形成条件反射。增加髓鞘质的途径就是精深练习,髓鞘质公式:精深练习 $\times$ 一万小时 $=$ 世界级技能。

小学时,班里的同学分为几拨,有的爱运动,有的爱玩游戏,有的不爱运动,但很喜欢玩游戏,我是既爱运动,又爱玩游戏,但沉迷程度不是最强的,我记得我的一位小学同学,他那个游戏瘾非常之强,他家人把他锁家里,他会叫其他同学去他家楼下,他把钥匙从六楼扔下来,然后叫同学跑上去从外边开门,他家的锁从外边锁了之后,里面即使有钥匙也打不开,必须从外边才能打开,所以他就打电话叫同学过来帮忙开门,就是为了放他出去打游戏。游戏对于学生党的吸引力是极强的,很少有学生不玩游戏,这类学生是很少见的,绝大多数都有在玩游戏。到了发育期,对我吸引力更强的东西出现了,那就是色情,到了那个年纪,突然被色情的内容所深深吸引,这种吸引力甚至超过了游戏对我的吸引力,色情就像毒品一样,让我深陷其中,无法自拔。我还记得那几年租黄碟买黄碟的经历,非常荒唐,非常堕落。把黄碟放进影碟机时,我是极度期待的,甚至手都是抖的,心脏也快跳出来了,紧张到窒息,就像打开毒品一样。在初中阶段,可以说游戏、色情和手淫恶习毁了我,一点没错,因为沉迷游戏,耗费了大量的时间和精力,对于学习就不上心了,因为沉迷色情与手淫恶习,我的脑力迅速下降,体质大幅下降,伤精症状开始爆发,那时无知还吃补药纵欲,十几岁时就是慢前患者了。

到了大学,同学之间的话题也离不开游戏,那时网游已经出现了,我从高中起就很少玩游戏了,一方面是学业的压力,另外就是我对网游不大感兴趣,我那时的兴趣在文学和运动方面。我在大学时玩过枪战类的网游,但从来没有沉迷过,玩得也很少,也从来没有因为玩游戏通宵过,记忆中从来没有过一次玩游戏通宵的经历。我现在想想为什么神经症是在十几年后才找上我,而不是手淫的前几年,因为以前的我从来没有熬夜,也很少久坐,那时我热爱运动,作息饮食规律,后来我开始熬夜久坐纵欲,这三样伤身的全占了,结果没几个月神经症就开始爆发了。现在很多十几岁的孩子就得上了神经症,原因就是熬夜久坐纵欲,光有纵欲,还不至于废得那么快,如果三样全占,那就危险了,而导致他们熬夜的最大因素就是沉迷网游,熬夜久坐久视,身体得不到恢复,甚至在熬夜时还看黄手淫,这几种因素综合在一起,就很容易得上神经症,得上神经症就生不如死了,到时就苦大了。那几个耳熟能详的练级打怪类的网游,我一次都没玩过,那似乎不是我的菜,我大学最多玩玩枪战类、运动类和益智类的游戏,对于要花费大量时间、精力乃至投入金钱的网游,并不感冒。

我们戒色后应该戒掉沉迷游戏,杀戮类的游戏不要去玩,会增长自己的杀心和戾气,打赢了就骄傲,得意忘形,打输了就嗔恨,要报仇,这都是负能量。家人不让玩游戏,就和家人吵架,玩游戏入迷了,其他什么都不管了,戒色文章也不看了,一切的注意力都放在玩游戏上了,和正常生活都脱轨了。之前就看过一些戒友本来戒得很不错,后来就是犯了游戏瘾,结果导致戒色状态一落千丈,频繁破戒。所以很多戒色前辈都说要戒掉网游,甚至直接说戒掉一切游戏。我不会说戒掉一切游戏,有些益智类的游戏还是可以玩一玩的,但容易让人沉迷的游戏则应该要戒掉,否则对自己的生活、学业、事业的影响都会很大。看过新闻,有的人贪污巨款玩网游,有的人通宵玩网游猝死在网吧,有的人连续一个月待在网吧,吃喝拉撒睡觉都是在网吧。以前去网吧时,看到那些人大多都在玩网游,这是网吧电脑屏幕上最多见的画面,我已经好些年没去过网吧了。我家里有电脑,也没必要去网吧了,我也不喜欢那样的场所。

\subsubsection{沉迷游戏的危害}

\begin{itemize}
    \item 严重影响视力。眼睛长时间对着电脑屏幕,造成视力下降、飞蚊症等,有损眼睛健康。
    \item 易致心血管疾病。沉迷于网络游戏的人精神高度集中,在屏幕强光与噪声的刺激下,容易导致血液加速、心跳加快、血压升高,从而引起头昏、头痛、心律失常等症状。
    \item 易致幻觉与焦虑。在网吧玩游戏时间长了之后会产生幻觉,注意力下降,反应能力变差,在心理上还会产生焦虑情绪。
    \item 占用大量时间,影响学习成绩,一旦上瘾,坐在电脑旁就是好几个小时,甚至还有好几天的。另外,在上课时总是想着网络游戏,会产生走神、精神不集中的现象。网游一玩就是几个小时,很占用学习时间,游戏有强烈的吸引力,孩子的自控能力普遍很低,如果家长不严加管教,孩子就会沉迷游戏,无法自拔,成绩一落千丈。
    \item 更可怕的是心灵的腐蚀。不良网络游戏往往充斥着暴力和色情的内容,对暴力的崇尚和追求,以及强者通吃的法则,对于性的夸张歪曲表现,诸如此类内容都会对青少年的心理健康造成危害。
    \item 形成孤僻的性格。玩网游就会很少跟身边的人沟通,长此以往,就会形成孤僻的性格,不愿与外界的人接触。我们称此类人为宅男,宅在家里,不愿与外界接触,不愿与别人沟通,只把自己沉浸在游戏的世界里,不合群,不会与人沟通,自闭,对孩子的发展是极为不利的。
    \item 影响恋爱。经常为了玩网游而跟女友吵架,在恋爱的过程当中,一个男人过度沉迷于游戏,女人就会感觉被忽略,这样就会影响到两个人的关系,而有的人因为打游戏,总是在约会的时候迟到,这让女方觉得非常难受,很多女性都不喜欢对方打游戏。
    \item 浪费大量的金钱和时间。想要好的装备肯定是要花钱买的,想要厉害,光靠技术是远远不够的,花钱买装备,花钱开会员,花钱充值各种钻。沉迷游戏不仅浪费时间,也浪费大量的金钱。
    \item 经常在游戏里找自豪感,而回避现实。沉迷游戏不能自拔的人,往往在游戏中自认为是很了不起的角色,在游戏里风光无限,不愿回到现实中来,可是游戏毕竟是游戏,不管在游戏中是怎样的大神,回到现实中,自己还是一个无名小卒,有的人会有很大的失落感,甚至严重者有愤世的倾向。
    \item 影响工作,下班的时候玩,影响休息,影响第二天上班。如果是上班族,晚上打游戏经常会熬夜,很晚才睡,势必会影响睡眠质量,睡眠不足,导致上班时精神不振,不能集中精力工作,甚至会经常出错,造成不必要的损失。上班的过程中也会经常想着游戏中的事,而影响工作效率。这是非常不好的,长期以往,不仅得不到重用,也会职位不保。
    \item 虚度了人生最好的时光。在人生最美好的时候不选择奋斗,而去沉迷网络游戏,以后事业无成的时候,肯定追悔莫及。
    \item 引起角色混乱,诱发犯罪行为,网络游戏给了扮演各种各样角色的机会,同时也为角色混乱埋下隐患。长期玩砍杀等游戏,内心也会充满暴力,容易诱发犯罪。
    \item 没钱时会非法获取,一旦玩游戏捉襟见肘,便会想着搞钱,通过非法的渠道获取,比如偷钱、抢劫等。
    \item 损害身体健康,影响正常发育,学生正处在长身体的时期,长时间僵坐在电脑前缺乏适当的锻炼,会引起他们视力下降、精神疲惫,严重的直接猝死。当游戏者把大量的精力用来玩游戏时,自然除了坐在电脑前,基本不可能再去做其他事情,这样天长日久,自然对身体健康带来诸多危害,如视力下降、体能下降、饮食无规律导致的消化系统疾病等。长期坐于电脑前会导致神经紊乱,体内激素水平失衡,使免疫功能降低,引发各种疾患。
    \item 人格异化,性格扭曲,淡化道德约束。沉缅于网络游戏的学生,往往过度依赖于网络的虚拟世界而不能面对现实世界,而且长时间玩网络游戏,会淡化他们在现实社会的社会规范和道德约束,导致现实生活中人际关系不和谐,家庭反目。
    \item 影响身体恢复,戒色后要注意养生,沉迷网游很伤身体,会导致恢复变慢。
    \item 不少网游加入了色情暴露的元素,让人看了容易起邪念,这样就会诱发破戒。
    \item 打游戏打输了影响情绪而破戒,打赢了又狂妄无比,负能量太重,也会间接导致破戒。假如输了,心里肯定不甘,非要赢一把不可,赢了一把,还要再玩一把,而且一玩就是五六个小时,甚至经常通宵。
    \item 影响戒色状态,会导致戒色状态严重下滑,把精力和时间都放在游戏上,不再认真完成戒色日课,变得懒散和敷衍。
    \item 影响正常的学习、生活,把大量时间消耗在虚拟世界时,留给现实生活的时间自然就很少了,这样一来,根本不可能有更多的时间和精力来做好现实生活中该做的的每一件事,于是出现成绩下降、上课没精神、逃课等现象。内心一直想着玩游戏,其他都不管了,也不感兴趣了。沉迷在游戏里不能自拔,一旦玩游戏上瘾,就对身边一切事物漠不关心,全部的精力都放在游戏上面。
    \item 脾气变得暴躁,谁不让自己玩游戏,就和谁闹,不孝父母,顶撞父母。
    \item 在游戏中打打杀杀,到了现实生活也是看别人不顺眼,嗔恨心很重,很容易动怒和起杀心。
    \item 语言不文明,在游戏中充满了骂人的肮脏语言,这些语言接触多了,自然会受到影响,久而久之,就可能在现实生活中脱口而出。
    \item 具有暴力倾向,由于在游戏中经常是一言不合就拔刀相向,久而久之自然会对游戏者产生一些影响,而这影响在不知不觉中就被带到了现实生活中,所以出现一些学生因很小的事情就发生冲突,甚至导致流血事件的发生。
    \item 神经衰弱,沉迷网游,熬夜久坐,长时间处于疲乏状态和体力的超负荷状态,引起大脑皮层的兴奋和抑制功能紊乱。整天面对电脑或手机,不但导致近视、失眠,还会出现心血管肠胃不适等疾病,会降低记忆力,让人变笨。长时间沉迷网络游戏会导致植物神经紊乱,体内激素水平失衡,使人免疫功能降低。
\end{itemize}

\subsubsection{如何界定游戏成瘾}

诊断标准一共有十二条:

\begin{itemize}
    \item 对玩游戏的渴求(玩游戏的行为、回想玩游戏和期待玩游戏支配了个体的日常生活);
    \item 不能玩游戏时出现戒断症状(可以表现为易怒、焦虑、悲伤);
    \item 耐受症状(需要玩的时间越来越长);
    \item 无法控制要玩游戏的意图;
    \item 因游戏对其他爱好丧失兴趣;
    \item 即使知道玩游戏的潜在危害仍难以停止;
    \item 因玩游戏而向家人朋友撒谎;
    \item 用游戏逃避问题或缓解情绪;
    \item 玩游戏危害到工作、学习和人际关系;
    \item 对玩游戏的控制受损(比如时间、频率、持续时间、场合等);
    \item 玩游戏的重要程度高于其他兴趣爱好和日常生活;
    \item 即使导致了负面影响,游戏行为仍在继续和升级。
\end{itemize}

\subsubsection{如何戒游戏瘾}

有句话说得好:“一入网游深似海,不知不觉中,已经带你进入万丈深渊,你的学业、你的青春、你的健康、你的人生价值观、父母的期望等等,都将成为昨日的泡影。”认识到游戏瘾的危害,就要认真下决心,并且要掌握一定的方法,严格去执行和落实,慢慢就能克服游戏瘾。应该培养广泛的兴趣爱好,从其他活动中获得成就感和满足感,尽量远离网络游戏。这其实就是一种替代,用其他活动来替代网络游戏。培养一些兴趣爱好,比如太极拳、瑜伽、乐器、打球、摄影、画画、书法、烹饪、种植、旅行、学习一门外语……或者是自己感兴趣的任意一件事情,当然必须是正当的事情,不能是不良的嗜好。也可以多行善积德,从助人当中感受到快乐,你会发现与网络游戏相比,这类正能量的活动会让自己更快乐、更充实,生活也会更有意义,一旦自己充满正能量了,人际关系也会自然改善。父母希望看到一个充满正能量、积极向上、阳光开朗的孩子,而不是疯狂沉迷网游,在打打杀杀的虚拟世界里迷失自己。

一个游戏者的深刻反省:“三年,5000 小时,从大三到大学毕业,如果不是玩游戏,我可以用这些时间做很多有意义的事情。正值 22 岁到 24 岁的年纪,是一个人成长学习提升自我的黄金时期,我却把时间都浪费在了游戏上面。本来想出国深造,本来想好好学习拿奖学金,本来可以保研,都因为玩游戏耽误了一天又一天。大把的时间被浪费了,即使是在毕业之后准备考研的四个月,我也没有离开游戏,总共学习的时间大概一个多月,剩余的三个月都是在游戏中度过。如果不是玩游戏,5000 个小时,三年的时间,我可以多出去走走,锻炼身体,用 1000 个小时学习英语,准备出国考试,用 1000 个小时学习学校的知识,用 1000 个小时准备考研,用 1000 个小时学学编程,深度学习,搞科研。而我,却在游戏中换来了毫无意义的人生。出国无望,保研错过,考研失败,勉强毕业,三年游戏带给我的是失败、遗憾、丢人。”

我戒到现在没玩过一次网游,也没喝过一次酒,也很少出去聚会玩乐,这是我保持戒色状态的一种方法,我不想把时间浪费在虚拟世界里打打杀杀,换来一身的负能量,我更希望自己在现实世界里行善积德,一点点培养自己的正能量和提升自己的德行,平时多看大德开示,多做笔记,不断总结、提炼和领悟,感受顿悟之乐。我有自己的兴趣所在,不需要在网游中寻找某种成就感,那种靠打打杀杀得来的成就感也是我不喜欢的,那种成就感太肤浅,不管在网游中取得多高的地位,在现实中还是一条可怜的蛆虫。在网游中混得很厉害的人,往往都是非常狂妄的人,大家观察下,基本上都有这个特点,非常狂妄,目中无人,觉得自己很牛逼,其实负能量都爆表了,还那么无知和肤浅。

要戒掉网游,可以这样思维:

\begin{itemize}
    \item 不玩网游并不会失去什么,反而节省了很多时间和精力,可以用于其他的正事;
    \item 网游迟早会玩腻的,花费那么多时间和精力,实在不值得;
    \item 玩网游容易起负面念头,对自己很不利;
    \item 父母不希望看到自己沉迷网游;
    \item 玩网游容易影响戒色状态和身体恢复;
    \item 沉迷游戏会让身体变差;
    \item 现实世界才真正值得投入;
    \item 告诉自己已经受够了沉迷游戏的生活了;
    \item 这不是你真正想要的生活,你还有更重要的事情要做;
    \item 玩物丧志,与其沉迷游戏,不如为理想拼搏;
    \item 游戏迟早会玩腻,到时你就不会觉得它好玩了。
\end{itemize}

深刻认识到危害,好好下决心,建立其他的兴趣爱好,学会念头管理,这样是完全可以戒掉网游的,不要把时间浪费在网游上,应该把主要的精力和时间放在学业和事业上。我偶尔也会玩游戏,但不是网游,益智类和体育类的游戏偶尔可以玩一下,我目前没有什么玩游戏的冲动,想起来了半个月或者几个月玩一次,每次十分钟或者半小时,先控制好时限,然后玩一下,很注意时间,不会让自己沉迷其中。记得以前我有迫不及待想玩的冲动,那种冲动很强烈,而现在那种冲动彻底消失了,游戏已经变成生活中可有可无的东西,是生活中很小的一个内容,不会再占据我生活的主要。我觉得不玩游戏真的挺好的,我可以节省很多精力和时间来干其他有意义的事情,这样生活就能更充实、更有意义和价值。

\subsubsection{如何戒手机瘾}

电脑的出现让很多人沉迷其中,而智能手机的出现则加剧了这一情况,智能手机没有地点限制,更容易沉迷,随便一点就能上网,各种游戏,各种应用,层出不穷。智能手机的覆盖面更广,下到几岁小孩,上到八十老翁,手里都在捧着手机,手机低头族随处可见。随之而来的问题就是手机依赖症,玩手机成瘾,这个问题已经变得越来越明显了,越来越普遍了。我们应该认识到手机成瘾的危害性,自己也能体会沉迷手机后的一些变化,除了视力的下降、颈椎腰椎的酸痛、睡眠不足等身体健康方面,还会导致浪费时间,学习效率变低、现实人际互动减少、情绪易烦躁等心理层面的隐患。

看过一篇报道:一位妈妈求助,如何帮儿子戒掉手机瘾。

\begin{quote}\it
    “儿子有很严重的手机瘾,这两年,我跟他作了无数斗争,身心俱疲。”昨日,陈女士打来电话,哽咽讲述她儿子手机成瘾的情况。陈女士的儿子从小成绩都挺好。初三时,家人送给他一部手机,之后儿子就像变了个人。吃饭、走路、坐车,无时无刻不捧着手机。晚上,玩手机到凌晨一两点,有时候三更半夜了还对着手机唱歌,成绩也一落千丈。为此,陈女士没少跟儿子发生争吵,几乎“两三天就要闹一回”。只要一说“别玩手机”,儿子就会凶巴巴站起来,跟她顶嘴。手机没话费了,儿子就把她的手机抢过来给自己充值。上个月的一天晚上,陈女士让儿子关机睡觉,儿子不听。陈女士要收走手机,儿子就犯犟,甚至于时不时地做出一些过激的行为,为此,陈女士也是非常头痛。为了管住孩子的手机瘾,陈女士也请了很多人帮忙劝说。孩子总是当时答应得好好的,没过两天就又玩起来了。“这两年,我不知道因为他的手机瘾,哭了多少场。”陈女士向记者求助,询问有什么办法能帮助她儿子戒掉“手机瘾”。
\end{quote}

手机依赖症,你有没有?你吃饭的时候有没有玩手机?走路的时候有没有玩手机?是不是总想拿着手机上网、看电影、聊天、游戏、刷朋友圈?不管手机响没响,亮没亮,在几分钟或者十几分钟之内,都会把它拿过来看一看?如果手机没在身边,还会感觉坐立难安?如果这些症状你都有,表示你已经患上“手机依赖症”了。这类人群,一旦手机没电或忘带手机,就会出现心理上,甚至是生理上的种种不适,他们必须机不离手,才能获得安全感。长期过度使用手机会造成不同程度的身体症状和心理症状,症状包括眼球胀痛、视物模糊、视疲劳、眼干、眼涩;颈部疼痛、肩僵脖痛、手指发麻、腰酸背痛;焦躁不安、睡眠不稳、失眠、情绪低沉、无精打采、暴躁易怒等。

一位手机瘾者的自述:

\begin{quote}\it
    最近发现自己在手机上花费了好多时间,而且情况还愈演愈烈。这让我开始反思自己,本来制定好的计划,结果因为玩手机导致计划比原来滞后很多。例如:原计划三个月的学习,结果每天晚上回来抱着手机,一玩就是大半夜,等回过神的时候已经到了睡觉的时间了,这种状态一直持续了好几个月。很多时候悔恨自己不应该这么做,事情发生了只有想办法去补救,于是下定决心戒掉可恶的手机瘾。每天早上一醒来拿起手机看看天气,再打开微信,刷刷朋友圈,结果半个小时没有了。接着上班,一有空就想起早上看的搞笑视频,拿起手机接着看,半小时又没了。吃饭时拿起手机接着玩,半小时又没了。晚上没事干,放肆地玩手机,结果一不小心就到十二点多了。一整天都在围着手机转,其实手机上没有那么多要紧的事要你处理,紧急情况都会打电话的,浪费时间大多都在一些娱乐 APP 上。手机瘾的危害我自己是深有体会,第一,颈椎受不了,时间长了驼背很明显,气质形象明显下降。第二,眼睛受不了,工作本来就是每天盯着电脑,好不容易下班了。还要继续看手机,眼睛经常干涩。有时候早上起床一照镜子,发现眼睛红红的。第三,自己也变得沉默了,和身边的人交流也少了很多。所以要下定决心戒掉手机瘾,改一改这些坏毛病。
\end{quote}

手机瘾真的太普遍了,特别是养成习惯后,会花越来越多的时间在上面,一有空就捧着手机,吃饭也捧着,上厕所也捧着,洗脚时也捧着,坐电梯也捧着,坐车也捧着,坐地铁也捧着,手机真的占据了太多的精力和时间。捧着手机,很快半小时就过去了,很快一小时就过去了,很快几小时就过去了,而自己还是一事无成,很多正事、要事都没有去做,自己和别人的交流也少了,沉迷在手机的世界里不能自拔,身心都出现了不好的感觉。

以下这段来自于一位戒手机瘾的高中生:

\begin{quote}
    作为一个手机重度中毒的高中生,走路刷吃饭刷上课刷睡前刷,把所有的东西都刷完还要重复刷,有一段时间甚至觉得为什么大家的微信朋友圈更得这么慢,都不够我刷的,热门的剧也挨个轮了一遍后,居然开始觉得空虚起来。回到家以后,我惊恐地发现老妈为了防止我闲得长毛,竟然给我报了瑜伽班,我似乎听见了全身骨头咔咔作响的声音,瑟瑟发抖。但还是没拗过老妈,乖乖地去了瑜伽班,回来以后似乎忘了手机这回事,只想安慰我发抖的腿。于是我就愉快地开始了每天擦屋子、学做饭、练瑜伽的养老生活,闲的时候翻两页书跟老妈扯两句皮,陪老妈逛逛菜市场,偶尔和朋友出去联络个感情,每天都让自己忙起来,没手机的日子就没那么难熬了。对手机的感觉也从第一个星期的心痒难熬,变成第二个星期的偶尔自己闲下来的手痒,最后变成了想用某个软件时才会想到手机,而手机上面的灰尘,也让我认识到:原来真的没有人找我,我也没有那么重要,自抱自泣中。一个月后,我已经没有了看朋友圈的欲望,更重要的是,由于早睡早起练瑜伽,我的身体素质明显比以前一走三歪的状态好了太多,起码去超市的时候老妈可以把我当儿子用了。除此之外,还学习了自己喜欢吃的菜,当然由于和老妈沟通变多,我们已经很久没吵架了,似乎有了时间后,我的耐心也回来了。我不否认手机的便利性,但把手机摆在它该有的位置,我看到这个世界更多的东西,有更多的时间去陪身边的人,也更珍惜现在拥有的一切。
\end{quote}

报了瑜伽班,也算是培养兴趣爱好,转移注意力,接触手机的时间少了,在其他方面投入的时间多了,慢慢就开始摆脱手机瘾了,手机也回到了它本来就该处在的位置,而不应占据生活中太多的时间。首先要从内心重视起来,要真正意识到这些问题才会想办法去改掉它。当然放下手机是一件不太容易的事情,因为你已经习惯了,刚开始放下手机的时候容易烦躁不安,无所事事,这时可以到公园去散散步,或者运动锻炼一下,可以规划一下自己的生活,可以干一些其他的事情,这样就占去自己的空闲时间。

充实自己的生活,有空了打扫一下卫生,为父母分担点家务、看看有益身心健康的书籍等等,健身也是一个很不错的选择。发展一下自己的兴趣爱好,平时把时间花在自己的兴趣爱好上,你会发现生活增添了不少乐趣。打打球、弹琴唱歌什么的都可以,多和身边的朋友聊聊天,不要总是低头捧着手机封闭自己。总的来说,戒掉手机瘾需要一定的勇气和决心,意识到危害,就要下决心去改,并不是很难,多坚持一下就可以了。放下手机做一个自律而优秀的自己,加油!

三个控制:\begin{description}
    \item[控制手机使用时长] 控制手机使用时长,把更多的时间用来做其他的事情,不能让手机占据太多的时间。每次使用手机前,先设置时限,比如这次使用手机的时限是十分钟、二十分钟等。不设置时限,时间就会飞逝,回过神来,已经半小时、一个小时过去了。设置时限,你就会主动关注时间,把握时间。
    \item[控制内容] 控制浏览的内容,低俗的内容不要去看,擦边暴露的内容不要去看,一些不良应用要卸载掉,自己必须坚决一些,平时可以看看正能量的内容。
    \item[控制接触次数] 控制接触次数,你可以统计一下一天接触手机的次数,比如一天二十次、三十次,然后开始减少接触次数,不是特别必要就不要去碰手机,不要有事没事就捧着手机,让手机安静躺在那儿,到用时才拿起它。戒手机瘾需要自律,对自己的生活有更加细致的规划,也要及时反省和总结,一步步优化和加强控制。很多人早上醒来的第一件事,是打开新闻客户端,看看每日头条再起床;睡觉前,一定要进朋友圈浏览朋友们的信息,或者刷视频。这个习惯要改,早起和睡前尽量不要接触手机,早起接触手机容易延误正事,睡前接触手机容易造成熬夜。
\end{description}

一位手机瘾者的反省:

\begin{quote}\it
    说实话,我很早就想改变自己的这种状况了。对我来说,看手机有两大我不能忍受的坏处:第一就是浪费时间。一天清醒的时间假如是十个小时,而我要花费三、四个小时的时间在看手机上面,这是多么奢侈的一件事情。而精神层面的原因是,看手机让我焦躁不安,我总有一种在浪费生命的感觉。看完手机,心中满是空虚和自责。这种滋味最难受。第二个原因是我的眼睛受不了。前段时间看电视剧,加上看手机,还写文章,我的眼睛长时间疼得难受,睁开很困难,弱光情况下都很不舒服。出现这种情况的原因自己心知肚明,只能怪自己太毫无节制,看完觉得空虚无聊。我已经意识到了问题的严重性,所以剩下的事情就是如何改变的问题。第一点是每次拿手机之前,我都会告诉自己:哎,放下,放下,看手机只会浪费你的时间,看完你会后悔的,你的眼睛会疼,你会懊恼,你会暴躁。这一点是非常关键的,在潜意识中注入这样做的危害,那么你就会放弃这样的行为。给自己设定看手机的时间,这包括次数和每次时长。我还规定自己上班时间除了往来短信之外,看手机不能超过两次。一次在中间,一次在下班前半个小时,集中处理一上午积攒的需要浏览的东西。而且我也在尽量通过电话,而不是微信联系。
\end{quote}

谁把我们的时间偷走了?手机!这是现在非常普遍的情况,不仅偷走时间,还会造成生理的不适,心情的烦躁,浪费时间,浪费生命,更空虚更无聊。我们意识到手机瘾的危害就要做出一些改变和调整,有时不带手机,出门走走,顿感轻松很多,可以看看花草树木,发现以前忽略的美好。树立一个高尚的目标,这样你的心思才能转到正确的道路上,人没有了理想就会浑浑噩噩放任自己,毫无章法,如果你有正确的理想指引你,我相信你会自觉减少玩手机的时间,把更多的时间和精力投入在人生奋斗上。

\subsubsection{如何戒网络小说瘾}

家长一方面担心孩子过于沉迷网络小说影响学习,另一方面是担心孩子在网络小说中接触到一些不健康的东西,特别色情和暴力已经渗透在网络小说里,很多小说灌输的价值观人生观也是错误的,会对青少年造成误导。网络小说还会造成每日浮想联翩的情况,有的人把自己想得神通广大无所不能,靠打打杀杀成了万人敬仰的英雄,网络小说的很多情节都是非常自私的,也是非常荒唐的,与现实脱轨。长期接触这类小说,就像精神鸦片一样腐蚀自己的灵魂,很多网络小说,纯属为了赚钱而写的意淫式的东西,其内容大多都是都市艳遇、香车美女,男主角天下无敌,征服世界,征服黑道,不服的人直接秒杀,最终成为全世界的主宰,这种意淫式的网络小说充满了权利欲、色欲和名利欲。很多青少年受了误导,在生活中也想靠打打杀杀征服别人,充满暴力,没有德行,处于一种非常自私狭隘偏执猖狂的状态,甚至对家人也是暴力相向,因为网络小说灌输给他的价值观就是靠暴力征服别人,成为主宰,从此走上人生巅峰,收获香车美女,放纵享受,睥睨一切。这种价值观是非常错误的,也是非常愚昧无知的,会对青少年造成严重的不良影响。

若有以下几点:\begin{itemize}
    \item 每天看网络小说时间超过五个小时;
    \item 每天不看网络小说会浑身难受,全无精神,且容易暴躁发脾气;
    \item 经常通宵不睡看小说;
    \item 与父母交流减少,以前的兴趣都放下了;
    \item 每天看手机的时间持续增加。
\end{itemize} 符合以上几点即可定性为病态小说瘾,是应该要戒除的。有的人躺床上一动不动看五六个小时,早上起来第一件事就是拿起手机看小说有没有更新,晚上也不睡觉,熬夜看小说,甚至通宵看小说,一边看还一边幻想自己就是里面的主人公,充满各种荒唐的幻想。看网络小说已经占据了太多的时间,影响到正常的学习和生活,这时候就应该要引起充分重视了,一定要下决心戒除了。戒除方法:\begin{itemize}
    \item 充实生活,转移注意力。可以试着培养其它的爱好,不会总想着要看网络小说。
    \item 自我奖励。如果能坚持一天不看网络小说,就奖励自己一个小物件等。
    \item 请人监督。如果自己自制力不够,可以请家人或者朋友来监督。
    \item 多和人交流。平时要多和别人交流,多参加一些活动,让自己没有时间去看网络小说。
    \item 多读圣贤书籍,培养正确的人生观和价值观,种德立命,修身治世,培养正能量和德行。
\end{itemize}

\paragraph*{总结}

这季三大样都讲到了,戒游戏瘾、手机瘾、网络小说瘾,这三大样戒掉了,就可以节省太多的时间和精力,可以更好地规划和安排自己的生活,为人生的理想拼搏奋斗,活出正能量的人生。与其沉迷于虚拟的世界中,不如在现实世界中行善积德,这样才能真正完成逆袭,从此走上人生巅峰,不是靠打打杀杀走上人生巅峰,而是靠改过迁善、行善积德,这样一步步走上人生巅峰,那些成功人士都是具备一定善德善行之人,才能坐稳那个位子,如果德不配位,根本就坐不住的。童蒙养正,少年养志,成年养德,我们不需要在虚拟世界里找满足和成就感,那种满足和成就感是极其肤浅的,在现实生活中加强投入,好好孝顺父母,行善积德,这比什么都强,当你正能量起来了,生活的境遇和心态都会大幅改变,考上理想的学校,找到满意的工作,到时的满足感和成就感不比在虚拟世界里打打杀杀强百倍吗?沉迷游戏,和家人关系紧张,荒废学业,这不是走上人生巅峰,而是落入了人生的低谷,充满戾气和负能量,自己的精神和健康也遭到了严重损害,这不是我们真正想要的生活,有此类问题的人应该好好认真反省一下了。戒掉这三样,充实自己的生活,培养新的兴趣爱好,我们可以过得更有意义,活得更有价值!

一位戒友的破戒总结:“敷衍学习、沉迷游戏、不认真学习戒色文章、没有远离邪淫的内容、没有管控好自己情绪,最重要的就是放松警惕!警惕的螺丝真的不能松懈,一旦松懈便会被打回原形。”另外一位戒友的反馈:“我初中戒过 169 天,上高中以来沦为戒油子,很大一部分原因就是接触了某款游戏,我已经玩了接近 8000 局了,沉迷游戏让戒色的好状态一去不复返了,并且开始了连续的熬夜,并有多次通宵熬夜,成瘾程度令人发指。”很多戒友破戒的原因跟沉迷游戏有很大关系,游戏占据了太多的时间,沉迷游戏后就无心学习戒色文章了,所以戒色后也必须要戒除沉迷游戏,不能一玩就几小时,网游最好不要玩,不仅浪费时间,还浪费金钱,这种投入实在太不值得,还搭上自己的身心健康,严重影响戒色状态,一定要充分认识到这一点,狠下决心戒沉迷游戏。戒游戏是远离消极负能量的自己,回归积极向上的自己,游戏已经吞噬了太多的时间、精力和精神,一定要认识到游戏瘾的巨大危害。戒瘾四步走:认识危害、下定决心、充实生活、念头管理,一旦冒出想玩游戏的想法,马上思维危害,这就是思维对治,从念头上进行干预和控制。也要从根本上转变对游戏的认识和理解,游戏不是你生活的全部,生活中还有很多更重要的事情等待你去完成,你的精力和时间要用在真正有意义的事情上。

你觉得游戏好玩,所以才会不断去玩,如果你换一种思维,觉得它不过如此,打打杀杀也没啥意思,玩心自然就淡了。熬夜久坐也很伤身体,容易得上神经症,玩到最后一场空,迟早会厌倦的,还不如现在就收手,狠心卸载游戏,在卸载的一刹那,也许你会有一些不忍,毕竟投入了那么多,但是你要知道这是你重回正轨、走上人生巅峰的关键一步,因为你已经成瘾,所以必须要彻底戒掉,一定要拿出最大决心来做个了断!你是在拯救自己!刚开始不玩,是容易烦躁不安的,这也是戒断反应,充实生活,积极锻炼,培养其他兴趣爱好,转移注意力,坚持一段时间戒断反应就消失了,到时你会觉得,原来不玩游戏也挺好的,自己收获了其他方面的乐趣,到时你也会觉得,一天到晚坐在那玩游戏真的很可怜,多么宝贵的青春年华就被这样白白浪费了!多少的时间和精力就这样白白葬送了,实在太不值了!当你开始为理想奋斗,而他们还麻木不仁地坐在那打打杀杀,浪费时间浪费精力浪费钱财,这何止天壤之别啊!不要觉得玩玩游戏也没什么,一旦陷进去,玩几个小时、十几个小时,到时你就完全变了一个人了,生活中其他事情都不关心了,一心就只想着玩游戏,游戏瘾和色情瘾会一起发作,把你逼入人生的死角,到时你就只能瘫坐在角落里哀嚎绝望了,这是多么悲惨的场景啊!

一位资深戒友:“之前我也是一个网瘾青年,但是因为网络游戏极容易引发人的嗔心,而且也有许多擦边内容,所以我也戒网游三年了。直到如今,我都从未玩过一款手机游戏,因为我深知,玩游戏绝对是弊远远大于利的,也意味着久坐久视,不利于身体恢复。”这位资深戒友戒掉了游戏,他戒色方面也取得了很大进步。还有一位戒友说得很好:“看到身边的朋友一个个事业有成、生活美满、才艺出色、生活快乐的时候,你的内心一定会被触动,反观自己,除了虚拟的一身装备和游戏中虚拟的满足感,一无所有!无限的青春和时间,换来的是一无所有,还在沉迷游戏中的你,是否能看到我这个若干年后的你的样子?也许若干年后,你也会跟我一样坐在电脑前,打下这悔恨的文字,劝导那些少年,远离游戏。不过人真是很奇怪,过来人的例子从来不会吸取教训,而选择亲身尝试,到最后伤痕累累,才幡然醒悟,流下悔悟的泪水,可是青春不复,当皱纹爬上脸庞,镜子中再也不是当初那个狂傲无知的少年,也许你就成长了。”

很多戒友都有沉迷游戏这个问题,这个问题是要深入谈一谈的,要给大家一个深刻的告诫和警醒,《戒为良药》来到 \ref{134},我才深入谈这个问题,似乎有点晚了,应该前几年就开始谈这个问题,但我相信现在谈也不算太晚,现在正是我想法比较成熟的时候,所以能谈得比较深入一些。我之前看过新闻,有一位做生意的老板也沉迷网游,不仅花了很多钱,而且生意也懒得打理了,就是整天沉迷游戏之中,最后公司也倒闭了,还背上了一身的债务,本来他生意做得很好,有几百万的身家,后来沉迷游戏,人生陷入了低谷。沉迷游戏不仅对于青少年,乃至对于成年人都是一个很大的问题,60 后很少有玩网游的,而 70 后、80 后、90 后、00 后基本都是玩游戏长大的,所以更容易沉迷网游。这个问题要引起充分重视,我的建议就是网游不要去玩,太浪费时间精力和钱财,益智类和体育类的游戏可以偶尔玩一下,但也不可沉迷,每次玩的时间不要超过一个小时,多培养其他的兴趣爱好,让生活充实起来,多行善积德,培养自己的正能量,到时就会发现不玩游戏,照样可以过得很好,照样可以在其他方面收获满足感和成就感,而这一切远胜于在虚拟世界里打打杀杀,当别人走上人生巅峰时,而你还挂着两个黑眼圈在虚拟世界里熬夜奋战,就算在虚拟世界里练到最强,那又能怎样,关掉电脑和手机,屁都不是,还是一个可怜虫!最后很多玩家都恨死游戏了,因为游戏吞噬了他们的青春和健康,他们觉得是游戏害了他们。请记住:现实生活才真正值得你去奋斗和投入,不要在虚拟世界里虚掷美好的光阴了,戒除沉迷游戏,在生活中积极奋斗,行善积德,你的人生将从此走上辉煌灿烂的巅峰!

下面分享一首戒色诗歌。

\begin{poem}[戒色诗]
    \begin{multicols}{2}
        \centering~\\
        当初求种像条狗,如今撸完嫌人丑。 \\ 当初下片疯如魔,如今撸完忙删片。 \\ 当初着迷性趣浓,如今撸完没意思。 \\ 当初连撸不要命,如今撸完悔恨生。 \\ 当初看片似饿狼,如今撸完懒得动。 \\ 当初轻信无害论,如今撸完症状虐。 \\ 当初身体百强健,如今撸完废如渣。 \\ 当初纯真又无邪,如今猥琐且龌龊。 \\ 浑浑噩噩一撸蛆,精耗人废灾难重。 \\ 结缘医院常吃药,人财两空苦不堪。 \\ 脑力精力狂下降,学业事业难应付。 \\ 脾气暴躁人缘差,顶撞父母不孝顺。 \\ 鬼气缠身人厌恶,自照镜子也自卑。 \\ 症状爆发陷苦海,生不如死神经症。 \\ 看片撸管危害大,赶紧回头把自救。 \\ 删删下下何时了,不如狠心把色戒! \\ 待它百日筑基时,逍遥快乐笑颜开! \\ 哈哈哈哈哈哈哈,原来戒色这么爽!
    \end{multicols}
\end{poem}

下面推荐一本书。

\begin{book}[《庭训格言》,康熙皇帝]
    康熙一生兢兢业业,修身、齐家、平天下都十分认真,可谓耗尽心血和精力。康熙平时在宫中经常给皇子皇孙以教诲,雍正即位后对康熙的家训加以追述,并整理汇编成《庭训格言》,凡一卷,246 则。今日我们读《庭训格言》,依然会得到许多有益的教诲。

    \begin{quote}\it
        训曰:“当无事时,敬以自持;而有事时,即敬之以应事物;必谨终如始,慎修思永,习而安焉,自无废事。盖敬以存心,则心体湛然。居中,即如主人在家,自能整饬家务,此古人所谓敬以直内也。”
    \end{quote}

    感悟:康熙的比喻非常贴切,一个人如果能做到“敬以存心”,身心都会湛然澄澈。就好像一位精明的主人,能够将自己家中管理整饬得井井有条。而心如果不敬,就像一个家没有了主人,各种混乱就会随之而来。敬就是谨慎的态度、敬就是不懈怠的状态,人只有保持这种状态,才能不断进取,似有天助。

    \begin{quote}\it
        训曰:“《大学》《中庸》俱以慎独为训,是为圣第一要节。后人广其说,曰:不欺暗室。所谓暗室有二义焉:一在私居独处之时,一在心曲隐微之地。夫私居独处,则人不及见;心曲隐微,则人不及知。惟君子谓此时,指视必严也,战战栗栗,兢兢业业,不动而敬,不言而信,斯诚不愧于屋漏,而为正人也夫。”
    \end{quote}

    感悟:康熙认为,慎独是为圣贤的第一要节。他把慎独分成两个方面,一个是在自己独处的时候修慎独。另一个方面是说在我们内心的隐微之处,要修慎独。这两个地方都是只有我们自己能察觉而别人看不到的地方,所以一定要战战栗栗,兢兢业业,不动而敬,不言而信。《庭训格言》值得我们学习,有很多为人处世的道理,也谈到了修心的道理,是比较全面的一本书,不愧为皇帝的家训,含金量颇高,值得一读。
\end{book}
