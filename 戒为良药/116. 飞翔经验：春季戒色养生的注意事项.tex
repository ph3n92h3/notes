\subsection{春季戒色养生的注意事项}

前言:

这季前言谈下远离邪友的问题,前几天看到一个帖子,楼主被身边的朋友带去那种场所,这种事情应该很多人都会遇见,学生党可能少些,工作后这类事情很可能会遇见,身边的同事或者朋友有在嫖娼,然后就会向你吹嘘嫖娼的经历,还可能把你往那种场所带。也许他是出自好意,大家一起玩得开心,但是他却没有意识到这种行为的严重后果。大家都知道孟母三迁的故事,一个良好的环境对人的成长及品格的养成至关重要,近朱者赤,近墨者黑,除非你有很强大的定力,否则很可能会被拉下水。嫖娼是违法的,而且也很容易上瘾,弄不好还会染上性病,之前看过一个案例就是嫖出了病,本来要订婚了,结果查出梅毒,女方和家人果断怒了,然后好好的婚事就这样不了了之。一次喝酒后,他哭了,说得病后感觉世界一片灰暗,后悔不已。嫖娼的花费也不少,经常嫖娼不仅危害身心健康,而且最后也可能穷困潦倒、失意落魄,天道祸淫最速,事业人生直走下坡。很多人第一次嫖娼都是被邪友带去的,本来没有那种想法,喝完酒邪友就开始怂恿了,酒后很容易失去理智的判断,也容易把持不住自己,于是就跟着邪友去嫖娼了。喝完酒的人,一,容易听命于心魔;二,容易听命于邪友。这是喝完酒的特点。我们一定要注意远离邪友,发现身边朋友有在嫖娼,你可以试着劝一下,如果不听,你就要和他保持一定距离了,不需要闹翻,但一定要保持距离,不要和他去喝酒,不要走得太近。

下面摘自黎家明《最后的宣战》,一次嫖娼就得了艾滋病。

\begin{quote}\it
    那是一个平常的星期二,没有风和雨,一项繁重的工作结束后,我和我的同事出去吃饭,我们都喝多了一点酒,庆祝工作顺利完成。烈酒在我们年轻健壮的身体里狂野地奔腾,我的同事邪邪地对我笑道:“我带你去一个地方。”我问道:“什么地方呀?”他说:“你先说你敢不敢吧?”我说:“谁怕谁,我有什么不敢的!”我已经意识到他说的地方意味着什么。其实,在我清醒的时候,每每路过那些美容院、洗头房、桑拿浴室、酒吧,我都是那么不屑一顾。我鄙视那些隔着玻璃窗诱惑的眼神和肢体。从大学毕业就一直一个人在外地工作,严格的家教、父母的警告和信任,一直让我远离那些场所。但年轻酒后的我,那一天,迷失了……
\end{quote}

下面分享一些案例。

\begin{case}
    今天有个同学夸我长得帅,好久没有被这样夸了,自从开始手淫后,七年时间,刚开始一两年还行,到了后面脸几乎变了形,大小眼、痘痘、没精神……太久时间一照镜子就会失落,而现在戒色将近八个月,虽然还有些痘痘,但眼睛几乎一致,且有些灵气,虽然对相貌不再那么在意,但被女生夸后还是非常感谢戒色吧,感谢飞翔吧主,感谢陈大惠老师,感谢前辈们的经验,我会一直坚持戒下去的,《戒为良药》也会一直看下去,永远放在手机电子书里,看完一遍接着看,永驻戒色吧,懂多少就帮多少,同时宣传戒色吧,不管是网络还是现实生活,能帮多少就帮多少。没来戒色吧前,我真不敢想象可以不手淫生活,感觉对自己的控制越来越清晰,很微妙,很自在,快乐应该就是可以从控制自己的念头开始。

    \textbf{附评} 这位戒友坚持戒色近八个月,他逆袭了,容貌气质改善很多。“灵气”这两个字用得很好,纯净的孩子是富有灵气的,他们比成年人更接近本源,不像成年人已经忘记与纯净真我的神圣连接。灵气最主要的外在表现就是一双眼睛的清澈度和神采,有的孩子真的是灵气十足,心地是那么干净,几乎没有什么污染。进入发育期后,开始手淫泄精了,这时候灵气值就会大幅下降,眼睛会失去神采,会变得空洞无神,眼睛也会浮肿变形,出现不对称的表现,眼珠也会变得混浊。撸者常常有一种呆滞的神情,那就是能量被掏空后的表现,变得呆滞晦暗而缺乏灵气。戒色可以让人找回久违的美好与自信,戒色的确比手淫更快乐,只有与纯净真我再度连接,你才会真正快乐起来,这是发自内心的大快乐。反之,沉迷于看黄手淫的堕落生活,只会让自己越来越不快乐,甚至可以说是惶恐与痛苦,因为能量被不断耗损,最后导向的结果必然是负面的。撸者是用真正的大快乐去换那几秒短暂的快感,这是得不偿失的愚痴之举,智者已经完全认清了,所以他们不再盲目追求快感,他们知道真正的大快乐在哪。曾经,我也认为我不可能戒掉手淫了,因为那时我一直失败,心魔实在太强大,那时的我根本不是心魔的对手,简直不堪一击,就像小孩和大人在扳手腕,完全就是一边倒,没有任何胜算,后来我学会专业戒色后,逐渐变得强大起来,这时候才有战胜心魔的把握。这位戒友说得很好,当你可以控制自己的念头了,自然就会体验到快乐自由的感觉,真正的大快乐是本自具足的,是盲目追求快感让你变得不快乐。
\end{case}

\begin{case}
    飞翔您好,我在高二那年无意间从同学那接触手淫,然后一发不可收拾,现在我三十岁,在当医生,因为手淫我开错药,吃了医疗官司,即将被开除,家里人十分伤心。翻看高中毕业照,我照片上一脸死气,其它同学洋溢着青春的气息,我就像刚从棺材里爬出来。我接触戒色吧以来,一直想戒色,可是总是破戒,我感觉人生毫无希望,多次想自杀,我很痛苦,非常无助,没有人能够倾诉!

    \textbf{附评} 《The Porn Trap》里讲道:“色情给人虚幻的快感,前一刻它让人飘飘欲仙,后一刻它就牢牢地将人拖下无底深渊。”还说道:“色情的另一面是毁灭。我丢了饭碗,还差点跟老婆离婚。如果你一直看色情,那么总有一天,色情会毁了你全部的生活。我认为,现在很多人都还没有意识到色情的毁灭性。”国外对色情和手淫的危害研究得很深入,国外虽然也有无害论,但近些年来关于色情与手淫危害的书籍变得越来越多,人们已经逐渐意识到色情与手淫的巨大危害了,国外的网上书店有很多关于戒除色情和手淫的专业书籍,而我国这方面的书籍还很少,希望将来能够有专家教授开始研究色情和手淫的危害,到时无害论将会被彻底淘汰。如果是高校或者某研究机构的教授来写书论述色情与手淫的危害,这无疑更具有说服力,从科学角度来论述危害更容易让人接受。这位戒友高二接触手淫,现在三十岁,手淫影响了他的脑力,影响了他的工作状态,也影响了他的容貌气质,他的生活几乎被色情与手淫恶习给毁掉了。撸到一定程度的确会有一种行尸走肉的感觉,“就像刚从棺材里爬出来”,那种灰暗颓废无神的感觉实在太糟糕了。另外一位戒友说:“色情就是毒品,邪淫就是在吸毒。这十五年的邪淫史,夺走了我太多太多,健康、学业、事业,夺走了心底的纯净与快乐,除了邪淫时那短暂的饮鸩止渴般的满足,几乎无时无刻不是活在阴霾下、痛苦中。”快感是短暂的,快感带来的满足会引发更多的不满足与焦渴,如饮咸水,多饮多渴,欲壑难填,肾精有限。要戒掉手淫恶习,必须多学习戒色文章提高觉悟,专业戒色非常强调学习,当你的觉悟和断念水平提升了,到时自然可以降伏心魔,否则就会被心魔拖入生不如死的活地狱!
\end{case}

\begin{case}
    两年前买个飞机杯,如今在医院发帖!2016 年去中医院检查身体,医生说我三十岁的人,六十岁的身体,已经精血耗尽,肾精枯竭,几近死亡,如果再不戒,只有五年活头儿了。我不信这个邪,回家又撸了一年,前些日子撸的时候身体痉挛,小腹剧烈疼痛,昏死过去……姐姐拨打了 120 把我送到医院抢救,才捡回一条命,手淫造成的濒临猝死,情况很危急,还好抢救及时。如今我躺在病床上用笔记本告诉大家,千万不要再拿自己的生命开玩笑了,珍重身体,拒绝手淫。

    \textbf{附评} 这个案例让我想到了四个字:以身试法!有点愣头青的感觉。医生通过脉象知道他的身体已经严重衰败了,三十岁的人,六十岁的身体,身体已经伤不起了,但他还是“不信这个邪”,一意孤行,继续沉迷手淫恶习,真的是不撞南墙不回头,不见棺材不掉泪,撸到濒临猝死,才幡然醒悟,还好被姐姐及时发现,否则小命不保。很多人都买过邪淫的玩具,那些东西只会掏空你的身体,到时身体就危如累卵、危机四伏了。身体好的时候不觉得,症状一旦爆发,那就相当痛苦了,年轻时就要懂得保精养生,千万不可滥撸滥泄,随意糟蹋自己宝贵的肾精能量。疯狂追求快感必然会导致悲剧的结果,撸出濒死感的案例数不胜数,在中国,每年死于猝死的人数多达 55 万,其中,中青年人猝死的比例在不断上升,平均每天一千多人猝死,这一千多人中肯定有不少撸者。之前有个新闻,就是监控拍到上班期间看黄手淫,然后猝死的整个过程,手淫后真的可能会猝死,有的动物交配后就挂了,这种事情是非常突然的,让人猝不及防,突然把人抛入生死的边缘,多么恐怖!死于看黄手淫,这种死法真的是太悲哀了!手淫会把生命力射掉,严重透支就会导致猝死,也许刚才还撸得起劲,射完没多久身体就开始抽搐痉挛,倒地不省人事,慢慢变成一具冰冷的尸体,这种死法太丑陋,也太猥琐!刚开始很多人都是抱着“偶尔撸一次没事”的想法,他们根本没意识手淫的高度成瘾性,总以为自己能控制,到最后越来越失控,心魔变得越来越强,这时候就骑虎难下了。心魔就像怪兽一样,一次次喂养,最后就会变得越来越变态,贪心越来越强,然而身体根本吃不消这种疯狂的耗泄,过了某个临界点,症状肯定会大爆发,撸得狠了,双手一摊,两脚一伸,去见阎王了……
\end{case}

\begin{case}
    今晚奶奶又问我女朋友的问题,虽然我回答很平静,但是我心里在滴血,我已经被邪淫给弄得人不像人鬼不像鬼了。不停地遗精,我身体里面的东西全都被抽干了,哪还有精力找女朋友?哎,弄成今天这样的结局,以后那么长时间,真不知道怎么面对家人。虽然是公务员,别人看来很体面,可是我却经历了如此艰难的生活,没有人会理解。在里面干活,因为精神不振,经常被骂,自己有苦也只能往肚里咽。都怪年少时那个无知无耻的自己,染上这样的恶习,还沾沾自喜,却不知道走了一条不归路。

    \textbf{附评} 能成为体面的公务员也很不容易,但是邪淫对他已经造成了很大的负面影响,还出现了频遗的症状,身体的精华不断漏失,不管是脑力还是精力、体能等都会大受影响,身体废了,家庭事业都会遭遇很多挫折。到了年纪家人就开始催婚了,但是身体却因为邪淫而废掉,这种事情也很难和家人说出口,只能找其他借口来推脱或者暂时答应下来。考上公务员了,生活应该不错,但是却经历了如此艰难的生活,的确让人难以理解,只有亲身经历后,才知道沉迷色情与手淫对一个人的生活和健康会造成怎样的负面影响,随着伤精史的延长,这种负面的恶果就会变得越来越明显。一个健康而充满精力的身体对于一个人而言是异常重要的,身体废了,怎么谈女朋友?精神萎靡,在单位肯定会被领导经常骂,成为领导的出气筒。邪淫会让一个人的生活和事业出现各种的不顺,这是千真万确的事实,最近我和一个朋友聊天,他过去就经常邪淫,现在结婚后经常和老婆吵架,闹离婚,搞得他焦头烂额,烦恼得不行,邪淫会感召不如意的眷属,结婚后会出现种种的不和谐。万恶淫为首,这个淫指的就是邪淫,邪淫就是在积累负能量,邪淫后会产生各种负面的念头,导致自己的气场很不好,就像一块磁铁一样吸引负面的事物,要改变这种糟糕的处境,那就必须狠下决心戒除邪淫的行为,好好忏悔,多行善积德,这样境随心转,慢慢就能改变命运。我们要不断壮大心中的浩然正气,从废铜烂铁化身为钢铁侠,无坚不摧,威武雄壮!让戒色点亮你的生活,让灵魂沐浴在纯粹的美好与喜悦中。
\end{case}

\begin{case}
    飞翔叔,我是一名大一学生,戒了一个月,今天破戒了非常后悔。昨天评上了一等奖学金,沉浸在成就感和大家的赞美声中,有点骄傲了,然后今天念头来的时候我就没及时断掉,脑子里有个声音:“没事的,就当犒劳自己吧!”然后就放纵自己 YY,今天下午正好室友不在,就看黄破戒了,现在我好后悔,也好害怕明天心魔还会找我麻烦,飞翔叔我该怎么办?

    \textbf{附评} 看了这个破戒的案例,我的总结就是:德行有亏,戒色必败!戒色最忌讳的就是骄傲,曾国藩说过:“天下古今之才人,皆以一傲字致败。”人在骄傲的时候很容易放松警惕,结果被心魔钻了空子,所谓骄兵必败!当处于一种狂欢的情绪中就要格外警惕了,因为狂欢会滋生放纵,这是经验之谈,心魔会给出针对性的怂恿——用看黄手淫来犒劳自己!一旦认同这个怂恿,就会开始放纵,心魔的确很狡猾很阴险,一定要学会识破心魔的套路,自己也要注意提升德行,一定要好好培养自己的谦德,不管出现什么情况,都不可骄傲自满,一定要谦虚谨慎,戒骄戒躁。我有时看帖,有的帖子的题目就有骄傲的成分,这类戒友肯定戒不长,骄傲是一个人很大的弱点,藏族有句谚语:“在骄傲的山上存不住功德的泉水。”戒到最后就是德行的比拼,德行不够,肯定会败下阵来,在戒色的同时应该多学习圣贤教育,慢慢完善和提升自己的德行,德行就像地基一样,楼造得越高,对地基的要求就越高,地基必须足够稳固,厚德才能载物。我非常注重谦德,戒色的前辈们都很注重谦德的培养,如果想进入更稳定的戒色层次,必须努力提升自己的德行。《周公诫子》云:“夫此六者,皆谦德也。夫贵为天子,富有四海,由此德也。不谦而失天下,亡其身者,桀、纣是也。”易曰:“天道亏盈而益谦。”满招损,谦受益。《了凡四训》里有四个字我很喜欢,那就是——种德立命!谦虚有道德的人才能立得住!才能成为真正的常青树!反之,有点成就就骄傲自满,这类人很难有大发展,一个骄傲自满的人也容易招致别人的攻击和厌恶。《了凡四训》最后一篇是“谦德之效”,袁了凡先生列举了丁敬宇“惟谦受福”、冯开之“虚己敛容”、夏建所“谦光逼人”、张畏岩“折节自持”等身边友人力行谦德而获得功名的实例,阐释谦德对于人生成就的神奇效验。每一则都是折节谦恭的典范,很值得今人效法。戒色的确是一门很深的学问,要达到更高深的造诣,必须多学习圣贤教育,圣贤教育有几千年的智慧在里面,专业戒色的最高层次是与圣贤教育完美对接的。当你的德行提升了,不管对于戒色,还是对于为人处世,抑或是将来事业的发展,都很有帮助。圣贤的开示平时应该多读、多思,反复领悟和消化,最终落实到行动上来。
\end{case}

\begin{case}
    三月八日前我一直破戒的原因就是因为没有学习戒色文章来持续提高戒色觉悟,另外在那之前也一直没有弄明白戒色最核心的部分是控念,经常把念头当作自己,敌我不分。通过三月八日的破戒,我明白了戒色最为核心的部分就是念头实战。我重拾信心告诉自己这次绝对不能破戒,但最终还是在三月十七日败下阵来,睡午觉时躺在床上玩手机,玩着玩着,脑海中不知不觉中就有了想要看看擦边新闻的想法,于是看了擦边新闻,接着就是图片、文字,最后到色情网站浏览视频,当时脑海中有一个声音,那就是:“看看没事,不撸就好。”但是看着看着我就开始撸了,直到撸到一半我才发现我正在撸,于是猛然惊醒,从床上起来,那之间的过程(观看的时间)大概持续了两个多小时,起床后在室内做了会运动,欲望才消退,结果晚上睡觉前,因为上床比较早,七点半就上床了,心魔又来了怂恿我说:“反正才七点半,你也睡不着,不如找点刺激的内容看看。”说实话那种感觉很微妙,属于非常细微的念头,我当时几乎没有察觉到,我那时已经有一种很微妙的不详的感觉了,但是可能是出于自己认为看看也没事吧,于是听信了这个念头,先是在网上看了些擦边视频,最后看到色情网站去了,看着看着开始撸了,然后就破戒了,破戒后一直睡不着,脑海中 YY 变得非常活跃,怎么断也断不掉,结果又连破两次。观看色情影片的时间大概持续了五个小时。最后睡觉时都已经凌晨十二点了。

    \textbf{附评} 很多戒友对断念的理论还是有一定领悟的,但在实战中却做不到,一次次被心魔攻破。对于心魔的怂恿没有识别,有的戒友虽能识别,但还是听信了心魔的怂恿,在实战中没有做出正确的选择。断念平时就要不断练习,不断地磨刀,工欲善其事必先利其器,每天都要练习观心断念,时刻保持警惕。黄念祖老居士开示过:“日日防盗,夜夜防贼。”这八个字说得实在太好了,心魔贼就在自己的心里,不要让其得逞!必须时刻防着心魔!一定要学会识别心魔的怂恿,然后一概不听,必须狠一点!心魔尽管怂恿,如果你不听,它也没办法!问题就是你听了,你认同了怂恿的念头,结果就会鬼使神差般破戒!脑海中的战斗真的很微妙,大德对于念头的入侵用过很经典的四个字来描述,那就是:不请自来!当你独处时、无聊时,那种念头就很容易冒出来,真的是不请自来,邪念就是木马病毒,必须立刻杀灭,就像杀毒软件杀灭病毒一样!防火墙和杀毒软件必须时刻开启,以防心魔黑客的不时攻击!这位戒友一次次被心魔攻破,他把自己的心理过程描述得很详细,相信很多戒友破戒前也是这个套路。首先,用手机时绝对不要浏览擦边新闻,不要没事就胡乱浏览新闻,现在很多新闻都有诱惑图片,手机上更是铺天盖地,比比皆是。我现在已经不用手机浏览新闻了,因为只要打开,肯定会有诱惑的内容,几乎是百分百的概率。其次,平时要减少玩手机的时间,一个人躺在床上玩手机,这时候很容易放松警惕和戒备。在戒色实战中的确有很多细节要格外留心和注意,很多人都是败在细节上,戒色高手在独处时都是很警惕的,也会严格限制玩手机的时间,当心魔怂恿时,他们能够立刻识别并断除,绝对不听信怂恿的念头。在破戒后一定要认真总结和反省,为什么心魔会屡次得逞?破戒前的心理过程是怎样的?自己一定要深入分析和研究。曾经,心魔对我的怂恿每次都成功,我被心魔虐得体无完肤、遍体鳞伤,被心魔耍得团团转,简直弱爆了!后来我补强了觉悟,强化了断念,心魔的怂恿再也没有得逞过,心魔可以在一小时内攻击你几十次乃至上百次,甚至几小时不间断地攻击你,为的就是把你拿下,把你变成撸管肉机,如果你能强悍地断念,不听信、不跟从,那心魔的攻击每次都会失败!它根本无法得逞!你必须足够警觉,足够强悍!拒绝被心魔攻陷!!!
\end{case}

\begin{case}
    一直坚持戒色,但是一直没戒掉。昨晚破戒了一次,两点多才睡,早上起来早餐不吃又窝在被窝撸了一次,还是强行的。那时候肚子已经很饿了,什么都不吃还是在撸,刚才洗了澡出去吃饭,感觉整个人轻飘飘的,手脚无力,吃饭的时候吃了几口,瞬间感觉想吐,整个人都没力气,全身出汗,脸色都白了,那时候一直在大口吸气,额头一直不断地冒汗。全身发冷,双手抖得非常厉害,很麻痹,就感觉那时候快要死了,觉得透不过气,心非常慌,心跳极速加快,甚至想上厕所的冲动,一直想吐又吐不出来,我艰难地在那熬了几分钟,我不敢动,不敢说话,闭着双眼念阿弥陀佛,我真的感觉很无助,我觉得我快撑不住了,过了几分钟后,感觉舒服点了,我艰难地买了单,一步一步走回我的宿舍,我觉得腰都挺不起来,整个人快死了的感觉,那时候我真的怕了,我发愿以后都不敢撸了,这应该就是戒友说的濒死感,真的太可怕了!如果那种症状再持续几分钟,估计我就倒在地上了。现在躺在被窝里,手脚还是冰凉的,想吐,心还是很慌,发这个帖真的让各位戒友赶紧回头是岸,别让色带走你的生命。阿弥陀佛,我要休息了,我真的怕我睡了醒不过来!

    \textbf{附评} 这又是一个撸出濒死感的案例,熬夜撸管,早上饿着又强撸一发,结果恐怖的濒死感就来了……心魔附体后,一个人可以连续看几小时的黄甚至看一个通宵,真的像着魔一样,以前我也有过类似的经历,不断地下片,时间不知不觉就流失了,我发现找黄看黄的时间是过得最快的,沉迷在色情信息里似乎根本感觉不到时间的概念,一转眼,时针已经指向了凌晨,而我还是不死心,还在疯狂找片,已经到了丧心病狂、穷凶极恶的地步了,两只眼睛似乎都在冒着绿光,两只眼球似乎都要贴到电脑屏幕上去了。那种状态就是不射出绝对不甘心,一定要射出才爽,就像完成任务一样,很多次在完成“射精任务”后,我都感觉自己形销骨立,快要形神俱灭了,一照镜子,气色像鬼一样,从找黄看黄到撸完,就像跑完一场马拉松一样,让人精疲力尽。有时射完一次还不死心,过一会还想来第二次,一定要彻底掏空才肯罢手,在那种状态下把一切都抛之脑后了,达到了废寝忘食的程度,只想撸撸撸,射射射,贪求更多的快感,那种贪心是非常强的。这位戒友也算捡回一条命,关键时刻他念佛了,也许是佛菩萨的加持才让他慢慢缓过来,劫后余生他真的醒悟了,真的不能再撸了,撸管恶习就是生命的毒瘤,这个毒瘤不除迟早会爆发严重的症状,弄不好就一命呜呼了。父母好不容易把孩子养大,还等着你尽孝和养老送终,千万不要把自己撸没了,到时就会上演白发人送黑发人的人间悲剧。人来到这个世界上,应该干点有意义的事情,给天地增加正能量,多行善积德,做一个负责任的好男人,而不是一个沉迷于色情和手淫的猥琐男。国外戒色文章里讲道:“如同其它成瘾物质一样,黄片会让大脑充满多巴胺。多巴胺一次又一次地释放,不断编织大脑中的奖赏回路,最终改变了看黄者的大脑组成。对,你没看错,色情会引起大脑实质上的变化。”“一个人看的黄片越多,大脑损伤越严重,也越难从中解脱出来。但也有一些好消息:神经元受到的损伤是可逆的。也就是说,如果我们上瘾者能够戒除这些不良行为,大脑受到的损伤又能恢复。”黄片就是一种毒品,看黄手淫就是在吸毒,这是一种高度成瘾的恶习,必须要下最大决心来戒掉,要有壮士断腕的勇气、破釜沉舟的决心!再也不看黄,再也不手淫!再也不!!!
\end{case}

\begin{case}
    飞翔老师,您好!我因国庆节那天破了戒,戒色状态消失了,一直到现在都没有找回,到现在为止破戒三次。请问怎样找回以前的戒色状态?另外,手淫害我不浅啊,前几天工作差点出了事故(工作的时候分神),差点生命就结束了,若不是在最后一刻同事喊一下我,我就不在人世了,这百分百是手淫导致的,手淫伤脑,注意力、记忆力、理解力等等都大幅度下降,严重影响工作,总犯低级错误……手淫害我真的太多太深了,说也说不完。

    \textbf{附评} 手淫后容易出事故,这是真的,大家看看撸者的神情就知道了,大多都有点呆滞和恍惚,双眼空洞而无神,有一种被掏空抽干的感觉,以这种状态去做事,很容易出事故,之前就有撸完第二天开车,因为注意力不集中而出车祸的,把人给撞了,赔了很多钱。有的工种是要求注意力高度集中的,如果不集中,也许手就绞进去了,或者出现别的更严重的事故,这都有可能。看过球赛的人都知道,有些球员在场上的表现可以用“梦游”来形容,就是完全不在状态,表现极差。大家想象一下,如果用梦游的状态去做高危的工作,那出事故的概率就很高了,一不小心就可能变成残疾。之前看过一个新闻,一个小伙在往机器里放东西,自己没注意就把手绞进去了,右手截掉了,变成了残疾人,这个小伙是否手淫无从知晓,但他出的这个事故很值得我们警醒,很多工作必须要注意力集中,稍微不集中就可能酿成大祸。一个良好的精神状态和脑力状态实在太重要了,而手淫恰恰会让人的注意力、记忆力、理解力下降,手还会发抖,会严重影响一个人的工作状态,脑子不行了,就会经常犯低级错误,被领导骂,甚至还会被炒鱿鱼。上面那个医生戒友,因为手淫开错药,吃了医疗官司,即将被开除,开错药属于非常低级的错误,手淫让人心不在焉、注意力涣散,在工作中就容易犯低级错误。一个手淫恶习就可以让人丢掉饭碗,乍一听,好像不大可能,但事实就是如此,中医讲肾上通于脑,手淫是会导致脑力下降的,脑力不行了,一切事情都难以做好,如果出了严重的事故,那真的后悔终生。破戒后一定要认真反省和总结,加强学习补强觉悟,必须注重断念实战,平时要好好练习断念。心魔很狡猾,必须吃透心魔的套路,不要上心魔的当!在破戒后戒色状态很可能会严重下滑,在短期内很可能被心魔连续攻破,这时候要及时调整状态,稳住军心,每天坚持学习戒色文章不要间断,养成良好的学习习惯,坚持下去,慢慢就能重新步入正轨,到时回来的就是一个加强版的你,只有强者才不会破戒,必须强过心魔!
\end{case}

\begin{case}
    本人 95 年的,我不抽烟,不喝酒,上大学后,就很少熬夜。我是从初中时候,一次同学带我去网吧通宵的时候,看到了别人放的色情电影,从此就学会了撸管,一发不可收拾。到现在有九年左右的历史,在 2016 年下半年,我在读大三,因为课少,撸管和看黄的次数就增多了,我也强戒过,戒到一个月左右,心魔一来,就马上破戒。直到大家开始去实习时,我就感觉自己的身体已经不行了,那时,就是特想回到爸爸的身边,可能是当一个人的阳气和能量极度缺乏的时候,就会希望在自己亲人身边,希望亲人能给自己一点能量吧。这也符合当一个人快要死的时候,会特别希望亲人在身边吧,当我回到家里和父母在一起时,我没有什么事,就是没精力。因为要吃酒的缘故,我去了一趟贵州,离开了父母,结果回来就发热到 40.1 度,输了一天的液,感觉好多了,没听父母的话让我再输一天,只是吃药,结果就复发了,又输了一天,就在那天心魔来了,我就看黄撸了一共三次,过后我就发现我喝水是苦的,当时还以为是心里发热,结果吃药和输液都没效果,水是越喝越多,身体一天比一天瘦,然后马上去医院检查,一测血糖,三十点几,正常人的空腹血糖是不超过 7 的。我已经是糖尿病酮症酸中毒了,虽然说撸管不一定会直接得糖尿病,但是我得糖尿病肯定和撸管有着极大的关系,因为身体已经虚弱得不行了,可怜我的爸爸妈妈,爸爸知道我生这个病,一晚上头发两侧白了,胡子也白了,我真是痛心,我住院时候每天测七次血糖,饭前、饭后两小时根据血糖高低来打胰岛素,打三次针,还要打一针输液,睡觉前还要测一次血糖,血糖高了还要打一针输液,每天就是看白天看晚上,我共住了十一天的院,也就是说我一共被戳了上百个洞了,在住院时我才深入地去看戒色吧,才看到飞翔老师的《戒为良药》,后悔啊!要是早点看到,我就不会得这个病。看看糖尿病并发症,我现在很害怕,也很迷茫,我才二十多岁!

    \textbf{附评} 这位戒友是含着泪写下的帖子,95 年生人,还是那么年轻,但已经患上了糖尿病,他犯了一个很大的忌讳,那就是在发高烧期间看黄撸了三次!身体还没好透,尚处于很虚弱的状态,这时心魔来了,结果疯狂破戒,糖尿病爆发了。糖尿病对人类健康产生了重大影响,糖尿病的并发症可导致心血管病变、肾功能衰竭、失明、截肢等。据世卫组织统计,中国二型糖尿病发病率在过去数十年中呈“爆炸式”增长,1980 年只有不到 5\% 的中国男性患有糖尿病,而目前这一比例超过了 10\%。中国人口世界第一,糖尿病患者人数也居全球首位。数据显示,2015 年中国糖尿病患者人数为 1.09 亿人。上季文章里也有一个糖尿病的案例,但那个戒友四十岁了,而这位戒友才二十多岁,中医专门讲到纵欲会导致糖尿病,彭鑫博士讲过:“糖尿病还有一个名词叫做消渴。消就是消失的消,渴呢,就是口渴的渴。古人把消渴分为三种。一种是上消,一种是中消,一种是下消。中消就是指的,饮食不节所导致的中焦脾胃的这种疾病,会导致糖尿病的发生。那么下消呢,就是指的房事不节、纵欲,所导致的糖尿病。而下消的这种糖尿病呢,在现代的临床中尤其多见。大家注意啊,不是只是因为饮食不节,所导致的这个糖尿病发病率很高。现在有很多这种因为纵欲过度所导致的糖尿病也层出不穷。”在身体虚弱时是特别忌讳泄精的,而这位戒友看黄撸了三次,身体那么虚弱了,还拼命手淫,身体就是这样被彻底撸垮的!自己身体不行了,还连累到家人,家人也跟着愁肠百结、担惊受怕,如果之前就戒掉手淫恶习,也就不会遭这个罪了。这位戒友不抽烟,不喝酒,也很少熬夜,但就是有看黄手淫的恶习,这个恶习把身体掏得太狠了,让他年纪轻轻就患上了糖尿病。在被心魔附体后,那真的不是人,比畜生还畜生,比禽兽还禽兽,就是一个劲地撸,把最后一滴精都要掏出来,心魔就像一个抢劫犯,极其蛮横地掠夺肾精资源。对付心魔,必须够狠!心魔入侵,一记觉察犹如后手重拳直接将心魔撂倒,直接把心魔 KO!要有这种断念的狠劲与魄力!六祖大师说过:“邪念之时,魔在舍;正念之时,佛在堂。”邪念上来时,就看你的实战表现了。戒色烈丈夫,眉宇间皱起一股刚烈的正气,眸光如刀,祭出最强悍的杀招,把心魔打得满地找牙!必须降伏自己的心魔,拒绝再被心魔奴役,只有战胜了心魔,才能真正把握自己的人生和命运。
\end{case}

下面进入正文。

之前写过夏季和冬季戒色篇,这季写下春季戒色篇,每个季节戒色养生的要点都有所不同,虽然戒色的根本核心是修心,但在戒色的具体过程中有很多需要注意的细节,每个季节的特点存在一定的差异,了解和把握这些特点,及时做出调整,这样才能戒得比较稳定。

一年之计在于春,春季是一年中的第一季,也是非常重要的一个季节,进入春季后不少戒友破戒了,春季天气转暖,日照时间延长,人的生理会随着季节变化而产生波动,同样人的心理也会随之而发生变化。在乍暖还寒之际,变化多端的气候常使人难以适应,会出现情绪不稳、多梦、亢奋、精力难以集中等症状表现,春季很容易发生心理疾病,正因为春季容易发生心理失调,所以春季是情绪破戒的高峰,进入春季后心情变得烦躁易怒,莫名其妙发脾气,这时候自己一定要加强情绪管理,在情绪差时,人很容易失控,心魔也会乘虚而入,对你疯狂怂恿!日照时间长了,多晒会太阳,晚上就容易勃起,也容易冲动,这是季节转换带来的变化,自己一定要多注意,在季节转换时要加强修心,保持警惕,注意心态和情绪的调整,不要让负面的情绪主导自己。

进入春季很多人都会吃韭菜,韭菜又名壮阳草,前几天一位戒友吃了韭菜后,邪念猛烈进攻,就像给心魔打了兴奋剂一样,结果欲火中烧,破戒了。记得前几年我也吃过一次韭菜,吃完没多久,邪念就开始进攻了,比之前要活跃很多,特别是一坐下来,身体一放松,邪念更是接二连三地冒出来,于是我赶紧站起来,恢复到比较警觉的身体状态,这样才战胜了邪念。韭菜属于五辛之一,佛经有记载,熟食者发淫,生啖者增恚。我们这个淫欲心重跟吃荤和吃五辛是有很大关系的,吃完后会增加人的欲念,让人很容易破戒。戒色后应该减少吃肉,以素食为主,能够吃全素也很好,五辛尽量不要去吃,否则会给戒色带来很大的难度,吃完那些东西,你就要面临一场念头的硬仗甚至是恶仗了,心魔会疯狂进攻,如果不是断念高手,那就很难招架住,对于一般的戒友而言,根本顶不住心魔的狂轰乱炸,很快就会被心魔攻破。

戒色后,饮食方面我们要格外注意,能引起性欲的食物尽量少吃或者别吃,饮食控制好了,戒色就会顺利很多。关于饮食的告诫,前辈的文章里多有提及,这方面要引起我们足够的重视。前段时间一位戒了两年的戒友,喝酒破戒了,而且还是嫖娼,酒后很容易乱性,戒色后,酒最好也能戒掉,实在做不到,也应该尽量少喝为妙,否则喝完酒定力会下降很多,这时候心魔就容易得逞,把人变成一个疯狂纵欲的禽兽!非常可怕!那位喝酒破戒的戒友极度后悔,本来他戒得不错的,但就是在喝酒这件事上不够谨慎,所谓一着不慎,满盘皆输。戒色从某种程度来讲,就像高空走钢丝,稍不注意就会掉下去,有的人好不容易戒了几百天,就因为自己不够谨慎和小心,最终被心魔攻陷,再次做了心魔的俘虏,感觉有点亡国奴的意味。心魔做了主,还有你好日子过么?不仅把身体掏空,而且还造下了邪淫重罪,将来果报惨烈啊!

进入春季,随着温度的升高,会出现无精打采、困倦乏力、精神不振等“春困”的表现,出现这种现象时会影响到自己的戒色状态,感觉看不进戒色文章,看一会就感觉很累,总是想睡觉,这时候要及时做出调整,自己要注意休养,适量锻炼,饮食保持清淡,坚持一段时间即可恢复正常,一般通过建立良好的运动习惯和生活习惯,春困的问题就会大大缓解。人的精力状态是有起伏的,自己要懂得管理自己的精力,季节转换时很容易出现不适的症状,自己一定要注意休养和调整。有的戒友进入春天后就感觉打不起精神,身心都很累,困乏无力的表现很明显,这时候一定要及时调整,否则就会影响到自己的戒色状态,也容易被心魔攻破。季节转换时,心魔很容易得逞,原因就是季节转换时会出现不稳定的现象,情绪不稳定,心态不稳定,身体症状也容易反复,戒色状态会出现短暂的下滑,如果自己不注意调整,那就会被心魔攻破。

下过棋的人都知道,如果自己在关键时刻出现了明显的破绽,并且被对方抓住,那局势就会急转直下,一下就进入了难以挽回的败局。戒色的每一天都要很小心,不要离开戒色文章,每天看看戒色文章和戒色笔记,这样可以很好地保持戒色状态。有过亲身体会的戒友都知道良好的戒色状态是多么宝贵,也许刚开始还能保持不错的戒色状态,然而一旦出现破戒了,良好的戒色状态就可能完全消失了,要花很久的时间才能找回来。之前一位资深戒友本来戒得挺好,戒色状态非常不错,后来不慎破戒了,到现在一年多的时间都没找回良好的戒色状态,又重新变得充满戾气和负能量,良好的戒色状态是很容易失去的,自己一定要潜心学习戒色文章,不可中断,并且要加强德行的培养,这样才能越戒越稳固。我见过不少戒友就因为某个环节没做好,然后突然就被心魔攻破了,就像咸鱼翻身后,再次被心魔翻了回去,又苟活在心魔残暴的统治之下。

在与心魔无数次的交手过程中,我深感心魔的狡猾与强横,它会专攻你的弱点,而且还很会挑时机,你必须在念头实战中一次次去体会和总结,彻底摸清心魔的套路和出现的规律。虽然心魔神出鬼没,但它的出现还是有一些规律可循的,比如在独处时它很容易向你发动攻击,很多学生党平时上学没事,一到周末就不行了,周末独处时间增多,心魔就会大举进犯!戒色说到底,就是一场念头的战争,我以前的戒色文章也专门讲到过,这场战争看不见硝烟,一个人坐在那儿似乎什么也没有发生,但他的内心正在交战,战斗也许就在眨眼间,那个念头断不掉,接下去就是看黄手淫,那个念头就是一个指令、一个木马病毒,当杀毒软件拦不住,木马病毒就会为所欲为肆意破坏。南怀瑾先生讲过“心兵难防”,心魔对你用兵,这就是一场脑海中的战斗,一次次被心魔虐,甚至被虐了十几年,当接触到戒色文章,才知道问题出在哪,一场史诗级的戒色蜕变之旅才真正拉开序幕!

一位戒友说:“戒色的每一天才是真正的大快乐,那种快乐难以言表。以戒色为乐,以戒色为荣,破戒才是最大的痛苦,短暂的罪恶快感,换来空虚、悔恨、症状缠身、无地自容。”他说得非常好,戒色的大快乐是那么微妙,真正的智者不再追求快感,他们享受的是纯净美好的感觉,享受的是内心泉涌般的大快乐、大喜悦,像孩子一样开心快乐,活在纯净的奇迹里,而不是活在邪淫的地狱里。春季很多人都会去踏青,在看到大自然的美景时,你的内心会很愉悦,你并不需要躲在房间里撸管,你完全可以换一种更纯净的活法,活出自己内心的崇高与光明,活出真正纯粹的自己,纯粹才是力量之源!

不撸反而更快乐,这个结论撸者很难相信,但事实就是如此,戒到一定程度就会感受到那种特别纯粹、特别美好的大快乐,这种快乐与欲望无关,当你处于无欲状态时,才能感受到这种大快乐。一颗简单纯净的心,一份久违的童真情怀,蔚蓝的天空下微风轻柔地吹拂着柳条,孩子们在草地上欢快地奔跑着,累了就躺下望着天空的风筝,一切都显得那么单纯与美好。春季可以召唤出内心某种纯真柔软的感觉,你知道你属于这种感觉,你知道不管自己的年纪有多大,只有你愿意,你就可以做回纯净的孩子,一个人只有保持心灵的纯净,他才会真正快乐起来,反之不管撸多少次,最终还是不快乐,乃至空虚、惶恐、痛苦、悔恨,生不如死。

戒色可以让你从一个纯真的视角来感受这个世界,一切似乎都变得纯真美好起来,那种纯真美好的感觉非常久违,因为沉迷手淫,就无法感受到这种纯真美好的感觉,脑子被邪念占据后,就和这种纯真美好的感觉断开了,就像宽带连接被断开一样。埃克哈特·托利在《宁静在说话》里说:“当你与内在的宁静失去连接,你就失去了与你自己的连接。当你与自己失去连接,你就在这世界迷失了自己。你内在最深处的关于你是谁的自我意识,与宁静不可分离。这就是‘我是’,它更深于你的名字与外在形式。宁静是你最真实的本性。什么是宁静?宁静就是你的内在空间或觉知,在这内在空间里,这一页的文字正在被解读,然后成为思想。没有那个觉知,就没有感觉,没有思想,没有世界。”小时候我们的内在空间是非常干净的,没有邪念的污染,邪念就像垃圾一样,必须被清除出去,当心灵得到净化后,你的身心感觉就会焕然一新,就像一个肮脏发臭的房间重新被整理干净,窗明几净,整洁明亮,让你感到舒服惬意、心情愉悦。

很多人并不想过堕落的生活,他们也想戒除手淫恶习,但是心魔是那样强大,一次次把他们拉入怪圈,邪念会主动入侵,不请自来,不速之客,自动冒出,关于这点有很多人都没认识清楚,也包括一些戒色前辈,如果你对这点有很深的认识,你就会时刻提防着心魔,因为你知道它会来!它会主动入侵!就像黑客会主动发动攻击一样,你把杀毒软件和防火墙全部开启,严阵以待,时刻都保持警惕,念头一起来,马上就断除,必须要快!不怕念起,唯恐觉迟!有的人以为悟懂一个道理,或者用不净观把贪恋的心理转变过来,就不用断念了,其实不管懂了什么道理,念头还是会来的,懂道理是为了更好地断念,并不是懂了道理,就不用断念了。之前快乐戒撸法的思想误区就在这里,以为懂了道理就不用断念了,就想当然地以为自己是非撸者了。一些前辈都在这个问题上认识不清,要真正认清这个问题是需要一次顿悟的,在一次次念头实战中,你会渐渐发现邪念出现的规律,很多时候并不是你主动去想,大多都是它来了之后,然后你跟着它跑,戒色高手不会跟念,他们会立刻断念!上次和一位戒友聊,他说洗澡时心魔突然入侵,怂恿他撸,然后他识破这是心魔的怂恿,坚决断除,不听信,不跟从,于是就没破戒。他的这次实战表现很棒,古德云:“顿悟虽同佛,多生习气深,风停波尚涌,理现念犹侵!”光懂道理没用,关键还是要看断念实战!

现在春季,天气转暖,大街上的诱惑慢慢多了起来,自己一定要做好视线管理,这个问题前几天就有戒友反馈,天气热了,穿得少了,考验又来了,到了夏天考验更多更猛烈,要求更强的对境实战意识,元音老人讲过:“内不随念转、外不为境迁。”这十个字就是实战的最高指导。内,不要跟着邪念跑;外,不要陷入诱惑的对境,要做到视而不见、心如止水,对境不动心才是大力量人,对境动心,就是定力差的表现。通过不断对境练心,最后自己的定力就会变得很坚固,当心地光明磊落了,对色情诱惑不反应了,这时候才能真正把握自己的人生,否则看见色情诱惑,一次次陷进去,根本就是身不由己,在邪淫堕落中不断消耗宝贵的肾精能量,不断消磨自己的斗志和动力,最后只会变得消沉和颓废。前两天有一位戒友发帖说在火车上旁边坐一排女的,他就受不了,看照片人家穿得很正常,他却觉得受不了,经常手淫的人看见女性就容易起邪念,真正刚强的戒者就是旁边坐一排女优,都毫不动心,用刚烈的正气秒杀一切,根本不稀罕,根本不在乎,就是这么有骨气,就是这么正!要做戒色敢死队队长,任何诱惑不能动,任何邪念攻不下,死硬到底,视死如归!

下面就春季养生做一下分享,具体如下。

春季天气回暖,身体恢复速度会加快,但也容易出现症状反复,因为气温变化比较大,有时比较暖和,有时突然下雨,天气变凉,昼夜温差大,这时候就容易出现症状反复。每当季节转换时,就像飞机遇见气流出现颠簸一样,过段时间就好了,自己要注意及时调整。中国有一句养生谚语叫“春捂秋冻”,说的是早春时节不要急着把棉衣脱掉,以免感受风寒;初秋来临,也不要一下子穿得太多,适宜的凉爽刺激有助于锻炼耐寒能力,能促进身体的物质代谢,增加产热,提高对低温的适应力。这是古往今来善于养生者都十分重视的保健经验。养生的确是一门很深的学问,戒色之后应该多看养生的书籍,努力提高自己的养生觉悟,养生做得好,身体恢复速度会加快,而且也有利于保持良好的戒色状态。

《黄帝内经》:“春三月,此谓发陈,天地俱生,万物以荣,夜卧早起,广步于庭,被发缓形,以使志生,生而勿杀,予而勿夺,赏而勿罚,此春气之应,养生之道也。逆之则伤肝,夏为寒变,奉长者少。”春天草木发芽,天地一派生机,要懂得顺应春气而舒畅条达,不要损害、克伐它。要顺应春天的养生之道,违背这个法则就会伤及肝气,到夏天就会发生寒性病变。每个季节养生的特点都有所不同,春属东方,五行归木,于脏为肝,因此,春季养生宜顺应肝的生理特点,注意养护肝阳。肝为刚脏,喜条达而恶抑郁,条达为舒展、条畅、通达之意。肝气宜保持柔和舒畅、升发条达的特性才能维持其正常的生理功能,宛如春天的树木生长那样条达舒畅、充满生机。

\paragraph{春季要多亲近大自然,感受那种蓬勃向上的朝气与生机}

在春光明媚的春天,应该多到大自然中走走看看,登高远眺,踏青赏花,置身于鸟语花香的春光中,呼吸着新鲜空气,舒展一下浑身的筋骨,会让人倍觉神清气爽,精神为之一振。这段时间我也抽空到公园去走了走,爬爬山,看看春天的美景,心情也变得格外舒畅,到自然中去走一圈,内心的烦恼也会烟消云散。春季也比较适合做树疗,一位戒友说:“今天在公园和大树依偎了半个小时,内心的确清净了许多,感觉非常的奇妙,但又无法言表,下次有空再去!”亲近大自然会给人一种轻松的感觉,大自然会让你忘却烦恼,依偎在树旁是特别纯真美好的体验,树疗的同时也可以听听轻音乐或者佛乐,不要选择人太多的地方。进入春季会出现情绪不稳的现象,去大自然踏青正好可以帮助调整情绪和心态,我那时得神经症时就经常去公园溜达,看看风景,散散步,坐在湖边吹吹风,这种惬意美好的感觉对身心的康复很有帮助,我那时去公园时也会听听《金刚经》和一些佛教的歌曲。大自然可以给人痊愈的力量,亲近大自然就是一种莫大的乐趣,很多戒友可能比较宅,一直窝在家里,这样对身心恢复不太好,总是面对电脑、电视和手机,时间长了眼睛受不了,人的心情也容易变得烦躁。戒色之后应该定期去大自然中走走看看,用心去感受那份和谐、静谧与美好,我喜欢在公园人少的地方坐下来,在那里,远离了人世的喧嚣,周围很安静,偶尔一声鸟鸣传来,非常悦耳动听,微风从我的脸旁温柔、轻盈地拂过,那种感觉特别美妙,可以触动内心最纯真、最柔软的部分。快乐就在当下,快乐就是如此简单,这需要自己用心去体会。

春天我个人比较喜欢的树木就是杨柳,看上去有一种柔韧、条达、舒展的感觉,非常柔美和优雅,柳树慢慢地吐出它那嫩绿色的柳芽,很招人喜欢,小时候喜欢扎成圈戴在头顶上,上次去公园也看到家长给孩子戴上柳条扎成的圈,让我想起了小时候。春季的大自然给人一种纯真的感觉,这种美好的感觉可以追溯到童年,那时的我们还没有进入欲望和快感的世界,活得比较单纯,所以能感受到这种简单纯粹的幸福与快乐,一旦进入发育期开始追逐快感了,就会彻底迷失,脑子里经常浮现的就是黄片里那些淫秽污浊的场景,再也难以感受到这种纯真美好的大快乐、大喜悦。一个纯净的孩子看到一朵花时,他的喜悦就可以完全爆棚,开心得不得了,就像狂喜一样,大人很难理解孩子为何这么开心,大人很难感受到这种喜悦,因为大人的头脑已经被邪念污染了,开始变得麻木不仁,只知道一味追求快感。戒色后我才逐渐恢复那种单纯喜悦的感受,原来快乐是如此简单,就是因为盲目追求快感才让自己变得不快乐,心灵净化后,那种纯粹的大快乐真的如泉涌一般,心灵的净度越高就越快乐,整个人都被喜悦撑爆了,原来一朵花、一棵树、一声鸟鸣带给我的喜悦远胜于几百 G 的黄片所带给我的快感,不管看多少黄、撸多少次,我都不会真正满足,我都不会真正快乐起来,反而变得越来越空虚、越来越惶恐。曾经我错误地以为看黄手淫会让我快乐起来,然而这个恶习却让我越来越迷失、越来越不快乐。当心灵恢复纯净后,我才真正明白快乐的真义!孩子有纯净、美好、开心、满足的神情,因为他们心地干净,因为他们不邪淫。

\paragraph{春季精神调养——戒郁怒}

《黄帝内经》认为,养生重在顺其自然,肝属木,与春相应,所以春季应以养肝为先,养肝最重要的是调畅情志和气机。调畅情志,重点是保持精神的愉悦,不宜抑郁或发怒,尽量不着急、不生气、不发怒,保持乐观的心情,以保证肝脏气机的条达。戒色后一定要注意情绪管理,情绪不好时要及时调整,不要让自己处于生气失控的状态,不要随便发脾气,要学会保持心平气和,生活中肯定会遇见挫折和不顺心的事情,不管出现什么情况,都要沉着冷静地面对。有经验的资深戒友都知道,生气后容易破戒,会把手淫作为发泄的出口,所以学会情绪管理非常重要,生活中尽量避免吵架,应该以和为贵,懂得谦让,戒色和一个人修养和做人密不可分。春季特别要注重戒郁怒,因为进入春季很容易出现心理的失调,很容易出现郁怒的表现,这时就要学会及时调整。《菜根谭》名句:“宠辱不惊,闲看庭前花开花落;去留无意,漫随天外云卷云舒。”一种平怀,泯然自尽,学会培养超然洒脱的心境。南怀瑾先生去见贡嘎上师,贡嘎上师的侍者对南怀瑾先生说:“我跟他四十年,没有看他发过一次脾气。”贡嘎上师从不发脾气,永远是笑脸,修行非常好,南怀瑾先生非常之佩服。

现代医学研究表明,不良的情绪易使神经内分泌系统功能紊乱,免疫功能下降,春季是肝病、心脑血管病、消化性溃疡等疾病的好发季节,春天应注意情志养生,保持乐观开朗的情绪,戒郁怒以养性,要做到心胸宽阔、乐观豁达。“怒”是养生最忌讳的一种情绪,它是情志致病的罪魁祸首,对人体健康危害极大,中医专门讲到“怒则伤肝”,而且一个人经常生气很容易影响到自己的人际关系,大家都不喜欢爱生气动怒的人。生气动怒不但会伤害感情、僵化关系、影响团结、激化矛盾,危害家庭与人际关系,还易引发脏腑功能失调,使气机升降紊乱,气血运行不畅,新陈代谢出现障碍,容易引发各种疾病。人在生气时会分泌大量内毒素,因此动辄生气的人很难健康长寿。不少戒友在戒色后依然在抱怨嗔恨,经常生气,这样不管对身体还是对人际关系都很不好,生气后也容易出现症状反复,生气对养生很不利。记得以前我也很容易生气,后来戒色了,首先转变过来的就是心态,开始发感恩心、惭愧心、谦卑心,开始有意识地通过行善积德来增加自己的正能量,渐渐就变得很少生气了,情绪变得比较稳定了。戒色后应该注意培养开朗乐观的性格,多一点幽默感,懂得知足常乐,生起感恩之心,这些都是调节情志的灵丹妙药,对于维持人体的生理和心理健康极其重要。戒色后也要注意协调好自己的人际关系,若人际关系好,就会引发愉快的情绪反应,大家相处其乐融融,好的人际关系可以产生安全感、舒适感和满意感,心情自然恬静舒畅。在春光明媚的假日,可以约上好友一起去公园散步踏青,陶冶性情,振奋精神,感受春季勃勃的生机和活力,使肝气顺达,气血调畅,起到防病保健的作用。如诗如画的美景可以让人直接进入临在的状态,脑子里没有念头,非常纯粹的状态,这时语言是多余的,灵魂已经徜徉在无限的美好里,偶遇春游的小学生,更是感概万千,看着他们欢喜雀跃的背影就联想到了自己无忧无虑的童年。

\paragraph{春季养肝补脾}

春季饮食以平补为原则,重在养肝补脾,孙思邈:“省酸增甘,以养脾气。”要少吃酸味,多吃甘味的食物以滋养肝脾两脏,中医讲:“肝主青色,青色入肝经。”青色的食物可以养肝,而辛辣、刺激、大鱼大肉、油炸的食物会增加肝的负担,因此建议多吃时令新鲜蔬菜、水果。肝脏最怕八件事,分别为:肝怕大怒,大怒时,气机上逆而不顺,会严重影响肝的功能,是养生家最忌讳的情绪。因此,要尽量保持精神愉快;肝怕醉酒,饮酒不能贪杯过量,最好能戒酒,肝脏代谢酒精的能力是有限的,大量饮酒,会使肝脏负担加重,甚或出现酒精肝等疾病;肝怕熬夜,肝藏血,“人卧则血归于肝”。现代人经常熬夜,在夜间 23:00 时不能进入睡眠状态,便错过了养肝血的最佳时间,久而久之容易出现肝胆疾病;肝怕抑郁,肝喜条达而恶抑郁,心情压抑、郁闷会令肝气郁结,进而引发胃痛、高血压、头疼等一系列病症。此时,一方面可以通过调控情绪来缓解,一方面可以在医生指导下服用疏肝理气的药方;肝怕用眼过度,中医说“肝开窍于目”,过度用眼也会损害肝脏的功能,建议养成良好的用眼习惯,避免过度用眼;肝怕乱服药,肝脏是人体最大的解毒器官,绝大多数药物都需经过肝脏代谢。乱吃药会加重肝脏负担,容易导致药物性肝损害;肝怕肥胖,过度肥胖容易形成脂肪肝,甚至引发肝纤维化,影响身体健康;肝怕纵欲,中医讲肝肾同源,精血互生,纵欲会导致肝肾不足。春天在五行中属木,而人体五脏之中肝也属木,所以春季肝当令,春季要格外注意养肝,伤肝的行为尽量不要去做。脾为后天之本,气血生化之源,如果长期脾虚,就会气血不足,营养不良,甚至免疫力下降。山药是中医健脾益气的主要食疗食物,做成山药粥,健脾更好,胡萝卜和南瓜也有健脾作用,可以做成南瓜胡萝卜小米粥,效果更好。另外,莲子芡实粥、板栗核桃粥也很不错。

国医大师朱良春活了 99 岁,生前他曾说:“长寿而不健康,活受罪;健康而长寿,是真快乐。”老先生悬壶济世一生,一辈子没有退休,去世前几个小时还在病榻上修改书稿,93 岁的时候日诊 30 多患者,多的时候 40 - 50 号,他一定要把病人看完才会休息。坐诊之余又飞来飞去出席国内外各种会议,精力充沛,体力许多年轻人都赶不上。老先生旺盛的精力来自哪里?据悉老先生每天早晚服用一碗粥,坚持了 60 多年。2010 年朱老接受央视《中华医药》采访的时候,曾详细分享了他的养生方法。1938 年,朱良春跟随老师章次公在上海行医,恰逢上海霍乱急性蔓延,求诊者络绎不绝。朱良春每天看七八十号人、百十号病人,白天晚上不停地接诊。尽管年轻,他也感到体力不支。“疲劳、消瘦、每天工作到晚上都感觉到很累。到三十岁左右的时候,身体就有点拖垮了。”朱良春回忆,当时的状态,他母亲看在眼里,母子俩一合计,设计出了一种粥,吃了几个月以后,他逐步感觉不到疲劳了,精神比较好了。此后,这种粥朱良春每天早晚坚持吃,吃了一辈子,老先生说:“对我的健康、对我精力的恢复、对我衰老的延缓都有很大帮助,所以我一直持之以恒,常年吃它。”

朱良春黄芪长寿粥(五天食量)食材:绿豆、薏仁、扁豆、莲子各 50 克、大枣 30 克、枸杞 10 克、黄芪 250 克。制作方法:绿豆、薏仁、扁豆、莲子、大枣清洗干净进锅待用。黄芪先浸 20 分钟,煎 15 分钟后将水滗出,加一碗水再煎后滗出水,两遍煎后的水合在一起用做煮粥之水。急火煮沸后慢火炖 40 分钟,加入枸杞一起再煮 10 分钟。服用方法:每天取五分之一,早饭前和晚上各吃一半。功效:补益五脏,甚能缓解疲劳。

朱良春先生说,中药里黄芪是大补元气的药,它能提高机体的免疫功能、能够补气。所谓“气”就是功能,用现代的语言讲就是一个人的功能、就是能量,气足了他的能量就强,气虚了以后,能量就不够了,就容易疲劳,耐力就差了。煮粥最大的秘密在黄芪,黄芪作用可不凡,明代医学家李时珍说它为“补药之长”,《本草求真》中,黄芪被推崇为“补气诸药之最”。相传古代的苏东坡擅长中医养生,就常常用黄芪来进补,还留下了“黄芪煮粥荐春盘”的诗句。据说胡适在中年以后,常常会感到疲惫,力不从心,也是用黄芪泡水,代茶饮用,特别是在讲课之前,他会先喝上几口黄芪水,精力倍增,说起话来声如洪钟,滔滔不绝。现代药理研究也表明,黄芪确有明显的强壮作用,而且还能够降低动脉压,加强心肌收缩力,防治循环衰竭。绿豆、薏仁、扁豆、莲子、大枣、枸杞,可以补五脏,在这个基础上再加上黄芪的补气作用,强强联合,提升精力体力。黄芪还有一样功效备受朱老的推崇。黄芪含有微量元素硒,它能增强体力,提高免疫功能,能抗癌,能防癌。朱老的这方子里,黄芪用到了 50 克,可能有人会担心,黄芪是温性的,长期吃会不会上火啊?朱老对此有解说:“这粥中,稍微凉一点的是莲子、绿豆,黄芪、红枣带一点温性,所以它们之间相互之间能协调、平衡,所以大家都能吃。”当然,并不是说今天吃了,明天力气就很大了。要坚持吃过一段时间后,你才会发现自己应付疲劳、抵抗疲劳的能力增加了。黄芪泡水喝最好能控制在 15 克左右,网上有些方子建议使用 20 - 30 克,这要具体情况具体分析。另外,黄芪泡水的使用量和煮粥的使用量是不一样的,如果是泡水喝就是 15 克左右,如果是熬粥,那么最好使用 50 克左右。无论是熬粥还是泡水,黄芪的使用量都是有明确规定的,不可过量使用,而且如果是肾阴虚、湿热以及热毒炽盛的患者,则不宜服用黄芪泡水,否则会加重病情。

我个人比较推崇粥疗,具体吃什么粥,可以根据自己的体质来选择。

\paragraph{春季养生推荐}

春天梳头益处多,中医认为,经络遍布全身,气血也通达全身,这些经络或直接汇集头部,或间接作用于头部,头顶的“百会穴”就由此得名。通过梳头,可以疏通气血起到滋养和坚固头发、健脑聪耳、散风明目、防治头痛的作用。早在隋朝,名医巢元方就明确指出,梳头有通畅血脉,祛风散湿的作用。北宋大文学家苏东坡对梳头促进睡眠有深切体会,他说:“梳头百余下,散发卧,熟寝至天明。”《养生论》说:“春三月,每朝梳头一二百下。”为什么要特别强调春天梳头?这是因为春天是大自然阳气萌生、升发的季节,人体的阳气也顺应自然,有向上向外升发的特点,表现为毛孔逐渐舒展,循环系统功能加强,代谢旺盛,生长迅速,故而春天梳头正是符合春季养生强身的要求,能通达阳气,宣行郁滞,疏利气血。我家里有一把绿檀的木梳,现在每天梳梳头,感觉脱发变得更少了,每天不超过五根,而且梳完头感觉精气神变好了,睡前梳头也很好,可以促进睡眠,提升睡眠质量,梳下头就感觉头部的气血很舒畅,心里也美美的。

春天是万木争荣的季节,人亦应随春生之势而动。可以根据自己的身体情况进行适量的运动锻炼,让筋骨得到舒展,让精神得到振奋,为一年的工作学习打下良好的基础。合理适量的运动能够使人体的气血流通,百脉畅达,可以增强人体的生理功能,提高人体的抗病能力,使得机体强健而达到祛病延年的目的。《黄帝内经》里讲到“广步于庭”,即在环境优美、空气清新的庭院中悠闲地散步。《老老恒言·散步》曰:“散步者,散而不拘之谓,且行且立,且立且行,须得一种闲暇自如之态。”说明散步是在随意不拘、悠闲舒适、逍遥自在、闲暇自如的状态下进行的一种运动,完全没有任何思想负担。因此,散步有助于精神的放松,有助于身心的陶冶,有助于气血的运行,可以达到养生延年的目的。我那时得神经症时,锻炼就是从散步开始的,刚开始只能散步五分钟左右,当时身体已经相当虚弱了,走五分钟都感觉很累,得神经症后是忌讳出大汗的,因为大汗伤阳,身体本就虚弱,是不宜出大汗的,应该以散步和养生功法为主,等身体恢复差不多了,可以考虑慢跑或者球类运动。我那时经常去公园散步、看风景,感觉心旷神怡,别有一番真趣,是心灵上的一种享受,我很喜欢公园内里那种安静的氛围,很容易就让我放松了下来,我那时社恐、强迫、疑病都有,后来坚持戒色养生,这些症状就渐渐缓解消失了。在恢复的过程中,经常亲近大自然可以大大加速疾病的恢复进程,去公园走一趟,心情就舒畅很多,就像做了一次心灵按摩一样。春天放风筝也很不错,而我更喜欢躺在草地上看风筝在天空飞,双手枕着头,用特别纯真的眼神看着风筝飞舞,这种感觉就像回到了童年,我想永远做一个纯净的孩子,虽然身体会长大,但心灵永远保持纯净、单纯和美好。春天骑行也是不错的选择,不负春光不负美景,去户外感受一下那种自由自在迎面而来的清风,感受骑行带来的快乐,享受大自然带来的阳光和美景。自己可以安排一下骑行的路线,为了避免久坐,可以骑一段,下来推一段,散步和骑行结合,这样效果更好。

最后总结:

春季是一个生机盎然的季节,不像冬天那么萧瑟凄凉,也正是这份生机可以最大地激起戒色的斗志与热情,春季有一种积极向上的氛围与力量,看到那种蓬勃的朝气,内心深处就会受到某种感召,希望自己也能找回那种积极向上的状态。春季也是一个容易破戒的季节,进入春季时,情绪容易变得不稳定,很容易出现情绪破戒,自己要做好情绪管理。春季天气多变,气温乍暖还寒,冷暖骤变,气温差异较大,症状很容易反复,自己要注意休养和调整。春季日照增多,会导致欲望变得活跃,邪念进攻次数会变得频繁,这时候要格外警惕,做好断念,不要被心魔攻破。春季就像童年,希望每一位戒友都能在春季找回最纯真的自己,只有做回纯净的孩子,你才会真正快乐起来,每一天都活在新鲜的喜悦里,开开心心过好每一天。

下面分享一首戒色诗歌。

\begin{poem}[找回最纯粹的自己]
    \begin{multicols}{3}
        \centering~\\
        有多少次紧盯屏幕 \\ 两眼冒着绿光 \\ 为那些诱惑疯狂不已 \\ 然而射过之后 \\ 一切都变得索然无味 \\ 他就像经历了一场抢劫 \\ 呆呆傻傻地躺在床上 \\ 两只死鱼眼睛 \\ 空洞地望着天花板 \\ 这就是多巴胺的骗局 \\ 一次次上当受骗 \\ 肾精资源 \\ 被一次次无情掠夺 \\ 他感觉自己被困住了 \\ 看黄手淫的堕落生活 \\ 就像一个牢笼 \\ 他必须打破这个牢笼 \\ 才能重新获得自由 \\ 漆黑的房间里 \\ 传出撕心裂肺的呐喊声 \\ 他彻底受够了心魔的奴役 \\ 不能再这样活! \\ 与其在邪淫的地狱里凄惨哭嚎 \\ 不如彻底推翻心魔的暴政 \\ 彻底解放和拯救自己 \\ 必须找回最纯粹的自己 \\ 真正的力量来自纯粹 \\ 做一个纯粹的人 \\ 直到死亡
    \end{multicols}
\end{poem}

下面推荐书籍

\begin{book}[《钻石途径》,阿玛斯]
    阿玛斯(A. H. Almaas),21 世纪最重要的心灵导师之一,“钻石途径”创始人。1944 年,生于科威特,学术专长是物理、数学及心理学。阿玛斯具有完整的现代心理学背景、丰富的个案经验,并受过最高层的佛法训练和东方修炼法门,又撷取了葛吉夫的教诲、苏菲神秘主义、金刚乘和禅宗的精髓。他最主要的贡献是将西方心理学与东方智慧整合起来,发展出一条循序渐进、直接而精准的“钻石途径”。钻石途径一共四本书,是一个系列,我一共做了 521 条笔记,阿玛斯用的也是“本体”这个词,钻石途径意味着与本体连接,培养钻石般的觉察力,里面讲到:“本体才是我们快乐、喜悦和满足的源头。”本体即妙明真心,就是纯粹的觉知,真正的自己,我们来到这个世界上的终极目的就是认识自己的本体,活出自己的本体。《钻石途径》是非常好的书,不过作者用的是哲学和心理学的语言,读起来不是很好懂,但只要你真正明白那几个关键词的含义,就一下把握了核心,心灵导师可以用很多种方式来描述真理,就像有很多种路牌,但指向都是同一个,那就是纯粹的觉知,阿玛斯书里提及的“人格”这个词相当于《当下的力量》里的“小我”(ego),几个关键词理解正确了,就相对好懂许多,就能真正明白作者究竟在讲什么。这套书籍我个人比较喜欢,网上有 \texttt{.pdf} 版本,我花了几天就看完了,收获很多,笔记已经复习了很多次,每次复习都有新的领悟。感兴趣的戒友可以看看这套书籍,希望每位戒友都能发展出钻石般的觉察力,对付心魔就需要极其敏锐且强悍的觉察力。
\end{book}
