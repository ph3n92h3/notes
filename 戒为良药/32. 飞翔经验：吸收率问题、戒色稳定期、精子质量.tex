\subsection{吸收率问题、戒色稳定期、精子质量}

\paragraph*{前言}

最近看到一个戒友发帖,说到带新人的问题。大量的新人涌入戒色吧,其实是很需要老戒友指导的,新人有两个显著特点:一,思想误区多,很多人大脑里还有无害论的残留;二,问题比较多,戒色过程中的问题是非常多而且复杂的,我到现在回答过的问题有几千个,各式各样的问题,五花八门。新人大脑中的思想误区一定要得到及时的纠正,否则他心存疑惑是会导致破戒的。要戒色成功,首先要把遇见的问题都要搞懂,搞明白,否则带着疑惑戒色,是会影响到信心和决心的,有疑惑的戒友是极其容易产生动摇的。所以,我们老戒友,要及时帮助到新人,消除他们的疑惑,纠正他们的思想误区,帮助他们建立戒色的信心,在戒色过程中给予及时的指导,让新人们能早日进入戒色正轨,而不是盲戒和强戒。

记得以前我回答问题,首选的基本都是零回复的帖子,我那时的想法就是要消灭戒色吧零回复的帖子,尽量给予新人及时的指导。现在我主要在自己帖子回答问题,偶尔也会在其他帖子答疑。我现在的生活状态就是,不能把自己搞得太累,每天回答问题在三十个左右,多则五十个,一般就限制在五十个以内,国庆节回答的问题比较多,有几天超过了每天八十个问题,因为国庆节是休息日,我就多回答了些问题,多回答问题明显会感到自己的精力不够用,容易造成身累心累的现象。所以,我现在尽量控制答疑的数量,以便调整体能,更好地坚守在戒色吧。

其实很多问题,我都有回答过,有的问题甚至回答过了几十遍,但对于每个新人,我都会保持应有的耐心,因为我觉得新人是很需要耐心的指导的,你在指导新人的同时,其实也是在强化自己的正能量,也是在增加自己的定力,同时,你也会把这种度人的精神传达给新人,新人迟早会变成老人,老人再带新人,这样戒色吧就会形成良好的氛围,发展就会越来越好,大家也会戒得越来越专业。我相信将来戒色成功的人会有很多,当然现在已经有一批戒友成功了,他们已经进入了戒色稳定期,他们也正在无私地帮助更多的新人。现在的新人是幸福的,前辈的路已经帮你们铺好了,只要你自己够努力,不断学习提高觉悟,彻底戒掉撸管是迟早的事情。戒色吧也很强调养生的重要性,你在这里也可以学到养生方面的知识,这样对于你身体的恢复也是很有帮助的,戒是一方面,养生恢复又是一方面,两手都要抓,两手都要硬,这样身体恢复就比较理想了。

我们一定要多帮助新人,帮助新人就是在帮助自己,度人即度己!新人也会变成老人,老人再帮助下一拨新人,薪尽火传,我们要把这种度人的精神传下去。让更多的人认识到撸管的危害,帮助更多的人彻底戒掉撸管,重新恢复阳光健康。大家加油!

对于新人,我的建议就是多学习,建议先把戒色吧精品分类里的文章多扫几遍,很多问题在精品文章里都能找到明确答案,通过精品文章的学习,你的觉悟会提升得比较快。

下面进入正文。这季就吸收率问题,戒色稳定期,精子质量这三方面详细论述一下,具体如下。

\subsubsection{吸收率问题}

这季说的吸收率,并不是说肠胃对食物的吸收率,而是说学习戒色文章的吸收率,在我帖子里,还有私信里都有戒友反映戒了一年多,还是戒不了,反复破戒,症状如故,搞得很灰心丧气,戒得很绝望。问他戒色文章看了没,他说都看过了,戒色的道理他都懂,但就是戒不掉,记得有位戒友说,他对戒色文章都免疫了,看到文章的标题就知道里面讲什么,有的戒友甚至出现了厌倦情绪,厌倦学习戒色文章,我在前面的文章有讲到过,出现厌倦情绪,一定要注意及时调整,让自己重新找回戒色的良好状态。

要彻底戒掉撸管,说到底,就是一个不断修炼觉悟的过程,觉悟必须得到持续的提高。

有的人开始进步会很快,但学到一定程度,就似乎进入瓶颈期了,觉悟不再上升了,很多戒友戒色知识和道理他都懂,但问题就出在,他懂得不够深刻,吸收率太低了,看完一遍文章,所能记忆和复述的内容太少,自己思考领悟的内容也太少了。就像我问你,你看过某某电影吗?你回答说看过,然后我问你,里面的内容是什么,你只能记得大概,很多细节性的内容却没有任何印象了。这其实就是吸收率太低,就像你吃一种食物,你只吸收了其中一部分的营养,而最关键最重要的营养你却错过了。对于一本好书或者一篇好的文章,我都会反复看很多次,不断提炼,不断总结,提炼其中的精华思想,这就像沙里淘金的过程,你一定要把文章里的灵魂给提炼出来,然后彻底吸收它,这样你的觉悟就会变得更强大。你必须找到一篇文章的灵魂,其实一篇文章的灵魂也就那几句话,你把重要的话记下来,都放在一个笔记本上,这样时常看看笔记本,尝试主动去记忆这些内容,能达到自由复述的程度,这样就基本到位了。如果你能结合自己的体验,有更深入的思考与领悟,那就更好了。

我当初觉悟的提高用了一年多,这一年多我一直在做笔记,把我认为好的句子和思想观点记录下来,不断地复习这些内容,这样我的觉悟就能得到稳步的提高。我刚开始和心魔的差距是这样的,我被心魔虐了十几年,对心魔总是无能为力,见一次心魔就被虐一次,特别是周末一个人无聊时,心魔就喜欢跑出来,把我虐惨了。我那时觉悟太低,根本没有任何学习的意识,所以戒了十几年都是失败的。后来,我意识到学习的重要性,不断地学习提高自己的觉悟,然后我发现了一个现象,再遇见心魔,我能降伏它了。这就像和心魔下棋,过去你老输,是因为你觉悟不行,当你通过学习提高觉悟后,你就会发现,你可以降伏心魔了,你能战胜心魔了,心魔动不了你了。而要做到这一点,唯有通过坚持学习持续提高觉悟。否则你放松学习,放松警惕,迟早还是会破戒的。有戒两百天破戒的,也有一年多破戒的,所以我们应该不要离开戒色文章,不要离开戒色吧,每天尽量上来看一下,五分钟也好,过来帮助下新人,也是在强化自己的正能量,如果不能上网,应该把你的笔记本拿出来,复习一下戒色知识。一旦进入戒色稳定期,每天并不需要多长时间来学习,能够温故而知新,然后保持警惕,这样就可以了。

要提高自己对文章的吸收率,最好的办法就是:做笔记!做笔记是提高吸收率的利器!

我到现在还经常做笔记,做笔记的过程就是不断总结,深入记忆和理解的过程,你的笔记本就是精华思想所在,你认为所有重要的戒色思想和观点都应该归纳在一本笔记本上,我现在这样的笔记本有十多本,就是通过这样的积累,我才有现在的思想觉悟。很多戒友都是学生,学生应该都有记笔记的习惯,老师应该也强调过记笔记的重要性,所以,在戒色方面,我们也应该发扬这个优良传统,不断记录精华的戒色句子,然后不断复习它,温故而知新,每次复习你都会有新的认识,这样你的觉悟就可以得到持续的提高。

下面是戒友“让一切清晰”的一段话,大家可以看看:

\begin{quote}\it
    大哥,我来了!四五天没上网了,把精力差不多都用在工作上!但每天醒来、每天累的时候、每天睡觉之前都时刻想着戒色的警惕性!今天一有空了,就拿着笔记本拿着笔,记了您一些精彩的图片语录!还没记完,因为太多又需要慢慢领悟!一起加油!大哥!!
\end{quote}

清晰这位戒友,现在戒得比较稳定,一是因为他肯学,学习的方式也很好,也是做笔记的方式,通过提炼文章的精华思想来加深自己对文章的理解,提高自己对文章的吸收率。之前我观察过,另外一位戒友汤姆,也很善于学习和总结,总结就是再学习再理解再思考,通过自己的一次总结,将会有新的领悟。其实总结也是一种加深记忆的方式,我学习的方式也是不断总结,不断提炼,然后汇集到一个笔记本上,然后经常翻阅复习这个笔记本,我的觉悟就是这样一步步上来的。

这季我总结了一个戒色成功的公式:\textbf{戒色成功 = 觉悟高 + 警惕强}。

真正戒掉的戒友,一定是具备这两个特点的人,而觉悟高如何获得呢?只有一个途径,就是通过不断学习提高觉悟,学习有方法,做笔记就是一个很好的武器,我们应该捡起来。严格地说,警惕意识也是觉悟的一部分,我把警惕专门分开来讲,就是为了强调警惕的重要性,已经有无数反馈表明,一放松警惕就会破戒,一放松警惕就会被心魔吃掉。所以,戒色的每一天,我们都应该保持高度警惕。警惕就像手中沙,一放松警惕,肾精就流失了……

大家要提高觉悟,应该广泛地吸取各类文章的精华思想,不止是我的文章,只要你认为好的文章,都应该去吸取它的精华部分。比如养生类的文章,传统文化的文章,还有其他门类的文章,都是可以吸取的,你就像一棵树,当你吸收了足够的营养后,你的觉悟就会变得更强大,到时降伏心魔就不在话下了。

\subsubsection{戒色稳定期}

下面谈下戒色稳定期。

戒色稳定期,是众戒友梦寐以求的一种戒色状态,一旦进入戒色稳定期,你会发现“煎熬感”不见了,天数也变得不重要了,你会觉得时间过得很快,没有很难熬的感觉。因为进入戒色稳定期的人,已经基本能降伏心魔了,心魔很少出来骚扰他了。在戒色稳定期,每天只要花不多的时间复习一下戒色知识,然后保持警惕意识即可。当然,如果你觉悟很高,也不一定要每天都学习戒色文章了,一周学一次也可以,但一定要保持高度警惕,这是非常关键的。没有警惕意识,就会被心魔吃掉。不管你戒两百天还是几年,失去警惕意识就会破戒!

那么戒友会问,多久才能进入戒色稳定期?

有文章写过是七十天,或者三个月,根据我的研究,到底多久才能进入戒色稳定期,这是因人而异的,因为每个人觉悟的提高速度是不一样的,有的人觉悟提高快,可能半年就进入戒色稳定期了。有的人戒色半年,不学习,觉悟还是很低,还是反复破戒。如果他不提高觉悟,那就永远进入不了戒色稳定期。要进入戒色稳定期,它的通行证就是:你的觉悟等级要够!就像有的单位对于英语水平会有要求,比如四级或者六级。当你的觉悟达到级别了,你才能进入戒色稳定期,否则不学习,觉悟永远那么低,要进入戒色稳定期就遥遥无期了。

我那时进入戒色稳定期大概花了半年时间,但是那时候我并没有放松学习,因为我那时觉得自己要学的太多太多,现在的我也是如此,虽然已经懂了很多,但是学无止境,我还是想多学习,懂得更深刻的道理,我现在过的是一种悟道的生活,不断学习,不断领悟道理。

如何判断自己进入了戒色稳定期,你可以这样来观察,一般进入戒色稳定期,心魔就很少出现了,不会有很强的破戒冲动,只要不放松警惕,一般是不会破戒的。比如在进入戒色稳定期之前,你一天想破戒的想法是二十次以上,当你进入到了戒色稳定期,你会发现,可能一天有破戒的想法不会超过三次,甚至一次也没有,我现在的状态就是一次也没有,已经很久没破戒的冲动了。

YY 刚一起来,还没形成冲动,就被我断掉了。我现在断 YY 就讲究两个字:快和狠!绝不容忍 YY,对 YY 采取零容忍的态度,决不妥协。进入戒色稳定期,还有一个很大的特点就是“煎熬感”消失了,就是没有那种很难熬,很忍不住的感觉,对于天数也无所谓了,因为你的戒色境界已经完全超越了“天数党”所在的层次。

值得注意的是,刚开始戒色后会进入欲望休眠期,一般欲望休眠期在一个月左右,很多人在这段时间 YY 很少,基本没破戒的欲望,这个阶段可不是戒色稳定期,这只是欲望休眠期,这一定要弄清楚,很多人在欲望休眠期时会洋洋自得,以为戒色很简单嘛,一旦过了欲望休眠期就是破戒高峰期,这时候才是考验你觉悟和定力的时候。

最后总结下,进入戒色稳定期的标志就是:觉悟要够!觉悟如果不行,戒十几年都进入不了戒色稳定期。

\subsubsection{精子质量}

最后来谈下精子质量的问题。

先上几个案例:

\begin{case}[精子质量]
    婚前有过一年的过度 SY 史,导致婚后生活射精无力,要小孩没要上,去做了一次精液常规检查,发现精液只有 3 \unit{\ml},无精子,精液含有白细胞、红细胞。有 SY 史前,身体体质好,SY 后发现体质变弱,经常腰痛。
\end{case}

\begin{case}[精子质量]
    我大概七八岁的时候就 SY,阴囊潮湿,后检查无精子,经多次检查都一样,还有没有生精的希望,想问问我还需要做什么检查?
\end{case}

\begin{case}[精子质量]
    我 SY 五年,总是压迫它,有一次射出一黄豆大的白色颗粒,之后检查无精,能治好我的病吗?我的睾丸也很小。
\end{case}

\begin{case}[精子质量]
    2001 年春的一天,我又一次 SY 后,脑子感觉一下子天好象塌了下来,被笼罩得昏昏沉沉。我以为过几天就没事了,可是这昏昏沉沉的感觉一天比一天严重,浑身还潮热不堪,我当时还没有结婚,也不敢把 SY 的事及出现的情况告诉家人。到 2002 年春,我感觉身体越来越差,潮热也越发严重。我就到医院就诊,经检查说是生殖器龟头,有细小的赘生物是尖锐湿疣。我害怕极了,赶紧找钱治疗。治疗后。身体脑子还是越发不好。有时候,站着腰很疼,脑子昏沉什么也不想干,胡思乱想。突然有一天,我感觉腰部没了什么感觉,也使不上力,到医院什么也检查不出来,让吃的药很贵。我就去街道的门诊看了,当时门诊医生让我吃仁汇肾宝,我一盒没有吃完,就觉得身上的潮热又狠了,后来门诊医生又让我吃他自己配制的中药补肾胶囊和大量西药维生素,六味地黄丸,杞菊地黄丸等等,我越吃越感觉不好,连睡觉都出汗,记忆力也好象没了似的,身上的劲也越发使不上了,好像地狱一般的生活,什么也不感兴趣。2003 年冬,我就这样结婚了。可是,洞房时还没动几下就不行了。我又找了一个老中医就诊,他说我前面吃的那些药都不对,他给我开中药调理,就这样我离不开了中药,一直吃却一直不见好转,早泄能戴着避孕套坚持,可终究还是不行,房事后感觉被刮了一层似的,身体脑子都疲倦得很,犹其左边的脑子麻木、紧皱、瞥闷,后颈也有这样的感觉。精子化验 a 和 b 级才 30\%,三年了都没有生育。
\end{case}

\begin{case}[精子质量]
    我一直都有 SY 的习惯,已经好几年了,就是没改掉,现在娶了老婆也有五年了吧,还是一直都没怀上小孩,我结婚后也经常会 SY,我很担心是因为 SY 导致的。
\end{case}

\begin{case}[精子质量]
    由于我在青少年时期,有严重的 SY 史,婚后精液检查为弱精症。A 级太少,活率太低,怎么办?
\end{case}

\begin{case}[精子质量]
    我结婚前一直都是 SY 来满足自己,原来也不知道会有什么后果,一周要 SY 五天左右吧,结婚后一年都了没怀孕,然后去检查说我有弱精症,我都晕死了。
\end{case}

\begin{case}[精子质量]
    本人男,34 岁,结婚五年,一直没有孩子,我和爱人去年检查,发现我有中度弱精证。
\end{case}

\begin{case}[精子质量]
    我今天二十岁,和女友同居半年都没有怀孕,我十三岁的时候开始 SY 的习惯,而且比较频繁,平均一天一次,有五六年的历史。到了十五岁的时候经常遗精,现在 JJ 也是十三岁时候的大小,只有八厘米,喉结也不是很突出,胡须不是很多,感觉好象还欠一段没有完全发育。还有易疲劳、腰酸背痛、冬天冷得发抖、夏天出很多汗。前段时间去医院检查精子,A 级 9.78\% 、B 级 7.68\% 、酸碱度 7.0,只有这三项不正常。
\end{case}

\begin{case}[精子质量]
    从十三岁开始就有 SY,有时较频,有时几周一次。婚后发现,存在严重的早泄现象,五年未有好转。去年三月份开始,晨勃现象消失了。JJ 以前很易勃起,但不是十分坚硬,自晨勃消失后,勃起较以前困难。婚前,我总觉身体长年都特别热,易出汗。稍运动一下就满身大汗,婚后稍好些。记性差、易困、精力不集中,常觉得腰酸背痛,脸色黑。95 年曾得过肺结核,98 年大学毕业时,医生检查时说完全好了。本人从事的是产品设计工作,每天用电脑较多,有时画图时,一坐就两三小时不动。多次检查均为死精(0\%,冶疗吃药后有时会达 5\% - 10\%,但从未超过 10\%。药停后又为 0\%),精子异型率特高有 60 - 80\%,彩超检查有轻微的精索曲张。
\end{case}

SY 是会伤害精子质量的,这在西医方面也证实了,SY 可引起尿道炎,前列腺炎,精索,精囊炎,阳痿,勃起不坚,习惯性早泄等。一旦得上了前列腺炎,势必会对精子质量产生不良影响。另外,精索静脉曲张也会影响到精子的质量。当然不是所有前列腺炎患者和精索患者都会不孕不育,这也要看病情的严重程度,很多人虽然有慢前和精索,但去检查精子质量,还是有生育功能的,就是畸形精子和死精有不少,有的戒友去检查显示活力低下。

虽然有些戒友还是有生育功能,但也不要以为没事,因为有生育功能,不代表你的后代就是健康的,在慢前和精索环境下产生的精子势必没有在正常环境下产生的精子质量好。卫生部发布《中国出生缺陷防治报告(2012)》显示,我国每年新生儿出生缺陷率为 5.6\%,约九十万例。根据 2007 年公布的成都市新生儿数量以及缺陷发生率估算,成都每年约有 1200 名有缺陷的新生儿出生,主要为唇腭裂、先心病、听力障碍等。王菲和李亚鹏的孩子就是兔唇缺陷儿,大家应该都知道的。纵欲无度以致精神萎靡不振、困倦、心悸、头昏眼花、腰腿酸软,这表明性生活已经超过身体负荷,不仅会使人意志消沉,影响健康,而且还会增加睾丸负担,不利于高质量精子的生成,妨碍优生。很多戒友虽然还有生育功能,但精子质量其实很一般,古代名医就有讲到过,纵欲生出的后代容易夭折,身体的体质也不好。所以,我们在婚前要好好坚持戒色养生,把身体养好,把精子质量养好,这样才符合优生的条件,这是对自己健康的负责,也是对家庭、对子女后代的负责。

一旦出现精子质量问题,一方面要坚持戒色养生,戒色养生是精子质量改善的基础,然后再积极治疗,这样精子质量会慢慢恢复正常的。如果你一边吃药一边纵欲,精子质量是否能恢复就要打个问号了,即使暂时恢复,你这样纵欲,难保又不行。纵欲的危害实在很大,不仅影响你自己的身体健康,还会影响到下一代。所以,我们一定要好好坚持戒色养生,为了自己,也为了下一代。

上季推荐了五本书,大家的反响还不错,但是也有戒友说,文言文看不懂,其实我推荐的书,只要有高中学历,看懂文言文应该没问题,当然你也可以看文白对照的版本,相术方面的书一般有文白对照,中医方面的书就没有了,要读懂中医书籍,是需要一定的古文功底。但也有例外,《黄帝内经》就有白话文版本的。这季继续推荐五本书:

\begin{book}[《九种体质使用手册》,王琦]
    王琦是北京中医药大学教授,博士生导师。这本书挺不错的,明确自己的体质,可以更好地去养生。
\end{book}

\begin{book}[《秘密》]
    这里埋藏着一个隐藏两千多年的富有、成功、健康的秘诀。自从有人类的那一天开始,人类就开始寻找这个秘密。这本书土豆吧主也推荐过的,讲的就是吸引力法则,其实就是心理学方面的暗示原理,如果你暗示自己积极的内容,你全部的能量就会朝着积极的方向去努力,如果你暗示自己消极悲观的内容,你就会变得很消极,这样就会增加失败的可能性。
\end{book}

\begin{book}[《名医类案》,明代江瓘]
    这是一本医案书籍,是明代以前著名医家临床经验的总结,案例非常丰富。这本书我非常喜欢,读医案可以让你知道,人是怎么病的,当你知道了这些致病因素,你就会学会避免。很多中医医案都有讲到纵欲伤身的案例,有的医案书籍甚至每隔几页就出现一例。所以,研究中医医案,对于提升我们戒色方面的觉悟也是很有帮助的。
\end{book}

\begin{book}[《一万小时天才理论》]
    一万小时天才理论说的是:要想成为这一领域的世界级专家,你就需要花大约一万个小时来练习。没有其他的法子,如果你想成功,那就必须得花一万个小时。这本书里面的理论很有启示意义,髓鞘质是交流、阅读、学习技能、人之所以成为人的关键。髓鞘质这个概念很有意思,相信看过这本书的读者应该会知道这个名词。
\end{book}

\begin{book}[《灸除百病》]
    在我身体恢复的方面,艾灸的确立过大功,特别是在我胃肠功能的改善和鼻炎的改善方面,艾灸的确很管用。扶阳之法,艾灸第一。\textit{阳气若足千年寿,灸法升阳第一方。(《黄帝内经》)} 宋代的著名医学家窦材在《扁鹊心书》中认为自古扶阳有三法:灼艾第一、丹药第二、附子第三。大家看到这里,肯定会觉得艾灸很好,但我要说的是,艾灸虽然看似简单,但里面的水真的很深,所以如果你想尝试艾灸,最好先多学习相关的知识和经验,可以看书,也可以看艾灸视频。艾灸虽好,但一个有烟,还容易起泡,这方面要多注意,多看多学是可以协调好这两个问题的。用对用好了艾灸,你的身体会恢复得更快、更好。艾灸我自己深入体会了两年多,在这方面懂得还是挺多的。戒色是一方面,如何养生也是同样重要的,很多戒友戒了一年多还是恢复不理想,其实就是在养生方面做得太差,很多人根本没有养生意识,还在久坐熬夜,不运动。所以,我建议大家如果想更快更好地恢复,必须在养生方面下足功夫,努力提高自己的养生意识。
\end{book}
