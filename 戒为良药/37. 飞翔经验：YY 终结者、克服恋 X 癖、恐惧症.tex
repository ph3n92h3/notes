\subsection{YY 终结者、克服恋 X 癖、恐惧症}

戒色吧发展加快,大量新人涌入,这是好事,戒色吧是需要新鲜血液,可以带动更多的人戒色。但我看很多新人的发帖,很多人其实戒色还未入门,还停留在强戒阶段,破戒了归咎于自己的毅力不行。如果把破戒的原因归咎于毅力,那永远也戒不掉,必须意识到学习的重要性。学习提高觉悟,觉悟战胜心魔,光靠毅力是注定失败的。

要戒色成功必须坚持学习提高觉悟,把戒色文章这个思想武器捡起来,好好武装自己的头脑,不要一破戒就拿毅力说事,那样戒一百年也戒不掉的。还有的戒友会说,为何我学习了还会破戒,我的回答就是:觉悟未到。觉悟的提升是一个连续的过程,并不是你看了几篇戒色文章就能成功戒色的,你必须通过不断学习了解戒色的方方面面,戒色是一门学问,要学习和实践相结合,破戒后多总结经验教训,多写心得,多温故而知新,这样觉悟就能持续提升,到时候就有实力战胜心魔大 BOSS 了。否则你不学习,你就永远是戒色菜鸟,永远被心魔虐,没有还手之力。

戒色后,你的敌人是谁一定要明确!很多人戒了很久,都不知道敌人是谁!其实敌人就是你自己的心魔。有人会问,什么是心魔。其实心魔很好理解,说白了,就是放纵的想法和堕落的冲动。人是有这种自毁倾向的,因为年轻无知,很多人刚开始并不会认为这有什么不好,但症状出来就开始悔恨了。戒色就是为了一个目标:战胜心魔!心魔像一种动物,那就是野马。你必须要学会驯服这匹野马,而不是让它牵着你跑。如果你无法战胜心魔,那么你就是欲望的奴隶,迟早会被症状缠上。而降伏心魔,只有一种途径,那就是靠学习提高觉悟,觉悟高强即可降伏心魔。

很多资深戒友已经深刻认识到了学习的重要性,我看很多老戒友回复新人,都会提到要多学习精品文章。打个比方,你不学习,你永远是小学一年级水平,如果你不断坚持学习,那么你的等级和觉悟就会持续提升,可以念到大学乃至博士。如果你不学习,在戒色方面永远是菜鸟水平,永远无法和心魔大 BOSS 抗衡。

我给新人的一句忠告就是:多学习,多总结,多做笔记,好好练级!很多戒友都是靠学习突破一百天,乃至半年,甚至一年以上的。没有别的办法,只有靠学习提高觉悟。决心再大,没有学习意识,也是垃圾一个,三分钟热情是无法戒色成功的,必须养成良好的学习习惯,让觉悟持续得到提高才是戒色王道。

最近有戒友破戒后和我说,说他破戒了才想起保持警惕性的重要性。还有的戒友则是和父母吵架,情绪破戒。戒色吧现在很开明,并不忌讳破戒,而是希望广大戒友破戒后好好总结反省来提高自己的戒色水平,所谓吃一堑长一智。破戒的精费不能白交,一次破戒,一定要好好总结经验教训,这样你才会戒得更好!保持警惕性是异常重要的,\textbf{戒色成功 = 觉悟高 + 警惕强}。失去警惕的羚羊会被豹子吃掉,失去警惕的戒友会被心魔吃掉,一样的道理。戒色的每一天都要保持高度警惕,时刻防着自己,特别是防止自己 YY。很多戒友因为缺少亲身体验,所以看我的文章可能无法深刻理解和认识,我曾经多次提到保持警惕性的重要性,很多人看了也没有共鸣。直到他破戒了,他才突然意识到原来保持警惕性是如此之重要。然后,他的警惕意识水平就会上升很多,因为他亲身体验到了,吃到亏了。戒色是强调体验的,缺少体验和反省意识是很难戒色成功的。还有的戒友是情绪不好导致的破戒,这种破戒类型我以前的文章专门总结过,管理好自己的情绪也是非常关键的,等你学会管理自己情绪了,你就会戒得更稳定更长久。

股市有一句话叫:“因为你是一流的输家,所以你才是一流的赢家。”何谓一流的输家?一流的输家就是会从失败中总结经验教训,不断完善自己。戒色领域也是如此,你是一流的破戒者,所以你才是一流的戒色成功者。破戒后不能就这样算了,你必须从这次破戒中获得应有的教训,这样下次才能戒得更好。不断总结,不断学习,觉悟就会提升很快,离戒色成功就不再遥远了。我们戒色只有两种结果,要么你降伏心魔,要么心魔废掉你!!!

再来讲讲戒色吧的宣传,一个贴吧要不断壮大是离不开宣传的,现在戒色吧较之过去发展更快了,当然离不开广大戒友的自觉宣传。在戒色吧成为一个觉悟者后,很多戒友都会出去宣传,在自己的活动圈子里宣传,比如有的戒友在猫扑、天涯或者人人上宣传,你这样宣传其实就是在行善积德了,强化自己正能量的同时,也是在改变别人的命运。不少来到戒色吧的戒友,都会默默感谢“那个人”,那个人其实就是在别的吧或者论坛宣传戒色吧的人,宣传很可能会引来误解,但同时也会拯救一批人,能被点醒的人都是善根深厚之人,骂你或者不理解你的人,也不要去争辩什么,点到为止,清者自清,他将来会悔悟的。有些戒友宣传手法更绝,只发帖宣传,从不跟帖,这样就避免了和别人争辩。善帖是发给有缘人看的,无缘人看了也不明白。\textit{大德大善能遇之,无德无善不明白。(宣化上人)} 不明白也就算了,实际宣传中被骂也是常有的事情。我们宣传也要讲究策略,要学会避免不必要的争吵,以免乱了自己的心性。前任吧主玛雅人的宣传方法也不错,就是贴宣传单,你可以在生活中贴,也可以在论坛里贴,我觉得大学的布告栏是个不错的选择,你打印几份贴上去就有望唤醒很多人。这是一件善举,将来会有很多人在心里默默感激你的。

现在冬季,反映症状反复或者身体状态不佳的戒友比较多了,冬季风寒雨雪,气候本来就容易致病,如果再加上破戒或者遗精,身体就容易出现症状反复了。马俊仁曾经说过,藏獒交配一次,有好几个月都缓不过来,其实人也是一样的,特别是冬季泄精就更难恢复了。据我研究,冬季破戒出症状的概率也远远高于其他季节。冬季草木凋零、冰冻虫伏,是自然界万物闭藏的季节,中医认为,此时人的阳气也要潜藏于内,冬季养生提倡“保持阳气”、“养肾防寒”。冬季是一个伤不起的季节,我建议大家一定要注意保养,养生功法常练,食疗方面也可以加强些。适量运动,尽量不要出大汗,动则升阳,有利于提升阳气。保持心情乐观开朗,喜则升阳,但是不要大喜大笑伤心气。还有就是善则升阳,多做好事多接触积极正面的信息是有利于提升阳气水平的,这就是中医讲的三阳开泰。

最近有个戒友分享了头发恢复的情况,帖子名为《头发变密了,有图为证》,有脱发困扰的戒友可以贴吧内搜索看一下,增加自己的信心。脱发问题我讲过多次,单纯的肾虚型脱发通过坚持戒色养生是有望恢复正常的,如果你还有雄秃或者家族遗传,要恢复难度就更大了。但不管怎样,好好坚持戒色养生,发质是会有所改善的,至少比过去要好很多。

下面步入正题:这季就 YY 终结者、克服恋 X 癖、恐惧症这三个方面详细论述下。

\subsubsection{YY 终结者}

如何断意淫,我 \ref{17} 已经讲得很详细了。但很多人还是无法做到断意淫,对于意淫还是无法自控,和不少戒友的聊天中,他们都会提到“我无法控制自己的念头”,然而要戒色成功就必须要断掉意淫,必须要学会控制意淫,如果你能做好断意淫的思想工作,那么你戒色就成功了一半。你必须时时刻刻看住自己的念头,出现意淫的念头马上就断掉它。凡是成功戒色的人,断意淫都是极快极狠的,完全是压倒式的、铁面无私的。不怕念起,就怕觉迟,有些戒友反应太迟钝,已经意淫十几秒了,才突然发现自己是在意淫,这时候意淫的火苗已经烧起来了,要灭掉它,难度会上升很多。意淫就像火,越早灭掉,越容易控制,否则等它形成燎原之势,那就很难控制了。所以,断意淫贵早、贵快、贵狠!

断意淫的规律:意淫的时间越长,就越难断掉;意淫的时间越短,就越容易断掉。

我推荐的断意淫时间是一秒内,如果你意淫超过一秒了,其实你已经断慢了,断晚了。你必须先知先觉,在它刚露头时,就无情地干掉意淫!

断意淫的三条原则:

\begin{multicols}{3}
    \begin{itemize}
        \item 不犹豫
        \item 不纵容
        \item 不妥协
    \end{itemize}
\end{multicols}

对待意淫的态度就是:零容忍!有你没我,有我没你!

断意淫口诀:\textbf{念起即断,念起不随,念起即觉,觉之即无。}熟背这个断意淫口诀,其实就能打败意淫了,很多戒友都知道这个口诀,但不会用这个口诀,他们还希望有其他断意淫的方法,其实这个口诀就是最好的方法,参透这个口诀,就能彻底降伏意淫。

很多人知道这个口诀却不会用,这季我就把用的方法和大家说一下,拿到这个口诀后,我的建议就是每天背诵几百遍甚至上千遍,有空就背诵这个口诀,要达到什么程度呢?要不断重复直至形成条件反射,意淫出来,不用你思考,直接就断掉了,这个道理就像电脑的杀毒软件,病毒一出来就马上自动杀除了,不要经过大脑思考,完全是自动化的操作。而要达到这个境界,就必须大量重复、重复、再重复,重复到一定程度,你就会发现断意淫原来可以这么轻松,根本无须思想挣扎。我就是这样做到的,我就是这样降伏意淫的。而且随着你断意淫的技术炉火纯青,你会发现自己的意淫也会变少了,因为意淫见到你都怕了。

断意淫口诀就像一门技术,学精这门技术,用好这门技术,断意淫就轻而易举了。所谓,千招会不如一招精,断意淫口诀用到位的人,意淫对于他就不再是难题了,因为他已经可以完全降伏住意淫了。

我推荐戒友每天重复断意淫口诀至少五百遍以上,当你感觉已经形成条件反射了,那火候就差不多了。

说到条件反射,我想很多戒友都应该有点体会的,很多人因为一直接触的是邪淫的内容,结果他对黄毒已经形成了强烈的条件反射,稍微一点刺激都能激起他的自动反应。比如有的戒友对我说,说他看帖子都会漏,看到撸管这个词都会漏,已经到这种程度了。其实他看邪淫内容,就是在不断重复,重复到最后就会形成条件反射,他自己根本无法控制自己。当一个人纵欲形成条件反射了,其实中毒已经很深了,心瘾已经非常重了。要从这种心理怪圈中解脱出来,一定要多发忏悔之心,下大决心去戒色,最后要不断重复输入正气,多学习戒色文章提高觉悟,慢慢地把思想意识转变过来,这样就能戒色成功了。

心魔分两种:

\begin{multicols}{2}
    \begin{itemize}
        \item 意淫的形式
        \item 怂恿的形式,怂恿你去试。
    \end{itemize}
\end{multicols}

对于意淫的形式,很多资深戒友的警惕性还是很高的,但是对于怂恿的形式,很多人就放松警惕了。特别是看了无害论,戒色立场出现动摇,这时候心魔的怂恿就很容易成功。比如有的人会出现这样的念头:撸一次没事的,最后一次。心魔有时候并不会直接诱惑你,而是怂恿你去放纵,像一个教唆犯,真是变着法儿让你破戒。如果你很在意自己的性功能,心魔就会怂恿你去试,如果你有了想试定力的想法,其实就是心魔在诱使你破戒。对于戒色后遇见的各种问题,我们应该有一个正确的认识,然后一定要时刻防着心魔,不管是意淫还是怂恿,都要时刻保持警惕。戒色立场一定要坚定不移!不动如山!不管心魔说什么,我都不听,就是不破戒!就像汉奸来诱使革命烈士投降,革命烈士回给他的就是一口唾沫!我们戒色,就应该有点烈士的气魄和决绝!

其实断意淫口诀不仅对意淫有效,对治其他妄念也是很有效果的,不少戒友虽然基本无意淫了,但是控制不住其他妄念。这种情况,通过这个口诀也是可以控制自己的念头的。戒色就是控制与被控制,被心魔控制永远无法成功,你必须要让自己成为一个掌控者,去牢牢控制自己的念头!看住自己的念头!戒色的每一天都应该如履薄冰,不可大意和放松警惕!

大家应该重视断意淫口诀,用好断意淫口诀,不断重复直至形成条件反射!条件反射是需要不断强化的,如果一段时间不强化,它可能就会产生消退。所以,我们应该时常背诵这个口诀,让它保持在条件反射的程度。

这个口诀是个宝贝,希望你是一个识宝的人。千万不要手捧金饭碗去讨饭,把钻石当玻璃了。

\subsubsection{恋 X 癖}

下面谈下恋 X 癖。

有戒友希望我谈下恋 X 癖,说他破戒都是因为恋 X 癖。比如恋女人的东西,或者恋女人某一个部位等等。其实达到恋癖的程度,就是成瘾了,这种倾向同酒瘾、烟瘾、药瘾、毒瘾、网瘾、赌瘾等类似,都是一种成瘾行为,当然撸管本身也是一种成瘾行为。有这方面癖好的戒友,可以用白骨观和不净观对治,把自己拉回正常轨道,否则中毒这么深,是万难戒色成功的。之前有个戒友喜欢搜集 AV,说是搜集了几百 \unit{\giga\byte},后来他都下决心删掉了,因为他知道这些 AV 就是绑在他身上让他沉入地狱的石头,必须要抛弃掉。还有的戒友恋丝很严重,达到不能自拔不能自控的地步,这类戒友戒色也比一般戒友要困难。

其实每个人有自己的喜好倾向,只是有的人轻微,有的人中度,还有的人则是重度迷恋,深陷其中,甚至会搜集或者偷窃女性用品,自己根本无法自控。如果是重度患者,我建议可以去看看心理医生,听听专业的治疗意见,这样更有利于矫正。

恋癖者一定要有矫正自己异常行为的坚定决心和信心,多发忏悔改过之心,加强道德修养,多学习传统文化提升自己的道德水平。主动参加有益身心健康的社交活动,公益活动则更好。不要接触不良刺激,在这个网络时代,不良刺激到处都是,所以一定要学会控制自己的念头,看住自己的念头。断意淫口诀也可以用来控制恋癖的想法,当你能控制自己的念头了,恋癖矫正就不再困难了,就怕控制不住自己的念头,反而被心魔所控制,那样肯定难以矫正过来。

恋癖心理方面的治疗方案,一般包括:

\begin{multicols}{2}
    \begin{description}
        \item [认知疗法,进行系统的性教育] 学习正确的性观念
        \item [厌恶疗法,对标的物反感] 建立厌恶感
        \item [橡皮筋疗法] 给予轻微惩罚
        \item [社交疗法] 改变自己的性格
    \end{description}
\end{multicols}

“刺激 $to$ 反应”,其实我们戒色,就是在修炼自己的反应,面对诱惑刺激,我们的反应不能是喜欢迷恋,而是要深刻意识到邪淫对身心的危害,然后采取一种排斥拒绝或者厌恶的态度,通过不断学习戒色文章,把自己的思想意识转变过来,这样才有望戒色成功。对于恋癖者来说,学习也是矫正的前提,只有深刻意识到了危害,才有可能发大决心去戒。

有戒友可能会担心厌恶疗法会影响婚后夫妻生活,会担心将来对女人没兴趣,出现性冷淡,影响夫妻感情。其实这大可不必担心,这只是一种治疗的手段,是让你回归正常的轨道。最终的目标就是让你建立正确的性观念,学会控制自己的欲望,学会掌控自己的念头。

\subsubsection{恐惧症}

最后来谈下恐惧症。

看了无数戒友的案例,其中很多人都出现胆小的倾向,这其实和肾虚是密切相关的。严重的戒友甚至会出现恐惧症,比如社恐,恐艾,幽闭恐惧,广场恐惧等。还有的戒友怕光,像活在阴暗角落的可怜虫子。很多戒友感叹自己以前不是这样的,以前的自己很阳光,很有底气,自从沉迷撸管,就一步步变得胆小怕事了。

\textit{人有五脏化五气,以生喜怒忧思恐。(《黄帝内经》)} 情志与五脏的对应关系是:心在志为喜,肝在志为怒,脾在志为思,肺在志为忧,肾在志为恐。肾主恐,恐反过来又伤肾。肾虚的人出现胆小是比较普遍的现象,曾经的我也是这样的,出现过严重的社恐,还有很强的疑病倾向。那段日子真是暗无天日的感觉,自己莫名其妙就恐惧了。现在想想觉得有点不可思议,后来通过戒色养生,底气回来后,我一点也不怕了,感觉完全不一样了。上次回复一个戒友的帖子,他的经历也是如此,戒色一段时间,马上就雄赳赳气昂昂了,底气回来了,冲劲回来了,以前的猥琐胆小一扫而空。

出现恐惧症,你去看西医,一般按神经症和心理疾病治疗,当然也是有一定效果的。如果你去看中医,就是按内脏功能失调来治疗的,就是通过中药调理你的内脏,然后恐惧感就会消除。应该这样说,中西医治疗各有千秋,但我们作为患者,一定要认识到戒色养生的重要性,否则你一边治疗一边撸管,再好的医生也难看好。如果你不是很严重的患者,通过半年以上的戒色养生,完全是有望不药而愈的。肾气足,万邪熄。不过我要提醒大家的是,在恢复过程中是容易出现症状反复的,所以我们应该注意养生,尽量减少症状反复的困扰。

出现恐惧症,很多患者就迷茫了,到底是心理出现问题,还是身体出现问题了,如果你懂了养生方面的知识,其实就知道了,是先肾虚,然后人就会出现胆小的倾向,严重的患者会出现恐惧症。当你通过戒色养生和积极治疗,底气回来后,自然就不恐惧了。如果这个道理你想不明白,那就像进入了迷宫,患者很可能会这样想:我到底哪里出了问题,真的是我心理的问题吗?既然是心理的问题,为什么一直治不好?他从来没想过恐惧症和撸管有关,因为他不懂医理。所以恐惧症要痊愈,也是必须先明理,明白道理,找对方向,再努力就能成功。否则一直想不明白怎么回事,一辈子都在恐惧症的迷宫里徘徊挣扎,更可悲的是,家人根本无法理解他的恐惧,都以为他心理有问题。其实,问题的根源就在于撸管上。撸管导致五脏功能失调,继而就会出现情绪心理方面的改变,肾虚不只是躯体有症状,其实很多人都忽略了肾虚的心理症状,肾一虚,人容易变得急躁易怒,也极容易出现胆小的倾向。明白了这个道理,恐惧症要恢复就好办了,一句话:反其道行之!伤肾的事情不要去做了,撸管肯定要坚决戒掉,然后不要熬夜久坐,应该积极锻炼,养生功法也可以练起来,总之就是养生的事情做加法,伤肾的事情归零,这样坚持半年以上,恐惧症就会痊愈。我就是明白了道理以后痊愈的,那时候我吃过很多药物,都没用,只能暂时缓解下,吃一段时间还容易吃耐药,所以我把恢复的重心放在了戒色和养生上,特别是养生上。站桩和打坐,八段锦,六字诀我都有修炼,然后十一点前上床睡觉,久坐则是每四十分钟起来活动一下,尽量避免久坐。病去如抽丝,而我是一个明理的坚定者,所以最终我痊愈了。只要你能明理,有戒色养生的意识,并且有恒心,那就能痊愈。

不要以为恐惧症好不了,我就痊愈了,我把痊愈经验传授给病友,只要你们真正明白了,也是可以痊愈的。要看到希望,多给自己积极正面的暗示,只要你坚持下去,症状会逐渐消失的。那种阳光自信的感觉会重新注入你的身体,你会找回遗失的美好!

这季继续推荐三本书:

\begin{book}[《气场的秘密》]
    这本书让我了解了什么是气场以及气场的奥秘。气场的确是存在的,但它非常微妙。通过戒色养生,无疑可以增强一个人的气场,气场强大,人的命运也会随之发生改变。而沉迷撸管则会削弱一个人的气场,让人感觉污浊、猥琐、无力、不阳光不健康。气场的感觉在第一眼就可以给对方留下深刻印象,气场有强度大小,有颜色之分,也有清浊之分。通过学习这本书,可以让你懂得气场的奥秘,从而更好维护和增强自己的气场。气场和健康是密切相关的,气场和行为也是密切相关,有什么样的行为,就有什么样的气场。你撸管,就会有一个撸管者的猥琐气场;你戒色,就会有一个戒色者的正气气场。我建议大家可以读读这本书,这本书虽然不厚,但很有启示意义。
\end{book}

\begin{book}[《遵生八笺》]
    中国古代养生学的集大成之作,中国名士养生第一经典,明代高濂撰。我买的是白话文版本的,这本书的养生知识非常丰富,当初读这本书时,的确让我开了眼界,老祖宗在养生保精方面已经达到了极高的境界,可以说是几千年的经验积累,我们应该多汲取前人的经验,学会养生之道。只戒不养的戒友恢复很慢,也容易反复。懂得养生是异常关键的,我一直很强调养生的重要性。我现在读的养生类书籍已经很多了,包括一些专业的中医书籍,基本每本书里都有强调戒色保精的重要性,如果你深谙中医养生之道,对于戒色是非常有帮助的,是戒色的一大助力!
\end{book}

\begin{book}[《张锡纯医案》]
    张锡纯,中西医汇通学派的代表人物之一,近现代中国中医学界的医学泰斗。这本书我是非常喜欢的,一个人纵欲会得很多疾病,但致病因素是非常多的,七情致病,六欲伤身,“风、寒、暑、湿、燥、火”这六因也会导致人生病,还有劳累,饮食方面的因素也会导致人生病,多看医案,有利于提高自己在中医养生方面的觉悟,当你有了很强的养生觉悟,那么得病的概率就会下降。养生是一门学问,保养好自己,也可以把知识分享给家人和朋友。
\end{book}
