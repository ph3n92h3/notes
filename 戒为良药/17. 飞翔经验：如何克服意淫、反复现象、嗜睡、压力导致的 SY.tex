\subsection{如何克服意淫、反复现象、嗜睡、压力导致的 SY}\label{17}

\subsubsection{如何克服意淫}

意淫是大家开始戒色后普遍会遇到的难题,很多人可以不 SY,但是意淫就像脱缰的野马,几乎每几分钟就意淫一次,在中医来讲:心动则精自走。意淫属于暗耗精气神,也很伤身体,所以必须克服意淫。如果你满脑子意淫,总憋着不 SY,也容易出问题。要杜绝意淫,必须深刻地认识到意淫是什么,YY 其实就是念头,邪淫的念头。这个念头像一样东西,这个东西就是“火”!

有了这个认识,大家的觉悟就会上一个层次。接下来,大家再来思考一个问题:火在什么时候比较容易扑灭?有点常识的人都知道,火在刚开始起来时最容易扑灭,几乎不需要多大力气就可以扑灭。否则,当火星变成了大火,那根本无法控制,\textit{毛主席} 曾经说过:\textit{星星之火,可以燎原。}一旦等它发展壮大了,就不是你可以控制的了,而在刚开始时,你还是可以完全控制的。我们的意淫也是如此,当邪淫的念头刚开始出现时,我们马上断掉,这叫“念起即断”,在它还没成气候时,把它扼杀在摇篮里。否则任其发展壮大,最后就是欲火焚身,极易导致破戒。所以,断意淫是有窍门的,这个诀窍就是断意淫的时机,当意淫刚出现,必须马上断掉,否则它就会“越烧越旺”,必须在它还是火星时灭掉它,不能犹豫,不能妥协,千万不能错过断意淫的最佳时机,切记!

\textbf{断意淫口诀:念起即断,念起不随,念起即觉,觉之即无。} 这个口诀很管用,大家背熟它就可以形成条件反射,一有意淫念头,自动就会断掉。当然,前提是背得滚瓜烂熟,不断重复再重复,深入潜意识,这样就能形成断 YY 的条件反射。我们一定要保持极高的警惕和敏感,不能等到它发展壮大了,再去断,那时候就很难控制了。所谓不怕念起,就怕觉迟,就是这个道理。

下面介绍两种佛教修心的方法。

\begin{itemize}
    \item 一种虚云法师开示过,就是不要去理会妄念,妄念来了,不要去管它,这其实就是念起不随,不要去跟随妄念,跟随妄念就是在强化它,不跟随就是断!很多戒友都是跟着邪念跑了很久,才突然发现自己是在意淫,这时候发现已经晚了,邪念已经起势了,就像小火星已经变成大火了,这时候断意淫就比较被动了,已经失去了最佳时机。
    \item 还有一种叫转念法,当意淫出现时,立刻转成佛号,当然不只是意淫,包括贪嗔痴等妄念,一旦出现,马上转成佛号,一句“阿弥陀佛”就转过来了,这种方法很好,我用得比较多,我本人信佛,这个方法对我很管用,有佛缘的戒友可以尝试下。
\end{itemize}

总之,要戒色成功,意淫关必须过;身体要恢复,频遗关必须过。

戒色的确不是那么简单的,不是说我意志力强,就可以戒掉,戒色要成功,必须多学习戒色文章提高觉悟和定力,这样才有望彻底戒除,一旦停止学习和放松警惕,那就很容易破戒。戒色必须专业,什么叫专业,大家都知道有职业玩家和业余玩家之分,篮球有职业篮球和业余篮球之分,戒色也是如此,必须通过学习让自己戒得更专业更到位。不学习戒色文章,你永远无法变得专业,永远只能处在菜鸟级别,要想彻底戒色成功,真的很难。

真正有觉悟有善根的人,必定是每天坚持学习的人,每天都有所领悟,这样他的觉悟和定力提高得就非常快,我在戒色吧这么久,也的确发现不少戒友进步很快,觉悟很高。而有部分戒友,戒了很长时间还是不行,戒色意识还是很模糊,很多问题还是认识不清,这怎么能够成功?如果你对戒色还有疑惑,立场还不够坚定,那要戒色成功,无异于痴人说梦。只有通过学习改造思想,让思想认识有了飞跃,这样才有可能彻底戒除,永不复手!

\subsubsection{反复现象}

下面谈下反复现象。

有些戒友会问,戒断反应和反复现象有何区别?这两者的共同点都是身体会出现症状反复,但两者还是有所区别的。区别如下:

戒断反应一般在戒色后一个月内会出症状,有些戒友戒色前没事,戒色后就出症状了,这就是戒色后正气有所恢复,潜在的病邪自然就会显现,坚持戒色,症状会逐渐消失的,有些戒友不懂戒断反应,一出现症状就慌了,还以为是禁欲有害,结果就又掉进 SY 陷阱不能自拔。所以,要戒色成功必须建立起正确的认知,否则只有失败。

而反复现象出现的时间跨度就比较大了,有人戒色后三个月出现症状反复,有人是戒色半年出现症状反复,有人一年左右也会出现症状反复,出现症状反复是很正常的,据我研究,很多戒友在遗精后都会出现症状反复,还有就是最近熬夜、劳累、久坐、生气、饮食不节、感冒等原因,都会导致症状反复的出现。这个道理其实很好理解,当你还没养足肾气时,遗精乃至不良生活习惯都会伤到肾气,那就很容易出现反复症状。还有就是季节转换的原因,因为每个季节人体的阳气水平都不同,一般在季节转换时容易出点症状。出现症状反复,不用担心,好好休息几天,按时作息,保持饮食清淡,症状会慢慢消失的。

\subsubsection{嗜睡症状}

再来谈下嗜睡症状。

肾虚的表现有一个名词叫“嗜卧懒动”,不仅是 SY 后,乃至遗精后,人都会出现这种表现,就是变懒了变得爱睡觉,睡不醒。很多戒友因为睡不着,会通过 SY 来让自己产生困倦感,然后会更容易入睡,这种方法其实非常不好,当肾气伤到一定程度,就会出现睡眠障碍,到时候就是你睡前 SY,也不容易睡着了,所以必须改变这种睡前 SY 的习惯,必须及时纠正它,就像一棵树长歪了,必须把它弄正了,不能让它顺着歪劲长。

分享一个案例:

\begin{case}
    我 14 岁时开始有 SY 史,而且很频繁,不过后来知道了 SY 对身体不好就停止了,但是现在晚上经常睡不着,白天头晕脑涨,腰部酸痛,小便发黄,是不是我 SY 过度导致了肾虚啊!我该怎么办我已经有好几年没睡个好觉了!
\end{case}

有睡前 SY 习惯的戒友,必须纠正这一习惯,习惯的力量是非常强大的,但必须学会克服它,否则等待你的就是症状。SY 后嗜卧懒动,其实就是气虚的表现,有人一次不过瘾,连着来两次,这种人将来肯定会出现早泄阳痿倾向,因为“欲不可强、欲强则毁”,必须懂这个道理。

\subsubsection{压力导致的 SY}

最后来谈下压力导致的 SY 问题。

据我研究和总结,一般戒友中普遍会出现两种不良倾向:

\begin{description}
    \item[睡前 SY] 就是通过 SY 来让自己更容易入睡,这种习惯太伤肾气,积累到一定程度,反而睡不着。
    \item[压力导致的 SY] 就是把压力变成 SY 行为,这种情况也极其普遍。
\end{description}

睡前 SY 我上面讲到过了,下面就来谈谈压力导致的 SY。

压力导致的 SY,分为好几部分,有求职压力、学业压力、父母的压力、人际关系压力等,人活在这个世界上,会受到各方面的压力,所以有句话叫:生容易,活容易,生活不容易。

我遇见的戒友,其中有求职失败,或者被老板骂,被父母骂,心中情绪失控,精神压力非常大,在这种情况下寻求 SY 的慰藉,希望通过 SY 来摆脱不良情绪,说实话我以前也经常这样,被骂了或者没考好,晚上肯定 SY,好像 SY 是一个发泄的出口,就是希望 SY 把自己搞累搞空虚,然后什么也不想,一天就过去了。

这种发泄的方法其实是不可取的,因为这种方法对身体的健康是有损害的,虽然是可以起到缓解精神压力的作用,但总的来说是弊大于利,所以必须克服这种不良倾向,生活中有压力可以通过其他方式来疏导,并不一定要通过 SY,比如运动方式就比较好,当然要注意适量运动。我提倡的方法就是注重调心,压力使你的心理失衡,你要学会把它调整到平衡的状态,让自己保持在心平气和的状态,这种调心的能力就是情绪管理的能力,大家可以多看看情商方面的书,对大家更好地管理自己的情绪是很有帮助的。

我咨询的戒友中,很多人破戒就是情绪出了问题,生活中的不如意,生活中的压力,导致暂时的情绪失控,这种心理状态就比较容易破戒了,心里不爽,没处发泄,不如 SY 吧,就这样鬼使神差般地破戒 SY 了。

我写这篇文章,就是让大家能更清楚地认识到情绪管理的重要性,有一个稳定的心理状态,戒色才有望彻底成功,永不复手!
