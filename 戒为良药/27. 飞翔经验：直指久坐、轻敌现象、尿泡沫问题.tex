\subsection{直指久坐、轻敌现象、尿泡沫问题}

\paragraph*{前言}

最近前任大吧传奇玛雅人发了一个帖子,介绍了戒色心得,也梳理了戒色吧发展的历程,他是最早一批戒色吧的元老,为戒色吧做出了卓越的贡献,值得大家尊敬,他的亲身经历对大家戒色是一个很大的助力。关于 75 的论述,让大家明白了为何无害论会如此猖獗的原因,因为背后有人或者组织在专干撒播无害论的事情,无害论很多都是编造出来的,然后好好伪装一番,冠以某国的科学研究,对于无知的新人很有迷惑性。玛雅人关于 75 的论述,可以让大家更加清楚地认识到无害论的真面目,以及背后的推手。

其实无害论早已有之,我记得在 90 年代末时,我看杂志时,就有文章在宣扬无害论。我相信那时的无害论应该是作者的认识误区所致,不像现在的无害论,论调变得越来越夸张,越来越耸人听闻,好像不 SY 就会死一样。90 年代末的时候,家庭电脑还未普及,基本就接触不到有害论,更别提有人传授戒色的经验,那时的我就像井底之蛙,我记得那时我看到一个广告,是卖补肾药物的,上面有一句:肾虚,百病丛生。就这么一句,就是我那十几年唯一接触到的正面信息,而且只是说肾虚,并没有直指 SY。现在是网络时代,更多的青少年有机会通过网络这个平台接触到有害论,应该说,网络时代的青少年是幸运的,但也是不幸的,因为这个时代是黄毒泛滥的时代,90 年代有黄毒,但不像现在如此泛滥,上网到处是黄毒,门户网站也充斥着暗示性的内容,而且上网下视频太容易了。很多青少年都有网瘾,有网瘾的人久坐电脑前,体质本来就容易下降,再加上 SY 的摧残,身体出症状的可能性就太高了。

要戒色成功,必须认清无害论,这其实就是戒色的第一关,认不清无害论,迟早会产生动摇,很多戒友一看无害论,马上就产生了动摇,戒色决心不再那么坚定。所以,我们要远离无害论,远离邪知邪见,适度无害论的最大漏洞就是:它完全忽视了 SY 行为的高度成瘾性。人人都以为自己能做到适度,但我可以明确地告诉你,普通人根本就做不到适度,成瘾了再谈适度是荒谬的,另外就是没有一个人知道度在哪里,大家都知道开水烧到 100 \unit{\degreeCelsius} 会沸腾,而 SY 伤到什么程度才会出症状,这个度只有鬼知道。因为每个人的体质是不同的,有的人一周三次没出症状,但有的人一周一次就出症状了,而那个一周三次没出症状的人,你敢说三年后他不出症状吗?他一周三次,可以保证他三年不出症状,因为他体质好,能够伤三年,但是伤到第四年就会出症状。还有人说,我能控制一周一次,但我要说的是,你能控制 YY 吗?YY 是暗漏,一般 SY 的人,经常会沉迷于 YY,他以为一周一次没事,其实他 YY 漏得更多。就这样,不知不觉间,症状就找上门来了!

我第一次来戒色吧时大概是在一年多以前,当时戒色吧的会员人数只有几百人,现在能接近三万,其实和大家的宣传是分不开的,戒色吧的繁荣是大家共同努力的结果。我和玛雅人没交流过,和土豆倒是交流过不少次,土豆虽然现在不当大吧了,但他还是常驻贴吧的,土豆能坚持到现在很不容易,这要克服很多阻力才能做到的,是要有大愿力才能做到的。戒色吧现在的整体氛围比之过去要好很多了,已经有一批有觉悟有学识的戒友在崛起,在带动戒色吧更好地发展。这在过去是不可想象的,过去很多问题发出来都是没人回答的,所以现在的新人是幸福的,戒色吧的前辈已经为你们铺好路了,你们一进来就能把你们引导到正确的戒色道路上来,可以避免走很多弯路。作为新人,你们要好好珍惜学习戒色文章的机会,让自己戒得更专业。

我那时来戒色吧,发现一个现象,很多帖子都有关于 SY 危害的论述,也有戒友发自己的经历来警 示大家,但是真正深入研究 SY 行为的很少,给出具体指导的也较少,很多疑问戒友都得不到满意的答案,大家应该知道,一有疑惑就容易起退心,一有疑惑也会导致破戒。那时,我因为得焦虑症和神衰,研究神经症已经很久了,经过和上千个神经症患者的聊天,经过对致病原因的大量比对和深入分析,终于得出了一个公式:熬夜 + 纵欲 + 久坐 = 完蛋。只要是男病友,基本都符合这个公式。我的研究是基于大量案例的,并不是凭空猜测的结果,做任何研究都要注意搜集第一手案例资料,然后经过深入分析才能找到其中的规律,我那时聊过的焦虑症患者上千,各个年龄层都有,最大的有五十多岁,最小的十六岁。然后我意识到,纵欲和这个病密切相关,肾为元气之根,脾为生气之源,纵欲伤了肾,久坐既伤肾又伤脾,而熬夜伤精更严重,这三者就像三把斧头砍向生命之树,不出症状才是怪事。接着,我就从研究焦虑症转向研究 SY 行为,因为这两者其实是共通的,现在 SY 戒友也聊了上千,搜集的案例有几千例,基本上什么症状我都见过。

我是过来人,也是痊愈者,曾几何时,我也处在无知幼稚的思想状态,那时的我也以为前列腺炎是无法痊愈的,那时的我也以为脱发了就难以恢复了,但现在我通过坚持戒色养生,这两样我都恢复了。一句话,如果你想痊愈,必须多和痊愈者交流,如果你一直和没痊愈的人交流,就会被灌输无法痊愈的想法。我的主张就是戒色一定要彻底,养生一定要到位,这两点做到了,就有了痊愈的基础,然后再配合积极治疗,很多疾病都是可以痊愈的,如果你这两点做不到,吃再好的药也难以痊愈,即使暂时痊愈了,也极有可能复发。

我注重戒色,更注重养生,一直看我文章的戒友应该会知道,我对养生是极其重视的,因为我发现,戒色并不代表万事大吉,相反,戒色仅仅是开始,很多戒友戒了八个月乃至一年,身体恢复情况都不理想,还是没有多大改善,其实就是不懂养生,没有养生的意识,在其他方面漏得太多,这个问题我在 \ref{12} 讲得比较详细,没看过的戒友可以看看,赶紧把养生知识学起来,这样对于你身体的恢复才会比较有利。再举个案例,有个戒友虽然戒色两年,但频遗关始终无法克服,这样戒了两年身体还是老样子,苦不堪言。所以,戒色后的频遗关和 YY 关必须过,否则对恢复实在太不利了。切记。

下面步入正题。这季就久坐问题、轻敌现象、尿泡沫问题详细论述一下,具体如下。

\subsubsection{久坐问题}

久坐问题我在 \ref{7} 曾经讲到过,现在觉得有必要再强调下,也可以讲得更深入些,曾经看过的一本养生书籍上写到:生病就在一穿一脱之间。意思就是你觉得热,把衣服脱了,然后就着凉了生病了,而我要说的就是:生病也在一站一坐之间。久坐伤人,是温水煮青蛙的方式,刚开始让你觉得很舒服,不知不觉间就开始伤你了。久坐,就这么一坐,再简单不过的姿势,就把肝肾脾给伤了!久坐必久视,否则你久坐在那干嘛?久坐伤肾,压迫膀胱经,久坐伤脾,脾主肌肉,所以久坐也伤运化,久视则伤肝,因为肝开窍于目。很多人对于久坐不以为然,其实中医早就有讲到久坐会折寿,而且现在西医的研究也证实了这一点,西医也有很多关于久坐危害的文章。很多戒友是戒色了,YY 也控制得不错,但是网瘾还是很强,有久坐久视的不良习惯,结果就是戒色半年乃至一年,恢复还是不理想。而且我发现,久坐会产生一种惯性,按照某些人的说法就是,有一双无形的手把自己强按在椅子上,站不起来,欲罢不能,不知不觉间几个小时就过去了。所以我们如果想要更好地恢复,必须深刻认识到久坐的危害:

\begin{itemize}
    \item 肥胖,当摄入的热量大于消耗的热量时,体内的脂肪容易堆积,体重便会上升。肥胖是引发多种慢性病的危险因素。
    \item 颈椎病,人保持长时间坐姿,全身重量压在脊椎骨底端,加上肩膀和颈部长时间不活动,容易引起颈椎僵硬,严重者甚至导致脊椎变形而诱发弓背及骨质增生。
    \item 食欲不振、消化不良,久坐缺乏全身运动,会使胃肠蠕动减弱,消化液分泌减少,日久就会出现食欲不振、消化不良以及饱胀等症状。
    \item 肌肉萎缩,久坐可使体内携氧血液量减少,氧分压降低和携二氧化碳血液量增多,二氧化碳分压升高,引起肌肉酸痛、僵硬、萎缩。
    \item 妇科疾病,许多妇女得宫颈炎疾病并不是因为卫生习惯不好,而是与久坐有关。久坐会妨碍免疫细胞的生成,导致抵抗力下降,加上血液循环不通畅,容易引发宫颈炎等妇科疾病。
    \item 记忆力下降,久坐不动,血液循环减缓,则会导致大脑供血不足,伤神损脑,产生精神厌抑,表现为体倦神疲,精神萎靡,哈欠连天。久坐思虑耗血伤阴,会导致记忆力下降,注意力不集中。
    \item 前列腺炎:调查发现,慢性前列腺炎患者中,办公室职员、开车司机、电脑工作者等尤其是汽车司机(长途汽车司机)占较大比例,并且不易治愈。因为从事这些方面工作的人,要长时间久坐不动。
    \item 精索静脉曲张:研究显示久坐可引起精索,并且有可能会加重精索。
    \item 伤心,久坐不动对心脏工作需求量减少了,但是可致心脏功能降低,引起心肌萎缩,给高血压、冠状动脉栓提供了可乘之机。
    \item 伤筋,久坐不动,会使肌肉松弛,血脉不畅,发生瘀积,容易导致静脉曲张,还容易引起痔疮及坐板疮。因为久坐不动,人的全身重量几乎都压在屁股肌一点上,造成其被“压死”,血不畅通而坏死。
    \item 伤肌,久坐不活动,血液不畅,会使肌肉僵硬、酸痛、萎缩、失去力量和弹性而发生痉挛。一次,一个人打牌通宵不动,当他起身时,竟然跌倒在地。原来久坐不动,两腿僵硬不听使唤。
    \item 伤骨,长期坐着工作的人,由于全身重量压在脊椎骨底端,或坐姿不当,加上肩膀和颈部长时间不活动,会引起上端的四节脊椎骨僵硬,还会导致脊椎骨干燥而诱发多种脊椎病,最常见的是躬背及骨质增生。
    \item 伤脑,久坐不动则大脑供血会不足,当人突然站起时,会感到头晕、眼花甚至欲呕吐等。
    \item 伤神,久坐不动还会产生精神压抑,使人无精打采,倦怠无力,哈欠连天,有时还会引起虚心上火,出现牙、咽疼痛,耳鸣及便秘等症状。另外,久坐还会妨碍免疫细胞的生成等。
    \item 伤肾,久坐不动会压迫位于臀部和大腿部的膀胱经,造成膀胱经气血运行不畅,导致膀胱功能失常,而肾经与膀胱经相表里,这样就会引发肾功能异常,所谓“久坐伤肾”就是这个道理。而肾气不足慢慢就会导致气血双虚,出现皮肤瘙痒、面色苍白或黝黑、失眠多梦、心情烦躁、便秘、经血量少等。而这些问题反映在颜面上会表现为可怕的色斑。色斑的出现其实是身体在告诉我们:它的内部气血发生了瘀堵,即中医所说的气滞血瘀。
    \item 久坐使人的全身血管血容量(外周血容量)减少,心脏功能减退,加重中老年人的心脏病,提前发生动脉硬化、冠心病和高血压等病症。
    \item 久坐使胸腔血液不足,导致人的心、肺功能进一步降低,加重中老年人心脏病和肺系统疾病如肺气肿感染,迁延不愈等。
    \item 久坐还会使人的脑供血不足,导致脑供氧和营养物质减少,加重人体乏力、失眠、记忆力减退并增大患老年性痴呆症的可能性。
    \item 久坐不动会引发全身肌肉酸痛、脖子僵硬和头痛头晕,加重人的腰椎疾病和颈椎疾病。
    \item 久坐容易引起肠胃蠕动减慢,消化腺分泌消化液减少,出现食欲不振等症状,加重人的腹胀、便秘、消化不良等消化系统症状。
    \item 久坐可使直肠附近的静脉丛长期充血,淤血程度加重,从而使人的痔疮加重,导致大便出血、肛裂等症。
    \item 由于有些中老年人经常处于静态且不爱说话,会加速语言功能的衰退,并使大脑反应能力变得迟钝,这与“用进废退”理论是一致的。
    \item 久坐还会导致人的心理压抑,爱发无名之火,精神状态欠佳,对外界兴趣逐渐降低直至全无兴趣。
    \item 长期久坐易患肾结石、胆结石等结石类毛病。
\end{itemize}

这 24 点是我摘抄总结的,基本上久坐的危害都在里面了,有句话叫:久坐能杀人!这句话并不是危言耸听,这种温水煮青蛙的杀人方式,的确是杀人于无形,不知不觉间你的健康状况就被削弱了,而且有的人比较无知,还不知道问题出在哪里。世界卫生组织报告指出,每年有两百多万人因久坐少动而死亡。久坐能坐出病来,久坐会与死神亲密接触,是这样吗?是的,这千真万确!然而,现在坐着的人越来越多,坐的时间过长的人越来越多。有人进行过观察,办公室人员每天上班坐的时间平均为五小时以上,连续坐的时间平均为两小时以上,还有电脑族、开车族……

有的戒友会问,我工作学习没办法啊,每天都要久坐,怎么办呢?我给出的答案就是,每四十分钟起来活动一下,给自己定个电脑闹钟,网上有专门的闹钟软件下载,大家可以下载一个,到时可以提醒你起来活动一下。另外,我可以透露一下我自己的办法,我在家都是站着上网的,就是在桌子上放个椅子,这样我就避免了久坐,然而又出现了一个新问题,那就是久站,中医:久站伤骨,所以我一般站一会就坐下休息会。这样方式我觉得挺适合自己的。大家有兴趣也可以尝试下。

还有的戒友会问,打坐算久坐吗?打坐是一种练功态,的确有高僧大德打坐入定几小时乃至半天,甚至几天的也有。因为是特殊的练功态,所以一般是无害的,我推荐是每天打坐一小时,你也可以根据自己的情况来安排,如果你境界达到了,打坐入定几小时也是没问题的。当然我也看到过打坐时间久了有害的文章,那篇文章讲的是双盘打坐时间久了,到老了容易出现腿脚方面的毛病。写那篇文章的是道教方面的知名人物,阅历深厚,他的意见大家可以参考下。我现在一般以散盘和单盘为主。

久坐这个问题怎么强调都不过分,希望大家能深刻认识到这个问题。戒色后,我是主张积极锻炼的,但运动要注意适量,并不是越多越好。实践也证明:经常做有氧运动的戒友恢复相对比较快,也比较理想。

\subsubsection{轻敌现象}

下面来讲下轻敌现象。

轻敌的帖子在戒色吧里时有出现,比如发出来的论调是:戒色很简单嘛,戒色很容易嘛。一般发这样帖子的戒友,都处在欲望休眠期,看过我 \ref{4} 文章的戒友应该会知道,欲望休眠期过后就是破戒高峰期,很多戒友在欲望休眠期还是能控制心瘾的,欲望也不重,YY 也不多,有的人欲望休眠期只有几天,有的人欲望休眠期有一个月左右,在欲望休眠期内,很多人容易滋生一种轻敌的情绪,就是觉得戒色有什么难的,现在我不是戒得很好嘛,发出来的帖子也是这种论调。

如果大家很细心,一定会发现过不了多久,发出来的帖子变成了这样:我太自信了,破戒了。我太大意了,破戒了。从刚开始的轻敌变成了破戒后的懊悔。所谓:骄兵必败。毛主席说过,战略上藐视敌人,战术上重视敌人。我们战略上可以藐视心魔,但战术上一定要重视。如果戒色那么简单,那怎么会有那么多人反复破戒?而且很多人都是研究生,都是高智商的人群。套用一句电影台词:你把戒色想简单了!要戒色成功必须专业系统地学习戒色知识,不断提高戒色觉悟和思想意识,这样才有望彻底戒除。

\subsubsection{尿泡沫问题}

再来谈下尿泡沫的问题。

尿泡沫的问题,很多戒友都问到过,很多戒友都经历过,尿液中泡沫的形成,主要与尿液液体的表面张力有关。一般来说,液体表面张力越高,越容易形成泡沫。在各种情况下,尿液中的各种成分发生变化,如蛋白、黏液和有机物质增多,就可使尿液的表面张力增加,尿液中容易出现泡沫。而尿液中泡沫多,到底是不是病?则需视不同情况进行具体分析。

尿泡沫现象,我们也要分清是生理性的,还是病理性的。

如果属于下列情况,那么尿液中泡沫增多并不属于病变:

\begin{description}
    \item[尿道中存在精液成分] 男性如果尿道中存在精液成分,可以引起泡沫尿。如逆行射精(常见于糖尿病同时伴有自主神经功能紊乱症患者);经常性兴奋使尿道球腺分泌的黏液增多、遗精后等。
    \item[排尿过急] 排尿过急时尿液强力冲击液面,空气和尿液混合在一起,容易形成泡沫,但较易消散。此外,排尿时站得过高,在重力作用下,尿液对液面的冲击力较大,也容易形成泡沫。
    \item[尿液浓缩] 在饮水过少、出汗过多、腹泻等情况下,人体因水分不足引起尿液浓缩,造成尿液中蛋白及其他成分浓度较高,容易形成尿中泡沫增多。
    \item[其他原因] 便池中的消毒剂或去垢剂也是使尿液形成泡沫的原因之一。
\end{description}

偶尔出现的泡沫尿,多半是生理性的,大多能找到引起泡沫尿的诱因,如排尿过急、尿液浓缩等。如果去除上述诱因后泡沫尿消失,且没有伴随其他异常症状或疾病,则无需太担心。

以下情况则要警惕:

\begin{description}
    \item[蛋白尿] 尿液中蛋白含量异常升高是引起泡沫尿最常见的原因之一,也是各种疾病尤其是肾脏病的重要临床表现。各类原发性肾脏疾病,如各类原发性肾小球肾炎等和各类继发性肾脏损害,如糖尿病、高血压、痛风、肝炎等,均可以导致肾脏损害,尿液中蛋白增加。而多发性骨髓瘤、急性血管内溶血、白血病等,虽然肾功能正常,但因血液中出现大量异常蛋白,尿液中也有蛋白漏出,形成蛋白尿。有蛋白尿时一般都会出现泡沫尿,这种泡沫尿的特征是尿液表面漂浮着一层细小的泡沫,久不消失。
    \item[泌尿系统感染] 泌尿系统感染可以引起尿液中泡沫增多。常见的包括尿路感染、膀胱炎、前列腺炎等,大多同时伴随有尿频、尿急、尿痛等症状。
    \item[尿糖增多] 尿液中的有机物质(葡萄糖)和无机物质(各种矿物盐类),也可以使尿液张力增强而出现泡沫,但这种泡沫一般较大,且很快消失。糖尿病病人因血糖升高,继发尿糖升高,容易产生泡沫尿。
\end{description}

我自己的经历,我也出现过尿泡沫,一般泡沫久久不消失,考虑身体内有炎症的可能性,比如前列腺炎。当然,如果蛋白质摄入过多,也有出现尿泡沫的情况。如果偶尔出现一次,就不用太担心,如果尿泡沫的情况一直持续,或者经常出现,那最好去医院做个检查。

\paragraph*{后记}

戒色后如何克服频遗关的确是戒友们比较关心的问题,大家都想尽量减少遗精次数,我当年也和大家一样,在网上搜索了无数的防遗的方法,经过我自己的切身体验,我最后还是选择了八段锦里的固肾功,这个功法有中医的理论支持,而且经过了无数人的验证,可靠性比较高,固肾功不仅对遗精有效,对于滴白问题也同样有效。我在第三季介绍了固肾功,经过很多戒友的反馈,很多人的确做到了减少遗精次数,把遗精频率控制在一月一次,这样的遗精频率对于恢复是比较有利的,当然还有不少戒友做了固肾功,但遗精照旧,一方面考虑是动作没做到位,另外就是其他导致遗精的因素没有注意避免。如果其他导致遗精的因素没有注意避免,那么即使做了固肾功还是有可能会遗精。比如有的戒友做了固肾功,但是白天运动过量导致劳累,这样晚上还是会遗精。还有的戒友 YY 关克服不了,遗精问题还是无法解决。

最近吧里有很多戒友分享了减少遗精的方法,对待这些方法我们可以去尝试,但请记住了,不要把全部希望都寄托在一种功法上,因为如果你不注意避免其他导致遗精的因素,那么依然有可能会遗精的,关于导致遗精的其他因素我专门写过一季,大家可以参照着看看。
