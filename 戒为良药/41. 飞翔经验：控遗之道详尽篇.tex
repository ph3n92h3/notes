\subsection{控遗之道详尽篇}\label{41}

\paragraph*{前言}

戒色吧不少戒友都会分享自己的撸管经历,分享经历很好,能够给别人以警示和启迪。但分享放纵经历时,要注意避免涉及细节,因为那些敏感的文字容易让人产生邪念,从而导致破戒的发生,有的戒友看着看着就漏了,有的戒友看了以后久久不能平静,因为勾起了他自己的放纵回忆。所以我们在分享自己的经历时,要淡化放纵的具体过程和细节,应该突出撸管的危害。这样既可以起到警示的作用,又不至于让人看了产生邪念。

寒假和过年放假是破戒高发期,吧里的破戒帖子明显增多了,有的戒友甚至一天一次或者一天多次破戒,搞得自己很无奈,发帖叫救命,他自己就像失控的飞机,知道要坠毁,但自己就是无法控制心魔,无法克服撸瘾。想当年我做学生党的时候也是如此,平时上学还好,一到周末或者放假就会连续破戒,根本就不是心魔的对手。心魔最喜欢在你无聊时出来考验你,如果你走的是强戒路线,那肯定完蛋,因为强戒之人,觉悟并没有实质的提升,注定是失败!\textbf{戒色成功 = 觉悟高 + 警惕强}。这就是我总结的成功公式,如果你一味强戒,那么等待你的只有失败,破戒后不要灰心丧气,关键是要从破戒中汲取经验教训,这样下次才能戒得更好。有的戒友破戒后从来不反省,也不忏悔,也没有总结经验教训,这样只会越戒越差,越戒越没有信心。冬季破戒的恶果是加倍的,因为冬藏精,冬季收藏得不好就会影响到来年的身体状态,而且冬季破戒出症状的概率也是相当高的。大家如果在冬季不慎破戒了,也一定要注意避免连续破戒,连续破戒太伤了。只有觉悟提高才有望战胜心魔,大家每天都要提醒自己:一定要千方百计提高自己的觉悟!提高觉悟就像练级,级别到了才能和心魔大 BOSS 抗衡。否则免谈!只有被虐的份。

有好几位戒友问我,到哪去找戒色文章,想让我每期推荐些戒色文章。我当初戒色是从精品文章开始看的,所以我推荐初戒者应该把精品分类的文章多看几遍,多做笔记,多思考多总结,然后要多复习,温故而知新,这样觉悟就会慢慢上去了。如果你有很强的学习能力和悟性,那么你觉悟的提升会很快的。对于好的戒色文章应该反复看,并且尝试记忆重点句子和段落,这点也非常重要。对于前辈的经验帖,应该更加重视,因为这是前辈实战的总结,并不是泛泛而谈,有很强的针对性和指导意义。在看戒色文章的同时,也可以多看书籍来提升觉悟,我现在基本都是看书提升觉悟了,觉悟提升才是戒色的王道。

上季我有讲到戒色成功的时间标准,我说的是半年或者一年,有戒友就说某戒色平台的标准是三年算成功,半年是不是太短了。其实半年还是三年,两者并不矛盾,三年标准是比较保险也比较稳妥的说法。而我说的半年标准其实是对上根人说的,中根人则需要一年左右,下根人可能需要两年以上。打个比方,初中三年数学,一般人要花三年才能学完,而有些上根人,就是天资聪慧异常之人,可能半年就能把三年的数学全部学完,甚至他还会觉得“吃不饱”,还想学高中乃至大学的数学,这其实就是根器的差别。

有些戒友提高觉悟像爬楼梯,而有的戒友提高觉悟是坐电梯,觉悟提升超快,常驻戒色吧的资深戒友应该会注意到此类上根戒友的存在。戒色吧就像个大班级,总有冒尖的学生,也总有悟性较差的学生,但只要你坚持学习,总是会成功的,就像龟兔赛跑一样,不要放弃,即使是下根人也是可以抵达终点的。

戒色成功的最关键标志就是:进入戒色稳定期,基本无意淫!

戒色稳定期是很多戒友梦寐以求的终极戒色状态,就像金字塔的塔尖一般,当你进入戒色稳定期了,基本就可以算成功了,只要你能保持足够的警惕守住戒色成果,那就可以算彻底成功了。上根戒友在半年左右即可进入戒色稳定期,戒色稳定期的最大特点就是:基本无意淫!甚至十几天都不会有一个邪念,处在非常稳定的戒色状态。这时候煎熬感会消失,根本就不会在意天数了。进入戒色稳定期后,就不一定要天天学戒色文章了,但应该时常看看,然后要注意保持警惕意识。很多资深戒友进入戒色稳定期后,意淫极少,这时候他就以为不会再破戒了,从而放松了警惕意识,殊不知放松警惕就会导致破戒,不管戒多长时间,失去警惕就会被心魔钻空子。所以,警惕意识怎么强调都不为过。特别要警惕心魔的怂恿,怂恿你去试定力和试性功能。

不断学习提高觉悟,觉悟提高到一定程度,自然就能进入戒色稳定期,所谓水到渠成。之所以你能进入戒色稳定期,其实就是因为你的觉悟已经强大到可以降伏心魔了,心魔就像野马,当你能够驯服这匹野马了,它就会变乖,就会变得稳定不闹事,意淫就会消失或者变得极少。但即使是已经被驯服的野马,偶尔也会发一回脾气,所以降伏后也要每天保持警惕,小心驶得万年船!

天下戒色是一家,大家的观点可能有所不同,但动机是一样的,那就是为了帮助更多的人觉醒,帮助更多的人找回阳光健康的自己。说大点,社会是由人组成的,净化人也就是在净化社会,扬正气促和谐。大家团结一致,戒色大业才能越做越好。

下面步入正题,这季就控遗之道详细论述一下,具体如下:

关于遗精问题,我写过几季,这季再好好谈谈。如果把戒色比作一项运动,可以说是铁人三项。断意淫是其中一个专项,警惕意识也是一个专项,如何控制遗精也是一个专项,情绪管理也是一项,养生意识也是一项,学习意识也是一项。准确地说,应该是“戒色六项”。就像学生党的中考和高考一样,并不是只考一个科目,是考好几个科目,比如 3 + 2 或者 3 + 综合等。这六个戒色专项,可以概括为一个词,那就是:总觉悟!就像高考的总分一样。

这季就控遗这个专项深入谈一下。

身体要更好地恢复,控制遗精频率就显得异常重要了。很多戒友虽然在坚持戒色,但是频遗始终无法克服,结果就是漏垮了。记得有位戒友,戒色大半年,但是他频遗漏了大半年,以前还没有精索,漏了大半年,然后再去医院检查就有精索了。中医:久遗八脉皆伤!频遗是非常伤身体的,很不利于身体的恢复,频遗也很容易出现症状的反复或者加重。所以我们一定要严格控制遗精频率,控遗是可以单独分开来作为一门学问来研究的,戒友奥斯汀就在专攻频遗问题。我曾经也被频遗问题所困扰过,刚开始戒色我并没有出现遗精问题,因为以前透支严重,所以戒了很久才恢复正常的遗精频率。有的戒友因为从发育期就开始撸管,所以从来没有出现过遗精,他根本就不知道遗精是什么情况。还有的戒友体质虚弱,在撸管的同时也会出现遗精困扰,简直就是双重摧残,双斧伐木,没多久身体就吃不消了,症状百出。

我那时为了控制遗精频率,也费了不少功夫,天天在网上查资料,查了几百页的资料,我那时研究的重点就是两个方面,一,是什么原因导致了遗精;二,有什么方法可以控制遗精。我那时有做遗精记录,每月几号遗精,几点遗精,遗精间隔,遗精时的具体细节都有详细记录,然后我就研究是什么导致了遗精,把可能的原因想出来,然后上网查,看看是否有案例和我一样的情况。然后就把一个个原因给找出来了,导致遗精的原因很多,必须要学会避免这些遗精诱因,这是非常关键的。这季再具体罗列一下:

导致遗精的因素如下:

\begin{multicols}{3}
    \begin{itemize}
        \item 白天意淫
        \item 白天劳累
        \item 喝酒
        \item 吃肉太多
        \item 趴着睡
        \item 裸睡
        \item 晒被子
        \item 盖太厚太重
        \item 睡前打坐
        \item 内裤太紧
        \item 顶着或者夹着被子
        \item 艾灸不当
        \item 打坐意守下丹田
        \item 熬夜久坐
        \item 睡前喝水太多
        \item 运动过度
        \item 生病
        \item 肾亏无梦而遗(滑精)
        \item 饮食偏辣偏重
        \item 紧张(包括梦魇)
        \item 挤压(包括趴睡)
        \item 生气(导致气血紊乱)
        \item 受凉(包括吃凉食)
        \item 按摩穴位不当
        \item 炎症导致的遗精(属于中医肾亏范畴)
        \item 还阳卧(有不少遗精的反馈)
        \item 睡前泡脚(水过热)
        \item 憋尿(易致遗精)
        \item 睡时使用电热毯或取暖器
        \item 睡前有过剧烈的运动
        \item 包茎包皮过长
        \item 思虑过度
        \item 回笼觉
        \item 失眠再睡
        \item 大量出汗
        \item 吃中药(特别是不对症)
        \item 长时间做足疗
        \item 黑芝麻吃多了
    \end{itemize}
\end{multicols}

我 \ref{16} 总结了 23 条导致遗精的因素,这季添加至 39 条。其中很多都是我自己亲身体验,并且进行案例调查得出的结论。还有一些是根据戒友的反馈,然后进行分析调查得出的结论。遗精问题是比较复杂的,也是因人而异的。比如两个人同样喝冷饮,可能一个回家就遗精了,另外一个却没有出现遗精,这就是体质和健康状况的差别。冷伤阳气,但有的人体质还行,还伤得起,所以没出现遗精,但有的人体质较差或者已经很虚损了,再一吃冷饮,马上就会出现遗精。有的人吃很多肉都没事,但有的人肉吃多点,马上就会出现遗精。还有的人临睡前热水泡脚就会出现遗精,但有的人怎么烫脚都没事。而且我还发现一个事实就是,每年人的体质和健康状况都会发生微妙的变化,可能去年你热水泡脚不遗精,但今年就可能会出现遗精。去年你运动一下午都不会遗精,今年可能稍微运动多点就会出现遗精。另外,遗精和季节也是密切相关的,很多人在春夏遗精很少,到了秋冬遗精就开始频繁了。很多人刚开始戒色,基本很少遗精,但戒到一定程度就会出现频繁遗精,给他带来很大的困扰。我们要更好地恢复,必须有一个稳定的遗精频率,如果一年十二个月,每个月的遗精频率都少而稳定,这对于恢复是很有利的。我罗列的这些导致遗精的因素,很多戒友可能看了印象不深刻,只有当遗精发生后,才可能会加深印象,下次才能真正做到避免。

遗精一般分两类:

\begin{multicols}{2}
    \begin{itemize}
        \item 生理性遗精
        \item 病理性遗精
    \end{itemize}
\end{multicols}

一般一月三次以内算正常,还有的文章说三次左右。如果你一月达到六次、八次了,或者连续几天都遗精,那就不正常了,属于病理性遗精。如果能把遗精频率控制在十五天一次或者二十天一次,已经可以算成功了。如果想更好地恢复,最好能控制在一月以上一次遗精。毕竟很多戒友的身体透支多年,出现了巨大亏空,一次遗精都伤不起了,一次遗精就可能让他难受好几天才能缓过来。这里再说说精满自溢,如果严格地说,是不存在精满自溢的,《圆运动的古中医学》里面就讲到不存在精满自溢一说,之所以有精满自溢,其实是对凡夫说的,凡夫色心一动,元精即化为浊精。修行高人,是可以突破遗精障碍的。而我们生在红尘俗世,诱惑太多,要真正做到“无漏”,难度是很高的。一方面你修心要做到位,另外就是要有明师传授方法。有的戒友会觉得浊精不好,其实浊精是很有营养的,富含多种营养成分,并不是垃圾。所以,浊精也不可随便泄漏,对于浊精我们也要学会爱惜。

戒色吧里大多数的戒友都会遇见遗精的困扰,也许你现在没遇见,明年就遇见了。遗精后的心理调护也异常重要,很多人在遗精后都会出现破戒,因为遗精后容易出现思想动摇,也容易出现不良情绪,所以遗精后是破戒的一个高发期,我们在遗精后一定要注意调整心态,正确看待遗精,并且保持高度警惕,做好情绪管理,这样就能平安度过遗精后的那段时间。

中医一般把病理遗精分以下五类:(西医则归因于炎症表现,如前列腺炎等)

\begin{description}
    \item [心肾不交型] 表现梦中遗精、头晕、倦怠无力。此为心火亢盛、水亏火旺,扰动精室所致。
    \item [肾气不固型] 表现滑精频繁,面色少华,畏寒肢冷。此为病久不愈,阳精内涸,气失所摄,精关不固。
    \item [肾阴亏虚型] 表现为遗精、头昏目眩、耳鸣腰酸、失眠盗汗。此为房事过度,耗伤肾阴,干扰精室,致使封藏失职。
    \item [肝火偏旺型] 表现为梦中遗精,烦躁易怒,胸胁不适,口苦咽干。此为肝火旺盛、火扰精室所致。
    \item [湿热下注型] 表现为遗精频繁,或在排尿时精液外流,口苦或渴。此为湿热下注,扰动精室。
\end{description}

古人认为:“遗精不离肾病,但亦当责之于心君。”\textit{有用心过度,心不摄肾,以致失精者;有因思色欲不遂,精色失位,精液而出者。(明代医家戴元礼《证治备要遗精篇》)} 时至清代,对遗精指出“有梦为心病,无梦为肾病。梦之遗者,谓之梦遗;不梦而遗者,谓之滑精”。又将遗精分为梦遗和滑精,后世医家多沿用至今。

很多戒友最后都遗怕了,甚至害怕睡觉,而有的戒友则是经常滑精,这类戒友应该去找个好中医调理一下,药物调理也是控制遗精频率的一个有效途径。被频遗深深困扰的戒友,应该积极尝试药物治疗,特别要找一个好中医,现在庸医多,开的很多药都不对症,吃下去反而会加重遗精,记得有戒友吃了中药后,两天遗精四次,像泄了气的皮球。另外,很多戒友喜欢自己找药吃,或者听某某人推荐吃哪种药,他就去买来吃,这样吃药很可能不对症,因为适合别人的药不一定适合你,比如有戒友听说吃知柏地黄丸好,就去买来吃,结果反而加重了遗精。所以,有遗精困扰的戒友,如果你想吃药,最好去看中医,具体诊断后再对症下药,这样稳妥些。中医讲究辨证治疗,不是说遗精吃一种药就全部管用,吃得不对只会起到相反的效果。有的戒友吃金锁固精丸效果不错,因为比较对症,但有的戒友就不适合吃了,要看具体病情才能开药,还有的戒友是吃了封髓丹控制了遗精频率。

下面我推荐几种控制遗精频率的方法:

\begin{multicols}{3}
    \begin{itemize}
        \item 固肾功
        \item 睡姿吉祥卧
        \item 缩肛功
        \item 仰卧起坐或者仰卧举腿
        \item 铁板桥
        \item 站桩
    \end{itemize}
\end{multicols}

固肾功就不多说了,我 \ref{3} 的文章讲得比较详细,还拍过一个视频分享给大家。如果你把固肾功做到位,感觉找对,是可以做到有效减少遗精频率的。记得有个戒友原来一月梦遗八次,坚持做固肾功,就减少为一月一次,他一下又恢复了活力和自信,底气又回来了。当然,杜绝其他导致遗精的各类因素也很重要,特别是我上面罗列的各种因素。

睡姿吉祥卧也是非常重要的,固肾功加吉祥卧,就相当于“双保险”了。我看到有戒友在推荐婴儿卧,其实婴儿卧和吉祥卧类似,只不过吉祥卧是右侧,并且对手的位置有要求。吉祥卧刚开始容易产生不适应,一个是手垫着不舒服,还有就是腿一直弯曲憋得慌,我刚开始也是感觉不适应,坚持一个月,慢慢就适应了,现在我睡吉祥卧已经很舒服了。关于吉祥卧的原理,我想和大家分享一下,因为很多人都知道吉祥卧很好,但真正知道背后原理的人却不多,吉祥卧是把人体折叠,这样一折叠就相当于关闭了下漏的通道,我打个比方大家就理解了,好比一根吸管,你在上面灌水,如果吸管下面呈折叠状,水就不易漏失,具体折叠的角度,我推荐小于或者等于 90 \unit{\degree},如果是 135 \unit{\degree},那还是很容易漏失。我自己睡觉一般都 90 \unit{\degree},小于 90 \unit{\degree} 也很好,但难度有点大,不容易做到。古大德都很重视吉祥卧,很多都曾苦练过吉祥卧,甚至把自己绑着定型入睡,他们能克服遗精关,和吉祥卧是有一定关系的。一般坚持吉祥卧一个月,就能慢慢适应了,刚开始肯定会感觉到不大舒服。

有了固肾功和吉祥卧这“双保险”,再加上避免其他导致遗精的各类诱发因素,控制遗精频率就有把握了。

控制遗精频率的功法很多,网上应该不下十种,我自己试下来的最好的动功就是固肾功,当然吉祥卧也很好,不过是属于睡姿静功。另外,缩肛功大家也可以练练,仰卧起坐和仰卧举腿也可以起到固肾腰的作用。铁板桥不少戒友反映也不错,但贵在坚持。站桩也是可以有效减少遗精频率的,有戒友反馈通过站桩他大大减少了遗精频率。

\paragraph*{总结}

我把自己控制遗精的心得和大家分享,希望能给大家带来启示和帮助。频遗关过了,恢复才有望。否则这样频遗,直接可以把一个人漏垮掉,漏出一身的病。当你有了一个少而稳定的遗精频率后,再加上养生之功,身体就会进入恢复的正循环。阳光、自信、底气会重新回到你的身上。人就像一只皮球,是由精气充起来的,精气泄掉了,人就像泄了气的皮球,无精打采,萎靡不振,面容猥琐,甚至面带鬼气。当你重新把精气充足后,那种感觉真的就如猛虎出笼,不怒自威,眼睛有神采,精力旺盛,有很强的自信和底气,不再惧怕任何事物,充满干劲。希望大家都能找回这种积极向上的身心状态。

\paragraph*{后记}

最近有戒友在我帖子里说,他看了适度无害论后思想又动摇了,立场又不坚定了。他想听听我的意见,其实对于适度无害论,我已经在文章里讲过很多遍了,适度无害论是谬论,完全忽略手淫的高度成瘾性,新人看了以后就想“适度”一下,找到了所谓的理论支持,撸得更加心安理得了,其实完全是自己欺骗自己,自己害自己。适度无害论不知害惨了多少人,上瘾了能适度吗?上瘾后能自控吗?

关于性生活的次数,\textit{药王孙思邈} 是这样说的:“\textit{人年二十者四日一泄,三十者八日一泄,四十者十六日一泄,五十者二十日一泄,六十者闭精勿泄,若体力强壮者一月一泄。}”彭鑫博士说的次数就是引用药王孙思邈的这段话,不过这段话是对古人说的,古人结婚比较早。鉴于现代人精气神耗损比较严重,彭鑫博士推荐现代人的性生活次数是一月一次。但大家看到这个次数,千万别以为这是普适的,这种标准是对健康人而言的,很多人以前不知透支过多少次,都已经症状缠身了,如果还想完成每月的“指标”,那无异于是雪上加霜了。很多名中医都建议禁欲半年乃至一年以上,加上中药调理和养生之道,身体才有望彻底痊愈。

还有一点,药王孙思邈还特别说到:“\textit{男不可无女,女不可无男。无女则意动,意动则神劳,神劳则损寿。}”撸管就是单方面的性行为,阴阳不交,其实是很折寿的。所以,我们一定要戒掉撸管恶习,把最好的自己留到结婚后节制使用。让那些把自慰合理化的文章见鬼去吧,不知祸害了多少青少年,青少年没有深厚阅历,也没有分辨力,就这样被那些歪理邪说给迷惑了,最后搞得一身的症状,容貌也变丑变猥琐了,还影响到身体的发育。最后很多戒友都来上这么一句,真是意味深长的一句:我就是被适度无害论给害了!多么讽刺的一句话。

大家如果想彻底戒掉撸管,一定要过“适度关”,否则你的戒色立场不可能坚定,适度这两个字很有诱惑力,但我把这两个字看得很清楚,对于很多戒友来讲,适度其实就意味着一发不可收拾。我绝对不和心魔讨价还价,戒色不是买菜,戒色是很严肃的事情,也是很残酷的事情,因为戒不掉就很有可能废掉,实际情况是很多人已经废掉了才想到要戒色。

大家应该做一个坚定的戒色者,立场要坚定,信心要坚固,否则看了无害论是很容易被迷惑的。孙悟空有一双火眼金睛,像 X 光一样,能看清本质认识到真相。我们戒色也需要一双火眼金睛。认清适度无害论的真面目,远离无害论的文章。无害论就是一种毒药,这种毒药很甜,吃下去很舒服,但药力一发作,就要你好看了。适度无害论也很像酒,喝下去人就糊涂了,分不清对错了,我从来不看无害论,因为我知道那是害人的东西。
