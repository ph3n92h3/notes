\subsection{断念口诀的学、练、悟}

\paragraph{前言}

最近又有戒友向我反馈频遗的问题,本来控制得很好,一个月没遗精,然后出现一次遗精,接着就开始频遗,后来我帮他分析原因,考虑健身劳累所致,当然也不排除其他因素,我分析他的反馈,劳累的可能性比较大,我建议他先不要做力量训练了,可以适当做做有氧运动,注意休养,避免劳累。戒色后一定要学会严格控遗,这是恢复的重中之重,如果出现频遗,身体就很难恢复,甚至会爆发新的症状。我建议大家每次遗精后都要记录下来,做一个遗精的总结,已经有很多戒友在这样做了,就是每次遗精后写下日期,遗精的大致时间,比如三点到四点,还是五点到六点之间,最关键的就是分析遗精的原因,首先要关注的因素就是饮食,比如吃辣、肉吃多了、喝酒、补的东西吃多了、坚果类食物吃多了、葱姜蒜、韭菜等等。过年期间很多戒友都出现了频遗,就是饮食方面的问题,也许过去你的身体对这类因素不太敏感,但戒到一定程度就可能会变得敏感,所以要格外注意。其次要关注的因素:劳累、睡前泡脚、艾灸不当、睡前打坐、回笼觉、吃中药、憋尿、按摩穴位不当、意守下丹田、白天意淫、趴着睡、思虑过度、恐惧过度等。每次遗精后一定要注重分析,把可能的因素写下来,做个记录,下次尽量避免。这个过程就像侦探破案一样,遗精发生后,要仔细回忆之前做了什么,吃了什么,是什么因素导致出现遗精。时间长的遗精可以不分析原因,比如几个月才出现一次的遗精,一般一个月左右出现的遗精,都应该大致分析一下原因。如果排除了外在的因素,那就可能是身体失调或者亏损所致,这是需要看中医调理的,一定要学会克服频遗,一般控遗的方法就是固肾功、吉祥卧和养生桩等,固肾功要做到位,找准拉紧感,但不要出现劳累的感觉。遗精后身体会感觉变差,情绪也会变得低落,邪念也会开始变得活跃,还可能出现症状反复,这时要放平心态,注意情绪管理,注意休养,加强修心,保持警惕,一般休息几天,身体就会逐渐恢复。遗精的频率尽量控制在一月三次以内,最好是一个多月一次,如果能做到几个月一次,那就更好了,控遗是门很深的学问,并不是说行善了就不会频遗了,导致频遗的因素真的有很多,各方面都要注意。戒到一定程度,每一位戒友都可能出现频遗,关键是要学会控遗,及时调整,这样才能稳住身体的状态。

上季有一位戒友执著于体验,因为他看了修行的书,对某些特殊的体验很感兴趣,也想体验一下,于是心里一直想着,很执著,是一种躁妄心,有点走火入魔的感觉。我已经告诫他了,不要去求体验,关键还是心空净,这样水到渠成,自然会有一些体验,但不管什么体验,都不要执著,就像没发生一样,最好也不要说出来,要保密,否则说出来很容易增长我慢,对修行很不利,这点要切记。 \textit{近来修行者,多多着魔,皆由以躁妄心,冀胜境界。勿道其境是魔,即其境的是胜境,一生贪着欢喜等心,则便受损不受益矣。况其境未必的确是胜境乎。倘其人有涵养,无躁妄心,无贪着心,见诸境界,直同未见。既不生欢喜贪着,又不生恐怖惊疑。勿道胜境现有益,即魔境现亦有益。何以故,以不被魔转,即能上进故。须知学道人要识其大者。否则得小益,必受大损。勿道此种境界,即真得五通,尚须置之度外,方可得漏尽通。若一贪着,即难上进,或至退堕,不可不知。(印光法师)}

下面分享一些案例。

\begin{case}
    不知道怎么表达自己的心情,刚看完了前言和飞翔哥收录的案例,已经湿润了眼眶。如果我没有遇到戒色吧,如果我没有学习《戒为良药》,练习断念,我会继续放纵下去,我已经撸了二十年了,几近油尽灯枯,案例中的频死感、肾衰竭一定是我跑不了的!尤其是前言中大家所议论的邪淫让人冷漠,我真的是感同身受!可是命运发生了转折,我遇到了戒色吧,我千百次失败依旧没有放弃,如今我戒色 360 天!一年了啊!回首邪淫的经历,真的有种恍若隔世的感觉,心情真的是五味杂陈。我撸了几近二十年啊,二十年啊,如今戒色一年,可以说变化非常非常明显,但是恢复的路却是一步一个艰难,大部分症状都恢复得不错,但是腰椎间盘突出和飞蚊症却可能是邪淫在我身上留下的永远的伤疤。真的戒色要趁早,别等到器质性病变,那就积重难返了!
    \subparagraph{附评} 这是一位资深戒友看了上季的文章留下的反馈。强大的戒色灯塔照亮沉沦欲海的人,他们在苦苦挣扎,等待别人的唤醒和拯救,遇见戒色吧的确是人生的一大转折,否则很可能在放纵的轨道上越陷越深,不能自拔,最后得上严重疾病,在病榻上苟延残喘。\textit{善摄生者,要当先除六害,然后可以保性命,延驻百年,何者是也?一者薄名利,二者禁声色,三者廉货财,四者损滋味,五者除佞妄,六者去妒忌。去此六者,则修生之道无不成耳。若此六者不除,盖未见其益,虽心希妙理,口念真经,咀嚼英华,呼吸景象,不能补其短促,盖损于其本而妄求其末,深可诫哉。(《太上老君养生诀》)} 古圣先贤已经为我们留下了宝贵的养生知识,其中看淡名利、戒色、饮食清淡、养生功法、不过劳、心态平和、修德是主要的内容。\textit{欲求神仙,唯当得至要,至要者在于宝精行气。(葛洪《抱朴子》)} \textit{精者神也,宝精则神明,神明则长久。(《云笈七签》)} 古人的保精和宝精意识极强,一个保护,一个宝贝,都在强调肾精的重要性,肾精充足,人的感觉马上就不一样,一旦泄掉,就会感觉差很多,即使遗精也会感觉差不少,戒色后要严格控遗,尽量减少耗损,说到底就是能量管理,能量养足了,真的是势不可挡,一派气壮山河的感觉,整个人都被一股气给撑起来了,雄赳赳气昂昂,甚至英雄气概也出来了。在我戒色之后也有恍若隔世的感觉,好像又投胎了一次,再看这个世界感觉清新美好了许多,之前脑子被邪念占据,满脑子污秽不堪的想法,症状爆发后真的很惶恐,很绝望,生不如死。后来戒掉后就像枯木逢春一样,一种久违的生机和活力又回到了我的身上,走路带风,都能蹦起来,像孩子一样轻松、快乐,纯真的感觉真的太美好了。戒色后要加强养生恢复,这位戒友之前撸了近二十年,如今戒色一年,变化虽然明显,但毕竟掏了二十年,还需慢慢回填。腰突要积极治疗,注意保养,飞蚊症我也有,戒色养生后会减轻的,完全恢复有难度,因为这个病和用眼过度也有很大关系,现在这个网络时代看电脑和手机的时间比较长,普遍用眼过度,可以自己控制下,注意保养眼睛。戒色的确要趁早,有的戒友认为要得上重大疾病才可能戒掉,这是不一定的,有的人即使得上严重疾病,也不一定能戒掉,戒掉关键是掌握方法,下决心学习,学会修心。等到得了严重疾病就很被动了,恢复也需要很长时间,所以要趁早戒掉。
\end{case}

\begin{case}
    本人今年 32 岁,在中学担任一名体育老师,工作兢兢业业,29 岁结了婚,夫妻感情开始也不错,但是后来,学校开展大学生来我们学校实习活动,其中有一个挺漂亮的实习老师分到了我们体育组,我们每天工作交流,在办公室里娱乐,很快互相产生了感情,我没有遵守夫妻忠诚与她多次发生约会,渐渐发生了性关系,纸包不住火,事情被我老婆发现了,她坚决要求离婚,我父亲因此从今不让我进家门。在办理离婚那天我浑浑噩噩,没有任何伤心。离婚后,那个实习老师不久被调走了。我又和一个高中女生开始了恋爱,但不久就分手了。现在每天性欲非常强烈,每天放假下班回家都手淫,甚至对所教的一名可爱的小女生有那些想法。现在我忏悔我的邪淫,重新做人,做一名合格的人民教师,希望大家原谅我。
    \subparagraph{附评} 婚外情的案例之前没有讲到过,这季正好根据这个案例讲一下。这位戒友是刚来戒色吧的新人,职业是体育老师,他的问题就是没有把握好同事间交往的分寸,结婚后一定要注重责任感,生活中和其他异性要严格把握好分寸和距离,不可走得过近,这方面要特别注意。他还把歪脑筋动到自己学生头上,这实在愧对“为人师表”这四个字,如果被学生家长知道了,闹到校长那里,那真的要被开除。邪淫的行为让他的生活陷入困境,老婆坚决要求离婚,父亲不让他进家门,一人邪淫,伤害的是两家人,真的不能搞婚外情,否则老婆痛苦,孩子痛苦,双方父母都很痛苦,一个男人最重要的是正气和责任感,这是很多年后我才知道的,男人最重要的不是颜值,而是正气和责任感,有正气和责任感的人是最帅的,对家庭负责的男人是最帅的,这种帅真的是帅到骨子里了,帅得很深沉,很有智慧,很有分量。否则即使和一百个女人发生关系,也是毫无意义的,只会让自己越来越垃圾,越来越空虚、颓废,浑浑噩噩,到处乱搞,沦为纵欲的禽兽,还以为自己欲望强,实则是虚则亢的表现。邪淫的人要倒霉,这是很多戒友的亲身体会,以前我的两个同事都是因为嫖娼被开除的,邪淫后领导就开始看他们不顺眼了,要倒霉了,因为整个人的能量场变差了,变成负面了,会感召和吸引负面的经历和事物。一个男人真的要懂得戒邪淫,也要知道什么是邪淫,古圣先贤一致反对邪淫,是必须要戒掉的。不戒邪淫,人生真的危机四伏,有的明星搞婚外情,潜在损失达几千万乃至上亿,身败名裂。一搞邪淫,你之前建立起来的道貌岸然的高大形象,一瞬间就坍塌了,那些崇高的理想和抱负完全就是在打脸。\textit{邪淫是开启万祸之门的钥匙,你一旦犯了邪淫以后,就开启了万祸之门。这个万祸,它不是一个数字,不是一万种灾祸,可以说是无量灾祸。(秦东魁老师)} 还记得初中一位同学的作文,讲的就是他和妈妈一起回家,正好撞见他父亲和一个女人睡在床上,他妈妈上去就是怒骂和打架,这一幕被幼小的他看在眼里,留下了深深的伤害。婚外情对孩子的伤害是很大的,孩子如果知道父亲在外面乱搞,对家庭不负责任,幼小的心灵是很痛苦的,这种痛苦不应该由孩子来承受,做一个对家庭负责任的父亲显得尤为重要,男人应该要有分寸、底线和原则,家和万事兴,千万不能搞婚外情,也不能嫖娼、约炮、一夜情等,一定要懂得戒邪淫,懂得修心,对治自己的邪念,把家庭经营好,好好孝顺双方父母。一位戒友曾经说过:“我认为一个男人最帅的不是睡过了多少女人,这是最垃圾的行为,不是不报,时候未到,最帅的男人是高度自律的,是懂得尊重自己,尊重别人。把最好的自己留到婚后。”他说得很好,最帅的男人懂得高度自律,对自己负责,也对别人负责,正气凛然,光明磊落,绝不乱来,真正的帅就帅在那股正气和责任感!
\end{case}

\begin{case}
    当年我就是这样,身心一大堆毛病,做什么事都不如意,做什么都不成功。在我快陷入绝望之时,一次偶然的机会,我经过一个念佛堂,在书架上我第一次接触到佛法,心里觉得像一股清泉。我本能地觉得佛法能治愈我的身心疾病,能帮我走出困境。再不回头不行了,我从《印光法师文钞》入手,晚上家人都睡了,我还读得津津有味,并在平时的生活中努力落实,我这样做了两年,感觉症状减轻了,底气足了,敢于正视别人了,同时人缘好了,运气也变好了,我又能融入集体了。后来就在贴吧里遇到了飞翔哥,通过看飞翔哥的文章,我如梦初醒,痛改前非,通过飞翔哥,我又读到了元音老人的文章,同样是身体力行。2018 年我通过了事业单位考试,进入公检法系统,改变了自己的命运。当我通过面试的那一刻,我都不相信是真的,这是在做梦吗?我这样一个笨手笨脚的人能考上,简直不可思议。这个消息也像一枚原子弹,强烈地震撼了我原来工作的单位,一些平时瞧不起我的人,再也不敢轻视我了。在这里,我十分感恩佛法,感恩各位大德引导我走上了正路,我一定会用我所学来造福社会,维护公平正义,传递正能量。同时我也会一如既往地低调行事,决不会因为成为国家工作人员就生出一种傲慢之心。今天简要地写一下我的经历,希望能给大家带来信心。
    \subparagraph{附评} 这位戒友的逆袭很给力,在快陷入绝望之时,也就是强势反弹之时,在绝境中往往蕴含着逆袭的机会,关键就是自己是否能把握住,我相信老天会给每个人这样的机会,就看自己的努力了。这位戒友在困境中接触了佛法的甘露,认真研读大德开示,的确是下了很大的功夫,家人都睡了,他还在认真学习,并且在生活中努力落实,坚持了两年,改变很大。学习是非常重要的,闻思修,闻思是放在前面的,认真学习、思考和领悟,真正理解大德的开示,逐步深入地体会,不断精进,转入转深,认识越来越深刻。我也学习了《印光法师文钞》,非常殊胜的法宝,不仅教人修行的道理,还讲了很多做人的道理,那个朴实慈悲恳切的风格,令我顶礼膜拜。在人生遭遇不顺和困境时,就是一个大转变的时刻,如果还在原来那个放纵堕落的轨道上,是不思悔改的,甚至以邪淫为荣,只有遭受到了痛苦的报应,这时才幡然醒悟,继而有幸遇见善知识的指点,这样就能有一次强势反弹。戒掉邪淫,改过迁善,积功累德,能量场越来越强,越来越正,因为戒色,精力和脑力水平都得到了提升,这样就可以更好地备考。这位戒友进入了公检法系统,一次考试改变了命运,原来他是一个笨手笨脚的人,在单位还被人轻视,这下考上了,没人敢轻视他了。因为学习了大德开示,所以他知道这时不能傲慢,要懂得谦卑和感恩,大家看看他这段文字,就知道他的德行真的上来了,普通人很可能骄傲自满,得意忘形,而他知道要谦虚、要感恩、要低调,这就是有德行的表现,他的德行与他的位子是匹配的,这样位子就坐得稳,将来前途不可限量。原来他身心一大堆毛病,做什么事都不如意,做什么都不成功,那种状态真的很糟糕,身心脑力都不在状态,很多事情都做不好,也没那个精力去做了,很容易半途而废。现在他彻底逆袭了,这个案例真的非常励志,一个人的正气实在太重要了,一犯邪淫,整个人的能量场就会大幅下降,这时候就开始走霉运了,如果懂得戒邪淫,积累正能量,那命运必定会随之改变。\textit{汝年尚幼,须极力注意于保身。当详看安士书中欲海回狂,及寿康宝鉴。多有少年情欲念起,遂致手淫,此事伤身极大,切不可犯。犯则戕贼自身,污浊自心。将有用之身体,作少亡,或孱弱无所树立之废人。又要日日省察身心过愆,庶不至自害自戕。否则父母不说,师长不说,燕朋相诲以成其恶,其危也,甚于临深履薄。……慧佐之死,乃其父母祖母所致。其家生此聪颖之子,不告以保身寡欲之道,乃早为娶妻。又不说节欲之益,纵欲之祸。彼二青年只知求乐,不知速死。及已经得病,尚不令其妻归宁。以致年余大病,以至于死。将死见其妻,尚动念,故咬指以伏欲心耳。(印光法师)} 印光法师也提倡戒手淫,因为“此事伤身极大”,又告诫婚后要懂得保身寡欲之道,千万不可纵欲。一旦得病,就要禁绝房事,否则身体很难恢复,甚至会越来越差,大病期间行房事,有可能会导致死亡,得病期间一定要和妻子沟通好,等身体彻底恢复后才能行房事,这点一定要格外注意。
\end{case}

\begin{case}
    这两天天气很好,我也抽空来到公园做树疗了,这是棵合抱的银杏树,而且这里下午来的人很多,但是我没有管他们,我戴着耳机听着喜马拉雅电台边听边做,每次一个小时左右,我感觉树疗最大的功效就是让我浮躁混乱的内心世界平静了下来,在与树拥抱的时候我可以发自内心地微笑了,好像童年的感觉又回来了,还有我爱上了骑行这项运动,我每天骑行 50 \unit{\km} 左右,在骑行的过程中我内心也感觉非常快乐,车轮在动,人在飞,春风轻轻地掠过我的脸庞,看着周围的花花草草,青山绿水,内心非常欢乐。不过这几天我觉察到了邪念变得很活跃,一天要上来十几幅邪淫的图像,还有各种怂恿,好在我听了话,好好练习修心口诀,在它们冲起来的时候我立刻和它们拉开了距离,瞬间就把它们消灭了,在邪念很活跃的时候,我一直保持着觉察力,死死锁住内心,皱着眉头,直到这波邪念消失我才放松下来,我还要继续努力练习,我一定要戒好 2019 年。
    \subparagraph{附评} 这位戒友的体验非常好,找回了童年欢乐的感觉,这种感觉非常可贵。前段时间我看见一个小孩,问他几岁了,他天真地说:“我四岁啦。”好一派天真无邪,这个四岁小孩的状态就是纯真喜悦,眼睛是带笑意的,跑跑跳跳,非常开心快乐,这种快乐与性无关,是内心纯净产生的快乐,孩子生活在游戏与玩乐的层面,不懂大人的世界,童年的感觉是非常美好的,在第一次撸管后,这种单纯美好快乐的感觉就会逐渐消失,似乎整个世界都从彩色神奇而变得呆板灰暗了,失去了纯真的乐趣。最近有一天路过小学,正好赶上放学,孩子们那种纯真欢乐的小身影让我颇为感动,我的内心能感受到一种久违的美好与幸福感,仿佛回到了童年。大家有空的话,可以在小学放学时,在校门口附近站一站,再次感受那种纯真喜悦欢乐的状态,自己的心情也会变得像孩子一样,真的是这样,那一张张纯真无邪的笑脸是非常感人的。树疗的可贵之处在于让你亲近大自然的同时改善自己的气色和肤质,去感受大自然那种和谐的振动频率,很快内心就会放松安静下来,脸上会浮现会心的微笑,听听鸟叫,呼吸一下新鲜的空气,看看蓝天白云,闻闻树木和花朵的气味,真是一种莫大的享受,在那种和谐当中,突然就有一种似曾相识的感觉,那种感觉来自于童年。做树疗对树木很有讲究,要选择向阳干净的大树干,树种也要注意选择,有的树阴气重,表面过于粗糙,不适合树疗,做了以后可能导致头部不适或者皮肤受损,这点要格外注意。春季有点小虫子也正常,蚂蚁比较多见,先观察一下树干,做的过程中如果有蚂蚁爬手上,就用树叶引导到地上。亲近大自然真的会让心情愉悦,心态也会变得格外放松,仿佛进入了一片秘境,上个月我去湖边散步,一阵春风吹来,带着花香,灵魂一下就轻盈了,喜悦了,惬意了,那短暂的几秒感觉非常享受,心情变得大好。纯净有纯净的大快乐,放纵有放纵的快感,最终你会发现纯净的大快乐才是自己真正梦寐以求、魂牵梦绕的,那是最纯净的自己所感受到的快乐,远非邪淫的快感所能比。春天骑行也是很不错的体验,看看油菜花,看看抽芽的柳条,感觉春回大地的一片生机盎然,春风拂过脸庞,一瞬间有种回到童年的感觉,内心非常快乐、喜悦。心灵纯净的人,表情和眼神是那么美,眼睛都是亮晶晶的,充满灵气,浑身焕发圣洁的光彩。不过春天情绪容易波动,要注意情绪管理,气温回暖,日照增多,邪念也可能变得活跃,自己要保持警惕,加强修心。这位戒友的断念实战做得很好,相信 2019 年他会戒得更好。
\end{case}

\begin{case}
    终于再次戒 180 天!在以前连破了八个月,每几天就想破戒,破了就懊悔,眼泪都逼出来了,一直破了接近百次吧,还是后几个月开始重新学习戒色文章,感受戒色知识的阳光感。然后破了两个月后,仍然咬定青山不放松地加大量学习戒色文章,每天学习一小时二十分钟。然后一直戒到现在,感慨万千,感觉破戒不破戒只在一念之间,如古人说的:不为圣贤,便为禽兽。戒色也是有成功方法的,念头管理和对戒色文章的理解程度就是成功的方法。还在破戒的戒友不要灰心,坚持学习戒色文章就能战胜心魔。
    \subparagraph{附评} 专业戒色强调学习,学习提高觉悟是非常重要的,觉悟就是对戒色的认识程度和理解程度,要真正把握戒色的原理和规律,这点至为关键。修行方面也是很强调学习的,学习大德开示,学习经论,并不是只是念佛持咒而已,有四个字很好,那就是解行并重,解即是对理论的学习和了解,理论不明容易盲修瞎炼,行法不深,好比饥者说饱,只有解行并重,才是正路。就像断念口诀一样,先把断念的理论搞清楚,弄明白,真正吃透,这样练习起来就会事半功倍,通过练习和实战,再反过来复习和研究理论,写实战总结,这样实战水平自然会越来越厉害。这位戒友戒了 180 天,也算戒色小成了,之前一直破戒,一直失败,多达近百次,他没有放弃,还是捡起了戒色文章,注意积累戒色知识,慢慢提升戒色觉悟和实战水平。最后他悟到了:破戒不破戒只在一念之间!念头是行为的先导,必须注重念头管理,一念稍疏,陷溺难返。戒色的关键就是修心,就是观心断念,无论断念口诀、念佛持咒、思维对治等,都是为了修心。对待破戒的正确态度是:从失败中吸取教训,认真反省和总结,加强学习补强觉悟,坚持练习达到质变。这位戒友破了两个月,仍然咬定青山不放松,不气馁、不放弃,坚持学习,终于学有所悟,觉悟一下提升很多,他的这种精进和对待失败的态度很值得大家学习,即使暂时失败,还是不放松学习,一点不气馁,就像打游戏,一开始一直失败,但还是不放弃,一直打一直打,并且学习高手经验,加上自己摸索和总结,就逐渐成为了高手。我反对沉迷游戏,但可以拿游戏来说明相似的情况,BOSS 是强大,但等你强到一定程度,就有战胜 BOSS 的把握了。戒色并不是叫你沉迷戒色文章,但想要戒掉恶习,肯定需要持续的投入,有投入才有产出,有的戒友戒了好几年,都没有认真练习过断念口诀,戒来戒去,实战还是那么差,关键自己要拿出决心和恒心来练,要加大投入,但不要影响正常的学习和工作,自己做好时间管理。
\end{case}

\begin{case}
    我也分享一下自己的戒色案例吧,希望能对大家带来帮助。我从 2014 年 7 月 31 日至今,到现在未破戒快五年了,五年期间,我真的收获了很多,无论是自身的身体和精神状况,学校的学习情况,还是家庭的经济状况,都在呈上升趋势。不过接下来,我也要为各位兄弟们,尤其是在戒色之后,取得了一些成绩的兄弟们提个醒。从去年下半年开始,因为戒色之后,自己的气场和能量等级得到了改善和提升,加之自己也是大龄适婚青年,的的确确受到了一部分异性的关注,有的异性表现得很主动和热情,其中甚至包括已经结婚生子的异性……也多亏自己通过学习《戒为良药》打下了良好的戒色基础,才没有导致破戒情况的发生。因为这种情况是我在戒色之前从未遇到过的,可以设想一下,如果没有坚实的戒色基础作为保障,那么这几年的戒色成果肯定会毁于一旦。希望大家可以读到这个案例,来警醒自己。另外我现在和异性相处起来,也更加地游刃有余了,比起戒色之前,显得更加的从容和自然。大家闲暇之余,也可以通过网络授课平台,或者一些书籍等等,来学习一下和异性相处的方法。学习它可不是为了泡妞那么肤浅,而是里面有很多心理学等方面的知识,虽然是学习和异性相处,但是我觉得在提升了和异性相处的技巧之后,和同性之间的社交技巧也得到了提高,之前我都没有想过这个问题,看来戒色给大家带来的好处是方方面面的,很多东西已经渗入到了我们的灵魂之中,可以使我们成为一个更好的人,成为超乎想象的自己。我最近在背诵《道德经》和学习日语等等,一切都很好,真的很感谢上天能让我遇到您,飞翔哥。谢谢《戒为良药》带给我的一切,对您真的是无论怎么感恩都觉得不够,哈哈!
    \subparagraph{附评} 这是“浑河大鲨鱼”的反馈案例,他是资深戒友了,还记得当初他的头像是某著名街球手,而现在他已经戒色快五年了,在我眼里,他是一位很有冲劲的戒友,学习的热情和劲头很足,学习能力很强,具备决心和恒心,立场很坚定,是一位很值得大家学习的戒友,也是我比较看好的一位戒友。五年间,他蜕变了,学业方面取得了巨大的进步,考上了名校的博士,真的是一位卓越的人才,家庭等方面也呈上升趋势。戒色真的太重要了,一个充满负能量、充满戾气的撸者会给整个家庭带来灾难,各种不顺和倒霉,因为能量场差了,自然会感召不好的事情,身心脑力再一差,很多事情都会半途而废,没有精力去做,做也做不好,浑浑噩噩的状态,做什么都很难成功。戒色之后精气神会提升,人也会变得纯真喜悦,有正气,这样的确会增加魅力,引起异性的关注,这时因为有了戒色觉悟,所以就具备了相当的定力,知道不能乱来,也知道严格把握交往的分寸,这样就能守得住。我看过古代的戒色案例,拒色不淫是会得厚报的,人在做,天在看,你面对诱惑展现正人君子的风范,老天都看在眼里,老天爷都要为你的正气竖起大拇指,为你点赞,如果放纵自己,肆意乱来,老天爷都会摇头,为你叹息,祖宗都会蒙羞。面对女色的诱惑要有一股刚正之气,就算百般妖娆,千般妩媚,也无法动摇一丝一毫,视若无物,心如止水。“浑河大鲨鱼”也提到了社交技巧,这类的教程是有一定益处的,并不是为了泡妞,而是为了更好地沟通和相处,人际交往能力强了,生活中也会如鱼得水,可以避免很多尴尬和误解,一个好人缘的确很重要,经营好自己的人脉,对自己的事业发展异常重要。如何把握交往分寸是重中之重,有极强分寸感的男人绝对是高度自律和负责的人,不会越雷池一步。和已婚的异性交往,更要严格把握分寸,保持适当的距离,不敢有一丝一毫的非分之想。做事、做人都很强调分寸,要做一个有分寸、有底线、有原则的男人。到了一定年纪,是该考虑恋爱结婚的问题了,有正气和责任感的人能感召好的姻缘,不要太注重对方的颜值,关键是人品,颜值是考虑的一个方面,更重要的是对方的人品。女人心地善良,人品好,脾气好,通情达理,婚姻观很正,懂得勤俭持家,贤惠孝顺,也懂得尊重男人,这样的女人是择偶的理想选择,当然投缘是很重要的,你的振动频率自然会感召类似的女子,所以关键是自己要有一个好的振动频率,这样才能感召一个好的婚姻。

    \begin{quote}\it
        书云天道福善祸淫,盖此一关,是理欲关,是净秽关,是通塞关,是贵贱关,是死生关,是天堂地狱关。何以言之?人之一心,非理即欲,而好色者欲之根也,一好色而诸欲皆萌矣,一觑破则万善咸集矣,故曰理欲关。心本至清,好色而清者浊矣;身本至洁,好色而洁者污矣,故曰净秽关。此中浩浩,何在不宜,一著于色,便生窒碍,甚至父子因之暌离,功名因之阻滞,学问因之无成,非通塞之关而何?吾气刚大,上凌太空,吾情慈悯,下济万物,何等高贵,乃一涉淫私,事机泄露,甚至奴颜不知羞,婢膝不知耻,才子混身于下隶,书生行等于穿窬,非贵贱之关而何?若夫精神完固,而寒暑难入,骨髓流滑,而百病丛生。更有少年之科第,九五之尊严,千年之道行,一念不禁,莫能救药,真死生之关也。至于天堂不必在天,存光明之性体,无处非天堂也;地狱不必在地,陷贪恋之火坑,无处非地狱也。更或前念迷,即是地狱;后念觉,即是天堂。迷觉分于俄顷,堂狱遂判云泥,真天堂地狱之关也。诚可慨也夫,诚可畏也夫。(《家庭宝筏》)
    \end{quote}

    戒色对一个人是至关重要的,人一思淫,心田即暗,中正之心已邪,则光明正大之气遂失。一搞邪淫,运势就会变差,懂得戒色自律,就可以避免很多灾难。有句话说得很好:“你内心的高贵,就是上等的风水!”戒除邪淫,内心才能清净,才能高贵,当你的能量场好了,风水自然就来了,到时人生真的会风生水起。最近“戒得太迟”又出来发帖警醒大家了,他得了尿毒症,经常做透析,手臂上的血管都变形了,异常鼓起触目惊心。真的要懂得戒邪淫,否则沉迷放纵,指不定什么灾难等着自己,到时就痛苦无量、苦不堪言了。一位戒友曾说:“每个人的戒色史几乎都是一部宏伟的史诗,那才是真正的英雄史诗。”我深表认同,做戒色的英雄,拒绝邪淫,做正能量的自己,远离放纵和堕落。真正的英雄向内征服心魔,完成史诗级的逆袭,这是最激动人心的绝杀!全场沸腾!为戒色英雄欢呼!
\end{case}

下面进入正文。

这季是讲断念口诀的,这个口诀我很早就在文章里推荐过,这季会讲得更详细、更深入、更全面一些,比之前的文章要充实丰富。很多戒油子在等待这一季,这季会让很多戒油子顿悟,让他们真正明白戒色的核心。我想说的是,戒油子也有春天,戒油子差就差在实战上!一位戒油子说:“我真的道理危害懂得很多,可是每次邪念一上来我就阵亡了,实战一塌糊涂。”一旦他们领悟了真正的核心,努力训练,就能突破怪圈,完成逆袭。另一位戒友感叹:“实战打不赢,一切等于零!”如何提升实战水平是最核心的问题,对于那些对断念已经有了正确认识的资深戒友而言,这季也会让他们的认识更深刻,会了解到更多的细节,这季的论述会更系统、更完整、更切中要害。

这是一个传承千年的修心诀,它不是我写出来的,而是来自于大德开示,我只是把关键的词语组合在了一起,形成了十六字的断念口诀,也就是大家熟知的:\textbf{念起即断、念起不随、念起即觉,觉之即无。} 这个修心诀对于戒色异常重要,我戒到现在八年多,和领悟、练习这个口诀是密不可分的,是这个修心诀让我学会了真正主宰自己的内心。这个口诀的核心就是——觉!就是提升你的觉察力,觉察力上去了,就可以降伏其心!《金刚经》讲的就是“降伏其心”,很多大德都强调“降伏其心”,如一人与万人敌。上季苏格拉底教给丹的方法,也是提升觉察力,苏格拉底提到的核心,也就是内观,向内看,向内觉察,征服心智。苏格拉底教导的方法就是观心,观察自己的念头,这和我一直强调的观心断念是一样的。

刚开始背诵口诀是为了熟悉这个口诀,看断念的文章是为了深入理解这个口诀,慢慢就会真正领悟这个口诀的含义。在背诵的过程中,其实也就是在练习觉察力,发现自己被念头带跑了,马上拉回来,渐渐地,拉回来的速度越来越快,觉察力在变强。练到一定程度,就不用背口诀了,你会发现自己看到念头时,念头就消失了,这就是觉之即无。到了这个程度,就不用背了,生活中只需保持警惕与观心即可。

这个修心诀,无数的高僧大德都曾推荐过,虽然不一定是这十六个字,但意思是一样的,就是通过觉察来降伏其心。

\begin{quotation}\it
    问:当念头起来的时候,要如何控制它?怎么样不跟坏念头跑?

    宣化上人:念起即觉,觉之即无。
\end{quotation}

\begin{quote}\it
    瞥然一念生起不要理它,念起即觉,觉之即无。(净慧长老)
\end{quote}

\begin{quote}\it
    大抵最上治心,当下清净;才动即觉,觉之即无。(《了凡四训》)
\end{quote}

\begin{quote}\it
    一起便觉,一觉便转,此是转祸为福、起死回生的关头,切莫当面错过。(《菜根谭》)
\end{quote}

\begin{quote}\it
    随他多少邪思枉念,这里一觉,都自消融。真个是灵丹一粒,点铁成金。(《传习录》)
\end{quote}

\begin{quote}\it
    千万个修,抵不过我一觉。觉则心空,此是最上福德。(《般若花》)
\end{quote}

\begin{quote}\it
    念起即觉,觉之即无,修行妙门,唯在此也。(圭峰禅师)
\end{quote}

\begin{quote}\it
    妄起即觉,觉即妄离。(虚云法师)
\end{quote}

\begin{quote}\it
    念起即觉,觉即照破。(憨山大师《观心铭》)
\end{quote}

\begin{quote}\it
    在妄念生起时,能时时觉照,不随之迁流迷失。此如古德所云:念起即觉,觉之即无。(南怀瑾先生)
\end{quote}

\begin{quote}\it
    不怕念起,唯恐觉迟。念起即觉,觉之即无。(普照禅师)
\end{quote}

\begin{quote}\it
    经过成千上万次的斗争,由一开始的念起不觉,到念起能觉,由觉而不能转,到一觉即转。”“内不随念转,念起即觉,一觉即空。念起了不知道,看不见,跟念头走了,不觉还有念头,那就迟了。念起了马上就知道,立即转掉。”“古德云:不怕念起,只怕觉迟。所以大家须于念来时即看见,不跟着跑。”“是以悟后,常须照察,不可玩忽,妄念忽起,或一凛觉,或提佛念,以至无为,方始究竟。(元音老人)
\end{quote}

这些开示最核心的一个字,就是觉!要觉察、觉知、觉照,当你掌握了觉察的功夫,就像拥有了强大的激光武器一样,可以瞬间消灭念头怪。

很多戒友都是通过学习和练习这个口诀,从而真正学会了修心,这个口诀已经让很多戒友受益匪浅,经过成千上万次的实战,对口诀的理解和体会越来越深,也深感这个口诀的价值无量。这个口诀虽然来自于佛法,但不用信佛也能练,因为它是处理念头的一种方式,不需要念佛持咒或者念经等,所以比较适合专业戒色,大家只要认真练习这个口诀,就有望降伏心魔,主宰内心。这个口诀是最具普适性的,即使不信佛也可以练习断念口诀。念佛持咒也很好,合适有佛缘的戒友,刚开始很多人容易接受科学专业的戒色途径和方法,等到慢慢了解佛法之后,才有可能接受念佛持咒,所以是一步步来的,戒色吧主要是走专业戒色的途径,就是从覆盖面和接受度这个角度来考虑的。

一位资深戒友说:“很幸运遇到飞翔与《戒为良药》,也很幸运自己坚信着飞翔与《戒为良药》,专心重复研读并落实《戒为良药》,才能戒除四年多,四年多的蜕变经验也让我总结了断念绝对是核心中的核心,这是不管什么理论都无法推翻的事实。”

断念是戒色实战的核心,不管口诀、念佛持咒还是思维对治等方法,都是为了断除念头,不让念头连续下去。有的戒色前辈为了争第一,写了好几篇文章来诋毁断念口诀,用练习不得法、有思想误区、走极端的失败案例来反对断念反对学习,这是完全错误的,明眼人一看就知道是偏见和误导。真正有德行的前辈是不会争第一的,而是会互相谦让,注重和谐与团结。这个修心诀是无数高僧大德共同一致推荐的,不是某些人区区几篇误导的文章所能诋毁的。大家对断念口诀要有绝对坚定的信心,不要看那些诋毁和误导的文章,那些文章完全是胡说八道,误人子弟。推荐念佛持咒是很好,但如果为了争第一而贬低和诋毁断念口诀,这就不对了。这个修心诀完全可以化解念头,化解欲望,并不是压念,而是觉而化之,这是无数高僧大德推崇备至的修心诀,传承千年,经过无数实战检验,是非常值得学习和练习的一个金口诀。参透这个口诀,练习这个口诀,真的可以立于不败之地,我戒到现在八年多一次未破,就是最好的明证。

\subsubsection{断念口诀的两种用法}

\paragraph{实战时念口诀}

这种用法就是平时熟背断念口诀,达到条件反射的程度,每天几百遍,零碎的时间也可以背,然后用冒出的其他念头来训练,比如你脑海里冒出一个过去的回忆,你第一时间发现,马上念口诀来转,或者你脑海中冒出一个骄傲的念头,也可以马上念口诀来转,或者冒出一个嗔恨的念头,也可以念口诀来转。强迫的念头,其他的胡思乱想,也可以念口诀来转。关键要发现快,不要让念头起势,马上念口诀,人家是“快使用双节棍”,我们是“快使用断念诀”!这个用法的原理类似于念佛的一念代万念,莲池大师:“\textit{念一佛名,换彼百千万亿之杂念也。}”你一次次训练自己,那个发现的速度会越来越快,如猫捕鼠,那个觉察力会变得越来越敏锐。

\begin{case}[实战时念口诀]
    现在我的情况是,发现念头,然后要念一句“念起即断”断掉它,而不是一觉就自然消失。
    \subparagraph{解析} 这位戒友就是念口诀来断念的,实战时,十六字可能有点长,可以念前四个字,也就是念起即断,这样更简单有力些。关键平时要背得熟练,念头一来,马上念口诀,那个发现的速度要极快,这样断念就很容易。这位戒友虽然还没达到“一觉即空”的程度,但已经做得不错了,继续强化下去,就有望做到觉之即无,一觉即空,一觉即胜!
\end{case}

\paragraph{直接觉察消灭}

如果说实战时念口诀是初级阶段,那高级阶段就是觉察即消灭,觉察即降伏,一觉即空,闪电般断念,就像刀客出刀,瞬间解决战斗。刚开始很多人是做不到的,强化到一定程度就可以做到了,是突然发现自己能做到了。当第一次做到时,会很兴奋,好像发现了一个宝藏一样,可以兴奋一整天,断念的实战水平一下有了质的飞跃,那种顿悟的感觉真的太爽了,比中大奖还高兴。到了高级阶段,你会发现觉察力就是最强的武器,那个发现,其实就是觉察,本来当你发现了,还需念口诀或者念佛来转,当你继续强化,某一天会出现这样的情况,那就是当你发现念头了,念头就消失了,只是发现,只是看到念头,念头就消失了,还没等到念口诀或者念佛,念头就消失了。那个发现是一瞬间的事情,极快,一下就解决战斗了,真的是快至颠毫,眨眼间就决出胜负了。

\begin{case}[直接觉察消灭]
    我开始狠下心来,痛定思痛,深刻反省了自己戒色一直失败的原因,一直卡住我的那个点?限制我的那条“线”?笼罩我的那张“网”?慢慢地,答案的轮廓逐渐开始清晰了,原来如此,到底还是断念实战啊!那天,仿佛整个世界都光明了,我看到了战胜心魔的希望!我彻悟了戒色的核心,没错———就是断念实战。之前也了解过这方面的内容,但也只是走马观花,似懂非懂,没能明白其中的深刻含义。于是我开始疯狂练习观心断念,因为我对佛号比较有信心,从小也有佛缘,选择了念弥陀圣号用以断念。但是那时还不太懂得观心断念的真正内涵,反正感觉念佛肯定有用。于是我拼命念佛,一天到晚,只要让我抓住空闲时间,我就玩命地念。慢慢地,我发现邪念上来,马上就自动地转为佛号了。后来随着学习和练习的深入,我发现我可以直接做到觉察消灭了,邪念冒出,觉之即无,看见念头的那一刹那,战斗就结束了,很轻松,一点不费力,就是简简单单地一觉。这个“觉”,真的不可思议,奥妙无穷,分量极大,价值千金!
    \subparagraph{解析} 这是资深戒友戒色一年的感悟,写得非常好。戒色一直失败的原因是什么?自己一定要好好反省,可能导致失败的原因有很多,但是最后都是断念没做好,或者不想断念,放纵自己。念头是行为的先导,要戒掉恶习,必须懂得修心,对治邪念。这位戒友有佛缘,坚持念佛,慢慢做到了念来就转成佛号,这时候他的觉察力已经得到了强化,随着学习和练习的深入,他进入了高级阶段,做到了觉察即消灭,看见念头的那一刹那,战斗就结束了。到了高级阶段,还可以继续精进深造,就像跑进十秒大关,还能继续进步一样。这个“觉”,是无数高僧大德一直在强调的,也是《了凡四训》《菜根谭》《传习录》中真正有分量的字。这个“觉”字,它可以是一个名词,比如觉悟,也可以是一个动词,觉察!一觉即灭,一觉即空,一觉就完事了,这个觉,实在太厉害了!刚开始可以从转念练起,念头一来,马上念口诀或者念佛来转,也就是一念代万念,首先要把口诀或佛号念得纯熟,保证每天的日课不断,口诀一般建议每天五百遍,佛号则是三千、五千、一万等,可以根据自己的时间来安排。
\end{case}

\begin{case}[直接觉察消灭]
    飞翔哥,我几天我终于明白了,发现念头的一霎那,战斗就已经结束了!剩下的就是去感受那个临在,连回忆、断除的念头也不要有,念起即觉,然后继续去感受。
    \subparagraph{解析} 这位戒友终于做到了,进入了高级阶段,念起即觉,念头就消失了,就这么简单,剩下的就是纯粹的觉知,安住一会,好好感受一下。
\end{case}

\begin{case}[直接觉察消灭]
    不管做了什么功课,最后要靠实战来检验。就是看自己是否能念起即断,觉之即无。行善非常重要,但是修心断念是更重要的最核心。一定要注重实战,念起即断才能避免破戒,因为即使行善和学佛,念头也还会起来,起来就面临着能否断掉的问题。
    \subparagraph{解析} 这位戒友总结得很好,不管是行善、念佛持咒、念经、放生等,即使做得再多,也会发现念头还是会上来的,戒到一定时候还会遭遇猛烈的翻种子,到时就很考验一个人的断念能力,断念就是在修心,是最核心的东西。上季一位戒友和我反馈,说他住道场,但还是破戒了,道场每天有日课,而且作息规律,怎么还会破戒?就是看到了女众,心里有了不良的想法,不知对治,一方面要思维不净观对治贪恋,另外一方面就是不良念头冒出来时,要及时断掉,这样就不会破戒了。即使到了道场,也要注意对境实战,尽量不要去看女众,尽量不要和女众说话,管住自己的视线。
\end{case}

\begin{case}[直接觉察消灭]
    我戒色也快一年了,戒戒破破,放生,行善,学习戒色文章一直都有在做,但是每次欲望来了,特别是看黄的念头根本就抵制不了,立马缴械投降,然后变成禽兽,真的,邪淫的人真的是禽兽。每次发泄完了,心中无限悔恨。也就是前一个星期,破戒了,然后我内心开始挣扎,呐喊一定不能这样下去,于是我总结反省,觉得根本问题还是断念不行。就这样,我下狠心,每天早上跑步时练习断念口诀,可以说进入疯狂状态,就这样突然开悟了,知道这个口诀的真正妙处,原来就是觉,念头一来,立马觉察,盯着念头看,发现它就消失了,有时候感觉要来,就一直这样保持觉察状态。我以后每天跑步都会练习,一直强化,真的不能再邪淫了,不然下一次就可能撸进医院了。
    \subparagraph{解析} 这位戒友也开窍了,往往疯狂练习后就会开窍,就像禅宗师父会逼徒弟一把,帮助徒弟开悟。所谓的疯狂,并不是叫你失去理智,而是勇猛精进的一种状态,就像“林疯狂”开挂般的表现。有的 NBA 球星在输球后会疯狂加练,那种状态势不可挡,火力全开,知耻而后勇。看过一篇文章,名字叫:全队最刻苦!库里疯狂加练,三分 100 中 91,真变态准。库里疯狂加练投篮,最后一个离开球馆,有着非凡的职业精神。连练球很勤奋的杜兰特都离开了,库里还是留了下来,疯狂加练。库里为什么那么准,或许看看他平时的训练就可见一斑。这位戒友戒戒破破快一年了,也是该开窍了,每天坚持练习断念口诀,迟早会开窍的。不开窍,隔座山;一开窍,隔层纸。不开窍,怎么都进不去;一开窍,就会觉得原来如此简单。开窍就真懂了,之前是假懂,似懂非懂,开窍后继续精进练习,实战水平就会越来越高,越来越强。
\end{case}

\subsubsection{断念口诀的思想误区}

\paragraph{把断念误解为压念}

憨山大师说过“念念斩断”,元音老人也说过“断念保护”,并不是叫你压念,而是叫你觉察。断念是要狠一点,要有一股猛力、狠力,坚决果断,斩钉截铁。把断念误解成压念,这是常见的思想误区,压念就是主观不想让念头起来,念头一起来,就想压住它,结果越压越反弹,会感到压抑和挫败。断念则是不怕念起,就怕觉迟。如果对念头有抗拒心理,越不想让它起,它就起得越厉害,有抗拒心理很容易导致压念,不要抗拒念头,而是懂得通过觉察来化解念头。\textit{念起即觉,不压不随。(元音老人)} 一觉就完事了,一点不压抑,很轻松。

\paragraph{以为光练习断念就行,不用学习了。}

这也不对,提升综合觉悟也是非常关键的,要有一个深入的理解,反复学习是必要的,一般过一段时间,就会感觉自己的领悟更深了,更透彻了,更扎实了,在实战中能够游刃有余地把握各种局面和情况,实战表现更稳定了。学习可以让你借鉴前辈的实战经验和研究成果,这对于提升自己的实战表现是非常重要的,实战中有各种情况和各种细节,这些细节往往很重要,有时一个细节就决定着成败。对于学习,我自己也深有体会,有时吃透大德一句开示,需要好几年的时间,之前以为懂了,其实懂得很浅,等到几年后,突然就顿悟了,有了更深层的理解,对于本质有了更准确、更简明、更深入的理解和把握,这时真正吸收到骨髓了,之前停留在皮毛。要反复学习,反复复习笔记,在复习过程中,突然某条笔记就可能让你“炸眼”,突然一下就很有感觉,就像炸药遇见了火花,点燃了顿悟的火药桶,开出了绚烂的烟花。之前那条笔记,也许你看过很多遍都毫无感觉,但是突然有一天,感觉来了……你真正读懂了,真正明白了那个道理。顿悟实在太兴奋了,那个欢喜雀跃啊,无法言喻,整个人好像受到了巨大的激励和鼓舞,完全不一样了。感觉顿悟后,整个人的气质、眼神和行为都不一样了,一股强大的力量已经灌入了。

\paragraph{两种开示的圆融理解}

一种开示提倡斗争,消灭念头;另一种开示则说不要斗争,只要不随,或者安住于觉性,让念头自动消融。

看大德开示肯定会发现这两种貌似矛盾的说法,其实这两种都是可以的,看开示要全面,不可偏于一方,也要圆融理解,否则就会产生严重的误解。从不随的角度来讲,很多大德是反对与念斗争的,因为怕你压念。但是从觉察的角度来讲,很多大德是主张斗争的,斗争不是压念,而是觉而化之。明就仁波切讲过:“一看念头,念头就消失了。”这就是觉而化之。要圆融地理解开示,而且要广泛地了解修心,不能只是看了几篇文章就局限地下结论,有的人甚至会根据一方而反对另一方,自己陷入狭隘偏见而不知。总体而言,我是比较认可斗争的,大多数的高僧大德都是讲斗争、讲降伏、讲征服、讲消灭,用的词汇都是比较强硬的词,佛经中用的也是降伏,修行是要制心的,这是毫无疑问的。但是大德说不要斗争,我也能圆融理解,因为大德看到压念的情况,一些人因为压念而感到挫败和烦恼,乃至神经兮兮的,这时候大德针对实际情况,就说不要斗争,只要不随,其实不随就是断!目的其实都是一样的,都是不让念头连续下去。

\begin{case}[两种开示的圆融理解]
    感谢大哥!我刚好在看《当下的力量》,在第九章最后的时候遇到问题!内容:就像你不能与黑暗抗争一样,你也不能与无意识抗争。如果你试着这样做,事物对立的另一面就会得到加强。你就会被其中一个对立面认同,你会创造一个敌人,并把你自己拖入无意识状态中。无论如何请确保你内心没有抗拒,没有仇恨,没有消极力量。心魔利用这段话,让我不要和心魔对抗,我感到恐慌!估计我理解有误!
    \subparagraph{解析} 心魔很狡猾,利用思想误区怂恿他,动摇他。托利说不要抗争,意思就是不要强压,而是学会带进觉知之光来化解无意识的模式。强压肯定不行,越压越反弹,关键是要学会化解。是觉而化之,不是压念。托利的这段文字要善巧地理解,否则就会产生误解,不要抗争,并不是不要和心魔斗争,托利强调的是觉知,带进觉知之光,黑暗就自动瓦解消失了。念头一出现,你一觉察,念头就消失了,这就是利用觉知之光来让念头消失,这才是正确的斗争方式。看开示一定要善于圆融地理解,要明白作者是从哪个角度和哪个层面来讲的,这样就不会产生矛盾。
\end{case}

提倡斗争的开示,这类开示会给你一股明确而坚决的力量,但容易让人误解为压念,我们一定要正确理解断念。而提倡不要斗争的开示,好处就是让人比较放松,一般不会产生压念,但缺少一股斗争的狠劲,弄不好会变得懦弱不作为。这是两种风格的开示,就像武将和文官一样,虽然看似矛盾,实则各有千秋,应该圆融理解,充分尊重这两种开示,我个人偏向于斗争的开示,正确的斗争方式是觉察,而不是压念。要懂得斗争的技巧,不能盲目斗争,盲目压念。

\textit{所言格物者,格,如格斗,如一人与万人敌;物,即烦恼妄想,亦即俗所谓人欲也。与烦恼妄想之人欲战,必具一番刚决不怯之志,方有实效。否则心随物转,何能格物?……战之一字,关系甚深,人欲天理之际,若不以力战,则理被欲蔽,俾理必隐而欲必著矣。(印光大师)} 印光大师提倡斗争,提倡力战,元音老人也提倡斗争,很多大德都提倡斗争,斗争的开示是那样坚决,那样有力量,那样立场鲜明,古德有云:“打得念头死,许汝法身活。”\textit{妄心杀而真心现。(《菜根谭》)} 提倡斗争,提倡杀内心贼,这类开示我是比较喜欢的,当然我知道如何斗争,这样就没有任何问题,与心魔斗,其乐无穷。

\subsubsection{实战水平提升的过程}

学习 $\to$ 练习 $\to$ 实战体会 $\to$ 继续学习 $\to$ 继续练习 $\to$ 实战总结,觉悟和实战水平是螺旋上升的,途中可能有暂时的退步,是为了更大的进步在蓄力。每次实战后,应该要回到戒色文章,回到戒色笔记,这时候再看,就会心领神会,因为有了实战体会后再看,就会很有感觉,好像说到心坎里一样,特别有感触。自己平时也要经常写实战体会,反省一下自己的不足之处,哪些做得还可以,哪些还有所欠缺,还需继续完善和提升。实战主要包括两方面,对境实战和断念实战,内不随念转,外不为境迁。这两方面都非常重要,都要尽力做好。不管是对境实战还是断念实战,值得研究的内容是很多的,实战中会出现各种情况和各类问题,自己一定要沉着应对。

一位戒友的实战总结:

\begin{case}[实战水平提升的过程]
    刚刚看完文章,打开 QQ,经历了一次对境实战,心有微动。实战过程非常短,但是也反映出一些问题:\begin{multicols}{2}
        \begin{itemize}
            \item 首先,打开 QQ 空间,好奇心点进了空间,此时没有注意观心,没有观察到这个好奇的念头,好奇那个人是谁。
            \item 其次,一进空间,看见了擦边图片,实战表现不够坚决,不是立刻避开,不是强烈地关闭,而是有所迟疑,说明我的实战意识依然不够强,实战操作不够娴熟和果断坚决。就像防震演练,一个人操作不够熟练,就会容易惊慌失措,应该把这种避开、不回看、不停留反复强化反复思考内化成意识,并且通过反复训练成为一种条件反射,就是不假思索的,程序化执行“避开、不停留、不回看、不回忆”,这样子就能大幅优化实战表现。
            \item 另外,看到擦边图片时,除了落实避开、不停留、不回看、不回忆、不看第二眼之外,还需要把握好念头,因为念头也会随着对境而起,很细微、很迅速,如果不能及时发现并控制念头,就会一下子扭转你的走向,使你从前一秒的持戒,立刻变成搜黄(因为扑灭火苗不及时,火苗烧起来了,才想起来扑,这时候已经身不由己)(这种时刻就是决定性的时刻,是决战时刻,是拼刺刀的,不能有丝毫迟疑懦弱,否则,阵亡的就是你!拔剑决战!果敢决绝!)
            \item 实战时,避开以后,看准了念头,因为念头肯定会在这时起来,一冒头,立刻击毙。
            \item 上网需谨慎,无事不上网,上网时时刻保持观心。
            \item 不去随便看陌生人的空间,少看 QQ 空间,不要随便点击,不要好奇,好奇害死猫,很可能是一个埋伏圈。
            \item 继续加强断念,观心时刻开启,时刻保持警觉,时刻觉察,尤其是上网时,就像杀毒软件是时时开启一样。全天候观心,了解自己的每一个念头,认识你自己。必须拿出最大的热情,以最认真的态度,对待断念练习,平时多流汗,战时少流血,邪念一冒立刻拦截,自动化,零误差!
        \end{itemize}
    \end{multicols}
    \subparagraph{解析} 这位戒友的实战总结很好,很细致,实战总结要经常写,经常反省,这样才能不断优化和提升。网络时代,很容易遭遇各种对境实战,到处都是擦边图和擦边新闻,跳出各种诱惑图片,加上好奇心的驱使,很容易去点击,不是每次好奇都会导致破戒,但放任自己的好奇心,迟早会陷进去。对境实战是我一直着重强调的,也是需要反复强化的,对境时很考验你的实战意识,特别是视线管理,即使是戒了好几年的资深戒友,对境实战也可能做得不太好,也可能去看几眼,自己一定要认真反省和总结,不断优化自己的实战表现。对境时会遭遇各种诱惑,非常考验一个人的实战表现,真正的定力是通过对境时练出来的,成千上万次对境,表现越来越优化,视线管理和念头管理越来越强。并不是叫你主动去找黄来练,如果出现这种想法,很可能是心魔的怂恿。生活中和网络上,肯定会看到的,包括我做戒色图片,也会看到擦边图,关键自己要管住视线,不要去看第二眼,不要停留,多思维不净观,不要觉得诱惑好,这点非常关键,如果你觉得诱惑好,诱惑就对你产生了作用,产生了强大吸引力。上次一位戒友说要有平等观,我觉得他说得很好,看一切平等,虽知道美丑,但却不去分别美丑,每个人的标准是不同的,你一分别好坏,你肯定强烈喜欢好的。就像一个戒友说自己总是盯着美女看,很难控制自己,首先他认为那是美女,这个观念就强化了吸引力,而在我看来就是臭皮囊,我的观念就淡化了吸引力,他的思维他的观念强化了诱惑的吸引力,这就是最大的区别,观念转变一下,就能对治贪恋,这点非常重要,很多人戒来戒去,观念还没转变过来,对境时一点定力都没有。狗看到美女有反应吗?狗没觉得那是美女,狗没那种分别念。对境的秘诀,就是不要觉得美,不要觉得好,这样吸引力就减少或消失了。如果你觉得美,觉得好,那吸引力就大幅增强了。
\end{case}

\subsubsection{练习过程中的问题}

\paragraph{走神了}

只要开始练习断念口诀或者念佛,肯定会分心走神,被念头带跑,这是肯定会出现的现象。关键就是在分心走神的那一刹那,及时发现,拉回来,慢慢拉回来的速度会提升,越来越快,越来越敏锐,拉回来这一下,其实就是在觉察。不要怕走神,关键要及时拉回来,慢慢走神的次数会减少的,因为你的觉察力和专注力提升了,这也就达到了练习的效果了。过去你被邪念一次次带跑,现在你有了强大的觉察力,可以及时发现消灭邪念。

\begin{case}[练习过程中的问题]
    问好飞翔老师,念断念口诀的时候没办法把注意力集中在口诀上怎么办?每次练习断念口诀的时候,虽然嘴上是在念口诀,但是心却不在口诀上,老是想些其他的东西,很难集中注意力,这样下去肯定效率不高,飞翔老师这该如何去改进?求解。
    \subparagraph{解析} 这位戒友说的就是走神问题,一边念口诀,一边想其他东西,念佛时也会出现这种情况,虽然口念佛,但是心却在想别的,这样肯定是不行的。自己首先要知道,这是练习过程中必然会出现的现象,每个戒友都会遭遇,我也经常遭遇这种情况,关键就是拉回来那一下,发现自己分心走神了,马上拉回来,刚开始也许要好几分钟才意识到自己走神了,练到后来,分心的念头一出现,你马上就能看见,就像鹰眼一样敏锐,就像鹰爪一样强悍。关键就是要敏锐地感知念头入侵、念头进来的那一刹那,对那一刹那要有敏锐的感知,要对念头越来越敏感,念头进来时,反应要超强烈,就像一颗小石子进入眼睛一样。刚开始都差,很少一开始练习就能做得很好的,关键是持之以恒地练习,那个发现的速度,拉回来的速度会变得超快,你实战时那股气息和眼神也会变得超强,念头贼一进来,就被你发现了,很难带跑你了。
\end{case}

\begin{case}[练习过程中的问题]
    练习断念的时候,是肯定会走神的,会跟杂念跑,刚开始练习的时候,等你跟着杂念跑了好久才会发现自己走神。随着练习的加深,你会发现走神的次数少了,被念头带跑的时间短了。这就是“断念速度”的练习。练习观心断念意义之一其实就在于走神的那一刻,走神的时候把念头拉回来,然后立刻觉察。走神的时候立刻觉察,其实就像实战时候邪念来了你一记觉察。练习的时候,如果你走神了,你觉察得越快,实战里断念的速度也会越快。
    \subparagraph{解析} 这是一位戒友的总结,他总结得挺好,拉回来那一下其实就是在觉察,发现、觉察、拉回来,这三者是一体的,是在一个瞬间完成的,当你发现,你就是在觉察,当你觉察时,你已经拉回来了。关键就是走神的那一刻,及时发现,拉回来,就是在训练那个拉回来的速度,原来是跟念跑几分钟,都不知道,练习后,念头一来,马上就知道,这就是进步的指标。练习口诀和念佛号,其实都是在训练你的觉察力,觉察力永远是最核心的东西,缺少觉察,就会跟着念头跑,这肯定是不行的。有了觉察力,才能主宰内心,缺少觉察力,就会被念头带跑,没有主宰权。
\end{case}

\paragraph{如何计数}

有的戒友会问如何计数,一般有两种计数方法,一就是根据时间来算,二就是用计数器来算。第一种,比如一分钟你可以算下自己能念多少遍,念一分钟来录音,数下录音里自己念了几遍,然后按这个速度来算五百遍大概需要多少时间。第二种,就是买个计数器,网上有,我用的就是计数器,这样比较准确些。根据时间来算,可能会有误差,而用计数器就比较好,可以准确地完成遍数。过去的修行人是用佛珠来计算的,比如 108 颗,念一遍就拨动一颗,念一圈就是 108 遍。不过相比之下,还是现代的计数器更方便一些。量的积累最终会迎来质的飞跃,只要你正确理解原理,坚持练习,你就会发现自己的觉察力在不断变强,就像一块肌肉在练习后变强变硬一样,断念口诀练的就是觉察力,发现念头的能力。

\paragraph{没有坚持练习}

有的戒友也练习了,但没有坚持,这是最可惜的,只要练下去,肯定会有所进步的,自己要有恒心,每天保质保量地完成任务,就像刷牙一样去坚持,慢慢习惯了就好,如果学业和事业忙,自己则要学会挤时间来练习,把零碎时间利用起来,废时利用可以挤出很多时间。坚持练习一段时间后,你就会发现自己的觉察力在提升,这样就可以激励自己更勤奋地去练习,会自愿加大练习的投入,觉得越练越喜欢练,能够进入这种状态是最好的,热爱练习,练习得法,这样进步就会突飞猛进。

\paragraph{练习断念时过于紧张、过于用力}

有的戒友练习时脑袋疼,那是过于紧张,过于用力了,注意力不要过于集中于头部,否则头部是可能出现不适的,把注意力轻微地集中于眉心的位置,也就是第三眼的位置,是有利于强化觉察力,但不要过于用力,否则可能会出现头痛的表现,要学会放松。一旦出现头痛,要及时调整,一般建议放松,注意力不要集中于头部。一般练习断念也不需要把注意力集中于头部,只是去感知念头的出现而已,把注意力轻微地集中于眉心,这是一个小技巧,并非一定要这样做。

\paragraph{不知道怎么提升觉察力}

这个问题不少戒友也问过,上面已经讲了,就是发现念头,不跟念跑,发现自己走神了,及时拉回来,是这样不断训练的,刚开始被念头带跑很久,才知道自己陷进念里了,后来念头一出现,你马上就发现了,这就是觉察力提升的表现,那个速度可以练得极快。当你跟着念跑时,那种感觉好像自己就是念头,自己成了念头,当你突然发现,突然觉察,这时你就成了观察者,这是你的真实身份。

\paragraph{断念水平不稳定}

有的戒友反馈断念水平不稳定,有时做得很好,有时做得较差,这种情况是经常出现的,即使资深的戒色前辈也不是每次都做得极好。邪念上脑是极快的,那种念头、图像、微妙感觉,一瞬间就上脑了,大概就是零点几秒的时间,就上来了,一瞬间的事情,觉察断念也是眨眼间的事情,关键就是坚持练习,慢慢断念水平就稳固了。不过即使断念水平很高的人,偶尔也有做得不好的时候,比如两到三秒才断掉,而不是零点几秒就解决战斗,这种情况很常见。

\begin{case}
    已经意淫两三秒了才察觉到,而有时就能快速反应一秒断念。
    \subparagraph{解析} 这位戒友反馈的就是这种情况,随着坚持练习,实战中基本都能做到零点几秒解决战斗,偶尔两到三秒秒,要看整体的平均水平,就像跳水高手,一直跳出高分,但偶尔也会失误,跳出一个不太理想的成绩。高手能稳定在较高的水平,是看平均水平的,如果平均水平在两到三秒,这个成绩就不太好,平均水平在一秒内,乃至零点几秒,就比较好了,这是觉察力敏锐、觉察力强的表现。
\end{case}

\paragraph{急于求成}

这种心态,做很多事情都会出现,断念水平的提升不是一蹴而就的,需要坚持学习、练习、总结,是一个渐进提升的过程,关键就是坚持练习,练到一定程度自然就熟极自神、出神入化了,就像练一个魔术手法,一下把念头变掉了。急于求成的戒友,他的心态就是浮躁的,他的练习效果也是比较差的,一定要有耐心和恒心,慢慢就能看到自己进步了。也像练习一门乐器,不能指望一开始就达到很高水平,需要坚持练习,持之以恒地练习,要热爱练习,慢慢有了心得体会,就更爱练习了。前辈也不是一下就达到零点几秒的觉察速度,而是坚持练习才做到的,这是一项值得投入的练习,你会发现自己正在变得越来越强,那股觉察力的品质已经今非昔比。

\paragraph{看不到效果而放弃}

这种现象也经常出现,铁棒磨成针,不是短期能看见效果的,关键是坚持,一天天坚持下去,突然有一天你发现自己的水平真的上去了,已经不是过去那个被心魔随便虐的菜鸟了。这个提升的过程也许需要几个月,或者一年以上。看不到效果,但其实已经在进步了,开水没滚,但已经在朝着沸点冲刺了,坚持加热,最后就彻底沸腾了,要明白这个进步的原理,贵在坚持,每次练习,你都在变得强大,每次拉回来,你的觉察力都在进步。突然某一天,你发现自己的水平已经上了一个全新的层次了,进入那个层次,实战的感觉完全不同,你开始有了主宰力。你的断力会变得极其强悍,你开始主宰两耳间了,你开始登堂入室了!

\paragraph{日课脱离实战}

有的戒友每天背诵口诀或者每天念佛,但是实战时却没有用出来,这就是日课脱离实战,一定要明白日课最终是要和实战挂钩的,修行是要以实战为核心的,不能纸上谈兵。有的人只强调日课,不强调实战,这是有所欠缺的,即使念佛念到非常清净了,也是暂时的,戒到一定程度会经历猛烈翻种子,平时还会遭遇对境的诱惑,各种考验、各种念头都会袭来。记得犟牛居士说过一句话:“光念佛不知道修心,尝不到法味。”对于这句话,我记忆很深刻,很多人虽然在念佛,但却不知道用佛号来断念,印光大师:“\textit{当杂念起时,格外提起全副精神念佛,不许他在我心里作怪。果能如此常念,则意地自然清净。当杂念初起时,如一人与万人敌,不可稍有宽纵之心。否则彼作我主,我受彼害矣。若拌命抵抗,彼当随我所转,即所谓转烦恼为善提也。汝现能常以如来万德洪名极力抵抗,久而久之,心自清净。心清净已,仍旧念不放松,则业障消而智慧开矣。切不可生急躁心。}”要用念佛来抵抗,如一人与万人敌。一句佛号犹如金刚王宝剑,什么都可以斩断,能不能斩断,就看你的力量。\textit{教人称念弥陀名号,以一念而除众念,妄念既除,烦恼自断。一句佛号,如金刚王宝剑,烦恼妄念,喻如劫贼,贼若来时,宝剑即举,贼当自退。念佛之法,亦复如是,贪心烦恼起时,即一心念佛,而贪心自息,嗔心痴心等起时,悉皆如是。(圆瑛法师)} 大德很强调念佛实战,日课是很重要,但最后关键还是看实战,日课不能脱离实战。这点切记!

\begin{case}[练习过程中的问题]
    感谢飞翔老师,一直在等您发新帖,我很惭愧,屡戒屡破三年多,最近终于懂了,还是自己断念能力不强,以前也练习过断念,刚刚学会点断念就以为自己已经非常厉害了,后来破戒后看到说持咒更加殊胜,就放弃了断念口诀,开始念佛,但那会没有领会到念佛也是为了断念,而我一直是为了增加自己的福报去念的,所以就导致后来的一直破戒。也就是前段时间看到您说的,无论何种破戒,都是最后没有把念头断掉!意识到自己一直在走弯路,所以现在又重新开始练习断念,每天坚持学习,但是有个问题得向您请教下,就是戒色应该怎么对待玩手机?我是那种自控力比较差的,比较容易沉迷手机。
    \subparagraph{解析} 这位戒友本来练习断念口诀,后来听说持咒更殊胜,就去练习念佛持咒,其实断念口诀和念佛持咒并不冲突,断念口诀其实教导的是观心,而观心是大总持,修任何法门都要观心的,观心是基本功,观心本身也是最深奥的法门。念佛时也要观心,发现杂念来了,马上专注在佛号上,观心可以说贯穿所有法门。念佛持咒是很殊胜,但也要看到断念口诀的殊胜之处,断念口诀最终的操作是离开念头的,只是觉察。如果有佛缘,可以好好念佛持咒,但要懂得念佛持咒的原理,不能为了福报去念,那样念来念去,不知修心,结果肯定还会破戒。我自己也是念佛人,念佛其实也是为了提升觉察力,最终也要用于断念,念了几千万的佛号,最终念头还会上来的,这个我深有体会。断念口诀和念佛持咒都是很好的,明白原理去练,都是可以成功的,有的人推荐念佛持咒,就会贬低和诋毁断念口诀,我们应该有一个包容和尊重的态度,对于断念口诀和念佛持咒,都应该充分尊重,这样就不会陷入狭隘和偏见。关于手机的问题是比较普遍的,那就是沉迷玩手机,自己要节制用手机,减少刷新闻、刷图片、刷视频,我平时也用手机,但很注意控制时间,避免沉迷手机。我越来越发现刷手机也是会上瘾的,会浪费很多时间,还会遭遇各种诱惑的内容,这方面自己一定要警惕,可以培养一下其他的兴趣爱好,减少用手机。我们可以用手机看戒色文章,看大德开示,看修行的书籍,这样就比较好,用于有意义的方面,比无聊刷手机要强百倍,刷来刷去,只会感觉更空虚更无聊,这时就想看黄找刺激了。沉迷手机,手机就会变成手雷,会毁掉你的生活,有句话是这样说的,要毁掉一个人,就给他一部手机,说得有点严重,但充分说明沉迷手机的确危害很大。
\end{case}

\begin{case}[练习过程中的问题]
    飞翔哥,你帮帮我啊!我手淫 16 年了,从初一开始,现在 29 了,身材矮小瘦弱,找不到对象,前列腺炎,尿频尿急尿不尽,阳痿早泄秒射,神经症,焦虑症,抑郁症,我也在学习《戒为良药》,可是,我现在的问题,黄瘾非常重,我常常是上午出去念了佛号五千声,一回来立马搜黄看黄,黄瘾很重,就想看黄,看了就疯狂地射,最少两次,自己控制不了黄瘾,今天腊月二十九,我戒了十四天,今天办公室就我一人,看黄三小时,连续破戒两次,我也念佛,已经念佛号 290 多万声了,快三百万声了,但是黄瘾太重了,就想看黄,控制不了,飞翔哥救救我!求你帮帮我!
    \subparagraph{解析} 这位戒友的问题就是日课和实战脱离,念佛很好,但也要注重用念佛来断念,当想看黄的微妙念头出现时,及时发现,马上念佛来转,发现一定要快。念佛的日课也要保证质量,不要一边念佛,一边胡思乱想,发现被念头带走了,要及时拉回来,这样觉察力才能逐步提升,下次念头上来时,就能及时发现,马上念佛。这位戒友才戒了十四天,一切才刚开始,黄瘾重也是过去不断看黄导致的,现在戒色后要学会对治想看黄的念头,坚持学习戒色文章,坚持念佛,慢慢黄瘾就会变淡的。念佛持咒要注重日课的质量,更要注重实战,日课不能和实战脱离,否则迟早是要破戒的。坚持念佛,慢慢内心是会清净的,但之后念头还是会上来的,甚至会猛烈地上来,到时很考验断念实战的能力。我以前念佛,有一年左右的时间,内心真的很清净,甚至想主动起念头,都感觉吃力,清净到那种程度,后来戒到一定时间,念头又开始猛烈上来了,经过了不知多少次的断念实战,可以说是久经战阵了,后来才明白修心真的是如一人与万人敌,佛陀所言非虚,念佛也是要用于断念实战的。\textit{这觉的速迟与除的快慢是用功的力量问题。……假如妄念势强,觉而不能断,就赶快念佛。……念头是要来的,念头来了,不要睬,念头一来就把咒提起,那么这个样子念头就化了,就是用咒来化念,把它化掉,不是压念不起。(元音老人)} 元音老人关于断念实战的开示已经非常明确了,不管是断念口诀还是念佛持咒,都是为了断念,断念是根本,觉察是关键。
\end{case}

\subsubsection{实战中的问题}

\paragraph{断得太慢,很容易跟念头}

断速是修炼出来的,一次次被带跑,一次次拉回,成百上千次这样拉回,渐渐地拉回的速度变快了,觉察力变强了,断念速度就上去了。古德一直强调的就是,不怕念起,就怕觉迟,觉迟了,念头就起势了,就像火苗已经窜起来了,火越烧越大,欲火中烧,到时就完全失控了。一定要早发现,早断除,不要跟念头,要有斩钉截铁的断力,而且反应要极其敏锐,念头一来,马上就做出反应,要异常灵敏,不能念头一来,没有反应,却跟着念头跑了,缺少觉察,这样就容易破戒。断速要达到零点一秒,现在如果做不到,坚持练习,将来某一天就能做到的,那是充满爆炸力的断速,快、严、烈、狠!

\paragraph{警惕性不高}

警惕和觉察是连在一块的,没有警惕性,也就谈不上觉察了,要警惕,但不能过于紧张,这个要自己调节。如何找警惕的感觉,打个比方,就像过马路时,警觉地看着汽车,但身体还相对比较放松,就是那种感觉。身体不要紧绷着,那样很容易累,警惕性不高,就容易被念头带跑,念头是带跑专家,你必须时刻严防,时刻开启观心模式,就像监控一直开着,坏人进来了,就去抓住。高手的警惕性高出普通戒友一大块,这是练出来的,成千上万次的实战磨练,警惕性越来越强,对念头的出现极其敏感。

\paragraph{因对抗产生的压念}

这个问题上面讲过了,对念头有抗拒心理,不想让念头起来,想压制念头,这就会导致压念,压念会导致念头反弹和挫败。断念的方法就是觉而化之,如果平时念佛持咒,也可以马上念佛持咒来转,也可以通过思维对治,这几种方法都是可以的,关键就是熟练,搞懂原理,练到纯熟的境地,到时实战时自然就有力量了。斗争要讲究技巧,不能蛮干,压念完全就是在蛮干,正确的斗争方式是觉而化之,或者马上念佛来转,也可以思维对治等。

\paragraph{舍不得断,贪恋太重}

这种情况就是思维对治没到位,对女色太贪恋,对邪淫危害认识不足,自己平时要多学习戒色文章提高觉悟,把观念转变过来,要深刻认识到贪色的危害,要多思维不净观,看破女色,其实就是那张皮而已,揭开皮没一样让人喜欢的,放大镜看那张皮也是充满污垢的。\textit{看见画皮没有?画皮戴上了,好像美女,要是把画皮一揭,是具骷髅。类似的问题,是让你渐修、渐观,渐观久了,可以克服你的贪欲心。(梦参法师)} 我戒到现在,看过很多大德的开示,提到戒邪淫这方面,大德法师们基本都在讲不净观,的确可以对治贪恋,关键要正确理解,不要误解不净观,就是思维观想一下,不一定要看太恶心的图片。

\paragraph{断了一个,又来一个,缺少威慑力}

这个问题也多有反馈,好像断不光一样,一波又一波的邪念往上冲,如果觉察力强大就不怕了,就像你掌握了强大的激光武器,再多的念怪,都可以瞬间消灭,而且会形成强大的威慑力,你这一发狠,心魔就发抖,关键是你能狠到什么程度?你的实战实力达到了什么级别?我曾经也遭遇过这种情况,就是断了一个,又来一个,后来觉察力强了,一般不出三个,邪念就不敢上来了。练到一定程度,你会发现自己已经具备威慑力了,只要稍微一发狠,那边就不敢动了,这有点像打群架时,这边有一个不要命的狠人,那边的人就不敢轻举妄动了。也像张飞暴喝,吓退曹军一样,威慑力真的很重要,是实力和级别的象征。当雄狮走出来,那边一群野狗,没一个敢上的,这种威严和级别,戒者要修炼出来。

\begin{case}[实战中的问题]
    为什么会断了很多次然后念头还是接着来呢?就感觉潮水一样!断了一个又来了!该怎么应对这种情况呢?
    \subparagraph{解析} 实力不济时,可以断念加转移注意力,出去走走,避免独处。实力强时,根本不怕它,几下就打成躺尸了,动不了了。就像拳击,打退了对方的进攻,对方一会又上来,如果你实力够狠,一记重拳直接把对方打成躺尸,解说员大喊:“It's over!”直接 KO,直接打晕毫无还手之力,要狠到这个程度,一记觉察重拳,直接把心魔打成躺尸,要努力让自己具备这个实力,做断念的重炮手,强势终结心魔的进攻!做断念的狠角色,闪耀八角笼!世界第一大力士马瑞斯重拳 KO 柔术高手格雷西,那场比赛我看了很多遍,开场没多久,马瑞斯一记重拳直接把格雷西打倒,加上几下狂暴的砸拳,比赛就在解说员声嘶力竭的疯狂叫喊中结束了,那场比赛马瑞斯展现的气势和力量,让我很震撼,我觉得我们断念也应该具备这样的气势和力量,真正的戒色高手不会陷入和心魔的缠斗,而是一记觉察重拳结束战斗,我们要努力让自己具备这个强大的终结能力。
\end{case}

\paragraph{未战先慌,未战先怯,不够沉着冷静}

这就是实力不行,实战经验缺乏,实力不够,面对心魔就没有把握,很容易被攻陷,实力强大,就很笃定,仿佛一切尽在把握之中,一点也不慌,一点也不怯。大家都是学生党过来的,如果自己把知识点都掌握了,到时考试时就有把握了,能考出高分,如果自己没复习,很多知识点不懂,到时面对考试真的没把握,内心也很慌。戒色也是如此,要通过学习提高觉悟,要练习断念来强化实战水平,当你具备实力了,你就可以决胜实战,不会慌乱,我现在每次实战都很笃定,不怕心魔,我已经具备取胜的把握了。

\begin{case}
    六年的戒油子,目前戒色 150 天(潜力无限),说下我的一点感悟吧:戒色,终究还是要拿实力说话,你打不过心魔,就是因为你太弱,必须不顾一切地强大起来。严格训练,实战第一!大道理谁不会讲,真要做到的有几个?经过五年屡戒屡破的沉淀,我现在更喜欢拿实力说话,用行动证明。是骡子是马,拉出来溜溜!要说戒色方法,我只会一招:南无阿弥陀佛。别小看这六个字,这一招,我可以把它耍得出神入化。什么怂恿、图像、微妙感觉、意淫、贪恋、侥幸心理、看黄的好奇心、怀疑动摇等等,来一个,杀一个!来两个,我杀一双!一起上好了,没再怕的。
    \subparagraph{解析} 这位戒友说得很好,他是走念佛路线的,能把一句佛号念熟,完全可以决胜实战,关键是下功夫,而且要正确理解念佛实战的原理。念头可以千变万化,而我们可以一招制敌!自己要具备识别能力,发现要快,马上念佛来转,这样就可以立于不败之地。他的这段话也表明了他的自信与底气,戒色的确要拿实力说话,必须让自己强大起来。严格训练,实战第一!这八个字说得很好。他之前五年屡戒屡破,也是戒油子,后来他开窍了,特别注重实战,注重练习,他的实力提升了,突破怪圈了。这个逆袭案例很好,我们必须以实战为核心,实战不行,迟早要破戒,最后你会发现其他方法只能管一时,最重要的还是断念实战,把握了这个根本,就能进入极稳定的戒色层次。
\end{case}

\subsubsection{各种断念方式的比较}

断念口诀,这是我主要推荐的,普适性最强,不用信佛也可以练习,进入高级阶段,只需觉察,不需起念。正确理解,坚持练习,就能达到较高的境界,这个修心诀是无数大德推崇备至的,传承上千年,讲的其实就是观心,观心是大总持,是根本。\textit{观心是佛法的根本法门。(元音老人)} \textit{修行——观心——是个根本的法门。(净慧长老)} 对于断念口诀一定要正确理解,正确练习,这两点很关键,如果产生误解,盲目练习,没把握要领,没抓住重点,那就很难取得效果。任何方法都要正确理解,理解是第一位的。我很看重悟性,有悟性很快就能明白前辈在说什么,能牢牢抓住重点,这样练习起来就能渐入佳境,突飞猛进。

念佛持咒,有佛力加持,当然也需要自力行持,两方面都需要的,不能马马虎虎念佛,要认真完成日课,也要正确理解念佛实战的原理,注重念佛实战。断念口诀快在它不需要起念,而念佛持咒在于佛力加持,两者都需要发现快。有的戒友做恶梦醒不过来,马上念佛就能醒过来,这种经历我也有过好几次,念佛持咒的确很殊胜。念佛持咒可以让内心清净,关键是你行持的力度,要精进,保证日课不能中断,坚持几个月,效果就出来了。念佛持咒适合有佛缘的戒友,有的戒友可能不大能接受,可能机缘还不成熟。有佛缘的戒友最好能修净土宗,每次做完日课回向求生西方极乐世界,这是末法时代最好的选择,我自己修的就是净土宗,每天有念佛持咒,念佛持咒和断念口诀都很好,我两者都很喜欢。

思维对治,当邪念来了,也可以思维对治,思维不净观,思维邪淫的危害等等,思维对治也是很重要的,它的长处就是可以对治贪恋,可以让你警醒,一般都是搭配思维对治,因为对治贪恋和思维危害是非常重要的,这个部分是不能缺少的。

还有断字诀、呸字诀等,也很好,适合无人之时,也可以用一用,一字诀喊出来是很有气势很有威力的,这点是其他几种断念方式不具备的。元音老人在开示中喊过“断”,黄念祖老居士在开示中也喊过“断”,喊断的开示是我特别喜欢的,因为断念真的太重要了,那个断字极具威力,从大德口中喊出来,感觉特亲切,让我激动万分,因为我知道自己听到了最核心的开示了。喊断的开示我会反复听,听几十遍、上百遍都不会厌倦,因为真的太殊胜了。喊断的那一刹那也极具戏剧性,似乎一切都停止了,似乎空气都凝固了。

一般就这几种主要的断念方式,每种断念方式都很好,都各有特点,可以综合使用,并不矛盾。其他观呼吸、邪念来了看大德开示、适量锻炼等也有一定效果,可以配合使用。

\begin{case}[各种断念方式的比较]
    三年了我终于学会戒色了,三年前在快十年的找黄看黄狂撸滥撸只剩下一副快要毁灭的躯壳时,戒色吧接引了我。那一次我凭着初心猛烈,一口气戒了两百来天。期间猛学戒色知识,对戒色感到新奇和新鲜,长达十年的黄脑一下子被一道瀑布给洗刷了。而且那时我自我觉得已经得上神经症了,只可惜后来还是经验不足,加之也骄傲了起来,又萌发了寻找刺激的念头,再次被心魔俘虏。自破戒之后再也找不到最初的戒色状态了,虽然一直在学习,因为不得要领,总是在断断续续破戒,最多只能戒三个月,那还是在每天记戒色笔记的情况下,一般则难挨过一个月。在有一阵子没破的情况下我把自己认为是一个不会撸的正常人,以为以平常的心态去工作,去处理日常,不用再学习戒色了,没想到正中心魔的下怀,看了那么久的《戒为良药》现在我终于明白了,戒色即实战,就是断念,杀心魔贼,我是纯粹的觉知,那个怂恿我去找黄看黄不是我,每次都是阵亡在同一个圈套,仿佛都走不出那个怪圈了。现在我终于知道了,我还会一如既往不断地看飞翔大哥的《戒为良药》,听他讲实战经验,学以致用,知行合一。最后汇报我的战绩,现在又一次冲刺到半年了,但这次不一样,我已经不会任心魔宰割了,比第一次的两百天要稳固。戒色就是每天学习戒色实战经验,每天断念。
    \subparagraph{解析} 这位戒友之前的破戒经历很典型,因为经验不足,骄傲,觉得自己不用再学习戒色了,结果就破戒了,心魔一直在虎视眈眈,必须注重学习,注重断念实战,不能放松警惕。戒色是终生修为,戒到一定程度可以减少学习,但不能骄傲和放松警惕。三年了,这位戒友终于学会戒色了,期间的心路历程也很坎坷,戒过两天,再陷入怪圈,后来坚持学习终于顿悟,领悟了戒色要注重断念,真实身份是纯粹的觉知,相信他这次会戒得更好。
\end{case}

\begin{case}[各种断念方式的比较]
    本人已经戒色 330 天了,其中在 176 天时心魔怂恿破了戒,又发愤图强戒了 154 天,刚好 330 天。在这里感慨万千,如果不是有戒色吧,如果不是有大家,有飞翔大哥,或许我还是个 loser。在大家的帮助下,我离这种习气越来越远了,变得越来越好了。在这里我要告诉大家的是,断念至关重要,修心才是根本。
    \subparagraph{解析} 这位戒友戒得还不错,176 天出现破戒,被心魔怂恿了,后来他发愤图强戒到了 330 天,最终领悟了断念的重要性,修心才是戒色的根本,修身为辅助。一位戒友说:“很多人都在找最好的戒色方法,方法也很多,但个人试过很多方法,最终觉得,最好的方法就是断念!念起即觉,觉之即无,一发现淫念来了,马上切断、不跟随,淫念就消失了!我自己以前也是破戒无数,后来专心看《戒为良药》,重复看、反复看,才慢慢戒成功了!”另外一位戒友说:“如果忽视了修心和断念实战,看起来再好的方法也是虚的,哪怕天天锻炼,哪怕我自己也是学生物的,对危害了然于心,照样被心魔按在地上揍!断念是根本,这是我的血泪教训啊!”看似很好的方法,但如果偏离实战,必定是要失败的。念佛持咒是要用于断念实战的,仅仅完成日课是不够的,关键还是要看断念实战,实战永远是核心,就像打仗是要上战场的,不能纸上谈兵!心魔套路很多,要知己知彼,熟悉心魔的套路,这样才不会上当,实战时才能成竹在胸。
\end{case}

\subsubsection{断念口诀的念法}

\begin{multicols}{2}
    \begin{itemize}
        \item 念口诀要尽量清晰,有节奏,让自己耳朵听到,不能念得太快,自己听不清楚,一般建议小声念,默念也可以,默念是用心去听。
        \item 念口诀不能心浮气躁、急于求成,每念一次要认真体会一下口诀的含义,但不要去思维,只是去感受。
        \item 念口诀不能马虎敷衍,要认真对待练习,拿出热情来练习,那个断字和觉字,可以用短促的重音来念。
        \item 每天坚持练习,不能中断,注重积累和坚持,注重实战。暂时看不到效果也不灰心丧气,有信心继续坚持下去。
        \item 从每天五百遍开始,其他时间有空也可以练习。不要紧张,要身心安定来念,感觉紧张了,就放松些。
        \item 念的过程中,会出现被念头带跑的情况,不要气馁,及时拉回来即可,关键是拉回来的那一下,要快。
        \item 断念口诀练的就是觉察力,发现越快,拉回来越快,实战时就会越强。明白原理,坚持练习,就能变强。
        \item 平时念十六字,这十六个字是一个整体,实战时可以只念“念起即断”,这样比较方便。念到“断”字要狠一点,就像榔头砸下去。
    \end{itemize}
\end{multicols}

先熟背断念口诀,然后平时可以用普通念头来练习,也可以用其他负面的念头来练习,要练到一出现念头马上就念出断念口诀,反应要极快,达到条件反射的地步。

\paragraph{tips 1} 对于上脑极快、会引发强烈冲动的念头,可以在觉察的时候配合左右小幅度快速晃头一两次,这个实战技巧可以有效把念头晃掉。就像把糖放入水中,一晃就消融了,念头来自于空,晃一下就能消融回空,我称之为“消融之晃”,之前看到有戒友在使用这个技巧,这个实战技巧我也经常用,既是晃,也是甩,把念头甩掉。幅度不要大,很小的幅度就可以。

\paragraph{tips 2} 介绍一个训练注意力和专注力的方法,盯着一个物体看,比如一支笔,盯着它看,十秒,不许有念头,有念头就算失败。可以称这个训练方法为“挑战十秒”,如果觉得自己做不到,就从三秒、五秒开始,要纯粹地看,你的真实身份是观察者,这个看的力量要不断强化,不要让自己陷入念头。觉察力的本质就是一种注意力,所以训练注意力可以有效提升觉察力。练习一段时间,你会发现这个看变强了,变集中了,变持久了,不断强化这个看的品质与强度。

\paragraph{tips 3} 快速反应训练,可以下个反应速度测试的应用,平时有空的时候可以玩一玩,对反应速度的提升是有一定帮助的,一般是 0.2 秒、0.3 秒居多,能练到 0.1 秒就非常棒了。念头上来时,必须快速反应,要第一时间觉察消灭或者马上念口诀、念佛,反应必须要快!

\subsubsection{断念后安住纯粹的觉知}

断念后会出现一个空当,一个间隙,那正是无念的真我,也就是纯粹的觉知,我们要学会安住其中,也就是第 99 季讲的无敌模式。在《灵性的实相》那个视频里,讲到了处于这个状态时,就能接受宇宙能量,念头是阻碍,是要断除的。安住一会儿,就会有妙不可言的感受,的确是一种很棒的享受,你会发现其中有莫名其妙的快乐与满足,因为你处于真我的状态。埃克哈特·托利在书中说过“注意那个间隙”,那个无念的间隙,那个念与念之间。如果你不具备断念的能力,那是很难安住的,因为一个念头接着一个念头,你几乎找不到间隙,如果你具备强大的断念能力,就可以通过觉察来创造间隙,然后安住间隙。就像喝茶一样,你可以品味那个间隙,那个间隙的感觉真的妙不可言,一切尽在不言中。刚开始你也许感受不到什么,甚至有点无聊,持续安住,奇妙的感觉就来了,那个内在的空,看似平淡,实则让人回味无穷,安住得越深,感觉越美妙。有一种惬意就是安住真我,享受内在的宁静与美好,这是最高级的享受,而且还是免费的。茶和禅都是一个味道,回归天真、单纯、简单、没有分别的状态,做真正的自己,才能真正地快乐起来。

安住的方法:短时多次。一开始只能安住几秒,然后就被念头带跑了,这时不要气馁和感到烦恼,每次就从几秒开始,就像滴水一样,坚持滴,就可以滴满一浴缸,也可以水滴石穿。慢慢安住的时间会自动延长的,安住的感觉也会加深,会变得稳固。通过这个方法,你的觉察力也会得到很大的提升,已经有很多戒友在这样做了。这个方法也可以用于实战,当念头来了,你马上就安住纯粹的觉知,开启无敌模式。当你安住真我时,你就在散发光、爱、美、欢乐、和谐与优雅,你会越来越喜欢安住真我,真我无味乃有至味!

一般静坐一段时间,比如半小时,入静程度比较高,整个人就会感觉喜悦、清新和明亮,因为接受到了宇宙能量,身心焕然一新,眼睛特别喜悦,整个人就像全新一样。

我坐电梯时也会让自己安住纯粹的觉知,生活中有很多这样的时刻,可以充分利用起来,每次安住都是在变得更强,更有觉知。

\paragraph{tips 1} 突然喊“停!”就像我们小时候玩的游戏“一二三,木头人!”不许动,身体完全静止,动作也停在那里,脚抬起来就定在那里,手前摆就停在那里,这里加一条,喊停的一刹那,念头也停止。就是双停,动作停,念头停。试下这个技巧,挺有效,因为我们有时候会习惯性地跟着念头跑,陷入念流而不自知,这时候突然喊停,脑子一下就安静了,一下就从念流里抽出来了,同时进入了纯粹的觉知。喊停,可以小声喊或者心里喊,不一定要发大声。这其实是禅宗的一个技巧,后来就演变成一个小孩玩的游戏了。

\paragraph{tips 2} 逆腹式呼吸是指吸气时腹部自然内收,呼气时小腹自然外鼓。逆腹式呼吸可以扩大肺活量,改善心肺功能,练习时舌抵上腭,做一次华池水就下来了,缓缓咽下去,华池水是人体中的一大奇宝,是人体自产的大药,古人也称之为玉泉、天泉、神泉、神水。我每天都会练习逆腹式呼吸,对于调息安神很有效果,练习后感冒少了,感觉体质增强了,精力也改善了,那一口华池水咽下去就像吃仙丹一样。练习的要点:\begin{multicols}{2}
    \begin{itemize}
        \item 呼吸要深长而缓慢;
        \item 呼吸要深长而缓慢;
        \item 一呼一吸掌握在十五秒种左右,即深吸气三到五秒,屏息一秒,然后慢呼气六到十秒,功力深时可以适当地延长时间;
        \item 练习时背部要挺直,不要低头;
        \item 可以选择站姿或坐姿,尽量放松身体。
    \end{itemize}
\end{multicols} 以上五点是逆腹式呼吸的普遍要点,我的改进版是,吸足气,屏息一到两秒,呼尽气,再屏息一到两秒,这是一次,然后再做下一次。刚开始有的人身体差,可以不屏息,慢慢练习一段时间感觉肺活量富裕了,就可以尝试屏息。屏息是一个最关键的要点,在《灵性的实相》里,讲到了神圣本源是一个没有呼吸也没有念头的状态,无呼吸比无念更进一步,达到了绝对的静止。我们要试着安住这个状态,但不要刻意去憋气,而是学会屏息的同时保持适当的放松,不需要屏息太久,身体可能会出现不适。通过这个方法,安住的感觉会更强。安住纯粹的觉知,不一定要屏息,但屏息是一个更高的技巧,可以交替使用。

\paragraph{tips 3} 发现自己陷进念里了,快速用力拍手三下,就能出来,就能回归纯粹的觉知。觉察力强的话,在发现时就出来了,如果陷入很深,念头惯性很大,念头动得很厉害,这时拍手就是很好的选择,拍手可以有效提起注意力,打断念头,还可以振奋精神。

\subsubsection{如何战胜微妙的感觉}

微妙感觉是实战的难点,很多戒友都反馈栽在微妙的感觉上,要具备更高的警惕和更强的觉察力才能战胜微妙的感觉。微妙的感觉是念头更细微的形式,就像水有液态、固态和气态一样,微妙感觉就像气态,你能感觉它来了,也能知道它的含义。没有明确的念头或图像,只是一种非常细微的感觉在渗透进来,就像你坐在家里,突然闻到一股煤气的味道,这时你就要高度警觉了。微妙感觉一出现,马上念口诀或者念佛号,觉察力强的话,直接觉察消灭,反应一定要快。我自己也有无数次的微妙感觉的实战体会,就是突然一种很微妙的感觉来了,就像未成形的念头,类似一种气体的侵入,我能明确感觉到它的含义,我只要及时发现它,它就很难继续下去,一般在觉察后,它就会消失,有时还会残留一些感觉,我安住一下无敌模式,就彻底消散了。微妙感觉就像打仗时的毒气弹一样,虽然不是子弹,但危害也很大,对于这类进攻方式要格外警惕。

\begin{case}[如何战胜微妙的感觉]
    十几天前戒得好好的,一点念头都没有,突然在图书馆的一瞬间,有一个微妙的感觉,之后自己开始惶恐,结果就破了,那一天真的人都是懵的。之后第二天起来就发热,头昏,又不像是感冒,人特别难受。
    \subparagraph{解析} 微妙感觉会在一瞬间把人罩住,就像一个屁把人罩住一样,屁是很微妙的,看不见,摸不着,微妙感觉也类似这样。出现微妙的感觉,不要怕,对于这种进攻方式要沉着应对,心魔的几种套路都要熟知,这样临战时就不会慌了,心魔主要的进攻方式就是念头、图像、微妙的感觉,这三板斧!然后会怂恿你,乃至让你怀疑动摇,这些都是常见的套路。这位戒友一个微妙的感觉就让他惶恐了,他肯定意识到心魔进攻了,自己抵挡不住,实战时没把握,很慌乱,结果就破了。打仗时有的新兵会被吓得手足无措甚至尿裤子,但老兵很淡定,对于敌人的进攻方式都了然于胸,仗该怎么打,都非常熟悉了,一点也不慌。微妙的感觉一来,要及时发现,不要认同,不要跟随,只是看着,它就会消失,觉察力强,看见的一瞬间就消失了。
\end{case}

\begin{case}[如何战胜微妙的感觉]
    飞翔老师,继昨天破戒之后今天又被邪淫回忆的微妙念头攻破了,感觉每次破戒以后定力会下降很多,遇见心魔没有必胜的信心了,然后那种微妙念头像一条蛇一样缠绕不断,唉!每次破戒好像都会连续破戒直到掏空为止,被邪淫图像的微妙念头没完没了地纠缠,一直抵抗最终实在是扛不住了……哎!
    \subparagraph{解析} 这个案例虽然不是微妙的感觉,但念头本身也是微妙的,邪淫回忆往往印象很深刻,也是一下就上脑了,速度很快。我最近也遭遇了好多次这类进攻,有时甚至在念佛时都会跳出来,就是过去放纵的回忆,先是一幅回忆的图像一下出现在脑海中,然后就开始变成回忆的小电影,如果断得不快,就会出现漏的感觉,然后身体就会出现一些不适。能跳出来的邪淫回忆一般都是极度诱惑的,才能变成很深刻的记忆,过了这么多年还能跳出来拉人下水。这位戒友两个“哎”,表明了他的无奈,我很理解他的心情,毕竟是连续破戒,心情的确很糟糕,很无奈。也让我想起了“平时多流汗,战时少流血”这句话,平时要好好备战,练为战!战为胜!努力提升自己的实战水平,这样才能决胜实战,否则水平不行,实力不济,到时真的要被心魔虐成狗。
\end{case}

\begin{case}[如何战胜微妙的感觉]
    唉,心魔真的是魔王级的,太难办了。前段时间心魔进攻不猛,就是一个念头来了,断掉,然后隔一会念头又来,那段戒得比较轻松,心里也比较高兴,感觉自己还是可以和心魔抗衡一下的。结果这几天心魔疯狂进攻,我体会到心魔大举进攻时有几个特点,首先是邪念一个接着一个,不会间断,其次心魔会三种方式都用,比如来一个图像,再怂恿你,再来点微妙感觉,我一下子就蒙了,然后就被攻破了,这两天欲火中烧,还好没撸,但漏了很多,也不亚于破戒。心魔真的难办,我还需要更多练习断念。
    \subparagraph{解析} 心魔也可以说是拳王级的,一套组合拳,这位戒友就趴下了,心魔的进攻火力的确很猛,有点像泰森的组合拳,不过心魔更狡猾和阴险,它会怂恿你,特别是针对你的思想误区来怂恿,无所不用其极,怂恿的内容也是五花八门,甚至会有很多荒唐的怂恿,如果没有坚定的立场和较高的觉悟,很容易就会中招。大概一个月前我看了一场格斗比赛,那场比赛的取胜方式就是——站立拼拳!这四个字我记得很深,比赛一开始,就是拼拳,没有任何花哨的动作,两个人就直接拼拳,大概半分钟就决出胜负了,让人看得热血沸腾。我们平时就要把觉察重拳练强练快练狠!到了实战时,心魔要来组合拳,你一记强悍的觉察重拳就把心魔撂倒了,满场欢呼!百万戒友为你的实战表现点赞!解说员也陷入了咆哮般的疯狂呐喊:“It's over!unbelievable!”你的实战表现要让心魔闻风丧胆!!!不怕贼强,只要将猛!正所谓:戒色虎将杀气横,万千邪念瞬齑粉!气势汹汹心魔来,一记觉察成躺尸!
\end{case}

\begin{case}[如何战胜微妙的感觉]
    飞翔哥好,我现在就是天天练习口诀,对于普通 YY、图像能秒断,但一到欲望高峰期就感觉到一种蠢蠢欲动的微妙感觉轮番进攻,没实质性的念头只有微妙的感觉,但是上来极快,瞬间就能让人起很强的冲动,然后伴随着那种微妙的感觉,紧跟着就是图像、怂恿,感觉断得力不从心。主要还是微妙的感觉一上来,欲望一下就起来了,不知道这种情况该怎么克服?之前破戒基本都是这种情况,微妙感觉一来,立即有种欲火焚身的煎熬感,前后时间估计都不到一秒吧。
    \subparagraph{解析} 这位戒友对于普通 YY、图像能秒断,但对于微妙感觉还不行,对付微妙的感觉,必须具备更强的觉察力,还需要对这种进攻方式有充分的了解、熟悉和敏感,要保持更高的警惕。微妙感觉上脑极快,就像一团气体上来,我们能读懂微妙感觉的含义,一般都是非常诱惑的内容,能让人瞬间起很强的冲动。实战中我也遭遇过无数次,我也是深有体会,关键还是觉察,觉察力强,觉察速度快,就能克服,觉察力弱,觉察慢,就会很被动。实力继续提升,就能战胜微妙感觉,就像头几次打 BOSS 一般都会失败,甚至会失败几十次,之后实力提升了,摸清套路和规律了,就有战胜的把握了。断念一定要快,真正的顶尖高手都是零点几秒就断掉了,反应要极快,因为我们的对手也非等闲之辈,上脑速度极快,我们的断速也要极快,才能跟得上实战的节奏。
\end{case}

\begin{case}[如何战胜微妙的感觉]
    飞翔哥,我昨天晚上八点左右经历了一场实战,那个念头有微妙的感觉,跟它跑了一小会,微妙的感觉就开始附体慢慢发展壮大,等它壮大了,我再断念肯定是不行的,就算断掉了也很难受,所以我就立马断念不跟随观察念头,把这一波念头干掉了,我就躺床上睡觉了,最后念头还一波波上来企图控制我,我都不跟随直接观察念头,后来我就睡着了,我觉得躺床上观心断念是很容易入睡的。
    \subparagraph{解析} 断念就是头脑里的暗战,表面看这个人好像没发生什么,但只有他自己知道他脑子里究竟发生了什么!一场惊心动魄的实战!如果失败,就会沦为疯狂的撸管肉机,不掏空不会罢休。好在这位戒友已经具备相当强的实力,虽然跟了一小会,但立马断念了,保持观察,不跟随,他做得很好。入睡前可以保持观心,这样更容易入睡,一般入睡前容易胡思乱想,这时正好练习观心,一会就睡着了。\textit{妄念生起时,单纯觉知它,妄念就会失去力量。(大宝法王)} 当微妙感觉出现时,就直直看着它,它的力量就会削弱,就像用强光照射一个动物,那个动物就无法动弹一样。关键就是觉察、觉知、觉照,发展看的力量。及时发现,强烈地看,看那个微妙的贼,在强烈的注视下,贼就动不了了,突然就消失了。你不在看,就在跟念,你在看,就等于和微妙感觉拉开了距离,就像一束激光照射在了微妙的感觉上,最终结果必然是微妙感觉消失。
\end{case}

\paragraph{总结}

古人云:“一念之欲不能制,而祸流于滔天。”戒色的核心是修心,要在念头上修,之前有的戒友听信了诋毁言论,不注重念头管理,结果破得一塌糊涂,这时他才知道戒色真的要注重断念实战,否则肯定会破戒的。还有一个戒友本来戒了两年多,听了诋毁的言论,看擦边图意淫,不断念了,结果连破两次!诋毁的文章真的不能看,会误导你,让你不注重断念实战,让你对断念口诀产生严重的误解和偏见。我们对断念口诀要有绝对坚定的信心,这个口诀传承上千年,无数大德一致推荐,是最值得练习的修心诀。一位戒友说:“戒色一年多最终还是破了,就是没有在断念上下功夫,我终于体会到了断念非常重要,超级超级超级重要,是戒色的核心,断念不行,必有破戒的一天。”另外一位戒友说:“我花了四年半的时间才明白断念的重要性,多么痛的领悟,这个领悟来之不易。”还有一位戒友说:“2018 年注定是难忘的一年,上半年放松学习,不练习断念,结果夏天止不住地破戒,挡都挡不住,下半年强势崛起,如今戒了快五个月了,真的越来越好,发誓戒一辈子,再也不碰手淫这害人的玩意!”不管是口诀、念佛、思维对治等,都是为了断念!断念就是实战的核心,平时一定要好好坚持练习,关于断念的文章也要反复研究,多做笔记,多复习。对于暂时的失败,不要气馁和灰心,不要自暴自弃,要有一个正确的态度,认真反省和总结,从失败中吸取经验教训,继续加强学习补强觉悟,失败后再学习,往往能吸收到更多的东西。关键要学会慢慢提升自己的实战能力,这样最终才能做到降伏其心。一个人一开始是无法具备问鼎冠军的实力的,在比赛时也会遭遇多次败北,但还是继续系统的训练,注重慢慢提升自己的实力,不会因为失败而放弃,每位冠军的心路历程都是这样的,都经历过失败,关键是学会培养自己的实力,强到一定程度,就可以了。说实话,我一开始也是菜鸟,也很无知,就是通过学习和练习强大起来的,我很注重积累,慢慢实战能力就提升了,超越一个临界值,就会发现心魔很难攻破自己了,这就是一条练级之旅。一开始被虐是正常的,因为你弱小,所以注定被虐,等你强大了,剧情就翻转了,你开始完爆心魔了。

不注重学习、不注重断念实战的戒色方法,必将失败。这句话是我说的,是我深入研究戒色这么多年、看过无数案例得出的结论。一位戒友说:“我最高戒过两年半多,那时候我认为我的方法是最好的最完美的,我也不会将我的方法告诉任何人,直到 2018 年底我破戒了之后进入了破戒怪圈,我发现我之前用过的所有方法都没用了,我开始绝望地拿起《戒为良药》反复地查看,最后飞翔大哥的话点醒了我,所有戒色方法只要不注重观心修心断念,都会失败的。”修心永远是最核心,也是无数高僧大德反复强调的重点,偏离了这个核心和重点,肯定会破戒。念头是行为的先导,修心是最根本的,即使禅宗破初关之后,还要继续修心对治,保任除习气。某些人会说自己戒了多少年,甚至有说自己戒了十几年的,但是他的戒色方法完全偏离修心,这样的戒色方法是存在误导和缺陷的,也存在很大的水分,偏离修心的戒色方法是不可能戒十几年的。戒色界也是鱼龙混杂,有的人动机不纯,还想靠戒色赚钱,然后夸大自己的戒色方法。大家如果真正把握了戒色核心,有了很高的觉悟,就可以分辨哪类戒色方法存在水分。具备正见真的很重要,有正见的人不会随便动摇,因为他已经把握了核心,就像内心有了定海神针一样。

很多事做过头或者认识上有误区,都可能走火入魔,包括修行方面也有这样的案例,屡见不鲜。对于断念口诀要正确理解,认真练习,但不要走极端,不要过于紧张,出现紧张时,要学会放松调整。练习断念不要急躁,不要急于求成,要静下心来坚持练习,保持理智和耐心。同时,也要处理好自己的生活,不要影响到学业或事业,自己要做好时间管理,好好安排自己的生活。先学习断念的理论,明白断念不是压念,这样念头来了,就不会试图去压制,压念会导致压抑和挫败,会导致各种烦恼,越想压,越不行,搞得自己很苦恼。断念是觉而化之,也可以念口诀或者念佛号来转,以一念代万念,也可以思维对治,思维不净观,思维邪淫的危害等。断念的理论先认识清楚,掌握原理真的很关键,这样就不会走入误区。我戒到现在八年多,能让我立于不败之地的就是强悍的断念水平,还有就是德行的培养,综合觉悟的提升。只要真正下功夫去做,是有望戒掉的,很多戒友都成功了。断念口诀是一个金口诀,希望大家重视起来,认真学习和练习,特别是戒油子,他们的特点就是实战差,相信戒油子看了这篇文章会有所醒悟,戒油子也有春天,戒油子也可以开出一片金光灿烂的油菜花!

下面分享一首诗歌。

\begin{poem}[春风里的戒者]
    \begin{multicols}{3}
        \begin{center}~\\
            不再看黄了 \\ 不再放纵自己的欲望 \\ 从那个堕落的世界里 \\ 抽离 \\ 一切都开始变得纯粹 \\ 内心也美好起来 \\ 这种感觉很久违 \\ 眼前的世界似乎由灰色 \\ 渐渐变回了彩色 \\ 生命开始变得鲜活 \\ 看着镜子里的自己 \\ 充满灵气的眼睛 \\ 明亮而坚定 \\ 一个纯真无邪的笑容 \\ 仿佛回到了八岁 \\ 满满的少年感 \\ 春天来了 \\ 万物复苏了 \\ 绿柳儿随风飘舞 \\ 花儿也绽放了 \\ 纯净的灵魂徜徉在春风里 \\ 只是存在,不思考 \\ 用神圣的临在 \\ 去感受春天的美好 \\ 我睁开双眼 \\ 看见一望无际的花海 \\ 闪烁着耀眼的光芒 \\ 孩子们在欢快地奔跑 \\ 我幸福地笑了 \\ 那种纯粹,那种美好 \\ 刻骨铭心 \\ 春风温柔地掠过我的脸庞 \\ 就像一双无形而慈爱的手 \\ 在抚慰我的灵魂 \\ 当春风抚慰我时 \\ 我感觉自己就是纯真无邪的孩子 \\ 我站着闭上双眼 \\ 用心感受这神圣的片刻 \\ 脸上写满了幸福的笑意
        \end{center}
    \end{multicols}
\end{poem}

下面推荐一本书。

\begin{book}[《大国医》,王耀堂版]
    好久没推荐养生的书籍了,这季推荐一下《大国医》,这本书集合了三十位国宝级“国医大师”的养生绝活与治病经验,是一本可以增长养生知识的书籍,书中很多经验值得我们学习和借鉴,国医大师在养生方面的认识真的很深刻,真的很有一套,也是几十年经验的总结。我们戒色后一定要在养生方面多学习,多下功夫,这样身体恢复就能加速,我看过不少戒友就因为养生没做好,恢复不太理想,如果具备较好的养生意识,那恢复就能加快,恢复也会比较理想。关于大国医,有很多版本,王耀堂版还是不错的,网上也有 pdf 下载,不过书中的一些内容要选择性吸收,因为会推荐荤菜的方子,还有一些中医药方可能会助长欲望。有病最好还是去中医院就诊,具体诊断后再拿药服用,因为对症是非常重要的,每个人体质有所不同,要具体诊断后才能下结论。为人子弟不可不知医,很多年轻人是不懂养生的,熬夜、久坐、纵欲,年纪轻轻就把身体搞垮了。养生是一门很深的学问,古人是比较重视养生的,养生其实就是健康管理,在年轻时就要对养生有所专研和实践,要有养生意识,这样对自己将来的人生是大有裨益的,戒色后要认真研读养生的书籍,增长这方面的认识。
\end{book}
