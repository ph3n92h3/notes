\subsection{关于魔考、憋与射之间(附戒色考卷)}

\paragraph*{前言}

上季向大家推荐了树疗,不少戒友都跃跃欲试,有的戒友已经收到了良好的效果,这季关于树疗再补充一些内容,先分享两个反馈案例:

\begin{case}[树疗反馈]
    老师,来晚了。心魔攻击的次数越来越多,一直不断地怂恿我,都怪我,我开始戒色时没做好拼死的决心,现在心魔越来越强大,感觉快支持不住了。还有老师,您的那个树疗法效果很好,我就抱了六天左右,而且每天只抱了十分钟,感觉脸上越来越光滑,越来越细腻,开始恢复弹性了,我很高兴,再次感谢您,飞翔老师!
\end{case}

\begin{case}[树疗反馈]
    我最近四天抱了三天,每次都是一个多小时,最明显的变化就是贴过树的地方脸色明显的亮了起来,出现脏东西排出后的小坑,这部分的皮肤也没有这么油了,细腻的变化是显而易见的,树疗真的很神奇!不枉我三天来的努力。
\end{case}

第一个是戒友刘昂的反馈,第二个是戒友 Fizra 的反馈,所有反馈案例中,Fizra 是最有树缘的一位,他已经真正体验到了树疗的神奇效果,并且已经开始享受树疗的感觉,他的确很有悟性。当你开始享受树疗了,真正开始用脸亲近树木了,奇迹就会发生。上季我和大家分享了树疗的恢复方法,有的戒友处在观望状态,半信半疑,还有的戒友则是苦于没地方做树疗,很多学生党在学校是不大方便,如果同学看到了,可能会觉得你是怪人,然后你就成了学校的“风云人物”。所以,我一般推荐大家去公园,去人少安静处,尽量不要被熟人看见,这样心理压力就会少很多,其实当你习惯树疗了,也没有什么可难为情的。比如有人刚开始剃个光头,走在马路上,人家可能会指指点点,时间长了也就完全适应了,自己有自信了就不会在乎了,习惯成自然了。有的戒友怕白天被别人看见,说晚上行不行,我一般不推荐晚上做树疗,晚上阳气内收,植物也是如此,晚上做的效果肯定没白天好,另外,我们白天做树疗时,应该选择树的阳面,什么意思呢,就是树干被太阳晒的那一面,树和房子一样,有阳面和阴面,我的经验就是在阳面做树疗,效果会更好。

关于树的干净问题,有的戒友说马路边的树很脏,我一般不推荐在马路边上做树疗,因为马路边的树是比较脏,而且人太多,我们应该选择人少安静处做树疗。然后要注意选择树木,如果你仔细观察,就会发现每棵树的干净程度都是不一样的,同样是香樟,有的树干就很干净,有的则比较脏,我之所以推荐公园,就是公园污染少,干净的树相对多些。干净的树是要自己找的,当你找到属于自己的树后,就会慢慢和这棵树建立感情,当你和树拥抱时,树也可以给你一种心灵的慰藉,慢慢地,你就会真正喜欢上树疗,真正开始享受树疗的感觉了。

要改善容貌气色,必须用脸去亲近树干,分别是印堂、双侧正脸颊,以及脸颊的侧面,这几个部位轮流做。有的戒友说,不观想行不行,不观想其实也可以,不过我认为观想的效果会更好一些。有的戒友感觉无聊,也可以一边听 MP3,一边做树疗,这都是可以的,还有的戒友抱这棵树半小时,然后再抱另外一棵树半小时,这都是可以的,反正公园里的树很多。有戒友说我推荐的树疗和气功采气类似,气功采气我初中时就有了解过,一般采气要和树木保持一定距离,而我推荐的树疗则是主张和树直接亲密接触,其他的要求次之,灵活度比较大。

如何看树疗效果,相信实践过的戒友,都能感觉到这四个字:轻松愉快。和树亲密接触是可以调整你的心理状态的,是潜移默化的影响。心理压力大时,建议多亲近大自然,这样可以让你放松下来,是可以帮助你调整心态的,做树疗可以让你获得一种宁静祥和的气场,让你感觉轻松愉快,有一种愉悦的感觉,这是心理方面的效果。很多戒友因为撸管容貌被毁,很想恢复容貌,而恢复容貌的办法,我首推树疗,其次是有氧运动,最后才是护肤品或者物理治疗。关于树疗对容貌恢复的效果,我这季再详细讲讲,树疗是非常神奇的,它可以让你的皮肤气色恢复到发育前,很微妙很不可思议,有一种时光倒流的感觉。

具体变化如下:

\begin{description}
    \item[皮肤会变光滑细腻] 刘昂提到了光滑细腻,Fizra 也提到了细腻。皮肤细腻,一,是视觉的感觉。二,就是触觉,摸上去很滑很细腻,不粗糙,触感极好。这就是做树疗后的感觉,不需多久,每天一小时,坚持三天,不熬夜,马上就能感觉到这种微妙的好变化。我自己的感受也是如此,的确变得很光滑细腻,很神奇的感受。
    \item[清洁作用] 这一点说出来有点匪夷所思,要说最好的清洁办法,大家可能会首选洗面奶或者清洁面膜,但我可以告诉你,肌肤最好的清洁方法就是树疗。Fizra 就提到了树疗的清洁效果,脏东西会排出来,毛孔会变得很干净。这种干净不是表面的清洁,而是由内而外的清洁,树疗可以净化你的气场,调整你的内分泌,皮肤自然就会变干净。
    \item[皮肤气色会变明亮] 树疗会提亮你的皮肤气色,看上去会很阳光很有精神,这点 Fizra 也体验到了,这种明亮的感觉是非常明显的,一照镜子就能看出来。
    \item[控油] 树疗调整内分泌后,皮肤的出油问题会得到很大的改善,会慢慢恢复正常。
    \item[皮肤恢复弹性] 这点刘昂的体会里面有,皮肤会重新变得饱满富有弹性,那是小孩子才有的弹性,非常健康的感觉。
    \item[脸部去肿塑形] 撸管导致身体失调后,反映到脸部就很可能会出现微肿现象,五官形态会发生微妙的坏变化,而树疗可以大大改善这类问题,可以消肿并且紧致皮肤,甚至可以大大改善眼袋问题。
    \item[口有余香] Fizra 在我文章的回复里提到了树的香气,这一点我也有很明显的感受。记得有一次树疗,我在公园做了三个多小时,然后回家路上,感觉嘴里和鼻子里非常非常香,那种香气非常美妙,既是味觉的美妙体验又是嗅觉的美妙体验,这种香味可以在嘴里保持一天甚至几天,非常奇妙,也非常美妙。
\end{description}

树疗的好处多多,暂列这七点,当然上季我也罗列了一些好处。希望有更多的戒友去认真体验树疗,找到属于你自己的那棵树。真正用脸去亲近树木,当你把树疗当作一种莫大的享受时,就不会再感到无聊了,因为你很享受那种感觉,就像抱着树睡觉,同时也是一种莫大的心灵慰藉。当你真正去亲近树木了,很欢喜地亲近,奇迹就会发生,改变会让你惊喜。希望有更多的戒友把体验反馈给我,你的真实体验可以启发并帮助到更多的人。关于选择什么树木,就要看你所在的区域了,南方有南方的树,北方有北方的树,好树肯定有,就要看你自己的观察与体验了,多做几次体验,然后你就会找到你自己喜欢的那棵树。

下面分享一个答疑案例:

\begin{case}
    飞翔大哥你好,戒色已有一段时间,但一直有 JJ 流液体的毛病,像昨天和相亲对象只是打电话聊天也会流好多(并没 YY),然后今天就有睾丸坠胀腹部左下的位置胀疼的感觉,已经一天了好难受,之前也有过几次只是过一会就好些了。通过休息和坚持戒色会好转吗?还是需要去医院做个检查什么的?

    \textbf{答} 嗯,你的问题很多戒友都反映过,和女友打电话会条件反射地漏,即使没聊敏感内容也会漏的。所以有女友是戒色的一大障碍,只要一漏,那就很容易出现症状反复。我建议你注意休养,积极锻炼,这样症状是会缓解的。如果症状持续难受,那就应该及时就医检查治疗,考虑前列腺炎了。加油!

    \textbf{分析} 有女友戒色是比较麻烦些,而打电话漏,这种情况是比较多的,即使完全正常的电话也会漏,因为只要听见女人的声音那就很容易漏,属于条件反射。我以前的文章讲到过,不管漏什么,都有可能会出现症状反复。这个戒友就是如此,一漏身体马上就难受了。所以,有女友的戒友更应该加强学习,注意修心,在身体尚未完全恢复时,应该减少亲密,否则不利于你身体的恢复。老漏的话,可能还没结婚就废掉了,那结婚可能就是一场悲剧了,记得有几个已婚戒友就是因为废掉了才离婚的。
\end{case}

关于频遗问题,我再补充下:

在第一次遗精后要格外引起重视,要注意避免出现连续遗精,因为在第一次遗精后,固摄能力已经下降,所以很容易出现连续遗精的情况。这时候我们应该把固肾功做到位,但大家注意,我说的做到位,并不是让你猛冲猛打,如果睡前搞得太累,那也是可能导致遗精的。我们做固肾功要注意把握节奏,动作柔和,循序渐进,慢慢下压,切记不可猛冲。有的戒友做固肾功过猛,结果把韧带给拉伤了,我们一定要注意热身,慢慢下去,动作保持柔和,节奏保持稳定,这样既不会拉伤韧带,也不会把自己搞得太累,这方面应该多注意才是。

吉祥卧的难点就是固定睡姿,很多人遗精醒来发现自己已经变为平躺了,古大德就专门强调过睡姿的固定,吉祥卧是需要练习的,甚至是绑着练。有戒友推荐在膝盖那放枕头,这也是一种办法,还有的戒友甚至在床上放箱子,就是为了固定睡姿,估计他是遗怕了。吉祥卧大家应该多加练习,其实也不是很难,慢慢就习惯了,刚开始固定睡姿可能有点困难,毕竟会有不舒服的感觉,练到最后就睡得很舒服了。

下面步入正题,这季就关于魔考,憋与射之间这两个主题详细论述一下,具体如下。

\subsubsection{魔考}

魔考这两个字,在我做的一张图片里有写到,相信很多戒友都看过,这季从魔考这个角度展开来讲讲。

所谓魔考,其实就是心魔的考验,戒色后心魔会时不时来考验你。考验你什么呢?考验你的觉悟和定力,学习提高觉悟,觉悟产生定力,靠什么降伏心魔?靠觉悟!有的戒友说,对待戒撸,要像对待高考或者考研那样,全力以赴地学习,每天学习每天复习,备战迎考,要拿出那种拼劲来学习戒色文章。我比较认同他这种观点,他有这种观点,其实就是深刻意识到了学习的重要性,只有学习可以提高觉悟,觉悟到了即可战胜心魔。我们对待戒撸也应该认真严肃点,不要马马虎虎地戒,其实戒撸比高考更残酷,戒不掉很可能就废掉了,实际情况是,很多人是废掉后才开始戒的,而高考失败,你的身体并不会废掉,还可以复读重考。

我们戒撸其实就像在考试,只不过我们的考官是心魔,是心魔在出卷子。每次破戒,就是考试不及格。魔考错一题就是不及格,所以比高考更难。但是通过每次破戒,我们可以看到自己的不足之处,然后找到正确答案,这样下次就能戒得更好。而很多戒友破戒了也不总结经验教训,也不分析破戒原因,每次都戒得很糊涂,学习也很马虎,这样要戒掉真的很难。回想我那时学习戒色文章的劲头,也的确像高考,我记得我高考时也没做过那么多笔记。有一颗渴望学习的心,并且能够养成良好的学习和复习习惯,让觉悟持续得到提升,这样坚持下去,彻底戒掉撸管就不是难事。其中的难点就在于持续,提高觉悟就像爬山,有的戒友爬到一半,不动了,出现了懒散和懈怠,没有动力和热情了,然后就是一连串的破戒,而且每次破戒,他都发现自己越戒越差。这其实就是单纯靠热情戒色,热情一消退,厌倦感一出现,马上就又掉入了破戒怪圈。要克服热情消退和戒色厌倦期,那就必须养成良好的学习习惯,养成了习惯,就无所谓厌倦不厌倦,就像你不会厌倦刷牙一样,因为已经成为固定的习惯了。

每天看看戒色文章和戒色笔记本,温故而知新,自己有所思考和领悟,这样觉悟就可以持续提升,并不一定要看新的戒色文章,复习才是学习之母。我那时就像一块海绵,疯狂吸收戒色知识,然后自己也在不断思考和领悟,当我达到戒色稳定期以后,我也没有放松,还一直在学习和悟道,因为我有一种求知的渴望感。直到现在,我也一直在学习,我家有很多书,其实每本书都和戒色理论相通,就看你能不能悟得到。提升觉悟最快的方式就是悟戒,当你真正领悟到一条戒色原理或者规律,觉悟就会提升一个层次。经常处在悟戒的思想状态下,觉悟就像坐电梯般飞升。当觉悟达到一定境界时,心魔就动不了你了,对于心魔出的任何题,你都可以正确应对。

关于心魔的怂恿,你是否能意识到这是心魔在怂恿你破戒,你是否能正确应对。有时心魔会怂恿你破戒,他不会直接诱惑你,而是怂恿你去破戒,心魔会帮你找破戒的借口。比如这样的念头:一周撸一次没事的,或者试试性功能恢复没有,又或者精液没营养的,憋着对身体不好。诸如这类念头很有迷惑性,在这些问题上如果认识不清,那就很可能会导致破戒。首先,对于这些问题一定要有正确的认识,其次就是一出现这类念头,马上就要意识到这是心魔在怂恿你破戒,对于这类怂恿的念头,我们也要做到念起即断,千万不可听从这类念头。心魔的怂恿其实也是心魔在给你出题,看你是否能正确应对,如果你很有觉悟,知道这是心魔在怂恿你,那么你就会坚决断掉这类念头。

撸管后会出现非常多的情况,五花八门,对于每一个常见的问题,我们都要有正确而清晰的认识。如果你有疑惑,那么就会导致疑惑破戒,如果你没意识到情绪管理的重要性,那就很可能出现情绪破戒。每次破戒后我们不应该气馁,而是应该从破戒中不断总结经验教训,这样下次才能避免犯同样的错误。其实很多戒友都是在犯同一个错误,比如每次遗精后都会破戒,其实他之所以遗精后会破戒,就是因为他对遗精没有正确的认识,在心魔的考卷上写了错误的答案,结果就是破戒。当你写了正确的答案,那就过关了,只要真正过关一次, 那么就永久过关了,因为你知道如何正确应对了。就像一道难题,你知道正确答案,那就一辈子都会了。

当你出现破戒了,我可以肯定地告诉你,你在戒色的认识上还存在不足,因为认识上有缺陷或者误区,就会导致无法正确应对心魔的考核,继而就会出现破戒的情况。当你觉悟提升到一定的高度,对于心魔出的题,你基本都能正确应对了,那么你的戒色天数就会不断突破。心魔考官发现无法考倒你,他就会减少考核,就像被你驯服了一般,慢慢地,你就会进入戒色稳定期,一旦进入戒色稳定期,煎熬感就会随之消失,是否计算天数也无所谓了。进入戒色稳定期后,就不一定要每天学习了,但一定要保持警惕意识,有戒友戒了两年多,以为自己不会破戒了,然后一放松警惕,马上又开始破戒,所以警惕的螺丝千万不可松,对于怂恿的念头更要特别警惕。通过破戒,要认识到自己的不足之处,然后要补强认识上的不足,这样下次就会戒得更好,就像一件设计产品,经过上百次的修改完善,最终就会变得完美。记住,破戒是让自己的觉悟变得更强更完善,而不是一蹶不振,越戒越差。如果你不总结破戒的经验教训,那就会犯同样的错误,结果就是越戒越不行,越戒越气馁。飞机能飞上天,是经过无数次的修改和完善的结果,只要我们能保持进步,通过破戒意识到自己的不足,然后补强自己的不足,那么我们的觉悟就会变得越来越完善,面对心魔的考核就会成竹在胸,因为你已经掌握了所有的正确答案,不管心魔出什么题,都考不倒你。

为何会破戒?千说万说,还是觉悟不够。学习提高觉悟,觉悟产生定力!\textbf{戒色成功 = 觉悟高 + 警惕强}。没有其他办法,提高觉悟才是戒色的王道。而觉悟提高是自己努力学习的结果,没有人可以替代你。就像你看别人吃饭,永远也不会饱,你只有自己去学去吸收,觉悟才能持续提高。师父领进门修行在个人,只要你有一颗渴望学习的心,并且能养成良好的学习习惯,要彻底戒掉撸管并非难事,悟性高的戒友进步会很快,悟性差的戒友进步会慢些,但只要保持每天进步,迟早会迎来“大翻身”的。对于戒色吧的很多戒友,特别是新人,我观察过他们的帖子,其实很多人根本就没入门,还在一味地强戒和盲戒,而老戒友则是不厌其烦地向新人强调学习的重要性,试图把新人引入戒色的正轨,无奈有些新人根本听不进去,心态也比较浮躁,看不进戒色文章。有戒友就和我说过,以前因为心浮气躁看不进戒色文章,戒一段时间后,心能静下来了,这时看戒色文章才发现真的很管用,这时他才入了戒色的大门,以前连门在哪都没摸到,完全就是一个门外汉。

对待戒色,我们就像对待一场人生的考试,这是一场很残酷的考试,考试失败,很可能身体就废掉了。十几岁二十岁的身体相对好恢复些,毕竟年轻恢复快,到了三十岁,恢复速度将会变慢许多,到了四十岁恢复速度就更慢了,男人四十身体就正式开始衰败了。男走八,五八四十开始正式衰败;女走七,五七三十五开始正式衰败。所以,我们应该趁着年轻把撸管彻底戒掉,有一个好身体才是最关键的,当失去健康才会意识到健康的宝贵,健康的感觉比几千万几亿都珍贵,健康是花钱买不来的。对待魔考,我们不必害怕,心魔的考题就那些,你可以通过学习前辈经验来获取正确答案,有了正确答案,就不怕心魔来考你了。如果你的答案是错的或者你根本不知道答案,那么等待你的就是破戒,等待你的就是症状的折磨,这很现实,也很残酷,比任何考试都残酷,因为你一旦答错了,输掉的是健康、容貌和整个人生,同时也输掉了金钱,因为看病吃药是需要很多钱的。就这么残酷!对待魔考,你必须提起十二分的精神,像烈士一样戒撸,心够决!对自己够狠!正确应对魔考是你唯一的出路,答案就像密码,输入正确答案,你就过关了,否则就是破戒。

\subsubsection{憋与射}

下面讲讲憋与射的问题。

憋与射的问题,应该是每个戒友都会遇到的问题。憋与射的选择,其实就是心魔出的一道考题。这道题是选择题,当你做出错误选择时,那就会破戒。因为你心里没有正确答案,只有正确答案才能通过考验。这季我就把正确答案和大家分享一下。一般出现憋的情况,有以下几种:

\begin{itemize}
    \item 梦遗最后时刻憋住或者憋回去一部分
    \item 撸而不射,就是在快射时停止
    \item 性生活时憋着不射
\end{itemize}

大家遇见最多的情况就是梦遗憋精,有些戒友本来戒得好好的,然后突然有一天梦遗憋精了,这时候他脑袋里跳出了一个想法:憋着不好,会导致炎症,憋着对身体有害,应该排掉。这个念头其实就是他看的有关文章里的观点。关于憋与射这个问题,其实正确答案就是两者都有害,憋精其实已经精失其位了,伤害已经造成了,很多戒友在憋精的第二天马上就会出现症状,一出现不适症状就更加重了他对憋精有害观点的认同,在这种观点的驱使下他就会破戒。

憋与射是一个两难的选择,之所以两难就是因为两者都有害,这时候我们就要学会权衡利弊了,就像把这两个选择放在天枰上,看哪个选择更合理。憋,容易出症状;射,其实也很容易出症状。但射对于戒色是很不利的,因为很可能会一发不可收拾,严重打击你的戒色信心。而憋精虽然会有不适的感觉,但只要注意休养几天,适量锻炼,这种不适的感觉就会消失。另外,憋精的情况毕竟不多,偶尔憋一次问题不大,如果经常憋精,那是很不好的。如果只是一月憋一次或者几月憋一次,那就不用放在心上,继续坚持戒色养生即可。如果你选择了射,那就完蛋了,你把自己的闸门一开,精泄掉不说,戒色的信心、勇气、士气都会一泄千里,很可能就找不到戒色的感觉了,可能要调整很久才能重新找回良好的戒色感觉。所以,我的主张就是选择憋,这是权衡利弊的结果。如果我选择射,那我肯定早就失败了。

当你在憋与射这个问题上认识不清,那么“憋精有害”的观点将变成心魔有力的怂恿,心魔会不断和你说憋精很有害,不如射掉。如果你一时糊涂,听信了心魔的怂恿,那么肯定会破戒,很多戒友就是在这个问题的理解上存在缺陷,结果就是不断破戒,悔恨不已。我曾经在以前文章里有讲过一句话:不管发生什么都不要破戒!请大家牢记这句真言,戒色之盾要够坚固,心魔之矛才不会攻破。

其实心魔和你之间是在进行一场拔河比赛,当你觉悟很少时,就像小孩和大人在拔河,根本不是心魔的对手,一拉就过去了。当你通过不断学习,觉悟提升了,你将变成少年,少年和成人是可以拔一拔的,但还是不行,毕竟心魔是成人,还不是心魔对手,还是会输掉。当你坚持学习下去,当你的觉悟变成大人了,这时候就可以和心魔抗衡一下了,戒色天数也会取得突破。当你继续学习提高觉悟,你的觉悟将会变成大力士,这时候心魔就很难打败你了。当你的觉悟变成大山时,心魔就动不了你了。这就是觉悟修炼的过程,通过拔河这个形象比喻,大家可以知道自己处在哪个层次。很多戒色新人无疑处在“小孩阶段”,被心魔一拔就过去了,根本不是对手,希望大家好好坚持学习提高觉悟,几乎每季我都会强调学习的重要性,特别是持续学习持续提高觉悟,觉悟修到了,自然可以如如不动,不被境牵,不为境转!

这季总结了 35 个问题,这些问题在我的文章里都有讲到,相信很多问题大家心里都有正确答案。提问的方式,可以很好地检验你学习戒色文章的吸收率,很多戒友看完戒色文章就忘记了,等于什么也没看,觉悟一点也没长进,所以吸收率是非常重要的,可以通过复习提高吸收率,也可以通过提问的方式检验吸收率,当你吃进去的戒色文章都消化吸收了,你的觉悟就会一天天变强大,觉悟变强才是戒色成功的硬道理。

附戒色考卷:

\begin{multicols}{2}
    \begin{enumerate}
        \item 你对适度无害论的态度和理解。
        \item 出现憋精情况时,你会选择:\begin{multicols}{2}
            \begin{enumerate}
                \item 憋
                \item 射掉
            \end{enumerate}
        \end{multicols}
        \item 遗精后应该注意什么?
        \item 破戒类型有哪些?
        \item 导致遗精的因素有哪些?
        \item 断意淫口诀是什么?
        \item 什么是欲望休眠期?
        \item 什么是戒断反应,具体有哪些?
        \item 遇见诱惑时正确的第一反应是什么?
        \item 什么是正确的戒色动机?
        \item 意淫的变化规律:\begin{enumerate}
            \item 随着时间变强
            \item 随着时间变弱
        \end{enumerate}
        \item 手淫导致腰痛的原因?
        \item 手淫导致腿软的原因?
        \item 为什么撸管的人胆子会变小?
        \item 手淫为什么会导致脱发和白发?
        \item 撸管后为什么睾丸容易下垂?
        \item 戒色时你是否可以控制自己的情绪,是否能管理自己的情绪。
        \item 手淫为什么会导致脑力下降?
        \item 手淫为什么会导致鼻炎或者加重鼻炎?
        \item 如何克服戒色厌倦期?
        \item 为什么手淫后会变得嗜卧懒动,哈欠连天?
        \item 出现晨勃时,应该如何对待?
        \item 如何克服梦撸?
        \item 手淫为什么会使人变丑?
        \item 光看不撸是否有危害?
        \item 哪些念头是心魔的怂恿?
        \item 为什么意淫要在第一时间断掉?
        \item 精液的正常颜色是?
        \item 久坐和熬夜的危害是?
        \item 意淫的危害是什么?
        \item 进入戒色稳定期的标志是什么?
        \item 为何肾虚后大小便容易出问题?
        \item 为何手淫的人容易自卑?
        \item 撸管为什么会导致耳鸣?
        \item 靠什么降伏心魔?
    \end{enumerate}
\end{multicols}

这季推荐一本书:

\begin{book}[《密勒日巴尊者传》]
    密勒日巴尊者是一位深受藏人爱戴的瑜珈修行者、哲人及诗人。在图画的描绘中,他往往以手支耳作聆听状。这象征了他凭借诗歌传达佛法智慧的方式。他的肤色泛绿,这是由于在多年的闭关静坐中,他常仅靠荨麻煮汤维生。对于佛教的发展而言,他亦有相当的重要性,他提供了一个凡人能在一世之间证道的实例。这本书我自己是比较喜欢的,也专门看过连续剧,拍得很不错,密勒日巴尊者的一生是一个传奇,这本书里的很多教言都值得反复研究和体味。
\end{book}
