\subsection{戒色状态的调整、发育障碍、紧张性遗精}

\paragraph*{前言}

最近浏览贴吧的帖子,看到很多戒友破戒,原因还是过不了 YY 关,而戒色要成功,必须过 YY 关,可以这样说,如果你能够杜绝 YY,其实就成功了一半,你的戒色天数就会猛增!如果你过不了 YY 关,等待你的必然就是反复的失败。失败不可怕,可怕的是不学习,只有不断学习提高觉悟,不断总结经验教训,才能戒得越来越好。失败了不要灰心,要看到自己的进步,提高觉悟有一个过程,当你觉悟修到了,自然就戒掉了,所谓:水到渠成!

断 YY 贵早,就是要断得早,一定要先知先觉。

有的戒友和我说,欲火中烧了怎么办?做俯卧撑可以压制欲望吗?我的回答就是,你断 YY 太晚了,YY 的念头刚起来时犹如小火星,断掉很容易,当 YY 的小火星变成燎原之势,变成欲火中烧,你想扑灭它,太难了,等到它发展壮大了,你才后知后觉,想到要去扑灭它,很有可能就会适得其反,越压制越旺,越压制越反弹。不怕念起,就怕觉迟。在 YY 的念头刚出来时 0.01 秒就要察觉到,马上断掉它,不犹豫,不妥协,不拖延,要果断,干脆利落!手起刀落!断得早!断得狠!断得快!你不降伏心魔,心魔就降伏你,没有第三种可能,被心魔降伏,就等着被症状虐吧!

关于断 YY,我 \ref{17} 的文章写得比较详细,大家可以再看下,我所写的文章中,\ref{17} 是比较重要的一季,因为里面涉及到如何断 YY,过了 YY 关,戒色才能日趋稳定,否则就是戒了很久,还是失败,还是打不过心魔。

我上季有张图片讲到,戒色如下棋,坐在你对面的就是心魔。和现实中的下棋一样,刚开始你和一个高手下棋,只有被虐的份,怎么下,都下不过。因为人家是高手,有着丰富的实战经验。而你是新手,什么都不懂,还存在很多思想误区,在这种情况下,如何能下赢心魔?心魔段位比你高,而你只是一个菜鸟。其实,要下赢心魔,只有华山一条路,那就是不断学习提高觉悟,等你觉悟修到了,段位上去了,再战心魔,你就会发现心魔不是你对手了,你可以降伏心魔了,可以战胜它了。戒色要成功,就是一个觉悟不断提高的过程,等你有了学习意识,有了良好的学习习惯后,离戒色成功就不远了!

还有一个问题,就是勃起时,流出的透明液体,到底是什么?我研究过的文章,一般有两种说法,一种是前列腺液,一种是尿道球腺分泌的液体,有润滑的作用,有的戒友会叫它润滑液。尿道球腺液,它透明而粘稠,可拉长成丝,有润滑尿道的作用,并构成射出精液的最初部分,也是组成精浆的成分之一。而前列腺液也是精液的重要组成部分。所以不管流出的是前列腺液还是尿道球腺液,其实都对身体会造成损伤。很多戒友,光看不撸后就出现了症状反复,YY 是暗漏,中医有讲到:心动则精自走。我看过的中医医案中,就有记载 YY 导致疾病的案例,YY 是能够致病的。这点千真万确!光看不撸之后,会有什么后果,你可以照照镜子,你会发现自己的气色下降了,然后很有可能会出现睾丸痛,小腹涨,或者尿频的现象。这就是 YY 暗漏的后果,这是大量戒友反馈的结论。大家一定要坚决杜绝 YY,做到彻底戒色!

下面进入正文。这季就戒色状态的调整,发育障碍和紧张性遗精详细论述一下,具体如下。

\subsubsection{戒色状态的调整}

最近反映戒色状态不佳的戒友有很多,比如出现乏力,懒散,情绪低落,动力丧失等现象。

这种情况我也有过,出现这种表现和季节是有密切关系的,“春困秋乏夏打盹儿,睡不醒的冬三月”,季节转换对人的心理和生理都有着深刻的影响。夏季天气炎热晚上休息不好,白天容易犯困不难理解,然而到了秋高气爽金秋时节,人们为什么还会出现秋乏现象呢?因为,在炎热的夏天,人的身体大量出汗造成了水盐代谢失调,肠胃功能减弱,心血管系统的负担加重,人的身体处于过度消耗阶段。夏去秋来,气候由炎热变得凉爽宜人,人体出汗也明显减少,人的机体进入到了一个周期性的休整阶段,水盐代谢开始恢复平衡,人的心血管系统的负担也得到缓解,消化系统功能也日渐正常,然而此时人们的身体却有一种说不出来的疲惫感,这就是人们常说的“秋乏”。其实这是不同季节人体的自然生理反应。经过一段时间的调整,秋乏现象会自然而然地消除。

所以,如果你最近戒色状态不佳,建议注意休养,不要搞得太累,注意情绪管理,慢慢就会过去的,戒色的好状态会回来的。自己一定要学会调整,这异常关键。人的戒色状态是起伏的,这很像运动员比赛状态的起伏,梅里特现在跑 12 秒 80,他下次还能跑 12 秒 80 吗?也许状态就回落了。戒色也是如此,刚开始不少戒友看了几篇戒色文章,马上决心极大,戒色热情高涨,但是过了十几天,马上就丧失热情了,找不到动力了。看的时候疯狂看,不看的时候几周都不看戒色文章。这样觉悟的提高就没有持续性可言了,就像爬山,爬到一半就半途而废。所以,我们要戒色成功,必须尽量保持稳定的戒色状态,刚开始时热情高涨,可以多摄入些戒色文章,当热情消退时,我们不要不看戒色文章,而是把摄入量减少些,每天按时看些戒色文章,就像每天刷牙一样,养成习惯,习惯成自然。有了稳定的戒色状态,觉悟的持续提高就有保障了,就像你打网游,不是打一天就休息十几天,而是每天打,这样练级才快。当然,我是不赞成沉迷网游的,只是打个比方而已。

导致戒色状态发生变化主要有以下几点:

\begin{description}
    \item[戒色厌倦期] \ref{26} 的文章有专门讲到;
    \item[季节因素] 季节转变是可以影响到人的情绪和生理的,从而影响到戒色状态,这点是比较容易被忽视的;
    \item[看了无害论] 一看无害论就容易动摇,从而产生退心和疑心;
    \item[做事的计划性] 做事有计划的戒友,会戒得更稳定,每天给自己安排多少量的学习,他心中有数;
    \item[生活琐事] 如果琐事很多,是容易导致分心,或者心一累,就不想看戒色文章了;
    \item[工作学业压力] 人在压力大的情况下,情绪容易紊乱,从而影响到戒色状态;
    \item[有老婆或者女友] 这类戒友的戒色状态,是很难稳定的,具体什么原因,我不说大家也应该知道;
    \item[环境污染] 比如寝室有人看黄,或者因为应酬要去那些场所,这都会影响戒色状态;
    \item[遗精后] 遗精后出现破戒的情况还是很多的,遗精后是容易产生思想动摇的,一定要保持警惕;
    \item[饮食因素] 肉吃多了,补药吃多了,容易产生妄念,从而影响戒色状态。
\end{description}

我们要戒色成功,必须要让自己的戒色状态保持稳定,就像走钢丝一样,要让自己保持稳定,找到一个平衡点,这样才能戒得长久,戒得稳定。

\subsubsection{发育障碍}

下面谈下发育障碍。

我搜集了几千个受害者案例,聊过上千个戒友,其中反映发育障碍的还是很多的,主要集中在以下几点:

\begin{adjustwidth}{-2.5em}{-2.5em}
    \begin{multicols}{3}
        \begin{itemize}
            \item JJ 发育障碍
            \item 身高发育障碍
            \item 面容发育障碍(看上去偏小很多)
        \end{itemize}
    \end{multicols}
\end{adjustwidth}

前两点,相信大家比较容易理解,第三点,肯定有人会说,面容偏小不是好事吗?其实面容偏小,不一定是好事,很多戒友正在为此发愁,在同龄人当中面容偏小,容易让人看不起,而且二十几岁的人,看上去像高中生或者初中生,将来找工作也会遇见麻烦,用人单位会觉得你不够成熟,无法胜任工作。这里的偏小不是偏小一二岁,而是在五岁以上。SY 后,面容会出现两种倾向,一种明显变老,一种就是发育延迟,面容偏小。这两种都会给人带来烦恼,而SY后变老变丑比较多见一些。

JJ 发育障碍比较常见,JJ 属肝,而 SY 伤肝肾,是会影响到 JJ 发育的,有的人本来体质就不行,再加上疯狂沉迷 SY,结果就会影响到 JJ 的发育,出现短小的现象,搞得自己很自卑。很多戒友在 10 \unit{\cm} 以下,有的则在 8 \unit{\cm} 以下,有的则是太细,也是非常让人烦恼的问题。出现 JJ 发育障碍,应该要赶紧戒掉 SY,注意调理身体,这样是会有所改善的,至少勃起质量会有所提升。

身高发育障碍,我在 \ref{8} 有专门讲到,有身高烦恼的戒友可以看看。影响身高的因素很多,SY 是其中一个因素,沉迷 SY 是有可能影响骨骼发育的,因为中医:肾主骨。沉迷 SY 恶习会导致骨骼发育障碍,从而影响到身高。但并不是所有 SY 的人都会影响到身高,因为影响身高的因素有很多,SY 只是其中一个因素,如果其他因素都做得很好,比如吃得很好,运动到位,按时作息,基因优异,这样还是可以长到理想的身高,但可以肯定的是,SY 绝对会影响到骨密度,SY 会导致腿软,在剧烈运动时,容易出现骨折骨裂的现象。

SY 的的确确会影响到发育,但每个人体质不同,后天的营养和生活运动习惯的不同,导致影响程度也不同,有的人热爱运动,无不良嗜好,按时作息饮食,营养跟得上,虽然他也 SY,但影响程度比较轻微,而有的人天生体质不佳,再加上久坐熬夜,不运动,吃得也不好,这样就很有可能会出现严重的发育障碍。所以在发育期的男孩们,一定要及早认识到 SY 的危害,免得将来欲哭无泪,过了发育期就定型了,很多东西很难改变了。

\subsubsection{紧张性遗精}

最后谈下紧张性遗精。

紧张性遗精还是很多的,经常会出现这类问题的戒友。

我知道紧张会导致遗精,是在初中时代,记得当时期中考试,有一个同学就在考场遗精了,因为他还有题目没答完,但时间已经到了,他一紧张,就漏了,当时这件事在同学之间还传为奇谈,觉得不可思议。而我现在做 SY 行为的研究,才发现紧张性遗精其实还是很多的。

大家先看几个案例:

\begin{case}[紧张性遗精]
    飞翔大哥!请问一下晚上睡觉做的不是春梦,而是梦见在考试没写完,一着急就忍不住射了。这种情况是遗精还是滑精?
\end{case}

\begin{case}[紧张性遗精]
    飞翔哥,我是一名大四学生,快要考司考了,这几个月倍感煎熬,从今年一月戒到现在,破了很多次,由于考前特别紧张,最近一连四天都遗精,包括昨晚,我对未来失去信心了,希望你帮帮我!再遗精下去整个人都没了!
\end{case}

\begin{case}[紧张性遗精]
    飞翔哥,昨晚遗精了,没有坚持到一百天,遗完后非常沮丧,是梦遗,梦见自己被老师点唱歌,我不知道唱什么好,一着急,就遗了,一遗就醒了,才四点半。当时很沮丧,没有了原来的兴奋感。我还要继续努力,坚持固肾功,向下一个一百天奋斗。
\end{case}

我初中那个同学,是白天清醒时紧张遗精,而很多戒友是晚上做梦一紧张就遗精,或者白天紧张情绪很强烈,日有所思夜有所梦,晚上也会出现紧张性遗精。出现这种情况,大家不要慌,一定要学会管理自己的情绪,多给自己正面的暗示,不要多想。再加上坚持练习固肾功和积极的治疗,这种紧张性遗精是可以克服的。记得以前有一个戒友总是因为梦魇而遗精,一直很困扰他,后来我建议他积极治疗并且坚持练习固肾功,平时则要学会情绪管理,让自己保持在心平气和的心理状态,后来他反馈遗精比以前好了很多。

不少人一紧张就六神无主,心烦意乱,这时候你应该告诉自己,深呼吸,保持从容,保持心理状态的安详。多给自己这方面的暗示,时间久了,自然会克服紧张情绪的。保持淡定是非常重要的,你如果在白天能够保持从容淡定,慢慢是会影响到潜意识的,到时候,在梦里出现紧张性遗精的情况也会慢慢减少乃至消失。这是一个心结,你必须学会解开它,而不是越搞越紧张。

\paragraph*{后记}

很多戒友存在把不良情绪转变成 SY 的行为定势,就是一遇到挫折或者烦恼的事情,马上就想通过 SY 来发泄,有这方面倾向的戒友一定要学会情绪管理,纠正这种错误的行为定势。我在破戒类型那季里面,专门有讲到情绪破戒,情绪破戒太普遍,但危害实在太大了,会让你陷入恶性循环。管理好自己的情绪,对于戒色成功是必须要具备的素质。有的戒友被爸妈骂,然后情绪压力很大,回房间就破戒了;有的戒友被老板骂,情绪失控,想不通,回家就破戒了;还有的戒友是乐观情绪破戒,比如高考后,一放松就破戒了。或者为了庆祝胜利,庆祝成功,这时候人也有放纵的倾向。还有就是无聊情绪,周末一个人在家无聊,心魔就跑出来了,然后鬼使神差般地就破戒了。情绪破戒太普遍,我们一定要意识到管理情绪的重要性,好好管理自己的情绪,让自己戒得更稳定。
