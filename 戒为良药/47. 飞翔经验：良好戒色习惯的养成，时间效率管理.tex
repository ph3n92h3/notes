\subsection{良好戒色习惯的养成,时间效率管理}

\paragraph*{前言}

这季前言分享一个反馈和两个答疑案例

\begin{case}
    到了午休的时候,我就躺在床上看飞翔老师更新的 \ref{46} 文章,看完我就开练了,而且一站就是 45 分钟以上,以前的打坐、浑圆桩、抱树都从未有过,打坐是因为我妄念纷飞,用数息法也很难摄念,浑圆桩稍微一站久手就很累,我也坚持不住,抱树也只是三分钟的热情,可是子午内养功站桩往那一站,我就钟爱此桩法了(绝不夸张,此桩最大的特点就是不累,功力却很威猛!而且不需要完全摄念,站的过程中我还是会用数息法摄念或者念佛,还有就是从心里感恩飞翔老师),接下来奇迹就在我身上悄然发生,站桩过程中,飞翔老师说的“首先劳宫穴开始发热,发麻跳动至丹田发热”,我都有感触到,瞬间打通任督二脉以及全身大小经脉,几个月身体气血阻塞瞬间疏通,从未有过的清新爽朗的感觉顿时涌上心头!站完桩后,我手心出汗,背全湿了,顿时心里的阴霾一扫而空,重见天日啊!萎靡不振、猥琐不堪的我顿时可以笑容满面、活蹦乱跳啊!飞翔老师,我从心底里向您致敬!您让我省下了高昂的心理医疗费,让我可以信心十足继续和心魔战斗啊!而且有意思的是,站完桩后我还兴高采烈、兴致勃勃去公园抱了一个多小时的树,效果和此桩一样,都是立竿见影!回来后照镜子,凡是脸颊有贴过和没贴过的地方就是不一样,贴过的地方变得细腻白皙了,很是高兴!(净心者)

    \textbf{附评} \ref{46} 我推荐的子午养生桩比较容易上手,很简单,也不累,易于坚持,效果是很明显的。我现在每天都有站此桩,这个桩得气是比较快,特别是手上的感觉。记得我以前很在意气感,有气感则心里很高兴,没气感则觉得效果不好,其实是掉进气感的窠臼里了,后来我才发现最重要的不是气感,而是舌抵上腭下来的那口华池水,那才是补益元气的宝贵药材,也就是人体的大药。现在我并不在意气感,而是很注重口中的华池水,每次站桩平均吞咽十次以上,华池水下来是清香甜美的滋味,和普通唾液完全不一样,在练功时分泌下来尤为明显,在平时也可以舌抵上腭,但没有练功时分泌那么多。我推荐的都是养生功法,并不是什么神功异能,我也从来没反对过看医生,该治疗的应该积极治疗,但三分治疗,七分养生,养生才是重头戏,不应该把希望都砸医生那,毕竟伤精的症状不是小感冒,不是吃几粒药丸就可以痊愈的。肾虚患者一定要注意戒色保精,否则上补下漏,万难痊愈,即使暂时好了,一撸就极易复发。记得有几位精索戒友去做了手术,手术后不注意保养,而且又开始破戒,结果精索又复发了,这能怪谁?所以我们自己一定要注意戒色养生,一定要注意保养,把痊愈希望都压在医生和手术上是非常错误的认识,一定要记住三分治疗,七分养生,孰轻孰重,一目了然。症状严重或者持续则应该积极治疗,然后配合戒色养生,这样恢复就比较快了。
\end{case}

\begin{case}
    已戒两百天,近来一次遗精后心魔进行了一次猛烈进攻,当时有些迟疑,然后就很难找回当初清心寡欲的戒色感觉了。到现在 YY 严重,如煎熬一般。自我分析与戒色稳定后自我意识增强和当初戒色觉悟不扎实有关,不知道该怎样脱离苦海?

    \textbf{答} 遗精后是破戒高发期,心理容易产生动摇,情绪也会出现不良变化。所以,必须意识到遗精后容易破戒,一定要提高警惕,加强学习,并且注意调整情绪,做好断念,这样就能平稳度过这段破戒高发期了。加油!

    \textbf{附评} 遗精后是一个特殊时期,之所以特殊,就是因为这段时期是破戒高发期,遗精后不警惕是极易破戒的。一遗精,人的情绪不再平稳,就像飞机遇见气流开始颠簸,这时候就容易出现混乱,所以遗精后一定要注意情绪管理,让情绪重新恢复平稳。情绪管理这项能力,相信资深戒友会极其重视它,因为很多人就败在情绪破戒上。情绪管理和情商有关,而情商是可以习得的,就是通过学习是可以提高情商的,自己多看看情绪管理方面的书籍,提升这方面的理解和认识。做好情绪管理,戒色才会更稳定。戒色成功需要一个稳定的心理环境,我曾经把戒色比作走钢丝,你应该保持高度警惕,并且保持心理的稳定,如果你心理混乱,那肯定会掉下来的。有的戒友听我说要保持高度警惕,然后他看见女人就害怕,走在街上很紧张,整日把自己搞得神经兮兮的,影响到了正常的生活。其实他误解了我的意思,保持高度警惕,并不是过度警惕,就像走钢丝,过度紧张也是会掉下来的,我们戒色就像弦,不能绷太紧,会断,也不能太松,容易破戒。不紧不松方合中道。保持高度警惕,就在不紧不松之间,这个需要自己好好把握。
\end{case}

\begin{case}
    飞翔大哥,您说只要熟背断 YY 口诀就可以彻底降伏 YY,而且所有的 SY 都是 YY 在前的,那我们还要提高觉悟干什么?每天只要像念佛一样背口诀不就好了?您在您的文章中始终没有点透这一点,希望您能去掉我的疑惑。我准备加诵四种清净明诲,不知可有用处?

    \textbf{答} 过了意淫关,戒色就成功了一半。断意淫也属于觉悟的范畴,戒色觉悟还包括情绪管理、遗精控制、还有养生之道等。熟背断意淫口诀是必须的,提高综合觉悟也很重要,否则综合觉悟不高,那还是有可能会破戒的。持诵四种清净明诲很好,如果你有佛缘,有这个信力,建议可以每日持诵四种清净明诲,这样可以帮助你净心,足够虔诚的话自然能获得加持力,加油!

    \textbf{分析} 我是建议大家平时多背诵断意淫口诀的,你如果觉得四句有点长,可以背两句,也可以背一句,只背个念起即断即可。很多戒友断意淫不得力的原因就是临阵磨枪、临时抱佛脚,到了意淫出现时,再想起背断意淫口诀,发现不怎么管用。其实,要真正用好这个口诀,就必须在平时大量重复背诵,就像念佛号一样,每天念几百遍直至形成条件反射,邪念一起,马上自动就断掉了,完全自动化了,根本不需要经过大脑的思想挣扎,如果你还要挣扎或者犹豫,那就说明你功夫没到家。当这个口诀念得烂熟时,就像口头禅一样,下意识就会断意淫,就像电脑的自动杀毒软件一样运行,到时候断意淫就得力了,一点都不会思想挣扎了。其实念佛号也是如此,大量重复佛号,有妄念了,马上转成佛号,完全自动化,完全下意识,根本不需要思想挣扎,也不会出现犹豫不决。要达到这个境界,唯有大量重复,一定要达到条件反射的地步,达到自动化断除邪念。千万不要平时不念,到意淫攻击你时,再想起来用,那是钝刀,不给力!断意淫口诀就像一把刀,你平时一定要磨锋利了,到时候意淫来时,就能给上力了。否则到时候你犹犹豫豫,还要经过一番激烈的思想挣扎,弄不好最后还是会破戒,那就很被动了。
\end{case}

下面步入正题。这季就良好戒色习惯的养成,时间效率管理和大家做一个分享,具体如下。

\subsubsection{良好戒色习惯的养成}

很多戒友都提到了习惯的重要性,因为他们看到了习惯的巨大力量,好的习惯和不好的习惯都有巨大的冲力,就像巨石从高处滚落,带着你前进或者带着你自毁。把破戒归咎于习惯是不少戒友的真实想法,的确有些习惯是容易导致破戒的,还有一些不良习惯则不利于身体恢复,养成良好的戒色习惯其实也属于觉悟的一部分。下面例举几个不良习惯:

\paragraph{不良习惯之赖床}

赖床是一个很不好的习惯,赖床也容易导致破戒,赖床破戒我在破戒类型里有专门总结过。赖床时意淫容易跑出来,而且赖床对身体健康也不大好,中医讲久卧伤气,一直不起床,会影响人体气机的生发,有的戒友甚至不吃早饭,这样对身体更不好。赖床这个习惯一定要改掉,醒了就应该马上起床,不要在床上东想西想,更不能沉迷意淫。要克服赖床习惯,首先要发大决心,要对自己说:从明天开始永远不再赖床。其次,可以借鉴酒店的叫醒服务(Morning Call Service),住过酒店的人都知道,一般的酒店都有 Morning Call Service,定好闹钟,放在够不着的地方,到点马上“鹞子翻身”或者“鲤鱼打挺”。起床要够坚决!

\paragraph{不良习惯之睡前必撸}

睡前撸管的戒友非常多,不撸管他就睡不着,养成习惯了。一撸管射掉,人就会有疲倦感,会打哈欠,中医:肾为欠为嚏。射掉的一刹那,有的人甚至会感到大脑被瞬时抽空,中医:肾上通于脑。其实肾与脑是相通的,肾虚后脑力会下降,脑力下降在戒友中也异常普遍,这就不用我多说了,多看案例即可知道。射掉后,人就容易有疲倦感,有了疲倦感就容易入睡,而且在射掉的第二天早上也容易睡懒觉,就是睡不醒。中医:肾虚嗜卧懒动。所以前天晚上撸管,很容易造成第二天睡不醒或者赖床。刚开始是睡不着撸管,所以很多人形成习惯了,但是恶果在后头,伤到一定程度后就会出现神经症,常见的是神经衰弱,到时候就是越撸越睡不着,特别煎熬和难受,真的是苦大了。而且一旦伤到神经,恢复的时间是以年计的,严重的至少要一年以上。

\paragraph{不良习惯之懒散拖沓缺乏耐心}

肾虚后人就容易变懒了,人本身也是有惰性的,我们要更好地戒色,必须要克服懒散的习惯,养成良好的学习习惯。懒散与拖沓似乎是一对兄弟,懒散的人基本也有拖沓的习惯,总是说明天再做,总是觉得自己有的是时间,等到明天了,又说后天,不知拖到猴年马月。我们要做成一件事,在考虑成熟后是需要立马行动的,执行力要强,行动要果断。戒色神力来源于学习,学习提高觉悟,觉悟战胜心魔。很多戒友懒于学习,而且学习拖沓、没计划,并且缺乏耐心。撸管的人,肾精一丢失,人就容易变得浮躁,所谓心烦气躁,脑力再一下降,根本无法安心学习,自己意识到这个问题,就要学会调整,尽快找回良好的学习状态。我的建议就是一点点学,刚开始不要给自己太大压力,让觉悟一点点积累,准备一个戒色笔记本,看到好的戒色句子就摘录下来,一天摘录十句,一个月就是三百句,时常复习摘录的句子,这样觉悟不知不觉就上去了。

关于拖沓的习惯,我建议可以给自己定时,在一个时点前必须完成学习任务,给自己一个时限,给自己一点压力,把自己的积极性调动起来。当你养成良好的学习习惯后,即可战胜懒散和拖沓,慢慢进入状态后,学习也会变得富有耐心。每天都要反省,今天是否学到了新的内容或者有了新的领悟,必须让觉悟持续提高,在学习的过程中一点点积累,合抱之木,始于毫末。很多戒友的问题就是学习不定时,不定量,想起来就学,没热情时十几天都不看戒色文章,所谓一曝十寒。这样觉悟是无法持续提高的,而有的戒友每天都有进步,虽然是小小的进步,但他没有停止进步,他很注重每天的积累,这样坚持一段时间,他的觉悟就会飞升,量变产生质变,最终会迎来大翻身。

\paragraph{不良习惯之久坐熬夜}

不少戒友是在戒色,但是他的养生意识非常差,这样对于身体的恢复是很不利的,久坐伤脾伤肾,熬夜则更伤。熬夜和久坐也是温水煮青蛙式的伤害,不知不觉就在废你了。我们要让身体更好更快地恢复,一定要注重养生之道,养生意识要建立起来。特别是神衰戒友更不能熬夜,熬不起啊伤不起。熬夜会减慢身体恢复的进度,熬夜久坐的危害我在前面的文章有多次提到,这两个习惯一定要改变,另外,保持良好的饮食习惯也很重要,按时吃饭,按时休息。不要想起来就吃,不要等到很累很累了再睡觉,保持规律的饮食作息异常关键。不要说撸管纵欲,就是熬夜和久坐就能导致很多疾病,属于生活方式病。意识到了,我们就要学会改变,久坐每四十分钟起来活动下,舒展一下身体,走动一下,让身心放松一下。学生党熬夜的问题,应该尽量提高学习效率,避免熬夜的发生。住宿的可以和室友好好沟通下,十点半熄大灯,不睡觉的可以开台灯,不要出声影响别人休息。工作熬夜就比较难办了,我现在聊过的戒友,有不少要上夜班,很伤身体,只能自己多注意保养,加强食疗,避免劳累。记得一位戒友因为夜班吃不消,后来就辞职了。如果你又撸管,又熬夜久坐,你就会废得很快,相当于三把斧头砍一棵树,没多久树就倒下了。

\bigskip

养成良好的戒色习惯非常重要,我们戒色更要注重修德,戒色的人应该是一群有素质有教养的人,是一群真正具备正能量的人。很多戒友,特别是新人,还是心胸狭窄,充满嫉妒心,看到别人破戒则幸灾乐祸,看到别人破戒就嘲笑挖苦一番,如果你是想骂醒对方帮助对方,那也无可厚非,而有的戒友则是出于阴暗心理,见不得别人比他好,看到比他差的则嘲笑羞辱,这其实是一种自卑和缺乏安全感的表现。戒友之间应该互相鼓励,交流经验,互相促进。别人戒得比自己好,应该为他感到高兴才是,别人如果戒得比自己差,则应该尽量帮助对方,把自己好的经验分享给对方。希望对方比自己好,才是君子之想。君子心地光明磊落,充满正能量,让我们消除嫉妒心,让我们做一个无私奉献的人。

修德很重要的就是学习传统文化,传统文化论坛的内容我是专门有学习过的,相关讲座也听了很多,的确受益良多。传统文化和传统价值观正在沦丧,现在很多年轻人以放纵为乐,互相攀比的都是吃穿用,甚至攀比自己的放纵经历,其实到最后还是害了自己,而他因为愚昧无知,根本不知道自己这样做有什么不对。他周围人都在这样干,他也这样干,大环境是放纵的,他自然也就被同化了。传统文化论坛里的王双利讲座,我看了好多遍,他讲座的题目是《醉生梦死浪子归》,学习传统文化前,他是吃喝嫖赌抽五毒俱全,学习传统文化后,他彻底醒悟了,所谓迷途知返是勇者,他真的改好了。传统文化论坛推广的是《弟子规》,我曾经也向戒友推荐过弟子规,培养正确的人生观是异常重要的,否则一味攀比物质,一味满足自己的感官,这只会让自己的精神更空虚,因为欲望是无底洞,永远不可能彻底满足,而你纵欲是要消耗肾精的,肾精是人体最宝贵的物质,不少人没有任何中医养生常识,还以为吃几个鸡蛋就能补回来,真能补回来,还会症状缠身吗?真能补回来,那些皇帝怎么年纪轻轻就挂了?皇帝不比你吃得好?最顶级的食物加上太医都不行。关键还是要学会戒色保精,很多古代名中医都有要求患者禁欲的嘱咐。华佗与顿子献就是很好的一则医案,大家可以搜着看看。精少则病是绝对的真理,经过几千年的验证,老祖宗是不会欺骗子孙的。

传统文化里面也强调孝道,百善孝为先,弟子规里也很强调孝顺。中医讲三阳开泰,动则升阳,善则升阳,喜则升阳。而百善孝为先,你如果很孝顺你父母,你的心境也会好起来,心境好了,身体的不适也会跟着好起来。身心是互相影响的,尽量不要惹父母生气,有什么事情可以好好沟通,沟通失败可以先顺着,孝顺孝顺,要懂得柔顺之道,千万不要顶撞父母。惹父母生气很不好,因为生气是会导致疾病的。说话要注意语气,即使父母有不对的地方,也要学会好好沟通。我们戒色也要注重孝顺,不少戒色文章都强调过孝顺,孝顺属于正能量,你多孝敬父母,多孝敬长辈,也是在增长自己的正能量,有了正能量戒色也更容易成功。

\subsubsection{时间效率管理}

下面和大家分享下时间效率管理。

我们大家都有一项极其宝贵的资源,这就是时间资源,如何用好时间,如何在有限的时间内达到效率的最大化,这是一门学问,我专门研究过时间管理和效率管理,不管做任何事,都要学会使用时间,管理时间,并且努力提高自己的效率。效率!效率!效率!这个词太关键了。我们戒色就是要提高学习戒色文章的效率,提高对戒色文章的吸收率,不能看过就忘记,那样是不行的,很多人看戒色文章就像看电影,看过一遍就不看了,看一遍和看十遍的吸收率是完全不同的,走马观花看一遍,很多知识都没吸收到,就像吃饭,大家同吃一碗饭,有人吸收率只有一粒米,有人吸收率就是满满一碗饭。吸收率高,长觉悟就快。长觉悟和长身高一样,一方面要吃,另外一方面就是要能吸收。否则光吃不吸收,那就等于零。

有个戒友看《戒为良药》看了十一遍,到现在都没破,能看十一遍的人,绝非一般人。对于好的书籍,我也会反复阅读,每次阅读都有新的收获,如果仅仅看一遍,很多精华的东西都会错过。对于戒色,大家应该采取钻研的态度,不断理解、不断认识,真正掌握戒色的规律和原理,当你真正掌握了,戒色成功将变成可能。我在前面文章曾经说过,戒色吧的确存在一批天赋极高的戒友,这类戒友觉悟提升极快,大家应该多向他们学习。

下面关于提升效率的要点和大家做个分享。

\paragraph{强劲的戒色动机,强动力的想法}

大家戒色基本都有恢复健康,恢复容貌的需求,这是普遍的戒色需求,而有的戒友是被症状逼上梁山,不得不戒,再不戒就完了,甚至有的是已经废掉了才开始戒,有的戒友戒色动力超强烈,他对自己要求很严格,对待戒色他拿出了十二分的热情与勇气,而且比较持久。他非常渴望学习戒色文章,如饥似渴,我当年的学习状态就是如此,太渴望学习了,就像一块海绵疯狂地吸收戒色养生知识,我想知道得更多更深入,而不是浅尝辄止。我对自己的要求也很严格,就是彻底戒色,包括意淫也戒掉,因为我通过学习中医知识,知道意淫伤害也很大,所以戒色必须彻底。当时我的想法就是一定要戒掉,别无选择,因为我已经伤不起了。那时我已经成了医院的常客,经常跑医院,但我发现医院也救不了我,只有靠我自己。那时我的出路就是破釜沉舟,这就是我的戒色决心和动力。戒色是需要一个强劲的动力的,否则你的潜能就出不来,人都是逼出来的。很多戒友看了不少案例,出于恐惧而戒色,因为他不想等到废了再戒,那样恢复难度就很大了,出于恐惧也是很强烈的动力,因为你已经看到了下场,不戒就和他一样惨。有了强劲的戒色动力,效率就容易出来,因为你一直不断地想学,就像迷上网游一样,那种全身心投入的状态,不知疲倦,也不会觉得厌烦,就是那种状态。我那时的学习状态就是那样,得到一条养生知识如获至宝,马上就记下来,生怕错过。那时我觉得自己像个挖宝的人,不是在山上挖宝,而是在书上挖宝,挖知识的宝。现在我不提倡那样着迷戒色,还是应该注意休养,在学习和养生之间找到平衡。以前我一天可以看完一本书,现在我一般分几天看完,让自己收着点,以免过度劳累伤身。大家如果能把看 H 的兴趣和热情转移到戒色上,那一定会戒色成功的。就怕你戒得不坚决,戒得马马虎虎,那样就难了。你是想戒色成功?还是十分想,还是百分想,还是千分想,还是万分想,亦或者一定要!从你对待戒色的渴望程度和迫切程度,我就知道你戒色成功的可能性。

\paragraph{做笔记}

学习戒色文章最好的办法就是多做笔记,我那时天天在看书做笔记,其实现在我天天也在看书做笔记,不过是以佛法方面的内容为主。做笔记其实已经在吸收了,看只是微吸收,看的吸收率很低下,真正能过目不忘的人极少,当你拿起笔开始做笔记,对文章的认识就会深入很多,在做笔记的时候,很容易产生自己的思考和领悟。大家都是学生党过来的,做笔记的重要性,相信每个老师都强调过的。不断做笔记,并且不断温习笔记的内容,温故而知新,还要尝试背诵重点句子,这样觉悟就能稳步提高,做笔记和复习笔记是觉悟提高最好的方式。而笔记的内容也可以灵活掌握,只要你觉得重要的句子都可以记下来。

\paragraph{激励热身运动}

在开始学习前,应该要学会激励自己,可以说些激励自己的话,也可以稍微活动一下,让自己兴奋起来,这样就比较容易进入状态。NBA 球员在打球前会通过听歌让自己兴奋起来,他们还会相互喊话鼓劲,然后全身心地投入比赛。我们学习戒色文章也要学会找状态,像学校早晨的广播体操,其实就是一种激励热身运动,通过身体的适量运动,很快让心理找到感觉,一旦身体振奋活跃了,心理也会很快进入状态,从而告别昏昏沉沉的不良精神状态,变得更容易集中注意力,这样学习效率就会提高。这里要提醒大家的就是不要做太激烈的运动,稍微活动一下即可,不能搞得太累,那样反而不利于学习。

\paragraph{学习计划}

曾经有戒友要我给他做戒色计划,其实我自己也没详细的计划,所谓计划赶不上变化,我只有一个模糊的计划,遇见具体情况是需要作出调整的,我那时就是保证自己每天都要看一页书,至少一页书,状态好时,可以多看,状态差时,不能少于一页书,就是保证每天都有学习,一天都不中断。这样坚持一段时间,就会真正看进去,然后就把页数规定为五页,不能少于五页,当然是要做笔记的,不是走马观花看过就忘记。不断地看、不断复习,这样觉悟就能持续提高。龟兔赛跑,不怕你跑得慢,就怕你彻底休息。只要你在跑,你就会越跑越好,最终量变产生质变,迎来大翻身。实际戒色中,很多戒友毫无学习计划,单纯靠热情戒色,热情一消退,对戒色文章就毫无兴趣了,再也找不到感觉了,越戒越差。如果能养成每天学习的习惯,那就不会出现热情消退的尴尬了。持续提高觉悟是戒色成功的王道,大家应该保持良好的学习习惯,不要单纯靠热情戒色。有了学习计划,就可以按进度去做,就不会一曝十寒,不要规定自己多久能戒色成功,做好每一天才是真的,每一天都做好,自然会迎来戒色成功。给自己压力,但不要给自己太大压力,太大压力很可能会导致失去信心。我的学习计划,就是每天都有学习,不一定要多,但不可中断。

\paragraph{全断隔离}

就像把你关进监狱,断网,断电视,断手机,禁言,只给你戒色文章还有纸笔。你什么都不能干,只能干一件事情,那就是学习戒色文章。全断隔离,其实就是为了排除干扰,现实生活中干扰太多了,让你分心的事情也太多了,这样就很容易影响到学习的效率,看一会书,马上去玩手机了,看一会书,马上上网打游戏了,没有定性,不能集中注意力干好一件事。如果有人命令你,今天必须看一百页的书,看不完不能吃饭睡觉,你会怎么样?你肯定会很投入去看书,当你排除一切干扰,你会发现自己的潜能完全出来了,原来你真的可以一天看一百页书,甚至两百页、三百页,乃至一本书。有些戒友对戒色文章完全入迷了,如饥似渴地想学想看,他的投入程度决定他的进步速度。如果你排斥学习,或者三心二意马虎学习,那样进步是很缓慢的,效率低下。

\paragraph{休息最少化}

一周当中可以安排一两天休息最少化,休息最少化就是为了保持学习的持续性,也就是尽量不中断。一鼓作气看下去,如果一中断,原来那股气势就会削弱很多。就像打篮球一样,突然让你下场,再上场可能完全找不到感觉了。尽量把更多的时间用于学习,省掉闲聊的时间,省掉喝咖啡的时间,省掉逛街的时间,省掉打球的时间,把更多的时间倾注在学习上,这样你可以一天学习八小时,或者十小时,这一天就是为了高效学习而特设的。就像运动员有高强度训练日,教练一般会先安排中低强度的训练日,状态出来时,教练则会安排高强度的训练日,这样进步会更明显。戒色学习和网游练级一个道理,一天投入八个小时进步快还是投入一个小时进步快?当然,我不提倡你每天都学八个小时,但应该保证一周有一天是高投入的学习状态。

\paragraph{快进加速}

让一切活动都开启快进键,就像 DVD 的快进键,你可以走路快些,你可以吃饭加快些,不要边吃边闲聊,你可以刷牙稍微快些,一切的加快都是为了把时间挤出来用于学习,让自己的生活具备一些紧迫感,不要懒懒散散,一点都紧不起来,那样是无法进入最佳的学习状态的。有了一定的紧迫感,全身的能量就会被激活,干事情也更容易集中。这其实也是给自己一点压力,无压力则无动力,有压力才能出油,稍微给自己一点压力,这样效率就会出来。不过,大家要注意,不要给自己太大压力,那样很可能会适得其反。

\paragraph{切割时间}

所谓切割时间,就是把时间分成一段段的,然后在每一段时间内都尽量达到最大效率,就像把时间都切割成 45 分钟,在每个 45 分钟内,你要给自己上紧发条,努力在这 45 分钟内达到最高的效率,比如你原来 45 分钟内只能看十页书,但是如果你有这个效率意识,你就会给自己鞭策,在 45 分钟内就可以看 15 页书,这就是效率的提升。高效的 45 分钟过后,可以休息 15 分钟,然后再开始下一个高效的 45 分钟。

\paragraph{经常复习}

我一直坚信复习是学习之母。温故而知新,每次复习都会有新的收获,每次复习都是再巩固再加深认识。大家应该时常复习戒色笔记本,你的戒色笔记本就是文章的精华所在,你要通过复习,把那些精华的戒色知识和意识转化成自己的,如果你能有自己的思考和领悟则更好。有的戒友不能每天上网,但千万不要离开戒色文章,在定力尚浅时,更应该每天都要抽出时间来复习戒色笔记本,高三党时间少,我建议每天至少抽出十分钟来看下戒色笔记本,然后保持高度警惕。不断复习是提高吸收率的最好方式,大家一定要注重复习。

\paragraph{学会放松}

在学校老师应该会说劳逸结合,我们学习戒色文章也是如此,有休息最少化的高效日,但也没必要每天都绷得很紧,马放南山是为了养精蓄锐,高效日每周安排一两天即可。其他时间可以自己调整,累的时候少看些,状态好的时候可以多看几页,但尽量不要中断,至少至少要看一页,养成良好的学习习惯很重要。

\paragraph{尽早完成}

尽早完成任务,其实就是避免拖沓,很多人都喜欢拖沓,拖到不能再拖才想着去做,那样其实就很被动了。比如可以下午完成的任务,一定要拖到晚上熬夜,时间流逝是很快的,不知不觉就过去了,所以我们要抓紧时间,时间管理有把任务分为:重要、不重要、紧急、不紧急。我们要首先完成紧急的任务和重要的任务,把握主动,切忌拖延,越拖延越被动,最后导致仓促完成,效率很低下,甚至出现无法完成的情况。对于需要完成的任务,我只想说四个字:马上行动!不要再拖延了,请果断些。

\paragraph{寓乐于学}

就是让事情变得有趣,不要把学习当作一件很痛苦很困难的事情,而是要把学习当作一件很自然的事情,让学习变得有趣很重要。学习戒色文章也是如此,主动学习者永远比被动学习者强百倍,主动学习者就像牛自己吃草,吃得很欢。而被动学习者是要人按着才肯吃,甚至按着也不吃,这就没办法了,师父领进门修行在个人,你自己不看戒色文章,谁也帮不了你。我们要学会纠正自己的错误想法,学习并不痛苦,只是为了让自己懂得更多,有了觉悟才能战胜心魔,每学到一条戒色知识,你应该感到高兴才对,并且要有学习更多知识的渴求,这就是求知欲,强烈的求知欲决定着你的进步速度。为何有人看一遍戒色文章就厌烦了?他以为自己懂了,其实只是走马观花,而有的人可以看十遍以上,每次看都有新收获。你说这两种人,哪种人更容易戒色成功?不言自明。

\paragraph{设置专用时间}

专用时间很好理解,就像吃饭有吃饭的时间,睡觉有睡觉的时间,而我们学习戒色文章也最好能做到有规律且定时定量。每天什么时间学习戒色文章先规定好,然后到点就好好用心学习,在专用时间里,尽量达到效率最大化,其他任何事情都靠边,这段时间戒色专属,雷打不动,专心学习。

\paragraph{分解目标}

要完成一个大目标,可以分解成若干小目标,比如一本六百页的戒色书籍,你可以每天看十页,六十天即可看完,保证进度不中断很重要。如果你每天看二十页,三十天即可看完,如果你每天看五十页,十二天即可看完。把大目标分解成每日可以轻松完成的小目标,这点技巧非常重要。很多人看着六百页的书望洋兴叹,这猴年马月可以看完啊?其实掌握了方法,即使以前页的书也可以很快看完。就像造房子,保证进度,万丈高楼平地起。就怕你无法保证进度,三天打鱼两天晒网。

\paragraph*{总结}

这季分享了时间效率管理,就我的经验和认识和大家分享了一下,时间效率管理是一门学问,有兴趣的戒友可以自己买些相关书籍看看。当你掌握了相关的方法,效率就会提升好几倍,学习工作有了效率,你就可以更好地安排自己的生活。
