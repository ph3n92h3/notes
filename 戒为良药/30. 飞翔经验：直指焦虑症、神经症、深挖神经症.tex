\subsection{直指焦虑症、神经症、深挖神经症}

\paragraph*{前言}

这几天看到一个帖子,是前列腺炎戒友,他说他有腹痛的情况,每次排精后就不痛了。我的建议就是,不要希望通过 SY 排精来缓解炎症。有炎症应该积极治疗,控制住病情,然后好好坚持戒色养生,这才是身体恢复的正道,否则,你靠 SY 排精来缓解炎症,这只会让你陷入恶性循环,身体会一步步垮掉的,前列腺炎永远别想好了。我们治疗前列腺炎,指导思想一定要正确,前列腺炎主要靠养,其次才是靠治,三分治,七分养,治疗控制病情,然后就是要靠戒色养生了,如果你没有戒色养生意识,前列腺炎就极有可能复发,为什么前列腺炎的复发率在 90\% 以上,原因就是他们的指导思想出现了误区,还以为靠药能治好,其实前列腺炎靠药是无法根治的,前列腺炎是养好的,必须学会养生之道,建立起戒色养生的意识。这样前列腺炎才能慢慢痊愈,一般慢前患者的恢复时间在一年左右。如果你一边吃药一边 SY,那花多少钱都治不好了,很多戒友花了好几万,用了很多种治疗手段,先进的机器都给用上,各种疗法都给用上,还是不行,就是不懂得戒色养生,钱花冤枉了,身体依然好不了。医生也不会告诉你为什么治不好,一方面,有的医生思想存在误区,认可无害论。另一方面,有的医生也希望你治不好,这样医院就可以搞创收了。很多医院,医生看病是吃回扣的,你觉得他会和你讲 SY 有害吗?

另外,大家非常关心的晨勃问题,其实也可以用雄性激素的分泌来解释,一般清晨五点左右,雄性激素分泌达到高峰,然后就会出现晨勃现象。而在发育期时,雄性激素分泌极其旺盛,这样就可能会每天都有晨勃,记得我刚发育那会,的确有一段时间每天都会出现晨勃,后来沉迷 SY,晨勃现象就减少很多了,直到后来彻底消失了,当然影响晨勃的因素有很多,比如饮食因素,吃肉比较多,吃补肾的食物比较多,晨勃就容易出现。还有就是情绪因素,情绪和晨勃是有密切关联的,这点很多人容易忽视,情绪可以影响雄性激素的分泌,从而影响晨勃。比如你生活中遇见不开心的事情,然后这种悲观的情绪是会导致雄性激素分泌下降的,这样就可能出现晨勃消失的现象。情绪和激素是互相影响的,很多时候情绪低落都与雄性激素分泌失调相关,激素是会反过来影响一个人的情绪和行为的。比如有的人今天醒来,感觉情绪很好,心里有愉悦感。但是过十几天后,情绪就突然变得很低落,这其实和体内的激素分泌密切相关。女性更年期,随着卵巢功能逐渐减退,雌激素减少,然后她的情绪和外貌都会发生变化,情绪的显著变化就是伴有敏感、多疑、烦躁、易怒等不良情绪。男性其实也有更年期,男性到了更年期,随着体内睾酮的减少,情绪和身体也会发生一系列的变化。所以说情绪和激素是互相作用的关系。

还有一个影响雄性激素分泌的显著因素就是运动习惯,适量的有氧运动和力量训练是可以促进雄性激素的分泌的,我每次杠铃深蹲第二天早上就会出现晨勃,深蹲是可以刺激雄性激素分泌的,有氧运动也可以提升雄性激素的分泌水平,但有一个前提,就是不能过度锻炼,过度锻炼会导致雄性激素的下降,不管是有氧还是力量训练,过度了都会导致激素分泌下降。季节因素也不可忽略,现在秋季,一年中雄性激素分泌最高,我感觉自己晨勃次数多了很多。

如果从生理角度去理解,可以说,人的本质就是激素,人体会分泌七十五种以上的激素,它们在人体内扮演着各自的角色。体内荷尔蒙浓度高的女性,比体内荷尔蒙浓度低的同龄女性看起来要年轻很多。而体内雄性激素分泌旺盛的男性,进取心和攻击倾向,也比一般男性要强很多。

有戒友会说,戒色后晨勃会不会恢复,因为他沉迷 SY,晨勃已经消失很久了,我的回答就是,坚持戒色养生,晨勃是会恢复的,我以前晨勃曾一度消失很久,后来坚持戒色养生又恢复了。还有的戒友会问,SY 到底会导致雄性激素分泌增加还是减少,其实这个问题很好回答,SY 导致的失调问题,基本都走两个极端,有的戒友沉迷 SY 后,雄性激素分泌减少了,比如我,SY 后感觉胡子少了。而有的戒友则是 SY 后,胡子体毛生长更加旺盛。之所以出现两个极端,其实和个人的体质有关,中医有把人分成九种体质,SY 后的激素分泌失调,要看个人体质而定,在有的人身上表现为雄性激素分泌增加,在有的人身上表现就是雄性激素分泌减少。分泌增加其实也不是好事,容易造成雄秃的可能性。我所看过的案例,以 SY 后雄性激素分泌减少比较常见,大概占 70\%,还有 30\% 就是 SY 后雄性激素分泌增加。

关于晨勃问题,大家也不必强求天天要有晨勃,随着年龄的增长,你会发现晨勃会慢慢减少的,晨勃和多种因素有关,关于晨勃我们要懂得顺其自然,不要强求。我现在主要关注的并不是晨勃的频率,而是晨勃的质量,一次质量好的晨勃,可以超过一百次差的晨勃。持久坚挺的晨勃为好的晨勃,也可以作为早泄是否恢复的指标,早泄是不可以随便试的,可以从晨勃质量中观察到是否恢复了,如果你去试,很可能又掉进 SY 的陷阱。很多人虽然有晨勃,但晨勃质量很差,晨勃不坚不持久,硬度不行后劲不足,而且很多人晨勃是有 YY 的晨勃,并不是真正意义上的晨勃。另外,憋尿的晨勃也不是真正意义上的晨勃。

下面步入正题,这季就一个主题:直指焦虑症,神经症,深挖神经症。具体论述如下。

\paragraph{关于神经症}

神经症,全称是神经官能症,对于很多戒友可能很陌生,因为很多戒友还没伤到那个程度,难以理解神经症到底是什么感觉。没有经历过神经症的戒友可以大致了解下,增长一下自己的见识,也可以警示一下自己。

神经症是伤精患者的一道分水岭,伤到神经了,要恢复相对就比较慢了,轻微的神经症还好恢复些,严重的至少一年以上才有望恢复。这个病是需要悟道才能好的,靠吃药是极难痊愈的,很多人始终没有开窍,始终不明白为什么,就像进入一座迷宫,进去后就迷在里面了,几年乃至十几年都出不来,每天都靠药维持着,生活质量大受影响。很多特别严重的,会出现自杀倾向,媒体上经常有抑郁症焦虑症自杀的新闻,有的人有钱有名,资产上亿,结果他也自杀了,普通人想不明白,为什么会自杀呢?其实一旦得上了严重的神经症,想自杀的极多,我也曾经出现过自杀的念头,那时的我,每天活在症状地狱里,没经历过的人,的确很难体会到那种感觉,那是对身心的极大摧残,很多人得病后,性格都发生了巨变,判若两人。

我那时分别出现过恐惧症(社恐、恐艾、恐癌)、疑病症、强迫症、焦虑症、神经衰弱、胃肠神经症、心脏神经症,头也昏沉过,也头痛过,还有全身游走性刺痛,简直就是地狱般的刺痛,简直就是酷刑,躺在床上我老出现地震的感觉,后来才发现是严重的躯体震颤感,肌肉也乱跳,我那时真的快崩溃了,全身没一个地方舒服的,我之前不相信有人间地狱,得了神经症,我信了。身处症状地狱,的确是一种莫大的折磨。

神经症其实非常强调体验,如果你没体验过,你是无法真正理解那种感觉的,就像我和你说蹦极的感觉,你只是听我说,但没有去尝试过,这样你的体验和感受就不会很深刻,也不会很直观。很多医生没得过神经症,他们对神经症的理解来自于书本,而我可以很负责地说,书本上很多理论还不够完善,特别是西医在这方面的理论还处在研究阶段,所以很多人对神经症会产生误解,包括很多医生。不可否认的一个事实就是,庸医多,真正明白的医生少之又少,看几十个医生,能遇见一个真正懂的,就算是你的造化了。神经症患者被误诊是极其普遍的,有的患者心脏不舒服,当心脏病来治疗,而不是当神经症来治疗,这样治了大半年,花了好几万,还是看不好,也没有多大缓解,搞到最后才知道是得了神经症,而之前一直当心脏病来误诊,所以,神经症患者对于医生是颇有微词的。对于神经症,我推荐还是看中医比较好,西医的药副作用大,也容易造成依赖,很多病友都依赖上了药物,药物上瘾,不吃不舒服,其实吃了也没好多少,在一种恶性循环中苦苦挣扎。

神经官能症导致的常见疾病:

\begin{multicols}{2}
    \begin{itemize}
        \item 慢性咽喉炎、口腔溃疡;
        \item 肠易激综合症、结肠炎、慢性胃炎;
        \item 神经性头痛、头晕、头昏、失眠 、多梦;
        \item 抑郁、焦虑、恐惧、强迫、疑病症;
        \item 多汗、虚汗、盗汗、怕冷、怕风;
        \item 心脏神经官能症、胃神经官能症;
        \item 脖子肌肉僵硬 、关节游走性疼痛、幻肢痛;
        \item 记忆差、反应迟钝、神经衰弱;
        \item 早泄阳痿、易感冒、免疫力低下。
    \end{itemize}
\end{multicols}

神经官能症患者有的人以胃肠神经官能症为主,有的以心脏官能症为主,有的则是头部极其不舒服,还有的患者症状很多,基本都有。我聊过的病友中,很多人都有十几种乃至几十种症状,千奇百怪,出症状的规律是此起彼伏、层出不穷,搞得自己很恐慌,其实根源只有一个:那就是植物神经功能紊乱导致了免疫系统的功能紊乱。那是什么导致植物神经紊乱了呢?我当时也很困扰,一直在找原因,最后经过我的深入研究,聊过上千的病友,得出一个公式:熬夜 + 纵欲 + 久坐 = 完蛋。光纵欲,那需要伤到一定程度才会出现神经症,我过去十几年频繁 SY,只是前列腺炎和精索,虽然人也变丑好多,但是神经一直是好的,后来我久坐熬夜,不好好吃饭,然后就出现神经症了。应该这样说,神经症的出现是众多因素共同作用的结果。女病友则是生气、压力大、熬夜导致的,还有精神刺激,比如家庭变故等。

\paragraph{真假神衰}

很多戒友因为沉迷 SY,出现了记忆力和理解力大幅度下降的情况,这种情况,有的戒友就会觉得自己得了神衰,其实据我观察,很多戒友只是脑力下降而已,不一定是神衰。

选了几个典型神衰的表现,大家可以对照下:

\begin{case}[典型神衰的表现]
    本人男,二十五岁,由于小时候(十二岁)一次偶然机会不幸染上 SY,SY 两年后开始出现不适症状:头晕耳鸣、全身乏力、失眠、视觉昏暗,感觉眼前的世界很不真实,好像自己活在做梦一样,腰部酸疼、精神萎靡、胸闷气短、心慌心悸、记忆力下降、脸色黑黄、瞌睡倦怠、额头两侧脱发、手心脚心出汗、大便不成形,还有泌尿方面的疾病如尿频、尿不尽。
\end{case}

\begin{case}[典型神衰的表现]
    飞翔大哥你好,小弟 SY 最少有十年的历史了。今年二十一周岁。今年五月末突然发现脑力不足,精神恍惚,觉得整个世界有空虚感。因为正在备战考研,真的很无助,身体一直还不错,也经常打球啥的。经过了解我大概确定应该是有神经衰弱,最严重的时候晚上根本无法入睡,每天就两个多小时,晚上盗汗,人消瘦了五六斤,尤其是脊背瘦多了,肚子上还好,下眼皮突然变黑,白天觉得眼睛不好使,眼睛老是被重压着很难睁开,白天就像喝醉酒似的,天哪!真不敢想象,那时候都不敢上街,害怕被车撞了……
\end{case}

\begin{case}[典型神衰的表现]
    我今年二十岁,身体发黄瘦弱,五年前开始无节制的 SY,起初身体没什么症状,可是一年后的一天在上楼途中突然脑后一沉,然后就开始头昏目眩,开始只是认为是发烧引起的,可是后来一直是这样没有好过,直至现在还是这样。五年期间人一天一天的衰老下去,全身无力、精神萎靡、记忆力减退、注意力不能集中,休息睡眠更是差得要命!人慢慢地变成了废人,也看过不少大夫吃过不少药,可是都不见效。
\end{case}

\paragraph{感恩焦虑症}

虽然焦虑症给了我很大的痛苦体验,但我还是要感恩焦虑症,因为没有焦虑症,就没有现在的我,是焦虑症度了我,是焦虑症让我开始自学中医,是焦虑症让我开始悟道的生活,把我的精神世界提升到了另一个境界,塞翁失马焉知非福,是焦虑症点化了我。我是以病入道的,据我所知弘一法师李叔同也得过神经症,神经症是他出家的一个助缘,文摘如下:

\begin{quote}\it
    今得阅弘一法师述,高胜进笔记的《我在西湖出家的经过》,始知其绝食出家的本因:患有神经衰弱症。其文曰:“到了民国五年的夏天,我因为看到日本杂志中,有说及关于断食方法的,谓断食可以治疗各种疾病。当时我就起了一种好奇心,想来断食一下。因为我那个时候患有神经衰弱症,若实行断食后,或者可以痊愈亦未可知”。
\end{quote}

\paragraph{濒死感体验}

焦虑症,严重的会有惊恐发作濒死体验,先选两个案例,大家可以看下:

\begin{case}[濒死感体验]
    今年一月独自上街,突然觉得胸闷、心悸心慌、呼吸困难,濒死感,随即送医院,当时测血压正常,心跳 100 每分,心电图正常。此后不敢独自上街,或在人多的地方或与人聚餐是就容易紧张,现在还不定时发作,发作时气透不过来的感觉、心悸、出很多汗、全身发冷、濒死感、不真实感,要发疯感或失去控制感,每次发作约一刻钟左右,非常难受,现在还不时有头晕,脖子痛。做过 SCL-90 的测试,恐怖重,焦虑中,其余轻度升高。看过中医说,舌淡苔薄白,脉沉偏细数。我有 2 型糖尿病,担心服西药副作用大,给身体负担更重,但服用中药、针灸、耳穴压豆,但效果一般,担心吃西药副作用大,成瘾或依赖,希望在医生指导用药,降低我的忧虑。
\end{case}

\begin{case}[濒死感体验]
    我得焦虑症四年了,经常惊恐发作,发作时心跳加速,达到一百四五十次,一开始还以为是心脏病,但做了五次 B 超和冠脉 CT、两次心机酶、三次心电图、一百多次二十四小时动态心电图、X 射线甲亢胸片,全正常。发作时感觉从血管里发冰、发冷,然后莫名的心跳加速,非常恐惧!简直就象要死了!!发作时心跳得超级快,但打出来的心电图却只是心动过速,医生诊断惊恐发作。哎!也没给任何药物,大概过了二十分钟,心跳就自动恢复了,恢复到七十次左右了。痛苦啊!
\end{case}

有病友会问我,惊恐发作会不会死,这个问题其实不好回答,因为死人不会说话,能救活的人基本都会说濒死发作不会死。全国每年五十五万人猝死,你觉得里面是否有焦虑症患者呢?

\paragraph{SY是会伤神经的}

SY 的确会伤到神经,但要伤到一定程度,如果再有熬夜久坐,这样出现神经症的概率会加大很多。伤到神经是非常痛苦的,那是一个真正的症状地狱,那是一个深渊,不被理解的深渊。掉进去后,也很难爬出来,必须彻底悟道才有可能真正摆脱它。

\paragraph{得病过程}

我频繁 SY 十几年,因为热爱运动,按时作息,合理饮食,所以没有出现过神经症。后来开始熬夜久坐了,就不对了,身体一天天不行,被神衰、焦虑、植物神经紊乱折磨着,过着生不如死的生活。而我得病前后就一个月,也就是说,我放纵了一个月,然后就出现了神经症!而且更让人难以置信的是,那段放纵的时间,我每天都做几百个俯卧撑,卧推有 120 \unit{\kg} ,衣服脱了很强壮的感觉,一点都不像会生病的感觉,但我也得上了神经症,而且是那种生不如死的神经症,并伴有严重的恐惧、焦虑,几十种躯体症状轮番折磨我,后来我研习中医才知道,强壮并不代表健康,很多强壮的人其实都有毛病,属于暗疾,只是别人不知道。真正会养生的人,才是健康的人。强壮的人,不是,外强中干的很多。

\paragraph{误诊经历}

我开始是胃肠神经官能症,然后去消化内科看,当消化问题治疗,但效果很不好,后来才知道被误诊了,其实是神经的问题,还看过皮肤科,然后按皮肤病治疗,效果也不好,西医分科有缺点也有优点,缺点就是神经症这类病,很容易被误诊,有的心脏官能症的病友,一误诊就是大半年,花费几万都看不好,光检查费都上万。后来我自己查,才慢慢觉得自己很可能是神经症,再和病友一交流就更加确定了,然后我去神经内科一看,确诊为焦虑症。

\paragraph{疑病倾向}

得了神经症后,出现疑病的病友极多,因为疑病,总是不断地去医院检查,检查费惊人,关于这点,其实医院也承认,疑病倾向的人,的确在为医院搞创收。这些人都以为自己快不行了,以为自己得了大病,希望能检查出来,检查后,一切正常,心理舒缓些,过段时间又不舒服了,然后再去检查,就这样很多人的钱都花在了检查上,其实他们不知道,是神经的问题,根本就检查不出什么器质性问题。当然,该检查的检查下,也没必要经常去检查,检查一次排除了器质性的问题,就不用经常去检查了。一定要认识到问题的所在,是神经出了问题。

\paragraph{注定被误解}

身体出现很多不适症状,第一反应肯定去医院检查,但是神经症检查很多次,你会发现基本没什么大的问题,但是症状是明显的,让人崩溃的,让人恐慌绝望的。但检查出来,却没多大问题,这种情况下,家人就会认为你装病,或者说,病是你想出来的,是你自己多想了,其实你没病。这种对神经症的误解太多太多了,不仅家人会误解你,很多医生也会误解你,因为医生也会认为,病是你想出来的,这样很多病友,处在无法被人理解的困境中,好在现在有网络,才能找到同病相怜的病友,否则一直这样不被理解,真的有可能走上绝路,很多病友之所以那么喜欢聊天聊症状,就是图一个心理安慰,知道有很多人和自己一样。其实光聊症状也不行,好不了,必须要找到真正的病因,病因认识不正确,真的万难痊愈了。有的病友,检查费好几万,甚至十几万的都有,还有的病友经常抢救,经常住院,医生也拿他没办法,检查不出什么毛病,但身体就是不行。有病友会问我,他的身体到底怎么了,后来我和他说,其实神经症基本都是功能性的问题,又不是器质性的问题,基本检查不出什么问题,打个比方,你一部自行车骑了一年多,很旧了,你这时候感觉旧车和新车已经没法比了,旧车骑着很累很费劲,但是你把旧车送去检查,零件都正常,就是旧了,功能没新的时候好了,其实神经症就是这种状况,功能大不如前了。

还有一种误解,来自普通人,因为抑郁症和焦虑症,一般人都以为是心理出了问题,和躯体没任何关系,是一个人想多了导致的,其实这真是大错特错了,很多人发病前正常得不得了,乐观向上,突然的焦虑症把他击垮了。所谓心理疾病,并不是单纯的心理问题,而是有生理基础的,不断 SY 伤肾气,熬夜久坐也伤肾伤脾伤肝,然后就出现了神经症,如果你单单认定为心理疾病,是想出来的,这是不正确的,现在普遍这样认为,说是心理疾病,其实是错的,我是以一个深刻的体验者和研究者的身份来讲的。我觉得没深刻体验过的人,很多想法都是有偏颇的,包括很多医学教材上的论点,都不是正确的,并不符合真正的事实。就像无害论一样,现在无害论也上教材了,难道它是正确的吗?事实证明它是错误的,尽信书不如无书,前人正确的论点,我们要继承,如果是错误的,我们要有勇气去纠正。事实证明,心理疾病并不是简单的心理问题,并不是想多了。打个比方,惊弓之鸟的成语大家都知道,鸟本来有伤,然后听见弓的声音,一紧张,伤口破裂掉下来了,大家想一想,如果这只鸟没伤,能掉下来吗?肯定不会,其实神经症也是如此,如果你肾气充足,会得神经症吗?中医:肾气足,万邪熄。所以说,心理疾病的发作是有生理基础的,原因是你身体虚掉了,然后给一个刺激就突然发病了。心理疾病也不是那样简单,是有一大堆躯体症状的,焦虑情绪不代表焦虑症,抑郁情绪也不代表抑郁症。有的病友是这样反映的,他原来是非常乐观的人,是突然发病后才变得焦虑,并不是焦虑导致焦虑症,而是发病后开始出现焦虑、恐惧、强迫等倾向。我自己也是这样,我根本没多想,一直很乐观向上,结果熬夜久坐纵欲,然后就得了焦虑症。

如果说,是不是有想出来的焦虑症,我想如果真有这种情况,也是占很小比例,比如有的人天生体质极差,肾气严重不足,这样受点刺激也是有可能出现焦虑症的,但我通过上千的案例比对和分析,可以说 99\% 都是有生理基础的,而这种发病的生理基础并不是先天体质不好,而是后天放纵自己和不良的生活习惯导致的,并不是想出来的病。

\paragraph{吃药回忆}

吃药,是我不堪回首的一段经历,因为我严重依赖过药物,把药当饭吃!我现在家里还有一大箱没吃完的药,很多都过期了,我还留着,就是提醒自己,我曾经活在症状地狱里。我不想再放纵自己,我不想再重返那个生不如死的症状地狱。我那时谷维素吃了几十瓶,没多大效果,黛力新天天吃,刚开始吃,效果不错,吃多了就耐药了,然后看医生,就给我换药,然后再继续依赖药物,光吃药就花费了我不少钱。关键是吃了还不能痊愈,最多只能缓解,那真叫一个生不如死。年纪轻轻,靠药活着,不吃药,比死都难受。普通人无法理解那种感受,只有病友可以理解。

\paragraph{直指病因}

病因在上面的文章已经提到,男病友就是:熬夜 + 纵欲 + 久坐导致的,这是我聊了上千个病友得出的答案。肾为元气之根,熬夜、纵欲、久坐就像三把斧子砍向生命之树。当然个别人吸毒、酗酒、暴怒,这也很伤身体,也很容易导致神经症的出现,另外遗传、压力大、家庭变故等因素也可能诱发神经症。

\paragraph{康复过程}

神经症的康复过程,基本就是两个方向:

\begin{multicols}{2}
    \begin{itemize}
        \item 治心
        \item 治身
    \end{itemize}
\end{multicols}

身心是合一的,治身可以影响心理,同样,治疗心理也可以影响身体,心理的健康也是极其重要和关键的,所以,得神经症后,很多病友都开始信佛,信佛对修心是非常好的,通过修心来影响身体的健康。我现在也信佛了,当然我有佛缘,没佛缘的病友可以多学习其他方面的传统文化,也是很利于修心的,比如《弟子规》就很好。心对,身也会跟着对。

治身方面我推荐中医调理,然后自己要多学习养生知识,可以选择适合自己的养生功法坚持练习,这样身心同治,神经症是可以慢慢恢复的,严重的至少需要戒色养生一年以上,很多病友不懂得戒色,不理解戒色,这类病友其实根本就没有认识到保精的重要性,肾为先天之本,脾为后天之本,要恢复,必须懂得保精,也要懂得养脾。这样各方面都做好了,养功到位,神经症才有望恢复,否则很多病友五年没恢复,甚至十几年没恢复的都有,就是不懂得养生,不重视保精!还在漏,还在 SY,还在自我摧残,这类病友就是没有悟道,不悟道,神经症太难好了,即使暂时痊愈了,也极易复发,很多病友在放纵后又复发了,非常常见。所以,神经症病友必须认识到戒色的重要意义,否则你的神经症太难好了,症状地狱会让你生不如死的。戒色养生是痊愈的基础,否则吃再多的药,也会被你漏掉。

\paragraph{就医指导}

我推荐看中医,而且是百里挑一的老中医、好中医!因为庸医太多,你去看医生,最好先在网上了解哪个医生好,哪个医生的医术高超,可以去所在城市的医院网站或者中医 QQ 群了解,否则盲目就医,很可能会遇见庸医,有的庸医看不好就罢了,甚至还会向你灌输无害论,或者说你的病是想出来的,这类庸医和患者之间缺少真正的理解,他们的医术很值得怀疑。好中医永远是少数,得真传有心得的更是少之又少,所以一定要找好中医看。

再打个比方,一个中医学博士,和一个初中学历,但行医四十年的老中医,你会选择谁?我想懂行的人肯定会选择老中医,老中医阅历深厚,在治疗方面有着丰富的经验,这在学校里和书本上是学不到的,只有通过实践和反复体验才能悟得真机。否则纸上谈兵,永远成不了好中医。有很多科班出身的中医,你要说理论,你不会比过他,他背书能力强,口若悬河,但我要问的是,真正的实践经验有多少呢?中医是讲究师承的,能得到大师指点,口传心授,这样进步提高才会比较快。真正的好中医永远是少数,他知道的比普通中医要多要深。

\paragraph{最后总结}

我是以病入道的,应该算体验派,自己有深刻体验,然后再开始研习中医理论,我也是痊愈者,因为悟道所以痊愈。明道了才知道该怎么去做,我希望神经症患者看到这篇文章,能得到有益的启发,我是从症状地狱爬出来的人,太清楚那种感觉,希望我的文章能帮到你们,按照我指的路去坚持,相信你们会痊愈的。但恢复是一个缓慢的过程,需要保持耐心。加油!
