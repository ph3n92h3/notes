\subsection{学习《了凡四训》的心得体会}\label{129}

这季就《了凡四训》和广大戒友做一个分享,几乎每位戒色前辈都有研读《了凡四训》,也是戒色后必读的一本书,戒色十规里也专门提到了这本书。《了凡四训》和《俞净意公遇灶神记》可谓中国改造命运学的扛鼎之作,让大家了解和掌握改造命运的原理与方法。这两本书各有特点,总体而言,《了凡四训》更全面、更系统一些,而《俞净意公遇灶神记》则更有针对性,对意恶这个问题的指出更直接,更一针见血。这两本书我都很喜欢,了凡先生和俞净意公是中国改造命运的两位巅峰人物,他们的事迹与教导我们要认真学习,对于我们自己改造命运是很有启发和帮助的。

《了凡四训》又名《命自我立》,是明朝袁了凡先生结合自己亲身的经历和毕生学问与修养,为了教育自己的子孙而作的家训,以其亲身经历教诫他的儿子袁天启,认识命运的真相、明辨善恶的标准、改过迁善的故事。阐明“命自我立,福自己求”的思想,指出一切祸福休咎皆自当人掌握,行善则积福,作恶则招祸,并现身说法,结合儒释道三家思想以自身经历体会阐明此理,鼓励向善立身,慎独立品,自求多福,远避祸殃。该书自明末以来流行甚广,影响较大,了凡先生在没有遇到云谷禅师前,光是知命认命,听凭命运的安排,那是消极而被动的,而后根据立命之学自强不息地改造命运,则是积极而有益的。读了《了凡四训》,可以使人心目豁然,信心勇气倍增。《了凡四训》实在是一本有益世道人心,净化社会风气不可多得的佳作,越来越被后人誉为家教典范。《了凡四训》虽然文章篇幅短小,但是寓理内涵深刻,所以数百年来历久不衰,为各界人士所尊崇。印光大师对这本书也极力提倡,弘化社印送这本书约在百万册以上,由此可知,印光大师对这本书高度重视。

先了解下了凡先生,了凡先生全名袁了凡,即袁黄(1533 - 1606),初名表,后改名黄,字庆远,又字坤仪、仪甫,初号学海,后改了凡,后人常以其号“了凡”称之。浙江嘉善县魏塘镇(今属嘉兴)人。晚年辞官后曾隐居吴江芦墟赵田村,故一作吴江人。

袁了凡碰到两位高人,在慈云寺,碰到了一位老人,相貌非凡,一脸长须,看起来飘然若仙风道骨,这位老人就是算命高人——孔先生。孔先生算命算得很准,说他将来在县考中可以考到 14 名,府考能考到 71 名,提学能考到第 9 名。结果第二年的考试中,果然如这位孔先生所料,考试的名次分毫不差。孔先生说他只能活到五十三岁,做官只能做三年半,膝下无子,半生无福。因为算命先生的话屡屡应验,所以袁了凡对此深信不疑,自此之后,听天由命,不思进取。直到有一天,他遇到了一位高僧——云谷禅师,云谷禅师是明代中兴禅宗的大德高僧,云谷禅师对他说,你虽然命中注定没有多少功名,也没有儿子,但是这一切都是可以改变的。

孔先生的算命水平是非常高的,一般的算命不会算得那么细,连考试名次都能算出来,实属罕见。孔先生的确是一位高人,能真正读懂命理信息,就像一张光碟所含的信息一样,播放出来就是含有时空的人事物,高人可以读出何年何月何时,发生什么事情,这一切的信息都是提前写好的,就像一个剧本一样。我以前也算过命,是看八字的,算得基本准确,说我会信佛,我那时的确开始学佛了,我后来查了自己的八字,是“华盖逢空”,所谓:“华盖逢空,偏宜僧道”,我虽然没有出家,但是对佛法还是很感兴趣的。我以前也专门研究过命理知识,的确是一门很深的学问,现在很多人号称能算命,其实只是略懂皮毛,有的也在靠算命骗钱,到处忽悠人,真正的算命高人永远凤毛麟角,万里挑一,有德行的算命人会劝你修行,而无德行的算命人只会吓你,忽悠你,骗你的钱。大多数算命的人只是一知半解,像孔先生这样的估计是算命界的极品了,能把名次都算出来,估计整个中国都没几个。

遇到孔先生只是一个铺垫,真正的高人是云谷禅师,云谷禅师是真正得道开悟的高僧,是憨山大师少年时代学习禅法的最重要的一位导师。云谷禅师老来悲心更切,即使是对七八岁的小沙弥,也一律以慈悲的眼色看待他们,以恭敬心对待他们。凡是动静威仪,都耳提面授,循序渐进、谆谆善诱。所以,见到老禅师的人都把他视为自己最亲近的人。云谷禅师平常教导别人,特别推崇净土法门,教人信愿念佛,求生西方极乐世界。各大丛林寺院,只要有开示禅法的道场,一定请他坐在首座。云谷禅师平日动静语默,安详稳重如山,沉默少言,一旦说话,便如空谷足音,醒人眼目。云谷禅师以定力摄持,住山清修,四十年如一日,夜不倒单,一生礼拜念诵南无观世音菩萨,没有一日间断。

云谷禅师传授的改命之法,有两个重要方面:一是行善改过、谦虚持德;二是持诵准提咒。为什么是准提咒,而不是其他咒?其实每种咒语都很殊胜,可能当时比较流行准提咒,抑或云谷禅师想特别推广这一咒语。我以前念过楞严咒、大悲咒、六字大明咒等,现在主要念往生咒 + 佛号,要看自己和哪种咒语有缘,对其有深厚的信心。袁了凡通过亲身践行,彻底改变了他原有的命运,按照孔先生的占算,了凡先生原本只能活到五十三岁,当一个小县官,也没有子嗣。结果,改运后却活到了 74 岁,当了高官,也有了儿子,而他的儿子也很贤能地考上进士,继续光耀家风。

《了凡四训》之所以让大家如此推崇备至,是因为它是改造命运的指南!很多人命运不济,苦苦找不到答案,看了这本书之后会有豁然开朗之感,也有了改造命运的勇气和力量。榜样的力量是无穷的,了凡先生现身说法,让你从根本处了解生命的意义在哪里,依据了凡先生给出的改命方案,知命、立命、改命,落地实修,从而重塑你的命运!《了凡四训》不可不学,它是改命学的集大成之作,我们应该认真研读,积极实践,它会影响和改变你的一生!不要抱怨命运的不公,更不要对生活充满怨气,那些困扰你很久的问题,都可以在《了凡四训》中找到答案。愿大家读完这本书之后,成为一个勇于改造命运、真正把握自己命运的人!

书中有立命之学、改过之法、积善之方、谦德之效四个部分,我会结合四个部分的笔记,和大家做一个详细深入的分享。

\subsubsection{第一篇:立命之学}

\begin{quote}\it
    云谷问曰:“凡人所以不得作圣者,只为妄念相缠耳。汝坐三日,不见起一妄念,何也?”
\end{quote}

\textbf{解析} 云谷禅师一句话点出了关键,凡夫之所以是凡夫,就是因为妄念相缠,做不得主,认念头为自己,跟着念头跑。要作圣,必须“克念”,战胜自己的念头,做回念头的主人。了凡先生的定功已然达到了不可思议的程度,可谓天姿超卓,普通人别说三天,就是一天、半天,或者半小时、十分钟、一分钟不起一个妄念,都是很难做到的。大家自己可以试验下,一分钟之内,能够保持观心的强度,不起一个妄念,能不能做到?如果你试验了,你肯定会发现,只要观力一弱,念头就进来了,把你带跑,很容易进入无意识跟念的模式。念头是会自动冒出的,即使你不去主动起念,它也会不断冒出,只有观力极强极稳定的人,才能做到长时间不起一个妄念,这是非常深厚的修心功夫。虽然了凡先生说自己是因为被孔先生算定才无妄想,其实他已经具备了很强的观照力,否则是不可能做到的。

\begin{quote}\it
    命由我作,福自己求。
\end{quote}

\textbf{解析} 不要有什么宿命论的包袱,宿命论容易导致消极,其实命运是可以改变的,人的福报是可以通过后天的努力修得的。只有明白了这一点,人才能真正地掌控自己的命运,而不是自怨自艾,怨天尤人。祸福无门,唯人自召,善恶之报,如影随形,\textit{夫心起于善,善虽未为,而吉神已随之;或心起于恶,恶虽未为,而凶神已随之,(《太上感应篇》)} 始作俑者其实就是自己的念头,好的念头会导致好的结果,坏的念头会导致坏的结果,真正明白了这一点就会懂得修心的重要性,要多发善心,学会对治邪念。自己的命运是自己造作的,福报也是自己修来的,一切都是自作自受。

\begin{quote}\it
    若不反躬内省,而徒向外驰求,则求之有道,而得之有命矣,内外双失,故无益。
\end{quote}

\textbf{解析} 能够反省,这是你改命的第一步,不去反省自己,认识不到自己的错误,那命运就无从改起。误区就是“向外驰求”,不知反省,不知修心,很多人虽然很努力奋斗,但是收效甚微,就是他的心态没有真正转变过来,还有很多负能量的念头,这样命运就很难改变。一个人若想得到外在的财富地位,必须从自身出发,反思悔改自己的的过错,断掉懒邪恶,坚持勤善正,去掉负能量,强化正能量,恒久力行,这样人生就会越来越好。曾国藩为什么要每日三省?就是怕自己有过错,过错会带来负面的影响,对自身很不利,所以要经常反省,及时发现,及时改正。

\begin{quote}\it
    汝不见六祖说:“一切福田,不离方寸;从心而觅,感无不通。”
\end{quote}

\textbf{解析} 六祖大师特别强调了“心”,修心是根本,而不是修身,修身的根本也在于修心,\textit{最根本的还是意业,人有念头,才会变成一言一行。(蔡礼旭)} 修行要注重修心,每个人的命运、缘分、风水都是由自己决定的,善心才是真正的护身符。真正的高人都懂得修心之理,要从起心动念上下功夫,学会控制自己的念头,这点是最根本的关键,也是最正确的认识,祖师大德都在强调修心。人的第一风水是心,人的善良、德行足以改变坏风水的影响,再坏的风水都抵挡不住有德者的光辉;无德者,即便占据天下最好的风水,也不能发挥作用,也不能长久。要怀着一颗感恩的心,思人恩德,想人好处,做到心善、行善、语善,保持一颗真诚的心。\textit{我讲的运气学,核心就是心念,一切运气由内在的心念而起,不在外部。所以,从根本上看,我们要内求,不要外求。(秦东魁)} 你的心念正了,做起事情来所携带的正能量是非常强大的,你有一种底气和自信,由这种自信会产生一种勇气和魄力,做事情也容易获得支持,事业就容易成功。

\begin{quote}\it
    科第中人,类有福相,余福薄,又不能积功累行,以基厚福;兼不耐烦剧,不能容人;时或以才智盖人,直心直行,轻言妄谈。凡此皆薄福之相也,岂宜科第哉。
\end{quote}

\textbf{解析} “余”指的就是我,文言文的用法。这是了凡先生自己的反省,中科举的人大多有福相,福气很薄,又不能积累功德来培养厚福,而且很没有耐心,不能容纳别人。时常用自己的才智来欺压别人,直心直行,说话很随意,说了很多妄言。像这样福气浅薄之人,怎么会中科举呢?这段反省很中肯,福相来自于行善积德,薄福之人士难以考取功名,因为他那个福德和位子是不匹配的,就像猴子坐在王位上,会显得很别扭,也不会长久,根本坐不住,王位是给雄狮坐的,雄狮象征德行和威严。古人很重视行善积德,很有行善的意识,特别注重德行的修养和提升,这方面我们要多学习古人的智慧,古圣先贤的确说出了真理,道出了规律。

\begin{quote}\it
    务要积德,务要包荒,务要和爱,务要惜精神。从前种种,譬如昨日死;从后种种,譬如今日生,此义理再生之身。
\end{quote}

\textbf{解析} 包荒,谓度量宽大的意思,也指原谅、宽容。一定要积德,一定要宽恕人家原谅人家,一定要和爱,一定要爱惜精神。宽恕他人,体现了自己的仁厚,人非圣贤孰能无过,水至清则无鱼,人至察则无徒,明白这个道理,培养那颗宽恕的心,会给你带来更大的福报,不宽恕别人就会导致狭隘和怨恨,这会产生很多负能量,对自己很不利。从前种种,就像死掉一样,过去了,从今以后,要重新做人,好好培养和提升自己的德行,人要往前看,要过正能量的人生。

\begin{quote}\it
    汝今扩充德性,力行善事,多积阴德,此自己所作之福也,安得而不受享乎?
\end{quote}

\textbf{解析} 这是云谷禅师在指点了凡先生了,关键就是提升德行,多做善事,多积累阴德。\textit{救人之难,济人之急,悯人之孤,容人之过。广行阴骘,上格苍穹。人能如我存心,天必赐汝以福。……近报则在自己,远报则在儿孙。百福骈臻,千祥云集,岂不从阴骘中得来者哉?(《文昌帝君阴骘文》)} 古圣先贤其实一直在强调改造命运的方法,几千年来核心主题一直未变,就是强调改过迁善,去除恶念,多发善念,多做善事,建立正能量的气场。\textit{积金以遗子孙,子孙未必守;积书以遗子孙,子孙未必读。不如积阴德于冥冥之间,为子孙长久之计。(北宋《司马温公家训》)} 阴德就如播种,只要播土下种,将来迟早会有收成,还会惠及子孙后代。

\begin{quote}\it
    誓行善事三千条,以报天地祖宗之德。
\end{quote}

\textbf{解析} 这是了凡先生发的大愿,这个大愿我是极为佩服的,我们看古人发的愿,就可以发现自己跟他们存心的差距。很多人可能还停留在比较自私的层面,所以发不出大愿,而古人已经给我们做出了榜样,我们要好好学习,最好也能发这样的大愿。刚开始发大愿可能还不太习惯,但是随着每天一遍遍发,这种强大的心念自然会产生一种莫大的力量,推动你去实现这个大愿,也会给你带来异常崇高的内心感受,这是我的亲身体会,当你有了济世度人的宏愿,当你发出这种性质的念头时,脸上的表情和眼神立马就会崇高起来,振动频率一下子就上来了,所以要多发大愿,要恒久力行来践行大愿。

\begin{quote}\it
    云谷出功过格示余,令所行之事,逐日登记;善则记数,恶则退除,且教持准提咒,以期必验。
\end{quote}

\textbf{解析} 功过格是很好的记录善恶的方式,能够让人时时警觉,以防自己造作恶业,每天晚上总结一下,也可以促进反省,不断完善。刚开始可能恶比较多,不一定要做恶事才叫恶,意恶也是恶,起了恶念也要记录,也要反省。每天尽量多做善事,比如每天帮助戒友答疑十个问题,这也算善行,或者宣传戒色一次,也算。生活中也可以做很多善事,捡垃圾扔垃圾桶,这种小善也可以多做,善事虽小,贵在坚持,一天做三件小善,一年就一千多件善事,真的是积小善,成大善,要注重积累,要坚持做下去,养成习惯,习惯成自然。

\begin{quote}\it
    此有秘传,只是不动念也。
\end{quote}

\textbf{解析} 《了凡四训》里最秘密的部分就是这句话,说的是画符,“执笔书符,先把万缘放下,一尘不起。从此念头不动处,下一点,谓之混沌开基。由此而一笔挥成,更无思虑,此符便灵。凡祈天立命,都要从无思无虑处感格。”虽然说的是画符,其实是叫你安住纯粹的觉知,也就是真我。做事前有意识地安住一会,然后再做,这样做事的品质和质量就会出奇地高,那些著名的画家和音乐家都在有意无意这样做,他们在创作前都会安静一会,定一定,然后再开始创作,创作的水平和质量非常之高。这的确是一个秘密,读《了凡四训》不能忽视这个秘密。

\begin{quote}\it
    余置空格一册,名曰治心篇。晨起坐堂,家人携付门役,置案上,所行善恶,纤悉必记。
\end{quote}

\textbf{解析} 治心篇,这三个字很好,为什么不说治身篇?因为心才是根本,\textit{三业之中,意业极重,凡一切善恶,俱起于意根,起念正则为十善,起念邪则为十恶。所以端正其心,以为根本。(虚云法师)} 了凡先生深知修行要从起心动念处下手,从心上改。“所行善恶,纤悉必记。”对自己严格要求,一个念头不正都要记下来,一念之微必使俯仰无愧,古人对自己颇为严格,所以达到的境界也不可思议。\textit{古人治心,防于念之初生、情之未起,所以用力甚微而收功甚巨也。(康熙教子《庭训格言》)} 邪念之初生,就要立刻对治,立刻断除,此时邪念尚未起势,所以用力甚微,但效果却很显著。\textit{人惟一心,起为念虑。念虑之正与不正,只在顷刻之间。若一念不正,顷刻而知之,即从而正之,自不至离道之远。(《庭训格言》)} 知之,即觉察之,觉察的刹那,邪念就化除了,心生邪念而不知,必成大患!\textit{凡人存善念,天必绥之福禄以善报之。今人日持珠敬佛,欲行善之故也。苟恶念不除,即持念珠,何益?(《庭训格言》)} 这段说到点子上了,修行不在外在的形式上,关键还是要断除恶念,这是最根本的。你天天手持念珠礼佛,但如果内心还是没有断除恶念,那是不会有效果的。

\subsubsection{第二篇:改过之法}

\begin{quote}\it
    今欲获福而远祸,未论行善,先须改过。
\end{quote}

\textbf{解析} 改过就像把漏洞堵住了,这样才能完成真正高质量的积累,否则即使做了很多善事,也会因为未改过而变得不纯粹,纯粹两字很难得,要做到纯善也很不容易,毕竟人有各种不良习气,只能自己下决心一点点去改,慢慢就能趋于正轨,臻于纯善。\textit{过而不能知,是不智也;知而不能改,是不勇也。(李觏《易论第九》)} \textit{不贵于无过,而贵于能改过。(王阳明先生)} 要改过,就要好好下决心,要痛下一番决心来改过。

\begin{quote}\it
    但改过者,第一,要发耻心。思古之圣贤,与我同为丈夫,彼何以百世可师?我何以一身瓦裂?第二,要发畏心。天地在上,鬼神难欺,吾虽过在隐微,而天地鬼神,实鉴临之,重则降之百殃,轻则损其现福,吾何可以不惧?第三,须发勇心。人不改过,多是因循退缩,吾须奋然振作,不用迟疑,不烦等待。小者如芒刺在肉,速与抉剔;大者如毒蛇啮指,速与斩除,无丝毫凝滞,此风雷之所以为益也。
\end{quote}

\textbf{解析} 知耻近乎勇,要有耻心,邪淫的状态完全是一种无耻的状态,满脑子邪思邪见,我们要发耻心,要学习圣贤,效仿圣贤,与其一身瓦裂不如完善自己的德行,改正自己的缺点与错误,人生是一场修行,要不断完善和提升自己。发敬畏心也很重要,存敬畏心和恭敬心,邪念就会减少很多,这里的畏惧,是因为知道过错的可怕,菩萨畏因,众生畏果,等到果来了你哭闹也没有用了,因为果已经成熟了,来不及了。怕果,你因就不要造。众生常作恶因,欲免恶果,譬如当日避影,徒劳奔驰。对于造因一定要很谨慎,一定要造善因,绝对不要造恶因,起心动念、言语造作,绝不去造恶业,自然没恶报。《了凡四训》是有谈及鬼神的,这类内容容易被误解,可能会觉得是迷信,其实鬼神是不同维次空间的存在,所以还是应该敬畏鬼神。我以前也不大相信,后来看了很多濒死的案例和书籍,就完全相信了,有的人甚至还记得前世,人死后脱离肉体会有中阴身,很多大德都提到过的。第三就是发勇猛心,要勇猛精进,势不可挡,摧枯拉朽,如猛虎出笼!人不肯改过,往往缺少一种勇猛的架势,一种你死我活的斗志,一种破釜沉舟、豁出命来拼死一搏的决心与勇气。我之前说疯狂学习戒色文章和疯狂练习断念,就是为了激发大家进入一种神勇无比的状态,这并不是叫你失去理智,而是叫你拿出最勇猛最精进的状态,大家都知道林书豪曾经的“林疯狂”表现,在逆境中奋然振作,强势崛起,我看了林书豪的纪录片,他去救济穷人,给流浪汉送食物送红包,真的很有善心,另外林书豪训练格外刻苦,非常疯狂,场上的疯狂表现是用场下的疯狂训练换来的,库里也是疯狂训练,苏炳添也是疯狂训练,这里的疯狂是一种勇猛精进的状态,要雷厉风行,全力以赴!就像百米赛跑选手们拿出最疯狂的状态来完成这一段跑道。曾国藩没有超人的天赋,也没有优越的背景,但却能够获得惊人的成就,这与他时时调动刚猛精神密不可分,总是在强调“凡事须下血战功夫”。做任何事都要有“勇”“刚”的一面,果断坚决,但也要兼具恒心,猛火煮和慢火炖要交替进行,要经常激励自己,鞭策自己,剔起眉毛,振作精神,勇往直前!不要再因循退缩了,不要再找借口了,到了决一死战的时刻,不能再懦弱了,也不能再逃避和拖延了,这时你要拔出战刀,勇猛动作,上阵杀敌!血战到底!杀他个片甲不留!所有的邪念,所有的负面念头与习气,统统杀光!杀净!一个不留!!!

\begin{quote}\it
    一息尚存,弥天之恶,犹可悔改,古人有一生作恶,临死悔悟,发一善念,遂得善终者。谓一念猛厉,足以涤百年之恶也。譬如千年幽谷,一灯才照,则千年之暗俱除,故过不论久近,惟以改为贵。但尘世无常,肉身易殒,一息不属,欲改无由矣。明则千百年担负恶名,虽孝子慈孙,不能洗涤;幽则千百劫沉沦狱报,虽圣贤佛菩萨,不能援引。
\end{quote}

\textbf{解析} 要悔改,要忏悔,而且要趁早,悔改也要猛厉,要下大决心。临死的时候其实已经有点晚了,还是要趁身体尚可时积极改过,积极行善,积极改造自己的命运。“幽则千百劫沉沦狱报”,这是讲地狱的,虽然一笔带过,但也足够警醒,这类内容善根深厚者方能接受。千百年来,关于地狱的开示一直都有,佛经中有,大德也讲过,传统文化的其他书籍也有记载,其他正统宗教也有讲到,这类信息是容易被误解的,即使暂时不能接受,也应该保持敬畏之心,认真改过。

\begin{quote}\it
    何谓从心而改?过有千端,惟心所造;吾心不动,过安从生?学者于好色,好名,好货,好怒,种种诸过,不必逐类寻求;但当一心为善,正念现前,邪念自然污染不上。如太阳当空,魍魉潜消,此精一之真传也。
\end{quote}

\textbf{解析} 根本还在于心,这段讲得非常明白,“一心为善,正念现前”,这八个字要认真体会和理解,有的人可能会理解成只要行善,邪念就污染不上,其实不然,“正念现前”才是关键所在,正念这个概念最初源于佛教禅修,是从坐禅、冥想、参悟等发展而来,是有意识地觉察当下的一切,正念有觉察之意,就是叫你保持警惕,时时观心,看住自己的念头,这点至为关键。光行善是不够的,因为邪念会自动冒出,邪念会入侵你的头脑,你必须具备对治之力,要立刻断除而不能被其附体。

\begin{quote}\it
    大抵最上治心,当下清净;才动即觉,觉之即无;苟未能然,须明理以遣之。
\end{quote}

\textbf{解析} 这段话就是断念口诀的意思,这个修心诀无数的大德都提到过,当然不一定是十六个字,也可能是后八个字,或者其他一段话,但含义是一样,那就是通过觉察来消灭念头,原理就是念起即觉,觉之即无。真正的断念并非压念,而是觉而化之,压念是有主观压制的想法,不想让念头起来,而断念则是不怕念起,而是要及时觉察。压念会带来挫败感,因为越压越反弹,搞得自己很苦恼,而断念则很轻松,一觉就化除了。持咒念佛是很好,断念口诀也很殊胜,思维对治也不错,这几种断念方式都是很好的,各有特点,各有千秋,持咒念佛有他力加持,但自己也要勤于用功才能达到一定的境界;断念口诀是自力,需要自己不断练习来提升觉察力,断念口诀最终是离开念头的,不需要起念,只是觉察。这种观心的功夫其实是基本功,\textit{唯观心一法,总摄诸法,最为省要。(达摩祖师)} \textit{三界之中以心为主,能观心者究竟解脱,不能观者究竟沉沦。(《大乘本生心地观经》)} 用念佛来对治邪念,需要先发现邪念来了,然后赶紧念佛,必须以观心作为基础,做念佛日课时,发现妄念起来了,马上专注在佛号上,也需要观心来发现妄念。观心是最基础的基本功,也是最高深、最终极的修心功夫,觉察力足够强大,一发现念头,念头就消失了,所以《了凡四训》把修心诀称为“最上治心”。当然,我是很尊重持咒念佛的,的确很殊胜,这几种断念方式都很好,我们应该尊重每一种断念的方法,不可贬低任何一种。我看过很多开示,高僧大德从来没有贬低过修心诀,而是极力提倡修心诀,很多高深的法门只教导观心,是可以获得最终成就的。

有的人虽然练习断念口诀,但还是失败了,原因有两方面,一,没有正确理解这个口诀;二,没有练到较高的水平。正确理解和坚持练习,这两者缺一不可,一定要有正确的理解,彻底弄懂断念口诀的原理与含义,不少人会误解断念为压念,想压制念头,不让它起来,这是完全错误的。按照压念去练习,只会南辕北辙,断念非压念,断念是化念,通过觉察来化除念头。真正弄懂原理后,就要勤于练习,也可以在练习的过程中加深理解,理解到位了,练起来也会事半功倍,练到一定水平即可降伏心魔。就像打游戏练级一样,要练到一定级别才有望打败 BOSS,这个过程需要坚持,不是一蹴而就的,打游戏可以花钱找代练,但是断念口诀只能自己勤于练习,刚开始要熟背,最后就不用背了,直接觉察即可,平时保持观心,保持警惕,看住起心动念。随着坚持练习,断力和断速都会有明显的提升,到时就可以主宰内心了。

\subsubsection{第三篇:积善之方}

\begin{quote}\it
    善日加修,德日加厚。
\end{quote}

\textbf{解析} 古人有积善意识,不断积累善行,不断提升德行,今人则提倡理财意识,对钱财看得比较重,而古人更看重的是善和德。修为好、心地清净、性情善良的人,头顶有灵光闪耀,整个人看上去也是光明开朗,神采奕奕,而作恶多端的人头顶灵光全无,甚至有一团黑气笼罩,显得灰头土脸,萎靡不振。对光明者,大家都仰慕恭敬;对黑暗者,众人都会厌恶乃至鄙视。从善如登,从恶如崩,做善事如登山,需要下很大决心来坚持,而向坏发展就像山崩一样迅速,学好很难,学坏极容易,很多人也是被邪友带坏的,交友一定要谨慎,发现对方有恶习,要尽量规劝,如果不听,那就远离。\textit{福以德基,非可祈也,祸以恶积,非可禳也,苟能为善,虽不祭,神亦助之。(《草堂集》)} 福在积善,祸在积恶,福之将至,观其善而必先知之矣;祸之将至,观其不善而必先知之矣。

\begin{quote}\it
    凡为善而人知之,则为阳善;为善而人不知,则为阴德。阴德,天报之;阳善,享世名。
\end{quote}

\textbf{解析} 善也是分类的,做善事不留名,不夸耀,非常低调,就像没做过一样,这类阴德的福报大,古人提倡多积阴德,这样福报才大,当然也不可执著于福报,只管耕耘即可。\textit{为善最乐,是不求人知。为恶最苦,是惟恐人知。(曾国藩)} 做了善事,内心会很快乐,也不需要别人知道,默默去做即可,如果做了善事,希望别人知道,希望别人夸奖自己,希望通过行善来获得好的名誉,这类人就比较浅薄,他们行善是和名利挂钩的,私心很重。古人极重阴德,阴德更显无私和低调,不求任何回报,不求名利,只是无私帮助别人,无私为社会做贡献,这种人境界就高。真正无私了才具有大力量,无私代表纯粹,纯粹的力量不可思议,真正的力量来自于纯粹!要做纯粹的人,达到纯粹的境界!

\begin{quote}\it
    为善而心不着善,则随所成就,皆得圆满。心着于善,虽终身勤励,止于半善而已。譬如以财济人,内不见己,外不见人,中不见所施之物,是谓三轮体空,是谓一心清净,则斗粟可以种无涯之福,一文可以消千劫之罪,倘此心未忘,虽黄金万镒,福不满也。
\end{quote}

\textbf{解析} 这条就是在讲执著与否,不执著就做圆满了,心里一直想着福报,那就大打折扣了。做善事要三轮体空,内心保持清净,这样产生的力道就会翻倍增长,一有杂念,一有执著,那力道就会大为削减。做善事也是很有讲究的,要做圆满,就要尽量做到三轮体空,即使做过一万件善事,也像什么也没做过一样,内心十分清净,不去想做了多少善事,也不去求回报,没有任何希求,心中空空如也。达此境界的人,即使做了一件很小的善事,也会产生巨大的力量。犟牛居士说过一个例子,说是一个人听说布施好,于是就布施了一千元,然后他就老想着得回报,结果好几年都未能如愿,他那种贪财的私心实在太重,布施的时候是带着贪心布施的,这样就会大打折扣,他那个发心就不纯,善行的效果自然很差,本来是善行,他却夹杂着贪心,贪嗔痴是三毒,他这样发心行善,怎能有满意的结果呢?你真正什么也不求,最后却不求自得,你真正无私了,那个善行的质量就非常之高,一丝一毫的回报都不求,只是纯粹地行善,要做到这种程度。哪怕求一丝一毫,那就是自私,自私的力量就很小,无私的力量才最大,效果也最好,差别就在这。

\begin{quote}\it
    随缘济众,其类至繁,约言其纲,大约有十:第一,与人为善;第二,爱敬存心;第三,成人之美;第四,劝人为善;第五,救人危急;第六,兴建大利;第七,舍财作福;第八,护持正法;第九,敬重尊长;第十,爱惜物命。
\end{quote}

\textbf{解析} 这十条是非常重要的,能做到这十条就非常有德行了。我们要长养德行,时时秉持着恻隐之心、爱敬之心、与人为善之心、成人之美之心,多做善事,多帮助别人,给别人带去正能量,用正能量去影响别人,走到哪里,都能自带善意的光芒,照亮身边的人。当你发出正能量的念头时,别人也容易发出正能量的念头,当你发出自私的念头时,别人也容易发出自私的念头,人在交流沟通时是很容易受到对方影响的,我们一定要多发善念,用高频的正能量去影响和带动别人。和高频的人在一起,你的频率就会自动被调到那个频率,内心会变得非常清净,能够以一个更高的视角、更高的智慧去看待问题和处理事情。我们要让自己成为高频的人,去影响周围的人,低频的人充满了负能量,他们也在影响周围的人,很容易导致争吵等不和谐的现象。很多人都反馈在邪淫后家庭容易争吵,邪淫者很容易和别人吵架,戾气很重,因为经常起邪念,散发出来的能量是低频的,具有攻击性,易于吵架。高频的人懂得断除负面念头,懂得行善和培养正能量,内心祥和而稳定,这样才能真正把握自己的人生与命运。

\subsubsection{第四篇:谦德之效}

\begin{quote}\it
    易曰:“天道亏盈而益谦;地道变盈而流谦;鬼神害盈而福谦;人道恶盈而好谦。”是故谦之一卦,六爻皆吉。
\end{quote}

\textbf{解析} 戒色十规首重“谦德”,谦,六爻皆吉。\textit{谦者,众善之基;傲者,众恶之魁。(王阳明《传习录》)} 保持谦逊,万事亨通,君子善终,大吉!谦对自己是非常有利的,傲对自己是非常有害的,傲慢其实助长了一个人的小我,而谦虚、谦卑则是真我的一种宝贵品质。清雍正年间,江水被推荐到朝廷做官。皇上召见时,他紧张哆嗦,不能对答,于是推荐他的学生戴震。戴震口若悬河,分析问题切中要害,说得清清楚楚。皇上大为兴奋。问戴震说:“你和老师比,谁的才能高?”戴震回答:“我的水平低。”皇上又问:“那水平高的反而不能回答,为什么?”戴震说:“老师年老,耳朵有些背,可他的学问,超过我一万倍。”皇上赞赏他的谦让精神,赐为翰林。戴震一方面尊师,另外就是非常谦虚,这是合天心的品德,所以吉祥,做事做人都要上合天心,这样才能大吉。

最近看了一篇职场的文章,里面讲到:“无论身在什么岗位,想得到领导信任,就要注意:其一,如果取得成绩,第一时间把功劳归给领导,因为没有领导的授权与支持,就不会有你的成绩;其二,越是取得成绩,在同事和领导面前要更加谦逊,千万不能得意忘形,更不能在领导面前炫耀自己的成绩和功劳,以免同事嫉妒和领导反感。总之,做人要有一颗感恩的心,取得成绩后,要不骄不躁,将成就与荣耀分享给同事和领导,这样方能得到大家的拥戴,得到领导的重用。”这段话讲得很好,真正的智者不居功,不自傲,取得一定成绩,也懂得让功,让功是非常深的智慧、非常高的德行,让功于众,让功于师,让功于领导,让功于圣贤,让功是君子所为,高风亮节,胸襟博大,堪为楷模。\textit{为而不恃,功成而弗居。(《道德经》)} 君子不居功,而是让功,小人往往会争功、贪功,品行一下就拉开了差距,因为不居功和让功,就能得到众人的支持,如果一味争功、贪功,只会引来众人的反感和反对。春秋时期,曾有过一段“三将让功”的佳话。据《通监记事本末》记载,晋国将领郤克、士燮、栾书这三位将领,在取得晋齐鞍之战胜利后,不争功,克己互让,谦虚谨慎,表现出良好的将德风范,一直为后世兵家所推崇。作为良将不但要有将才,精于谋略,善于战法,同时还必须具备良好的将德修养,胜而不骄,谦退不伐(不伐:不自我夸耀),不矜其功。晋军三将克己谦让的行为,堪称良将的楷模。

\begin{quote}\it
    书曰:“满招损,谦受益。”予屡同诸公应试,每见寒士将达,必有一段谦光可掬。
\end{quote}

\textbf{解析} 古人很懂谦虚的道理,今人很多人都不懂这个道理,谦虚会获得支持,一旦骄傲,就会引来敌意和攻击。大家应该都有这样的体会,就是某某人一旦暴露出了骄傲或者傲慢的想法,你心里就会觉得不舒服,甚至想攻击对方,觉得看对方不顺眼。如果对方很谦虚,不居功,不自傲,懂得恭敬礼让,这样一段谦光可掬,别人心里就会很舒服,这也是获得成就的征兆。如果一个人很骄傲很自满,这就是要受到打击的征兆,我看过很多事例,就是当事人发出了骄傲或者傲慢的念头,结果就引来了攻击他的言论,如果他能谦虚,就能获得众人的支持,一骄傲,一自满,目中无人,狂妄自大,出现这种表现,就要受到打击了。

\begin{quote}\it
    予曰:“惟谦受福。兄看十人中,有恂恂款款,不敢先人,如敬宇者乎?有恭敬顺承,小心谦畏,如敬宇者乎?有受侮不答,闻谤不辩,如敬宇者乎?人能如此,即天地鬼神,犹将佑之,岂有不发者?”
\end{quote}

\textbf{解析} 惟谦受福这四个字讲得非常好,地低成海,人低成王,圣者无名,大者无形。能谦虚,能低调,能恭敬,能敬畏,能做到这四点,就是能受福的表现。看人要观察他的德,他的举止,他的精气神,如果一个人的“神”平和端庄,很安定,表明他道德高尚,不会因周遭事物的变化而随意改变节操和信仰,敢于坚持正确的东西,意志很坚定。安静时,两眼光华熠熠,目光湛然清明,行动时,两眼安详沉稳,又敏锐犀利,这是清正的表现。\textit{才德全尽,谓之圣人;才德兼亡,谓之愚人;德胜才,谓之君子,才胜德,谓之小人。(北宋司马光)} 总而言之,德是最重要的,以德立人,以德立命,人无德不立,国无德不兴。曾国藩善用人,源于善相人,曾国藩对人才任用的成功正是得益于他对人才的分类观察,考量的标准就是品德端正,仪容端庄。

\begin{quote}\it
    道者曰:“造命者天,立命者我;力行善事,广积阴德,何福不可求哉?”
\end{quote}

\textbf{解析} 立命者我,真正立命的是自己,自己是可以改造命运的,不能消极被动,要力行善事,广积阴德,你的能量场变了,你的处境自然就变了。人具有主观能动性,完全可以通过改过迁善来改造自己的命运,怨天尤人的戒友我看到过不少,当他们学习了《了凡四训》就知道问题其实是出在自己身上,不应该怨天尤人,不能抱怨老天不公平,其实老天对每一个人都是公平的,关键是自己起了什么念,做了什么事,自然会得到相应的一个果报。明白改造命运的道理后,就要断除恶念,力行善事,坚持做下去,积小善成大善,日积月累,越积越厚,就像每天一张薄薄的面纸,坚持积累下去,就是厚厚的一叠,所以要注重积累,要坚持下去。

\begin{quote}\it
    稍有识见之士,必不忍自狭其量,而自拒其福也,况谦则受教有地,而取善无穷,尤修业者所必不可少者也。
\end{quote}

\textbf{解析} 谦则受教这四个字很重要,谦虚了,才能更好地接受圣贤的教诲,如果以傲慢的心态学习圣贤教育,甚至会觉得圣贤不如自己,从而产生毁谤之心,觉得自己了不起,其实浅薄得很。曾国藩书房题名“求阙斋”,别人求圆满无缺,曾国藩却求“阙”。阙者缺也,就是求欠缺用以提醒自己不能盈满,因为他地位愈高愈懂得谦恭的重要,所以能够保住自身的善始善终。善事要尽量做圆满,而做人忌满,有些事也不能做过头,所谓物极必反,做人的境界是大成若“缺”,抱朴守拙,谦卑低调。这个道理我们一定要懂!

\begin{quote}\it
    人之有志,如树之有根,立定此志,须念念谦虚,尘尘方便,自然感动天地,而造福由我。
\end{quote}

\textbf{解析} 真正有德行、有诚心的人是可以感动天地的,这点我是深信不疑的。人一旦立定志向,就要注重谦德,这样做事就容易成功,谦虚的人很容易得到别人的支持。与人方便,多帮助别人,这样就能得到拥护,这是德行的感召,自然能感动天地。造福由我,造恶也是由我,我们都有自由意志,既可以造善造福也可以造恶造罪,一条是光明大道,一条是地狱险途,全在我们自己。有觉悟的人,他深谙恶行导致恶果,所以他不去造恶,而是有意识地多行善积德,完善自己的德行,不断积累自己的福报,过正能量的人生,这才是真正有智慧的表现。

\paragraph*{总结}

这季和大家分享了《了凡四训》的研习体会,大家也可以看看《了凡四训》的电影,那部电影拍得很不错,很有古风的味道,秦东魁老师也讲过《了凡四训》,大家有空可以看看,喜马拉雅 FM 也有《了凡四训》有声书,平时听听也是很不错的,还有《了凡四训》的漫画,画风也很有意境。曾国藩对《了凡四训》最为推崇,读后改号涤生,“涤者,取涤其旧染之污也;生者,取明袁了凡之言:‘从前种种,譬如昨日死;从后种种,譬如今日生也。’”他奋然振作,精勤砥砺,终成晚清中兴第一名臣,将《了凡四训》列为子侄必读的第一本人生智慧之书。《了凡四训》影响了一代又一代的中国人,被称为东方第一励志宝典,了凡先生结合自己亲身经历和毕生学问修养来做的家训,加以现身说法,颇具说服力,也相当励志,这本书真的启发了无数的人,改变了他们的命运。决定一个人富贵贫贱的主要因素,不是风水、不是命数,而是一个人的“心田”,即你的思想、你的品德、你的行为,你的人生剧本是你自己撰写的,你输入负面的念头,放出来的影像就是倒霉的,你输入正面的念头,放出来的影像就是吉祥的。心是根本,大德一直在强调修心,境随心转,把心修好、修正,学会行善积德,积累正能量,到时你的命运就能完成逆袭。

下面分享三首诗歌。

\begin{poem}[空]
    \begin{multicols}{5}
        \centering~\\
        人站着 \\ 脑 \\ 空了\\~\\
        人坐着 \\ 脑 \\ 空了\\~\\
        念来了 \\ 念走了 \\ 脑空了 \\~\\
        念是过客 \\ 空一直在那里 \\ 我不是念 \\ 我是空\\~\\
        空 \\ 即纯粹的觉知 \\ 即真我
    \end{multicols}
\end{poem}

\begin{poem}[斩]
    \begin{multicols}{4}
        \centering~\\
        念头葛藤 \\ 缠绕上来了 \\ 越缠越多 \\ 越缠越密 \\ 唰! \\ 觉察的快刀 \\ 斩下 \\ 葛藤掉落一地 \\ 脑瞬间空了 \\ 那个惊人的空 \\ 那个平淡的空 \\ 那个不可思议的空 \\ 整个宇宙 \\ 都倒进 \\ 这个空里
    \end{multicols}
\end{poem}

\begin{poem}[笑]
    \begin{multicols}{4}
        \centering~\\
        人在走,车在开 \\ 耳边传来各种嘈杂声 \\ 我闭上双眼 \\ 站在那里 \\ 一动不动 \\ 我空了 \\ 不去追求任何事物 \\ 只是享受空的感觉 \\ 这个空的感觉 \\ 就是美,就是爱 \\ 就是喜悦,就是神圣 \\ 我笑了 \\ 笑得像一个纯真无邪的孩子 \\ 仿佛发现了天大的秘密
    \end{multicols}
\end{poem}

下面推荐一本书。

\begin{book}[《深夜加油站遇见苏格拉底》,丹·米尔曼]
    这是一本改变生命的书!作者是一个非常会说故事的人,在这本半自传体小说中,活灵活现地创造出三个让人一看就难忘的角色,一是代表作者本人,也同时代表无数对世界迷惘的年轻心灵,渴望了解存在的终极意义;二是那位神秘智慧的老人——深夜加油站的老工人,也就是作者昵称的苏格拉底,他视之为上师;第三位人物是古怪精灵的女友乔依,代表着作者失落了一半的灵魂伴侣。阅读《深夜加油站遇见苏格拉底》,有一种被加持、被点化的感受,尤其当书中的加油站老工人苏格拉底提出独特的反问时,读者会跟着叙述走,一起思考,一起展开对心灵的神秘探索。这本书 20 世纪 80 年代就已出版,2006 年被拍成电影。考上知名学府、获得世界冠军荣誉的丹在短暂的欢乐过后,深深陷入无可名状的空虚和恐慌,无法入眠的他在一家 24 小时加油站遇见一位被他称作苏格拉底的老人,后来又认识了精灵般难以捉摸的女孩乔伊。苏格拉底带他体验或浪漫、或惊险、或恐怖、或温馨的旅程,帮助他从车祸的打击中恢复健康,以捉弄、嘲讽、关爱、抚慰、激励等令人匪夷所思但恍然大悟的种种方式指导他修炼,最终指引他打开和平勇士之道的大门。这本书的名字很有穿越感,不过并不是苏格拉底穿越到了现代,而是作者对一位神秘老人的昵称,苏格拉底象征着智慧,刚开始以为这是一本励志的书籍,后来才发现这本书不仅仅是励志,更是一本修道的书籍,讲述了修行方面的真理。里面很多句子我都很喜欢,比如:

    \begin{quotation}
        苏格拉底替人加的不只是汽油,也许还包括那种光辉、那股能量或情感。

        始作俑者,正是人们的心智。

        所有的世人都被困在自己的心智所造成的洞穴中,无法自拔。

        你唯一的投资是修炼。

        看穿你的心智。

        就像决斗的武士,不是顿悟就是死。

        觉察力突破了心智的障碍。

        觉察力大幅度跃进并不会一下就发生,而是需要时间和修炼。

        觉察力是勇士的宝剑。

        征服心智。

        剑已磨利。
    \end{quotation}

    这些语句读来就很有加持力,心智代表的就是念头,就是小我。通过练习观心,练习觉察,来征服心智,觉察力就是念头的克星!觉察力就是勇士的宝剑!可以斩断心智带来的束缚,我很喜欢这本书直接而有力的表述,说的其实就是断念!很直接,很干脆!就是把觉察力这把剑磨利,征服自己的心智。《深夜加油站遇见苏格拉底》是一本改变无数生命的心灵圣经,畅销数百万,讲的其实就是修心之道,只不过是通过一个小说的形式来表述出来,这样更具有吸引力。这个世界有无数的迷惘的年轻心灵,他们急需智者的指引和点化,这本书的故事模式和《了凡四训》有点像,都是遇见智者,遇见高人,获得了指引和点化,从而逆袭了自己的人生。将来我想分享一下这本书的笔记和解析,敬请大家期待,有兴趣的戒友可以先阅读一下这本书。
\end{book}
