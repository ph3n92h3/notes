\subsection{我的学佛体会}

这篇文章是写给想信佛有佛缘的戒友看的,希望我的学佛体会能带给大家有益的启示。

我从小和大家一样,可能对佛教的最初认识是来源于影视作品,比如西游记或者一些武侠电影。那时对佛教并没有特别的感觉,遁入空门这个词对我来说有些不可理解,一般认为是受了刺激或者逃避世俗才会选择出家。小时候对于佛教的认识很肤浅,也缺少正确的引导,只能从影视作品获得很模糊的认识。记得小时候看西游记,孙悟空一个跟头十万八千里,但他却翻不出如来的手掌,那时幼小的心灵就觉得如来很厉害,是神一样的人物,而且属于顶级的那种。当然,这只是小学生的认识,很单纯也很幼稚。

后来,在我做健身教练时,有一个会籍顾问信佛了,衣服穿得很破烂,然后手里拿着一本佛教小册子,逢人就谈佛教。他的转变让我感到惊讶,因为之前我有和他一起打篮球,也算有点交情,我那时觉得他信佛前很正常,然后信佛后一下不正常了,那时我感到不可思议,也不可理解,心里想为什么要信佛,信佛有什么好处吗?那时我还是无神论者,说中国年轻一代是无神论者,是对的,因为我们从小就按照无神论来教育的,不过,说中国人都信佛,其实也是对的,因为祖辈和父辈多多少少会信佛,当然大部分只是求财求平安,并没有多少正知正见。相信不少戒友的奶奶或者母亲都有信佛的倾向,这样潜移默化,多多少少会影响到下一代。我奶奶也信佛,我小时候就跟着磕过头,当然奶奶从来没跟我谈过佛理,我那时就觉得佛教有迷信色彩,和我的教育背景相冲突,但我又不是坚定的无神论者,只是抱着可信可不信的骑墙态度。

再后来,我得了焦虑症,过着生不如死、暗无天日的灰暗生活。跑过无数次医院,做过非常多的检查,吃过无数的药,就是不见好,那时我活在绝望里,活在惶恐中,曾经出现过自杀的想法,还好没有实施。但也正是焦虑症,彻底点醒了我,否则,可能到死我都不会醒悟过来。那时,我对医院和吃药失望后,就开始自己钻研养生功法,才知道戒色保精实在太重要了,肾藏精,藏五脏六腑精华之气,肾为五脏之根,而撸管伤肾非常厉害,精少则病,真的是至理名言。

通过一段时间的养生,我觉得自己好很多了,病情在好转,我药也不吃了,因为基本都吃耐药了,而且还有副作用,不敢再吃了。我就是通过戒色养生一步步好起来的,当然好得很慢,过了一年多,焦虑症就基本无躯体症状了,但还是很虚,有一种大病初愈的感觉,继续坚持戒色养生,我的元气才真正养回来,之前我熬夜纵欲久坐,真的把自己透支太多,搞得元气大伤,焦虑症的爆发一下把我击垮了,但是祸兮福所倚,也正是焦虑症点化了我,点醒了我,我是以病入道的,病缘改变了我的人生,以病入道的案例实在很多,很多人得了病以后才开始反思自己的人生,才开始反思过去的做法是否正确。得病是大彻大悟的一个机缘,通过这个机缘可以看清很多过去完全忽视的东西。

我那时自己恢复差不多了,就开始在群里帮助病友了,和无数的病友聊过天,帮助他们认识病因,帮助他们认识戒色养生的重要性,很多病友在我的建议下,的确找到了痊愈之路,给了我很多感谢的反馈,我那时每天都在群里帮助大家,自己也搞过很多群。后来我发现,很多病友在得病后都开始信佛了,我当时对信佛没概念,但是也不反感,也许是善根成熟了,我自己也开始找佛教的内容看了。

我那时知道信佛对调心好,而身心是合一的,心对,身体也会跟着对,心错,身体迟早会出问题。佛教倡导的是和平宽容,倡导的是众善奉行,诸恶莫作,看到佛陀那慈祥的面容,我心里就自然升起无限的欢喜。我那时对养生已经有了一定的认识,知道行善是会提升人体的阳气的,所谓善则升阳,可以增加人体的正能量,我突然悟到,其实养生和佛法是完全相通的,这时我才发现佛法真的很不简单,佛教的内涵真的是博大精深。

有了这一层的思想认识,我就开始自己找佛教的内容进行自我熏陶。中国佛教出现过许多派别,主要有八宗。一是三论宗,二是瑜伽宗,三是天台宗,四是华严宗,五是禅宗,六是净土宗,七是律宗,八是密宗。现在基本就剩下禅宗、净土宗、密宗这三大宗派。我们学佛一般都从这三大宗派入手,千万不要去搞什么神通,神通其实我们每个人都有,修到最后自然会开发出来,如果专门搞神通,那就很容易着魔。

还有一些所谓的大师,自称是观音菩萨或者法力比释迦牟尼佛还高几十万倍,这不是佛说,这是魔说。要知道,佛菩萨再来是不会暴露自己身份的,如果他说自己是某某菩萨再来,那就会马上入灭或者装疯卖傻,就像济公一样。\textit{末法时代,邪师说法如恒河沙(《楞严经》)},我们一定要擦亮眼睛,学佛一定要正信,多去正规的佛教网站熏陶,多看大德高僧的开示,千万不要听信附佛外道的邪知邪见,那真的会误了我们的法身慧命。学佛的歧路还是很多的,很多人为了人天福报而学佛,或者求什么神通,或者求财求官,这都是学偏了,我们应该为度众生而学佛,为度众生而修道,这才是真佛子所该拥有的知见。

\begin{quote}\it
    娑婆世界是我们客居的地方,一切皆幻化不实,如戏梦一场,到头总是空,不要贪恋娑婆世界的一切,放下万缘,念佛求生西方,阿弥陀佛才是我们究竟归依处,是我们的故乡。(广钦老和尚)
\end{quote}

我发的第一个誓愿,就是看了广钦老和尚的行持语录,即:愿断一切恶,愿修一切善,愿度一切众生。这个誓愿,我现在每天还在发。我现在选择的就是净土法门,我也极力推荐广大有佛缘的戒友修持净土法门,印光法师和元音老人都说过,末法时代,众生障重慧浅,唯净土法门最当机,念佛一法,是如来普应群机而说,亦是阿弥陀佛大悲愿力所成就,无论上、中、下根,皆可修学。即使烦恼惑业未完全断除,但只要具足真信切愿,并且老实念佛求生西方,亦可蒙佛接引,带业往生,一得往生,生死就可了脱矣。

当然,禅宗和密宗也非常好。先说密宗,密宗的修法里面其实也包括往生极乐世界的修法。

索达吉堪布:又有人认为密宗与净土宗如同水火,互不相容,若选学一种则必须舍弃另一种。须知其实密宗中也同样包含了净土宗,在《法王如意宝文殊大圆满》、无垢光尊者的《四心滴》等众多大圆满法中,都有具体的往生极乐世界的修法。密宗中还有破瓦捷径往生法。如今法王如意宝晋美彭措显密并弘,平时一再强调弟子们应发愿往生西方极乐世界,并以自己发愿往生而为表率。法王指出修密法者必须同时兼修净土,修大圆满的人应在大圆满的基础上往生西方,既可确保往生,了生脱死,同时又可获得极高品位。并倡导每位弟子念满一百万阿弥陀佛名号及三十万阿弥陀佛心咒,对圆满这两个数目的弟子,法王以其加持力保证他们即生往生到西方极乐世界。

也有不少师兄选择了密宗,我觉得也很好,具体选择哪个宗派,就看你自己的缘分了。所谓:归元无二路,方便有多门。

再说禅宗,禅者,即吾人本具之真如佛性,宗门所谓父母未生以前本来面目。宗门语不说破,令人参而自得,即所谓参禅。末法时代,能亲证本来面目的人已经很少了,即使你大彻大悟,还是不能了生死,因为开悟后,惑业还在,还需悟后真修,一步步净除才行,如果有一丝惑业未尽,那还是无法了生死。

\textit{有禅有净土,犹如戴角虎,现世为人师,来生作佛祖。(永明延寿禅师)} 如果你彻悟了本来面目,再发真信切愿求生西方净土,那往生的品位就会很高,那真的就像戴角的老虎,没角的老虎已经很猛了,老虎再有角,那真是王中王了。禅宗靠自力了生死,难度颇大,所以禅宗一般接引上上根之人,对根器要求很高。末世众生,根器已经很差很差了,所以禅宗到后来基本都是法卷传法,并不是开悟之人传法。真正彻悟本来面目的人少之又少,即使大彻大悟,还需悟后真修,了生死的难度非常之大。所以,对于末世众生来说,最当机的就是净土法门,求生西方极乐世界才是明智之选。

我的推荐:

\begin{itemize}
    \item 《无量寿经》、《观无量寿经》、《阿弥陀经》,这是净土三经,大家都应该看看。
    \item 《观经四帖疏》、《善导大师的净土思想》、《印光大师文钞全集》,大家不要错过。
    \item 《净土圣贤录》、《往生传》、《善女人往生传》,看往生事迹可以增强往生的信心。
    \item 还有一个往生动画片也很好,叫《念佛成佛二十则》。
    \item 另外,黄念祖老居士和元音老人的开示也非常之好!不可错过,这两位居士是泰斗级的,是佛菩萨再来。
\end{itemize}

多看大德开示,获得正知正见正信,建立深厚的信心,然后发大愿,再加上老实念佛,这样往生西方极乐世界就有把握了。

我极力推荐广大有缘的戒友修持净土念佛法门!

最后,分享印光大师的一段话,和大家共勉!

\begin{quote}\it
    无论在家出家,必须上敬下和,忍人所不能忍,行人所不能行。代人之劳,成人之美。静坐常思已过,闲谈不论人非。行住坐卧,穿衣吃饭,从朝至暮,从暮至朝,一句佛号,不令间断。或小声念,或默念。除念佛外,不起别念。若或妄念一起当下就要叫它消灭。常生惭愧心及忏悔心。纵有修持,总觉我工夫很浅,不矜自夸。只管自家,不管人家;只看好样子,不看坏样子。看一切人都是菩萨,唯我一人实是凡夫。果能依我所说修行,决定可生西方极乐世界。(印光大师)
\end{quote}

后记:戒色恢复身心健康,有良好的状态奋斗自己的人生,这的确是戒色的好处,但毕竟不究竟,戒色应该有更高的追求。人生是无常的,2018 年走了很多名人,引来唏嘘一片,某著名主持人年仅 50 岁就因病去世,他的死讯震惊了世人,真的是三界无安,犹如火宅,一会车祸死了很多人,一会空难死了很多人,一会天灾死了很多人,得病去世的那就更多了。大丈夫生于世间当具超格知见,要深明了生脱死的重要性,要有勇气和决心超越六道轮回!戒色后应该好好修学净土法门,老实念佛,求生西方极乐世界,这是最究竟的选择。人身很难得,但是失去人身却很容易,得人身者如爪上土,失人身者如大地土,我们要牢牢抓住这辈子的修行机会,争取在这一生永超生死轮回,希望有缘的戒友都能往生西方极乐世界。

回向偈:

\begin{center}\it
    愿以此功德\quad 庄严佛净土 \\ 上报四重恩\quad 下济三途苦 \\ 若有见闻者\quad 悉发菩提心 \\ 尽此一报身\quad 同生极乐国
\end{center}
