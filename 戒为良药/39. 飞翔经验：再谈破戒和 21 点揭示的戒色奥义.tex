\subsection{再谈破戒和 21 点揭示的戒色奥义}

\paragraph*{前言}

戒色成功的标准,应该是坚持戒色一年以上,然后能严格控制意淫,做到基本无意淫,并且保持高度警惕。如果你能在半年左右进入戒色稳定期,基本无意淫,那么也可以算成功。很多老戒友虽然戒色很久,但他们最后只要一放松警惕,马上就会破戒,即使戒色两年以上也是会破戒。所以,我们必须保持警惕意识。\textbf{戒色成功 = 觉悟高 + 警惕强}。警惕意识是非常重要的,放松警惕就会被心魔吃掉。

传统文化讲到了戒色的重要性,中医养生也讲到了戒色保精,这两面的觉悟提高是有利于戒色的,但更关键的是专业戒色,有些戒友学中医的,深知撸管的危害,但也戒不掉,其实他所欠缺的正是专业戒色的知识,光知道撸管危害是远远不够的,一定要学会专业戒色才行。专业戒色可以从以下六个方面去完善自己:

\begin{multicols}{3}
    \begin{itemize}
        \item 断意淫的方法
        \item 情绪管理
        \item 遗精频率控制
        \item 警惕意识
        \item 养生意识
        \item 学习意识
    \end{itemize}
\end{multicols}

有戒友建议大家不要迷信我,这点我还是比较认可的,我曾经在文章里讲到过,我不是什么大师,我也不想做什么大师,我不需要任何盲目的崇拜,我只是一个普通的戒友,一个戒色前辈,分享经验只是希望能帮助到大家,仅此而已。你夸我两句,我也不会骄傲自满,你骂我两句,我也不会感到愤怒,我会一直保持一颗谦卑奉献的心,为大家服务。如果你相信我,来咨询我,我一定会给你我最好的建议。如果你不相信我,也没关系,戒色吧的高人很多,你也可以去咨询他们。但是迷信任何人都是不可取的,要信就正信,建立在正知正见上的相信。好的戒色文章可以给你很多启发和启示,但最终的路还是需要你自己去走,自己去独立思考和总结,师父领进门修行在个人。当你戒色入门后,就要靠自己努力去修觉悟和定力了。很多戒友拒绝学习前辈经验,他们想靠自己去戒色,他们的想法有一定道理,但是成功者基本都是善于学习阅读之人,前辈的经验可以让大家少走很多弯路,就像一位指路向导一样,一开始就给你指明一条正确的道路。如果你拒绝前辈的经验,那无疑对自己是一种损失,靠自己摸索会很艰难。我之所以戒色成功,也是吸取了很多人的经验和教训,一直在不断学习前人的知识和经验,然后结合自己的体验和研究进行独立思考,从而发现了戒色成功的本质。有很多人不愿意学习,其实就是心浮气躁,静不下心来,还是一味地强戒,结果可想而知。

有老戒友回戒色吧看看,发现我还在,然后他说了一句:飞翔,你怎么还在啊?是啊,很多老戒友都离开了,有的人是失败了离开,有的人是成功了离开,有的人虽然没有完全成功,但已经很少破戒了,这样的人很多也离开了,回归了正常的生活轨道,不再关心戒色了。我其实也可以离开,我做的事情完全是公益性质的,没有一分钱的收入,我把帮助别人戒色当作一份伟大的事业在做,虽然是零收入,但我做得很开心。写了这么多文章,回答了这么多的问题,如果按正常规律来看,应该早就厌倦了,但我之所以留下来,就是因为我发愿要帮助更多的人,仅此而已。如果你是资深戒友,一定会发现很多红极一时的咨询帖都沉了,因为回答者已经离开戒色吧了,刚开始回答会有成就感和新鲜感,回答多了则会产生厌倦感,从而心生退意。真正能在戒色吧留下来的答疑者,基本都是有愿力之人,真正愿意无私奉献的人,不求名利,只为帮助更多的人。戒色吧之所以能有现在的良好氛围,就是有一群这样的人在无私奉献自己。希望更多的戒友能成为奉献者,成为无私的奉献者后,你会发现你的人生会更有意义,你也会变得更加豁达和充满正能量。

最后总结下:大家可以学习我的文章获取经验,可以相信我,但不要迷信我,也不要盲目抬高我,我就是一个普通戒友。另外,我虽是信佛之人,但不会勉强大家信佛,信仰自由,一切随缘。对于有佛缘的戒友我会在答疑里说到佛教,也曾经说过放生,但只是针对信佛的戒友说的。如果你不信佛也没关系,可以走专业戒色的道路,这样也是可以戒色成功的。我的文章主要就是从专业戒色这个角度去阐述的,我走的就是专业戒色这条道路,在大量真实案例的基础上进行研究工作,研究手淫的具体危害,研究戒色规律和破戒规律,研究恢复之道,研究戒色后会遇见的各种问题。这几个方面就是我的主要研究方向。通过大量真实案例的比对和研究,让我发现了很多道理,所以我才会写文章把这些道理和大家做一个分享,希望能给大家带来有益的启示,希望大家戒得更专业更到位。

下面分享两个答疑案例。

\begin{case}
    我手淫史六年,频率也比较高(应该是很高),而且有强迫症。SY 给我带来很多不好的影响,我也深刻了解到过度 SY 的危害,决心要有所改变但是我还是有些疑问。戒色吧的宗旨是婚前禁欲、婚后节制。我想问的就是 SY 和 ML 都是伤身体的,SY 会成瘾,ML 就不会么?同样也会吧,为什么婚后节制的 ML 可以,婚前适度 SY 就不行呢?ML 可以节制,SY 很难?

    \textbf{答} SY 和 ML 都伤肾精的,ML 也可能会导致上瘾,也会导致肾亏。所以尽量在结婚后保持节制,并且要学会养生之道,不可贪欲毁身。

    第二个问题,想必很多戒友都想知道,为何要婚前禁欲,为何不说婚前就节制呢。戒色吧之所以倡导这个观念,是因为以下三方面:

    \begin{itemize}
        \item 源自道德层面的考虑,是和传统观念相接轨的,婚前属于邪淫,邪淫伤身败德,婚后属于正淫,正淫虽符合人伦,但也要注意节制。
        \item 就是避免婚前放纵,以免未婚先废,应该把最好的自己留到结婚后。我在戒色吧这么久,看到未婚先废的戒友相当多,家里催着结婚,但自己性功能已经不行了,早泄阳痿,勃起困难,这种事情也不好和女方说,一说人家就要提分手了,弄得自己很尴尬。有的戒友不仅性功能不行,甚至一身的病,一身的不适,这样的身体走进婚姻,会比较尴尬。
        \item 第三点很关键,其实戒色是一项修炼,为什么要婚前禁欲?这个道理其实和驯服野马是类似的,就是婚前要学会降伏心魔,心魔就像野马,婚后才能驾驭这匹野马。过程就是:先降伏再驾驭!如果你从来没有降伏过心魔,那就谈不上驾驭,那就无法真正做到婚后节制。也就是说,你婚前戒色成功了,降伏住心魔了,婚后才有望做到真正的节制。否则你从来没有降伏过心魔,婚后也是极有可能废掉的。记住了:是先降伏后驾驭!如果你降伏不了心魔,那么婚前婚后都有可能会废掉。我们要做欲望的主人,而不是做欲望的奴隶,做欲望的奴隶下场会很惨!
    \end{itemize}

    \subparagraph{分析} 中国的皇帝只有两个人活到了八十岁以上,一个梁武帝,信佛。一个乾隆,提倡“色勿迷”。皇帝一般也不需要手淫,后宫女人那么多,但皇帝为什么短寿呢?就是做爱太多了,皇帝看的是中国最好的医生,吃的补的是顶级的食物,但即便这样的生活条件,很多皇帝在二十多就挂了,就是过不了女色这关。很多人先天体质就弱,然而欲望却很强,这样伤肾,没几年就会被疾病缠上。所以即便是结婚后,我们一定要注重节制之道,否则很多人一结婚,做爱很频繁,结果身体也垮掉了。对于已经结婚的戒友,如果你身体出现了很多症状,那么你就应该要学会戒色养生了,可以和老婆好好沟通一下,把身子骨养养好再节制过性生活,不可贪图一时之快把自己给废了。我看很多中医医案,那些古代名医大都会要求病人禁欲一年,再配合吃中药调理,这样才有望彻底康复。否则上补下漏,永难痊愈,即使暂时痊愈了也容易复发。三分治疗,七分养生。养生里面就包括戒色保精。结婚的戒友必须和老婆沟通好,以免引起误解而引发家庭矛盾。如果你老婆比较体谅你也比较理解你,那就好办了。
\end{case}

\begin{case}
    继续请教飞翔哥,是否应该把戒色始终放在第一位,是否该有这个态度?这三个月来不断学习你的帖子和案例帖,我始终把戒色放在第一位,因为我感觉这涉及我的外貌、自尊和生命。我每天回来吃晚饭必须要学习戒色文章,三个月我已经做了三本笔记,但我不清楚是否有些过火,还是就该有此态度?请飞翔哥解答。

    \textbf{答} 嗯,你的问题曾经有人提出过。我们应该每天抽空学习和复习戒色文章和戒色笔记,持续提高觉悟,并且保持高度警惕。但我们也不能在戒色上花费太多时间,以免影响到正常的生活和学习,自己可以做个安排,比如每天学习戒色文章多久,比如一到两小时,做个时间安排,合理分配精力。在不影响正常生活和学习的基础上,我们可以多看看戒色文章和案例,但也不要把自己搞得太累。你进步非常快,如果你觉得很累或者影响你学业了,应该注意调整。加油!

    \textbf{分析} 关于我们和戒色的距离,应该是不近不远。大家戒色,也不需要一天十几个小时都泡在戒色吧,那样久坐久视,对恢复很不利。但你心里应该有戒色吧。每天都应该上来感受下戒色的氛围,然后多看看案例,多鼓励多帮助新人,你传播正气的同时也是在强化自己的正能量。当然也不一定要每天都上来,如果实在不方便,也可以把戒色笔记本拿出来复习,所谓温故而知新。我现在的时间安排,一般是白天用手机上戒色吧看看,应该有十次左右,每次就几分钟,看看案例,然后回答几个问题。然后晚上回家答疑一个小时左右,当然我还要做图片和构思文章。我不会把自己搞得很累,太累就容易起退心,所以我很注重劳逸结合和时间管理。高三党在戒色的同时,应该要更注重学业,高三这一年很辛苦也很重要,应该把更多的时间和精力分给学业。

    另外,还有的戒友会选择断网断手机,他把破戒的全部原因都归咎为电脑和手机里的诱惑,从来没有在自己身上找原因,很多人即使断网断手机依然会破戒,因为破戒有时并不需要外在的诱惑,通过自己幻想和回忆即可导致破戒,还有的人则是情绪破戒,并不是诱惑破戒。我的意见还是:要戒色成功,一定要多学习提高觉悟,否则是无法降伏心魔的。我们戒色应该要控制上网的时间,避免沉迷网络染上网瘾,但如果选择断网断手机,很多第一手的戒色文章和戒色资料就看不到了,对于觉悟的提高是不利的。当然还有一种折中的办法,那就是把戒色文章都下载到手机上,用手机阅读戒色文章,这样既远离了黄源又能继续学习戒色文章。
\end{case}

下面步入主题。这季就两个主题:再谈破戒和 21 点揭示的戒色奥义。

\subsubsection{再谈破戒}

戒色吧现在很多帖子都是破戒帖,占了很多,每天都能看见破戒帖,在破戒帖里可以看到前辈的鼓励和指导。在破戒之后不要灰心丧气,不要破罐破摔,当你学会从失败中总结经验和教训后,你就会戒得越来越好。失败可以是一块跳板,失败也可以是一个深渊,正确对待失败,从失败中学习,那么失败就是成功之母。

在戒色吧,一般会发现两类人:

\begin{multicols}{2}
    \begin{itemize}
        \item 越戒越好
        \item 越戒越差,甚至放弃
    \end{itemize}
\end{multicols}

越戒越好的戒友,基本都是善于学习、善于总结破戒原因的戒友,这类戒友有着良好的学习习惯,觉悟处在持续提升的过程中,这样一次破戒反而会让他的觉悟再上一层楼。他会开始注意导致破戒的原因,会戒得越来越精,比如有的戒友情绪破戒了,他下次就会意识到情绪管理的重要性,就会很注意及时调整自己的情绪。当他出现不良情绪了,他就意识到接下去很可能就会破戒,所以他会马上提高警惕并且调整好自己的情绪。这其实就是破戒的收获,破戒可以是消极的,破戒也可以是积极的,所谓吃一堑长一智。

第二类戒友基本都是强戒型的,或者只学了几篇戒色文章就以为自己可以了,其实要成功,必须持续学习持续提高觉悟,并不是看了几篇文章就能成功的。这类戒友也主要靠热情戒色,当破戒发生,戒色的热情就会消褪,然后就如强弩之末,再也找不到当初的戒色感觉。的确,破戒会掉士气,掉热情。但有句话是这样说的:失败可以让一个人气馁,也可以让一个人变得更强。那些杀不死你的,只会让你变得更强大。我主张理智戒色,不要靠一时冲动的热情去戒色,否则就是三分钟热情,一破戒马上就失去信心。我们应该保持理智,养成良好的学习习惯,就像每日刷牙一样自然,让觉悟持续提升才是关键。姚明不是一个晚上长到两米二六的,你要戒色成功,必须注重积累,不断复习不断反省,不断总结不断做笔记。有的戒友对于我写的文章,他真正看进去了,然后觉悟提升非常快,很多问题他都可以替我解答了。学习戒色文章我们更要注重吸收率,否则看了以后什么都不记得,那就等于白看了。

好的文章和好的书籍应该反复看,你这个月看这篇文章和下个月看这篇文章,感觉是截然不同的。对于这点我深有体会,第一遍看文章收获很有限,但过段时间再看,又会有新的发现新的领悟,再过段时间复习,又会掌握到新的知识。当你反复看过很多遍了,你就会发现自己真正吸取到了文章的精华。

对于有的戒友来说,破戒就是一块跳板,通过这块跳板,他能跳上更高的戒色层次,从而越戒越好。而对于另外一些戒友来说,破戒就像深渊,越破戒越不行,到最后就干脆放弃戒色了,选择了逃避,甚至有的戒友破戒后会跑到戒色吧来拉人下水,说什么你们戒不掉的,或者直接发黄图诱惑大家破戒。这类戒友其实就是做了叛徒,成了心魔的傀儡,成了心魔的传话筒。他们的想法我很清楚,无非是我戒不了,你们也休想戒掉,要废大家一起废,我好不了,你们也别想好。对于这类戒友一般都及时删帖,以免影响大家戒色。这类戒友应该好好忏悔下,回头是岸。

说句实话,我成功之前也破戒过无数次,当然那时我根本没有学习意识,觉悟很浅薄。那时我就强戒,最多 28 天。后来我通过学习觉悟上来后,很轻松就突破了一个月,然后就越戒越好,突破半年一年,到后来我已经不去算多少天了,我记得何时开始戒的就够了。

我们要不断尝试,不要害怕失败,要从失败中学习来增长经验,不断完善自己,这样才可能最终接近成功。当你养成良好的学习习惯了,根本就不需要失败那么多次,当你戒色入门后,有了一定的觉悟,到时突破半年其实很容易,以前之所以失败那么多次,都是因为戒色没有真正入门,还是门外汉。当你真正把握了戒色成功的真谛后,觉悟就会突飞猛进,戒色天数也会猛增。戒色吧的确有部分戒友觉悟提升得奇快,这部分戒友就是善于学习善于总结之人。觉悟上去后是什么一个感觉呢?我可以告诉你,打个比方,当你还是小学一年级时,做一年级的数学题你会觉得有难度,当你大学时,再回头做小学一年级的数学题,那简直就是小儿科,这就是觉悟提升的直接结果。当觉悟达到了,再回头看一切,就会觉得异常简单明了。但是,如果你觉悟一直上不去,那么就难以获得戒色成功。

对于一些老戒友,我提出的忠告就是:不要离开戒色文章,不要放松警惕!当你戒了一年以上或者两年以上,更不应该放松警惕。很多老戒友犯的错误就是,以为自己戒了这么久了,不会再破戒了,殊不知,只要放松警惕,不管戒多久都会破戒。警惕的螺丝不能松,否则戒色大厦将会轰然倒塌!当进入戒色稳定期后,很多人就放松了学习,放松了警惕,这样就给了心魔可趁之机,破戒就在所难免了。破戒后才会醒悟过来,悔恨自己大意失荆州。一般进入戒色稳定期后,意淫就很少了,很多人自然就放松了警惕,意淫少不代表意淫就彻底没了,心魔一直在找反扑的机会,当你警惕松懈时,就是心魔的可趁之机。所以,我们戒色必须每天都要保持警惕,保持在先知先觉的状态。当警惕的螺丝出现松动时,我们应该及时发现并且上紧警惕的螺丝,这样才可以保住戒色的成果。

戒色神力来源于学习!学习提高觉悟,觉悟战胜心魔!这三句话就是戒色成功的真谛!

保住戒色成果的根本在于警惕!严格来说,警惕意识也属于觉悟范畴,之所以分开来讲,就是为了强调警惕的重要性。

当你觉悟达到了,戒色成功将会水到渠成、瓜熟蒂落,请相信我的话。加油!

\subsubsection{21 点揭示的戒色奥义}

下面谈下 21 点揭示的戒色奥义。

大家可能会对这个题目感到疑惑,21 点是什么意思?其实我说的 21 点就是一种纸牌游戏,国外叫 black jack。不用我说,相信绝大部分戒友都应该玩过这个纸牌游戏,这个纸牌游戏非常容易上手,一般几分钟就能学会,极其简单。但这个简单的游戏要玩好却很难,我从这个纸牌游戏里悟到了戒色方面的道理,这季和大家做个分享。

21 点的规则有讲到,在要牌的过程中,如果所有的牌加起来超过 21 点,玩家就输了——叫爆掉(Bust),也叫爆牌。

其实 21 点就是爆牌的临界点,同理,我们每个人身体上也有一个出症状的临界点,每个人的临界点都不同,有的人体质好,生活习惯好,这个临界点就靠后,而有的人天生体质差,又不爱运动,这类人出症状就会比较早,也就是撸爆了。连续几天撸管或者一天多次撸管,这种情况就相当于 21 点的连续要牌,连续要牌爆掉的可能性是非常大的。连续撸管,出症状的概率是很大的,而很多人之所以连续撸管,是因为一次没爽够或者被瘾控制,身不由己,一破戒就是连续撸几次,这样出症状的可能性就很大了。我们要做的就是远离那个出症状的临界点,一方面就是坚持戒色,另外一方面就是注意养生之道,这样就能远离那个出症状的临界点了。不仅撸管会让你滑向那个临界点,熬夜久坐等不良生活习惯也会让你滑向那个临界点。过了那个临界点,就会出症状。

21 点的规则是,越靠近 21 点就越好,超过 21 点就爆掉。而我们戒色则是越远离临界点越好,越靠近临界点则越危险。

撸爆的戒友是极多的,因为他们不知道身体有那么一个点,也就是出症状的临界点,他们一发不可收拾,被瘾控制,疯狂撸管,结果没多久就会被症状缠上。体质差的人,出现症状比较早,体质好的人,症状出现会晚一些,但只要沉迷撸管,出症状就是必然的,只不过有的人是暗疾,表面上看着没事,但只有他自己知道,身体已经有了各种不适。还有的人属于后知后觉之人,身体已经有很多症状了,但他因为无知愚昧,从来没往撸管上想,还以为是其他什么原因导致的。当他看到戒色文章后,才恍然大悟,原来是这样,这么多年身体的各种不适,原来是撸管导致的,这时候他才醒悟过来。

21 点要赢,必须冒爆掉的风险,特别在 16 点、17 点时,这时候要一张牌很可能就会爆掉。而我们戒色,最高明的办法就是一次撸管都不要,因为撸管会上瘾,你要一次就会导致连续要,这样就会引发一个悲剧性的结果,那就是:你撸爆掉了!被症状缠上。

还要说明一点的就是,意淫也是在要牌,意淫也会导致爆掉,意淫后出症状的戒友也是非常多的,看过我以前文章的戒友应该知道意淫是暗漏这个道理。所以我们戒色,一定要克服意淫,过了意淫关,戒色就容易很多了,基本就成功了一半。否则即使你深知所有撸管的危害,但无法克服意淫,那也不可能会戒色成功。

我以前比较无知的时候,也是疯狂撸管,爆掉过无数次,但那时候没有高人点醒我,也没有人告诉我戒色知识,我那时就是井底之蛙。当我通过不断学习觉悟上来后,再看看以前无知的我,我觉得很悲剧也很可怜,那时的我,撸得疯狂,戒得盲目,结果就是无数次的破戒,无数次的失败。我那时试图在找一个动态平衡,我相信很多戒友都有这个想法,就是希望自己能在撸管和出症状之间找到一个平衡点,就是既能撸管,但又不出症状。后来我发现要做到平衡是极难极难的,因为撸管有高度成瘾性,会导致你连续要牌,最终就是爆掉的结局,等你爆掉后再戒,其实只有弥补之功,没有预防之力了。有部分戒友的确能做到一周一次,但他无法克服意淫,他在意淫上漏掉的更多,而他因为无知,根本不知道,还以为自己控制得很好。没过多久,他就会发现,他被症状缠上了。

21 点玩多了,自然会有爆掉的经历,撸管时间长了,自然会滑向那个临界点,结果就是撸爆了,常在河边走,哪能不湿鞋。我看很多戒友的案例,都是撸管前几年没事,然后从第三年或者第五年开始,症状就慢慢出现了,有的戒友去年还好好的,今年症状一下都爆发出来了。量变产生质变,伤精到一定程度,自然就会出症状,没有人可以逃得掉!精少则病,牢记药王孙思邈的这句话,永远不会错。

撸管爆掉的过程和玩 21 点爆掉的过程是非常相似的,当你连续要牌,等待你的最大可能就是爆掉。真正玩 21 点的高手,他会算牌、记牌和算爆掉的概率,但即便这样,他也有失手的时候,但是撸管则不同,你不撸管,就不会因为撸管爆掉。就像你拒绝玩 21 点,你就不会爆掉一样。你如果想玩,就很容易爆掉。最高明的做法就是:拒绝再玩,拒绝再撸!再也不会因为撸管而爆掉!

最后总结三句话:

\begin{itemize}
    \item 你不撸管,就不会因为撸管爆掉。
    \item 你一旦开始撸管,爆掉的可能性就会极高。
    \item 拒绝再撸,自觉远离那个出症状的临界点。
\end{itemize}

这季推荐一本书:

\begin{book}[《西藏生死书》,索甲仁波切]
    相信有些戒友应该知道这本书,我花了五天看完了这本书,这本书加深了我对佛教的理解,也加深了我对生死的认识,的确是一本很有启示意义的书,一下打开了我的视界。看完这本书,我又看了《中阴闻教得度》,看完这两本书我才发现我以前对于死亡的认识是多么肤浅,以前我是逃避死亡,比较忌讳死亡。看完这本书,让我对死亡有了一个全新的认识和理解,也知道了应该如何面对死亡的降临。有兴趣的戒友可以看看,应该会给你带来很大的启示。
\end{book}
