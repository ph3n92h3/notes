\subsection{戒色的十种力量}

前言:

上季一位戒友被一种念头所困扰,那就是怀疑的念头,怀疑戒色,也包括怀疑戒色前辈。他戒到一定时间,脑中莫名其妙就冒出这种怀疑的念头,搅得他内心很不安。这其实就是心魔的一种怂恿,是非常阴险毒辣的怂恿,直接动摇你的戒色立场和决心,如果信根不行,那就很容易出现动摇。心魔会怂恿你去怀疑,即使你已经亲身体验了手淫的诸多危害,也亲自体验了戒色之后的各种好处,即使这样,心魔还是会怂恿你去怀疑戒色,就像打仗时候的“离间计”一样,是非常贼、非常恶毒的怂恿。在修道方面,看一个人的根器,一方面要看他的悟性,另外就是要看他的信根如何,大根器对应的是大信根。经云:“信为道源功德母,长养一切诸善根。”如果信心坚固,那么在修行的过程中就可以克服诸多障碍,反之信根浮浅,那就很容易打退堂鼓,之前就有这类戒友,就是信根不行,立场不够坚定,最后又回到破戒的怪圈中去了。我们戒色必须要具备钢铁般的决心、信心和立场,死不动摇,死硬到底,你有这样的烈士气概就等于成功了一大半。信心就是一种力量,信力必须足够强大,我们选择戒色,并不是盲目的迷信,而是在深入的调查研究、亲身的体会和深刻认识邪淫危害的基础上才选择戒色的,也在戒色之后真正体会到了各种好的改变,这样就更加坚定了戒色的决心和信心。这种信是正信和智信,戒色是明智之举,古圣先贤告诫我们要戒色,国外最新的科学研究也证明了色情和手淫的诸多危害,选择戒色是绝对正确的。戒色并不是让你做和尚,而是让你学会控制自己的欲望。对于心魔的怂恿一定要学会识破,特别是怂恿你去怀疑戒色,这种怀疑的念头千万不可听信,必须立刻断除,自己必须特别坚定才是,只有真正的坚定者才能坚持到最后,动摇者必定会破戒,我们一定要做坚定的戒者。心魔会千方百计地搞垮你,各种诡计,各种套路,你所要做的就是学会识别心魔的诡计,立刻断除那类念头,必须坚定自己的决心和立场,千万不要上当。

有的新人会想通过健身来戒撸,觉得把自己搞累了,就不会想了,在锻炼的当天的确是如此,但是等休息好了,欲望反而会变得更强烈,因为力量训练会促进雄性激素分泌,到时邪念起来会更猛。就算你把自己练得像变形金刚一样,你依然无法抵挡住心魔疯狂的进攻,戒色最关键的是修心,不是修身。很多人强壮起来后会变得更加盲目,以为自己强壮就没事,于是疯狂撸管,最后把自己搞得外强中干,暗疾缠身。只要撸,身体迟早会有垮下来的一天,早点晚点,我见过很多强壮的人,最后身体垮下来时连普通人都不如,包括健美冠军还有大力士等,你和这些专业人士相比,你的强壮无法与他们抗衡,但是他们的身体也一样垮掉了,放纵的生活方式会击穿航空母舰。戒色后应该选择适合自己身体的锻炼强度,身体比较虚弱就以散步为主,多做做养生功法,等身体恢复得不错了,那就可以考虑做做力量训练,这样对于性功能的恢复是很有好处的,但是力量训练后一定要加强修心。如果你的身体强壮起来了,但是你却无法降伏心魔,心魔就会用你强壮的身躯疯狂邪淫,到时垮下来时就像雪崩一样。

下面分享一些案例。

\begin{case}
    男人三十而立,本该事业上一展身手,成家立业的年纪,而我三天两头跑医院,检查,吃药,活生生地把自己当成一个药罐子。如果你问我为什么,那就是这十四年来的手淫纵欲摧垮了自己,牺牲了幸福换来的只是每晚那几秒钟的快感。本人今年三十一,现在没有工作,没有存款(信用卡还透支两万)没有结婚,而且还一身病,这一切的一切就是从十四年前开始的,慢慢一点一点地侵蚀着我,把我变成一个没有活力、没有激情、没有动力、没有憧憬的废人。那年我才十七,也是受损友误导学会手淫渐渐地一手摧毁自己。刚开始我撸得很频繁,一般一天三四次,甚至五六次。过了半年就出现肾虚,半夜睡觉出虚汗的症状。好像差不多撸了一年多左右,有一次撸多了突然出现头疼的症状,说不出来的那种疼,很想呕吐,我想那一次是上天第一次对我警示。不过我底子很好一个礼拜就恢复了,一直没有出现过大问题。后来当兵去了,新兵连还没下连队就得了五更泻,每天天还没亮肚子就疼要上厕所,我刚到部队表现一直很好,就是因为这个天天拉肚子,身体素质跟不上来,训练也跟不上去,在部队的生涯中我记得正常的大便不超过一个礼拜,天天拉稀。期间也去过部队卫生院一直没看好,在部队期间我还是一直手淫的。这次又感染前列腺炎了,去年为了治这毛病花了好多钱,身心俱残。
    \subparagraph{附评} 三十而立,而邪淫之人,三十难立!即使立起来也是豆腐渣工程,说倒就倒,人家立起来,顶天立地,气撑乾坤,而撸者真的很难立,勉强立起来也不稳,很容易再次垮下去。别人的生活过得风生水起,家庭事业双丰收,而很多撸者则是躲在阴暗的角落里流泪悲叹,悔恨因为无知而放纵自己,掏空自己,让自己沦为废人。这位戒友原本的底子还是很不错的,但是他纵欲很疯狂,虽然没有得什么大病,但虚汗、拉稀和前列腺炎也够他受的,再发展下去估计就要得神经症了,或者得其他的大病,到时就更惨了。一个人生在天地间,一定要懂得规范自己的行为,哪些行为不能去做,自己一定要清清楚楚,放纵是很容易的,学好却很难,很多人都是无师自通,放纵就像坐滑滑梯,滑滑梯的末端就是地狱,滑下去真的太容易了,而戒色就像登山,需要你一步步扎扎实实地往上走,每一步都需要一定的努力,不是那么容易的。犹太人必须恪守《塔木德》所列的 613 条戒律,这些戒律中,有训戒 248 条、禁戒 365 条,内容覆盖日常生活的各个方面,人以戒立,在学校要遵守校规,在公司上班要遵守单位的规章制度,无规矩无以成方圆,规者,正圆之器,矩者,正方之器,要成正器,必须要有规矩的约束,古人强调“不逾矩”,一逾矩,灾难就来了。在性方面更要严格约束自己,千万不可放纵自己,世上无如人欲险,要毁掉一个人,就满足他的欲望,欲望是个无底洞,要了还想要,如饮咸水,越饮越渴,我们一定要懂得控制自己的欲望,邪淫的行为必须要戒除。《尚书》云:“天道福善祸淫。”你沉迷邪淫,就是在背离天道,就是在自取灭亡。一个人放纵自己的欲望,即使身体底子再厚,也会有变薄的一天,就像一块很厚的肥皂在不知不觉间就变薄了,撸管废人就是一个不知不觉的过程,它是一种能量的耗损,当你猛然惊醒时,身心已经出现很多问题了,当你想戒掉撸管时,却发现心魔是那样的强大,已成骑虎难下之势。五更泄,又名鸡鸣泄、肾泄。病因是由肾阳不足,命门火衰,阴寒内盛所致。中医认为,此病主要由于脾肾阳虚所致。好汉架不住三泡稀,频繁地拉稀腹泻会导致腿软无力,走路就像踩棉花,脚力虚浮,双腿打颤。当“噼里啪啦”响起时,你就知道撸管是要还的,积撸成疾,久撸必病。不少人得了前列腺炎,就去医院送钱了,少则几千,多则几万、十几万,好多年都没看好,因为他还在纵欲,如果不懂得戒色养生,慢前是无法真正痊愈的,往往在撸管后会再次复发,复发率高得惊人。本该在事业上一展身手、有所建树,而却因为这个恶习而沦为了药罐子、病秧子,那几秒的快感,代价实在太大太大了。
\end{case}

\begin{case}
    手淫十五年,人已经快要油尽灯枯了,请问还有救吗?我今年三十岁了,手淫十五年了,如今虽然已经戒了,但是身体已经垮掉了。浑身冷,经常恶心干呕,头晕脑胀,还有难以形容的难受,现在走路都很吃力,站都站不稳。真的几乎油尽灯枯了,我感觉我快要死掉了,每天都很遭罪很痛苦,手淫无害论害死我了,我还有救吗?为什么我戒了两个月了还是没有好转?现在身上阳气几乎枯竭了,手脚耳朵都冰冷。
    \subparagraph{附评} 精含乎骨髓,上通髓海,下贯尾闾,人身之至宝也,故天一之水不竭,则耳目聪明,肢体强健,如水之润物,而百物皆毓,又如油之养灯,油不竭,则灯不灭。这位戒友三十岁了,十五年的撸龄,几乎油尽灯枯了,他已经撸出神经症了,有神衰的症状表现,应该要积极就医治疗了。撸了十五年,而且症状已经很严重,恢复肯定需要一个过程,两个月还太短,至少要一年以上了,光戒不行,自己要学会养生之道,加强养生恢复。5年撸龄时,很多人感觉还行,症状不是特别严重,而当达到十年、十五年或者二十年撸龄时,所有的恶果将会集中爆发出来,就像火山爆发一样。我看过非常多的案例,那些撸者都是在达到一定的撸龄后突然废掉的,之前他们还以为自己身体没事,其实潜在的伤害正在不断积累,当达到临界值,就会突然垮下来。《千金要方》云:“若不能制,纵情施泻,即是膏火将灭,更去其油,可不深自防,所患人少年时不知道,知道亦不能信行之,至老乃知道。”色是少年第一关,不仅是少年,在青年、中年、老年时也都要懂得戒色保精,不可放纵自己的欲望。精是生命之灯的油,你一次次耗损宝贵的能量,其实就是在把自己推入万劫不复的深渊,你还愚痴地以为自己爽到了,其实你亏大了,根本就是巨亏!砖家的说法很扯,你懂的,当你症状爆发了,请问,砖家在哪里?当你在窗口付医药费,请问,砖家在哪里?当你躺上手术台,请问,砖家在哪里?当你被痛苦折磨时,请问,砖家在哪里?当你的人生撸成一坨屎,感到异常绝望时,请问,砖家又在哪里?砖家站着说话不腰痛,他们的歪理只会让你沉迷于手淫恶习,用无害论安慰自己,把无害论作为放纵的借口,这完全是在自欺欺人,最后身心俱废,形神俱灭,沦为行尸走肉。
\end{case}

\begin{case}
    好不容易戒了一百三十多天,今天早上起来突然好想看黄,心里说要忍住,但最后还是没有忍住看了,连续破了两次,现在心里好讨厌自己为什么没有坚持住,从今天开始戒两百天这是真的不能再破了,加油!
    \subparagraph{附评} 《家庭宝筏》云:“一星之火,可以燎原。罅漏之水。可以决堤。吾谓淫念亦然。立地起念,即要立地一刀斩断。著不得一些游移,容不得一毫缱绻,否则魔愈深,势愈炽。”断念的关键就是不能让邪念起势,一旦起势,就会被它所控制。断念要快,断念要狠,不能手软,不能犹豫,不能贪恋。戒到一定天数,心魔肯定会入侵的,这位戒友说“突然好想看黄”,看黄的微妙念头起来了,于是心里另外一个声音说“要忍住”,但是最后没忍住,破了!与心魔交战就在一瞬间,看你是否能够制服心魔,还是被心魔所俘。想看黄的念头就是心魔在入侵了,这种念头一出现在脑海里,就要引起高度警觉,必须立刻断除,如果做不到,那就是平时疏于练习,平时没有好好磨刀,到时邪念一上脑,就无法斩断。实战时有四种情况,第一种,起了念头,马上就去看黄,没有任何抵抗,这是最菜鸟的表现;第二种表现,念头来了,会去抵抗,但是断力不行,无法斩断,稍微挣扎一下又被心魔拿下,还是破戒;第三种,有时可以斩断,有时做不到,断念水平忽上忽下,不稳定,这类人也容易再次破戒;第四种,是真正的断念高手,完全是压倒性的胜利,胜负没有悬念,这类高手才是真正的戒色刀客,具备超一流的断念水平。大家可以看看自己是第几种,断念是需要不断练习的,就像刚开始学打字,你打得很慢,还要看键盘,手指也不会放,但是经常打,经常练习,最后就十指如飞了,打字速度越来越快,这就是不断练习的结果。刚开始水平都比较低,戒色觉悟也不够完善,对实战的认识也不深刻,所以必须不断深入学习戒色文章,对戒色的原理和规律有更深入的悟解,然后必须解行并进,有解无行,那是纸上谈兵,不管用,据我观察,空谈理论的人很多,搞理论却不注重练习断念,这样迟早还会破戒。有解有行,解行并重,这样就能越戒越好。平时应该养成观心的习惯,观心就是向内看,看自己的念头,用心感受念头是如何在你脑海中成形的,又是如何消失的,仔细观察那个过程,用心去体会,充分感受那个细腻微妙的过程。不断练习观心断念,你处理念头的能力就会越来越强,到最后,你就像一个老练的魔术师,可以让念头瞬间消失,一下就把它变没了。这位戒友虽然戒了一百三十多天,但显而易见,他的断念实战还有很大的提升空间,在面对心魔入侵时,他还不具备斩钉截铁的力量,还需要不断强化练习。有的戒友精进练习一段时间,就感觉断念的功夫大进,然后还须不断加强练习,以臻大成。这就像拿到一本武功秘籍,一方面要反复深入研读它,另外就是要不断练习,这样实战水平才能突飞猛进,从而彻底告别被心魔狂虐的岁月。
\end{case}

\begin{case}
    今天去看了一位老中医,号脉说我心脏不好,我问为什么,他说肾阳不足引起的脾虚,脾统血,导致心脏供血不足,头晕,肝胆不好。他说先给你开几服药,说去医院是省不掉的,心脏问题很大。我现在的心情真是……咳!根源就在肾阳不足,明天要去医院做心电图,看看心脏的具体情况,以后真的不想撸了。
    \subparagraph{附评} 撸到后来五脏六腑都会出问题,记得以前我心脏也疼过,当时很害怕,也做过心脏的相关检查,伤到一定程度,真是全身不对劲,好像没一个地方是舒服的,那种感觉很惶恐。中医讲肾为五脏之根,撸管就是在砍你的根,就算你这棵树再粗,也经不起经年累月地砍,小斧头可以砍倒合抱之木。到最后消化不行了,心脏也出问题了,呼吸系统也不给力了,肝胆也不好了,五脏六腑都失调了,刚开始可能还是功能性的问题,到最后就可能得上器质性的疾病,到时就可能要动手术了,就像一部破车进厂大修。我那时还有一段“惊心动魄”的经历,之前在戒色文章里没有提到,那是在戒色前,记得有段时间我很放纵,然后突然脖子下面长了一个硬块,有玻璃弹珠那么大,也不痛,当时我很恐慌,以为自己得了不治之症,赶紧去医院检查,在等检查报告的那段时间是最煎熬的,就像等待死刑判决一样,有的戒友应该有过类似的经历和体会。后来报告上说是淋巴结肿大,但我自己清楚,这次肿大很诡异,我之前撸管后淋巴结也肿大过,但这次的肿大很硬,像石头一样,而且和普通的淋巴结肿大很不一样,我知道这是身体给我的进一步的严重警告了,真的不能再邪淫了。刚开始身体会给你较轻的提示和警告,然后你戒不掉,身体就会给你更严重的警告,以告知你应该戒色了,不能再放纵了。《本草纲目》记载:“肾乃先天之本,生命之基;肾气衰,百病起。治病必求于本,本之为言根也、源也。世未有无源之流,无根之木,澄其源而流自清,灌其根而枝乃茂,自然之经也。故善为医者,必责根本。”一棵树,如果伤了枝干,它还能活,如果直接把根砍了,这棵树就活不了了,根实在太重要了,肾就扮演着根的角色,你每一次撸,都是在朝根上砍一斧头,砍到最后你就疾病缠身了,从这个角度来讲,撸管完全就是在自残,根本不是什么享受,那几秒的快感将会导致之后无量的惶恐与痛苦,那种生活犹如惊弓之鸟,惶惶不可终日,整日奔波在家和医院之间,花钱如流水,身体也差劲,真的陷入了极深的困境。撸管会把自己的身体变成“危楼”,随时都可能会坍塌,邪淫是一条荆棘之路,戒色修善是一条光明大道。戒色正士们,勇猛前行吧!
\end{case}

\begin{case}
    本人今年十九,你们感觉是不是有点小,但是邪淫超过了五年。没邪淫之前,我的身体非常好,长跑,短跑,全班第一。但是自从开始邪淫,我的身体真的一天不如一天,直到现在,我跑步跑一会,我的头皮就开始发麻,非常容易出汗,我现在每天晚上都失眠,动不动就腰疼,掉头发。每天都感觉生不如死,看以前和现在的照片,真的是两个人。我以前也尝试过戒,但是以前都是强戒,根本不行,强戒绝对戒不了。这两天我才发现,要从心里面去戒,每天多看看帖子,多锻炼锻炼身体,每天要保持好心情。希望刚开始邪淫的,记住,不是不报时候未到。
    \subparagraph{附评} 撸管对体能的影响是很大的,我初中是校队的,田径和篮球,在沉迷撸管后,明显感觉体能有所下降,容易气喘,持续强度能力下滑很厉害。有的人说自己撸管后症状不明显,其实让他去跑跑步,测验一下,就知道体能下降严重,很多事情要测验才知道的,就像听完课,你说你全懂了,但是一考试,证明很多知识并未真懂。这位戒友也算是班里的运动健将了,全班第一,可惜他染上了撸管恶习,身体就大不如前了,而他也就十九岁。现在的孩子面对的诱惑远超于 80 后和 80 后相同年龄段所遭遇的诱惑,现在是网络时代,获得那类资源很容易,而且诱惑图和擦边新闻太多太多了,只要上网,肯定会看到,孩子本来定力就低下,再一接触黄源,后果可想而知。这位戒友每天晚上都失眠,考虑有神衰了,每天都感觉生不如死,他的处境很糟糕。我们在受苦是因为我们无法做最纯净、最纯粹的自己,我们无法与本具的快乐和美好连接,我们一直认为看黄手淫会获得满足,却不知道无欲就意味着满足的状态,大家观察一下纯净的孩子,他们的脸上自然就有一种满足的表情与神态,感觉很幸福,内心极为开心快乐。手淫不会带给你真正的满足,那只是多巴胺的骗局,真正的满足只有纯净的心灵才能给你。从纯真善良的孩子变成面目全非的猥琐大叔,悲催啊!你明白那种感觉吗?手淫后深深的无力感与空虚感,那种被心魔操控、身不由己的感觉,那种看到自己容貌气色下降的自卑感与颓废感。戒色到一定程度,邪淫带来的厚重与压抑就会消失,就像乌云消散了一般,整个人突然变得好轻松、好惬意、好明亮、好清澈,如释重负的感觉,看看脚下,撸管沉重的脚镣已经卸掉了,一种轻松自由的感觉充满身心,你终于自由了,撸囚的生涯结束了。之前的 aitrue 小吧在戒色三年多后说:“我轻松了,我自由了。”他的这句话很简单,但是却是那么发自肺腑,那么意味深长,那么耐人寻味,只有亲身体会过的人才明白自由的宝贵。撸管不是自由,撸管会带来束缚,会给你戴上邪淫的枷锁,沦为撸囚的日子真的不好受,表面上你好像很自由,可以走来走去,可以找黄看黄,可以疯狂撸管,但其实你的内心根本感觉不到真正的自由,你一直在被心魔所奴役,你一直在过着一种隐秘堕落的生活,那些肮脏龌龊的邪念已经占据了你的头脑,这不是自由的生活,这根本就是在自毁!另外一位戒友说道:“感觉这么多年好像都在监狱里,现在有一种自由和纯净的感觉。”戒者一身白光罩身,散发着纯净的光芒,撸者周身黑气缠绕,散发着邪淫的恶臭。大家应该都有这样的体会,那就是撸到一定程度,自己都会觉得自己恶心,自己会痛恨自己,人都是有良知的,这样沉迷于邪淫对不起父母的养育之恩,也对不起列祖列宗,看着那副猥琐样,自己都觉得恶心,更别提周围人的感受了。这位戒友也意识到强戒是不行的,我们一定要学会专业戒色,懂得戒色的原理和规律,这样才有望彻底戒掉撸管。
\end{case}

\begin{case}
    今天是我戒色的 208 天,我以为我差不多快走向成功了,可是欲望来得太突然了,就在今天中午,我欲望大起,脑子里闪过一个念头,说:“看次黄吧!反正我已经不会去撸了。”可是没成想,一撸回到解放前。不甘心啊!!!
    \subparagraph{附评} 埃克哈特·托利在《Stillness Speaks》中说:“念头的续流有着巨大的力量,它能轻易地拖拽着你与它在一起。每个念头都假装它非常重要。它想要吸引你全部的注意力。”我们脑海中会冒出各种念头,我们可以观察这些念头而不陷入进去,一旦你认同了,你就会陷入进去,跟着念头跑,被念头带走。这位戒友脑中闪过一个看黄的念头,他相信了,他认同了,他跟了!他没有及时断除,结果就被这个念头攻破了,心魔的表现就是邪念袭脑,能够导致破戒的念头都可以称之为邪念。戒到一定程度,脑中会闪过念头或者图像,它们企图带走你,让你彻底陷进去。“看次黄吧!”这个念头明显是心魔的怂恿,他没有识别,也没有断除,他听信了,结果可想而知。戒色 208 天很不容易,破戒后恢复成果不会一下归零,但应该避免连续破戒,不管戒多久,都应该时刻保持警惕,每天都是实战,必须牢牢看住自己的念头。形意拳有句古诀:“遇敌好似火烧身。”这句话大家好好体会下,这种实战的警觉度有多强烈,火烧你的时候,你的反应是什么?当一个人被烫到的时候,他马上就进入了极度警觉的状态。邪念上来时,我们不能敌我不分,很多戒友犯的毛病就是敌我不分、认贼作子。念敌上来后,你要警觉啊!这是敌人,要立刻消灭之,不消灭,就附体。我看过很多大德的开示,对于妄念,他们用得比较多的词就是消灭、斩断、降伏、断除等,有些大德会说,妄念上来了不要理即可,其实不理不随就是断!斩断妄念的相续。对于处理妄念,大德的表述有很多种,但最根本的还是:降伏!也就是《金刚经》里的“降伏其心”,必须降伏自己的心魔,降伏闯入头脑的邪念!你必须强硬,不能在实战中表现软弱,一念之差,心魔就会攻破你。这位戒友说自己“不甘心”,我很理解他破戒后的心情,本来胜利在望,突然又被心魔翻了回去,再次被心魔打败,真的很懊悔很颓丧。就一个念头没处理好,结果就又回到撸管的怪圈中去了。一定要警惕啊!念头一上来,你就要极度警觉,不能跟!最近看了一个法制节目,给我很大启发,那个受害者收到一条短信,意思大概是“聚会的照片发你一份”,然后附了一条链接,受害者不知有诈,于是就点开了,里面什么也没有,就退出来了,过了一会,手机里收到银行的通知短信,说是五万元被转走了!这才知道上当受骗了,在点开那个链接时,其实就已经中了木马病毒了。我们戒色的过程中,心魔也会给我们的大脑发“短信”,如果你无法识别并且跟随了,那就等于中了木马病毒,心魔发给你的念头就是木马病毒,你还以为是自己的想法,于是跟了,结果就造成肾精账户被心魔盗刷。心魔很狡猾,它会冒充你,以第一口吻给你发短信,发进你的大脑,让你以为是自己的想法。资深的戒色高手已经学会识别心魔的这种诡计了,戒色高手不会跟,他们只是冷眼观察,在实战中一定要做“正确的事”!心魔的力量很强大,它一直试图把你拉入它的掌控之下,你能感觉到那股极其强大的拉力,一直在试图把你拉入那个怪圈、那个轨道,你必须时刻警觉,学会识别心魔的各种套路,学会击溃心魔所有的进攻!慈舟法师云:“佛法的仗,是正念与邪念打,败了再打,时久自然邪不胜正。”佛法的根本是修心,戒色也是如此,道理是完全相通的,和心魔作战,必须知己知彼,这样才能百战不殆。
\end{case}

\begin{case}
    今天是我刚刚结婚的第 21 天,也是我的生日,我今年 26 岁。从高中开始,我就开始SY,到我结婚前已经有最少十年的时间了,现在身体已经垮得不行了,已经被我自己给掏空了,结婚前我就有预感,虽然能硬起来,但硬度不够,秒射,所以,我吃药,希望通过吃药来弥补,但是你们想想,吃药真的能解决问题吗?结婚后的第二天我就买药,之前也没吃过这类东西,连续七天都有房事,之后这几天再吃药,已经不管用了,勃起的时候总是软软的,感觉不像我这个年纪血气方刚的样子。现在的我已经变成了一具空壳、一具尸体,肮脏的发臭的腐烂的尸体!!!媳妇现在已经睡沙发了,说心里很烦,其实谁都能看得出来,我不行!!! 都是手淫把我害成这样,我现在睡不着,在想以后的日子怎么过,我才刚刚二十多岁!!!还没有做爸爸。
    \subparagraph{附评} 新婚燕尔,很容易出现放纵的情况。这位戒友 26 岁,撸龄超十年,大家想想一件东西用了十年,都已经很旧了,甚至都坏过了。婚前禁欲非常之重要,应该把最好的自己留到结婚后节制使用,未婚先废的情况太多太多了,都是因为年轻无知和砖家的误导。戒色就是能量管理,你应该把你的能量用于学业和事业,而不是用在邪淫上,很多人还没结婚就已经撸成废人了,而家人还在催着结婚,这时身心俱废,都害怕结婚了。SY 就是在掏空自己,一次次撸,一次次掏,结婚后再放纵,那又是一顿猛掏,吃药纵欲,那完全是在自取灭亡,非常之可怕,《养生保命录》:“好色者,恃药以恣欲,此亡身之本也。”某球星吃药纵欲,最后口吐白沫昏迷过去,差点就挂了,人家是职业运动员的身体素质,也经不起如此的摧残。吃药纵欲,完全是在拿生命开玩笑。在年轻时就应该有保精养生的意识,这种意识非常重要,而且夫妻双方都应该有这个意识,不可在婚后疯狂放纵自己的欲望,应该安排好每月性生活的次数,一般建议三次以内。新婚的话,很难控制,但只要双方取得共识,然后具备较高的戒色养生觉悟和修心功夫,这样就可以做到。疯狂放纵的后果真的太可怕了,身体垮了,性功能废了,到时大多都是离婚的结局。古语有云:“精养灵根气养神,元阳不走得其真,丹田养就千金宝,万两黄金莫与人。”116 岁的罗明山老中医主张:“青壮年常节欲,年纪老宜分居。”并告诫人们,“肾精人之宝,不可轻放炮;惜精即惜命,固精人难老”。肾精是人身的至宝,一定要懂得珍惜,必须要有保精惜精的意识,很多年轻人非常无知,保精意识几乎为零!而且脑子里还有很多无害论,这样就会更加肆无忌惮,废得也特别快。男人都希望自己行,当性功能不行了,这时候就会很恐慌,其实这就是身体在警告你了,要读懂身体给出的信号,不能看到自己不行就吃药纵欲,这样完全是雪上加霜,就像一个人已经很穷了,你还要逼着他拿钱出来。我们一定要懂得中医养生之道,一定要学会积精累气、积精全神,达到“精足、气满、神旺”的能量状态,反之,不断耗损则会导致“精损、气耗、神伤”,真气耗散,疾病缠身。《紫清指玄集》中说:“能固其精,宝其气,全其神,三田精满,五脏气盈,然后谓之丹成。”《黄庭经》曰:“仙人道士非有神,积精累气以成真。”积精、保精、惜精的意识实在太重要了,肾精其实就是能量,肾精不足则形神衰败,精充气壮则气宇轩昂,我们必须要学会戒色养生,即使在结婚后也不能放纵,如果出现明显症状了,应该先戒色养生一段时间,等身体恢复了再节制性生活。
\end{case}

\begin{case}
    真的是太感谢飞翔哥了,我现在有种说不出来的喜悦和激动,就在两年前我还是一个昏昏沉沉的废人,现在戒色两年零七个月居然考上了黑带一段而且技术达到了黑带二段,并且被总教练看中,正准备送入国家示范团和龙英知府(中国水平最高的道馆)深造学习,我真的是零基础一年啊!之前完完全全没有接触过!飞翔哥!没有你,没有吧友们的努力,真的我不敢想象有今天,练习跆拳道本身就很消耗内力,没有很好的内力体力耐力是完全不可能练好的,而之前别说练功,走路都腿麻,很痛很痛,真是不知道说什么了,太感谢了!我现在戒骄戒躁依然努力!每天学习戒色文章和帖子两小时,依然每天练习跆拳道,再接再厉,加油飞翔哥!你改变了很多人的命运,因为你,让废人变成黑带精英!
    \subparagraph{附评} 这是“戒色集团军”的反馈,之前收录过他的案例,他现在已经考上黑带一段了,戒色两年零七个月,改变颇巨,可以说是从废人到黑带精英,彻底告别了东亚病夫的形象,蜕变成了一个英气十足、顶天立地的精武英雄。其实相对于他的跆拳道功夫,我更看重的是他的“戒骄戒躁”,这四个字与德行有关,最高的功夫其实就是德!一定要注重德行的提升与修炼,这点至关重要!很多人最后就是败在德行不够上,德不配位,必有灾殃,一骄傲、一生气、一急躁,然后就容易破戒。有了很好的德行,戒色成果就容易守得住,所谓厚德载物,“戒色集团军”很有这方面的意识,这才有了他奇迹般的蜕变。中国传统武术都会在练功前强调戒色保精的重要性,很多书籍在一开始就强调要戒色,要懂得保养自己的精气神,然后也会强调武德的重要性,习武先学德,打拳先行礼,武德的培养非常重要,“学拳以德行为先,凡事恭敬谦逊不与人争,方是正人君子。学拳宜以涵养为主,举动间要心平气和,喜气迎人,方免灾殃。”戒色也是如此,德行在任何一个领域都是至关重要的,戒色先做人,戒色先修德。之所以传统武术特别强调精气神和武德,是因为要把功夫练上去,必须要有这两样作为基础,不知戒色保精,则身体很容易因为纵欲而衰败,这样功夫就难以练上去;不知修德,就容易沦为争勇斗狠、闯祸惹事之辈,有的练武之人后来坐牢了,就是武德不行,用在了邪门歪道上。“戒色集团军”零基础一年就考上了黑带,可以说非常难得,正是因为有了戒色养生作为基础,他才能在这一年内突飞猛进,获得巨大的进步,而且他也注重修德,这样基础就更稳固了。他和过去判若两人,过去走路腿都麻,而现在可以做高难度的踢腿动作,看过他的视频,的确很潇洒,很有震撼力。古德云:“养生之士,先宝其精,精满则气壮,气壮则神旺,神旺则身健,身健则少病。内则五脏敷华,外则肌肤润泽,容颜光彩,耳目聪明。”干什么事都需要一个好身体,一个良好的精神面貌和充沛的精力,而戒色养生就可以给你这个基础,有了这个强大的基础,很多事情就容易成功,反观很多人的福报是很不错,但是最后他守不住,最终因为犯了邪淫而彻底垮下来。希望“戒色集团军”能够继续保持谦虚谨慎、戒骄戒躁,希望他在跆拳道方面能够有大的发展,希望他的人生路能够越走越好。真正能改变命运的,还是要靠自己,我最多负责指路,最终还需要自己去悟、自己去走,师傅领进门,修行靠个人。“君子有造命之学,命由我立,福自己求。祸福无门,唯人自招,善恶之报,如影随行!”好好改过迁善,多学习圣贤教育,这样人生命运即可发生逆转,到时就会迎来全新的蜕变。正所谓:“金鳞岂是池中物,一戒邪淫便化龙。”不戒撸,你都不知道自己到底有多优秀,当你达到了,你真的会感到惊讶。
\end{case}

\begin{case}
    飞翔大哥,您好!我终于领悟到了您所说的无念胜有念了。过程是这样的:我这一段时间遇到了严重的瓶颈期,就是断念断得很麻木,戒色道理都懂,可是到了实战经常当断不断,比如说邪念出现了,我一边念着断 YY 口诀,一边还是被邪念牵着走,很难断掉邪念。今天我打坐的时候邪念出现了,我直接一眨眼,念头就灭掉了,后来一直都是这样,邪念一冒头,我直接一觉察,一眨眼,念头就断掉了。我平日里都坚持看戒色文章,我一直坚信持之以恒地学习肯定会有所收获的,今天这个情况我想应该就是顿悟了吧,原来断念可以如此的简单,我终于明白飞翔大哥您文章里说的一秒钟断念是什么概念了,真的太奇妙了,我是纯粹的觉知,念头不是我,努力付出一定会有回报的,世上无难事,只怕有心人!感恩飞翔老师给予我第二条生命,感恩佛菩萨的加持,感恩古圣先贤的教诲,我深知戒色修善这条路我才刚刚开始,我会继续努力前进的!南无阿弥陀佛。
    \subparagraph{附评} 《太极拳论》:“由招熟而渐悟懂劲,由懂劲而阶及神明。然非用力之久,不能豁然贯通焉。”不断练习断念,到时你就会像太极高手一样,做到交手即扑、搭手即飞,瞬间制敌!然而要达到这种境界,必须不断练习断念,用力之久,豁然贯通,就像打通隧道一样,坚持下去,就会迎来顿悟和质的飞跃。在那一刹那,你突然能做到了,那种感觉真的很喜悦,就像第一次扣篮成功一样,整个人感觉非常振奋。断念水平的提升是需要一个过程的,这个过程就是一个不断练习的过程,神枪手不是一朝一夕练成的,是逐步强化的结果,强化、强化、再强化,最后就会越来越强大,在练习的过程中是会出现很多问题的,甚至会出现暂时的退步,这种暂时的退步往往是进步的前兆,就像潮水一样,先退回去,然后冲得更远。唯有继续坚持练下去,就会迎来断念水平的猛然提升,到时即可秒杀心魔,之前是心魔秒你,而现在你强大起来后,你开始秒心魔了,断得越来越快。戚继光说:“奇妙在于熟之而已。”熟能生巧,熟极自神,就像练一个魔术,不断练习,就会渐臻化境,有的手法魔术甚至要练几年才敢出去表演,这个世界上最奇妙的事情之一,就是把自己的念头变没,把一幢大楼变没都不及把自己的念头瞬间变没,这真的太奇妙了。要进入真正严格且高层次的戒色境界,那就必须不断强化断念,之前有位戒一百六十多天的戒友,心魔一来,又垮下来了,前几天有一个戒一年八个月的,当心魔再次疯狂进攻他时,他断念不力,结果就破戒了。要进入高层次、极为稳定的戒色境界,那就必须把断念修炼到极为强悍的地步,从登堂入室到登峰造极,全凭苦练而成。朝夕揣摩,刻苦研练,练懂技法,练深功力,要达到高深的造诣,必须要有大量练习的积累。武谚有云:“拳打千遍,身法自现。”千遍只是最基础的积累,要出神入化、登峰造极,何止千遍,要万遍、万万遍。《一万小时天才理论》里就专门提到了精深练习,里面讲到:“一万小时的法则的关键在于,没有例外之人,没有人仅用三千小时就能达到世界级水准,七千五百小时也不行,一定要一万小时——十年,每天三小时——无论你是谁。”我们要投入最大的热情去练习,并且持之以恒,这样断念的水平才会越来越强,不必担心暂时的退步,坚持下去,练下去!练到死!真干狠干!做断念的大师,做断念的绝顶高手!古德云:“理现念犹侵。”就是道理都懂,但是念头还是会入侵,光懂道理没用啊,夸夸其谈,一无是处,而且还懂得很浅。念头自动会冒出来,很多时候不是你主动去起邪念,而是它自动冒出,这就是“念侵”,你即使把几千篇戒色文章倒背如流,全部默写出来,你也无法击溃心魔,因为断念缺少练习!你把射击的理论书籍全部背下来,你会成为神枪手吗?神枪手是上万颗子弹喂出来的,天天都在练习,精深且持久地练习,最后才成为神枪手的。很多戒友犯的毛病就是“只学不练”,只学理论却不注重练习断念,大家应该充分重视断念的练习,这是极为重要的。这位戒友终于学会觉察消灭了,他之前已经完成了量的积累,现在终于迎来顿悟,断念水平有了质的飞跃,觉察就是全屏超杀,瞬间清空你的头脑,让你进入无敌状态。之前一位资深戒友戒了近两年,他说他还在努力练习断念,他说这条路是没有止境的,我的感受和他差不多,我戒到现在,即使每次都能击溃心魔的进攻,但我觉得自己的断念水平还有很大的提升空间,的确是“练无止境”。太极高手和太极宗师在达到高层次的境界后,还每天在练习,还在精益求精。我们一定要坚持练习断念,练到后来你会发现自己对念头的感知进入了更微妙细腻的层面,你的感受更精微了,对内心的细微波动都有极其敏感的感知,对念头的感受力明显深化和加强了。你就像一片大海,对海平面的微小波纹,都有着极其强烈的感知。戒色是一门很深奥的学问,而断念是一门很高深的功夫,当你逐渐进入高层次后,你一定会同意我的看法,越往高走,越觉得不可思议,只可意会不可言传。戒色修心会让你获得一种内在的力量感与主宰感,否则无论你外在多么强壮,都难掩内在的虚弱无力。真正的力量来自于降伏心魔!菜鸟面对心魔,不堪一击,而高手眨眼间结束战斗!绝不陷入拖沓、犹豫不决和贪恋,快刀斩念,没有任何拖泥带水,手起刀落,彰显戒色刀客的王者风范。
\end{case}

\begin{case}
    我是一名研究生,从初中开始 SY,到今天已长达十三年,占了人生的近一半时间,一半啊,这种对青春、对生命的浪费让我震惊、懊恼。我大约一年多前接触戒色吧,然后是传统文化和佛法,接触这些正能量后我确定找到了自己的病根,回想以往种种不堪,以及邪淫前的纯净,我坚信自己戒除后必定会前途无量、福报绵绵。甚至有一次在戒了一个月之后,好运连连,我深信戒色后的自己是无敌的,然而后来又放松警惕失败了。回到学校后已经差不多晚上六点了,疲惫感仍在,我就回宿舍了,准备还像往常一样躺床上看看电视,放松一下压抑的心灵。然而濒死感已逼近,在我看了会电视后,就感觉身体很不舒服,内心不得安定,心率很快,手脚冰冷,浑身不自在,越躺越难受,有点像晕车要吐的感觉,但不完全像,因为不想吐,就是心神紊乱,很难表达那种感受,我又忍着看了一会儿,发现实在不行了,头也晕,身无力,我就把手机关了,把自己裹到被子里,手脚怎么也捂不热,内心充满不和谐,不安定,焦躁无比,仿佛灵魂离体,突然我有一种奇怪的感觉,非常可怕,我快速坐起身来,意识到了一点,我的生命在流逝,可怕,可怕到无法形容,无助。我察觉到自己寿限将至,我悔恨无比,欲哭无泪,我立马跪在床上拼命磕头,念佛号,慌乱无比,左顾右盼,不知道要怎么做才能阻止体内生机的流失,我感觉到自己不受控制,我突然理解了新闻上死在宿舍的大学生死前的种种怪异行为,我理解了,因为我正处于这个状态,感觉到了灵魂在远离这副皮囊,也深刻地感受到做任何事都无济于事,这种无力感,焦躁感,恐惧感,可以让人做出任何怪异的举动,此时身边的一切都在远去,床位,桌子,新买的垫被,摆放整齐的各式皮鞋……一切都结束了,整个世界,都在离我越来越远。朋友拉着我冰冷的手直奔大路上打车,我一路上躬着腰,无力到快瘫软了,我向朋友痛苦地哭诉:“我知道自己有这一天,我早就想戒了,但做不到,这是报应来了。”边哭边说,一会儿功夫脑袋不能思考了,连交谈都不行了,朋友问我话我都不回了,全部能量都用来维持意识了,心想只要能赶到医院,即使失去意识说不定也还有救。坐到医生面前,我朋友就替我说症状:“浑身颤抖,无力,手脚冰冷。”我直接打断朋友的话:“我刚刚下午手淫了,现在有濒死的感觉,我感觉自己的生命在流失。”说着我低头一只手捂起了脸,哭,不是因为 SY 恶习说出来感到羞耻,而是在面对死亡时,对生命的眷恋,对生的执着,在生存面前,羞耻心已一文不值。医生给我量了血压,说血压没事啊,稍微高了点,心跳也快点,医生看着我紧张无助的脸,跟我说生命体征正常,让我放心,不会挂的。但我的濒死感那么清晰,主要让我痛苦害怕的是三个症状:后脑麻痹,心神不宁(我感觉死神就在旁边),后面两个腰子感觉跟被挖了一样空空如也,还跟着心脏一跳一跳的。
    \subparagraph{附评} 这是一位戒友帖子里面的摘录,他撸出了濒死感,很值得大家警醒。古德云:“人身之有精神,犹居家之有资财,财尽则穷,精尽则死,理之必然者也。”又云:“若耽于荒淫,则渐渐志识昏迷,心神衰耗,即使年少气盛,不即觉露,日复一日,终于不振,而百病随之,安所复望其学有进益乎?且此心一涉淫邪,正务必至懈弛,安肆日偷,正人自远,非类渐亲,气质委靡,举动苟且。”之前也有不少人撸出了濒死感,新闻也报道过撸管猝死的事情,这类事情远远比报道的要多很多,报道的只是冰山一角、九牛一毛,每年中国猝死的人数超五十万,这是一个很庞大的数字,其中肯定有不少撸者,撸管猝死不一定是发生在看黄撸管时,也可能是撸完后几小时或者一两天后,突然身体就不行了,那种感觉极度恐慌,那种生命力流失的感觉让人感到极度崩溃,这个世界的一切都在远离自己而去,一切似乎都与自己无关了。这位戒友是研究生,十三年的撸龄,也是该还了,这次濒死的经历一定会让他下最大决心来戒色,因为他真正尝到了果报的可怕。一般撸管猝死也查不出具体原因,最多就是心脏骤停,器官都是好的,生命力却没了,也就是能量枯竭了,就像一停电,电脑就不能用了,虽然电脑零件都是好的,但是无法打开了。肾精代表的就是人的能量、生机,是供给五脏六腑和大脑用的,这种能量怎么可以随便撸掉?国外一张戒色宣传图片给我留下了很深刻的印象,就是一个人因为撸管猝死在房间内,他母亲抱着他痛哭,现在这个社会诱惑太猛烈,加之无害论的毒害,很多人都会疯狂撸管,但他们却不知道后果将是什么,完全无知。很多孩子都想恢复邪淫前的纯净美好,那种生活才是自己真正想要的,但是无奈心魔太强大,有的人虽然学佛了,但还是无法战胜心魔,学佛一定要真正掌握核心,核心就是修心,就是要学会对治自己的邪念!否则只流于表面形式,这样内心就无法获得真正的转化。我们一定要和心魔斗争到底,这是你死我活的战斗,你必须不顾一切地强大起来,否则就无法摆脱被心魔狂虐的悲惨命运。国外戒色网站有一张宣传画就是扳手腕的图片,意思就是要战胜手淫恶习,战胜自己的心魔,比心魔强,就不会破戒。国外戒色文章也谈到了要和自己的邪念做斗争,要学会控制自己的念头,这其实就是在讲修心。刚开始戒色,毕竟觉悟尚浅,很可能会出现破戒,破戒后一定要好好反省和总结,不要气馁和灰心,要加强学习补强觉悟,强化练习断念,这样就能越戒越好,从敌强我弱的局面慢慢转变成我方强过心魔,到时就有胜利的把握了。我曾经被心魔虐了十多年,回想那十多年,一直在心魔的操控之下,而我完全是无知的菜鸟,后来学习戒色文章后,才慢慢转变这种被动挨打的局面,当我真正强大起来后,心魔就难以攻破我了。即使戒到现在,心魔还在试图攻破我,但我一次都没让心魔得逞,古人和心魔斗争了几十年,到最后才做到念头自然不起,所以这是一场持久战,必须做好持久战的心理准备。有的新人会说要忘记戒色忘记手淫,这种想法是错误的,因为戒色不是戒烟,烟是身外之物,可以丢掉可以忘记,而淫欲种子就在你的八识田里,它自动就会冒出来,你想忘记戒色,但是心魔却会疯狂进攻,所以说忘记戒色是行不通的,这是一种逃避,是对戒色形势认识不清。有篇戒色文章里是这样说的:“现在社会上的不少青年都会看黄手淫,我们仔细观察这些人,这些人已经没有灵气了,取而代之的是一种灰蒙蒙的邪气淫气,人的灵气随着欲望的增长逐渐消耗殆尽,都说人是万物之灵,这里的灵是纯净和智慧高贵的意思,但如果人一旦被欲望所蒙蔽,那么灵气就会尽失,到时候人就和畜生没什么区别了,我们可以看下现在的人的所作所为,都和畜生差不多了,因为人的灵性正在消退,所以人就会表现得像畜生一样甚至连畜生都不如,这都是贪婪淫欲和食欲财欲等欲望所导致的。”这段说得非常好,不仅是社会上的青年,还有在校的学生,他们因为染上手淫恶习而无法自拔,灵气渐消,戾气渐重,整个人灰头土脸的,一脸死气沉沉,像行尸走肉一样,有的人活了快三十岁了,有一半时间是在邪淫堕落中度过的,这真是一种悲哀,无法做最纯净的自己,就是最大的痛苦!体内的生机通过邪淫流失,身心容貌都开始衰败,到时恶报现前,真的悔恨无比,这个案例就是最好的警醒,可不惧哉!
\end{case}

下面步入正文。

这季将和大家分享戒色方面更深入更细致的体会,戒色的确是一门很深的学问,戒到最后你会发现戒色原来和做人、修德、修道是密切相关的,戒色有很多层次,最低的层次就是不学习,一味强戒,这样戒色注定失败,一般强戒在一个月左右就会破戒,短的几天就会破戒。即使有相当的觉悟,但还是可能会出现破戒,为什么有的人在半年时会垮下来,有的人在一年多垮下来,还有的人则在两年左右垮下来,这背后都是有原因的,一定是戒到某个阶段出现了问题,而他无法克服,随后就破戒了。放松警惕是常见的问题,恢复不理想感到灰心也是一个问题,另外也有德行方面的原因,德行是古圣先贤一直强调的,《孟子》:“仁人无敌于天下。”仁者无敌,要进入更高的戒色境界,必须注重修德。《论语》云:“志士仁人,无求生以害仁,有杀身以成仁。”志士仁人那种气节,那种崇高的精神境界,鼓舞了一代又一代中国人,使国人在民族危亡的关头,舍小家顾大家,前赴后继,奋勇前进。戒色先做人,戒色先修德,学会做人和修德,这样就有了稳固的基础,继续戒下去,慢慢又会被修道的内容所吸引,到时就会发现戒色和修道是高度契合的。当你脱离了欲望的束缚,能量就开始朝上走,到时脑子就会灵光很多,很容易明白大德到底在讲什么,很容易契入进去,到时就会感叹圣贤教育不可思议,圣贤教育有着专业戒色无法达到的境界,毕竟是几千年的深厚智慧,真的不可思议。

要明白那些极具深意的智慧话语,是需要一个过程的,你以为你懂了,其实你没懂,深层的含义你没认识到。刚开始看到一句话,你也许看过就看过了,但是当你觉悟提升到一定程度,再回过头来看那句话,就突然能明白那句话的分量与深意。之前就有戒友和我反馈过,说他过了很久才明白我文章里那句话的意思,真的明白后,觉悟就会猛然提升,之前看过,但没看懂,似懂非懂,坚持学习戒色文章,再结合实战的体会,突然有一天就看懂了,觉悟一下就上去了。我自己也是如此,大德的一句话我用了三年多才看得恍然大悟,有种欢喜雀跃的感觉,因为我真正悟进去了,尝到法味了,所以感到非常喜悦和兴奋,之前虽然看过很多遍,但因为体会不够,所以对那句话理解不深刻,现在体会越来越深入,对于那句话突然就有了很深的悟解,那句话虽然看似普通,但却极其重要。有的戒友戒了几百天,感叹自己的觉悟与当初的菜鸟状态真的不可同日而语,当初处于菜鸟阶段,都会问一些在以后看来极为白痴的问题,这也不能怪新人,新人的确会问那些问题,作为老戒友要耐心回答,好好引导他们。当他们觉悟上来后,他们就能明白前辈的良苦用心,前辈一遍遍地耐心回答,有的问题都回答上百遍了,但前辈没有不耐烦,前辈深知度人即度己,即使回答上百遍依然耐心地回答新人的问题,这里面就有一种无私奉献的精神在,这种精神也会通过言传身教传给新一批的戒友,让他们懂得什么叫无私奉献。前辈为什么选择不断地帮助新人和宣传戒色,因为这样做既可以行善积德,培植自己的福报,也可以起到稳固戒色状态的作用,又可以培养一种崇高的心态,崇高的心态对于戒色至关重要,崇高的心态可以让你远离邪淫的低级趣味。

戒色是一个完整的多层次的体系,要求你必须具备深入且全面的觉悟水平和较强的实战意识。有的人虽然觉悟不够完善,他也能戒到一定天数,但是再想进入更高的戒色层次,他就做不到了。最近看到一位戒友,他戒了三百多天破戒了,他说:“真是想死的心都有了!”他的一个同学在空间上传了不良视频,他好奇点开看了,于是就破戒了。这就是明知不可为而为之,在对境实战时做了错误的选择,让好奇心占了上风,他没有遵守戒色战场的纪律,看到诱惑应该立刻避开,而不是去撞子弹!那种图片和视频根本就不能看,一眼都不能去看,这点实战意识必须要有啊!放任自己主动去看,肯定会陷进去,就像着了魔一样,疯狂看那种视频,那种状态太可怕。能戒三百多天实属不易,但是一不警惕,就毁于一旦。戒色老兵为何能幸存?大家想过吗?老兵的警惕性是极强的,老兵知道怎样做到不阵亡!戒色吧每天都有大量戒友阵亡,基本都是自己送死去的,老兵看多了,知道戒色战场的残酷,所以他的实战意识和警惕意识已经远远高于普通的戒友。新人一看到图片,立马就陷进去了,那些图片就像陷阱一样,看到视频更是完全失控,而资深前辈不会犯这个错,深厚的经验让他只做正确的选择。现在手机新闻很多都是擦边内容,对于这类内容要高度警惕,不要去看,有的人会觉得看看新闻应该没事,但你要知道现在的新闻很多都配了诱惑图片,你无聊的时候点进去,很容易就会失控,很多戒友就在这上面栽了大跟头。在实战中一定要高度警惕,每天要认真反省,不断加强自己的实战意识。你要学会克服自己的好奇心,大德讲,常与自己逆,便是进功。这句话大家要深入理解,你放任自己的好奇心,那不是在找死吗?!王骧陆居士说过:“姑息二字,等于自杀。”纵容那种冲动,就像自杀一样,本来戒了三百多天,形势一片大好,然而一个诱惑视频就让他惨烈阵亡了,那个视频就像地雷一样,他主动去踩地雷,不是找死吗?!古德云:“见欲而止为德。”能做到止,必须要有很高的觉悟和警惕意识,“香饵之下,必有悬鱼”,遇见这类视频很多人都会去点,但是戒色老兵不会去点,为什么他与别人的反应不同,因为他的实战意识强啊!他知道那种视频不能去看,他坚决执行戒色战场的纪律!

前段时间一位戒了一年四个月的戒友也破了,他说自己觉悟不够,还说:“不论你戒色多少天,只要有朝一日掉以轻心就一定会破戒,我破了,我拿我血的教训告诉大家,戒色是一个持久战役,一刻都不能松懈。”我们生活在一个邪淫泛滥的时代,诱惑之猛烈是空前的,不出家门半步,你就可以看到让自己血脉贲张的诱惑内容,同样是网络,也可以让你接触到善文和善知识,所以网络是一把双刃剑,机遇与危机并存。要戒除恶习,那就必须多学习戒色文章完善自己的觉悟,而且要时刻警惕自己,当哪一天你对那种诱惑又有了某种说不清的渴望时,你就要格外警觉了,当你的视线不自觉地停留在擦边图上,你更要高度警觉。有时那种感觉是很微妙的,属于比较细微的念头,它一上来,如果不觉察,那你就会去看黄,就像上面那个戒友看到诱惑视频,有一种好奇想看的微妙感觉,这种感觉就在几毫秒间,是一瞬间的想点想看的微细念头。我们平时就要格外防着这种微妙的感觉,在对境的一刹那,这种感觉钻出来了,闪进大脑了,如果不立刻断掉,后果不堪设想。大家可以做个试验,进入你常去的一个正规网站,在你进去后,眼睛看到的刹那,你被哪个标题或者图片吸引了,进而去点击,那个“想看”的微细念头几乎快到感觉不到,但如果你有很强的实战意识和观心功夫,你就能捕捉到它,它就像一根毫毛一样,极其细微。你可以看着网站页面,但不做任何点击的动作,当你去点击了,一定是“想看”了,注意观察那个微妙的过程,当面对的是诱惑的内容时,要学会切断这个微妙的“想看”的感觉。有的戒友会说,看到破戒帖感觉很负面,其实破戒帖也是有区分的,不少破戒帖可以起到警醒的作用,就像在告诉你,如果你不警惕,下一个阵亡的就是你,还有的破戒帖有在反省和总结,这也值得我们学习,从而避免犯类似的错误。

构筑戒色体系需要全面完善的觉悟,要学会补强觉悟的短板,现在戒三个月、戒半年、戒一年多的戒友有很多了,如果大家想进一步提升自己,一定要更加完善自己的觉悟。下面我分享戒色的十种力量,如果这十种力量你都具备了,你就会戒得特别稳定,我戒到现在,一直在坚持这十种力量。

\paragraph{决心的力量}

生命的潜能往往在你下最大决心时,被完全激发了出来。戒色需要下最大决心,这是很重要的前提,很多戒友就是发不出大决心,不是很想戒,没有那种坚决果断的意志。当有了大决心,就会用最大的热情与动力来戒色,别人一天看一篇戒色文章,而你可以做到每天看五篇乃至十篇以上,因为下了最大决心,所以你就会真正全力以赴。决心的力量是不可思议的,但决心一定要落实到行动上来,光有决心没有行动,那是不行的,要决心猛烈,行动更猛烈,这样才给力!强大的决心对于克服强迫思维也很有帮助,记得以前我也被强迫思维纠缠过,有过强迫症的戒友应该对此有极其深刻的体会,念头总是纠缠不休,明知无意义却还要一遍遍地重复想,重复确认,后来我发现,只要下决心不去想,然后再加强观心断念,这样强迫思维就会逐渐减弱消失,非常灵验,断念后告诉自己再也不想了,下最大决心,这样就可以大大弱化强迫思维,要有一种壮士断腕的决心!有的戒友之所以发不出决心,就是对邪淫危害的认识不深刻,另外一方面就是自己对症状的体验不深刻,因为还没有撸到很严重的程度,所以他就得过且过,混日子,戒色也是马马虎虎不上心。有的人都习惯于破戒了,一点决心都没有,彻底沦为了戒油子,要突破现状,就必须狠下决心,像烈性炸药一样摧毁一切障碍!这股狠劲必须有!拼死戒,戒到死!这种刚烈的决心要经常发!

《孙子兵法》里讲:“死地则战!”退无可退,陷入绝境,这时候就会爆发出最强大的求生意志,会拿出最大的决心来拼死战斗,有的将领专门创造死地的局面和环境,好让士兵破釜沉舟,以一敌三,拼死得胜。前几周有一个戒友在我帖子里回复,说他撸出濒死感了,这次他下大决心了,因为已经陷入绝境了。如果善根比较深厚,又懂得专业戒色,这样就不需要等到染上重疾再下大决心,看看那些触目惊心的案例就足以下最大决心来戒。我当初就下了最大的决心,体现在行动上就是猛烈学习戒色文章,拼命练习断念。我渴望降伏心魔,我不想再过被心魔奴役的生活了,这种灰暗颓废的日子我真的受够了!在心魔的铁蹄和暴政下,我活得异常惶恐,完全就是处在大黑暗时代,被负能量撑爆,内心非常痛苦。我再也不想过这种放纵堕落的生活了,我渴望打碎邪淫的重枷,我渴望恢复久违的自由、纯净与美好。渴望改变的迫切程度必须达到最大值,这样才能激发出最大的潜力。必须发出最大的决心,然后积极行动起来,到时你就会见证自己的逆袭。决心一下,再一行动,能量和气势马上就不同了,非常快。你会发现,下了大决心之后,你的眼神都会发生改变,有一种无形的力量已经加在你身上了。

\paragraph{忏悔的力量}

《大智度论》云:“人罪能悔,已悔则放舍;如是心安乐,不应常念着。”忏悔则安乐,对自己的过错应该好好忏悔,但不要一直挂在心上,心里应该保持清净安乐,如果一直纠结于过去某个过错,这样心里就不清净了。忏悔的意思是,反省自己已犯的错误,愿意面对它,承担起责任,从此改正错误,决心不再犯。人非圣贤,孰能无过,知错能改,善莫大焉,忏悔本身就是一种修行。“往昔所造诸恶业,皆由无始贪嗔痴;从身语意之所生,一切我今皆忏悔。”这个忏悔偈我现在每天都在念,早晚各一遍,但我心里不念着过去的错误,心里应该保持清净,这样才会安乐。有的戒友总是纠结于过去的某个过失,搞得心里很不安稳,有忏悔心是好的,但不要纠结于以前的过失,关键是改过迁善,恢复内心的清净与祥和。罪从心起将心忏,心若灭时罪亦亡;心亡罪灭两俱空,是则名为真忏悔。忏悔的力量犹如热汤浇雪、朝阳照霜,只要诚心忏悔,所做的罪障就会消灭于无形之中。《金光明经》云:“所谓金光,灭除诸恶,千劫所作,极恶重罪,若能至心,一忏悔者,如是众罪,悉皆灭尽,我今已说,忏悔之法,是金光明,清净微妙,速能灭除,一切业障。”《心地观经》:“若覆罪者,罪即增长,发露忏悔,罪即消除。”

忏悔的力量非常大,但不是叫你故意作恶,有的人可能觉得既然忏悔可以灭除业障,那就随意作恶好了,有这种念头就不对了,忏悔是为了防止你作恶,把你引向正道,不是叫你有恃无恐地作恶,这点一定要明白。

\paragraph{勇猛的力量}

奥运会期间看过一条新闻:郎平肌肉竟完爆朱婷!郎平的肱二头肌的确练得很好,很强悍,铁榔头名不虚传,要获得胜利,那就必须要有勇猛风范。修行方面叫勇猛精进!蕅益大师在《百法明门论直解》中云:“精进者,于断恶修善事中,勇猛强悍而为体性。”戒色应该要拿出勇猛强悍的表现,在断恶修善中,刚毅果断,克服懈怠和拖延,把自己的冲劲和狠劲充分调动起来,就像冲进敌阵的将军,大劈大砍,杀出一条血路,用血战的功夫死拼到底。真正的戒者个个都是下山猛虎,个个都是敢死队员。我看到不少戒友戒得死气沉沉的,一点拼劲都没有,这样怎么能行呢!永嘉法师在《证道歌》里写道:“大丈夫,秉慧剑,般若锋兮金刚焰。非但空摧外道心,早曾落却天魔胆。”勇猛风范可见一斑,戒色就是一人与万人敌,你一个人要与成千上万的邪念作战!这必须是极为勇猛的战将才可以胜任,要多发勇猛心,多激励自己,披上勇猛精进的盔甲,冲锋陷阵,誓死杀敌,发扬血战到底之威势。

当年张自忠将军就是怀着必死的决心上战场的,我们要学习抗战老兵的铮铮铁骨和浩然正气,他们不畏生死,奋勇杀敌,百般血战,绝不苟且。他们这种烈士气概很值得我们后人学习,戒色就是一场残酷的战争,我们要做戒色的特种兵,戒出血性,戒出英雄气概,赢下这场战争!戒色需要的是一种极其强悍的战斗精神,一种不屈不挠的战斗意志。在喜峰口战役中,二十九军及其大刀队血战日寇,杀声震天,血光满地,威武雄壮,慷慨激昂。现在我们要把这种精神、这种力量用到戒色上来,“大刀向心魔头上砍去”,做戒色的烈丈夫,绝不做邪淫的孬种,挥舞大刀勇猛冲杀,苦练杀敌本领时刻准备战斗,绝对不能坐以待毙,必须冲起来,拿出你的拼劲,用一种势不可挡的勇猛气势来戒色,拿出最强悍的表现,摧毁心魔的进攻。

\paragraph{信心的力量}

关于信心的力量,我在这季前言里也提到了,信心对于戒色实在太重要了,古人讲“心香一瓣”,内心虔诚,恭敬谦顺,方能至诚感通。大家在刚开始戒色时,其实已经拉开了差距,因为每个人的信根和善根是不同的,有的人一戒就特别坚定,任何无害论不能动摇,而有的人还没戒,就在怀疑这怀疑那,内心很不坚定,极其容易动摇。大浪淘沙,真正坚定的人会坚持到最后,这类戒友是真正的戒色烈士,属于特别坚定、不可动摇的戒者,就像狼牙山五壮士一样忠烈顽强、不屈不挠、死硬到底。戒色吧出过叛徒,这类人信心浅薄、业障深重,被反戒分子所动摇,于是对戒色吧起了邪见。在中国革命的过程中也出过叛徒,这属于历史必然的规律,见怪不怪了。他们想继续过邪淫的生活,谁也拦不住,最后症状缠身、报应现前,也是咎由自取、自作自受。每个人的路都是自己选的,他们想回到堕落的怪圈中去,也只能随他们去。

佛法方面有“胜解信”,胜解信强烈、殊胜、清净、理性,是最坚固的正信,胜解信不是偶然冲动的信心,不是肤浅的信心,是深入了解和认识后的信心,是非常深厚且牢固的信心,纯一而不夹杂怀疑。我们戒色也需要有胜解信,信心必须异常坚固,立场必须特别坚定,这样你开始戒色,力量就会与众不同,信心本来就是一种极大的力量,有了深厚的信心,你才会全力以赴地去戒,你只要有一点怀疑,你的力量就不具备了,所以必须消除怀疑,坚定自己的信心。很多时候心魔也会向你发送怀疑的念头,企图动摇你的戒色立场,你一定要学会识破,并且通过思维来坚定自己的戒色立场,你可以这样思维:“就是死,我也不会动摇了!”、“不管心魔你怎么怂恿我,我都不会动摇,死硬到底,绝不动摇!”必须要拿出烈士的气概来应对心魔的这类怂恿,心魔非常阴险与狡猾,它把你搞动摇了,你就会自动破戒。我们一定要万分小心,时时刻刻提防心魔的怂恿,就像防贼一样防着心魔!

“信心是珍宝,昼夜趋善道。”希望每位戒友好好坚定自己的戒色信心,要像钢铁般坚定不可动摇,这样戒起来就勇猛了,就像威武的雄狮一样!有了坚固的信心支持,你的气场就会变得很强大,你的内心也会变得很安稳、很祥和。只有做到不可动摇,才能不断突破戒色的层次,才能日新月异、大有进境。

\paragraph{谦卑的力量}

关于谦卑,之前我在戒色文章里专门强调过,很多人都是因为戒到一定程度出现骄傲的念头而破戒的,骄傲意味着放松警惕,很容易被心魔钻空子,所谓骄兵必败!前段时间一位戒友就来向我诉说,说他戒了一年多,感觉自己很厉害了,起了骄傲的念头,结果没多久就破戒了,他很懊悔。我们真的不能骄傲,我看到不少新人的帖子都在轻敌和骄傲,老戒友戒到一定程度也有不少骄傲的,结果他们没多久就破戒了,心魔一直在等待他们骄傲,一骄傲,心魔就好下手了。骄傲即破绽,你露出破绽,就会被心魔抓住,被心魔一顿痛扁!王阳明先生云:“人生大病,只是一‘傲’字。”古人讲,虚己者进德之基。谦卑、谦虚太重要了,为什么说戒到最后拼的是德行?因为德行不够,就会自动败下阵来!戒色一定要注重德行的培养和提升,我看戒友的发言就知道他能戒多久以及能达到什么境界,只要他表现出一点点自傲的倾向,我就知道他肯定会破戒。人一傲慢,就像在心灵的清泉里拉了一坨屎,要多臭有多臭,是对纯净心灵的严重污染,所以绝对不允许这类念头出现在心里。《周易·谦》:“谦谦君子,卑以自牧。”又云:“劳谦君子,万民服也。”勤劳谦卑的君子,民众都敬服他。谦在卦象中,六爻皆吉,谦卑是修行人一生的功课,傲慢损福报很快,谦卑者可以迅速积累起福报。真正有修为的人往往安静谦卑,不轻易发脾气,人很稳重。学会谦卑是人生的一种智慧,谦卑,是一种美德,是一种涵养,更是一种成熟。谦卑的心灵就像深山里的泉水,清澈透明,能辉映出蓝天白云和我们纯净的灵魂。不成熟的麦穗竖得很高,越成熟的麦穗越往下垂,生命越成熟的人就越懂得谦卑与感恩。

有一次,一位内地企业家慕名前去拜见李嘉诚,会谈结束之后,李嘉诚特意从办公室出来,送他到电梯口。最让人惊叹的是,李嘉诚不是送到即走,而是毕恭毕敬地鞠躬,直到电梯门合上。李嘉诚作为一个功成名就的人,依然在修谦卑,最让人心生敬意的是,年逾七旬的他竟然亲自将客人送至电梯口,还毕恭毕敬地给客人鞠躬直至电梯门合上。李嘉诚做人、做事的谦卑和细致很值得我们后辈学习,为什么李嘉诚能成功,而且还是真正的常青树,这和他的德行是分不开的,他的德行能够匹配他所处的位子,这样就能守得住,才能长久不倒。李嘉诚有那么多的财富,但他依然保持着一颗谦卑的心,而正是这种谦卑成就了他的事业,使他的事业在走向成功的同时,也让他的人格走向伟大。谦卑是一种修为,地低成海,人低成王。即使功成名就,还是要把自己放低,保持谦卑,无私奉献,服务他人。

耶稣曾经把一个小孩子带到门徒的中间,要他们回转像小孩子一样。对他们说,人如果不回转变成小孩子的样子,就不能进天国。在你们中间,凡是谦卑像这个小孩子的,他在天国里是最大的。柔和谦卑实在太重要了,《易》曰,“刚而能柔,吉之道也。”谦卑的心是非常柔和、柔软的,太极拳理有云:“极柔软,然后能极坚刚。”柔软至极才能出纯刚,太极拳看上去很柔软,古代叫太极拳为“棉拳”,但是实战时却能显出惊人的威力,如棉裹铁,外现柔软,内含坚刚。就像水可以很柔软、很温柔,但也可以像海啸一样摧毁一切。水既能柔,也能刚,刚柔并济,方合中道。谦卑的心就是柔,面对心魔时的强硬就是刚,作为一个真正的戒者,刚柔都必须具备,你会发现当你把柔做到极致了,你才能把刚做到极致,极柔软才能极坚刚。如果你不修谦卑,那就是偏刚,偏刚易折,最后起了傲慢心,自己把自己给折掉了。锐气藏于胸,和气浮于面,外圆内方,骨刚气柔,憨山大师云:“学道人第一要骨气刚。”但同时也要懂得谦卑、谦下、谦和、谦己让人,懂得宽容和感恩,这就是柔。

\paragraph{惭愧的力量}

《大般涅槃经》云:“惭者,自不作罪;愧者,不教他作。惭者,内自羞耻;愧者,发露向人。”惭愧的力量非常大,戒色后要多发惭愧心,《杂阿含经》云:“世间若成就,惭愧二法者,增长清净道,永闭生死门。”惭愧心能生种种善法,是修行人的大宝,我们应该做一个有惭愧心的人。应该越修越感到惭愧,如果越修越贡高我慢,那就完全走错了。印光大师自称为“常惭愧僧”,多发惭愧心可以破除骄慢,很多戒友就是败在骄慢上,修惭愧心可以有效预防和对治骄慢心,这点非常重要!惭愧是“七圣财”之一,信和戒也属于七圣财,圣严法师说:“如果能够常把‘惭愧’两个字放在心头,则会有三大好处:第一是不敢懈怠,会非常精进、努力。第二则是非常谦虚,不但见到任何人都会尊敬,并且会无条件地帮助人。第三是能够忍辱负重,因为懂得惭愧,所以难行能行、难忍能忍、难舍能舍,这就是菩萨精神。”

因为有了惭愧心,所以就可以防止自己作恶,邪淫之人往往缺少惭愧心,这样就更加容易放纵自己而不知悔改,有的撸者甚至以放纵之事到处炫耀,完全恬不知耻,一点惭愧心都没有。经云:“若离惭耻,则失诸功德。有愧之人,则有善法。若无愧者,与诸禽兽无相异也。”有了惭愧心,就容易改好,所以要多培养惭愧心,我现在每天都在发惭愧心,一发惭愧心,我就觉得正向的力量增强了,内心变得很清净,有的戒友可能比较重视谦卑心和感恩心,对惭愧心的重视很不够,如果平时能够多发惭愧心,这样就会戒得更稳定,正能量会变得更强大。邪淫之念会引发很多负面的念头,而惭愧心会引发很多正面的念头,所以应该要多发惭愧心。记得以前看过一篇文章,讲的是有一位修行人惭愧心修得很好,别人赞扬他,他心里感到很惭愧,口里也说:“惭愧!惭愧!”那篇文章给我留下了深刻印象,我也想做一个“知惭有愧”的修行人,我自己也意识到了,不管戒多久,我其实还是很浅薄,别人的赞扬只会让我深感惭愧。当一个人发惭愧心时,有一种特别的振动像波纹一样在他四周弥漫开来,那种感觉非常好,而当一个人发傲慢心时,也会有一种波动扩散开来,让人感觉很不舒服,而且很容易引起别人的反感和攻击。修惭愧心对于戒色是非常重要的,希望大家能够充分意识到这一点,做一个懂得惭愧的人。

\paragraph{坚持的力量}

一个推销大师给人演讲,一句话不说,就是拿个小锤子一下一下地敲打一口很大的钟,但钟纹丝不动,观众不解,很多人纷纷离场,过了很久,那个大钟开始晃动了起来。坚持下去就会看到改变,在改变之前,往往会有一个无变化的阶段。有的戒友一直在学戒色文章,虽然感觉进步不大,但是他还是在每天坚持,因为他知道坚持下去就会迎来顿悟。当学习到一定程度,突然就开窍了,那就像一次灵感的爆发,一下就悟明白了,小戒靠忍,大戒靠悟,而悟的前提是有一定的学习积累,就像开水一直在加温,到了沸点就会沸腾一样,我们要明白这个原理,坚持学习下去,到时自然会迎来觉悟的突飞猛进。戒色贵在坚持,中断魔是很可怕的,你一中断,过去那种势头就找不见了,很多人都是一曝十寒,缺少持久力。唯有坚持下去才能迎来显著的变化,绳锯木断,水滴石穿,恒劲太重要了。你每天都在吃饭菜,你吃下去的饭菜已经可以堆成山了,你从生下来就不间断地在吃东西,这其实也是一种坚持,生存本能的坚持。学习戒色文章也应该变成一种本能的需要,每天看看戒色文章,复习一下笔记,坚持下去,不要中断,这样对于保持良好的戒色状态异常关键和重要。有的人刚开始戒得非常不错,很快突破百天,但是后来就不行了,戒色厌倦期一来,他的状态就一落千丈了,养成良好的学习习惯,这样自然可以克服戒色厌倦期。

\paragraph{善行的力量}

积善可以让你的内心变得崇高和光明,而邪淫和其他恶行会让你的内心变得下劣和黑暗。积善可以增加正能量,正能量会带来快乐和安稳,邪淫会积攒负能量,负能量最终会导致痛苦和不安。秦东魁老师就说要建立上等能量场,怎么建立呢?多行善积德,多孝顺父母,不能亏孝,不能犯邪淫。你可以试着观察每个人的能量场,你会发现每个人的能量级别都不同,行善的人能量级别就高,既行善,又不犯邪淫,这样的人能量场就会变得很好。乐观、积极、向上、充满热情、希望与信念,有着纯净美好的笑容,有着慈悲宽广的胸怀,这样的人带有正能量磁场,和他交流的时候会让我们感受到快乐、向上、信任的感觉,和他在一起会感觉到安全、放松和愉悦,让你感受到生命的美好与意义,你很想多和他呆一会。而邪淫戾气重、猥琐无神、情绪多变、喜怒无常、容易悲观、畏惧、喜欢抱怨、嫉妒、看什么都不顺眼的人,带有负能量磁场,人一旦接触到负能量,就会本能地感觉到不舒服,甚至很厌恶的感觉。邪淫恰恰就会让一个人进入负能量的状态,邪淫会引发其他各种恶念,会让一个人的内心变得无比阴暗,变得更加自私自利,邪淫完全就是一种腐蚀!把人变成行尸走肉,进入畜生不如的状态。

前段时间看了《今日说法》的一个节目,叫《幻灭》,讲的是一个叫丁庆平的人,原来他是语文老师,后来下海经商,赚了大钱,发迹了,谁知富贵之后他开始过上了邪淫放纵的生活,并且染上了赌博恶习,最后生意一败涂地,资金链断裂,欠了下巨债,最后逃亡了加拿大,七年后被抓了回来,已经穷困潦倒,几乎一无所有。古人讲富贵不能淫,很多男人在富贵后就会乱来,最后真是造孽深重,报应惨烈啊!丁庆平以前的照片,那真是相貌堂堂、一表人才,后来邪淫堕落后,那副样子真的充满了负能量,显得极度猥琐和阴暗。行善积德之人,头上就像有一个温暖的太阳在照耀,而邪淫作恶之人,头上就像笼罩了一片浓厚的乌云,整个人的色调都是灰暗的。《欲海慈航》里的一个故事:王文恪公鏊,未第时,有美女夜奔之。王书于壁曰:美色人人好,皇天不可欺。拒之。后登鼎甲,为宰辅。我自己亲眼看到的一件事也印证了这一点,记得有一年过年,和老家的一位远房亲戚一起吃饭,那时我才十几岁,而他大概四十岁出头,和我爸爸同辈。吃饭的时候,他说有一次他去外地办事,住在宾馆里,晚上来电话说提供那种服务,被他严词拒绝了。当时他在吃饭时提起这件事,依然一副正气凛然的样子,我那时就觉得他很正直,对家庭的责任感很强,不在外面乱来。他说那件事时,家境还不富裕,后来过了几年他做起了石子生意,一种建筑材料,真的发达了,赚了几千万,同村好几个人也做这个生意,但是都没他那么发达。我想这和他拒嫖有很大的关系,人在做,天在看,对于拒绝邪淫之人,上天定会予以厚报。

埃迪·格里芬曾安排美女为姚明提供“服务”,但遭到姚明拒绝,姚明经受住了考验,而格里芬过于放纵自己的生活,为他日后的悲剧埋下了伏笔。格里芬遭遇车祸不幸丧生,年仅 25 岁。驾驶一辆 SUV 无视铁路警告标志,强行穿越护栏,结果与一辆货运列车迎头相撞,车祸现场惨不忍睹,尸体被严重烧毁,无法辨认,据说格里芬开车时正在看黄。姚明严以律己,而格里芬纵情声色,两人的命运也因此天差地别。古德云:“天律于淫最严,人祸于淫最惨,小则戕生,大则绝嗣,近则削其福寿,远则灾其子孙,阳则受国宪之诛,阴则干神明之谴。”不邪淫的人,才能守得住!心正则兴旺,心邪必衰败!犯了邪淫就等于开启了万祸之门!人是灵性与兽性的合一体,既可以变得充满灵性的光明,也可以变得充满兽性的黑暗,就看你开发哪一边,如果你开发自己的兽性,那真是祸莫大焉!

\paragraph{誓愿的力量}

愿断一切恶,愿修一切善,愿度一切众生,这个愿我每天都在发,早课一遍,晚课一遍。金刚非坚,愿力最坚!一遍遍发,越来越坚固,越来越不可动摇,比金刚石还坚固。星云大师说过:“发愿,是一切成就的根本。”不能发空愿,一定要起行,切实去做,要有长远心、坚固心。因为发了大愿,所以就会尽全力去做,我坚持到现在,和每天发愿密不可分,也是愿力支撑我走到现在,发了愿,我就觉得自己有责任、有义务去帮助戒友,也有一种很崇高的使命感,当然能够帮到别人也是我莫大的荣幸,我也希望每一位戒友都能超过我,我做垫脚石也没关系,希望大家齐心协力,一起带动中国戒色公益事业的发展,让更多的人认识到邪淫的危害。现在已经有一大批真正具有戒色正知见的戒友崛起了,他们将会带动更多的人戒色,一股势不可挡的趋势已经逐渐形成了,我们都是见证者,我们也是参与者,让我们不断传递正能量,坚持做下去,直到生命的尽头。

有的新人也发誓愿,但他发的是毒誓,戒不掉就剁这剁那的,完全是一味强戒的表现,这类誓愿往往没有多大效果,很快就会自己打脸。发誓愿也是有讲究的,像地藏王菩萨发的大愿:地狱不空,誓不成佛;众生度尽,方证菩提。这个大愿就是一个无私利他的誓愿,毫无自私自利之心。我们戒色最好也能发一个利他的大愿,这样就会有一个崇高的动机,到时你行动起来的力度就不是那种自私狭隘的小愿所能比拟的,发了大愿就会有一种很崇高的使命感,这种使命感非常重要,有了这种使命感,就可以让你戒得更坚定,你的戒色立场也会变得异常坚固。一遍遍发愿,就是在把你导向那个强大的能量场,人有善愿,天必佑之,你有无私利他的大愿,老天都会帮助你,如果你狭隘自私,自己戒了,不管别人了,甚至还嘲笑别人,这样的人肯定会再次破戒,因为德行不行。发了大愿,自然就会朝着那个方向努力,再加上长远坚固心,就会达到不可思议的效果。有的人也发过大愿,但是缺少长远坚固心,中途起了退心,这种人就是善根浅薄,经不起考验。当你以一种崇高的心态去行动,你的行动将会带有特别的能量和品质,誓愿的力量不可思议,发出大誓愿,勇猛地去行动,不是为了自己,而是为了拯救万千沉迷于邪淫的孩子,你有责任和义务站出来,劝导他们,帮助他们回归正轨,时刻带着这种无私利他的崇高心态去做事,你将会充满无穷的力量,你的眼神也会变得崇高而坚定,拯救他们,这是你的天职!这是你一生的使命!!!猛烈行动吧!!!

\paragraph{娴熟的力量}

练习成就完美,熟练才能精通,“起念 - 看黄 - 手淫”,这个套路太熟了,熟到后来,半梦半醒都会撸,完全下意识了。而现在我们要做到念起即断,则需要不断地练习,从生疏到熟练,从熟练到出神入化,要不断地练,一遍遍地反复练习。《庄子》云:“始臣之解牛之时,所见无非牛者;三年之后,未尝见全牛也。”眼中没有完整的牛,只有牛的筋骨结构,比喻技术熟练到了得心应手的境地。 断念从某种程度来讲,也是一种技法,高手已经达到游刃有余的程度了,怎么达到的?一个字,练!两个字:狠练!三个字:拼命练!!!高手出刀就是干净利落,绝不拖泥带水,而菜鸟发现太晚,已经跟着念头跑了一大段,才突然发现自己是在意淫,这时候已经晚了,所谓不怕念起,就怕觉迟!要做断念的熟练工,要不断提升熟练度!久练自化,熟极自神!苏炳添在 2006 年的成绩是 10.59 秒,到 2015 年的成绩是 9.99 秒,跑进十秒大关,他训练了九年时间,他教练说的“两年一小进,四年一大进。”专业运动员的高超水平来自于长期专业系统的训练,台上一分钟,台下十年功。看过一位 NBA 球员,在全黑的环境下投三分,十投八中!先是让他站好位置,然后灯全部关掉,让他投,在看不见篮筐的情况下,还投中了八个,不可思议,我想这和他长期练习投篮密切相关,他的出手已经养成了一种极度熟练的感觉,即使看不见,也可以投中。还有的球员在训练中可以连续投进几十个三分,手感极为火爆,这都是大量练习的结果,练习会让人越来越精通。断念也是如此,段位越高熟练度就越高,对断念的理解也会越深刻。大量练习会让断念水平直线提升,当然也要练习得法,要深入学习断念的理论知识,然后在练习中一遍遍地深入体会、细心揣摩,在千万次的练习中越来越强。练之既久,功夫到了,“刀味”便慢慢开始渗透出来,那股无形的杀气绝对让心魔颤抖,功夫深了,对念头的波动就会变得极为敏感,彼微动,我先觉,断念越来越快,邪念一上脑,在瞬间就解决了战斗。觉察就是最厉害的刀,这是一把杀念的刀,刀气纵横,所有出现在脑海中的邪念和图像全部被强悍的刀气撕碎!磅礴刀意,威武雄壮,站立于高山之巅俯视云海的戒者,宛若置身仙境,谱一曲荡气回肠的戒色传奇。

最后总结:

戒色是一门很深的学问,要进入戒色的高层次,必须要注重德行的培养与提升,我戒到现在,发现戒色的学问是非常深奥的,在低层次你不会觉得戒色有多深奥,但是到了高层次你会发现戒色其实是非常非常深奥的,要进入极稳定的戒色层次,就应该多学习圣贤教育,多汲取圣贤教诲的养分,这样才能进入更高的修道境界。很多人可以戒个一百多天乃至一年多,但是到了某个阶段就会自动败下阵来,究其原因,有两方面,一方面断念实战不行,另外一方面就是德行不够,一嗔怒、一抱怨、一骄傲,马上就破戒了。德行就像地基一样,地基稳固了才能越戒越好,戒得越久对德行的要求就越高。这季的十种力量,主要还是侧重于德行的提升,最后拼的就是德,见过不少有才、有冲劲、有天赋的戒友,但是德行跟不上,有才无德,最后还是一败涂地,戒色先修德,不管戒到哪个阶段,都要注重德行的培养。《太上感应篇》云:“是道则进,非道则退;不履邪径,不欺暗室;积德累功,慈心于物;忠孝友悌,正己化人。”这三十二个字非常之重要,是做人的根本,也是戒色的根本,一定要好好落实。

吴泽云先生:“人自赋气成形而后,最重者莫如生命。然未能养生,安知保命,既知保命,即能养生,此不易之理也。乃近世人心不古,风俗浇漓,其最足戕贼人之生命者,要惟色为巨,色犹刃也,蹈之则伤;色犹鸩也,饮之则毙。”又云:“迨至陷溺已深,精枯髓竭,志气因之堕落。耳目因之瞆聋,形骸因之瘠厄,人格因之卑下,而一切虚弱瘫痪之病,又复乘隙而丛生,以致一身无穷之事业,绝大之希望,均消归于乌有。”邪淫的危害非常之大,本质上是一种能量的耗损,日积月累,迟早会自食恶果,必须要力戒邪淫。戒色的十种力量,每一种力量都极其强大,如果你具备了全部的十种力量,那就像具备了十种顶级的装备,到时你就会变得所向披靡,战无不胜,心魔再也不是你的对手。

下面分享三首戒色诗歌。

\begin{poem}[少年的沉沦]
    \begin{multicols}{3}
        \begin{center}~\\
            是什么把纯真的少年 \\ 变成了猥琐的存在 \\ 一个隐秘的恶习 \\ 他没有告诉任何人 \\ 他相信了砖家 \\ 因为砖家的理论 \\ 给了他放纵的借口 \\ 一个疯狂的动作 \\ 抽搐着的身体 \\ 贪婪的眼睛 \\ 紧盯着屏幕 \\ 一管管的精华 \\ 从体内射出 \\ 少年日渐憔悴和枯萎 \\ 双眼无神且耷拉下来 \\ 显得特别颓废 \\ 父母搞不清他到底怎么了 \\ 只有他自己知道 \\ 他究竟干了什么 \\ 少年已经被心魔控制 \\ 他开始过一种双面的生活 \\ 表面努力装作纯真的样子 \\ 背地里却在疯狂手淫 \\ 一切都面目全非千疮百孔 \\ 少年用尽全身的力气 \\ 发出最后一声悲恸的呐喊 \\ 我要戒撸!!!
        \end{center}
    \end{multicols}
\end{poem}

\begin{poem}[撸管变形记]
    \begin{multicols}{3}
        \begin{center}~\\
            短暂的快感过后 \\ 就是满目疮痍 \\ 在泄精后 \\ 脸上一层光华 \\ 就消失了 \\ 又重新变得晦暗 \\ 他感到无望 \\ 一次次被心魔虐 \\ 他想摆脱怪圈 \\ 可是做不到 \\ 镜中人已然变形 \\ 微妙的变形 \\ 却显得极其丑陋突兀 \\ 脸凹了,颧骨突出 \\ 眼皮下塌,五官变形 \\ 看上去很不对称 \\ 看着特别怪异 \\ 无处不愤恨的狰狞面孔 \\ 扭曲的脸,堕落的灵魂 \\ 因为沉迷撸管 \\ 他丑化了自己 \\ 特别丑,特别猥琐 \\ 特别龌龊,特别恶心 \\ 连他自己都觉得恶心 \\ 他蜷缩在床角默默抽泣 \\ 任眼泪肆意横流 \\ 流过变形的脸庞 \\ 流过所有的丑陋 \\ 撸管的生活太憋屈了 \\ 唯有降伏心魔 \\ 才能做回最好的自己
        \end{center}
    \end{multicols}
\end{poem}

\begin{poem}[纯白的灵魂]
    \begin{multicols}{3}
        \begin{center}~\\
            他的眼神非常清澈 \\ 没有任何的欲望 \\ 倒映着蓝蓝的天空 \\ 是那么的纯净与美好 \\ 孩子盛开的笑脸 \\ 嬉戏的声音 \\ 在记忆的某个角落 \\ 依旧耀眼,依然鲜活 \\ 孩子们从童话的世界 \\ 跌落到了欲望的世界 \\ 失去了纯净的美感 \\ 失去了纯粹的喜悦 \\ 散去邪淫的阴霾 \\ 他还是那个单纯的孩子 \\ 拥抱着纯真的浪漫 \\ 目光柔和而纯净 \\ 散发着无尽的温柔与善良 \\ 有一种特殊的美感 \\ 在脸庞浮现出来 \\ 徜徉在无限的美好里 \\ 是那么的快乐与自由 \\ 让我们恢复纯白的灵魂 \\ 活出本然的完美与纯净 \\ 活出内在的崇高与光明
        \end{center}
    \end{multicols}
\end{poem}

下面推荐一本书。

\begin{book}[《吃茶去:与星云大师一起参禅》]
    星云大师精选禅宗公案两百则,从现代人的观点,帮助您汲取禅师灵活幽默的智慧,活出充实自在的人生。禅宗是中国佛教中一个重要、有特色的宗派,与中国传统文化结合紧密。赵朴初说过,谈中国的宗教文化离不开禅宗。禅宗如一股源头活水,曾一度为中国文化带来过活泼生机,对中国文人画、山水诗、文学乃至理学都产生了深刻影响,把中国文化带入了一个注重自然、和谐、灵性、气韵生动的崭新意境,古代的文人墨客基本都有在学禅,古人参禅、谈禅的风气很盛。禅宗是历史文化的宝藏,值得我们去挖掘、整理、研究和借鉴。禅宗影响深远,传播广泛,它东传韩国、日本,影响遍及东南亚,近代更向西欧、北美传播。乔布斯也受到过禅宗思想深刻的影响,最终体现在他对企业的战略思考上,对产品简洁唯美的设计上。森田疗法又叫禅疗法,也是汲取了禅宗的思想和智慧。我本人是修净土宗的,但对于禅宗的一些公案和开示有时也会看看,对提升自己的思想认识很有帮助。星云大师这本书挺不错的,一个个小故事给人很大的启发,网上有 PDF 版本,有兴趣的戒友可以下载阅读。

    李开复对话星云大师,以下是李开复书里的摘录:

    \begin{quote}\it
        那时候,我常常怨天怨地、责怪老天爷对我不公平,我从内心深处发出呼喊:“为什么是我?我做错了什么?这是因果报应吗?”我是天之骄子啊!我有能力改变世界、造福人类,老天爷应该特别眷顾我,怎么可能会把我抛在癌症的烂泥地里,跟一群凡夫俗子一样在这里挣扎求生?朋友看我很痛苦,特地带我去拜见星云大师,并在佛光山小住几日。有一天,早课刚过,天还没全亮,我被安排跟大师一起用早斋。饭后,大师突然问我:“开复,有没有想过,你的人生目标是什么?”我不假思索地回答:“‘最大化影响力’、‘世界因我不同’!”这是我长久以来的人生信仰:一个人能在多大程度上改变世界,就看自己有多大的影响力;影响力越大,做出来的事情就越能够发挥效应……这个信念像肿瘤一样长在我身上,顽强、固执,而且快速扩张。我从来没有怀疑过它的正确性。大师笑而不语,沉吟片刻后,他说:“这样太危险了!”“为什么?我不明白!”我太惊讶了!“我们人是很渺小的,多一个我、少一个我,世界都不会有增减。你要‘世界因我不同’,这就太狂妄了!”大师说得很轻、很慢,但一个字一个字清清楚楚。“什么是‘最大化影响力’呢?一个人如果老想着扩大自己的影响力,你想想,那其实是在追求名利啊!问问自己的心吧!千万不要自己骗自己……”听到这里,简直像五雷轰顶,从来没有人这么直接、这么温和而又严厉地指出我的盲点。我愣在那里,久久没有答话。“人生难得,人生一回太不容易了,不必想要改变世界,能把自己做好就很不容易了。”大师略停了停,继续说:“要产生正能量,不要产生负能量。”
    \end{quote}
\end{book}
