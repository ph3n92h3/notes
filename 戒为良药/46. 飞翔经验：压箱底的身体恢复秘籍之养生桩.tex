\subsection{压箱底的身体恢复秘籍之养生桩}\label{46}

\paragraph*{前言}

先分享一个树疗案例。

\begin{case}[树疗]
    我一直在树疗,马上两周了,每天至少半个小时的树疗,我抱的是银杏树,确实是有效果,就是刚抱完树,皮肤会细腻白皙,但是过一会儿就恢复了我以前的皮肤状态,我每天都要用电脑很长时间,跟这个有关吗?请问如何才能留住树疗后的白皙细腻?

    \textbf{分析} 这个戒友体验了树疗,选择的是银杏树,我现在做的是香樟树和银杏树,银杏树一般比较干净,也比较温暖,做完效果是比较好,效果最显著的其实是第二天晨起,当然刚做完效果也是立竿见影的,但是这个戒友的养生意识很差,久视电脑会得“屏幕脸”,是会抵消树疗效果的。久坐伤肾伤脾,久视伤肝,如果打游戏熬夜,那就更伤了,熬夜很伤气色,会导致内分泌失调,很容易出现黑眼圈或者痘痘加重等。所以,我们在做树疗的时候,一定要注意养生之道,并且严格控制遗精,遗精一次至少抵消三天的树疗效果。还有就是树疗时间要保证,最好能保持在一个小时以上,两个小时以上则更好,当然不一定要一次站两个小时,你可以做会树疗休息会,再接着做,自己灵活把握。没时间的戒友至少保持每次半小时,无聊可以听歌。另外,做树疗要注意安全,太野生的环境可能有危险,比如有蛇,还有就是不要在河边做,特别是没有护栏的河边,一不小心可能掉河里。选择树和选择环境很重要,尽量不要跑到荒无人烟的地方去,可以在公园人少安静处,反正人家也不认识你,时间长了脸皮也厚了,无所谓了,习惯成自然。做树疗效果是不错的,但也要看你有没有这个缘分,有没有时间,有没有适合的环境,有缘坚持练习者,一定会真实体验到树疗的神奇效果,完全自然完全免费的皮肤保养。
\end{case}

再分享一个戒色新人的答疑。

\begin{case}
    本人也算是撸龄很久的一个,有十年左右了,看了你的戒为良药后,确实感慨很多,我非常同意你的观点,戒掉是必须的,但是有几个问题想问你下。\begin{enumerate}
        \item 你说了你之前有前列腺炎,但是在戒掉一年之后彻底恢复了,那么你是从检查报告上得到的结果说你恢复了还是说你个人感觉症状恢复了?如果是去检查了的话,能不能把你的检查报告发上来让我们看下。
        \item 你说你之前有早泄和重度阳痿的情况,那么你目前来看也戒掉三年了,你在你的戒为良药那个文章中也说到了一年左右恢复了,请问这个症状恢复的依据是什么?既然三年多没有破戒,那么怎么知道功能恢复了?
        \item 我想知道在什么症状的情况下是需要一直吃药 + 戒色的,在什么症状的情况下只是戒色 + 锻炼 + 食补就能好的?
    \end{enumerate}

    \textbf{回答如下} \begin{enumerate}
        \item 坚持戒色养生一年多,前列腺炎的症状全部消失了,曾经复查过前列腺液,指标已经完全正常。坚持戒色养生,积极锻炼,慢前是会慢慢恢复的,吧里有缓解恢复的帖子,你是新人,多看增加阅历,就不会再有怀疑。
        \item 恢复依据从晨勃质量即可看出,不一定要试,如果觉悟不高,很可能会一发不可收拾,重新开始沉迷撸管。我在《戒为良药》里面有讲到的,这个问题我好几季文章都讲到,建议你仔细看看。另外,早泄阳痿恢复的案例也是有的,上上季我一张截图就是一个戒友早泄恢复的真实案例,早泄阳痿恢复的案例还是不少的,家兴哥阅历也很丰富,他也知道很多人恢复了,他自己也有恢复的经验。坚持戒色养生,早泄阳痿是有望恢复正常的,有的戒友会去试,但我不会去试,因为我不想为了试而去破戒。
        \item 症状严重或者症状持续是需要积极就医治疗的,然后三分治疗,七分养生,彻底戒撸是必须的,否则很容易复发。
    \end{enumerate} 加油!新人疑问多,很正常。如果你有很深厚的阅历,这些都不是问题。

    \textbf{分析} 新人难免有诸多疑问,其实这三个问题我已经回答过无数次了,很多新人觉得前列腺炎是不可恢复的,因为他们阅历有限,没有遇见真正恢复的人,没有遇见不代表不存在。慢前患者真正彻底戒撸的有几人?都是用药缓解或者暂时痊愈了,之后就是继续撸管,然后就又复发了,久而久之,就得出这样一个结论,那就是:前列腺炎是终身的,是无法治愈的。其实这种观点是一种偏见,有痊愈的人,只是你没遇见。你没遇见,就说不能痊愈,实属以偏概全。很多人是戒撸了,但不注意养生,遗精频率也无法控制,结果就是恢复不尽理想,戒了一年多还是无法恢复,这其实就是恢复上做得不好。真正痊愈的人,必须是彻底戒撸,严格控遗,加上有自己一套系统的养生恢复方法,这样才有望恢复,戒色和养生这两方面都要做到位才行。很多戒友撸久伤深,恢复相对比较慢,这时候需要保持耐心,做好持久战的准备,在养生方面下足功夫,面对症状的反复期,也要有足够的心理准备。说实在的,很多戒友都不是在彻底戒撸,意淫也控制不了,养生意识也没有,这样要谈恢复,犹如痴人说梦。戒色以及恢复的思想误区实在很多,如果不弄清楚,结果就像进入迷宫,可能一辈子都转不出去了。

    晨勃质量的提升是早泄和阳痿恢复的一个表征,但有的戒友几乎没晨勃,但是他也没出现早泄和阳痿,这种情况也是存在的,他没晨勃其实肾气已经不足了,如果再透支下去,性功能很可能会出现问题。晨勃质量是浮动的,和季节、饮食、锻炼等因素都有关系,有一个相对稳定且高质量的晨勃,是性功能恢复的一个指标,大家可以参照这个指标去观察恢复情况,如果你一直想着去试,那是心魔在怂恿你,如果你仅仅是为了性功能而戒,那么肯定无法长久,因为你会一直纠结去试,越试越破戒,根本无法戒色成功,何谈身体的恢复?一般坚持戒色养生一段时间,晨勃消失的戒友会发现自己的晨勃有所恢复了,这其实就是肾气恢复的表现。刚开始戒色,晨勃可能会一度消失,属于戒断反应,有些新人不懂,就容易产生恐慌而破戒,还以为禁欲有害,其实这是戒断反应,只要坚持戒色养生,晨勃是会慢慢恢复正常的。当晨勃质量很好了,那就说明你性功能恢复得相当可以了,JJ 就像气球,就是靠肾气充满的,从晨勃质量完全可以看出你肾气恢复的情况,而性功能的恢复是和肾气的恢复密切相关的,所以,从晨勃质量即可看出性功能的恢复情况,高质量的晨勃是肾气恢复的一个金指标,而且这种高质量的晨勃不是那种昙花一现的,需要一定的稳定度。就像你偶尔碰运气打了十环,那是你真实水平吗?有人打十枪,七枪以上都是十环,而有的人打十枪,只有一枪十环,稳定度不行。

    有些人吃了助长性欲的食物或者药物,马上出现了好晨勃,然后不吃,马上就又不行了。这就像运动员打兴奋剂一样,打了兴奋剂跑得很快,不打就又慢下来了,并不是自己真正的实力。肾气是需要慢慢养回来的,并不是靠几副中药就能马上补回来的,姚明不是一个晚上长到两米二六的,是一个缓慢积累的过程。中医讲:服药百裹,不如独卧。吃一百副中药,不如自己睡一个晚上。意思就是自己睡觉不和老婆同房,你一同房,一百副中药的效果都泄掉了。在发育期,很多人可以天天都高质量晨勃,随着年纪的增加,晨勃频率和质量都会不同程度下降,甚至晨勃会彻底消失,一般坚持戒色养生一段时间,晨勃会有所恢复的。年纪大了,肾气不可能像十几岁时那么饱满了,自己一定要注意保养。

    关于我以前写的文章,我说恢复的就是恢复了,本着实事求是的原则,没必要骗大家。比如白发、脱发和慢前都恢复正常了,现在只有几根白发,原来真的非常多,不算头顶后面,就两侧就各有几十根,脱发完全恢复正常,也没有慢前的症状表现。慢前我现在还是很注意保养的,因为久坐熬夜、劳累或者遗精后,还是有可能会复发的,并不是一劳永逸的恢复,因为伤精的方式不止撸管一种。就我目前来说,恢复情况还是比较稳定的,戒到现在,慢前基本没复发过,因为我严格控遗,也未破戒,是彻彻底底的戒色,连意淫也戒掉了,另外我很注意养生之道,这样就把复发的概率降到了最低。我的鼻炎问题较之以前恢复很多,很少犯鼻炎了,和艾灸之功也是分不开的,痔疮也通过艾灸治愈了,这些都恢复了,鼻炎虽未彻底断根,但已经很不错了,较之以前已经好很多了。恢复是一个系统工程,我在恢复方面下的功夫很多,我能恢复成现在这样,我自己也相对比较满意,我把经验分享给大家,希望大家也能早日恢复到满意状态。

    有些戒友可能会觉得要医学证明才是恢复健康,其实这是一种误解,很多功能性的问题根本检查不出来,就像很多神经症患者,已经生不如死了,已经天天想自杀了,躯体症状多得很,但是他去医院做检查,什么也查不出来,家人甚至怀疑他装病。有的戒友去查前列腺液,完全正常,但是尿频尿急的症状却很明显,有些功能性的问题真的不是检查可以看出来的,医学仪器检查是有一定局限的,正因为这个原因,引起很多医患之间的误解和矛盾,患者自己觉得有病,而医生却说你没病,因为检查报告是正常的,你怎么说?毕竟人家是医生,你也不好反驳。后来不少患者转投中医,中医一把脉就知道你身体哪里失调了,配合中药调理和养生,身体就慢慢恢复正常的。中西医各有所长,不要一味迷信检查报告,检查正常不代表完全健康,很多问题是检查不出来的。

    很多新人戒色信心不坚定,和觉得自己恢复无望是很有关系的,因为看不到恢复的希望,所以就会自暴自弃地撸管,他对别人恢复感到怀疑,感到不可思议,其实是他自己孤陋寡闻、阅历尚浅。还有的戒友一出现症状反复期,马上就动摇了,以为自己无法恢复,结果就又破戒了。出现这种情况,一方面是觉悟还不行,另外,就是一开始信根就不坚固。信根不坚固,那就很容易产生退心。信心是成功的源泉,你不相信自己能恢复,怎么有动力坚持下去呢?

    很多戒友都在“试”这个问题上栽了大跟头,包括很多资深戒友,当头脑中出现了“试”的念头,比如试定力和试性功能。一旦出现了这类念头,你一定要格外警惕了,因为这是心魔在怂恿你了,怂恿你去试,心魔异常狡猾,特别擅长怂恿。很多戒友因为觉悟不够高或者放松警惕,一旦出现了试的念头,他马上就去试了,一试就会重新掉入撸管陷阱不能自拔,特别容易一发不可收拾。
\end{case}

附早泄恢复案例:

\begin{case}[早泄恢复]
    2012 年 12 月 20 号左右我开始关注本吧,那时候只能坚持三分钟,本人已婚,最严重的时候秒泄(各位别笑话我)。心灰意冷地过了一段时间,来到本吧后知道了是因为自己撸多了,就开始戒撸,中间破过三次,不过恢复得还算好,现在已经完全恢复正常了。谢谢本吧的各位前辈,谢谢戒色吧,我会坚持戒下去。
\end{case}

下面步入正题。这季即将分享我压箱底的恢复秘籍,对于伤精患者的身体恢复很有好处,特别是神经症戒友,希望你们不要错过这季的内容。

在很早的文章里,我就建议大家站桩或者打坐,养生要动静结合,方合大道。静功是非常重要的,光动不静,有失偏颇,光静不动,也不符合养生之道。这季就讲讲静功的重要性,主要讲站桩,我推荐的养生桩是高位桩,不是四平马步,也不是蹲墙功。我之前推荐的是武国忠的养生桩。下面引用武国忠一段文字:

\begin{quote}\it
    我从小练习中国传统武术,练的是大成拳。这个拳没有什么招数和套路,就是“两手往胸前一抱,静静地站着”。这个动作在武术里叫做“站桩”,我把这个源自武术的方法运用到养生治病上,称之为“抱住健康”养生法。至今为止,我发现,这个集养生、中医、武术于一体的保健方法,居然是目前所有锻炼身心的方法中最简捷,而且见效最快的一种。大成拳的创始人王芗斋先生,在七八岁的时候,身体极度虚弱,得了顽固性哮喘,这种病在当时是没有什么好办法治的,万般无奈时,著名的形意拳大师,武林中有“半步崩拳打遍天下”美誉的郭云深先生把形意拳中的不传之秘——站桩养生功传授给了他,王芗斋先生就通过练这种养生功来调养身体,不仅病好了,而且身上不知不觉有了很深的功夫。练“抱住健康”养生法时无须意守丹田,没必要摈除一切杂念,多想点令人愉悦的事吧,因为,心主喜,微微的喜悦比任何养心的药都好。这时,我们全身放松,无忧无虑,您看吧,没多久,您就会尝到什么是身体的大欢喜。(武国忠《活到天年》)
\end{quote}

武国忠这个养生桩效果是很不错的,其实它是武术里的一个桩功,兼具养生和武术的功能,养生界亦有很多人在习练此桩。中国传统武术的精髓就在站桩上了,“要把骨髓洗,先从站桩起”,百练不如一站。站桩在武术界是深受重视的,用极度重视来形容一点不为过。我以前的文章也一直向戒友推荐这个桩功,有不少戒友都收到了良好的效果。但这个桩功有一个累人之处,那就是手要架着,感觉像抱一个气球,要不紧不松,太紧球会炸,太松球会掉,就是要找那种不紧不松的感觉。武国忠的这个桩法当然很好,不过这季我要推荐另外一个养生桩,效果也是非常之好,就是子午养生桩,大家看下面这段文字:

\begin{quote}\it
    子午门内养功站桩:两小臂抬于腰间,与上臂成 90 \unit{\degree},两小臂平行,左腿向左横跨一步与肩同宽,沉肩坠肘,含胸拔背,头正身直,百会、会阴成一直线,下颌微收舌顶上腭,嘴唇微闭,全身放松,意守丹田,逆腹式呼吸,逐步达到呼吸细长、均匀,站桩时间,根据年龄与身体状况而定,由短而长,争取每次达到三十分钟以上。首先劳宫穴开始发热,发麻跳动至丹田发热,即有气感,只要练功者身体素质很好,练功时间长,就能在短期内打通全身大小经脉。练子午门内养功三日后有气感,十日后感到精力倍增,百日后可内气外放,外气内收。
\end{quote}

养生桩属于内养功,属于内壮。而力量训练属于外壮,不少练健美的人其实是外强中干,外面看着肌肉挺吓人,其实里面很虚弱,强壮不代表健康,最近有消息,曾经的全国健美冠军龙云雷得白血病了,原先他是先中风,后来又得白血病,而他只有 35 岁,命运真的很悲惨。你能想象强壮如牛的人中风吗?你能想象力量是你数倍的人得白血病吗?记得前几年北京有个健美冠军猝死,叫鲍普成,也就 36 岁。所以强壮并不代表健康,但是强壮的确让人有一种健康感,这是一种表面的感觉,就像有的苹果表面看着好好的,但是打开以后发现里面都烂了。表面现象是具有一定欺骗性的,力量大身体强壮,不一定能长寿,再举个例子,力量界的太阳神保罗安德森,深蹲超过一千两百磅,只活了 62 岁,深蹲五百公斤的人只活了 62 岁,有很多老头深蹲不超过五十公斤,轻轻松松活过八十岁。所以养生真的很关键,运动也要注意适可而止,运动分:养、耗、伤三种。养生功法属于养,出汗的有氧运动乃至力量训练属于耗,耗过头了,就是伤。中医:大汗伤阳!汗出津伤!年纪轻,伤得起,感觉不到。如果身体比较虚,应该选择不出汗或者微汗的散步,还有就是多练习养生功法。等身体恢复得差不多了,可以适当做些力量训练。

说起我与子午养生桩的结缘,那就比较早了,记得我初二时在一本杂志上看到介绍子午养生桩,属于梁山功夫的范畴。当时上面写到得气快,而且精力倍增,那时我就很感兴趣,然后通过邮购得到了“梁山功夫”一书,书里面的内容还是很多的,但我其他都没练,就奔着养生桩练习。这个桩法手不用架着,而是双手手心朝上,握空拳收于腰间,比较简单的一个姿势,而且也是高位桩,微蹲即可,但要注意膝盖不要超过脚尖。那时我练习的确是得气很快,特别是手上感觉很明显,精力的确变得很好,精力倍增绝非虚言。

大家可能会觉得养生桩真的有这么神奇吗?其实人体自有大药,可惜很多人根本不知道。这季我就把这大药和大家仔细讲讲。

\begin{quote}\it
    叩齿后,生唾液,唾液为肾津。一口唾,分三十次咽下。一口唾,等于一盒六味地黄丸。(曲黎敏)
\end{quote}

人体大药就是嘴里的唾液,但不是一般的唾液,是叩齿后分泌的唾液,叩齿后的唾液对人体很有益处,曲黎敏说相当于一盒六味地黄丸。而我说的大药不是叩齿后分泌的唾液,是叩齿唾液的升级版,就是舌抵上腭分泌的唾液,舌抵上腭即舌头抵住上腭部位。气功学中认为这是沟通任督二脉的桥梁,俗称“搭鹊桥”。中医学认为,督脉循背,总督周身阳脉,为阳脉之海;任脉沿腹,总任一身阴脉,为阴脉之海,两脉各断于上腭和舌根。舌抵上腭即可沟通任督二脉,使全身经络接通,上下之气通畅。常练此功,可疏通气血,条达经络,清爽头脑,强健体质。

金津、玉液二穴,藏津多液,为医家所共知。二穴所藏,俗称“华池之水”,系于肺而根于肾。其穴位在舌下之左右,舌为心之苗,足少阴肾经挟舌本,足太阴脾经散舌下,故二穴所藏,非独与肺肾有关,且能与脾阴互滋,与肾水互济,引肾水上抑心火,合脾土同养肺金。

程国彭谓此“华池之水”引而咽之,能治真阴亏损、阴虚火旺诸证,并赞其是“治阴虚无上妙方”。方载《医学心悟》,方法是:常以舌抵上腭,令华池之水充满口中,乃正体舒气,以意目力送至丹田,口复一口,数十乃止,每日二至三次,称此为“吞津液”法。此是引华池之水以缓火刑,取其坎离交媾、水火既济之意也。\textit{所有自来肾有久病者,可以寅时面向南,净神不乱,思闭气不息七遍,以引颈咽气顺之,如咽甚硬物,如此七遍后,饵舌下津气无数。(《素问》)} 此是引华池之水以滋肾阴,循金水相生之法也。上述舌抵上腭,引舌下津而咽之者,即金津、玉液二穴所藏。而此种引咽津液之法,其源于丹家,原意本在养生,后引其治病,既合乎医理,又顺应自然,法简而效彰。余留意观察,凡健康婴幼儿,舌常抵上腭、阴囊常紧缩,此乃阴阳互生,水火既济之外征也。舌为心之苗而藏金津、玉液;前阴为肾之窍而寓真气元阳,心火肾水,阴升阳降,源源不竭而生化无穷,故婴幼儿生机较成人旺盛。

这口“华池水”的威力真的非同小可,它就是人体大药,对于身体的恢复,对于神经症的恢复极有好处,见效很快,精力提升显著。

道教常称津液为玉液、玉浆、醴泉、灵液等,认为此液由炼气而产生,是五脏之精华,甜美清香。若人们勤加修炼,漱津咽液,那么就可以去病防病,强健身体。\textit{玉池清水灌灵根,灵根坚固老不衰。(《黄庭经》)} 《外景经》也很重视津液的作用,如“玉池清水灌灵根,审能修之可长存……玉池清水上生肥,灵根坚固老不衰……津液醴泉通六腑,随鼻上下开两耳,窥视天地存童子,调和精华治发齿,颜色光泽不复白”。

关于戒色养生,《太平经》主张“人欲寿,当爱气尊神重精”,《内景经》也强调说:“急守精室勿妄泄,闭而宝之可长活。”《外景经》也强调固精、宝精的意义,如“长生要慎房中急,弃捐淫俗专子精,闭子精门可长活……急固子精以自持,精神还归老复壮”。

我一般在打坐或者站桩时,舌抵上腭,不出五分钟,华池水就下来了,和一般唾液的味道不同,有甜美清香之感,然后咽下,相当于“吃仙丹”了,这真是人体自产的免费仙丹,可惜很多人根本不知道,还在向外求药,其实自己有个大宝贝,却不知道。站桩或者打坐时,刚开始练习会妄念纷飞,可以采用数息法摄念,关于守窍问题,子午养生桩是说守下丹田,而我自己则是守上丹田,也就是眉心位置,守下丹田是容易遗精的,睡前打坐也容易遗精,而且守法也有讲究,不能强守,而是“知而不守,先存后忘”。千万不要强守,强守是容易出偏的。或者你什么也不要守也可以,就专注于数息即可。一口华池水可以分几口咽下,但咽下也有诀窍,配合十六字锭金,就完美了。

何谓十六字锭金?

\begin{quote}\it
    道家气功养生十六字锭金:一吸便提,气气归脐,一提便咽,水火相见。
\end{quote}

这是道家十六字锭金产生于宋元期间,最早见于明朝初期冷谦《修龄要指长生一十六字诀》。从东汉末道家的产生到这十六字锭金的出现,其间一千余年。

十六锭金又称李真人长生一十六字诀,是道家养生的代表性功法。此十六字为“一吸便提,气气归脐,一提便咽,水火相见”。明代冷谦的《修龄要旨》初录此法,并誉它为至简至易之妙诀。其后《赤凤髓》《遵生八笺》《脉望》《养生秘录》《医方集解》等均介绍了此法。

明朝养生家冷谦在《修龄要指》中对此的解说有不同,很实在,称此十六字锭金为“长生一十六字妙诀”,“乃至简至易之妙诀也”。他指出这个妙诀比较实用,随处可行,人人可练,“久久行之,却病延年”。并通俗地介绍了修炼方法。今录于下:“口中先须漱津三五次,舌搅上下腭,仍以舌抵上腭,满口津生,连连咽下,汩然有声。随于鼻中吸清气一口,以意会及心目,寂地直送至腹脐下一寸三分丹田元海之中,略存一存,谓之一吸。随用下部轻轻如忍便状,以意力提起,使归脐,连及夹脊双关肾门一路提上,直至后顶玉枕关,透入泥丸顶内,其升而上之,亦不觉气之上出,谓之一呼。一呼一吸,谓之一息。气既上升,随又似前汩然有声咽下,鼻吸清气,送至丹田,稍存一存,又自下部如前轻轻提上,与脐相接而上,所谓气气归脐,寿与天齐矣”。

另外站桩是可以治疗频遗的,有频遗困扰的戒友也可以多尝试站桩,多吃“仙丹”,人体自有大药,不要错过了。田诚阳道长说过,站桩可以强身补气,气足自可摄精不遗。

关于华池水的作用,下面摘录了一些文字供大家学习参考:

\begin{itemize}
    \item 咽津功用。咽津养生在我国历史较悠久。古人称之为“胎食”的,指的就是咽津。\textit{习漱舌下泉而咽之,名曰胎食。(《汉武内传》)} 历代医家认识到口中津液是大有益于人体健康和长寿的宝贵物质,应该咽下。
    \item \textit{玉泉者,口中唾也。朝旦未起,早漱津,令满口乃吞之。(唐代医学大家孙思邈《千金要方卷二十七》)} 传统医学有关咽津养生的论述不少。《灵枢天年》、《素问刺法论》等都讲到“津液布扬”、“饵吞下津”对人体健康的重要作用和咽津的方法。先秦《子华子》讲到:“荣卫之行,无失厥常,六腑化谷,津液布扬,故能久长而不弊。”1973 年底马王堆三号汉墓出土的《天下至道谈》,其中阐述房中养生的“八益”,就谈及“吞服津液”的养生功能。这些都说明我国古人很早就发现了人体内唾液的功用,这与现代科学的研究论述是一致的。
    \item 道家养生家对练功过程中产生的唾液,称之为“金液还丹”,要“淙淙咽归丹田”(见《金丹大要》)。这种唾液有许多好听的名称:金浆玉醴、灵液、神水、醴泉、金津、玉液、金醴、玉津、玉池清水、玉浆、舌下泉等等。古人认为吞服这种津液有益于养生保健。明朝医药学家李时珍,在其所著《本草纲目卷五十二》中,对“口津唾”作了如下的说明:“人舌下有四窍,两窍通心气,两窍通肾液。心气流入舌下为神水,肾液流入舌下为灵液。道家谓之金浆玉醴,溢为醴泉,聚为华池,散为津液,降为甘露,所以灌溉脏腑,润泽肢体。故修养家咽津纳气,谓之清水灌灵根。人能终日不唾,则精气常留,颜色不槁;若久唾,则损精气,成肺病,皮肤枯涸。故曰远唾不如近唾,近唾不如不唾”。
    \item 咽津养生既有悠久历史,同时也符合现代科学。现代医学认为,人的唾液其成分除了水分之外,还含有淀粉酶、溶菌酶、粘液蛋白、氨基酸以及少量钠、钙等多种物质,具有多方面的生理功能,有利于消化、养生、强身。近年来科学家还发现唾液有消毒、解毒、抗癌和杀灭艾滋病毒的作用。
    \item 我国古代养生家非常重视咽津养生。道教徒较早地注意到咽津的养生功能。晋朝《黄庭内景经》多处讲到唾液的功能,《黄庭内景经》讲到“口为玉池太和宫,漱咽灵液灾不干,体生光华气香兰,却灭百邪玉炼颜,审能修之登广寒”。此五句意思是说,口是产生唾液的地方,漱咽唾液,能使病害不染身,预防疾病,达到延年益寿的目的。
    \item 关于咽津养生具体方法,在历代一些养生导引著作中都有叙述。《遵生八笺延年却病笺》中记载《八段锦导引法》就记有咽津方法:“赤龙搅水津,漱津三十八,神水满口匀,一口分三咽,以候逆水上,再漱再吞津,如此三度毕,神水九次吞。咽下汩汩响,百脉自调匀。”清代尤乘《寿世青编》说,咽津“将舌舐上腭,久则津生满口,便当咽之,咽下然有声,使灌溉五脏,降火甚捷,咽数以多为妙”。
\end{itemize}

\paragraph*{总结}

希望大家都能吃上“仙丹”,人体自有大药,不要抱了个大宝贝却毫不知情,我现在每天都吃“仙丹”,我极其重视嘴里这口“华池水”,这口仙丹对于伤精的各类症状都有良好的恢复效果,对于神经症和脑力的改善,效果也非常显著,精力倍增,但贵在坚持,从开始的每次 15 分钟,慢慢延长至每次半小时以上,最后能达到每次坚持 45 分钟就可以了。我那时神经症能恢复,和吃仙丹密不可分,我那时对药物彻底绝望了,而且很花钱,最后我就靠“吃仙丹”以及其他养生之道,坚持了半年多就感觉恢复了很多,坚持一年多,神经症就消失了。
