\subsection{戒色修心勇士之道}

\paragraph*{前言}

前段时间有戒友提到了“泰国杀妻骗保案”中的嫌疑犯电脑中有 158 \unit{\giga\byte} 的色情资源,表面上沉默寡言、老实本分,对妻子关心备至,另一面却是极度扭曲、被邪念所驱使的状态。一位戒友说:“这个人应该是邪淫得已经没人性了!真的是这么回事,我之前撸得特别严重,对亲人都非常冷漠!”另一位戒友留言:“邪淫确实会让人丧失人性,变得自私冷漠。这个我也有亲身体会,大家一定要戒除邪淫,找回那个阳光自信的自己。”还有两位戒友的留言:“邪淫的人基本没什么情感可言,已经是冷血动物,只知道发泄兽欲,其实比禽兽都不如”;“确实如此,邪淫的人就如同行尸走肉,只知道满足自己的欲望,根本不在乎亲情和友情,深有体会!”

这个案件我也关注了,千万保额和 158 \unit{\giga\byte},这两个数字让人印象深刻,之前上海杀妻藏尸案我也关注了,发现这两个人都是邪淫很重的人,本来都有幸福的家庭,都不知道珍惜,沉迷在色情邪淫的世界里不能自拔,走火入魔,邪淫泯灭了人性,真的很可怕。当一个人沉迷邪淫时,一定会变得自私、冷漠、烦躁、易怒、怨恨,变得具有攻击性,充满戾气和负能量,一定会进入这种状态,和家人吵架,和朋友疏远,只知道发泄自己的欲望,根本不顾别人的感受,非常自私和冷漠,有些人为了满足自己的欲望,不惜去犯罪。其实大家都体会过那种被邪念驱使的状态,在那种状态下会做出各种荒唐的事情,丧心病狂,丧失人性,为了满足自己的欲望,什么事都干得出来。古人讲万恶淫为首,真的一点没错,邪淫是必须要戒除的,在性方面要格外慎重和自律,绝对不能乱来。现在很多人的电脑或者手机里都存有不良资源,这是非常普遍的现象了,158 \unit{\giga\byte} 也许对有些人而言,还算少的,他们有好几个 \unit{\tera\byte},在这个色情泛滥的时代,我们应该来一场“删黄运动”,号召大家一起删除电脑、手机、移动硬盘里的不良资源,真正认识色情的危害,自觉远离沉迷色情的生活,让生命恢复纯净与光明,让内心的良知复苏,让人性回归。

下面分享一些案例。

\begin{case}
    飞翔哥,明天就是我戒色满两年了,中间一次未破。之前戒了好几个一年多,都没有突破,通过不断地学习,终于要过了两年这个坎。感恩您,飞翔哥。学习到现在我感觉我也才刚刚入门,真的,之前一直在外围,不知观心断念的重要性,现在终于有体会了,原来真正能断念的是自己的觉察力,断念水平的提升就是要不断提升自己的觉察力。看尽世间的一切不如看看自己的心,看住自己念头就能做自己的主人。我感觉找到了方向,我一定要更加努力提升自己的觉察力,希望有一天也能平稳安住于纯粹的觉知,做自己的主宰。

    \textbf{附评} 这位戒友戒得很不错,他终于领悟核心了,那就是提升觉察力,觉察力就是一种发现念头的能力,也就是观照力、观察力,观察的对象是自己的内心活动,是自己的念头。有很多戒友都在问,如何提升觉察力,通过练习断念口诀,觉察力就能逐步得到提升,刚开始练习时,很容易跟着妄念跑,一边念诵口诀,一边跟着妄念跑,胡思乱想,发现后就拉回来,刚开始被带跑的次数很多,很容易陷入念头里,反应也很迟钝,被带跑好一阵才发现,随着一次次练习,一次次拉回来,觉察力就得到了提升,拉回来的速度会变快,觉察力会变得敏锐和犀利,妄念一出现,马上就能看见。通过练习断念口诀,专注力和觉察力都能得到显著提升,关键是坚持练习,在练习过程中很可能出现暂时的退步,这很正常,有时退步是要进步的征兆,就像潮水先后退再前冲一样。练习到一定程度,你会发现一个现象,那就是念头一出现,你马上就能觉察到,那个速度越来越快,感受也越来越细腻入微,达到微动即知的地步,对念头有高度的敏感。练习断念口诀就是为了提升你的觉察力,觉察力强大后,念头就不是你的对手了,一觉即灭,刹那间就解决了战斗。当练习达到很高水平后,你会发现还能继续进步,还能做得更快、更好。断力很强大,综合觉悟也比较完善,立场也非常坚定,德行也很好,懂得谦虚,懂得警惕,这样心魔就很难攻破你了。越戒到后来,就会越来越深知观心断念是真正的实战核心,不管你做了多少,最后实战就看那一下,那一下不行,肯定会破戒,为何部队要大抓实战化训练?因为纸上谈兵不行,必须立足于实战,努力提升实战的战斗力,不能来假把式、花架子,否则总有一天要付出惨重的代价!这位戒友说得很好:“看尽世间的一切不如看看自己的心,看住自己念头就能做自己的主人。”观心是最殊胜的,很多大德都在强调观心,观心是基本功,是值得一辈子修炼的功夫,观心既是基本功,又是最高深的功夫。\textit{我不怕会一万种招式的人,我怕把一种招式练一万遍的对手。(李小龙)} 把观心断念练到精、绝的程度,练到极快、极狠、极具爆发力,邪念再上头时,你就能在极短的一瞬战胜之,从而立于不败之地,真正主宰和统治自己的内心。不要尝试去压念,而是要觉察!压念就像压弹簧,会产生一个很强的反作用力,越压念头越多,越想让它不起,它起来得越猛,让人感觉很挫败,如果你学会觉察了,就会发现念头上来时,一觉、一知、一看,念头就没了,消失了,被消灭了,这才是正确的实战感觉,这个实战感觉是可以越练越强的。有的人悟性高,练习勤,也许几个月就能熟练掌握这个感觉,有的人则需要一年以上,只要坚持学习和练习,肯定会开窍的,一开窍,实战水平就能得到极大的提升,到时你的战力完全可以碾压心魔。
\end{case}

\begin{case}
    我从事的是个光荣且令人尊重的职业,起因和经过就不细说了,大家都如出一辙。我撸了有十来年了,中间除了有半年有点事情,这十年间几乎每天都在撸,最多也就保持四天不撸,但大多数时候都是一天到两天一次。我在五年前查出患有痛风,这五年间我也在饱受病痛的折磨,我的尿酸要高出常人两三倍,有八百多,由于这期间一直在吃止疼药,加上熬夜、不经常运动和手淫,肾的亏损已经相当严重,半月前被查出慢性肾功能不全,现在每天打针、吃药。一开始扎肾康注射液,那一瓶就三四百元,大概四五天的时间我就花了六千多(这不是危言耸听)现在虽然不注射肾康,但每天光打针吃药的开销也在四百元以上。而且这五年我也得过乱七八糟的病,都是疼痛难忍。在两点多的时候,我又犯痛风疼醒,疼得实在不知道干嘛,于是又开始手淫,一直到现在还没有睡。和众多撸者一样,每次撸完都会后悔,发心戒色,但回过头又变回原来的自己,甚至有时会边看戒色文章边想淫秽的画面。因为刚才躲在被窝里哭了,所以我决定发篇文章,我也不知道该说自己什么好,即使我此刻下定决心,我可能转头又变卦,但正因如此我更要戒,为了身体和前途命运!这是我的案例,请大家引以为戒,请师兄开导。

    \textbf{附评} 这位戒友虽然有一个光荣且令人尊重的好职业,可惜因为十年的纵欲导致得上痛风、慢性肾功能不全,而且陷入了恶性循环,疾病发作时还撸,这样对身体的伤害更大,已经肾功能不全了,真的不能再撸了。纵欲肯定要还的,迟早的事情,病来如山倒,当疼痛一阵阵袭来,真的难以忍受,痛不欲生,度日如年。我之前收录过一个痛风的案例,痛风这个病和纵欲是有一定关系的,痛风的中医辩证分型:风湿热痹、风寒湿痹、痰瘀痼结、脾肾阳虚。有老中医明确指出,痛风就是由于长期纵欲、饮酒、饮食不节造成肾损害而得的病。喜饮酒、赴宴,喜食富含嘌呤、蛋白质的食物,使体内尿酸增加,排出减少。有医生统计,筵席不断者,发病者占 30\%,常吃火锅者发病也多。这是因为火锅原料主要是动物内脏、虾、贝类、海鲜,再饮啤酒,自然是火上浇油了。几种容易得痛风的人群:1. 长期有手淫习惯,包括其他方式的纵欲;2. 饮酒无度,酒后纵欲;3. 有痛风家族史;4. 饮食无节制,胡吃海喝,大量进食肉类、动物内脏等高嘌呤食物的人;5. 肥胖人群;6. 精神压力过大,经常熬夜。一位痛风患者说:“我发现我每次发病都跟纵欲过多有关系。”也就是纵欲过后就发作了或者症状加重了,痛风和纵欲是密切相关的,当然也和其他多种因素有关,现代社会色情泛滥,纵欲其实是比较大的一个原因。这位戒友如果没有染上手淫恶习,也许他的人生会过得很好,而现在却因为纵欲导致生活陷入困境,看病贵,还要忍受病痛,生活质量大幅下降,可以说活得很痛苦,很惶恐。是该好好下决心戒了,一定要拿出破釜沉舟的勇气,下大决心戒!就像斯大林格勒保卫战一样,如果不戒掉,继续纵欲,将来弄不好会得肾衰竭,这是关乎生死存亡的大事,一定要万分警醒,拿出大决心拼死戒!你想想,如果你倒下了,你父母怎么办?你家庭怎么办?前段时间看到一个帖子:“楼下的一个厨师死了,三十五岁,早上说是心肌梗塞,现在发现他的衣服和床铺下面都是壮阳药,酒店只肯出仁道的丧葬费。”纵欲的后果很可怕,年纪轻轻就把身体掏空,弄不好真的会挂掉,我们一定要警醒,要有危机感。戒色一定要学会修心,否则“转头又变卦”还会继续上演,一定要学会专业戒色,严格落实戒色十规,学会对治自己的邪念,只有这样才能真正突破那个怪圈。
\end{case}

\begin{case}
    持续一年的濒死感症状回忆,警惕自己和大家。想想自己躺在医院看天花板的绝望,想想自己上不来气胸口憋的濒死感,想想自己上不来气憋得肚子和手脚发麻的濒死感,想想检查都正常的无助绝望感,想想自己动个念头呼吸就会变弱的虚弱濒死感,想想自己被蚊子咬一下就快断气的濒死感,想想冬天每个晚上吸几口冷气就吸不进气的濒死感,想想严重失眠的痛苦感,想想失眠导致的严重耳鸣声,想想自己眼睛看几眼东西就吸不进气的濒死感,想想一年来每天都有的濒死感,仿佛光线、空气、风、颠簸都无法承受的虚弱。如今戒色 267 天了,症状接近消失,耳鸣都接近消失了,杀死心魔,永远不再做心魔的傀儡,不然心魔非把你搞到废,搞进医院,搞到死,搞进地狱。拒绝适度伪命题,大家悬崖勒马,不要走我的老路。

    \textbf{附评} 这个案例前半部分很沉重,撸出濒死感的确非常痛苦,身体会出现各种症状,极度虚弱,虚弱得不能受到一丁点的刺激。我过去得神经症时也非常虚弱,走几步路都要休息下,气喘严重,阳光和风都无法承受,最明显的感受就是睡了一晚,早上起床感觉就像没睡觉一样,特别累,那段时间我感觉自己大病临头了,疑病很严重,但检查基本正常,没有器质性的问题,但身体就是很难受,每天活得很绝望,那种症状体验只有亲身经历过才能真正明白。虚弱得就像一个老年人,想想曾经在篮球场上健步如飞,跳起抓篮筐,后来得病后连走几步都要停下休息,真的一个天一个地。这个案例后面突然柳暗花明,戒色 267 天,症状接近消失,非常激励人的逆袭,我那时是戒了一年多,症状基本消失,但还是很虚弱,等到戒了两年多,虚弱感没了,又生龙活虎了,又开始打篮球了,精气神回来后真的不一样,过去总是疑病、恐病,后来恢复后,根本不会去想了,那种健康的感觉自然会给你一种健康的自信,有一种久违的活力和生机,整个人都明亮鲜活了,走路很轻盈,都能一蹦一跳,开心喜悦充溢着内心,不像沉迷手淫时那种死气沉沉、鬼气缠身的感觉,就像行尸走肉一样,拉长着脸,耷拉着眼皮,看着特无神特猥琐。\textit{嗜欲之性,固无穷也,以有极之性命,逐无涯之嗜欲,亦自毙之甚矣。(《元气论》)} 我们一定要懂得保精之道,能量不能随便耗泄,懂得戒色,懂得积累能量,身体很多症状都会减轻或消失的,这是我的切身体会,也是很多已经恢复戒友的切身体会。有病是该积极治疗,但戒色养生是恢复的基础,三分治疗,七分养生,懂得戒色,注重养生恢复,身体就能逐步恢复健康。这位戒友最后几句话说得很直接,也很给力,一定要战胜心魔,否则迟早会被搞进医院,到时就痛苦无量了,为了那几秒短暂的快感而毁了自己的健康和人生,实在太不值得了,真的要悬崖勒马,赶紧回头。
\end{case}

\begin{case}
    戒色四个多月了,感受到了真的快乐,明白了真的快乐与性并没有关系。戒了,感觉到了很多没戒之前没有的快乐,那种快乐是纯真的,是无邪的。慢慢地,我懂得了该怎么做人,该怎么做事怎么学习,怎么对待生活中的繁琐事情,也明白了很多道理。现在不仅外表改变了,变帅了,变精神了,性格也变了,人也变得自信了,那种感觉真的很好,是还在撸管的你感受不到的。只有戒了一段时间后才深有体会,那种感觉太好太棒了,到那时候的你就不想再撸了,因为你不想放弃那种美好的感觉。

    \textbf{附评} 这位戒友的体会很好,真正的快乐与性无关,人往往有一种思想误区,就是觉得放纵会让自己快乐起来,其实放纵只是感受到了短暂的快感,在射出时脸上往往会出现痛苦难忍的表情,因为生命力要被分出去了,喉咙还会发出痛苦的声音,是不是?养命的能量要射掉了,本能地就有痛苦的感受,虽然感受到了快感,但也含着某种能量丢失所带来的难忍的痛苦。有的昆虫交配几天后就会死亡,那股生命的能量真的很宝贵,丢失后精气神就会大幅下滑,身体感觉也会变差许多。真正的快乐不是来自性,也不是来自钱,也不是来自其他物质,真正的快乐不需要任何理由,当你内心恢复纯净了,你就会变得快乐起来,看到简单的事物都会感觉很喜悦,比如看到一只鸟从你眼前飞过,你就会开心喜悦起来,真的是这样,小时候我们就是这样的,后来开始看黄了,内心龌龊了,你就失去这种单纯美好的体验了。戒色后我再次感受到了纯真无邪的快乐,真的很美好,那种纯净的质感真的魂牵梦绕,再次变得简单、纯粹和纯真,内心充满着喜悦,变得轻松了,爱笑了,那种感受是撸管无法给予的,撸管只会让你陷入深渊。国外一位撸者说:“我开始变得孤僻,在欲望中苦苦挣扎,对色情上瘾,让我痛不欲生,明知色情伤害我的身心,却又难以控制地去邪淫。”那种状态是失控的,不自由的,不快乐的,就像一个提线木偶一样被操控,内心会变得沉重,郁郁寡欢,每次放纵过后会变得空虚、沮丧和不开心。一位戒友回忆过去:“我能感受到那种生命力流失的感觉。”整个人变得憔悴、猥琐和衰败,毫无生气,能量频繁走漏,自然会进入枯萎的状态。另外一位戒友说:“当我们的思想开始变得纯净了,你就能找回这个世界上最美好的心灵感受!”他说得很好,那种自由感、轻松感、美好感和纯净感真的太宝贵了,内心很愉悦,很赏心悦目的美好,整个人都光亮了。心灵净化,灵性提升后,活出高贵、纯洁、光辉、慈悲与智慧,学会无私奉献,学会无条件的爱,是这样一种崇高的灵魂特质,进入高频振动的正能量模式。以前有戒友说:“能主宰自己就能感受到莫大的快乐。”的确如此,能主宰自己,就有真正的自由、真正的快乐和美好。戒到一定时间,会觉得戒色自律的生活才是自己真正想要的,会厌恶过去邪淫堕落的生活,再也不想撸了,但还是要严防心魔的进攻,因为很多人虽然信誓旦旦说要戒,但一念袭脑,没断掉,那个身影又开始猥琐起来、抽搐起来……
\end{case}

\begin{case}
    2015 年戒色,到如今已三年,回望这三年如白马过隙,细数这三年戒色真的给我带来了莫大的改变,从内至外,从一个猥琐怯懦的孩子成长成了一个自信坚强的青年,戒色三年让我收获了一个健康的身体,一份称心的工作,等等等等……不觉间我以为这世界已然恢复了曾经的美好,可不想噩梦却悄然而至,十年的撸龄已经在心头落上了一层厚厚的尘埃,随着时间的流逝我以为不需要坚持看戒色文章,不需要学习戒色知识,我已经恢复纯真善良。可情况并非是我想象的那样,在不经意间,一次擦边新闻,一张黄色图片,让我又堕落下这无尽的深渊!这让我不禁想起了一首诗:身是菩提树,心如明镜台,时时勤拂拭,勿使惹尘埃。这时我才了知,戒色就像冰冻三尺非一日之寒,需要活到老戒到老,需要不断地调节戒色和生活之间的关系。其实从另一方面来看,戒色即是生活。希望各位戒友以我为戒,将戒色进行到底,愿各位戒友早日戒除手淫、邪淫!路漫漫其修远兮,吾将上下而求索。

    \textbf{附评} 这个案例前段很励志,大家一定以为是一个戒掉的励志案例,后来画风一转,因为对境时没做好,又堕落了。戒色三年很不容易,也是很高的戒色成绩,给这位戒友带来了很多好处,但他放松警惕,以为不需要看戒色文章了,结果肯定就悲剧了,小心驶得万年船,不管戒多久,都要严防心魔反扑。这是个色弹横飞的时代,网络上铺天盖地,各种擦边图和擦边新闻,就是为了吸引眼球,让你起邪念,怎能不警惕?到了夏季,生活中也有各种诱惑在考验你,必须做好视线管理。戒色十规专门强调了对境实战,因为我深知对境时就是真正的考验,一定要特别重视,说时似悟,对境生迷,说道理似乎都懂,但是对境时就验出来了,比如一张诱惑图突然跳出来,你是否能立刻转移视线,开启散视,我也遭遇过很多这种对境实战的情况,我的经验就是不要试图去看清,普通人的第一反应肯定是盯着看,试图看清,而我们有实战意识,就要学会避开,开启散视模式,然后视线移到右上角的大叉,立刻关闭,手机可以点退出。这是我的实战经验,普通人的第一反应和戒色高手的第一反应是完全不同的,要学习高手的实战操作。对境时还有一点要克服,那就是好奇心,很多标题都写得让人起很大的好奇,吸引你点击,这时就要警惕了,因为你本能地会知道里面的内容有诱惑,就不要去点击。人的好奇心理往往很强,这方面自己要学会去克服,高手都和自己逆,否则顺着好奇心去点击,就很容易陷进去。一张诱惑图片出现在视野里,往往会有一股很强的拉力,会把你的视线吸引过去,这时候必须强硬一些,立刻关闭,退出来,思维不净观。那股诱惑的拉力是很强大,如果你的戒色觉悟和实战意识很强,你就能战胜这股拉力。一位戒友说:“就是因为每次对境实战不行,才破戒,看小说的时候老弹出来一些页面,有时无意点进去,真的是第二眼入魔,刚刚进去还能反应过来退出,又无意点进一次然后想看清那些图片,然后就疯狂找黄,破了。”对境时切记不要点进去,不要试图看清,不要看第二眼,要坚决避开!这个实战意识要反复强化!切记!

    修行是终生的,即使戒了很久,还会发现邪念、图像、微妙感觉,还时不时会入侵,特别是怂恿和邪淫的回忆肯定会再次进攻,所以必须保持警惕,不可掉以轻心,进入戒色稳定期后邪念会变少,几天甚至十几天都没一个邪念,但还是要警惕,警惕的螺丝不能松,否则戒色大夏会轰然倒塌的。古人修行都是几十年的功夫,不是几年,是持之以恒的坚持,坚持到最后就习惯成自然了,不觉得自己在刻意坚持,而是本应该如此。可以问下自己,对境时是否一点都不动心了?做不到就要反省,就要不断完善、优化和提升对境实战的表现,即使戒色状态已经很稳定了,还需小心,因为有时翻种子会很猛烈,还是要做好观心断念,多思维不净观看破女色的诱惑,多思维邪淫的危害警醒自己。活到老,戒到老,戒色即是生活,好好协调好戒色和生活的关系,并不是叫你沉迷戒色,而是把戒色融入生活,平时应该积极奋斗自己的人生,过好自己的生活。戒色文章和戒色笔记还是应该坚持学习,进入稳定期后可以减少学习量,但不应该离开学习,圣贤教育也应该坚持学习,这样有利于保持良好的戒色状态,也有利于德行和修为的整体提升。留在戒色吧帮助戒友答疑,这也有利于保持戒色状态,也是在行善积德,可谓一举两得。
\end{case}

\begin{case}
    本人七年的手淫,最多一天撸五次,有时候一天一次,有时候几天一次,去年也了解过手淫的危害,来到戒色吧学习戒色知识,但还是抵不住邪淫的诱惑,屡戒屡破,11 月 20 日晚上咳出血,本来想第二天去医院检查,没想到晚上睡到半夜又咳出大量暗红鲜红血,立马打 120 送往医院抢救,经医生抽血拍片检查为空洞性肺结核,眼泪立马掉了下来,恨自己平时没有照顾好自己,之前几个月喉咙一次感觉有异物有痰咳不出来偶尔感觉胸闷也不知道检查,经过一个星期的住院治疗,花了一万四的费用,一天两千块昂贵的治疗费用让我承受不了,想尽早出院回老家治疗,外省农保没有多少报,躺在床上吃喝拉撒都是爸爸和老婆照顾,血也止住了,就是拉尿的时候要等几十秒才尿得出来,等到出院手续全部办好之后,拉完早上的第一泡尿后突然感觉尿道被堵住了,右边肾突然剧烈疼痛,但我还是坚持出院让亲戚开车送我回老家,先是右边肾和睾丸疼痛,再过几个小时整个小腹膀胱也胀痛得厉害,一直以为是尿结石肾结石之类的,就这样痛了八小时到老家县人民医院住院治疗,先是抽血化验挂消炎点滴,塞止痛药,第二天做彩超,然后说我两边肾可能有结石,前面还有积水,膀胱没有尿也看不到具体的让我回去憋尿再来,于是一直喝水喝水,但还是没有一点尿,后来就开始呕吐,吃什么就吐什么,一天吐个十来次,到晚上还是没有尿,做不了彩超,跟医生说明了情况,医生还是认为是我喝水少了是我吐掉的原因导致没有尿,第二天早上又狂输液体喝水,第二天吐得更厉害了,什么都不吃都呕吐得很厉害,还是一点尿都没有,等到下午医生又给我抽血发现我有可能急性肾衰竭,叫我爸爸立马让我转院到市人民医院,医生说我怎么两天没有尿才来,说我再晚来一天命都没了,当晚 12 点做完肾堵塞手术,肾里面没有结石,两边肾都被不明黄色脏物堵住了排尿道,现在命暂时是保住了,肾什么情况我也不清楚,医生也没说,手术三天了也不知道情况怎么样,希望不用做透析,希望自己的肾能够好起来,这几天一直反思自己这么多年来的手淫看黄行为,真的是悔恨不已,也把这些肮脏行为告诉了爸妈并保证今后不再犯了,爸爸妈妈对我这么好,我却以这种局面对待她们,现在只有等病好了再好好报答她们,从肺结核到急性肾衰竭,不知道有没有转移到慢性肾衰竭尿毒症的可能,我现在好害怕,肾两边已经插管排尿了,在医院躺三天了医生没有给我结果,最后奉劝各位,真的别再手淫了,别再邪淫了,伤身败德,一辈子没好报,血淋淋的教训!!!

    \textbf{附评} 这个案例真的是肾衰竭了,真的到了生死存亡、性命攸关的程度了,中医认为肾为先天之本,肾精充足,五脏六腑皆旺,身体的抗病能力就强,反之,肾精亏损,则五脏虚衰,会多病早夭。肾为五脏六腑之根,人体气血阴阳皆系于此,手淫伤肾,时间长了五脏六腑都容易出问题,等到重病来袭,那真的苦大了,这位戒友也算遭了大罪了,真的是人财两空,他的经历很凶险,“再晚来一天命都没了”,年纪那么轻,如果走了,父母和老婆怎么办?一个好身体真的太重要了,男人要成长为家庭的顶梁柱,不能被色情的白蚁给蛀空了,成了朽木,那就惨了。男人一定要懂得戒色养生,把顶梁柱给彻底弄坚实了,撑起这个家,为家人撑起一片天。其实严重的疾病离我们真的很近,如果继续疯狂手淫,不珍惜自己的身体,指不定哪天就会倒下,这是说不定的,伤到一定程度,肾不行了,肝也不行了,脾也虚了,心脏也疼了,肺也不好了,过上了多灾多难的生活,这时才感叹健康的身体是多么重要,之前疯狂纵欲是多么愚蠢,多么无知。心、肝、脾、肺、肾,虽然肾排在五脏最后,但其作用却是最重要的。如果把肾比喻成一个“锅炉”,那么身体消耗燃烧后产生的“煤渣”(代谢废物)都必须经锅炉排出。一旦“锅炉”出现问题,废物排不出去,不仅会损毁“锅炉”,五脏整个系统也会崩溃。肾虚会累及心、肝、脾、肺等各脏器,并伴有心烦、失眠、头晕、眼花、消化不良、咳嗽等症状。肾病早期多为疲劳、乏力,眼脸浮肿、颜面苍白、尿中有大量泡沫、排尿疼痛或困难。接下来会产生食欲减退、恶心呕吐、腰痛、夜尿频繁、全身水肿、血压升高、呼气带尿味、骨痛、皮肤瘙痒、肌肉震颤、手脚麻木、反应迟钝等。如果严重了,上述各种症状继续加重,同时会导致心、肝、肺等多脏器功能衰竭。印光大师在开示中说手淫是杀身之利刃也,现在看来是千真万确的,有的人体会不深,不知道手淫的危害,以为自己身体好,没事,刚开始都以为没事,等到身体出现症状了,就知道了,等到身体全面垮掉了,器官罢工了,就知道严重性了,真的不是闹着玩的,是要命的!曾经我身体很强壮,看着也很魁梧壮实,自以为没事,后来神经症爆发,把我抛入痛苦的症状地狱,我这时才知道日积月累的纵欲是多么可怕,它不是一下把你搞废,而是一点点蛀空你,一点点蚕食你的健康,等你突然惊醒时,已经症状缠身了,已经身处痛苦之中了,每天都在煎熬,每天都很绝望,甚至想自杀。这位戒友最后奉劝大家“真的别再手淫了”,他的奉劝真的是诚恳到骨子里了,还带着深深的害怕,希望他早日康复,愿他的经历警醒世人。
\end{case}

\begin{case}
    目前戒色八个月啦,整个人精力充沛,感觉浑身充满了力量,阳光开朗了不少。脑子也好用多了,不再像以前那样迷迷糊糊,记忆力增强,背单词轻轻松松拿下。脸上的脂溢性皮炎也好了,记得之前一热起来整个脸像是有蚂蚁在爬一样,又红又痒,怕晒太阳,闷得难受!母亲带我去皮肤科医院看了几回,涂了不少膏药,却还是反反复复。现在好啦!皮肤已经重新变得光滑起来,真的是没涂药自己好的,没骗大家!还有还有,以前脸上全是油,尤其是鼻子上,出油量相当高,你能想象得到吗?一张又红又油的老脸,让别人看了都觉得恶心。幸好,戒色到现在好多了。胃口变好,消化很快,能吃。社恐也好了,充满着自信,敢跟任何人对视,心里一点不带怕的,社交什么的都 OK,no problem!脱发也减少了很多,以前一天至少掉几十根。最最重要的是浑身洋溢着正能量,特别快乐,没理由的开心,特别爱笑!哈哈,久违的美好!

    \textbf{附评} 这位戒友恢复很不错,整个人的状态非常好,大家应该都希望自己能找到这种健康的状态,充满正能量,快乐、开心、爱笑、美好,那种积极向上、开朗自信阳光的状态,真的很宝贵,如果沉迷邪淫,这种状态就会消失,身上和脸上都会多出很多负能量的感觉和阴暗气息,不仅自己能感觉得到,其他人也一样能潜移默化地解读你的气场。一位戒友感叹:“这种浑身向外散发着正能量的舒畅感觉实在太舒服了,这种状态太珍贵了!”心灵再次恢复纯净,那种由内而外的舒畅,脸上洋溢着幸福的笑容,心地光明,通透敞亮,轻盈喜悦,这感觉真好!亲身体验到了心中的那种纯净静谧的美好,还有一位戒友说:“戒色半年,我现在有时感觉到莫名的开心,像是小孩子一样,照相的时候笑容也灿烂,面相和善,而且眼睛变得很大,以前努力睁,都没有这么大,像是开了眼角,变长变大,有了精神,眼白以前布满血丝,现在渐渐地清亮了起来。戒色真的可以让你提高颜值,充满精气神!”另外一位戒友说:“最让我感动的是我感受到了纯净的大快乐。明天是我戒的第 100 天,我感觉过去 20 多年所有的快乐加起来,都没有这 100 天的快乐多。”莫名的开心,不需要理由,很多戒友已经找到了快乐的真谛和秘诀,那就是通过修心让心灵重新恢复到干净、纯粹、纯真的状态,处于那个状态自然就是快乐的,泉涌般的快乐从心底出来,没有理由,就是快乐,变成了一个开心果,就像小孩子一样,很容易快乐,那么单纯,那么美好。当身体好了,就有好状态去面对自己的学业和事业,而且你的心态会非常好,因为你内心是充满喜悦的,你会把喜悦的能量传递给别人,别人也跟着你一起喜悦,一起开心。你的眼睛是清澈明亮的,你的内心是纯真美好的,你的笑容是发自内心的,你处在纯净纯善的状态。老子提倡“复归于婴儿”,回到那种天真纯朴的初始状态,心灵非常干净,真的回归了那种状态,就能感受到真正的快乐,内心有极强的幸福感。\textit{看着你生命四周的圆满——照在你皮肤上的温暖阳光,花店门口摆放的美丽花朵,咬一口多汁的水果,或是沉浸在从天而降的充沛雨水中。在每一步中,都有着生活的圆满。感谢所有在你周围的丰盛,就会唤醒你内在沉睡的丰盛,然后让它流出。(埃克哈特·托利)} 每个人的内心都有丰盛的源头:爱、喜悦、自在、快乐、幸福感,学会净化心灵,学会感恩,就能开启内在的丰盛,在最简单的事物中感受到快乐、美好与圆满。
\end{case}

下面步入正文。

大家看到这季的标题可能会有疑问,知道会讲修心,但不知道具体会讲什么内容,其实这季是关于一本书的笔记分享与解析,在 \ref{129} 我向大家推荐了《深夜加油站遇见苏格拉底》这本书,这本书有一个副标题叫:和平勇士之道。这本书是讲修心的,而我是从戒色这个角度来讲的,所以就叫戒色修心勇士之道。丹·米尔曼(Dan Millman)的《深夜加油站遇见苏格拉底》是一本能改变生命的书!遇到生活问题的青年,在深夜遇见某个智慧的老人,于是开始一段灵修之旅。这本书的名字很有穿越感,苏格拉底是古希腊著名的思想家、哲学家、教育家、公民陪审员,苏格拉底怎么跑到现代的加油站了?其实是那位拥有神秘智慧的强悍怪老头——深夜加油站的老工人,也就是作者昵称的苏格拉底,说是苏格拉底,但整本书与古希腊哲学并无关系,丹米尔曼遇见的其实是一位化身西方人的禅宗师父,教授的是修心的道理,这本书的确吸引了我,不仅是因为小说的形式,而且书里提到的要点和所用的语言都是我所认同与欣赏的,我很喜欢这本书,大概花了五天读完,每天坚持读几十页,书也不厚,大概两百多页。如果这本书叫:佛在加油站。也许更恰当些,不过西方人可能比较仰慕苏格拉底,毕竟是西方人写的书,所以用苏格拉底也许对他们来说更有亲切感。

这本书是根据丹·米尔曼传奇般的生活经历改编而成,我相信书里的一些高峰体验都是真的,这本书不仅是在讲修心,而且也是一本揭示真相的书。大家对于真相这个词可能比较感兴趣,一定想知道所谓的真相到底是什么?真相是关于“我是谁”,是关于这个世界的本质。这本书都有讲到,虽然不是讲得特别详细,但已经点到了。考上知名学府、获得世界冠军荣誉的丹在短暂的欢乐过后,深深陷入无可名状的空虚和恐慌,无法入眠的他在一家 24 小时加油站遇见一位被他称作苏格拉底的老人,老人向他展示了超能力,引起了他的强烈好奇和关注,想知道老人究竟是怎么做到的,那个超能力就是跳到加油站的房顶,那个房顶至少四米左右,这一幕是发生在深夜,刚才看到老人还坐在椅子上,一转头已经站在房顶了,在深夜发生这种事,绝对诡异和匪夷所思,老人在背后无声地看着丹,不说话,有点恐怖片的氛围和感觉。其实老人是想用这种方式引起他的兴趣,把他吸引过来,从而接受他的教诲,老人想成为丹的师父,丹也需要一位师父来指点自己。

根据这本书改编的电影我也看了,虽然拍浅了,没有完全传达出书的内涵,但也不失为一部励志的电影,这部电影中有两处诱惑场景,是讲他放纵的,不过时间很短,只是象征性的,可以跳过不看。苏格拉底叫他戒色和吃素,以提升自身的能量水平,戒律对修行非常重要,戒律是修行的基础。苏格拉底是要把修心的知识传授给丹,丹一开始有很多年轻人常见的毛病,比如骄傲、自负、自以为是、炫耀、傲慢、容易生气等,老人以捉弄、嘲讽、关爱、抚慰、激励等各种方式来教导他,就像禅宗师父的手段,虽然严厉但却充满慈悲,让丹慢慢步入修行的正轨。老人叫他大笨驴、笨蛋,一开始丹还生气,后来丹说:“他叫我笨蛋时,我光笑不还嘴。因为就他的标准而言,他这样叫我并没错。”我觉得“大笨驴”这个叫法也挺搞笑的,在师父眼里,丹的行为确实很笨,思想也比较幼稚,毕竟还是大学生,很多方面还不是很成熟。叫大笨驴也是在调教他,因为他自负、自以为是,这样叫可以挫败他的自负,让他学会谦虚。老人的教导风格是睿智、慈悲且幽默,我很喜欢老人爽朗的爆笑,这一点对我很有吸引力,元音老人和黄念祖老居士也有着同样的特质,那种爽朗的爆笑仿佛可以粉碎一切幻象,我想只有佛才能笑得如此爽朗,如此开心,如此富有感染力。

从小学到大学,我们学了很多课程,语数外,地理、物理、化学、历史、生物、政治等等,但人生最重要的一堂课就是学会修心,这是最最重要的。学会修心,就能真正主宰自己,往高了说,还有望证悟宇宙人生的实相和真理。现代社会年轻人的主要关注就是娱乐,现代娱乐业很发达,电视网络手机的传播力度又如此之大,年轻人普遍关注的就是明星、女人、游戏、物质享受等,互相攀比,以放纵为乐,以邪淫为荣。很多人虽然生活富裕,有钱有车有房,但是心灵却一片荒芜,从放纵中得不到真正的快乐,只会让自己越来越空虚,负能量越来越重,精气神也越来越差,到了一定程度真的有江河日下之感,一年不如一年。就像丹一样,他拥有富裕的家庭、优秀的成绩和完美的体格,是人人称羡的大学中的风云人物。但是,丹每天晚上都会被噩梦惊醒,他的内心很惶恐不安,他在书里说:“尽管事事依旧如意,我却越来越忧郁。不久之后,梦魇迅速袭来,我差不多每晚都会惊醒,浑身冒冷汗。”失眠也是家常便饭,种种迹象表明,丹已经患上神经衰弱了,和他纵欲的生活方式是密切相关的,老人让他戒色,一方面是为修行打好基础,另外也是为了帮助他恢复身体。

回想作为学生党的我,我对修心这个词是完全不懂的,这个词好像离我十万八千里,我每天的关注主要就是学业、运动、游戏、女人,各种闲聊,当然也时常意淫,偷偷看黄手淫,那是一种浑浑噩噩的状态,整天不知道自己要干嘛,对人生很迷茫,每天都是得过且过,混日子,混吃等死。那个阶段的我不懂什么叫修心,也对修行没多大兴趣,根本不知道修心的价值和重要性。后来戒色后我不断研读大德开示,才真正明白修心的价值,修心的课程是最宝贵的,是无价的,或者可以说价值连城,老人教给丹的课程就是修心,让他的注意力从娱乐、女人、放纵上转移到内心的修炼上。刚开始丹抵触老人规定的戒律,书中也叫门规,后来渐渐地他感受到了戒律带来的好处,身体变好了,感觉轻松了,做事更专注了,于是他开始自觉遵守戒律了,当然现实中的诱惑还是很大的,特别是女人的诱惑,让丹很难把持住,不过丹最后还是通过了考验。放纵堕落是一种人生,戒色自律又是一种人生,刚开始都会倾向于放纵,因为年轻无知,被快感冲昏了头脑,最后会发现戒色自律才是自己真正想要的人生,男人要学会培养自己的正气,积累自己的能量,这样才能有一个好身体,才能干一番大事业。

在推荐序里,有这么一句话:“英雄之所以成为英雄,不仅是他打赢了多少外在敌人,而是他最终是否可以战胜自己。”这句话我深表认同,打赢外在的敌人不如战胜自己的心魔,能够主宰内心才是真正的英雄豪杰,才是真正的勇士!这本书的确让人豁然开朗,很适合年轻人读,可以向他们介绍修心,把他们的生活方式转变到修行的轨道上来,让年轻迷茫的心灵找到人生的正确方向,发现真正的大智慧。本书也描写了一些高峰体验,这些体验我在其他灵修书籍里也有看到过,其实都是在揭示同一个真相。对于合一体验,我是比较感兴趣的,与宇宙合一,成为整体宇宙!合一是真相,分裂是幻象。天人合一不是小说的情节,不是一种虚构的幻想,而是可以真实发生的体验。\textit{你最终的命运是进入合一的觉醒。(李尔纳)} 在这个世界上不管取得了多大的外在成就,都不会真正满足,这个世界的一切都是无常的,每个人都要死,死时一分钱也带不走,只有找到真我,亲自看见真相,进入永恒的维度,才能真正彻底地满足。

下面开始分享笔记和解析。

\begin{quote}\it
    生命必须采取正确的行动,才能让知识活过来。你的问题就在这里,你知道,却不采取行动,你不是勇士。
\end{quote}

\textbf{解析} 这条讲的就是知行合一,虽然知道,却不采取行动,这不是勇士,真正的勇士必定雷厉风行,行动起来很果断干脆,有很强的决断力、行动力和执行力,就像特种兵执行任务一样勇猛和专业。学习了戒色文章,知道前辈在强调学习提高觉悟,练习断念,就要勇猛去执行,拿出勇士的风范,要有势不可挡的冲劲。知而不行,是为不知;知而不行,终非真知;知而不行,谓之不诚。真正拿出决心、热情与斗志,去猛烈行动,真心实意去做,最终肯定会有一个好的结果,即使暂时失败,也能学到很多东西,所以贵在实行,贵在落实。苏格拉底在教丹行动的重要性,生命必须采取正确的行动,才能获得一个满意的结果。

\begin{quote}\it
    我调整望远镜,把焦点集中在这位体操选手身上,看出她来自俄罗斯。这么说来,我们此刻正身处一场于某地举行的国际表演赛。她步向高低杠时,我发觉自己听得见她在自言自语!“这场地的传音性一定很棒。”我心想,可是我看到她的嘴唇根本没有在动。我把望远镜头迅速移到观众席,听到许多声音在吼叫,可是观众却只是安静地坐着。我恍然大悟,不晓得什么缘故,我正在听他们内心的声音!
\end{quote}

\textbf{解析} 这是苏格拉底给丹的一次课程,老人开了他心通,可以让丹“听”到别人的自言自语,电影中这个场景和书中的场景不一样,电影是在体育馆的横梁上,书中则是在一场国际表演赛上,但都是为了揭示那个现象——脑内的自言自语、心智的喋喋不休、胡思乱想。书中的心智指的就是念头,英文中就是 mind,不管是托利还是李尔纳都揭示过这个现象,自言自语会覆盖真我,把人困住。电影中这个场景是比较震撼的,也许是这部电影最震撼的场景了,从上帝视角切入,看到众生是多么深地陷入到心智的陷阱中去,迷失在心智中,每个人都在自言自语,缺少觉察,跟着念头跑。我平时的修心体会也是如此,稍不注意,马上就会陷入念头里,自言自语,胡思乱想,还有很多评判的念头,回忆的念头,托利说:“需要在头脑里的自言自语发生的时候,就意识到它,将注意力从思考里抽出来,过了一阵子,你注意到自己又陷入思考,然后你选择重新将注意力抽离出来。”这就是一项练习,让自己不要陷入思考,随着一次次拉回,觉察力就会逐步提升。

\begin{quote}\it
    我破天荒头一遭领悟到,我为何如此热爱体操。因为它能让我得以暂时脱离嘈杂的内心,获得神圣的喘息机会。在我旋转摆荡和翻滚时,其他的一切都不重要了。我的身体在活动时,内心因为这宁静的时刻而得以休息。
\end{quote}

\textbf{解析} 一些体育运动的确会迫使人暂时脱离心智,获得内心的宁静,因为在做那些动作时必须保持专注,比如体操、攀岩、走钢丝等,我打篮球时也能体会到宁静的片刻,比如突破上篮或者拼抢篮板时,头脑往往是宁静的,那些片刻其实是很珍贵的,现在看来我之所以热爱篮球,就是因为专注打球时,脑子里什么也不想,这的确是很大的一种休息,身体得到了锻炼,脑子得到了休息,两全其美。

\begin{quote}\it
    其实你一直被自己迷乱的心智所催眠。
\end{quote}

\textbf{解析} 这句话一针见血。众生就是认心智为自己,跟着念头跑,最强力的催眠就是心智中的那个“我”字,那个“我”在冒充你,真正的你是观察者,不是念头。

\begin{quote}\it
    你是被自己的幻象所囚禁的俘虏,你对自己和这个世界怀有幻觉。你需要拥有比任何一位电影中的英雄更强大的勇气和力量,才能挣脱幻象,获得自由。
\end{quote}

\textbf{解析} 苏格拉底提到了“幻象”这个词,之所以这么说,是因为苏格拉底知道我们这个世界是心智的全息投影,是一个幻象。近些年有科学家提出了一个令我们十分难以置信的观点:宇宙很可能只是一个巨大的投影,宇宙中的万物皆是幻象!著名的量子物理学家玻姆提出的观点:宇宙可能只是一个巨大的幻影,宇宙万物都不是真实存在的,宇宙中的事物完全只是被投影的幻象!尽管宇宙看起来具体而坚实,其实宇宙只是一个幻象,一个巨大而细节丰富的全息图。物理学家已经开始通过实验来证实了,而这真相早已被修行者所亲证,有的戒友听到这种观点,也许会感到难以接受,我刚开始也这样,后来慢慢就接受了,我最喜欢的比喻是“梦”,梦是心智所造,我们被困在梦中,醒来后才发现这是一个梦,一个幻象,要醒来就要通过努力修行。虽然是梦,是幻象,但也要按照游戏规则来生活,如果乱来,肯定会遭报应的。

\begin{quote}\it
    你看不见自己的囚笼,因为栅栏是无形的。我工作的一部分就是要指出你的困境。
\end{quote}

\textbf{解析} 笼子就是 mind!黑客帝国最经典的台词:A prison for your mind!你的念头就是你的监狱!你的笼子!通过自言自语的方式把你持续困在里面。念头既是监狱又是工具,圣人稳固在纯粹的觉知上,然后在必要时才用念头做事,而凡夫是认念头为自己,跟着念头跑,停不下来,这就会导致自己被困住。栅栏是无形的,因为笼子是你的mind,mind就是念头,念头就是心智。上次有戒友推荐灵性觉醒电影:《三摩地》(Samadhi),非常好的影片,我之前就关注过这部影片,质量非常高的影片,是揭示真相的片子,推荐大家看一下。还有一部片子也非常好,那就是《内在与外在的联系》(Inner Worlds, Outer Worlds),也是揭示真相的片子,质量也非常高。

\begin{quote}\it
    丹,你正在受苦!你其实一点也不享受你的生活。你的娱乐、风流韵事,甚至体操,都只是治标不治本的方法,只是用来躲避隐藏在你心底的恐惧。
\end{quote}

\textbf{解析} 外在的事情,都只是治标不治本的方法,要治本必须学会修心,否则外在的事情做得再多,也只是一种逃避,无法控制念头就是在受苦。我以前沉迷手淫时就是一种逃避,逃避学业的压力和生活的压力,最深的恐惧其实就是怕死,怕死了就什么也没有了,所以活着要好好享受和放纵,其实放纵只是一种逃避死亡的方式,最终还不得不面对死亡,到时就会知道再多的放纵也是无意义的,也是不重要的。人死不是灯灭,看看濒死体验就知道还要继续轮回,万般带不去,唯有业随身,真正了达此理之后,行为必定有所改变,要尽量避免造恶,平时要多做善事。

\begin{quote}\it
    你的心智就是你的困境。
\end{quote}

\textbf{解析} 这句话更直截了当,指出心智就是困境,《三摩地》那部影片里讲:“思维可以被比作意识的陷阱、迷宫或监狱。……把我们牢牢地锁在思维的矩阵中。”思维就是念头,就是心智,我们要学会主宰自己的内心,学会控制自己的念头,这样思维就是工具。思维这两个字很有意思,思字上面一个田,这个田就像监狱的栅栏,对不对?下面一个心,代表这所监狱是由念头做成的,而维字是绞丝旁,代表把人束缚起来。念头比较口语化,而思维比较书面化,思维这个词就暗含监狱、束缚之意,思维的另一面就是工具,我们学习和生活都要用到思维,它是一个很好的工具,但如果你缺少觉察力,无法主宰内心,那它就是一座监狱,把你困在里面。

\begin{quote}\it
    意识并非心智,知觉并非心智,专注力并非心智。心智是障碍。
\end{quote}

\textbf{解析} 这句开示就是为了让你分清意识和念头的区别,纯意识是无念的状态,而念头会遮蔽这种状态,所以心智是障碍。如果你不知道心智是障碍,那就很难突破这个障碍,因为你在认贼作子,心智是两面的,一面是工具,另一面是监狱、障碍、束缚。关键就在于你是否能主宰自己,是否能够降伏它。

\begin{quote}\it
    如果你怎样都无法停止去思考数学题目或电话号码,或者老是不由自主在想一些恼人的思绪或记忆,这时就不是你的脑子在运作:而是你的心智在漫游。接着,心智就会控制你。
\end{quote}

\textbf{解析} 我们要学会做念头的主人,学会控制自己的念头,而大多数人认同念头为自己,跟着念头跑,内心不由自主地陷入自言自语,喋喋不休,想停也停不下来,持续被困在心智中,这种情况就是心智在控制你,你无法主宰自己。大家可以做个实验,就是停止念头一分钟,可以做到吗?估计很多人是无法做到的,不到三十秒,念头又开始带跑你了,你无法主宰自己的念头,如果连普通念头都无法控制,那邪淫的念头就更难控制了。心智作为工具,就好比汽车一样,你要开就开,要转弯就转弯,要停就停,完全由你控制,如果汽车失控,不受你控制,那是多么可怕的情况。如果念头不受控制,很可能会把你带入可怕的深渊,前段时间看见一个帖子,考完研究生去嫖娼,很悔恨,就是考完放松了,一个嫖娼的念头冒出来了,听信了,于是就去放纵了。还有一个戒友发了图片,一边是佛经,一边是伟哥,他内心在挣扎要不要去嫖娼,他那种嫖娼的念头已经起势了,已经变得很强了,这就是在起念时没有及时对治,导致最后陷入挣扎,欲火中烧。无法控制自己的念头,真的很可怕,有时一念之差就会犯下大错!

\begin{quote}\it
    你必须观察你自己,才能了解我说的意思,才会真正地明白。
\end{quote}

\textbf{解析} 修行首先要学会观心,观心就是观察自己的念头,能学会观察自己的念头,就能真正明白开示的真义,就能很快入门。\textit{唯观心一法,总摄诸法,最为省要。……心者万法之根本,一切诸法唯心所生,若能了心,则万法俱备,犹如大树,所有枝条及诸花果,皆悉依根。栽树者,存根而始生子;伐树者,去根而必死。若了心修道,则少力而易成;不了心而修,费功而无益。故知一切善恶皆由自心。心外别求,终无是处。(达摩祖师)} 修心是根本,修心首先要学会观心,心就是念头。

\begin{quote}\it
    你的思绪就像一只野猴子被蝎子螫到。
\end{quote}

\textbf{解析} 这句话就是形容念头多动的状态,就像猴子被被蝎子螫到,本来猴子就喜欢忙个不停,窜上窜下,被螫到后那动得更加剧烈。明就仁波切把念头形容为“Crazy Monkey Mind”,如果你观察过自己的内心,就会发现内在很疯狂很混乱,的确是那种 crazy 的状态,胡思乱想,停不下来,就像疯猴狂象一样,修行就是要学会降伏这个心。

\begin{quote}\it
    我开始在一本小记事簿上做笔记,把自己一天所有的思绪都记下来,只有练体操时不记,因为这时我的思绪已经被动作所取代。两天以后,我就得买较大的笔记本了,可是才过了一星期,也记满了。我看到自己竟然有这么多的思绪时吓了一大跳,更不要说它们大部分还都是负面的。这个练习让我比较能觉察到自己内心的噪音。
\end{quote}

\textbf{解析} 丹这个练习很好,就是把内心所想都记下来,这就是在开启观心模式了,通过这样的练习,可以有效提升觉察力。这让我想到了功过格,功过格侧重的是善念善行的培养,恶念恶行的警觉,首先都是要学会观察自己的内心,对于起心动念要非常敏感,能够及时觉察到念头的波动。

\begin{quote}\it
    我心智的杂音彻底占了上风。狂野、杂乱、愚蠢的思绪,自责、焦虑、渴望——全都是杂音。苏格拉底自始至终都是对的,我的确身陷囹圄。
\end{quote}

\textbf{解析} 丹这段描述了心智失控的状态,各种思绪涌了上来,大家应该都经历过这种状态,就是内心念头很多,很混乱,心智的噪音很大,搞得自己坐立不安,心神不宁,蠢蠢欲动,陷入这种状态就是被心智困住了,很多人总是陷入胡思乱想,停不下来,脑子一直被各种念头占据着,其中有很多负面的念头,这就是身陷囹圄,囹圄就是监狱之意。心智的确是一所无形的监狱,而荒唐的是人们把监狱当成了自己,并且一直在强化这所监狱而不自知。\textit{大多数人穷其一生囚于自己思维的囹圄中。(托利)} 认识到心智是监狱是非常重要的,我这几个月的顿悟就是加深了这一认识。

\begin{quote}\it
    你的心灵正在成长,你已准备好要接受剑了。
\end{quote}

\textbf{解析} 丹已经受过一些训练和教导,要真正接受剑了,剑代表的就是觉察力,核心就是觉察力的提升和强化,觉察力可以说是剑,也可以说是刀,能斩断念头的束缚,就像快刀斩乱麻一样。工欲善其事必先利其器,要想戒得好,要想在修行方面有所进步,就要好好磨快这把剑,功夫在平时,实战只是平时训练的检验,一定要注重平时坚持训练,这样实战时才有十足的把握。

\begin{quote}\it
    我欠缺专注力,而且心思游移不定。我们跑了几圈以后,杜威说,天空蔚蓝无云。我却光顾着想心事,根本没注意到天空。接着他往山上跑去,他是马拉松选手,我则打道回府,满脑子都在思考我的心智。天底下要是有“自找罪受”这种举动,这恐怕就是一件了。我观察到,在体育馆时,我的注意力集中贯注于每一个动作,可是一停止运动,我的思绪便又遮蔽了我的洞察力。
\end{quote}

\textbf{解析} 丹刚开始练习时,状态不太稳定,也有比较差的时候,总是陷入思考,想心事,想这个字就代表着笼子,上面那个相,就像古代的木笼一样,只不过栅栏是横过来的,很形象,我很佩服古人的造字智慧,一个思,一个想,已经把觉醒的线索暗含在里面了。陷入思考,自言自语,停不下来,满脑子的念头,一刻也不停歇,这种情况的确是自找罪受,一直思考,一直想,甚至强迫思维,让人感觉很累,内心也会很烦躁,患得患失,胡思乱想。丹观察到了,心智只有在他集中注意力做体操动作时,才会停止,一旦做完动作,思绪又开始遮蔽了,心智这种持续的占用,就像把人关进一个笼子,在笼子里面死循环。

\begin{quote}\it
    记住,碰到困扰时,抛开你的思绪,看穿你的心智。
\end{quote}

\textbf{解析} 生活中会碰到各种困扰,如果你有观心的经验,就会发现这些所谓的困扰都在心智的范畴内,当你抛开你的思绪,它们就不会困扰你了。大德说过要放下,不要去想,随缘而过,内心坦然,问心无愧。上季有一位戒友和我沟通,说他工作方面同事排挤他,家庭方面父母又闹离婚,搞得他很苦恼,内心很痛苦,这时候需要懂得放下,不要总是去想那些造成困扰的事情,要知道一切都会过去的,一切都会有一个结果,要坦然接受,积极面对,人生中的很多事情其实都是一堂课程,会帮助你尽快成熟起来。不管生活中遭遇了怎样的挫折,记得要发出正能量,要学会调整情绪和心态,不要陷入悲观消极的状态。看穿你的心智,它一直在冒充你,持续把你困在自言自语里,让你的内心陷入混乱和疯狂,看穿它,不要跟随它。

\begin{quote}\it
    你从体能的训练中已学到一件事:觉察力的大跃进并不会一下子就发生,而是需要时间与修炼。有个练习可以使你洞悉自己的波纹来源,那就是静坐。
\end{quote}

\textbf{解析} 苏格拉底对丹的开示,觉察力就是修炼的核心,觉察力的提升需要一个过程,不是一蹴而就的,不是练习几天就会大幅提升的,需要持之以恒的耐心练习,注重积累,要让觉察力变得敏锐、犀利和强悍,这样才能击溃心魔,主宰内心,这就是练级之旅,贵在持之以恒的投入。断念口诀就是为了提升觉察力,最后那四个字“觉之即无”,就是觉察到念头,念头就消失了。因为觉察力足够强大,看见念头时,念头就瞬间消失了,觉察力是极强的武器,就像激光炮,瞬杀念头怪。我每天也有静坐,静坐只是一个姿势,其实练习的就是观心,并不是往那一坐就可以了,有过站桩或者静坐经验的戒友应该知道,一旦开始站桩或静坐了,脑子里的妄念就会变得非常活跃,停不下来,通过练习观心,练习觉察,就能逐渐入静,念头一出现,就能让其消失。一开始肯定是嘈杂、妄念纷飞的状态,肯定是这样的,一次次被妄念带跑,一次次拉回来,就像拉锯战一样,你要知道,当你每次拉回来时,你的觉察力都在增长,一点点增长,渐渐你会看到自己的进步,拉回来的那一下会变快,反应速度会变快,发现念头的能力会变强,通过持之以恒的练习就能让内心安静下来,越来越能入静,觉察力强悍后,就能主宰内心了,刚开始肯定会被妄念带得七零八落。

\begin{quote}\it
    无声是勇士的艺术,静坐是勇士的剑。你有了这把剑,就能切断你的幻象。不过,有一点你必须明白:剑是否有用,取决于拿剑的人。
\end{quote}

\textbf{解析} 更准确地说,是静坐发展出来的觉察力是勇士的剑。有了这把剑,就可以斩断心智,最终切断心智造成的幻象。苏格拉底说得很明白,剑是否有用,取决于拿剑的人,这就是讲练习了,武器再好,如果不会用,能力差,那也无济于事,就像一把再好的枪,被一个枪法极差的人拿着,也很难发挥出威力。贵在练习,首先要正确理解,这样练习起来才能渐入佳境。

\begin{quote}\it
    勇士却以娴熟的技巧和透彻的理解,来使用静坐这把剑。他用这把剑,把心智斩成碎片,砍进思绪之中,暴露出思绪空洞的本质。你或许还记得亚历山大大帝的故事,他率领大军横越沙漠,看见两条粗绳绑成一大团复杂难解的结。从来没有人能打开这个结,但亚历山大毫不迟疑,拔出他的剑,用力一砍,结就断成了两半。勇士就该像这样去使用静坐之剑,你必须学会以这个方式攻击你的心智之结。
\end{quote}

\textbf{解析} 书里的这段话是我比较欣赏的,因为这段话有着斩钉截铁的力量,断念实战时就要狠一点,就像拔剑砍杀一样,坚决果断,干脆利落,毫不迟疑,绝不拖泥带水,绝不苟且妥协,实战时必须要强势,无比的强势,不要和心魔辩论,直接拔出觉察的剑,用力一砍,就完事了。亚历山大这个故事很好,实战时一定要强悍和勇猛,面对心智复杂的结,一刀下去,就解决了,断念要像勇士那样,以极为简明高效的方式犀利地解决战斗,剑已磨利,就等心魔上来送死!有的戒友还在想怎么和心魔辩论,而戒色高手直接拔剑砍杀,和心魔就不应该讲理,心魔擅长诡辩,歪理一套一套的,绝对的大忽悠,把你带到坑里。实战时不要辩论,直接觉察即可,心魔会针对你的思想误区发起怂恿,所以必须通过学习来完善觉悟,这样心魔的很多怂恿自动就会失效。

\begin{quotation}\it
    “那我需要什么呢?”我冲口而出。

    “更多的修炼。”他很快回答。
\end{quotation}

\textbf{解析} 苏格拉底的回答很干脆,更多的修炼!继续完成量的积累,最终迎来质的飞跃,很多戒友练习一段时间,感觉还不是很行,但应该看到自己的进步,比刚开始时肯定强很多了,已经学会观心了,只是觉察力还不是很强悍,还需继续强化。当你强到一定程度,就会突然发现心魔变弱了,心魔其实并未变弱,而是你的强大显得它变弱了,到时心魔变得很难攻破你,你的戒色状态开始变得稳固了。

\begin{quote}\it
    有那么一刹那,我觉得自己正从太空中某个有利位置,以光速在扩大,像汽球一般膨胀,不断向存在的最外极限涨大,直到我成为宇宙,再也没有分野。我已变成万事万物,我就是意识,体认到意识的本体;我是那道纯净的光芒,物理学家将之等同于一切物质,诗人则将之定义为爱;我是一,也是全部,让所有的世界都黯然失色。就在那一片刻,那永恒、不可知的,都在我眼前显现,呈现出就连笔墨也无法形容但确实存在的不朽。
\end{quote}

\textbf{解析} 这条是讲合一体验的,非常震撼的体验,也是我这几年一直在研究的体验。这个体验是苏格拉底给丹的,丹非常感激。\textit{意识扩散开来,我是一切,一切是我。……就像原子弹爆炸一样,唰,散开了,成了这个宇宙。(丁愚仁老师)} 我们的意识被局限在身体里,就感觉身体是自己,皮肤外面是外在的世界,和自己无关,这是分裂的模式,还有一种就是合一的模式,意识被释放出来,像原子弹爆炸一样极速扩张,成了整体宇宙,变成万事万物!进入永恒不朽的存在维度。大家看了这个体验,肯定会感到万分震惊,原来还有这种事情!还有这种可能!为什么之前一直不知道!我也被这个体验的描述彻底震惊了,不再局限在身体里,而是成了整体宇宙,这是怎样的体验!!!

萨古鲁在视频里也专门介绍过这个体验:“突然,所有的东西都变成我,我开始变成所有的东西,开悟最重要的体验,是与万物如是地合而为一。”萨古鲁在书里是这样描述的:“突然之间,我呼吸的空气,我坐的岩石,围绕着我的大气,所有的东西都变成了我的一部分,我变成了一个巨大的存在,我就这样坐在一块岩石上,泪水一直往下流,直到衬衫都湿了,我处于极度的狂喜中,在那个时刻,我就好像整个人都爆炸了,这种体验真的无与伦比。”

其他大师也有提到过这个体验,元音老人也开示过:“\textit{我们悟道后,快乐无穷,知道一切都是我自心显现的幻影,不再去追求。……亲证本性后,见一切事物都是我们的自性,没有你我他的分别。}”我到现在看过差不多十多位的大师介绍和描述这个体验,所以引起了我的高度关注,他们进入的是合一的源场模式,一切都是同一种能量的不同显现,能量可以显现为一棵树、一朵花、一座房子、一把椅子、一个人、一只动物、一阵风等等,但本质都是同一种能量的显现。我们现在活在分裂模式中,是什么造成的分裂?是心智!书中丹的顿悟:“始作俑者,正是人们的心智。”你误认为自己是身体,是心智,所以就被锁在思维的监狱里,无法进入到永恒的维度。

《三摩地》(Samadhi)影片中讲:“三摩地是指体悟到万物同一的状态。”《一个瑜伽行者的自传》里也提到了合一体验,也是上师赐予的。的确存在超越的体验,的确存在合一模式,只是我们不了解,不知道,我们平时关注的就是娱乐、游戏、女人、美食等,我们就像井底之蛙一样,死的时候两手一摊,两脚一伸,什么都带不走,迷茫地进入下一世的轮回,非常肤浅和无意义的一生。\textit{我们的真心遍满一切处。(黄念祖老居士)} 很多大德都提到了这句话,遍一切处,这是一个可以实际证到的体验,遍一切处,成为一切,成为整体宇宙,不再局限在身体里。

发生这个体验,是人一生中最重要的时刻,也不止这一生,可能千千万万生,几百万生,就是为了这个体验。丹获得这个体验后,以为自己都明白了,苏格拉底就说:“你的工作几乎还没有开始咧……你所见到的,只是幻象,而不是最终的经验。”开悟并非一劳永逸,还需继续修行,去除各种习气。最终的经验应该是整个梦境全部消失,梦境中的人事物全部消失,只剩下纯粹的觉知,但这个合一体验无疑已经是一种开悟的状态。

\begin{quote}\it
    这时我明白了,所谓觉察,指的就是人类体验到这股意识之光。
\end{quote}

\textbf{解析} 觉察就是真我的一个面向,真我就是空与觉察,真我就是纯粹的意识、纯粹的觉知。

\begin{quote}\it
    我领悟到当进入真正的静坐冥想时的所有过程——扩大觉察力,引导专注力,最终臣服于意识之光。
\end{quote}

\textbf{解析} 觉察力是提升和强化的核心,当一个人觉察力很强时,他自然就是专注的,学习和做事的效率都很高,也颇具创意。

\begin{quotation}\it
    他以问题回答问题,问道:“你想要看见什么东西的时候,是怎么做的?”

    我笑了:“嗯,注意看就是啦!你指的是静坐吗?”

    “核心就在这里,”他切着蔬菜,突然说,“静坐有两个同时并进的过程:一个是内观:注意逐渐冒出的思绪;另一个是放下:放下对冒出的思绪的挂碍。如此便能摆脱心智。”
\end{quotation}

\textbf{解析} 苏格拉底提到的核心,也就是内观,向内看,向内觉察,觉察力就是一股注意力,注意冒出的念头,注意力强,就可以直接觉察掉念头。另一个是放下,就是不要跟随,不要去关注,不要认同思维。苏格拉底讲的这两点都很好,就是教丹如何修心,如何摆脱心智的束缚。

\begin{quote}\it
    征服心智,只要你有兴趣的话。
\end{quote}

\textbf{解析} 苏格拉底的这句话很明了,征服心智,这样才能主宰自己,否则一个念头上来就可能让你彻底失控,进入疯狂纵欲、身不由己的状态,非常可怕的状态,就像失控的大货车一样,在马路上横冲直撞,最后车毁人亡。必须征服心智,这样才能获得身体真正的主宰权。

\begin{quote}\it
    你仍然认为你就是你的思绪,把它们当成宝贝一样,多方护卫。你误认为自己就是这个“心智”。
\end{quote}

\textbf{解析} 这条讲到了问题的关键,就是认同心智,误认为自己是心智,\textit{你不知道,这个贼,这个强盗,这个敌人,你不知道,他正是你自己妄心,所以咱们可怜。就迷,迷在此处。认贼作子,相信妄心,拿贼当儿子,最后绝对不能成就。你认贼为子呀,你以为是你自己的心哪,结果是什么呢,那是贼呀,那是害你的贼呀!这个是贼呀!那是认贼作子,由于认贼,那人修行就是煮沙做饭。自胜者,就是胜过了这个贼。所以《四十二章经》:“慎勿信汝意,汝意不可信。”(黄念祖老居士)} 我们一定要完成身份认知的转变,这点很关键很重要,否则修行就是煮沙做饭,修来修去,最核心的认知却没有掌握。

\begin{quote}\it
    你就是那意识,而非那带给你这么多困扰的幽灵心智。
\end{quote}

\textbf{解析} 苏格拉底的这句开示就是帮助丹完成身份认知的转变,真我是纯意识,纯粹的觉知,而不是那个冒充你的幽灵心智。幽灵这个词用得太好了,心智在脑海中出现,大家细心地体会那个过程,真的就像幽灵出现一样,一会出现一个念头,一会浮现一幅图像,一会微妙感觉开始渗透。如果你及时觉察它,它又会像幽灵一样消失得无影无踪,的确是幽灵心智,而觉察力就是这个幽灵的克星!觉察力一挥,这个幽灵就消失!

\begin{quotation}\it
    “苏格拉底,如果我并不是我的思绪,那我是什么?”

    他看着我,那副神情好像他刚说完一加一等于二,而我却问:“是,可是一加一等于多少?”

    他伸手从冰箱里抓出一颗洋葱,抛给我,“剥吧,一层一层剥。”

    他指挥道,我就剥了起来,“你发现了什么?”

    “另一层。”

    “继续剥。”

    我又剥了几层,“苏格拉底,只不过又多了几层。”

    “继续剥。”

    “剥光了,没东西了。”

    “错,有东西留下来了。”

    “是什么?”

    “宇宙。你走路回家时,好好想一想这件事。”
\end{quotation}

\textbf{解析} 剥到最后就是空,空就是最后留下的“东西”,真我是纯粹的觉知,纯粹的空,但不是死空,而是有觉知的鲜活的空。如果你顽固地认同于身体和心智是自己,那肯定会害怕空,害怕什么也没有,一旦真正认清了,就不会害怕空了,自然而然就接受了。我转变身份的认同大概花了两三年,刚开始也有点害怕空,后来慢慢就接受了,之前之所以害怕空,完全是因为误解,以为空什么都没有,好可怕啊,后来发现所谓的空只是去掉念头,纯粹地存在,没有念头,我依然可以存在,真正认清了,自然就不怕了,也坦然承当了。

\begin{quote}\it
    面对迫在眉睫的死亡,我的觉察力突破了心智的障碍。
\end{quote}

\textbf{解析} 这是苏格拉底在讲自己的一段修行经历,最后觉察力突破了心智的障碍,在面对危险时,人的觉察力往往会变得很专注,很强大,觉察力强了,就可以突破心智的障碍,心智是一种限制,会限制和遮蔽真我,而觉察的利剑可以斩断这种限制。

\begin{quotation}\it
    “苏格拉底,”我说,“你的身体周遭有闪亮的光,光是从哪来的?”

    “清净的生活。”他笑了笑。这时服务铃响了,他出去,表面上是替某人加油,其实是带给人欢笑。苏格拉底替人加的不只是汽油,也许还包括那种光辉、那股能量或情感。总之,人们离开时:往往会比来时还要快乐一点。不过,他最令我深受感动的,并不是那种光辉,而是他的纯真,他那干净利落、毫不拖泥带水的举止。我以前没有真正了解、欣赏这一切,而似乎我每学到一堂新的课程,就更深入洞悉苏格拉底这个人。我逐渐看清楚自己复杂的心智,在这同时,我领悟到他早已超越了他的心智。
\end{quotation}

\textbf{解析} 苏格拉底让丹戒色吃素是有道理的,对于提升整体能量和振动频率是非常好的,不再过放纵的生活,选择清净的生活,反而获得了纯净的大快乐,心灵纯净的人,他的身体周围好像有明亮的光辉。“苏格拉底替人加的不只是汽油,也许还包括那种光辉、那股能量或情感。”这段话是我由衷喜欢的,把自己的纯真、慈悲与幽默传递给别人,让人感受到巨大的鼓舞和激励,还有情感上的支持,这是多么好的一个人,苏格拉底的确已经超越了心智的限制与束缚,他的教导简单有力,切中修行的要点。

\begin{quote}\it
    你需要灌入分量十分庞大的气,才能冲破心智的迷雾,找到通往大门的路。
\end{quote}

\textbf{解析} 中医会讲气,人的能量足了之后,的确会产生一种微妙的气,手淫后照镜子会发现某种气被抽走了,气色一下就差了,意淫暗漏也会导致气色下降。人是有气场的,有的人往那一站,不说话,就能让人感受到他强大的气场和威严。气是非常微妙的能量状态,养足能量,才能更好地修行,一个纵欲的人很难在修行方面取得进展,一定要遵守戒律,把身体养好,有了精气神,这样修行方面才能取得突破。苏格拉底讲的气,应该也指一种强大的精神力量和上师的加持,只有这样,才能冲破心智的迷雾,看到真相。

\begin{quotation}\it
    “你目前的饮食或许的确给了你‘充沛’的能量。”他边说,边抬头看着一棵漂亮的树,阳光透过枝丫洒落地上,“但也使你昏沉无力,影响你的心情,并且削弱你的觉察力。”

    “改变饮食又怎么会影响我的能量?”我辩驳道,“我的意思是说,我摄取热量,而热量代表着能量。”

    “在某种程度上,这话并没错,可是勇士必须体会到更微妙的影响。”
\end{quotation}

\textbf{解析} 苏格拉底教给了丹饮食方面的注意,不要吃肉和喝酒,因为这样会拉低他的振动频率,吃肉的人很容易昏沉,也容易发怒,欲望也会加重。动物被宰杀时会感受到极大的恐惧,还有嗔恨,这两种低频能量会影响到血肉,吃肉的人自然会受到这两种低频能量的负面影响。虽然吃下去的时候感觉很亢奋,似乎津津有味,但是仔细体会吃肉后带来的微妙影响,就会知道吃肉的一些弊端。现在我已经吃素了,现在我吃肉会生病,身体已经很敏感了,不宜吃肉了,也不喜欢肉食了。以前我是无肉不欢,那种饮食结构会让我昏沉和易怒,而现在吃素,心情祥和很多,不再容易昏沉,也有利于培养觉察力。刚开始我没有完全吃素,而是从减少吃肉开始,后来身体适应后就吃素了,现在已经完全适应了素食的饮食结构,一切都很好了。饮食方面真的要很注意,尽量吃新鲜的蔬菜、水果、五谷杂粮和豆类等,这些食物的振频较高。

\begin{quotation}
    “我怎么可能达到每一项要求?”

    “想想佛陀对弟子说的最后一句话。”

    “什么话?”我问,等待开示。

    “尽力而为。”他话一说完,便消失在人群中。
\end{quotation}

\textbf{解析} 尽力而为,这四个字非常好,不要太注重结果,关键是尽力而为,注重过程,如果在过程中你真的尽力了,认真去做了,自然会有一个相对满意的结果。不要还没开始就丧失信心,觉得自己做不到,应该要尽力而为,努力去做,就像登山一样,看着一座高山,觉得自己爬不上去,但人家双腿有残疾的人都爬上去了,凡事尽力而为,不要想太多,不要害怕失败,只要真正尽力了,失败也是可以接受的,失败只是暂时没有成功,只要你继续努力,肯定会有成功的一天,不能被失败打败,而是要从失败中强势崛起,从失败中吸取教训,不断完善自己。估计苏炳添一开始也没想到自己可以跑那么快,他也只是尽力训练,尽力而为,苏炳添也输过很多次,但他并不气馁,还是尽力做好训练,一点点强大自己的实力。只要你尽力去做,就可能越做越好,最终达到的成就超乎你的想象。

\begin{quote}\it
    巨汉代表你一切苦恼的本源,他就是你的心智。他是你必须刺穿的恶魔,可别像那被击倒的勇士一样,被他欺骗了,集中注意力!
\end{quote}

\textbf{解析} 用巨汉来形容心智很贴切,心智的确很强大,就像巨汉一样,而你刚开始就像小孩,被巨汉耍得团团转,心智会冒充你,会一次次带跑你,控制你。刚开始你没有任何觉悟,也不懂得断念,肯定会被巨汉蹂躏,虐得遍体鳞伤,体无完肤。一个念头上来,没断掉,你就会沦为疯狂的撸管肉机,要战胜巨汉,就要强化觉察力,也要提升和完善你的觉悟,提高认识水平,识破巨汉的招数和套路,彻底击败巨汉。(心智代表所有妄念,而心魔代表负面的念头,会导致破戒的念头,心智的范畴更广。)

\begin{quote}\it
    我还需要更多的练习,才有能力把无比锐利的注意力,灌注到日常生活中的每一分每一秒。
\end{quote}

\textbf{解析} 练习使人强大,注意力强大了,就不会被念头带跑,念头上来也能及时发现,很多戒友反馈的问题就是缺少觉察,跟着念头跑了好一会才发现自己在意淫,这时候欲火已经开始燃起来了,火势已经很大了,很难扑灭了。戒色高手一直严密注意内心的活动,邪念、图像、微妙感觉一上来,马上就觉察消灭,非常锐利的注意力,就像快刀切念一样,一下就结束战斗了,那一下觉察充满了爆发力。丹说自己需要更多的练习,的确如此,每一分每一秒都要看住自己,要时刻保持警觉,严防心魔的进攻。一位戒友说:“心里想着只看一会不撸,结果还是没控制住。”只看不撸就是心魔的怂恿,他没识破,结果就中招了,必须学会识破心魔的套路,及时断除怂恿的念头。另外一位戒友说:“感恩飞翔老师,学习了您的文章并且在实战中加以运用之后,对于破解心魔的怂恿,我更有信心了。因为自己情绪的变化,今天出现了好几个怂恿念头,但都被我觉察到了,以前这个时候我一般都直接跟了念头,而这次,我一记觉察把怂恿念头打回原形,我体会到了断除这种念头的感觉,很不错,念起即觉,觉之即无。”这位戒友做得很好,真正理解断念口诀,坚持练习,熟练应用,就能决胜实战。

\begin{quote}\it
    你曾经沐浴在光明之中,曾经在最简单的事物中找到喜乐。
\end{quote}

\textbf{解析} 苏格拉底在唤醒丹的儿时记忆,小孩的特点就是很容易开心和快乐,小孩有一项能力就是能在最简单的事物中找到喜乐,比如一块小石头、一朵花、一棵树、一只鸟,都能让小孩感到无比的喜悦和开心,这是成人很难感受到的纯真喜悦,一定要内心非常干净,才能感受到。成人的脑子被邪念占据着,也被其他各种念头占据着,不再单纯,不再干净了,充满了各种欲望,这样就失去了在最简单的事物中感受快乐的能力。

\begin{quote}\it
    每个婴儿都活在明亮的花园中,直截了当地感受一切,不受任何思绪的欺瞒,没有信念,没有诠释,而且不下判断。你开始思考,开始替事物命名并知晓事情时,就堕落了。要知道,堕落的不只是亚当、夏娃,而是我们所有的人。
\end{quote}

\textbf{解析} 孩提时代感觉这个世界很明亮,内心很喜悦,邪淫后感觉这个世界很灰暗,内心很沮丧。婴儿的脑子是空的,直接体验一切而不去命名,婴儿处于纯粹觉知的状态,渐渐长大了,父母开始教他说话、认字,慢慢会思考了,失去了直接体验,就堕落了,这里的堕落指的是从纯粹觉知的状态进入被心智奴役的模式,这就是堕落,当然普通人是不知道这点的,他们以为心智就是自己。

\begin{quote}\it
    你在孩提时代有过的乐趣,是可以再度拥有的。耶稣说过,要进入天国,得先像小孩一样。
\end{quote}

\textbf{解析} 苏格拉底在教丹快乐之道,苏格拉底这个人就像一个老小孩一样,有慈悲的一面,有幽默的一面,有优雅的一面,也有纯真的一面,非常有智慧有魅力的一位老人。“要进入天国,得先像小孩一样。”要像小孩子一样单纯、纯真和纯净,这样才能进入天国,如果你的内心很污浊很龌龊很邪恶,无疑对应的就是地狱。每个人都应该学会净化自己的心灵,提升自己的灵性,让灵性的光辉再次围绕自己,变得圣洁而庄严,这样不仅能再次获得儿时的乐趣,也能开发更高的智慧。

\begin{quote}\it
    你惟一的投资是修炼。
\end{quote}

\textbf{解析} 苏格拉底的这句话我太认同了,真正惟一、真正有价值的投资就是修炼,内心的修炼,觉察力的修炼。很多戒友学会修心后一定会知道修心是多么重要,之前在学校里学到的任何知识和技能都无法和修心相比,这是最高殿堂的知识,是最值得学习、最有价值的知识。一个人一定要学会修心,提高自己的修为,真正主宰自己的内心,做身体和念头的主人。一位戒友说:“在一番冲突后将控制权直接交给心魔,直接做了它的傀儡,控制权如此容易地被抢夺,我真的没能想到。”心魔很容易就接管了他的身体,一个念头、一幅图像没断掉,控制权就会移交给心魔,然后就开始为非作歹,做尽荒唐事。如果你修炼出很强的觉察力,心魔就无法得逞,你就能牢牢地把控制权握在自己手里!

\begin{quote}\it
    苏格拉底向我揭示了,觉察就是宝藏。
\end{quote}

\textbf{解析} 核心就是觉察,能觉察就有望主宰内心,缺少觉察就会被念头带跑,身不由己,陷入内心的喋喋不休、自言自语,真我被妄念的噪音所覆盖。能觉察就能消除内心的噪音,回归真实的宁静与祥和,就能活出真我。觉察是真正的宝藏,也是最值得练习的。走路、吃饭、刷牙时都要保持观照,保持觉察,保持注意,不可分心走神,掉入心智的陷阱。每天生活中就可以练习,很多人吃饭时还在想东想西,胡思乱想,对不对?每个人应该都有体会,甚至吃饭时,邪淫的念头和图像都会上来,我自己也有过多次体会,关键就是时时观照,时时觉察,能觉察就能切开心智的束缚,就能获得自在,否则一个念头连着一个念头,还会裹挟着各种微妙的情绪,让人很不自在。如果是邪念,那危害就更大了,觉察就是心智的克星,你能觉察,就能制心。苏格拉底的核心教导其实很简单,就是通过训练提升觉察力,征服心智,尽力而为!

\begin{quote}\it
    心智却如幽灵,只活在过去或未来,它惟一的力量就是,转移你对当下这一刻的注意力。
\end{quote}

\textbf{解析} 当下这一刻,是无念的纯粹觉知,而心智幽灵则会把你带入过去或未来,一会陷入回忆,一会幻想未来,心智就是一直想、一直想,想个不停,想各种问题,想各种内容,心智就是通过想这个行为把你困住的,你肯定会发现自己停不下来,有强大的惯性迫使你胡思乱想,你会发现 90\% 的念头是无意义的,就是乱想,内心非常混乱。心智幽灵一次次把你带入自言自语、喋喋不休的状态,让你偏离真我。

\begin{quote}\it
    苏格拉底又说:“丹,所有的世人都被困在自己的心智所造成的洞穴中,无法自拔。只有少数勇士看见光明,挣脱束缚,放弃一切,因而能笑着走进永恒。我的朋友,你也会如此。”
\end{quote}

\textbf{解析} 在书的最后,丹和苏格拉底去了荒野的一个洞穴,苏格拉底从背包里拿出一捆木柴,点燃后他们的身体在洞壁上投射出怪异、扭曲的影子,苏格拉底讲起了一个故事:“以前有一个民族,终生都住在幻象洞穴里。数代之后,他们逐渐以为自己投射在洞壁上的影子,就是真实的实体。”苏格拉底继续说:“丹,古往今来,都不乏有福之人,他们从未受制于洞穴。有些人厌倦了影子的把戏,产生疑问,不管影子窜得有多高,都不再能令他们满足。他们成为追寻光明的人,其中少数幸运儿找到向导,向导指点了他们,带领着他们走出幻象,走进阳光中。”苏格拉底又说:“丹,所有的世人都被困在自己的心智所造成的洞穴中,无法自拔。只有少数勇士看见光明,挣脱束缚,放弃一切,因而能笑着走进永恒。我的朋友,你也会如此。”这个在洞穴的教导真的太好了,可能是全书最好的一个教导,心智的全息投影骗了几乎所有世人,人们都把这些投影当真的了,这些投影是立体全息的,看似真实,其实是一种投影。现代的全息投影技术已经出现了,有些已经做得很逼真了,而我们误认为真实的现实,其实也是一种投影。

智者知道这一切,知道真相,\textit{用这一种方法,把你的妄心息下来,心息下来,影子消了,那时候,身心世界,乓!都没有了,镜子现前了,你看见,啊,都是我的佛性啊!所以这样子叫做明心见性,哎,见到本性了。所以我们念佛也好,持咒也好,参禅也好,念到后面,就入禅定了,身心世界都化空了,都没有了,都没有了,相消了,真心现前了。所以一切法是一个法,都是你的心,就把你心清净,没有两样。(元音老人)}

\textit{所有的现象只不过是自心的投影,一旦你知道什么是绝对的真理,你将认清眼前所有的相对现象只不过是一个幻影,一场梦,并且不再执著于它。(顶果法王)}

憨山大师两次开悟体验:

\begin{quotation}\it
    一日粥罢经行,忽立定,不见身心,唯一大光明藏,圆满湛寂,如大圆镜,山河大地影现其中。及觉则朗然,自觅身心,了不可得。

    一夕静坐夜起,见海湛空澄,雪月交光,忽然身心世界,当下平沉,如空华影落,洞然一大光明藏,了无一物。
\end{quotation}

憨山大师的这两次开悟都提到了“影”,第一次是影现,第二次是影落,第二次更彻底一些,连影子都没了。大光明藏就是真我,真心。

\textit{物质是浓缩或冷冻的光,一切物质都是光的浓缩,以平均小于光速的速度,反复地以特定模式运动。[David Bohm(戴维·玻姆)]}

在玻姆的隐秩序理论中,高次元的隐秩序层级是具有所有可能量子态的“能量海”,而我们的三维空间只不过是从隐秩序中特定量子事件所投射或绽放出来的一种呈现(玻姆称为显秩序层级)。玻姆认为隐秩序层级的能量海,由包含所有电磁波频谱的广义的“光”所构成,光在显秩序层级的来回卷缩与绽放(folding and unfolding)中被凝聚或冻结(condensed or frozen),而形成我们三维空间中物质的稳定存在。也就是说:万物皆由能量形成,物质是浓缩凝结的光(光是振动频率极高的能量,真我就是光海)。

《三摩地》(Samadhi)影片中讲:“他们一生都被锁在洞穴里。”这个洞穴、监狱、笼子就是心智,就是念头,就是思维,而人们把心智当成自己,跟着念头跑,被牢牢锁在了这个洞穴和监狱里,看不到真相。一世又一世的轮回,一直重复这个模式,直到某一世遇见智者来指点他,唤醒他认识真相,教给他修心的方法,通过修心就有可能证悟真我,亲自看见真相。

\begin{quote}\it
    我没怎么开口说话,但不时发出笑声,因为我每次环顾四周,看着大地、天空、太阳、树木、湖泊和溪流,就会领悟到,这些通通是我,其间根本没有分野。
\end{quote}

\textbf{解析} 在洞穴的深处,苏格拉底强悍地给了丹一次觉醒的体验,这次是强制的,非常强悍,书中那一段看得我惊心动魄,内心也跟着紧张起来,读下去就释然了,苏格拉底扯下了丹的分裂滤镜,让丹亲自看见了合一的真相,领悟到一切都是自己。\textit{虽然现在看起来是无数的身体和分离,但当你看到真相时,不管你往哪看,看到的都是绝对的一,看到的都是你。(《终极自由之路》)} 界限消失了,一切都是自己, \textit{观者即所观之物。(克里希那穆提)} 之所以观察者和他的所观之物分裂为二,就是因为被心智障碍和限制,导致看不见合一的真相。

《幕后:一位觉者的实修日记》里面也讲到了这个合一体验:“我看见了我的后院,但它已经不再在我之外了,在我看来,我就是院子,我就是天空,我就是大地,我所看到的每样东西,我都是,在世界与我之间已经没有了分离感。”对真相感兴趣的戒友,一定不要错过幕后这本书,我开始对探索真相产生浓厚兴趣,就是从幕后这本书开始的,里面讲了很多体验,很不错的一本书,书薄料猛,非常震撼。

《内在与外在的联系》(Inner Worlds,Outer Worlds)这部影片里讲到:“当思维停止,那么你将会看到事物的本质,所有方面,树木、天空和地球、雨露、恒星都是一体,生命和死亡,自我和他人是一体,正如山脉和峡谷是一体……也即是一切都是连接到同一个振动源,有一个意识体,一个能量场,一种力量贯穿所有。”“当你从一场梦境中醒来,你会意识到梦中的一切都是你,正是你创造了它,每个人,每一件事情都是你。”“思考能创造分离,是一种有限制的体验,我们越是与思想同步,就越会远离源头。”“所有问题都来源于自我思维,你并不是你的思维。”

\textit{开悟意味着超越思维。(托利《当下的力量》)}

释迦牟尼佛在菩提树下证悟到宇宙人生的实相,第一句话即说:“奇哉!奇哉!一切众生,皆具如来智慧德相,但因妄想执着,而不能证得。”佛陀把他开悟的真谛,一语道破,原来所有众生都是平等无异,皆有佛性,都可以成佛,佛是觉者,觉悟了宇宙真相的人。现在我们因有妄想执着,真心被遮蔽了,所以沉沦苦海,在六道轮回。妄想执着就是心智,A prison for your mind!被心智的监狱锁住了,所以无法证得。

\begin{quote}\it
    你是世界,你是宇宙,你也是你自己和所有的人。一切都是上苍的美妙演出,醒来吧,重拾你的幽默,别担心,你自由了!
\end{quote}

\textbf{解析} 苏格拉底说:“丹,开悟并不是一种成就,而是一种体会。你醒来时,一切都改变了,同时又什么也没有改变。”见山是山,见山不是山,见山还是山。\textit{要知道它是梦,就是自心所幻出来的,从梦中醒过来,你一醒,是个梦!梦一醒全没有了,醒来就是!你醒了之后明白了,原来是个梦!根本是梦!(黄念祖老居士)} \textit{一醒,是梦!都没有了。(元音老人)} \textit{你们看到的整个宇宙都是一个梦,目前你们都在做梦。假设你在睡觉,我出现你梦中,告诉你:‘你此刻所见所闻都是一场梦!’可你不信。但醒来后便知我所言属实。事实上只是一场梦。然而,只要我不打开你的内眼,你就不会相信。(印度圣人美赫巴巴)} “生活的目的是从梦中醒来。”

美赫巴巴那句语录很有深意:“\textit{我不是来说教,而是来唤醒。}”

托利在《Stillness Speaks》里说:“\textit{灵性的觉醒就是从思维的梦境中醒来。}”

\textit{总有一天我们都会从梦中醒来,并看见这是一个梦境,然后对整件事情大笑一场。同时,在梦境中我也试着唤醒其它的人,如果他们想要醒的话。从分离的梦境中醒过来,在这之前,一切都和你是分离的;这之后,一切都在你里面。在这之前,世界似乎是极度的真实;这之后,你会视生命的全貌是一个梦境。(莱斯特·利文森)}

从虚拟现实的梦境中醒来,亲自看见真相!苏格拉底之前就叫丹醒来,最后也给了丹醒来的体验,丹无疑是幸运的,遇见了苏格拉底,一位真正的智者。

希望我也能早日醒来,也衷心希望每一位戒友、每一位众生都能早日醒来。亲自看见真相!!!

\paragraph*{总结}

《深夜加油站遇见苏格拉底》的确是一本好书,浅者获得了激励,从而告别消极颓废,积极奋斗自己的人生,尽力而为。深者则领悟到了更深的修行道理,所以是一本比较普适的书,能够畅销也是因为它的口碑的确很棒,任何人都可以从中获得启示和激励。平心而论,书比电影要精彩,电影经过了改编,有的内容没拍出来,特别是那几个高峰体验没有拍出来,最后洞穴里的教导也没拍,2006 年的片子,可能特效还不行,如果能用《奇异博士》里的特效来表现高峰体验,那就很震撼了。总而言之,电影拍浅了,没把书籍真正的深度拍出来,但也不失为一部激励人心的电影,可以让人感受到书籍里的一些场景和氛围。

\textit{一切境界唯心妄动,心若不起,一切境界相灭,唯一真心遍一切处。是故三界虚伪,唯心所作。(《起信论》)}

我们来到这个世界上,进入学校读书,会学习哲学的唯心和唯物论,有的人会误认为佛法是唯心的,其实佛法既不是唯心,也不是唯物,哲学里面把心跟物分做两块,而佛法是心物一如。朱清时教授讲的“心物一元”,即意识和物质的统一,就像睡觉时,你潜意识动了念头,然后变幻出一个梦世界,有人事物,感觉很真实,醒来后才知道是梦,这个梦是念头导致的,醒来后的现实还是一个梦,还能再醒一次,见到真相。昨晚的梦境是梦中梦,醒过来还是梦境,我们生活在两层梦境的世界,你以为早晨醒来就是现实,其实你被骗了,早晨醒来还是梦境,你只是在梦境中做了一个梦,其实醒来还是在梦境中。两个梦境的区别,床上做的梦,梦中场景是跳跃的,不是很连贯,而现实的梦境场景是连贯的,这样让你更加难以察觉到这是一个梦境。早晨醒来,你还是在梦境中,除非你能再次醒来,才能摆脱这个梦境。两个梦境都是念头导致的,如果你懂得修心,就有望从梦境中真正醒来。从昨晚的梦境中醒来,你还是有一具身体,因为你还是在梦境中,等你从现实这个梦境中醒来,你会发现自己的真实身份是无形的纯粹觉知。哲学家的逻辑思辨还在心智的范畴内,所以看不到真相,但他们对哲学的发展还是有一定贡献的,目前有些科学家已经知道宇宙是一个梦,一个幻象,但他们没有亲自看见真相,而修道的圣人则是亲证了真相,所以才是真正的智者。虽然知道是一个梦,但还是要按游戏规则来生活,这点很重要,因为梦未醒,你还是要因为作恶而受苦,所以必须要谨慎对待因果,不要犯邪淫,不要杀生,要改过迁善,好好孝顺父母,多做善事,努力积累正能量。

这季很特别,把我这些年研究真相的体会和大家做了一个分享,相信会引起大家对探索真相、了解真相的兴趣,人生不仅仅是吃喝拉撒,还有更高更神圣的目的,那就是觉醒!从梦中醒来!强化觉察力,做修心的勇士,主宰自己的内心,突破心智的限制,持之以恒地修行,迟早会水到渠成的。即使这辈子没能醒来,你也能过上一个正能量的人生,无愧此生了。

最后讲一个笑话,记得看完这本书之后,我出门看见房顶,都要在平地上跳一跳,想象自己一下跳到了房顶上,哈哈哈哈哈哈!苏格拉底爽朗的爆笑响彻了整个宇宙,彻底粉碎了整个投影,整个梦境……

下面分享一首诗歌。

\begin{poem}[梦中人]
    \begin{multicols}{2}
        \centering~\\
        从昨晚的梦境中 \\ 我醒来 \\ 来到了这个称之为“现实”的梦境 \\ 自从我早晨醒来 \\ 我就知道昨晚是一个梦 \\ 一个很逼真的梦 \\ 我在梦中有一具身体 \\ 走在马路上 \\ 场景一下跳跃到室内球场 \\ 我和某位NBA著名球星在打球 \\ 他叫我防守时要把双手举起来 \\ 我和他还说了几句英语 \\ 梦中的一切是那么逼真 \\ 在梦中我丝毫不怀疑这是一个梦 \\ 醒过来时,梦中的一切都消失了 \\ 我来到了所谓的“现实” \\ 我也有一具身体 \\ 有围绕着我的一切 \\ 我在这个现实中刷牙、洗脸、吃饭 \\ 上网、看电视、出门跑步等等 \\ 我丝毫不怀疑这是一个更逼真的梦境 \\ 直到上师告诉我 \\ 这是一个虚拟现实的梦境 \\ 生命的目的是从梦中醒来 \\ 亲自看见真相 \\ 亲证自己的真我 \\ 我在自己制造的梦中生活 \\ 在梦中欢笑,在梦中受苦 \\ 一世又一世,从未觉醒 \\ 自从上师告诉我真相 \\ 我决定醒过来 \\ 我渴望醒过来 \\ 亲自看见真相 \\ 我在你梦中 \\ 你在我梦中 \\ 我们都是梦中人……
    \end{multicols}
\end{poem}

下面推荐一本书。

\begin{book}[《幸运之书》,埃克哈特·托利]
    这本书语言精练,充满智慧,以格言的形式讲述朴素的真理,有长段落,有短句子,错落有致地将你带入自己内在的世界,探索心灵深处的秘密,去发现生命的价值与意义,去发现自我的本质,进而找到内心的平静和生活中的幸福。托利认为,当你与内在的宁静失去了连接,你也就失去了与自己的连接。当你失去了与自己的连接,你就会迷失在这个世界里。与外部环境的噪音相对应的,是内在思想的嘈杂;而与外部环境的安静相对应的,是内在的宁静。托利说,“一树、一花、一草,让你的觉知停留在它们身上,你会发现它们是如此的宁静,如此深植于存在之中。”当你看着一棵树,并感觉到它的宁静,你自己也会变得平和起来,这就是你在与这棵树在很深的层面上相连接。书中写道:“每个人都渴望幸运降临在自己的身上,幸运并不是一个人不用任何努力就可以得到好的运气。当下正是人生真正的幸运。专注于当下,你会集聚全宇宙的能量,而这种能量就是幸运。”当下就是纯粹觉知的片刻,真我显现的片刻,安住当下才是真正的幸运。张德芬在为《幸运之书》撰写的序言中写道:“埃克哈特·托利是我十分心仪的灵性导师,更是我心目中伟大、颇有深度的心灵作家。托利的新作《幸运之书》虽然字数不多,却很可能会是改变某些读者一生命运的书。”托利也是我个人比较喜欢的灵性导师,之前也分享过《当下的力量》的笔记与解析,这季推荐《幸运之书》,再推荐一个视频,可以百度搜:埃克哈特·托利在斯坦福大学的谈话,这个视频质量很高,也很清晰,一些关键的要点都讲到了,是很不错的一个视频,希望大家都能学习下。
\end{book}
