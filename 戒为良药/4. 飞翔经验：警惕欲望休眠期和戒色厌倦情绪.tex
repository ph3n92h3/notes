\subsection{警惕欲望休眠期和戒色厌倦情绪}\label{4}

每天都有无数破戒的戒友,破戒后的懊恼和对自己的失望是不难理解的,相信每个戒友都反复经历过。从开始发誓戒除到屡戒屡败,再到最后的彻底戒除,是一个漫长的过程,这个过程就是改造思想意识的学习过程,是一个思想认识不断提高的过程。这季就欲望休眠期和戒色厌倦情绪专门谈一下。

反复破戒的戒友,如果你有点中医常识,你就会发现:破戒也是有套路的,也是有规律可循的。破戒的规律和“肾气值”密切相关,当你感觉身体不行了,你就会本能地想戒色,当你戒色一段时间,肾气开始有所恢复,就特别容易破戒。刚开始戒色会经历一段欲望休眠期,很多人在这个阶段心瘾不重,还是能自我控制的,但欲望休眠期特别容易放松警惕,以为戒色成功了,其实欲望只是暂时休眠,等你肾气恢复到一定程度,它就会苏醒,这时候破戒往往是变本加厉地 SY,前功尽弃。欲望休眠期因人而异,有人只有三天的休眠,有人是二十天,有人是六十天,据我了解,欲望休眠期平均为三周左右,也就是 21 天左右。欲望休眠期接下去就是破戒高峰期,破完戒就是心理后悔期,这时候自我否定和后悔感会比较强烈。

这篇文章主要就是要让大家认识到欲望休眠期,在这个时期要提高警惕,保持警觉,不要放松学习,只有不断学习戒色知识才能提高你的戒色觉悟,才能提高你的戒色定力等级,戒色的定力等级和打网游练级是一个道理,刚开始等级低,被心魔虐,等你不断学习戒色知识,定力等级上去了,心魔就不是你对手了,就动不了你了。如果你不学习戒色知识,不开悟,定力等级永远那么低,遇见心魔,结果可想而知,遇见一次失败一次,看到黄源一点定力都没有,一看到心马上乱,马上跟着点击,这就完了,没有免疫力和抵抗力。

\begin{center}
    发戒心 $\to$ 欲望休眠期 $\to$ 肾气有所恢复 $\to$ 破戒高峰期 $\to$ 心理后悔期 $\to$
\end{center}

这就是破戒的过程,也就是怪圈,很多人几年,甚至十几年都出不了这个怪圈,我曾经就是十几年出不了这个怪圈,因为那时我没开悟,没有学习戒色知识和养生知识的意识,就是强戒和盲戒,所谓强戒就是以为靠意志力就能戒掉,其实强戒注定失败,因为戒色境界根本没有提升。盲戒,就是不学习戒色知识瞎戒,盲戒也注定失败,因为戒色境界也没实质的提升。

人脑“中黄毒”的机制和电脑“中病毒”的机制很相似,电脑中病毒后会影响系统的运行,同样,人脑中黄毒后,身体健康就会出问题,一般最先出现问题的是泌尿系统疾病,以前列腺炎为主。然后肾一虚,肾虚百病丛生,什么毛病都可能出现,脑力下降也很普遍,中医:肾上通于脑。SY 伤肾必伤脑力,记忆力和注意力都会下降。脑力不行,学业和事业都会受到很大影响。

再来谈谈戒色厌倦情绪。

戒色厌倦情绪实在是太普遍了,就像厌倦一件衣服,一道菜,一个手机一样。戒色知识看多了也是会让人厌倦的,一旦出现厌倦情绪,戒色就失败了一半,所以一定要学会调整心态和情绪,做好情绪管理,一出现马上调整,养成良好的阅读和学习习惯,做到每天学习戒色知识不放松,一般养成习惯后就不大容易厌倦了,就好像刷牙一样,习惯了就成自然了,哪天不刷牙也许你就会觉得不舒服,戒色知识的学习也要找到这种状态,当然不一定是戒色知识,养生类知识也很不错,因为养生和戒色是相通的。另外,一篇戒色文章其实可以看很多遍,因为温故而知新,你看得越多理解的程度就越深,而不是浅尝则止,看过了就忘。

\paragraph*{结语}

孙子兵法有云,知己知彼百战百胜,要戒色就要知道怎么破戒的,为何出不了破戒的怪圈,这是每个戒友都要深入思考的问题,等你想明白了,戒色的境界就上去了,如果你还停留在戒色初级阶段,强戒盲戒,那注定只有失败。只有不断学习,才能开悟,哪天你顿悟了,境界就不同了,定力等级就上去了,离彻底戒除就不远了。加油!
