\subsection{戒油子的蜕变与逆袭}

前言:

肉弹的春夏季攻势又开始了,大家走在马路上要保持警惕,非礼勿视,做好视线管理,不慎看到马上避开,不要看第二眼。如果你走在马路上,出现了枪战,这时你的反应是什么?肯定是想办法尽快避开,而现在进入春夏季,一场看不见硝烟的战斗正在激烈进行着,每天都有人阵亡,很多人看见外境的诱惑后,心里起了邪念,回家就看黄撸管,所以我们必须做好视线管理,不要被肉弹干掉,如果你盯着看,那就是在送死,自取灭亡。肉弹实在太狠了,我们一定要警惕、警惕、再警惕!管住视线,看住自己的念头,实战意识一定要强,菩萨见欲,如避火坑,凡夫见欲,飞蛾扑火!肉弹向你猛烈倾泻过来时,唯一正确的选择就是马上避开,实战中第一反应一定要正确,必须牢牢守住戒色的阵地!

我现在很理解阿拉伯国家的女子为何天天穿长袍,面纱更是将脸部裹得严严实实,只留下一双眼睛用以观察。她们那样穿,让男人无法见到她们的身材与容貌,这样也就不会让男人起邪念了,我国古代女子的穿着也大多是长袍的样式,大门不出,二门不迈。现代社会都在宣传性感,女子身体暴露的面积越来越大,直勾人的眼球,所谓性感无非就是让男人起邪念,起邪念后身体就容易漏,宣传性感其实就是在把大众导向纵欲主义。现在这种邪淫文化的宣传已经渗透到很多领域,包括很多网游都在大做这方面的噱头,可怜很多孩子年纪尚小,就被黄毒所染污。上次看过一个案例,一个十二岁的孩子就得上了尿毒症,怎么得的?疯狂撸管,一天多次,完全失控,完全走火入魔,尚未完全发育的身体遭受到了严重的摧残,十分可怜。我们一定要管好自己的视线,非礼勿视,万花丛中过,片叶不沾身。任其诱惑猛烈,我自视而不见!

最近有戒友反馈举报黄网不成,反而把自己栽进去了。很多戒友都在举报黄网,有时会去查看是否举报成功,这时候有可能会再次栽进去。黄网对于很多戒友来说就像心里的一个疙瘩,因为知道某个黄网,所以心魔就会怂恿他去看,很多新人恨不得把所有黄网都举报掉,这种想法是好的,一个无毒的环境对于戒色是比较有利的,但实际情况是难以达到的,因为有新的黄网会出现,何况还有很多擦边图和擦边新闻存在,提高自己的觉悟和定力才是关键。黄网可以去举报,但尽量不要再去点开那个网址,可以在举报的网站查看是否举报成功,如果再去点击黄网,万一没举报成功,那很可能会再次栽进去,就像栽进粪坑一样。有好些戒友就因为举报的问题而破戒,在举报黄网这个问题上,自己一定要保持谨慎。另外再说下反戒分子的问题,现在反戒势力分为三股:75 是第一股,也就是“自慰砖家”,真实身份是网特,老底是轮子;卖邪淫资源和性用品的人是第二股;被无害论洗脑、甘于堕落者为第三股。这三股反戒势力都有自己的贴吧,他们会来戒色吧拉人下水,特别是新人比较无知,辨别力不强,很容易动摇自己的戒色立场,所以在这季特别提醒新人不要上了反戒分子的当,反戒分子惯用的招数就是造谣、诽谤、冒充、诬陷和混淆是非。曾经见过一位反戒分子说戒色吧影响了他的生意,他所谓的生意就是卖性用品,现在戒色吧每天都有卖邪淫资源的人跑来发广告,一个资源卖二十元,吧务看到都会马上删除。戒色之路不是平坦的,既要打败自己的心魔,又要经受住外界的各种考验,没有正知正见、没有坚定决心的人是很难坚持下去的。对于反戒分子,不要理会即可,不要去争吵和辩论,以免影响自己的情绪。反戒分子肯定会存在,有正就有邪,但我相信邪不压正,我们必须坚定自己的戒色立场,不要被他们拖下水,真正有戒色正知见的人对反戒分子根本不屑一顾,他们的歪理邪说漏洞百出,根本不符合事实。

下面分享一些案例。

\begin{case}
    飞翔哥您好,我来戒色吧有十个月了,前半年不停地破戒,症状也一直在加深,但如今戒色已然四月有余,感觉焕然一新,不论是胆量、成绩,都有质的飞跃。前几天看自己的照片,发现比原来好了很多,同时似乎还有一股无形的气场环绕。回首往事,忽然发现纯净灵魂的美好。您曾在一首诗里写道:“当你最终爬出撸管地狱,呼吸到第一口纯净的空气时,你甚至会嚎啕大哭。”我现在真的是这种感觉!戒色逆袭比撸管爽多了,而且是全方面全天候的真实大爽,不是一时一地的多巴胺骗局。感谢戒色吧!
    \subparagraph{附评} 这位戒友戒色四月有余,他真正体会到了戒色的美好,有一种焕然一新的感觉。罗大伦博士在《朱丹溪如何看待女色》里说道:“还有许多小伙子,在各种色诱下长期坚持手淫,甚至每天一次,坚持了许多年(有很多是从初中就开始的)。现在身体怪病蜂起,婚后难以进行正常的性生活,出现的许多症状,看了无数的医院,连医生都绝望了。”撸管即地狱,掉进去后就意味着能量的不断耗损,当耗损到临界点,那就会导致症状的大爆发,到时就相当痛苦了。有的人身体的症状还能忍受,但心理的失调已经让他接近崩溃了,撸管是对身心的双重摧残,肾为五脏之根,肾虚百病之源,到时症状爆发就会给你上刑,让你生不如死。很多人会觉得撸管爽,其实这种快感极其短暂,根本留不住,就像过山车一样,而戒色就像坐高铁,很安全、很舒适、很愉悦、很持久。认为撸管爽的人,其实他们已经忘记了曾经拥有的纯粹的大快乐,观察下孩子,孩子的快乐并不需要通过看黄撸管来获得,只需保持心灵的纯净,自然就会感受到纯粹的大快乐。快乐是本自具足的,当你心灵恢复纯净后,你自然就是快乐的,看着花花草草会有一种特别美好的感觉。当一个人开始撸管后,他就会忘记纯粹的大快乐,转而寻求短暂的快感,在邪淫的陷阱里越陷越深。只有彻底戒除邪淫了,才能再次感受到纯粹的大快乐,这时候就会发现戒色其实比撸管爽太多了,每天都开开心心的,像孩子一样纯净,像孩子一样天真无邪而充满喜悦,每一天都是奇迹。唯有一点你与孩子不同,那就是孩子不懂得珍惜,而你失而复得,所以会倍加珍惜这种极致的美好。撸管会刺激多巴胺超量分泌,令人飘飘欲仙,但是当多巴胺的浓度迅速下降后,你就会感到深深的空虚,这的确就像一场骗局,你以为自己可以持久地获得这种快感,其实并不能,你只是在两极之间摇摆,一会顶峰,一会深渊,撸完就会感觉很没意思,因为多巴胺的浓度已经下来了,你最终感受到的肯定就是空虚,随着伤精程度的加深,症状就开始袭击撸者了,到时身心俱废,悲剧就开始上演了。有一种方式可以让你一直处于顶峰体验中,什么方式?戒色修善!戒色修善可以提升一个人的振动频率,振动频率越高,就越能感受到纯粹的大快乐,这种顶峰体验非常持久,远远胜过撸管短暂的快感,这才是真正的大爽!!!
\end{case}

\begin{case}
    我从初中开始频繁撸管,大学一直看黄片,导致后来的前列腺炎和肾炎,和老婆结婚一年没有怀上孩子,还有记忆力差,老婆现在要和我离婚,哎,奉劝大家千万不要再撸管了,以前不懂事,犯下的苦果只能自己吃了,老婆说和我没感情了,已经分居了,希望不要再有人听信手淫适度无害论了。
    \subparagraph{附评} 记得前段时间有位戒友因为早泄而离婚了,很多人婚前都有十年左右的手淫史,早已把自己的五脏六腑掏空了,不仅早泄阳痿,还有很多其他的伤精症状,到了结婚时已如强弩之末。撸到一定程度肯定会症状缠身的,各种炎症,各种虐,这个案例的戒友得上了前列腺炎和肾炎,精子质量肯定不行了,考虑弱精症或者无精症,这样老婆怎么能怀上孩子呢?现在他老婆要和他离婚了,这个短命婚很快就会结束了。在婚前是该好好戒色的,必须戒除所有邪淫的行为,包括手淫、嫖娼、婚前性等等,如果你在婚前放纵,那很可能会出现未婚先废的情况,有时报应的降临是完全出乎意料的,在初中时第一次撸就为以后的人生悲剧埋下了伏笔。一些无知的人会觉得戒色是在压抑自己的欲望,是在走极端,其实他们从根本上误解了戒色的含义,他们之所以会这样认为,就是被砖家洗脑了,砖家告诉他们,人有欲望就要发泄出来,不要压抑自己的欲望,不要把自己憋坏了。这种理论乍一听很有道理,但其实非常荒谬。戒色的真谛在于化解自己的欲望,转化自己的性能量,不能有了欲望就去撸,这样只会让自己越陷越深。撸管具有高度成瘾性,不能戒除就是因为已经成瘾了,真正能自控的人,一次都不会撸。砖家只谈发泄欲望,而不谈控制欲望,这无异于把人推入火坑,欲望本来就是一个无底洞,如饮咸水,越饮越渴,永远不会彻底满足。我们一定要学会控制自己的欲望,如果你觉得戒色是在压抑自己的欲望,那你已经步入了歧途,真正的戒色是化解而非压抑!当这股能量冲上脑后,我们要学会把它转化掉,如何转化,就在起心动念上转,不跟着念头跑,这样就化解掉了,这样能量就保存下来了,肾精可以被撸管消耗掉,如果能够保存下来,这样就可以转化为脑力和精力,可以让人进入不可思议的身心状态。砖家的无害论是祸国殃民的歪理邪说,关于这点,很多人已经彻底认清了,另外就是适度无害论,无害论前面加上适度,误导性更强,每次撸都是在耗损宝贵的肾精,就像打开水龙头,水表就会走,就像汽车上路,油表就会转,只要撸,肯定会造成耗损。撸管是极易成瘾的行为,国外比作毒品,成瘾了再谈适度完全就是在自欺欺人,有的人号称自己没成瘾,别说大话,先戒一年试试,没成瘾的人想断就断,断几年都可以,你能做得到吗?这个案例的戒友最后告诫大家不要听信适度无害论,真的是过来人的苦心劝谕,我们一定要吸取前辈惨痛的教训,以免重蹈覆辙。
\end{case}

\begin{case}
    我是去年六月二十几号接触戒色吧的,当时也是看了《戒为良药》决心戒色。当时一身的病,头晕、时时想睡觉、浑身无力、尿频等症状,吓得我不得不戒了。这一戒就到现在。感觉自己身体的变化翻天覆地啊!天天精神饱满,心情舒畅。
    \subparagraph{附评} 这是一个恢复的案例,非常棒的恢复,祝贺这位戒友成功完成逆袭。最近看的一篇国外的戒色文章,里面提到“很多戒撸者都说,不打飞机之后,他们过得更快乐了。”我觉得这句话说得异常精辟和到位,很多人错误地认为撸管的生活应该会带来更多的快乐,其实恰恰相反,撸管的生活会导向惶恐与痛苦,如果选择戒撸,反而可以获得更多的快乐与自由,个人的幸福感也会随之大大提升。撸管的本质就是在积累负能量,不断积累负能量肯定会导致负面的结果。这位戒友当时一身的病,已经到了不得不戒的程度了,好在他遇见了戒色吧,这正是他人生逆袭的转折点。肾精对人的生命是极为重要的,古人曾形象地将之比作为“灯油”,灯油(肾精)耗尽,人的寿命也就终止了。很多人撸到后来的状态就是行尸走肉,拍出来的照片都具有相似的特点,那就是双眼无神、睁不开,神情猥琐而颓废,甚至哭丧着脸,像吊死鬼一样,有一种凄惨无力的感觉,有的人脸部变形严重,不对称极其明显,变丑显而易见。在戒色一段时间后,身心、容貌、气质、精神都会有一个显著的提升,那是充满电、能量劲爆的状态,与昔日滥撸滥泄的状态真乃云泥之别,绝对不可同日而语。肾精亏虚后的主要表现:精力下降,无精打采,精气神不足,整体生命机能缺乏活力;气色不好,面色萎黄或晦暗无光;肾精亏虚不能充养脑髓,导致脑功能减退,出现记忆力下降、头晕耳鸣、听力衰减;肾主骨,泄精后出现骨质疏松,腰痛腿软,牙齿松动或脱落;生殖机能下降,男性早泄阳痿或精子质量下降造成不育;影响肾主水液代谢功能,出现尿频或夜尿多,身体浮肿;发为肾之华,肾精亏虚不能滋养毛发,致头发早白,脱落稀疏。以上这些都是比较常见而典型的伤精表现,有病是应该积极治疗的,如果比较轻微,也不一定要去医院,自己好好坚持戒色养生,即可不药而愈。
\end{case}

\begin{case}
    我刚才还在因为强迫症而陷入压抑与烦恼,但是我做了一件事,做完这件事之后我心情大好,心里特别特别开阔!我刚刚去戒色吧,把首页的帖子打开来,挨个鼓励,而且是非常非常发自内心的鼓励,结果你猜怎么着,强迫症的念头自动化解了!神奇吧!就是这么神奇!我以前不太相信奇迹,现在我信了,而且鼓励完戒友后,不仅强迫症消失,压抑烦恼消失,而且内心生出一种非常自然的喜悦,真的是这样,现在心里特别开阔,就像走进了夏天的小兴安岭一样(虽然我没去过,但是我爸去过),这种喜悦在心中久久不会散去,我现在终于理解为什么前辈们能越戒越好了,说实话,到今天才深刻理解这一点的我实在是太愚笨了!古语有云,自助者天必助之,原来助人者天也必助之!我会加油,我会努力,我不要邪淫那种肮脏的快感,我需要这种大快乐!衷心地感谢飞翔哥你所做的一切!
    \subparagraph{附评} 有句话叫“为善最乐”。我们来到这个尘世,有一门功课一定要学,那就是要学会无私地帮助别人,不求任何回报,是发自内心的纯利他的行为,绝对真诚地帮助别人,这就是无条件的爱,无条件的奉献。我非常喜欢纯印老人的一句话,那就是:“好人好自己,坏人坏自己。”当你学会无私地帮助别人了,其实你就是在帮助自己,种善因得善果,善恶之报,如影随形。这位戒友终于开窍了,前辈一直在提倡行善积德,很多负面的心理状态是可以通过行善来化解的,改变你的心念即可改变你的世界,整个世界就是你内在的一个投影。当你陷入负面的心理状态时,一定要学会调整,可以多给自己积极正面的暗示,对于强迫的念头不要去跟随,不要强化它,这样它的动能就会被削弱。通过行善来转变心态是非常好的办法,其实原理很简单,悟明白了就会有意识地去做。印光大师云:“要想得好报,必须存好心,说好话,行好事,有利于人物,无害于自他方可。倘不如此,何好报之可得。譬如以丑像置之于明镜之前,决定莫有好像现出。所现者,与此丑像了无有异。汝果深知此义,则将来必能做一正人君子,令一切人皆尊重而爱慕之也。”存好心,说好话,行好事,这九个字就是我的座右铭,这九个字的力量非常之大,你的起心动念都必须是善的,身口意三门应该尽量做到通善,如果心里冒出邪念,必须马上断除之。很多戒友看到前辈说行善积德,他们真的在努力去做,宣传戒色,鼓励新人,帮助新人,全心全意在做,这样的戒友很容易进入稳定的戒色状态,帮助别人即是在强化正能量。而有的戒友就没有主动行善的意识,他们还没悟到这一层道理,学习戒色文章的确需要相当的悟性,真正看懂了就知道怎么去做了,没看懂,就还是老样子。记住了,为善最乐,好好戒色,好好行善积德,真是妙不可言,比撸管掏空自己不知强过多少倍!
\end{case}

\begin{case}
    本人是 87 年的,今年虚岁 30 了。吧里面有很多比我小很多的能认识到 SY 和 YY 的危害,我真的很开心,我也挺羡慕你们能这么早了解到 SY 和 YY 所带来的危害。我大概是小学五年级的时候染上 SY 的,大概算一算,已经有十六七年的时间了。因为 SY 和 YY 频率很高,所以没戒色之前出了很多大事情,由于人体的精华都消耗得差不多了,导致人自闭、没自信、抑郁、脱发、前列腺炎、自杀倾向,基本上大家有的症状我都有,对了,还得过两次肺结核。生活上也有很多不如意,由于得病没有上大学,一技之长什么也没有,没有钱还被骗钱。姻缘也是,由于自己没有自信,所以刚开始都是女孩子主动,然后当我想跟女生交往的时候,又因各种原因而错过。再来说说我现在的变化吧(戒色近 600 天),脱发、前列腺还有其他心理症状已经好了 90\% 多了,人现在也非常有自信,很阳光,浑身充满了正能量,工作不累,一个月休息能有十天,收入一个月一万多,放到以前我是想都想不到的。我现在每天都很幸福,希望戒友们也能像我一样,重获新生。加油!人有正能量了,财运好了,身体好了,家里面也和睦了,人变得帅气阳光,异性缘也多起来,现在我也不会像以前一样,都不敢看一眼异性了。身上没有猥琐好色的负面能量了,女孩子也都很主动地联系我了。所以各位单身的戒友们,一定要好好戒,美好的未来在等待着你们。
    \subparagraph{附评} 这个戒友的帖子相信很多人都看过,戒色 600 天完成了逆袭,生活的各方面都有很大的改变,非常励志。80 后在青春期是学不到戒色的知识的,现在 90 后和 00 后真的赶上了,年纪比较轻,恢复就会比较快,过了三十岁,身体的恢复速度比十几岁和二十岁时要慢很多,过了四十岁,恢复的速度更慢,很多职业运动员过了三十岁就会选择退役,身体已经无法承受运动强度了。戒撸真的要趁早,伤精程度轻时,恢复会很快的,记得我大学时有次戒了三周左右,改变已然很大,重新恢复帅气和阳光,但是一破戒,马上就会打回原形,又变得灰暗颓废,眼袋也出来了。我那时戒三周能有很大改观,一方面是因为当时我只有二十岁,另外就是我在三周内经常锻炼,这样身体恢复的速度就加快了很多,只可惜那时的我处在强戒的层次,最后的结果必定是失败,根本不是心魔的对手,心魔虐我就像玩儿似的。这个案例的戒友邪淫了十六七年,可以说这十几年完全就是一个悲剧,各种不如意,还得肺结核,还被骗钱,姻缘也不行,运势差到极点了。然而在戒色 600 天后,他重生了!这 600 天就是一部励志大片,过去邪淫了十几年,生活和身体一塌糊涂,在坚持戒色后,身心的症状开始慢慢恢复,最可观的改变就是心理的蜕变,内心变得自信阳光,充满正能量,精力也变好了,工作不累,财运也随之起来了,月入一万多,真是不可思议。新闻曾报道一个三十多岁的年薪八十万的男子因为邪淫而犯罪,最后被抓了,警察去他家里搜,全是那些邪淫的资源,因为邪淫,他葬送了自己大好的前程,从年薪八十万到身陷囹圄,这个落差实在太大了。黄片真的教坏了太多的人,很多人最后的性取向都会变得越来越变态,连畜生都不如了,实在太可怕了。邪淫很损一个人的福报,本来年薪十万的人因为邪淫染上神经症,最后因病辞职,财源断了,只能在家养病。邪淫的人很难留住财,好不容易赚的钱很容易就会耗散掉,很多都是送给了医院,留不住钱财,福报都因邪淫的行为而漏掉了。当你开始戒色了,去除身上邪淫好色的负面能量,到时你的生活就会大幅改观,所谓境随心转,邪淫的心会导致不顺且糟糕的处境,这条真理已经被上万的真实案例所反复验证。你的心理状态就像一块磁铁,会吸引与之频率相同的事物,负面的念头吸引负面的事物,正面的念头吸引正面的事物,你的念头在解码你的现实景象。世上所有的东西,不论是固体、气体或是液体,不论是有形或无形,都是一种能量的表现。有形无形皆是不断振动的能量,两者的区别在于振动频率不同。你的念头就是振动的波形信息,每时每刻都在解码这个现实世界,我们一定要发出正面的、善的念头,千万不可发出邪淫的、负面的念头,子曰:“见善如不及,见不善如探汤。”看见善良的东西,努力追求,好像赶不上似的;遇见邪恶的东西,使劲避开,好像将手伸到沸腾的水里一样。不要心存侥幸,要心存敬畏!不可有丝毫放纵自己的想法,有时念头会自动冒出来,必须严格做到斩绝萌芽!断念必须要快,有时慢一秒都不行。邪念的力量会随着时间疯狂壮大,我们必须先发制人,对待邪念,就是七个字:斩立决、斩尽杀绝!
\end{case}

\begin{case}
    飞翔哥,我戒色七月有余,昨日不慎破戒,心中万分悔恨。总结原因有三:一是承平日久,武备松弛,淡忘了沉溺淫邪时的痛苦,放松了警惕和学习;二是情绪管理不当,成为诱因。因为这次破戒不是所谓的“欲火焚身”,而是沉溺情绪的自甘堕落;三是独处破戒后,身心无力软弱的感觉再次回来。渐有好转的焦虑再次转为惊惧,内心极惶恐不安,对周围抱有极大敌意。最扼腕叹息的是,我本渐有“致良知,觉自我”之感,能平静和自我对话,指引自我,现在倏忽间荡然无存。
    \subparagraph{附评} 这位戒友文采不错,总结和分析也比较到位。不管戒多久,都要保持警惕,警惕是我一直强调的内容,很多人进入戒色稳定期之后,慢慢就放松下来了,以为自己成功了,殊不知心魔正伺机而动,心魔就像潜伏在草丛中的猎豹,就等你放松警惕,然后心魔就会发起猛烈进攻。这位戒友戒色七月有余,再次被心魔攻陷,的确令人扼腕叹息。导致破戒的原因有很多,放松了警惕和学习是一方面,另外情绪破戒亦很常见,很多人情绪不佳时都会通过撸管来发泄,生活中不如意的事情是很多的,自己一定要学会调整自己的情绪,不管发生什么,都应该以一个积极的心态去面对,千万不可自甘堕落、破罐破摔。我到现在看过的案例,有的因为和家人吵架而破戒,有的因为和女友吵架而破戒,有的因为和同事吵架而破戒,有的因为考试成绩不佳而破戒,还有的因为辞职而破戒,生活中能够导致情绪发生波动的情况非常之多,沉溺于负面情绪很容易出现破戒,成功后的狂欢也易于出现放纵,这是两个极端,就像走钢丝时偏左或者偏右都容易掉下去,内心贵在保持平衡,保持一颗荣辱不惊的平常心很关键。憨山大师《醒世咏》:“红尘白浪两茫茫,忍辱柔和是妙方。”《佛遗教经》云:“能行忍者,乃可名为有力大人。若其不能欢喜忍受恶骂之毒,如饮甘露者,不名入道智慧人也。”忍是一种修为,是一种养深积厚,忍不是强憋着不生气,而是看开了自然就不生气了,星云大师在开示中说:“当有人对我们恶口毁谤、无理谩骂的时候,能够漠然以对,以沉默来折服恶口,才是最了不起的承担和勇气。”所有的不如意都是在考验你的定力,就看你怎么办!破戒意味着能量的耗损,连续破戒很容易导致身心再次出现失调,坚持戒色会积累出一种良好的感觉,充满自信与底气,然而连续破戒会让这种感觉突然蒸发掉,继而开始怀疑自我,乃至敌对他人,本来计划想干的事情,也会因为破戒而心灰意懒,提不起任何斗志与热情,就像泄了气的皮球。戒色的每一天都要很小心,很谨慎,当你进入戒色稳定期后,依然要保持一定的危机感,不要放松戒色文章的学习,另外应该多学习传统文化,这样才能进入更高的境界,《增广贤文》云:“守口如瓶,防意如城。”不管戒多久,都要牢牢看住自己的念头,必须时刻保持警惕,独处时更要进入高度戒备的状态,把自己的警惕性开大最大值!
\end{case}

\begin{case}
    我 SY 十几年,人们见到我都说我长得越来越像我的一个舅舅,他吸毒也十几年了。无论你做了什么事,有多隐蔽,生活最终都会让你原形毕露。色情如毒,请避而远之。
    \subparagraph{附评} 频繁手淫,最后把身体搞垮,走路扶墙,双腿打颤,形销骨立,精神萎靡,神情猥琐,面如鬼泣,行尸走肉,病魔常伴,霉运连连。这位戒友撸了十几年,容貌越来越像吸毒的舅舅。其实看黄撸管就是在吸毒,只不过人家吸的是白的,你吸的是黄的。国外这方面的研究已经比较成熟了,撸管成瘾和吸毒成瘾有着高度的相似性,戒毒的难度很大,一般需要强制隔离戒毒,而戒撸有着自身的难度,因为东西就长在你身上,而且现在是网络时代,邪淫的资源很好搞,很多资源也不用花钱买,从这个角度来讲,复撸的可能性是比较大的。撸了十几年后,人的容貌气质肯定会出现不好的变化,有时自己还不觉得,但和以前相比,肯定相差很大,微妙的坏变化在一点点蚕食你的帅气和自信,最终镜子里出现了一个畸形、不对称的怪物,自己看着都觉得恶心。撸者的脸庞多有变形,凹陷或者浮肿的表现非常多见,有的人甚至达到了面目全非的程度,吸毒人员看上去懒洋洋的,没精神,一副糟糕的模样,撸到一定程度,也是那种感觉和气质,非常相似。撸管虽属隐恶,但相由心生,当生命的能量耗泄掉后,肯定会反映在脸上的,就像花儿枯萎了一般,到时看到自己的衰样,再看看撸前的照片,真是无语泪先流,悔不当初啊!色情如毒,拒绝撸管,珍惜生命,远离邪淫!
\end{case}

\begin{case}
    飞翔哥,向你汇报下我戒色的成果吧,目前戒了有一年零五个月,期间也曾和心魔做过斗争,最近一段时间很好,身体各方面包括脸上的痘痘都恢复得不错。我现在读大二,脑力也恢复得不错,在此期间我获得了国家励志奖学金。
    \subparagraph{附评} 这位戒友的确很励志,一年五个月,各方面都恢复得不错,他战胜了自己的心魔,当一个人能战胜自己的心魔了,他的内心就会变得极为强大。如果一个人无法战胜自己的心魔,那么不管他外表多么强大,其实他的内心是很脆弱的,因为他是心魔的奴隶,也许他的福报很深厚,在某方面取得了成功,但只要心魔还在作祟,那就很可能遭遇人生的滑铁卢。大家都知道很多高官的下场都不好,就是因为贪财贪色。一个人如果无法降伏自己的心魔,那么必然就会被心魔所驱使,越贪越多,根本就停不下来。只有战胜了自己的心魔,才能获得真正的平安,否则心魔做主,永无宁日!必须终结心魔的统治,心魔就像一个暴君一样,必须推翻它的统治,真正做回自己身体的主人。撸者是身不由己的,那是一种被奴役的状态,心魔不断压榨身体的精华,最终导致身心俱废,陷入很糟糕的状态。这位戒友获得了国家励志奖学金,真是不错,一个人真正懂得了脑力与肾精的关系,他就知道戒撸对他学业和事业具有何等重要的意义了,戒撸会加满你的能量槽,让你以最佳的脑力和精力去应对学业和事业,这样取得成功的可能性就比较大了。邪淫削福报,戒色修善就是在积累福德,真正有道德修养的成功人士,他们不会去犯邪淫,另外他们还会积极投身到慈善活动中去,成立慈善机构帮助弱势群体,这样的成功人士就比较稳定,就像常青树一样,他们在位子上就坐得住,否则德不配位,必有灾殃。戒邪淫对于人生的意义太重大了,被纵欲主义思想和无害论洗脑的人是很难认识到这一点的,当有了相当的戒色觉悟后,你再回头看看他们那种思想状态,就会觉得他们太无知、太愚痴、也太可怜了。
\end{case}

\begin{case}
    真的是感谢飞翔哥,看了您的留言,一语点破,你说我这破脑袋怎么就想不明白呢!当初没能察觉心魔的诡计,还以为自己厉害了,最一开始有想要肆无忌惮的想法时就应该察觉,如果能够断除这种想法就不会有后面一系列的事。自己的识别能力太差,没能看透这种念头的伪装,虽然失败了也算是一种经历和成长,人还是要向前看,以后不要犯这样的错误就好了。这两天很高兴,一个是这两天答疑帮助人感觉自己虽然骨折但也不是完全没用的废人;二来是再次做您第 84 季文章的笔记,发现断念简单分为的三步可以融合为一步就是“觉”,觉本身就包括发现、识别和消灭,所以只要把“觉”练好了就可以了,我就在想得怎么练。之后看到文章里说:“一定要不断地反观内心”,突然明白每次的反观不就是练习吗?所以戒色其实就是练习“觉”,然后时刻保持“觉”,其实一点都不复杂反而很简单,就看你能不能做到,一次做不到就可能导致失败所以要一直做到。最关键的一个字就是“觉”,觉察力通过不断的反观内心会越来越强,真正做到“觉”了就不会被念头牵制,因为觉本身就有消灭念头的能力,邪念出现甚至刚出未出时就被你觉掉了,关键就在觉,重点是能不能一直保持觉。
    \subparagraph{附评} 这个案例是“这会必须戒撸”分享的,他是以前的小吧,之前我也分享过他的案例,大家应该记得他去年骨折的事情,到今年还没完全好,这一年他受了很多苦。“这会必须戒撸”是一位很有悟性的戒友,之前也写过十篇左右的戒色文章,分享了戒色心得,之前他戒过两次超百日的,后来就不行了。觉悟上的缺陷,还有就是整体道德修为跟不上,这样戒到一定程度自然就会破戒。我们戒色一定要远离酒肉朋友,尽量不要喝酒,喝酒后定力会大幅下降,到时候就很容易破戒,而且喝酒后起的邪念很可怕,比如萌生强奸别人的想法,所谓酒壮怂人胆,恶向胆边生。有的人喝酒后和朋友一起去那种场所,有的是出于工作应酬,喝完酒就把客户往那种场所带,美其名曰:商业性行为。实则害人害己!还有的人用喝酒来为看黄撸管助兴,上次看到一个帖子,那个戒友之前就是边喝酒边看黄,然后再撸管,躲着老婆自己一个人偷着乐。其实这哪是乐啊!这样疯狂放纵为自己的健康埋下了极大的隐患,喝酒纵欲的伤害会更大,弄不好哪天脑袋里的血管就爆掉了,到时弄个半身不遂就惨烈了。我们戒色后一定要远离邪友、远离那些以邪淫为乐的酒肉朋友,因为他们会形成带坏你的氛围,和他们混在一起,只会把自己拖下水。戒色后最好是不要喝酒,如果出于工作应酬,应该尽量少喝,喝酒后一定要保持高度警惕,我看过很多案例都是因为喝酒而破戒的,这些教训都很深刻,大家在喝酒这个问题上要保持谨慎,避免犯类似的错误。

    “这会必须戒撸”戒到一定程度,脑袋里出现了骄傲的想法,觉得自己断念强,所以就可以肆无忌惮,这种想法正是心魔的诡计,相信这种想法就是中了心魔的诡计。真正断念强的人是不会放纵自己的,“这会必须戒撸”没有及时识别出这种念头,反而他相信了这种念头,结果就开始放纵自己了。这其实还是觉悟上存在缺陷,对于某类念头认识不清,无法识别出,就像杀毒软件对于某类病毒无法有效识别,所以就无法及时消灭。“这会必须戒撸”这段对于断念的领悟是非常正确的,戒色的核心就是修心,观心断念就是在修心,最关键的就是一个“觉”字,觉就是在反观自心,在不断反观的过程中,你渐渐会发现,只要看见念头,念头就会自动消失,所谓念起即觉,觉之即无。觉察力在不断的反观中变得越来越强大,最后心魔就无法攻陷你了,到时你就会变得无懈可击。心生魔生,心灭魔灭,心即是邪念,心魔的表现就是邪念袭脑,当心魔入侵时,必须第一时间消灭!戒撸只有两种可能,要么你干掉心魔,要么心魔干掉你!破戒就是因为你打不过心魔,你不是心魔的对手。这就像一场拳击比赛,如果你的实力够强大,你就能完爆心魔,否则只有被蹂躏的份。“这会必须戒撸”的悟性的确很不错,但毕竟还太年轻,不够稳定,很多方面还有待完善和加强,希望他能够早日东山再起。
\end{case}

下面步入正文。

这季是关于戒油子逆袭的,我从初中就开始自己尝试戒撸,但一直没成功,屡戒屡败,被心魔虐了十几年,这十几年间我也曾经沦为了戒油子,所以我对戒油子的处境和心态十分了解。大家可能听说过佛油子,一般将学佛不求胜解,好夸夸其谈而不重实修的人称为佛油子。什么叫戒油子呢?第一种解释就是反复破戒,再无热情与决心认真戒色的人,态度不端正,马虎戒色;第二种就是表面上貌似都懂,但实际上一知半解,什么都没有做到。如古德所说:“说时似悟、对境生迷。”说的时候头头是道,但是实战却一塌糊涂、一触即溃、溃不成军,被心魔虐得体无完肤。我聊过很多戒友,其中很多人都具备一定的觉悟,比之新人要强很多,但是在实战方面却很差,知行不能合一,理论无法结合实战,这样久而久之就沦为了戒油子,他们会说自己都懂了,甚至一看到戒色文章的标题就知道里面在讲什么,但是他们的认识与理解其实很肤浅,并没有真正掌握戒色的原理与规律。

我那时沦为戒油子的时候是处在强戒的阶段,强戒就是不学习,以为单纯靠毅力就能成功,失败了就说自己毅力不行。那个时期也没戒色文章可学,接触不到任何戒色的善文,现在回过头来看那段时期,就像没带任何武器就上战场了,心魔手中拿着火神机枪,而我穿着裤衩拿着树枝就上去了,结果可想而知,简直就是去送死的,心魔虐我简直如同拳王打小孩,根本不是对手。在反复戒色失败后,我那时的戒色态度也出现了问题,刚开始靠热情冲起来,初心甚为猛利,可惜一旦破戒,就灰心丧气,甚至也有过不想戒的时候,就是完全放弃了,继而破罐破摔、自甘堕落、自暴自弃、自我否定。以前有一位戒友靠初心戒了四个月,初心是很珍贵的,有一种新鲜的冲劲与能量,但是初心不会长久,保持初心很难。因为初心会消退,学习戒色文章的热情也会下降,后来那位戒友就开始屡戒屡败了,甚至很难突破一个月,变成三天两头就会破戒,甚至还去嫖娼,他已经沦为了戒油子,再后来他就消失了,大概是放弃了。

戒油子首先要端正自己的态度,一定要扎实地学习戒色文章,好的戒色文章应该看几十遍乃至上百遍,自己做的戒色笔记也要反复地复习,把戒色知识真正内化成实战意识。很多戒油子都懒得看戒色文章了,这是戒油子很普遍的一个共性,那就是对于戒色文章的冷淡与厌烦,他们无法重复学习戒色文章,而提升觉悟的关键就在于不断重复学习,看一遍是远远不够的,看一遍所能获得的领悟简直少得可怜,像芝麻粒一样。孔子晚年喜易,反反复复把《周易》读了许多遍,不知翻开来又卷回去地阅读了多少遍,把串连竹简的牛皮带子也给磨断了三次,这就是著名的“韦编三绝”。要提高吸收率,那就必须注重重复学习,反复吸收。这季谈戒油子,并不是贬低,而是希望戒油子能够重整旗鼓,东山再起。因为我自己曾经就是戒油子,在那种怪圈里真的很无力、很无望,一直失败一直失败,就像被人一直击倒一直击倒,最后自己都懒得爬起来了,干脆就躺地上装死了,已经变得没有勇气站起来战斗了。根据我自己的体会与经历,戒油子也是可以逆袭的,关键就是要端正自己的戒色态度,并且养成良好的学习习惯,这样持之以恒地坚持下去,必然会迎来一次又一次的顿悟,到时觉悟真正上去后,在实战中的表现就会随之提升。

下面分享一个历史典故。

\begin{quote}\it
    宋朝苏东坡在江北瓜洲(现在杨州)地方任职,和江南金山寺只一江之隔,他和金山寺的住持佛印禅师是好朋友,经常谈禅论道。一日,苏东坡自觉修持有得,做了一首赞佛的诗,诗云:“稽首天中天,毫光照大千;八风吹不动,端坐紫金莲。”这首诗在赞佛陀的同时,又暗含自己已跟佛陀一样,达到了不为“称、讥、毁、誉、利、衰、苦、乐”八风吹动的境界。苏东坡再三吟咏,感到非常得意,立刻派书僮乘船从江北送到江南,呈给金山寺的佛印禅师观赏。禅师从书僮手中接看之后,即批“放屁”二字,嘱书僮携回。东坡一见大怒,立即过江责问佛印禅师:“你不夸我也就算了,何必如此讥讽与我?”禅师回敬道:“从诗偈中看,你修养很高,既已八风吹不动,怎又会一屁打过江?”东坡一听,默然无语,自叹修养远不及禅师。
\end{quote}

这个典故很有名,几乎修行的人都知道这个故事,修行是很注重对境实战的,说到而做不到是没用的,就像纸上谈兵一样。曾经看过一个节目,一个教授炒股知识的老师自己从来不炒股,但是他很能讲炒股的理论,讲得天花乱坠、头头是道,很多人都来听他的课程,后来他觉得自己已经把炒股研究得很透彻了,于是就自己开始炒股,结果被深深地套牢了,他的那些理论根本没用,因为他的理论完全脱离了实战,实战才能验出一个人真正的水平高低。有些戒友虽然懂得了断念的道理,但是“知而不行,是为不知”,他们只停留在理论的层次而没有把断念练到出神入化的程度,就像懂得了磨刀的方法,但是却不去磨,这样就等于零!执行力差是一个不容忽视的问题,在邪念入侵时,很多人忘记了断念,反而跟着邪念跑,邪念似火,无制则燎原,断念要斩钉截铁,毫不留情,不可有一丝一毫迷恋贪恋、当断不断之情。

\paragraph{真正认识到自己的问题所在}

\begin{quotation}\it
    你是否在破戒后认真总结和反省?

    你觉得你之所以破戒,问题出在哪?

    你犯了什么错而导致的破戒?

    你是否严格遵守戒色战场的纪律?

    你在断念时是否有犹豫和贪恋?

    你在破戒前,脑中有什么念头和感觉?

    你的断念速度如何?是否能做到念起即断?

    你是否能管理好自己的情绪?
\end{quotation}

失败并不可怕,可怕的是不学习、不总结、总是犯同一个错误,很多人自夸都懂了,但是魔考一来,立即垮掉。大家都做过学生党,不少题目你觉得自己懂了,但那是假懂或者是有点懂,并不是真懂和深懂,魔考就是试金石,你是假懂还是真懂,一考就见分晓了,真金不怕火炼。打开戒色卷子,映入眼帘的是一个个刺眼的红叉,本来不该错的题,你错了,很多都是低级的错误,但是你犯了,而且是一而再再而三地犯,从来没有真正醒悟过。有时你会说:“我知道自己错在哪了。”但是你之后还是在犯那个错误,你并没有克服那个错误。在破戒后一定要好好反省和总结,就像飞机失事后调查人员排查原因,到底哪里出了问题,为什么会出现破戒,导致破戒的原因一定要真正找出来,否则下次还可能犯这个错误。很多人断念也不够狠,当断不断,反受其害,还在犹豫还在贪恋,还舍不得断,这样怎么能行呢?晚一秒就可能被心魔拖入漩涡,必须要快!戒色战场的纪律有一条就是避开色弹,但很多人还在找黄看,这不是去送死吗?!很多人破戒几十次乃至上百次都没有醒悟,他们一直在违背戒色战场的纪律,这样怎能不破戒呢?

我很早就悟到的一个道理,那就是撸界没有新鲜事。你破戒的过程和原因基本都在前辈身上发生过,前辈也曾在某个坎上栽倒过无数次,直到不断学习不断总结才最终克服了那个坎,首先必须认识到那个坎,否则就会一直犯同样的错误。你可以反省下自己的破戒经历,绝大多数的破戒过程都是同一模式的固化与重复,如果你无法打破那个惯性的模式,你就无法跳出那个怪圈,你会被牢牢按死在那个漩涡里。我从来没有忘记一点,那就是从错误中学习,从自己的错误中学习,也从别人的错误中学习,然后要学会避免那些错误。一个人犯错很正常,但是如果他不能从错误中吸取教训,那真是冤大了,精费都白交了,就像你交了一大笔钱,结果屁都没学到,你觉得你会容忍这种事情发生在自己身上吗?破戒就是亏钱!这个意识一定要有!戒色就是止损,从某种意义上来讲,戒色也是在赚钱,因为肾精能量会转化为脑力与精力,让你以最佳的状态投入到学业和事业中去,戒色也可以让你少跑很多次的医院,所以戒色也是在省钱!

如果你想越戒越好,那就必须不断地完善自己的觉悟,也许你会经历破戒,但你应该看到,破戒可以让你发现自己的不足和漏洞,补强自己的觉悟,你就能越戒越好。在破戒后不应气馁与灰心,应该好好总结和分析,到底问题出在哪?找出问题,解决问题,下次不要犯同样的错误,这就像分析考卷一样,考完了,发现自己的错误,真正把题目搞懂了,下次就不会做错了。如果你一直破戒一直破戒,肯定你在某方面存在认识缺陷,错误得不到纠正,那就会一直错下去,智者会及时地认识到错误并且避免犯同样的错误,而戒油子总是在犯同样的错误,他们没有真正认识到原因,因为没有认识,所以就谈不上完善与解决。这样久而久之,越戒越油,越戒越烂,越戒越没信心,有的人甚至会出现怀疑与动摇,乃至叛变。

\paragraph{拿出拼命的决心}

《孙子兵法》云:“死地则战。”戒色要像处于绝境与死地,这时只有一个选择:拼命!必须拿出拼命的决心与勇气,舍得一身剐,敢把皇帝拉下马。对心魔作战一定要有一股摧枯拉朽、浩气冲天的气势,气势上要先胜出,否则还没戒,已经输掉一半了。如果你和很多新人聊过,你就会发现他们中很多人的决心不够坚强,缺少一股狠劲,既然选择戒色,心就要够决绝,不管干什么事,要做成这件事,都需要相当的决心与魄力。有的新人还没戒,就在想看黄的事情了,决心明显不行,对邪淫危害的认识严重不足,脑子里也有很多思想误区有待纠正。所谓拼命,就是要有杀出重围、杀出一条血路的气魄,黄檗无念禅师云:“剿绝内贼!”心魔即是内贼,最厉害的贼在你心里头,不要认贼作父,要把内贼坚决剿杀掉!大德开示多用“杀”字,不是叫你杀生害命,而是叫你杀自己的心魔贼,战胜魔军!有位戒友说过:“魔军已退。”意思是说刚才魔军大举进犯,现在已经被他打退,他就像戒色大将军一样镇守戒色城池,真乃雄猛大丈夫是也!战胜心魔的人,胜过战胜一座城池。

古德云:“第一要立坚志,盖志者气之帅也。人若立有坚志,如统军百万,威神八面。天日可贯,何事不成乎?凡畏难者,志不坚也;因循者,志不坚也。听言更移,中道自画。始勤终怠者,皆是志不坚也。”俗谚云,男子无志,钝铁无刚;女子无志,烂草无穰。你不拼命,你就不知道自己有多强,你就不知道自己的潜力有多恐怖!你拼命了,心魔就胆怯了,软的怕硬的,硬的怕横的,横的怕楞的,楞的怕不要命的!你拼命了,心魔的气焰就弱下去了;你拼命了,你的气势就威武雄壮了。拼命前,你是病猫;拼命后,你是猛虎!拼命地戒,拼尽全力地戒,不顾一切地戒,拿出破釜沉舟的决心,拿出视死如归的气概,猛烈学习戒色文章,猛烈做笔记,猛烈复习笔记,像打了兴奋剂一样勇猛精进,干劲冲天,干死心魔!真干狠干!!干干干!!!

\paragraph{不断激励自己}

美国哈佛大学的管理学教授詹姆斯所说,如果没有激励,一个人的能力发挥不过 20\% - 30\%,如果施以激励,一个人的能力则可以发挥到 80\% - 90\%。正确的激励可以把人的潜能激发出来,提高人的积极性与主动性,进而提高学习和工作的效率。戒油子基本都是在消极思考,他们不是激励自己而是在不断否定自己,这其实很好理解,在一次次被心魔打败后,自然会感到无力乃至丧失再次战斗的信心与热情。别人戒得风生水起,而戒油子却戒成一潭死水。要突破现状,必须发出大决心,然后平时要学会不断激励自己,不要消极思考,不要否定自己,消极思考带来消极的结果,积极思考带来积极的结果。同样是半杯水,消极思考者会说:“怎么只有半杯啊,太糟糕了。”而积极思考者会说:“太好了,还有半杯可以喝。”你发出消极的念头就会解码出消极的现实景象,一直消极思考就是在限制和削弱自己的能量。本来你可以表现得更棒更好,但是你觉得自己做不到,于是就不努力去做,变得消极和懈怠了。

\subparagraph{罗森塔尔效应}

美国著名的心理学家罗森塔尔曾做过这样一个试验:他来到了一所普通中学,在一个班里随便地走了一趟,然后就在学生名单上圈了几个名字,告诉他们的老师说,这几个学生智商很高,很聪明。过了一段时间,教授又来到这所中学,奇迹发生了,那几个被他选出的学生现在真的成为了班上的佼佼者。为什么会出现这种现象呢?正是“暗示”这一神奇的魔力在发挥作用。每个人在生活中都会接受这样或那样的心理暗示,这些暗示有的是积极的,有的是消极的。罗森塔尔教授是著名的心理学家,在人们心中有很高的威信,老师们对他的话都深信不疑,因此对他指出的那几个学生产生了积极的期望,像对待聪明孩子那样对待他们;而这几个学生也感受到了这种期望,也认为自己是聪明的,从而提高了自信心,提高了对自己的要求标准,最终他们真的成为了优秀的学生。这个效应留给我们这样一个启示:鼓励、赞美、信任和期待具有一种能量,它能改变人的行为,当一个人获得另一个人的鼓励、赞美时,他便感觉获得了社会的支持,从而增强了自我价值,变得自信、自尊,获得了一种积极向上的动力。

戒色的每一天都要保持积极正面的心态,多给自己鼓励,同时也应该多为别人加油,鼓励他人其实也是在鼓励自己,营造出一种积极向上的氛围对于戒色成功是非常重要的。人是很容易受到潜移默化的影响的,大家都知道我很喜欢用“加油”这个词,现在很多戒友也都喜欢用“加油”来鼓励别人,为别人加油,也为自己加油,始终保持在积极正面的心态,不做任何消极思考,即使暂时失败了,也要积极思考,只想积极正面的内容,这样就会最终把你导向积极正面的结果。如果你让自己沉溺于消极思考,这样很多事情就无法全力去做,当你转变一下想法,结果很可能就会截然不同,因为成功并不像你所想的那般困难,只有思想积极了,整体的能量才能积极活跃起来,这样离成功就很近了。

\paragraph{打心底真正热爱戒色}

非真戒色者,焉知戒色之乐哉?常人认为戒色是压抑,这是他们的思想误区,智者认为戒色是享受,因为智者懂得了如何转化自己的性能量,智者深知只要心灵恢复干净了,就能获得真正纯粹的大快乐。过去我认为歌星和影星很酷,曾经也喜欢过明星,后来学习传统文化后才明白圣贤教育是最酷的,降伏心魔是最酷的,戒色才是真正酷的事情。别人纵欲自毁,而你戒色重生,这就足够酷。酷要酷得有智慧,不能肤浅地酷,很多人认为撸管是很酷很爽的事情,在智者看来,完全就是傻逼式的自残。戒色是一项需要终生贯彻和培养的修为,应该打心底里热爱戒色,很多新人看不进戒色文章,他们不是真正热爱戒色,他们的思想观念还没转变过来。当你真正热爱了,当你把戒色当作一种享受和乐趣了,那么很多障碍自然就消除了,你会变得更加积极主动地去学习戒色文章,而不是坐下来勉强自己去学,真正热爱戒色的人会迫不及待地渴望去学习戒色知识,这是两种完全不同的状态,一种被动勉强,另一种主动积极且充满干劲,当你真正渴望学习了,这时候你会发现戒色比过去容易很多。有的戒友热爱足球或者篮球,抑或其他运动,有的戒友热爱音乐或者电影等,每个人都有自己的兴趣爱好,如果你能把戒色作为自己的兴趣爱好,你的进步会很快的。学习戒色知识可以让你明白很多道理,悟道之乐是非常殊胜的,顿悟道理的喜悦远胜于邪淫的快感,有时顿悟一个道理比中五百万还开心,有一种法喜充满的感觉。

\paragraph{戒油子需要一次真正核心的顿悟}

很多戒友也有过顿悟的经历,就是一下恍然大悟的感觉,就像一道题终于想明白了。戒色之后是会经历很多次的顿悟的,如果你没有经历过这种感觉,那就说明你还没达到那个火候。顿悟就像开水达到沸点,一般坚持学习戒色文章,自会迎来顿悟。小戒靠忍,大戒靠悟,如果经常有顿悟的感觉,那就会越戒越好,顿悟一般发生在学习戒色文章时,有时也发生在其他时刻,比如散步时或者做其他事情时,突然就明白了一个戒色的道理,就像一个从天而降的灵感一样,突然就领悟了前辈某句话的深意。前辈很多话的下面都是一座金矿,很多人挖到的只是表层的泥土,而真正的顿悟者能挖到深层的金矿,顿悟就有如此的穿透力,顿悟是一种深度的理解,当你真正明白了,你的戒色层次自然就上去了。顿悟可以有很多次,我现在差不多一月三次左右,是和修行有关的顿悟,古德云:“大悟十八回,小悟无其数。”在不断的顿悟过程中,你会发现自己戒得越来越稳定,越来越接近修道的维度,到时你自然会找修道的文章和书籍来看。戒油子也会经历顿悟,有的人经历顿悟后写出的戒色文章还很有道理,但戒油子往往缺少一次真正核心的大顿悟,那次大顿悟才是真正的分水岭,只有经历了那次大顿悟,才能在戒色方面真正脱胎换骨,那次大顿悟就像头脑的核爆一样,让你一下抓住了向上的关捩子,让你的思想意识进入了全新的层面。

\paragraph{对境时、念侵时}

戒色必须注重实战,一切立足于提升实战表现,戒油子能谈道理,能写文章,但是实战太弱,执行力太差,能说却做不到,《少林悬记》云:“后来明道者多,行道者少;说理者多,达理者少。”即使把《九阴真经》倒背如流,但如果不勤加练习,那也成不了武林高手。真正的敌人在内心,然而很多戒油子却一点警惕性都没有,一点实战意识都不具备。戒油子在平时也会讲到断念,也会说要保持警惕,但你会发现戒油子在实战中的表现和他所说的完全相反:断念不及时,反而跟着念头跑,当想看黄的念头出现时,他也不断,他听从那个念头。戒油子的问题就是理论无法结合实战,所言与所行是背道而驰的。

有的新人会问如何戒色,其实戒色很简单,就两方面:提升理论水平,也就是提升觉悟,另外,就是提升实战表现。这就是“理论 + 实战”,提升理论水平的终极目的就是为了提升实战表现,如果仅仅停留在理论的层次而和实战脱节,那依然还是会失败的。戒色的实战简而言之就是两方面:对境时你的表现、念侵时你的表现。这就两个方面,如果对境时你跟着点,盯着看,彻底陷进去了,那就是不行;如果念头入侵时,你不能做到念起即断,那也是不行。心魔袭脑有三种方式,我之前的文章也讲到过:念头、图像和微妙感觉。念头就是概念思维,就是某句话或者某个词从脑海中冒出来;图像就是形象思维,我更喜欢用“浮现”这个词来形容它,就是一幅场景,或者回忆或者幻想,从脑海中浮现出来;微妙感觉我喜欢用“逸出”这个词来形容它,像雾一样捉摸不定,但你能感觉到它上来了,情绪也可以归为这种微妙的感觉。我平时说的断念,就包括断除念头、图像和微妙感觉,图像和微妙感觉也可以看作念头的不同表现形式,图像是形象的念头,微妙感觉是极其细微的念头,就像水有液态、固态与气态一样,在本质上是一种。

王阳明云:“随他多少邪思枉念,这里一觉,都自消融。”觉即是觉察,觉察即消灭,前提是觉察力够强,在一次次反观内心的过程中,你的觉察力自然会增强,到时自然可以一觉即空、一觉即灭。《禅宗直指》里讲到:“佛法工夫。全在于觉。”觉即是激光炮,瞬间摧毁入侵的念头怪;觉即是快刀切念,划拉一下就全没了;觉即是黑板擦,脑中出现任何念头、图像和微妙感觉,都能立刻擦掉;觉即是屠魔刀,念来即斩,毫不犹豫!断念时必须够狠、够快、够绝!把断念练成绝活!出神入化、神乎其技!元音老人云:“念来即觉,觉即转空。”不怕念起,就怕觉迟,觉就是断念的最强利器,对内心保持强大觉察,入侵的念头自然就会遁形。《真心息妄法》云:“第一门觉察。此功夫最为紧要,由未悟以至彻悟,由初地以至十地,无一事无一时不要凛觉。不但才起一念要觉察,即日常应付人事,应对万机时,念念仍要觉察,觉察‘动念即乖’。”断念的功夫是一层层提高的,一层比一层厉害,一层比一层狠,一层比一层强,一层比一层快!断念就是最高深的功夫。《灵枢》云:“知一则为工,知二则为神,知三则神且明矣。”戒色高手知道得多,且知行合一,知深行猛;戒油子一知半解,且知而不行,或知浅行迟。知是理论,行是实战,唯有知行合一,方为真正的戒色行者!

\paragraph{水滴石穿,绳锯木断,积小胜成大胜}

戒油子往往不注重积累,要戒色成功,必须掌握正确的方法,产生持续不断的正确行动,完成戒色知识的持续积累,要做到这一点,应该养成良好的学习习惯,你可以给自己定个计划,比如每天记 20 条笔记,一月就是 600 条,每次记笔记前都应该复习之前的笔记,笔记要不厌其烦地复习几百遍,往往在复习的过程中就会迎来新的顿悟。苟日新,日日新,又日新,温故而知新,我到现在依然在做笔记,我极为重视做笔记和复习笔记,这是水滴石穿的功夫,刚开始你也许会觉得没什么变化,但只要持之以恒地坚持下去,很快就会出现变化。另外,一篇文章在几个月前可能看不大懂,但是几个月后再看就会感觉很不一样,到时就会有更深的领悟,就像小学生无法看懂初中的课本一样,等觉悟一点点上去后,就能完全看懂了。

戒油子也会做笔记,但无法坚持,记个几天或者半个月,他就停掉了,无法坚持,缺少连续性。做笔记必须要坚持不断,修行方面有“中断魔”之说,只要中断一天,就会连续中断,一曝十寒,甚至几个月都不做笔记了,沦为了真正的戒油子,戒油子的一大特点就是不认真学习、不坚持学习,声称都懂了,但实战时却是以卵击石,碰到心魔马上鸡飞蛋打。一次次单挑心魔,一次次铩羽而归,折戟沉沙,被心魔一次次无情地碾压,灰心丧气。戒色要成功,一定要懂得积累,365 天,你是否能做到每天都做戒色笔记,即使不能每天做戒色笔记,也应该每天复习戒色笔记,不要停止学习,不要离开戒色文章,有的戒友会说学习忙或者工作忙,其实再忙也可以挤时间,每天十分钟或者半小时的时间总是有的吧,在这十分钟内就可以看上百条的戒色笔记,这就是很大的收获啊!有时学习戒色文章就像吃药一样,吃一顿可以管一上午,中午吃一顿又可以管一下午,在不断的学习和复习过程中,你的戒色觉悟就得到了大幅提升,到时戒色就会变得简单很多。曾经的 aitrue 小吧也特别注重积累,冰冻三尺非一日之寒,水滴石穿非一日之功,绳锯木断非片时之磨,你可以问问自己,上次你做戒色笔记是什么时候了?你已经连续做了多少天的戒色笔记?是否有中断?只要不中断,继续记下去,平时多复习戒色笔记,这样就会迎来一次又一次的顿悟,到时觉悟就会日新月异、突飞猛进,总有一天你会脑洞大开、豁然开朗,迎来真正的大顿悟。

最后总结:

这季谈了戒油子的问题,戒油子首先要端正态度,重新下大决心来戒,然后踏实学习戒色文章,注重积累,不要中断,不要骄傲自满,时刻保持谦虚谨慎。只要你真正做出改变了,一定可以在戒色天数方面取得突破。真正热爱戒色的人会很容易成功,对戒色文章厌烦的人永难冲出怪圈,戒色的态度十分关键,必须静下心来认真学习戒色文章,当你真正热爱戒色了,自然就会乐在其中,可以不知疲倦地学习,像一块海绵一样疯狂地吸收戒色知识。戒色就是练级,要把打网游的热情和冲劲转移到戒色方面,戒色才是你真正值得投入的修炼,戒色可以把你的人生提升到全新的境界,戒色也是一种救赎,可以把你从邪淫的深坑里彻底超拔出来。

之前有位戒友戒了 200 天,因为那时他初心猛利,认真学习戒色文章,认真做笔记,一鼓作气戒了 200 天,但是后来他出现了骄傲的想法,继而放松警惕与学习,最后就破戒了,之后越戒越差,又陷入了屡戒屡破的怪圈。戒油子要好好反省和总结,扎实学习戒色文章是最关键的,一定要懂得积累的力量,千万不可中断学习。只有在不断学习的过程中,你才能进入更高、更稳定的层次与境界。不仅要学习专业的戒色文章,也应该好好学习圣贤教育,圣贤教育特别强调德行的培养,戒油子落败的原因之一就有德行的问题,谦德不行,容易骄傲自满,这也是年轻人极容易犯的错误,骄兵必败是千古不变之真理,《曾国藩家书》云:“天下古今之庸人,皆以一惰字致败,天下古今之才人,皆以一傲字致败。”学习圣贤教育可以有效提升你的德行,所谓厚德载物,无德留不住肾精,这个道理一定要深刻领悟!德行跟不上,戒到一定程度自动就会垮下来,德行就像地基一样,《菜根谭》云:“德者事业之基,未有基不固而栋宇坚久者。”学习圣贤教育可以夯实你的地基,不管是戒色、学业、事业、为人处世,都需要德行来做强大的支撑。我现在极为重视圣贤教育的学习,自感收获颇大,古德云:“一日不读圣贤书,便觉面目全非;三日不读圣贤书,便觉面目可憎。”随着学习的深入,我深感德行是最重要的基础,没这个基础,戒色不会真正稳固。戒色的人应该注重修德,谦卑自牧,谦光可掬,内心要真正谦卑下来,骄傲自满的想法一定要杜绝。

在连续破戒后,戒色的好状态就像开了闸的水,很容易一去不复返,有的人再次找回戒色状态可能需要几个月乃至一年以上了。要找回良好的戒色状态,认真总结和反省是必须的,然后一定要认真学习戒色文章,多做笔记,多复习笔记,很多人会做笔记,做了几大本,但是他不懂得复习笔记,这样吸收率还是很低,无法转化为实战意识。除了总结和反省,戒油子还可以通过积极行善来迅速积累正能量,从而转变自己的心态,这样有利于快速找回良好的戒色状态。破戒就是正能量的下降,坚持行善有助于快速提升正能量,每天帮助新人答疑或者宣传戒色,坚持一段时间,良好的戒色状态就会回来,而且我发现在帮助别人答疑的过程中,自己也会有相关的思考和收获,既是帮助别人,实则也是提升自己。

你的两个肾就是两个核反应堆,肾精足,能量就足,你的人生就会像火箭般上冲,在学业和事业上就能有上佳的表现。戒油子照样可以蜕变与逆袭,我曾经也是戒油子,甚至还放弃过,但后来我还是选择了坚持戒色,因为症状太痛苦了。作为戒油子的我,也曾经迷茫过,不知道该怎么戒,总是失败,心魔实在太强大了,心魔不知修炼了多少劫,而我在这一世就想干掉它,难度可想而知。然而在我坚持学习的过程中,我终于领悟到了戒色的核心,也就是——修心!圣贤教育就是在讲修心,专业戒色的最高层次也是在讲修心,修心就是修念,也就是观心断念。在刚开始戒色的半年里,我还会感到害怕和担心,担心自己某一天再度失控,担心自己再次被心魔拉回自毁的怪圈,那时的我戒得还不够稳定,但我一直在坚持学习,干劲冲天地学习,热火朝天地学习,拼了命地学习,像亡命徒一样,笔记一本接一本,笔芯一支接一支,几乎每天都在疯狂做笔记。在经历了一连串的小顿悟过后,终于迎来了醍醐灌顶的大顿悟,有了那次大顿悟后,我真正知道了该怎么对付心魔,然后就越戒越稳定了。

我们对待戒色文章要像对待初恋一样热情,要迫不及待地学习戒色文章,吸收戒色知识,必须拿出最大的热情与动力来学习戒色文章,对待戒色文章必须要有饥渴感,好比一个人在沙漠中走了很多天,突然发现水了,那种情景简直就像发了疯一样扑向水,疯狂地猛喝,不顾一切地猛喝。再举个例子,一个明天即将上刑场的死刑犯,你对他说:“只要完成某某任务你就可以免于死刑。”那个死刑犯肯定会拼尽全力去完成,在那种情形下就会爆发出最大的热情与动力。有的人打网游练级,可以连续十几个小时,甚至废寝忘食,如果能把这种热情的一半用在学习戒色文章上,早就戒色成功了。很多戒油子对戒色文章兴趣寥寥,甚至是厌烦,声称自己全懂了,其实一知半解,似懂非懂,这样肯定是不行的,必须转变心态,要激发出最大的热情与动力来专攻戒色文章。戒油子要逆袭,必须在态度上来个 180 度的大反转,要猛烈学习戒色文章,以排山倒海的气势不顾一切地疯狂吸收戒色知识,而且要养成良好的学习习惯来克服戒色厌倦期。弘一大师云:“发心不难勇锐,难于持久。”养成良好的学习习惯即可让你具有持久力,就像每日刷牙一般,哪天不刷牙反而觉得不对劲。

面对反复的失败,戒油子不要灰心丧气,一定要振作起来,在哪里跌倒就在哪里爬起来,不断总结失败的教训,补强自己的觉悟,强化观心断念,并且保持谦虚谨慎,最后的结果一定是这两个字——成功!要有坚韧不拔之志,要有万夫不当之勇,心魔击倒你 100 次,第 101 次站起来 KO 心魔!这就是最强的回应!觉悟就是你最强悍的肌肉,断念就是你的铁拳,彻底击溃心魔,完成荡气回肠的史诗级逆转!你行的!铁血男儿,握紧你的铁拳,站起来猛烈战斗吧!!!

下面分享六首戒色诗歌。

\begin{poem}[冲破手淫的魔咒]
    \begin{multicols}{2}
        \begin{center}~\\
            从手淫亿人坑中爬出 \\ 再次沐浴在纯净美好的感觉里 \\ 再回头看看,他们像蛆虫一般蠕动 \\ 苟且偷生、苟延残喘地撸着 \\ 太可悲,也太可怜了 \\ 他们是心魔的傀儡 \\ 现在是解除魔咒的时候了 \\ 必须学会主宰自己的心 \\ 不要跟随邪念,必须做到念起即断 \\ 拿出最大的勇气和决心 \\ 不顾一切地戒,戒到死 \\ 一夫当关,斩尽邪念 \\ 一人拼命,万夫难当 \\ 戒色烈士个个都是董存瑞黄继光 \\ 每一个都具备无比强悍的战斗精神 \\ 心魔狠,你比心魔更狠 \\ 心魔强,你比心魔更强 \\ 你必须降得住它 \\ 你必须在它之上 \\ 否则你就会被心魔踩在脚下 \\ 遭受无情的蹂躏 \\ 它想怎么虐你就怎么虐你 \\ 被心魔虐得体无完肤遍体鳞伤 \\ 一次次被心魔吊打,毫无还手之力 \\ 一次次发毒誓,一次次自己打脸 \\ 一次次冲向心魔,一次次被心魔碾 \\ 只有真正能把握自己的起心动念 \\ 才能降伏住自己的心魔 \\ 冲破魔咒的时刻已经来临 \\ 必须从心魔的暴政下解放自己 \\ 彻底告别奴役的状态 \\ 为自由而战!杀出戒色的生天!
        \end{center}
    \end{multicols}
\end{poem}

\begin{poem}[皮奴]
    \begin{multicols}{2}
        \begin{center}~\\
            为皮迷,为皮狂,为皮疯,为皮撸 \\ 画皮时代,多少人掉入了皮的陷阱 \\ 他们被皮包了饺子,彻底陷(馅)进去了 \\ 下地狱就像饺子下锅一样 \\ 一皮障目,不见脓血屎尿 \\ 为皮所诳,不顾礼义廉耻 \\ 为了这张皮,撸出慢前精索 \\ 为了这张皮,撸出社恐神衰 \\ 为了这张皮,撸到腰痛腿软 \\ 为了这张皮,撸到毁容射血 \\ 为了这张皮,撸成畜生不如 \\ 为了这张皮,撸成行尸走肉 \\ 皮下皆白骨,皮下即骷髅 \\ 色迷心窍者,只见皮,不见骷髅 \\ 皮就是他的一切,为了这张皮 \\ 他不断掏空自己身体的精华 \\ 像傻逼一样自残自毁自废 \\ 皮的魔力就是这么大 \\ 多少人看不穿这张皮 \\ 一辈子就迷在皮里面了 \\ 即使症状爆发还在想着这张皮 \\ 迷于皮的人生是可悲的人生 \\ 皮奴最终的下场会很惨 \\ 说白了,也就是一张皮而已 \\ 一张皮上九个孔,日日在流不净物 \\ 任它千娇与百媚,我自不动如磐石
        \end{center}
    \end{multicols}
\end{poem}

\begin{poem}[推送工]
    \begin{multicols}{2}
        \begin{center}~\\
            一个个黄昏 \\ 一个个夜晚 \\ 一个个百无聊赖的下午 \\ 他打开电脑 \\ 开始推送五脏六腑的精华 \\ 在射出前,片子的吸引力极高 \\ 在射出后,片子的吸引力暴跌 \\ 再看那些片子,简直就是垃圾 \\ 就像鸡肋一样没意思 \\ 他不知道为何要这样摧残自己 \\ 但他就是停不下来,就像着了魔一样 \\ 在一次次的推送后 \\ 他终于迎来了症状爆发 \\ 不是不报时候未到,时候一到立刻报销 \\ 身体开始给出严厉的警告 \\ 他知道该戒了,再不戒就是死路一条 \\ 有的人活着,他已经死了 \\ 像行尸走肉般活着 \\ 这就是撸到最后的惨状 \\ 太无望,太无力,也太悲催 \\ 他决定不再做推送工 \\ 推送工就是心魔的包身工 \\ 肾精被残酷压榨,毫无自由可言 \\ 最后身心俱废,沦为废渣
        \end{center}
    \end{multicols}
\end{poem}

\begin{poem}[凹陷的撸者]
    \begin{multicols}{2}
        \begin{center}~\\
            撸了几年后 \\ 他发现自己的脸颊凹陷了 \\ 颧骨突出,就像活骷髅一样 \\ 他端详着镜中的男人 \\ 如此熟悉又那么陌生 \\ 他做出了油画呐喊的表情 \\ 一种绝望而惊恐的感觉 \\ 开始撕心裂肺 \\ 他的灵魂因为手淫恶习 \\ 已经彻底扭曲和变态 \\ 那个纯真无邪的孩子 \\ 在沉迷手淫后消失了 \\ 原本脸上是饱满的 \\ 太阳穴那里也是平的 \\ 然而现在都凹陷下去了 \\ 看上去特别没有精气神 \\ 最可怕的凹陷出现在眼窝 \\ 给人一种衰老颓废的感觉 \\ 在学习戒色文章后 \\ 他明白了四个字:精竭容枯 \\ 在戒色一年后 \\ 他完成了蜕变 \\ 还是那面镜子 \\ 但里面的人 \\ 却是容光焕发神采奕奕 \\ 一扫之前邪淫的晦气 \\ 他变得自信满满 \\ 露出了久违的纯真微笑 \\ 那是心灵纯净的孩子 \\ 才能发出的笑容
        \end{center}
    \end{multicols}
\end{poem}

\begin{poem}[头发撸没了]
    \begin{multicols}{2}
        \begin{center}~\\
            一次梳头,左边肩膀几十根 \\ 右边肩膀几十根,还不算掉地上的 \\ 极为惶恐不安,不祥的预感已经降临 \\ 一次洗头,脸盆里漂浮着一层密集的落发 \\ 更加惶恐不安,已经能看见头皮了,接近崩溃 \\ 曾经浓密的黑发撸得稀稀落落,随时都可能脱落 \\ 秃顶男!这三个字从他的脑海中蹦出 \\ 太可怕了,三十岁还没到,就这样了 \\ 原本的自信再也看不到了 \\ 因为内心压力太大,转而又通过撸管发泄 \\ 这样就陷入了更深的恶性循环 \\ 肾气不断亏损,头发越来越少 \\ 看上去老了十多岁,简直无地自容 \\ 看着过去的照片,泪流满面 \\ 如果不撸,那该多好啊! \\ 撸管已经把他的生活完全毁了 \\ 惶恐!绝对的惶恐! \\ 崩溃!彻底的崩溃! \\ 撸者迟早会掉入痛苦的深渊! \\ 早知今日何必当初!
        \end{center}
    \end{multicols}
\end{poem}

\begin{poem}[两代人]
    \begin{multicols}{2}
        \begin{center}~\\
            爷爷在战场上冲锋陷阵 \\ 抛头颅洒热血 \\ 为解放全中国而战斗 \\ 孙子在电脑前色迷心窍 \\ 对着黄片开枪放炮 \\ 最后把自己的身体撸垮 \\ 依稀记得爷爷那刚毅正气的眼神 \\ 穿过漫漫岁月 \\ 依然那么震撼而温暖人心 \\ 一个邪淫的子孙 \\ 无颜面对列祖列宗 \\ 邪淫就是不孝 \\ 邪淫就是家族最大的耻辱 \\ 必须终结邪淫堕落的生活 \\ 一代人有一代人的正气 \\ 一代人有一代人的使命 \\ 虽然时代不同了,但正气不能丢 \\ 后辈要继承先烈的铮铮铁骨和浩然正气 \\ 与邪淫的生活一刀两断 \\ 彻底戒掉手淫恶习,自戒劝他 \\ 解放全中国的撸民
        \end{center}
    \end{multicols}
\end{poem}

下面推荐一本书。

\begin{book}[《现代因果实录》,果卿居士]
    果卿居士真名杨作相,天津人,已过花甲之年,他是宣化上人的弟子,是宣化上人亲自推荐的弘法居士,可信度很高。《现代因果实录》现在有一二三合集,由很多现代的因果故事组成,第一集的缘起部分写得很有意境,后面一个个真实的因果故事振聋发聩,给人很大的启示,我大约花了三天多就看完了,法喜充满,的确写得很好,值得反复阅读。果卿居士主要推荐的就是《地藏经》《梁皇宝忏》《楞严经》的四种清净明诲等,《现代因果实录》里讲的内容相对比较细致而全面,都是现代人常犯的毛病,里面也特别讲到了邪淫的问题,这部分的内容给我的触动比较大,现在这个社会邪淫的问题比较严重,对于邪淫的危害我们一定要有深刻而清醒的认识,一定要具备正知正见。果卿居士还写过一本《漫谈慈悲梁皇宝忏》,大家有兴趣的话也可以看看,里面也讲到了不少发人深省的真实案例。有些戒友在戒色后身体一直好不起来,他也在积极治疗,也很注意养生,但是改善不大,这时就要考虑业障病的可能,可以试试念佛、念经、放生和忏悔,很多人都是学佛修行后身体彻底好起来的。
\end{book}
