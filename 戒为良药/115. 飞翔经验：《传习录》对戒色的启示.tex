\subsection{《传习录》对戒色的启示}

\paragraph*{前言}

这季前言讲下沉迷游戏的危害,有的新人喜欢用沉迷网游来转移注意力,以为这样就能戒色成功,这其实是一种思想误区,一个人可以疯狂地玩游戏,但是心魔迟早还会入侵。在网游的世界里打打杀杀,打赢了起骄傲心,打输了起嗔恨心,想着要报仇,这种心态对戒色很不利,只会增加自己的负能量,让自己戾气缠身。前段时间一个打游戏的戒友破戒了,去网吧厕所破戒的,沉迷游戏会让人暂时忘了撸管这回事,但是人总有闲下来的时候,到时就容易被心魔攻破。觉悟没有任何实质的提高,只是靠打游戏来转移注意力,这种戒色方法实在不可取,现在的网游多植入色情诱惑的元素以增加人气,这样对戒色会起到相反的作用。记得我做学生党时也沉迷过游戏,那时玩的是三国志和一些闯关游戏,上午玩了下午接着玩,浪费了很多时间,也荒废了学业,后来我知道这样不好,就渐渐减少了玩游戏的时间,只是偶尔玩一下,我把自己的兴趣爱好更多地投入到阅读和运动上去了。现在的网游还要花钱买装备,玩游戏不仅浪费时间,还浪费大量金钱,沉迷于网游的人精神高度集中,伴随着血液加速、心跳加快,人的体力、精力消耗很大,有时在玩的兴头上就连吃饭、睡觉也忘了,致使过度疲劳,严重的话还可能发生猝死,网吧经常有猝死的事件,一坐十几个小时,吃饭也不规律,睡眠时间很少,而且还经常处于兴奋状态,这样很容易发生猝死。最近常熟一家网吧发生了一起悲剧,90 后小伙小陈在连续上网 23 个小时后,倒地猝死。小陈上厕所时,在洗手间门口突然晕倒,再也没有醒过来。据网吧工作人员称,当时小陈是突然直挺挺地倒下去的,怀疑是疲劳过度,心脏骤停导致的猝死,熬夜加上精神紧张,会对身体造成严重的损害。戒色后最好把网游也给戒了,我在戒色后把杀戮的游戏给戒掉了,现在偶尔会玩玩益智类的游戏,但也玩得很少,我感觉我的生活更充实了,可以节省很多时间来干其他的事情。网游的危害媒体已经做了很多报道,戒色后是不应该沉迷于网游的,自己必须高度自律,好好规划自己的生活,偶尔玩玩轻松益智类的游戏还是可以考虑的,但也不可沉迷其中。

下面分享一些案例。

\begin{case}
    大致 2006 年中旬开始,直至今天,不知不觉间我竟然做了这么久的 loser。我再也不敢相信自己的能力有多么的大,我已经悬在深渊之上。我的身体在前几年的时候就出现了状况,只可惜一直以来没有正确的认识,对自己的身体一点都不了解。去年年初,我请假回老家做了手术,因为当时检查出前列腺炎和弱精症,以及阳痿早泄。做手术的时候心里很憋屈,一方面是花费很高,另一方面是期望的效果并没有出现。男科医院的收费简直令人发指,总共十天花费了将近三万元,可恨的是医生连最起码的伤口恢复都没有处理好,回到工作单位我又陆续治疗了半个多月,短短的一个月我花光了积蓄。一直以来我总以为自己能够控制得住邪淫的思想,总是有一种侥幸的心理,直到问题发展到了做手术的程度,我才如梦初醒,内心悔恨不已。我总是在想,如果当初不过早地接触邪淫,是否今天又会是另一个场面?我总以为自己会突然地好起来,总觉得邪淫并不是那么可怕的事,直到自己一路跌跌撞撞不断地失败,乃至心生忧郁、颓废!邪淫侵害了我的思想,让我在不成熟的青春期过早地透支了生命的财富,而现在所发生的一切都是咎由自取,世上的事情有因必有果,种下了罪恶,收获的也只能是苦果。

    \textbf{附评} 一入撸门深似海,回首沧桑心亦哀。刚开始撸管时,往往太年轻无知,而且还被无害论洗脑了,快感的魔力又是那么强大,很多人初尝快感都会说太爽了!一下就被快感给征服了,给迷住了,然后就开始沉迷于看黄撸管的堕落生活,脑子被邪念牢牢占据。在那个年龄段,几乎很少有人知道这个恶习会导致什么样的后果,等到后来症状爆发了才渐渐明白获取快感是要付出代价的,很惨重的代价,最终的下场甚至可以用惨烈来形容。这位戒友因为撸管得了前列腺炎和弱精症,还有阳痿早泄,这都是很典型的伤精症状,也是因为无知,他选择了男科医院,这也许就是报应之一,邪淫者很容易耗损钱财,很多人都反映钱留不住。你掏空自己的肾精,医院就掏空你的钱包,看谁掏得狠!正规医院花费还好些,进了男科医院花费就大了,到时真的是人财两空,欲哭无泪。\textit{求名者因好色欲而名必败;求利者因好色欲而利必丧;居家者因好色欲而家业必荒;为官者因好色欲而官业必坠。考之往古,验之当今,有历历不爽者,且欲心即众恶之因也。恶因日积,罪孽日深,显则倾家荡产,一家之衣食无依。阴则削禄减年,一生之荣华尽丧。甚至精竭髓枯,神昏血尽,百疴丛起,一事无成,皆因好色一念害之也。可不畏哉?可不惧哉?(《少年进德录》)} 这位戒友之前认为邪淫并不是那么可怕的事,在某个阶段的确会产生这样的错觉,因为那时身体还好着,根本意识不到邪淫的危害,在那个阶段也很难相信邪淫的危害,总以为前辈的告诫是在耸人听闻,等到自己真的挨上了,才知道前辈的告诫句句属实。大家现在都知道吸烟有害健康,香烟盒上也印着这个警告,但有些人抽了几年依然好好的,难道就可以说吸烟是无害的吗?吸烟的危害有滞后性,撸管也是如此,恶因是慢慢积累的,量变产生质变,不能局限于自己有限的体验而片面地下论断。很多人都有侥幸心理,以为那些伤精症状离自己很远,好像不会降临到自己头上,这其实就是在自欺欺人,撸管的恶果迟早会显现的,甚至以你意想不到的方式把你彻底击垮。如果不撸管,也许人生就是另外一番景象了,治病花的钱就可以省下来,用于孝敬父母或者创业等等。因撸致病,因病返贫,这种案例实在太多了,自己不懂得珍惜,一次次糟蹋自己,最后症状爆发,人生就会变得极度灰暗。
\end{case}

\begin{case}
    戒色已经四个多月,身体状态好起来的同时,更让我感叹的是心理状态的好转,有很多人都说我的面相发生了一些变化,我觉得才四个多月容貌是不会改变的,改变的只是神态眼神和精气神。人有精气神,身体不虚,神态自然就放松安详,这不是伪装和自身努力纠正就能做出来的,如果感觉无论怎么做别人看你的眼神都不对,请从自身找找原因,是不是自己身体虚弱造成的。这是我目前戒色的心得,我会持续关注飞翔哥的文章,飞翔哥法布施功德无量!

    \textbf{附评} 这位戒友戒了四个多月,心理状态和容貌都有所改善,其实只要精气神提升了,容貌就会改变,精气神有塑形的效果,有的戒友戒色后原本浮肿扭曲的脸庞渐渐恢复正常了,变得紧实精致起来,到时真的焕然一新,五官和脸庞进入了最佳的状态,就像运动员进入了最佳的竞技状态一样,有一种特别的底气和自信。这位戒友也提到了神态和眼神的变化,戒色后的神态会显得放松祥和,有一种美妙的和谐感,国外戒色文章用到了“镇定和轻松”,戒色后的确有一种特别的气场,看人时的眼神都会很特别,好像两束强光一样,能量很集中很强,普通人看到你这样凛然正气的眼神就彻底服了,看别人一眼,别人就会敬畏你,戒色会带来这种不怒自威的效果,之前不少戒友都反馈过。有些人根本无法和你对视,你看他一眼,他的眼神马上闪躲,因为他感觉到了强大的正能量的威慑,而他的正能量没有那么强,所以根本接不住你的视线,只能选择躲避,就像一只老虎看向一条狗一样,那条狗马上就会感觉不自在乃至畏惧。能量养足后,两只眼睛就会炯炯有神,顾盼生辉,能量泄掉后,两只眼睛就会失去神采,变得空洞无神,眼睛也会出现微妙的变形,显得丑陋许多。戒色就是能量管理,现在比较流行的一种观点就是管理你的能量,而非时间。时间管理是很重要,但是能量管理显得更加重要,因为高能量可以创造高绩效!你可以把时间管理得很好,但是没那个精力、没那个能量去做事,那就等于零。中医讲:肾主志。这个志,就是一个人做事情的能量、斗志、动力,纵欲后人很容易变得消沉,甚至是斗志全无、不在状态,曾经的雄心壮志随着手淫这个恶习烟消云散,再也没有以前那种高昂的热情与斗志去做事情了。肾为作强之官,肾精不断耗损,身心就会趋于虚弱,短期可能看不出,时间长了变化就会很明显,你会发现自己的意志、斗志和胆量出现了严重的下滑,做事情会变得拖沓犹豫,做事效率会大幅下降。国外戒色文章说:“手淫会让人看起来像个失败者,并且会摧毁一个人的自信。”loser 本来就有失败者之意,被心魔打败,身体被掏空,完全就是一个失败者,记得一位戒友戒了接近三个月,他说:“整个人明显发生了改变,气色变好了,社恐也减轻了,人也自信了。”男人拼的就是肾,肾精就是核动力,一个人如果懂得戒色保精,那么他就有足够的勇气和强劲的动力去面对人生的一切挑战。
\end{case}

\begin{case}
    今天这个医生医德很好,给我千叮咛万嘱咐,千万不能 SY。看了不少中医没有一个给我说戒掉 SY 的,而这个大夫一直叮嘱我说,不戒掉的话永远不会好的,他说药物只是个辅助,这个病还得靠自己,还不厌其烦地给我心理疏导差不多半个小时,这一点我真的很感动。我断断续续戒了一年多了,大部分症状都好了,现在就是睡眠质量差、梦多,早晨大便不成形,记忆力不好。大夫说这是长期 SY 导致的脾肾两虚。各位戒友时刻要提醒自己适度 SY 会害死人,我就是一个活生生的例子,中医这几年没少看,钱也花了好几万了,还没有痊愈。祝大家早日找回从前那个健康阳光的自己!

    \textbf{附评} 中医方面有这样一句话:上医医心,中医医人,下医医病。真正的好医生,不仅仅是帮你看病,更重要的是指出你问题的根源所在,手淫恶习不彻底戒掉,那身体就万难痊愈,即使暂时好了,只要继续手淫,很容易就会复发。这位戒友断断续续戒了一年多,症状改善了很多,但还有一些症状没恢复好,我们一定要彻底戒色,尽量避免出现破戒,否则会影响到身体的恢复,当然身体的症状如果比较明显,也应该寻求积极治疗,三分治疗,七分戒色养生。睡眠质量差和多梦考虑轻微神衰的表现,大便不成形考虑脾虚,医生的诊断很正确,而且还告诫他要戒掉手淫恶习,这种好医生真的很少见,遇见医德好的医生也是自己的一种福报,有的人遇见的医生还向他灌输无害论,这类医生简直就是在误导患者。\textit{养生有五难:名利不灭,此一难也;喜怒不除,此二难也;声色不去,此三难也;滋味不绝,此四难也;神虑转发,此五难也。(嵇康《答难养生论》)} 分别指追逐名利、狂欢暴怒、贪恋声色、嗜食肥甘、情志不稳对养生的不良影响,第三难就提到了声色,色字头上一把刀,色是刮骨钢刀,撸者不是在养生而是在戕生,戕害自己的生命,他们因为无知还以为自己爽到了,其实是亏大了,养命的能量都被耗损了,等到症状爆发就悔之晚矣。“耳不闻淫声,目不视淫色”历来被视为正人君子修身养性的行为准则,只有下定决心与放荡堕落的生活一刀两断,才能与健康长寿逐步接近。\textit{人生而命有长短者,非自然也,皆由将身不谨,饮食过差,淫泆无度,忤逆阴阳,魂神不守,精竭命衰,百病萌生,故不终其寿。(《道机》)} 年轻人总是显得很无知,如果没有老一辈人的苦心告诫,那就很容易步入歧途,一旦体验到了快感,那就欲罢不能了,大剂量多巴胺的释放直接冲昏了头脑,让人如痴如狂,不顾一切想要快感,等到症状出来后才知道手淫是要还的,用痛苦还,用医药费还,把身体撸废了才知道快感就是一场多巴胺的欺骗,你什么也没有得到,除了一身的症状和无量的痛苦。很多人都是花了好几万、跑了很多次的医院才真正悟明白这个道理的,那些无知的撸者应该早点醒悟,否则执迷不悟只会把自己推入万劫不复的深渊。
\end{case}

\begin{case}
    前几天刚刚破戒,自以为是地认为我不会傻到连续破戒,而且一直以来都在学习,完全没想到失算了。破戒经历:上网搜索资料,然后突然冒出一个擦边图,心魔:“刚才那是什么,没看清,再看一次”,然后听从该指令,结果渐渐引人入胜,最后欲火中烧,不得不破。断念贵“早”!对境实战要保持警惕,稍有欣羡,便趋欲境!

    \textbf{附评} 这是一个对境实战的案例,上网有很多诱惑,各种小框会跳出来,还有很多擦边新闻,自己一定要保持高度警惕。对境实战很重要的一点就是不要看第二眼,这位戒友用的词是“引人入胜”,实则是引人入“陷”,往往第二眼就陷进去了,第一眼没看清,第二眼聚焦追看细节,就是这第二眼让人陷进去了。菩萨见欲,如避火坑,凡夫见欲,飞蛾扑火,这就是实战意识的差距,高手能够及时避开诱惑,菜鸟的视线则会粘上去,我们的视线要避免在诱惑的图片上停留,看见马上移开视线或者开启散视模式,应该立刻闭关页面,不可有丝毫的贪恋和犹豫,不可放任自己的好奇心,必须坚决果断执行,对境时的行动必须特别坚决!当你对境时遇见诱惑了,心魔也会跳出来怂恿你,国外的戒色文章把心魔比作内心的 ghost(幽灵),的确很形象,你不知道心魔何时会出现,但肯定会入侵,你必须时刻警惕,不要中了心魔的阴谋诡计。上次一位戒友戒了八十多天,突然有一天心魔怂恿他去试,于是他听从了那个念头,然后就破戒了。对于心魔的怂恿,一定要学会识破,不可听从那个指令,那个怂恿的念头就是木马程序,跟随即是安装,然后就像被附体一样,沦为撸管肉机,身不由己。一旦被心魔攻破,就很容易出现连续破戒,在破戒后一定认真反省和总结,加强学习提高觉悟,努力提升自己的实战水平,这样才能避免连续破戒。破戒最根本的原因就是断念实战不行,你可以说自己无聊,也可以说自己情绪不好,压力比较大,但最后肯定有某种念头驱使你去撸,那种念头不断掉,就会破戒!有的戒友说自己没意淫,直接就撸了,虽然他没意淫,但他心里出现了一种微妙的感觉,那就是“想撸了,想看黄了,想放纵了”。这种微妙的感觉其实就是比较细微的念头,必须立刻断除!\textit{佛法的要旨是,要自己作得了主,能降伏自己的妄心。(贝诺法王)} 戒色也是如此,要降伏所有的邪念,你自己必须当家作主,不能让心魔骑在你的头上作威作福。这位戒友说断念贵“早”,说得很对!但真正的考验在实战时,说到做不到,根本没用!功在平时,平时不断练习断念,断念之刃越来越快、越来越狠,到时心魔一露头,就能斩立决!你这里就是断头台,就等心魔前来送死!!!
\end{case}

\begin{case}
    我是 1988 年出生,年近三十,现在博士在读。从 2015 年 1 月 1 日开始戒色,前半年属于强行戒色,并没有系统地学习戒色文章,后来有了骄傲的情绪以为戒色很简单,加上受到黄色信息的扰动而破戒,再到 2015 年下半年破戒开始越来越多(自己有记录做了统计,2015 年破戒次数高达 32 次之多,都是集中在 2015 年下半年)。到 2016 年的 4 月份进入了破戒相对较少的时期,再后来中间回家和父母说明自己撸者的状态,父母刚开始对我有点失望,原来这么多年来自己状态不好的根源在此,但是后来父母还是鼓励我坚持戒色养生,因此,从 2016 年 7 月 18 日到现在从未破戒,到今天已经 188 天了,偶尔有遗精。戒色的过程中恢复的感觉很好,体力和脑力也非常棒,晨勃质量也较好,博士学业也提升,和周围同学关系也变得融洽。总之,戒色吧真的让我的生活越来越好。因为有了之前的经验教训,所以从 2016 年 7 月 18 日,每天都会登录戒色吧,每天都抽出时间坚持学习戒色文章,现在深刻地知道,我需要保持高强度的警惕性,过了欲望休眠期就是破戒高峰期,所以依然坚持学习戒色文章来提升觉悟,现在状态非常好,我仍然要继续坚持努力学习,保持高的警惕性。

    \textbf{附评} 之前也出现过博士戒友,能读到博士实属不易,但博士也会受到邪淫所带来的困扰,有的人智商很高,在邪淫的状态下都能考上名校,这类人是存在的,一方面是智商高,另外一方面就是伤精程度还不是很严重,但是随着年纪的上升、伤精程度的加深,邪淫所带来的负面效应就会变得越来越明显。这位戒友之前属于强行戒色,没有系统地学习戒色文章,他那时能戒半年也很不容易,完全的强戒很难撑过一个月,学习了戒色文章但未系统深入,这样很难撑过半年,之前有的戒友学习了十几篇戒色文章,戒了四个多月,但后来没有坚持学习提高觉悟,就越戒越差了,出现了频繁破戒。我们必须坚持学习提高觉悟,不能浅尝辄止,要有谦虚谨慎的心态,千万不可骄傲自满,一旦起了骄傲心,那就极易翻车!关于骄傲导致破戒的情况实在太多了,这个问题我之前也强调了很多次,大家一定要引起高度的重视,一念骄傲,祸害无穷!满招损,谦受益!不管戒多久,永远记得保持谦虚谨慎,要低调行事,不可有丝毫的张扬。\textit{醲肥辛甘非真味,真味只是淡;神奇卓异非至人,至人只是常。(《菜根谭》)} 真正有德行的人,往往显得平凡无奇,但是他的精神境界远超于常人。戒到最后拼的就是德行,厚德载物,一骄傲就全不是了,我戒到现在都不敢起一丝一毫的骄傲心,一起骄傲心,后果太可怕了。永远不要觉得自己比别人强,一定要懂得谦下!谦德圆满了,戒色的道路就会宽坦很多。这位博士戒友的善根和悟性还是很不错的,吃过一次亏,就立刻调整过来了,他现在真正懂得了警惕,也在坚持学习戒色文章。他真正体验到了戒色带来的好处,戒色可以提升一个人的能量水平,脑力、精力、体力变好了,心态也会随之变好,到时人缘自然会变好,戒色会让人获得一种如鱼得水的感觉,身心被肾精滋润后,那种感觉非常棒!你的高频率振动可以提升周围的能量场,为周围的环境设定振动的基调,别人会不由自主地接收它,当你处在一位非常慈悲、内心非常纯净的人身旁时,你的内心自然会升起美好的感觉,你会被那种崇高的能量场所深深感染。负面的念头会拉低你的振动频率,让你作茧自缚,而善念则会提升你的振动频率,让你感受到更高层次的快乐、充实与满足。
\end{case}

\begin{case}
    那个暑假是我人生中最快乐的一个暑假,因为我活在纯净的世界里。后来开学了,慢慢地由于戒色给了我福报,当时第一次月考直接考了全校第二名,要知道我入校是全级 61 名。在我戒色 81 天的时候受了心魔的怂恿去看黄了,但由于当时戒色的初心和自己那颗上进的心,我关掉了浏览器。那段日子,我学习了很多戒色知识,学习了传统文化,更从飞翔大哥的文章里学会了养生。当时我从不熬夜,每天锻炼,孝顺父母,人生可谓是到达了巅峰,当时的那段岁月真的很快乐。记得当时我一照相突然发现自己怎么这么帅了,精气神回来了就是不一样!

    \textbf{附评} 这是一个十五岁小戒友的帖子摘录。他戒过 98 天、169 天,后来就不行了,而且还迷上了网游。小孩毕竟心性不稳,很容易再次陷入怪圈,心魔大 BOSS 异常强大,必须要坚持学习戒色文章,不可有任何的松懈。戒色 98 天那次,他说:“这是我人生中最辉煌的时候。……那种感觉好爆了。”可惜后来他破戒了,虽然又戒了一次 169 天,但终究还是破戒了,后来他疯狂沉迷网游,更是荒废了青春,他在帖子里写道:“我迷上了这个游戏。我开始为之不断付出,每天拼命地打,一打就是十几个小时。晚上打到三点钟才睡觉,学习成绩一落千丈。看着曾经的竞争对手离我越来越远,我内心的痛苦是没有人懂的。自从打了这个游戏,我开始不学习戒色文章,打一整天的游戏,自然曾经的戒色觉悟没了,一下战斗力为 0 了,过着被心魔虐爆的生活,当时我一有欲望就根本抵不住心魔的怂恿!”迷上网游后,时间和精力都给了网游,戒色文章也不学习了,结果可想而知,这个小戒友从巅峰跌落到了低谷,他疯狂玩网游,疯狂破戒,和父母大吵,又变得负能量爆棚了。好在他认识到了错误,现在他又重新开始戒色了,网游最好不要去玩,太耗费时间,也对养生很不利,久坐久视甚至还要熬夜,也会让人进入走火入魔的状态,一头扎进网游的世界里,其他都不管了。在那个虚拟的世界里即使你练到满级,在里面称王称霸了,在现实生活中依然还是一只懦弱的蛆虫,心魔大 BOSS 你不去打,跑到网游世界里去打打杀杀,完全是搞颠倒了。你最该打的“网游”是在你的两耳之间,打自己的念头怪,真正的战场在脑海!戒掉手淫恶习,做回纯洁的少年,好好孝顺父母,把主要的精力和时间用在学业上,这样将来才有光明的前途,人的命运就掌握在自己手里,不要亲手葬送了自己!
\end{case}

\begin{case}
    邪淫带来的心里上的痛苦,记忆力思考力的下降,意志的消沉,让我成绩总是不尽如人意,高考理所当然失败,原本正常能考上重本的我,只考上了一个普通的一本。原来对我寄予厚望的家长伤心欲绝,得知成绩的那天,看着伤心的母亲,我跪在地上,狠狠地扇了自己无数个耳光,悔恨不已。不甘心的我选择了复读,在复读期间,我虽然破破戒戒,但心里憋着一口气的我把自己的时间全放在了学习上,而且也有意识地看戒色文章,加强正念,破戒的次数明显少多了。自己也有意识地记笔记,做记录,最高的一次戒到了 57 天,我觉得我还是能行的。最终自己也考上了一个重本,还找到了我的真爱。高四,我觉得因为戒色,让我的运势变得更好,精力也变得更好。我怀念那些日子,虽然我没有彻底地戒除,但是我觉得是因为戒色让我变得更好,让我明白了远离邪淫的日子是多么的快乐,多么的自在。我有生第一次尝到了戒色的甜头,真的很感谢戒色的那段日子,我觉得我获得了重生。邪淫的我,是多么的痛苦与不堪,而戒色的我是多么的阳光与积极,充满朝气与活力,我永远感谢那段日子。现在回想起来,戒色的日子,才是真正人活的日子啊!但是好景不长,由于暑假的彻底放松,整天无所事事,这时淫魔又来袭了,无聊的我再次被困住,一次沦陷以后,便步步沦陷,暑假期间,无数次地看黄手淫,并且学会了用种子,淫念一天比一天强烈,这时的我,完全没有了高四时的斗志,被淫魔整日地摧残,虽然我知道这不好,也很悔恨,但是一看黄,就什么都忘了,心思意念紧紧地被淫魔所抓住,挣脱不出,每天都混混沌沌,而且我发现,就是每次手淫过后,腰特别地疼,腿发软。现在回忆起来也是一阵后怕,我发现,只要一碰到黄色信息,便会被控制住,大脑完全被操控,到那时根本不由你,完全变成了一具撸管肉机,直到把精液撸出来为止,那时你会变得无比的虚弱与悔恨。这真是无尽的死循环,堕落于深渊而不知。

    \textbf{附评} 这位戒友之前也尝到了戒色带给自己的好处,戒色的自己是那么阳光与积极,远离邪淫的日子是那么的快乐,复读期间虽然没有做到彻底戒除,但已经带给了他很强大的能量,让他有足够的脑力和精力去冲刺高考,如果他能彻底戒除,也许会考得更好。远离邪淫就是远离束缚,做回阳光纯净的自己是非常开心快乐的,你能重新体验到儿时纯真无邪的那种大快乐,非常美妙的体验。这位戒友说:“戒色的日子,才是真正人活的日子啊!”这句话真的是发自内心的感悟,在手淫怪圈中挣扎、被心魔奴役的日子是十分灰暗和颓废的,因为无法主宰自己,总是一次次被心魔拖入手淫的怪圈,那种日子真不是人过的,看着自己变得禽兽不如乃至恶心变态,越来越远离纯净美好的状态,彻底沉溺在邪淫的低级趣味中,身上的负能量也变得越来越重。手淫看似可以帮助释放压力,但是之后却让人陷入了恶性循环,到时就会变得惶恐不安和更加地压抑难受。手淫不会让人真的快乐起来,短暂的快感并非真正的大快乐,看黄时你那禽兽不如的眼神充满着邪恶,那猥琐丑陋的动作让人大跌眼镜,有位戒友说:“撸管是这个世界上最龌龊的事情。”我觉得他说得很对,这个恶习是很隐秘的、见不得人的,就像在吸食隐秘的毒品一样。普林斯顿大学的 Jeffrey Satinover 教授将色情的影响如是描述给美国参议院委员会:“这就像我们发明的一种海洛因,色情观看者可在自己家中秘密使用并通过眼睛直接注射进大脑。”看黄手淫其实就是一种吸毒的行为,只不过吸食的是黄毒,虽然释放的多巴胺剂量没有海洛因那么大,但也会形成和吸毒一样的负面效应。很多撸者都觉得自己像行尸走肉一样,看上去就像一个吸毒佬。现在最新的科学研究已经证明了色情和手淫的巨大危害,过去那套无害论应该被彻底淘汰,最新的科学研究和中医理论是高度契合的,最近看的一篇国外的文章,里面写道:“According to Medical Daily, an organisation renowned for health and science issues, masturbation, which is highly addictive, could lead to fatigue, memory loss, hair loss, blurred vision, pains in the penis and impaired sexual function like erectile dysfunction, low sex drive and premature ejaculation.(根据医学日报,一个著名的健康和科学问题的组织,手淫,极易上瘾,可能导致疲劳、记忆力减退、脱发、视力模糊、阴茎疼痛和性功能受损,如阴茎勃起功能障碍、性欲低下、早泄。)”手淫这个行为是“highly addictive”,高度成瘾的行为,必须彻底戒除,在圣贤教育看来,手淫这个行为是一个染污灵性、伤身败德的恶习,对于手淫一定要有一个正确的认识,不能听信砖家那套歪理邪说,你可以拒绝相信手淫的危害,但最后的事实肯定会给你一记响亮的耳光,无害论是打着科学幌子的伪科学,这已经被很多人认清了。不要再做撸管肉机了,那是一种无力颓废、积累负能量的状态,在那个怪圈和死循环中,一个人的前途只会变得越来越黑暗。和心魔斗争是一场硬仗,要有死战之勇,拼了命也要和心魔干到底,干死心魔!做戒色的烈士,不要做邪淫的懦夫!
\end{case}

\begin{case}
    我今年 23,2016 年下半年大学毕业,撸了大概有八九年,到现在几乎把身体玩垮,耐力体力下滑得不行,跑个两公里要命了,头重脚轻,可能大家觉得两公里不错了,问题是我原来是篮球的体训生,两公里的强度顶不了一节正式比赛,三年前我是连着两场打满四节的节奏。然后年底去医院检查,把我吓一跳,胆和肾什么的到处都是问题,准备过一段时间去做切除手术,肤质差得要死,皮肤黄得一逼,气色差得要死,黄不拉几的还老长痘,我又手贱去挤,完了现在留下好多痘印。现在每天晚上都要起来小便一次,早上起来老是觉得没睡够,这几年就没真正睡饱过,睡眠质量差得要死,神衰记忆力衰退,在我身上体现得非常明显,别人说的什么我一下就忘了,东西放在这里,转身离开一下就不记得放哪里了,一思考什么题目脑子像是有浆糊,昏昏沉沉在那搅。

    \textbf{附评} 一个体训生也会因为邪淫而沦为废人,而这位戒友也只有 23 岁,记得曾经有一位挺有名的民间扣将,却有着伤精者的颓废面容,后来也做了手术,然后基本就销声匿迹了。身体再好也会因为手淫恶习而废掉,撸的时间长了,问题肯定会渐渐出现。有的人一身肌肉疙瘩,外表感觉很强壮,但双眼却空洞无神,我就知道他外强中干了,肌肉发达不代表健康,肌肉男也会患上神经衰弱。这位戒友作为体训生,身体素质应该远超常人,但是经年累月地掏空自己,最后的结果也不容乐观,身体失调后,内脏就可能长囊肿、结石乃至肿瘤。身体好的时候真的不觉得,但伤得深了就会发现一年不如一年,身体会给出严重警告,告诉你不能再手淫了,看着自己憔悴的脸庞,你就知道你的内脏是什么状况了,面部与五脏六腑是有对应关系的,中医有专门的面部全息图,如果发现自己肤质和气色差得要死,那真的要引起高度警觉了。曾经是满场飞奔的篮球体训生,多么阳光多么富有活力,但是沉迷手淫八九年后,体能、精神、容貌、脑力都大幅下降了,还要面临手术,这样的落差对人的打击真的很大。道书有曰:“人生欲念不兴,则精气舒布五脏,荣卫百脉,及欲念一起,欲火炽燃,翕撮五脏,精髓流溢,从命门宣泄而出。”大家看看邪淫的后果多么可怕,把养五脏的能量都给射掉了!欲火焚烧,精神易竭,遂至窒其聪明,短其思虑,有用之人,不数年而废为无用。普通人很难认识到邪淫的危害,他们被无害论洗脑了,等到他们废掉了才知道手淫的危害有多么大。有些人会说自己现在没事,但现在没事不代表将来没事,很多人之前都没事,但是等到伤得深了,症状就开始爆发了。撸进医院,撸上手术台,医生用那冰冷的手术刀切开撸者猥琐的躯壳,这边给你剪一刀,那边给你划一下,躺在手术台上感觉特紧张、特无助,做完手术感觉身体非常痛,也无法动弹,到时呆呆地看着天花板,悔意顿时涌上心头……我曾经也上过手术台,而且还不止一次,手淫后鼻炎变得很严重,最后接受了手术治疗,那种痛苦的感觉我记忆犹新,我那时鼻中隔偏曲、下鼻甲肥大,医生就用剪刀剪鼻骨,剪下鼻甲,动作非常粗暴,就像木工干活一样,虽然上了局麻,但还是感觉特别痛苦,如果不上麻药,那估计要直接痛死了!手术完了把又粗又长的纱条塞满整个鼻孔,几乎是塞到爆的程度,快出院的时候取纱条也很痛苦。做手术时是蒙着眼的,如果看到医生用剪刀伸进自己的鼻腔,那是非常恐怖的场景,患者会很抗拒,所以一般都是蒙着眼的,手术过程中医生不仅用了剪刀还用了锤子等工具,切骨头的时候那真是度秒如年,钻心的疼,现在我都不忍多回忆,实在太痛苦了。手淫真的是要还的,用痛苦加倍地还!沉迷手淫恶习,最终引爆的痛苦是无量无边的,到时体验到了就知道了。\textit{汝年尚幼,须极力注意于保身。当详看安士书中欲海回狂,及寿康宝鉴。多有少年情欲念起,遂致手淫,此事伤身极大,切不可犯。犯则戕贼自身,污浊自心。将有用之身体,作少亡,或孱弱无所树立之废人。(印光大师)} 手淫这个恶习必须彻底戒除,一定要下最大决心来戒!这是你死我活的战斗!拼死戒!戒到死!血战到底,正气冲天!!!
\end{case}

\begin{case}
    飞翔老师,求救啊!我元旦的时候熬夜两天加上手淫四次,就在前几天突然心跳加速,手脚无力,持续了几十秒然后就恢复了,过了几分钟又出现了,就这样反复了几个小时,太恐怖了!现在只要到了晚上都会这样,去医院检查了心脏也没有事,我现在该怎么办啊?好怕突然猝死!怎样可以快点恢复,我再也不敢 SY 了!

    \textbf{附评} 熬夜时手淫、劳累时手淫、酒后射出、饥饿时射出、连续射出,这几种情况相当危险,在身体虚弱时是很忌讳泄精的,熬夜本来就对身体的伤害很大,又熬夜又手淫,后果不堪设想,这简直是在玩命!不要让你的父母或者同学在第二天叫不醒你,他们发现的只是一具冰冷的尸体。现在网络色情的诱惑非常猛烈,可以提供无限的新鲜感,而且还有各种变态的片子,在这个时代,一个人如果沉迷于看黄手淫的生活,废掉的速度要比过去快很多,过去还需买黄碟,几个黄碟也容易厌倦,现在有黄网,可以免费看,新鲜的诱惑实在太多了,一个人很可能会陷入走火入魔的状态,陷入对色情的彻底疯狂和迷恋,有的撸者通宵找黄看黄,或者一看黄就是一下午,进入了废寝忘食的状态,看黄的时间过得太快了,这种堕落的生活太浪费时间和生命了。当你走火入魔了,很多事情就不考虑了,全部抛之脑后,唯一想到的就是要射出来。很多人都是在熬夜射出后出现的濒死感,到时那种恐怖的感觉一上来,那真的太让人抓狂了,可以让人瞬间崩溃,快感过后就是猝死,谁会想到这个结局?即使没死,出现濒死感也绝对让人恐慌。中医讲到:心为火,肾为水,心肾交济,阴阳相和,故人体康泰,反之因为纵欲心肾不交,就会出现心脏不适,中医强调“心肾相交,水火既济”,如果“心肾不交,水火未济”,身体就失调了,到时肯定会出现症状表现。疯狂泄精,心脏很可能会出现不适,比如心跳加快或者疼痛等,之前一个案例就是熬夜时连撸两次,最后心脏狂跳,眼前发黑,呼吸急促,感觉快要死掉了。有的人虽然没有出现濒死感,但手淫后心脏疼痛心脏不适的感觉很明显,手淫对五脏的危害都很大,伤到一定程度就知道了。如果检查没事,那一般就按心脏神经官能症来治疗了,必须彻底戒色,加上养生和积极治疗,这样身体才能慢慢好起来,伤到这种程度,恢复速度就比较慢了,可能要坚持半年以上才能看到明显的改善。快感只有几秒,但是代价却是极其惨痛的,身体垮下来后,前途一片黑暗,体内的洪荒之力要用在正道上,不能用在邪淫上,天道祸淫最速,没撞南墙的赶紧回头,否则等待你的无异于一场劫难!
\end{case}

\begin{case}
    本人自十岁起无师自通地学会手淫、意淫,严重影响身体发育及学习。身高 1.68 米、体重 75 公斤,喜食肉食。现年四十岁,已婚,婚后看黄、意淫严重,得病前还一直在看黄,越看越想,总是看不够,大家懂的,一年四季大街上看到漂亮女人就意淫。近三年来也想戒淫,知道不好但好像越戒越淫。看过陈大惠、彭鑫老师的视频,视频中讲到邪淫导致糖尿病、中风、截肢,感觉说得离自己太远了,自己身体好又没有家族史怎么会得糖尿病呢?感觉陈老师在唬人,不可能发生在自己身上,但事实还真就发生了,今年九月份本人因消瘦、乏力,四个月后查空腹血糖 11.7、餐后两小时 29.1,正宗的糖尿病。从五月份到九月份人瘦了一圈,从 75 公斤降至 65 公斤。现积极治疗、素食、戒淫(想淫也淫不起了,阳痿),体重逐渐恢复,但是精神面貌很差皮肤松弛老态。自己双亲七十多岁,血糖各种正常,务必请各位同道悬崖勒马。有上述症状者立即戒淫或检查,要相信陈老师讲的,没有骗我们,糖尿病、中风、截肢离邪淫的人太近太近了,想一想糖尿病的并发症,想一想年迈的双亲和年幼的孩子,该醒醒了。我们非但没有顶天立地撑起这个家,反而给家庭带来沉重的负担,望各位同道离开自己那个阴暗潮湿的邪淫世界,不要再受淫魔的奴役,找回本该属于你自己的那份自信、勇敢、担当,做个好男人。

    \textbf{附评} 这是名为“骨髓枯”的戒友所写的帖子摘录,吕洞宾写过一首诗:“二八佳人体似酥,腰间仗剑斩愚夫。虽然不见人头落,暗里教君骨髓枯。”这首诗写得相当好,肾主骨生髓,髓分骨髓、脊髓和脑髓,皆由肾精所化生,肾精的盛衰不仅影响骨骼的发育,而且也影响脊髓和脑髓的充盈。这位四十岁的戒友十岁就开始手淫了,估计当时还射不出,一般进入发育期才能射出,手淫明显影响了他的身高,只有 168 \unit{\cm},有的撸者虽然也能长到 180 \unit{\cm} 乃至 190 \unit{\cm},但在撸管的情况下长出的身高就像豆腐渣工程一样,感觉骨骼很脆,容易发生骨折,虽然长到理想身高了,但是骨质不佳。比如两幢同样高度的楼房,一幢是危楼,用的是劣质建材,而另外一幢用的是上好的钢筋水泥,经得起强震的考验。影响身高有很多因素,和遗传基因、营养、运动、睡眠等都有关系,撸管则是一个潜在的因素,最终的身高是看综合得分的,其他几项得分高还是可以长到理想身高的,如果能戒掉撸管,那就能长得更高、更好、更有气势。这位戒友四十岁了,之前一直在邪淫,近三年也学习了陈大惠和彭鑫的视频,但是他不懂得专业戒色,光知道危害还不行,要学会降伏自己的心魔,否则只能一次次失败,往往失败后会变本加厉地邪淫。\textit{《千金方》} 指出糖尿病由于“\textit{盛壮之时,不自慎惜,快情纵欲,极意房中,稍至年长,肾气虚竭……此皆由房事不节之所致也}”,\textit{房室过度,致令肾气虚耗,下焦生热,热则肾燥,肾燥则渴,(《外台秘要·消渴消中》)} 说明房事过度与糖尿病的发生有很大的关系。导致糖尿病有好几种因素,其中一种就是纵欲过度导致的,从年轻时一直纵欲到四十岁,身体越来越衰败,很多严重的疾病都可能得上。没得病时根本不觉得,甚至不相信前辈的告诫,认为前辈在夸大其词、耸人听闻,但是当自己真正得上了,才知道前辈所言非虚,伤精到一定程度就会跨过那个临界点,到时就惨了。很多邪淫者都有一个得意忘形的阶段,在那个阶段症状还没爆发,他会表现得盲目自信和不听劝告,以为自己不会有事,以为那些劝告都是唬人的,最后症状狠狠地教训了他,跑了几次医院后他开始认清事实了。很多撸者真的得上了很严重的疾病,之前就有不少这样的案例,我们要引起高度的警觉,那些得重病的人之前肯定有一个身体较好的阶段,后来才渐渐垮下去的,肯定有一个从好到坏的过程。在身体尚可时,一定要充分认识邪淫的危害,必须下大决心戒除邪淫,即使在婚后也要戒邪淫,不要看黄手淫了,意淫也要克服。如果从年轻时就开始漏,漏到中年身体就千疮百孔了,到时严重的疾病真的会找上门来,给你晴天霹雳!这位戒友最后的告诫很不错,也是他亲历了糖尿病后所发出的肺腑之言,一个沉迷邪淫的男人不仅害了自己,还会拖累自己的家庭。戒邪淫是一个人终生的修为,必须与那种堕落的生活方式彻底决裂!用你的正能量撑起一片天,做一个堂堂正正、充满正气的好男人!
\end{case}

下面步入正文。

大概半年前有一位戒友建议我讲一讲王阳明的心学与戒色的关系,当时我答应了下来,现在时机已经成熟,就在这季和大家做一个详细的分享。

王守仁(1472 - 1529),汉族,幼名云,字伯安,别号阳明。浙江绍兴府余姚县(今属宁波余姚)人,因曾筑室于会稽山阳明洞,自号阳明子,学者称之为阳明先生,亦称王阳明。明代著名的思想家、文学家、哲学家和军事家,陆王心学之集大成者,精通儒家、道家、佛家。穆宗皇帝曾评价王阳明:“两间正气,一代伟人。”我在学生时代就知道王阳明先生,只是当时年轻无知,并未对心学产生任何兴趣,也没有前辈引导,所以一直没有认真研读过王阳明先生的著作。现在回头想想,在那个年龄段即使想读,也很难读懂,因为对那时的我而言,这类书籍显得太高深了。戒色后智慧有所增长,才开始对圣贤教育产生新的兴趣,才渐渐读懂圣贤的开示,圣贤的文章是百读不厌的,越读越有味道,开卷有益,如服仙丹,对提升个人的思想境界很有帮助。

\begin{quote}\it
    提到中国文化儒释道三家,我今天清楚明了地提出一位与我们本题有关的明朝大儒,王阳明,他的本名叫王守仁,阳明是他的号。这位很有学问的大儒是浙江余姚人,他的思想就是有名的“阳明学说”,影响非常深远。他在明朝的历史上,功业很大,也很了不起。他的学说影响到后来日本的文化革命——明治维新,建立了这一百多年来的新日本,明治维新一开始采用的完全是阳明哲学,这在日本史及国际史上都很有名。明治维新采用了阳明哲学的什么观点呢?“知行合一”,即知即行,即行即知。人的知识跟行为常常配合不起来,知是知道,行却做不到,即知即行是很难的。(南师怀瑾先生)
\end{quote}

南师怀瑾先生这段话说得很好,我们戒色也是要知行合一,不可夸夸其谈,理论要结合实战!知行合一的思想,之前很多戒友写文章时都提到过,戒色是需要学习理论知识的,了解邪淫的危害和戒色的原理,然后更重要的是落实在行动上,知行要配合起来,佛法方面叫“解行并重”,意思是一样的。有的人可以说到,但是做不到,这样只会流于空谈,必须要好好下工夫去做,这样才能渐入佳境,之前有的戒友对断念的理论懂得很深,说出来头头是道,我也比较认可,但是等到断念实战时,他的表现却和菜鸟一样,这就是说到做不到,缺少精深的练习,就像一个人把某个工具的原理和结构解释得一清二楚,但还需要不断地练习以达到纯熟运用的境地,只有勤加练习才能让自己对理论的认识达到一个更深的程度,之前的理解好比刀划,勤加练习之后的认识好比刀刻,达到了入木三分的程度。

我之前看过纯印老人的传记,里面提到了王阳明先生的语录:“\textit{破山中贼易,破心中贼难。}”顿时对这句话很感兴趣,认为这句话说得太到位了。王阳明先生有着传奇的经历,曾经剿灭过山贼,所以他写出这么一句话是很有生活体会的,山中的贼寇容易剿灭,但是心中的心魔贼就难以对付了,很多人都没意识到有个贼在心里,凡夫的表现就是认贼作子,王阳明先生在《传习录》中也专门提到了“认贼作子”。心魔贼是非常狡猾的,一次次把你拖入手淫怪圈,要戒除手淫恶习那就必须降伏自己的心魔贼!之所以破心中贼难,一方面是因为自己的觉悟不行,对心魔没有全面深入的认识,另外就是断念水平不行,这样在实战中自然就会败下阵来。你要成功破贼,那就必须通过学习提高觉悟,强化自己的断念实战!当初看到王阳明先生的这句语录,就深知先生有着很多实战的经验和体会,才会有这么一句振聋发聩的语录。也就是从这句语录开始,我对王阳明产生了认可和兴趣,后来也买了王阳明的书籍。

王阳明于明武宗正德元年(1506 年),因反对宦官刘瑾,被廷杖四十,谪贬至贵州龙场(贵阳西北七十里,修文县治)当驿丞。他来到中国西南山区,龙场万山丛薄,苗、僚杂居,使他对《大学》的中心思想有了新的领悟。龙场悟道是王阳明人生最关键的时期,他先立为圣之志,继而经过艰苦的探索,终在龙场悟道,最后弘道,将心学弘传天下,其立德立功立言,彪炳青史。谪官龙场,是王阳明坎坷人生的一次劫难,也是促成他悟入圣人之道的大事因缘。在贵州的三个年头里,王阳明遍历种种苦难,受到无尽的折磨,却在贵州悟道成道,创立了自己独特的心学体系。接着他以自己的体悟印证于五经,无不契合。之后便提出了他的“知行合一”之说,并奠定“致良知”的理论基础,后来进而形成完整而系统的心学理论体系。阳明龙场悟道后,心中充满光明,万缘都已放下:千江有水千江月,万里无云万里天。朝已悟道,夕死可矣,龙场悟道是一大事因缘,非常传奇,正是在人生最困难的阶段,王阳明真正领悟了宇宙人生的大道。在王阳明先生的一生中,我个人比较看重的就是“龙场悟道”,龙场这个名字也极好,悟道之后似乎鲤鱼跳过了龙门,在龙场悟道,在龙场完成彻底的蜕变与升华。龙场悟道,太传奇了!

《传习录》之于王阳明恰如《论语》之于孔子,凡欲深入了解王阳明者不可不读《传习录》。之前有网友说读了王阳明的《传习录》,感觉王阳明得到了禅宗的真传,我的感觉也是如此,很多开示和禅宗如出一辙,只不过用到的名词有所不同,但指代的内容都是一样的。关于王阳明的去世,还有一段极具传奇色彩的传说。据说王阳明 57 岁那年,路过南安青龙镇丫山时,曾到山上的灵岩古寺。这是一座建于南唐时期的古刹。他走入寺中,看到一间屋子大门紧闭。古刹的住持见到他后,惊诧万分地告诉他,这是一位祖师的房间,祖师在圆寂前曾吩咐过,除非等到他自己回来,否则不能开门。因王阳明多次在梦中见到过一位高僧,所以他再三请求住持为他开门。住持怕得罪这位朝廷命官,便同意了。进入禅房后,眼前的一幕令他惊呆了,那位已经坐化五十余年的高僧,肉身一直未腐,相貌与自己几乎如出一辙。王阳明见书案上有本落满灰尘的书,掸去灰尘后,书页上赫然写着:“五十七年王守仁,启吾钥,拂吾尘,问君欲识前程事,开门即是闭门人。”看完偈语,王阳明自觉来日不多,匆匆离去,不久便撒手尘寰。

王阳明先生的前世应该就是高僧,这点应该没什么疑问,他能获得如此高深的领悟,和前世高僧的经历是分不开的,因为在前世就已经达到了相当高的造诣,所以在这世才能达到如此高超的境界。研读了王阳明的《传习录》,才发现这就是一本悟道的书籍,就是一本指导念头实战的著作。阳明心学是修心的学问,心学是一切学问的顶峰,就是金字塔最高的那块,佛法也是在讲修心。任何学问都无法和修心的学问相提并论,这是最根本、最高深、最有价值、最值得学习的学问。不管你学习了什么学问,如果错过了心学,无疑都是遗憾的。真正的战斗是心战,内心的强大才是真正的强大,内在的力量才是真正的力量,一个人一定要学会主宰自己的心!

毛泽东很早便通读过《王阳明全集》,并逐字逐句地批注。早在湖南第一师范念书时,毛泽东就以一篇《心之力》的作文脱颖而出,受到杨昌济的瞩目。作为毛泽东的授业恩师,杨昌济是王阳明的忠实信徒,在他的指引下,毛泽东对阳明心学进行了深入的研究。随着《蒋介石日记》的重见天日,历史学家发现蒋介石对王阳明的崇拜更是到了无以复加的地步。且不说台湾的阳明山、阳明大学都是由蒋公命名,即便“中正纪念堂”门前牌匾上的四个大字“大中至正”,也语出王阳明《传习录》之序言。据蒋介石自述:“我早年留学日本的时候,不论在火车上、电车上或渡轮上,凡是旅行的时候,总看到许多日本人都在阅读王阳明的《传习录》,许多人读了之后,就闭目静坐,似乎是在聚精会神、思索精义。”于是蒋介石跑到书店,抱回一大堆王阳明的著作,“不断阅读研究,到了后来,对于这个哲学,真是一种手之舞之,足之蹈之,心领神驰的仰慕……”1914 年至 1915 年,蒋介石研读王阳明、曾国藩、胡林翼的著作,自称“研究至再,颇有心得。甚至梦寐之间亦不忘此三集。”许多名人都对王阳明评价甚高,走进王阳明的精神世界,学习心学的理论,对于自己的人生很有好处。

下面我就《传习录》的十条笔记谈一下粗浅的认识和体会,是结合戒色来谈的,希望给大家带来一些思考与领悟。

\begin{quote}\it
    能克己,方能成己。
\end{quote}

\textbf{解析} \textit{子曰:“克己复礼为仁。”} 这里的“克”字,在古代汉语中有“克制”的意思,也有“战胜”的意思。克己就是一个人能够克制自己,战胜自己。\textit{景行维贤,克念作圣。(《千字文》)} “景行”是指崇高光明的德行,这八个字的意思就是要仰慕圣贤的德行,要战胜自己的妄念,努力仿效圣人。王阳明先生在《传习录》中提出了“克己功夫”,能克己,方能成己。克己就是克念,必须战胜自己的妄念,妄念是指不切实际或不正当的念头,不正当的念头就是邪念,一切负面的念头。克念其实就是在修心,我们戒色必须要战胜邪念,心魔的表现就是邪念袭脑,先生说过“破心中贼难”,心魔贼的确很强大,是 BOSS 级别的老怪!一个人必须要修炼克己的功夫,断除邪念就是走向光明,跟随邪念就是步入黑暗!圣贤提倡克己,因为圣贤知道每个人心中都有一个贼,如果让这个贼做主,后果不堪设想。很多戒友本来戒得好好的,但是突然某一天心魔贼跳出来了,把他重新拉入了怪圈。只有克己才能主宰自己,成就自己,必须战胜心魔贼!

\begin{quote}\it
    先生曰:“人若知这良知诀窍,随他多少邪思枉念,这里一觉,都自消融。真个是灵丹一粒,点铁成金。”
\end{quote}

\textbf{解析} 这一觉就是最快的刀!这一觉就是安身立命的功夫!念起即觉,觉之即无!这就是最高的断念实战功夫,只一觉,念头就消融了。这个断念功夫是需要不断练习的,这样才能日臻化境,不管何种功夫,都需要持之以恒地练习,最后才能达到出神入化的境界。看王阳明先生的书籍,良知、本体、天理这三个词出现得比较多,在心学体系中,这三个词都是同一意思,先生在《传习录》里说“良知,心之本体”,在《静心录》里说:“\textit{心之本体即是天理。}”良知和天理都有表层的意思,但在先生的开示里都指向了本体,什么是本体?本体就是纯粹的觉知,就是佛性、空性、本心、妙明真心、自性、本来面目。有无数个词都指向了本体,本体是圣贤开示永远的核心。你真正悟明白了本体,就一通百通了,看其他大德的开示一下就能抓住核心,否则很容易看得云里雾里的,我最早是从元音老人这里悟明白了本体,也从元音老人这里学到了断念的实战功夫,我无比感恩和永远顶礼根本上师元音老人。师傅领进门,修行靠个人,师傅可以告诉你方法,但关键还是要自己去做,断念一定要勤加练习。不管什么邪念入侵,只要你的断念够狠,都能将之瞬间消灭,反之,断念水平不行,肯定会被心魔再次攻破。

\begin{quote}\it
    问:“‘思无邪’一言,如何便盖得三百篇之义?”先生曰:“岂特三百篇?六经只此一言,便可该贯,以至穷古今天下圣贤的话。‘思无邪’一言,也可该贯。此外便有何说?此是一了百当的功夫。”
\end{quote}

\textbf{解析} 王阳明先生的这段开示说得异常明确,圣贤的所有开示都是为了让你学会战胜自己的邪念,让你回到思无邪的状态。曾经我们都是纯净纯善的纯真赤子,后来才慢慢被染污的,在思无邪的状态,我们是那么开心快乐,仿佛整个世界就像一个奇迹一样,那时的心地是多么干净,这种纯净的心灵自然就有真乐!我们来到这个世界上,就是从干净被染污的过程,小时候内心特别纯净,后来就越来越肮脏了,各种邪念缠绕。圣贤教育就是为了帮助我们回到干净纯真的心灵状态,能够读到圣贤教育,真的很有福报,但很多人并没有理解圣贤开示的真义,这是一种错过,也是一种莫大的遗憾。进入了快感的世界后,人很容易忘记曾经纯净的大快乐,转而一味追逐快感的体验。对快感的贪求最终会引爆无量的痛苦,很多人看不到这一点,到最后他们会发现越放纵越不快乐,体验的只是短暂的快感而不是真正的大快乐,在快感的世界里彻底迷失后,最终的下场真的很可悲。

\begin{quote}\it
    若违了天理,便与禽兽无异,便偷生在世上百千年,也不过做了千百年的禽兽!
\end{quote}

\textbf{解析} 先生这段话比较严厉,很发人深省。一个人如果过着邪淫放纵的生活,那简直连禽兽都不如,很多撸者都形容自己为行尸走肉,心里有很多邪念,甚至是变态的邪念,整个人充满着邪气,撸管前纯真无邪的眼神看不到了,看到的只是禽兽般邪恶的眼神。大脑被邪念所占据,过着隐秘的堕落生活,这种生活状态非常糟糕,在那种状态下很难体会到生活的美好,一个邪淫的人往往是脾气暴躁的,很容易给别人带去负能量。\textit{不为圣贤,便为禽兽;不问收获,但问耕耘。(曾国藩)} \textit{若不以学圣贤为事,则是行肉走尸。唯知饮食男女之乐,则与禽兽何异?(印光大师)} 一个人来到这个世界上,应该要好好学习圣贤教育,努力提升自己的精神境界,如果让自己沉迷于邪淫放纵的生活,那将来的前途就相当黑暗了。人有五常和八德,五常,即仁、义、礼、智、信,是用以调整、规范人伦关系的行为准则。八德,即孝、悌、忠、信、礼、义、廉、耻。万恶淫为首,邪淫会导致伤身败德,最后八德全废,进入禽兽不如的低劣状态。先生云:“\textit{天地虽大,但有一念向善,心存良知,虽凡夫俗子,皆可为圣贤。}”人生就是一场修行,要懂得遵循天理而生活,要断除邪念,好好培养自己的正念。

\begin{quote}\it
    谦者众善之基,傲者众恶之魁。
\end{quote}

\textbf{解析} 先生云:“\textit{人生大病,只是一傲字。……谦虚其心,宏大其量。}”谦虚是一种很好的德行,易经谦卦六爻皆吉。我们戒色后一定要学会谦虚谨慎,不管戒多久,都不可骄傲自满,有的人戒了几百天就觉得自己了不起,看不起别人,傲慢心一起,接下去就很容易破戒。戒色吧有不少青少年戒友,年纪尚小,取得一点成绩就很容易骄傲,这方面自己一定要学会克服。戒的时间长了,应该觉得自己还是很浅薄,要不断地修谦卑心和惭愧心,这样才能越戒越稳定。戒色后德行一定要跟上,很多人都是败在德行上,德行不够完善,那就很容易被心魔打败。观察一个人德行,就知道他能戒多久,有的戒友戒到一定程度,他自己也感觉到自己的德行不够,有骄傲自满的念头,其他负面的念头也很多,因为学习了戒色文章,所以他知道不能起这类念头,于是就坚决断除了。谦者众善之基,必须夯实这个基础,发誓永远不起傲慢的念头,有时念头会自动冒出,必须立刻断除。一个谦虚的人很容易获得好人缘,一个傲慢的人就像一堆臭粪一样,让人避而远之。秦东魁老师的一个视频的名字就叫:谦乃保身第一法。懂得谦虚,不管对戒色还是对为人处世都极为重要,大家一定要好好修谦德。

\begin{quote}\it
    常如猫之捕鼠,一眼看着,一耳听着。才有一念萌动,即与克去。斩钉截铁,不可姑容,与他方便。不可窝藏,不可放他出路,方是真实用功。方能扫除廓清,到得无私可克,自有端拱时在。
\end{quote}

\textbf{解析} 先生这段开示是讲念头实战的,我一直对念头实战的内容格外看重,因为道理可以说得很对,但最终的检验还是断念实战!念头入侵时,就看你怎么办?断不掉,就被邪念附体,身不由己而破戒。禅宗方面讲到断念,也提到了如猫捕鼠,要有猫那样的警觉性,一定要做到念起即断!先生在《传习录》里还讲到:“\textit{防于未萌之先而克于方萌之际。}”断念贵早,念头方萌之时就要立刻断掉!必须够狠够快!千万不可让邪念发展壮大,不能让其得势!先生用到了“斩钉截铁”这四个字,很给力!不可姑且纵容,必须斩立决!很多新人来到戒色吧,一开始根本不知道怎么戒色,比较盲目,脑子里还有许多思想误区,只有通过不断学习戒色文章才能渐渐入门,戒色文章不一定要看得很多,但一定要吸收得特别深!一定要彻底领悟前辈的思想精髓,并且付诸于持之以恒的练习,先生曾说过:“\textit{汝辈若不肯用功,连笋也不曾抽得,何处去论枝节?}”这句话一针见血,一定要奋发冲天之志,勇猛练习断念!把断 YY 口诀真正练到家,把口诀的含义真正吃透,并且在一次次断念实战中反复体会,不断提高。一定要精于实战,不能流于空谈,必须让实战意识入骨入髓,这样才能立于不败之地。

\begin{quote}\it
    先生曰:“此须识我立言宗旨。今人学问,只因知行分作两件,故有一念发动,虽是不善,然却未曾行,便不去禁止。我今说个知行合一,正要人晓得一念发动处,便即是行了。发动处有不善,就将这不善的念克倒了,须要彻根彻底不使那一念不善潜伏在胸中。此是我立言宗旨。”
\end{quote}

\textbf{解析} 这条也是讲实战,知行合一的关键就是从念头上修,否则知道得再多,把圣贤开示倒背如流,但是却不懂得修心,这样就是“行”的缺失,实战的懦弱和不给力!念头一发动,你就要行动了,把这个邪念给断掉,用先生的话叫“克倒”!你必须有实战的执行力,不能仅停留在理论的研究上,理论研究必须结合断念实战,这样才能越戒越好。先生云:“\textit{善念发而知之,而充之。恶念发而知之,而遏之。……克己须要扫除廓清,一毫不存,方是。有一毫在,则众恶相引而来。}”这些开示都是为了告诉学人一定要注重断念实战!知行合一,知要深知,行要狠行!前段时间一位戒友说:“戒色必须狠!”我觉得他说得很对,如果你不够狠,你就会沦为念头实战的炮灰!被心魔反复蹂躏,生活在暗无天日的邪淫粪坑!念头实战很残酷,你必须强过心魔!先生在《静心录》里讲到:“\textit{杀人须就咽喉上着刀,吾人为学当从心髓入微处用力,自然笃实光辉,虽私欲之萌,真是红炉点雪,天下之大本立矣。}”断念就是杀念,也需要掌握一定的技巧,杀人不算勇者,真正的勇者敢于杀念!杀自己的邪念!杀心魔贼!杀出一条血路!杀出戒色的黎明!\textit{然所忌者,无勇猛力,不能把断咽喉,不觉相续,则流而不返也。(憨山大师)} 断念就是要斩断念头的相续,你不觉察,反而跟着念头跑,这样就是认贼作子!敌我不分!阳明先生说过:“\textit{用功久,自有勇。}”平时要痛下决心练习断念,练之既久,自有斩钉截铁之勇力。自己也要不断学习戒色文章提高综合觉悟,关于断念的理论更是要反复学习和研究,深学深悟,融会贯通。

\begin{quote}\it
    先生曰:“只念念要存天理,即是立志。能不忘乎此,久则自然心中凝聚,犹道家所谓‘结圣胎’也。此天理之念常存,驯至于美大圣神,亦只从此一念存养扩充去耳。”
\end{quote}

\textbf{解析} “美大圣神”这四个字我很喜欢,《传习录》里还有四个字我也很喜欢,那就是“纯然洁白”,每个人一开始都是纯白的灵魂,是后来才慢慢染污的。先生云:“\textit{吾辈用功,只求日减,不求日增。减得一分人欲,便是复得一分天理,何等轻快脱洒,何等简易!……圣人述《六经》,只是要正人心,只是要存天理去人欲。}”为什么要去人欲,因为人欲会遮蔽天理,沉迷人欲会趋于禽兽之列。圣贤并不反对人伦,夫妻属于人伦,正淫要节制,邪淫必须要戒除。去人欲,并不是不结婚生子、不传宗接代,而是去除自己的邪念!去人欲可以用思无邪来解释,关键还是心地要恢复干净。\textit{为学日益,为道日损。(《道德经》)} 很多戒友在戒色前,每天都有很多邪念,他们的头脑简直就是邪念的根据地,各种回忆,各种幻想,看到女性就意淫,在戒色后学会修心了,不断对治自己的邪念,这样就是“日损”,每天都在减少邪念,最后就会发现心地恢复干净了,那种感觉太美好、太快乐,是真正的大爽!先生云:“\textit{圣人之所以为圣,只是其心纯乎天理而无人欲之杂。犹精金之所以为精,但以其成色足而无铜铅之杂也。}”正位凝命,如鼎之镇,去除邪念才能正气爆棚!战胜心魔是最伟大的壮举!攀登上珠峰也不及战胜自己的心魔,登上月球也不及战胜自己的心魔!打破吉尼斯世界纪录也不及战胜自己的心魔!一位戒友说:“脑子特清醒,心理特别祥和,甚至感到幸福的感觉。真的,好像恢复纯净了,脑子一下就充实了的感觉,又好像回到过去的快乐回忆中,反正这种感觉就是特别爽啊!我感觉不到痛苦了。”他的心灵恢复纯净了,那种感觉真的特别爽!手淫的快感根本无法与之相比,手淫就像沉溺在粪坑里以吃屎为乐,而心灵恢复纯净后,就像自由翱翔在天际,那种纯粹美好的感觉给人强烈的喜悦。戒色宛如奇迹,当你进入美大圣神的心灵状态时,你自然会感受到真乐,你知道这才是自己真正想要的感觉。

\begin{quote}\it
    先生曰:“汝若于货、色、名、利等心,一切皆如不做劫盗之心一般,都消灭了,光光只是心之本体,看有甚闲思虑?此便是‘寂然不动’,便是‘未发之中’,便是‘廓然大公’。自然‘感而遂通’,自然‘发而中节’,自然‘物来顺应’。”
\end{quote}

\textbf{解析} 先生这段话就是在直指本体了,本体这个词我比较喜欢,《传习录》里面讲到:“\textit{未发之中即良知也。}”又讲到:“\textit{天理即是良知。……本来面目即吾圣门所谓良知。……道即是良知。}”未发之中即为念头未起之时,也就是离念的灵知,一念不生、了了分明的状态。认识本体太关键了,本体是无念而清醒状态,本体是纯粹的觉知,本体就是你真正的自己,是你的真我。念头不是你,身体不是你,念头是工具,身体是载体。《雪狮的蓝绿色鬃毛》里第三十个故事“诺西的实相启蒙”,这个故事我特别喜欢,简要地分享给大家:诺西隆多是华智仁波切的大弟子。他在野外修行时,与上师学习“大圆满”的理论和修法达二十五年之久。有一次,华智仁波切与几个弟子住在一个野外隐蔽叫做那冲的地方。他习惯每天黄昏时,仰卧着,向上凝视,修大圆满的“凝视天空瑜伽”。那是一个非常殊胜的禅修法门,是要让一个人的心与无尽的虚空合一。有一天,华智又在做这样的禅修时,唤来了就在附近的诺西隆多,华智问他是否还未了悟自心本性,弟子据实回答:“还未。”然后华智说:“不用担心,事实上,没有什么你不懂的。先别管它!”上师咯咯地笑着,然后两人继续修禅。诺西隆多曾重复做过一个梦:梦中,华智仁波切为他解开了一团巨大如山的黑线,在线团中心现出一尊金质的金刚萨埵佛像。有一晚,华智又把隆多叫来了,要他躺在身边。“现在我们要揭开一切了,”他保证道:“保持清醒!”他们一起向上凝视,望入浩瀚无边、空无一物的虚空。远处有佐钦寺的狗在叫。华智对诺西隆多说:“亲爱的朋友,你听到狗的叫声吗?”“有!”隆多回答他。“那就对了!”上师大叫道。他又问:“你看到天上的星星了吗?”隆多肯定地回答了。华智叫道:“就是这样!那就是本然具足的觉醒的明觉、佛性。不要看别的地方!”就在那时,黄昏当中,隆多超越对立的智慧之眼打开了,隆多喜极而泣!这个故事令我印象非常深刻,因为这是在浩瀚星空下的开示,星空唯美而震撼,望向星空时头脑里往往是无念的,只是纯粹地看,没有任何念头,这其实就是纯粹的觉知,听也是如此,只是纯粹地听到狗叫,头脑里没有任何念头。大道至简,简单到不可思议,简单到难以置信的程度,只有彻底纯真的心灵才能了解和接受真相。认识本体、觉性后,还需要断念保护,光认识本体还不行,因为念头还会入侵,必须做好断念实战。妄念会遮蔽本体,凡夫都是认妄念为自己,跟着妄念跑,而我们要学会做念头的主人,安住于纯粹的觉知。《修炼当下的力量》中说:“只有当头脑静止下来时,你才能认识本体。”还说道:“所谓自由、救赎和开悟,就是知道你自己是思考者之下的本体、心智噪音下的定静,以及痛苦之下的爱和喜悦。”“本体才是真正力量的唯一源头。”埃克哈特·托利用的是“essence”,译者翻译过来就是“本体”这个词。

\begin{quote}\it
    人须在事上磨,方立得住,方能静亦定、动亦定。
\end{quote}

\textbf{解析} 这条开示是讲对境实战的,人必须经得起境界的考验,现在这个时代邪淫泛滥,生活中和网上会有各种诱惑,你必须经得起考验,这样才能立得住!对境不动心是内在力量的表现,是需要不断修心后才能达到的境界。毁誉荣辱、各种色情的诱惑皆不稍动其心,以沉着、冷静的态度泰然处之,高度稳定、处危不动、处变不乱,每临大事有静气,必须经受住严峻的考验。千魔不改,万魔不退!诱惑再大,你这里都心如止水、视而不见。上网时很容易遭遇诱惑图片,定力差的戒友,他的视线就粘上去了,然后就陷进去了,开始意淫和找黄了。戒色高手看到诱惑就像没看到一样,心里平静得很,在一次次对境实战中,他的定力变得越来越强。之前有的戒友说要看黄来磨练定力,这是完全错误的想法,因为看黄会导致暗漏,严格来说就算破戒!一位戒友说:“就在第十天早上的时候,欲望非常重,邪念纷飞,怎么断也断不掉,心魔像疯了一样地攻击我,我的脑袋快炸了,最后挣扎了十几分钟,受不了了,最后妥协了,看黄破戒了,接下来就是一连串的破戒。”一个人破戒可能是遭遇了诱惑,从而起了邪念,也可能是心魔直接入侵,比如一个人独处时,那种邪念就容易冒出来,心魔也会怂恿你去看黄,怂恿你去试,把你拉入怀疑和诽谤。我们必须要学会识破心魔的阴谋诡计,坚定自己的戒色立场,绝不动摇!在实战中感觉自己断念不行,那就要在平时狠下功夫练习断念,断念如断砖,是需要一个练习提高的过程的,看过一位练铁砂掌的师傅,一掌下去五块砖就裂了,人家也是练出来的,每天都在练,那个手掌练得跟熊掌一样,真的下了很大的苦功,最后达到了惊人的地步,感觉他断砖像切豆腐一样容易。戒色高手断念轻松而容易,同样也是练出来的啊!不断练习,功力就会逐步深厚起来。断念就是上阵杀敌,断念就是刺刀见红,戒色最根本的就是要赢得断念实战,在对境时一次次磨练,看看自己是否动心了?不动心是最好的,能够做到念起即断也很不错。要立得住,真的很不容易,你的定力必须足够坚固,像金刚山一般坚固,经得起各种诱惑的考验。阳明先生云:“\textit{时时刻刻须是一棒一条痕,一掴一拳血,方能听吾说话,句句得力。}”必须拿出一股狠劲和拼劲,不能坐以待毙,你必须不顾一切地强大起来,唯有强过心魔才能不破戒,否则就无法摆脱被奴役和蹂躏的悲惨命运。狠狠地学,狠狠地练,狠狠打击心魔!空谈误国,实干兴邦,必须实干、真干、狠干!

\paragraph*{总结}

《传习录》中讲到:“\textit{真知即是未发之中,即是廓然大公,寂然不动之本体,人人之所同具者也,但不能不昏蔽于物欲,故须学以去其昏蔽。}”要真正读懂《传习录》,那就必须对禅宗有比较深入的领悟,因为阳明先生完全是在用儒家的语言来阐述禅宗的道理,很多人看到天理和良知,很可能会理解为表层的意思,而对真正的指向并未明白。阳明心学不仅仅在于理论,更在于念头实战,认识自己的本体,然后就是断念保护,这和禅宗大德的开示如出一辙,圣贤教育是要让我们恢复本体,因为本体被私欲遮蔽了。一个纯净明鉴的心灵和一个肮脏龌龊的心灵两相比照,哪一个更美好可想而知,本来我们都是纯真无邪的小孩,从来不手淫,到了发育期开始染上这个恶习,心灵一下变得肮脏许多。学习圣贤教育可以让我们认识本体,可以让我们重新找回纯净美好的状态,来到这个世界上,应该要珍惜学习圣贤教育的机会。

这季分享了《传习录》的笔记解析,相信大家对阳明心学有了新的认识,阳明先生的一生堪称传奇,我最喜欢的就是龙场悟道那个经历,在最艰苦卓绝的困境中,阳明先生完成了彻底的蜕变,他悟道了。纵观王阳明的一生,作为军事家和政治家,立下不世之功,彪炳史册;作为思想家,开创儒学新天地,成为一代心学宗师。正如梁启超对王阳明的评价:“他在近代学术界中,极其伟大,军事上政治上,亦有很大的勋业。”梁启超还说:“阳明是一位豪杰之士,他的学术像打药针一般令人兴奋。”阳明先生是中国历史上罕见的立德、立言、立功三不朽的伟人,也是明朝最为杰出的政治家、思想家和军事家。他的一生跌宕起伏,充满了传奇色彩,身处各种逆境、困境、险境、绝境而心如止水,从容化解,这是激励和启发现代人最好的范本。他的心学思想融会了儒、释、道三家之精华,是救治当今浮躁社会的一剂清醒良药,读来的确给人很大的启示。哈佛大学教授杜维明先生预言,21 世纪将是王阳明的世纪,可见对其的评价相当之高。

王阳明在江西剿匪屡战屡胜,但是他却感叹“心中贼难破“,心中贼像幽灵一样神出鬼没,对人鬼使神差。最大的贼就隐藏在每个人的内心深处,圣贤认识到了这个贼,而凡夫却在认贼作子,跟着贼跑。心中贼的确难破,毕竟是大 BOSS 级别的,但是只要你的实力够,依然可以战胜心魔贼!对贼作战,你必须知己知彼,了解贼,识破贼,认清贼的所有套路,当贼入侵时,坚决剿灭之!阳明先生不管剿山中贼还是剿内心贼,他全部做到了,这点我最佩服先生,历史上很多人能征善战,军功显著,但是他们却无法打败心中的贼,他们甚至根本就没意识到心中有个贼!阿姜查在《这个世界的真相》里讲到:“去除心中之贼。”“真正的贼不是外在的贼。”圣贤致力于剿灭心中之贼,阿罗汉有一个称谓就叫“杀贼”!心中之贼不剿灭,人就不得安稳,这个贼会把他引入邪途、拖入怪圈,剿灭心中之贼,实在刻不容缓!\textit{恶从心生,反以自贼,如铁生垢,消毁其形。(《佛说孛经》)} 剿杀邪念,剿灭内贼,这是你一生的职责和使命!

我很喜欢阳明先生的书法《铜陵观铁船歌》,通篇字体修长,行笔快捷,骨力内涵,豪放中见沉着,遒劲中见秀丽,有米芾书法“沉着飞翥”的神韵。看先生书法,就能感受到那种豪侠之风,那种剿贼获胜后的英雄气概!铁骨铮铮,正气凛然,给人一种很大的精神享受。戒色的勇士们,让我们义无反顾地踏上剿贼的征程,百万壮士轰然前进,背影是那样的威武而壮烈,征服心魔贼,浩然正气镇乾坤!

下面分享一首戒色诗歌。

\begin{poem}[壮哉杀念者]
    \begin{multicols}{3}
        \centering~\\
        天空下起了鹅毛大雪 \\ 雪中站着一个人 \\ 纹丝不动 \\ 一场残酷的对决 \\ 正在他的脑海中上演 \\ 他闭上双眼 \\ 眉头略微皱起 \\ 他正在感受内心的波动 \\ 雪花落在了他的头顶 \\ 慢慢融化消失 \\ 远处的鸟儿发出几声叫声 \\ 传入了他的耳朵 \\ 冬天特有的静谧 \\ 笼罩着他 \\ 空气中流淌着一股杀意 \\ 浓烈的杀意 \\ 突然邪念上来了 \\ 一记觉察犹如出刀 \\ 快至毫巅 \\ 狂暴的心魔贼 \\ 瞬间被消灭 \\ 他还是站在雪中 \\ 纹丝不动 \\ 手中无刀 \\ 心中无念 \\ 安住于纯粹的觉知
    \end{multicols}
\end{poem}

下面推荐一本书。

\begin{book}[《幕后:一位觉者的实修日记》,马克·李维特]
    作者希望籍由此书能说明:开悟体验并不神秘,并不专属于某个其他的时间和地域的神话传说,而是一种视角的切实转化。这本书 162 页,可以说是一本比较薄的书,我对这本书的评价是“书薄料猛”!的确让人脑洞大开,非常给力非常震撼的一本书,这本书让我对真相有了进一步的了解和认识。这本书的可贵之处在于里面的经历是作者自己的亲证而非仅仅停留在理论层面,亲证的力量是不同凡响的,这本书的确让人大开眼界,也激起了我对修行更热切的兴趣。作者说禅宗的“骑驴找驴”是一种误导,因为作者采用的是更精确的表述,的确不是骑驴找驴,而是你就是纯粹的觉知,你找的东西正是你自己!禅宗的骑牛觅牛或者骑驴找驴,也不能说是一种误导,而是一种不求精确的类比,是针对不知自心是佛而到外面去寻找而言的,佛法讲:“莫向外求!”因为宝藏在里面,要向内看!向内观照,这样才能发现真正的自己。总体而言,这本书非常值得一看,对悟道感兴趣的戒友可以买回来看看,也不是很贵,二十多元,里面的内容极具震撼力,书是挺薄的,但是料却足够猛!
\end{book}
