\subsection{正确处理戒色与生活的关系}

\paragraph*{前言}

上季资深戒友“开始行动吧196”在帖子里的留言:

\begin{quote}\it
    非常感恩飞翔老师这么多年的坚守,戒色吧这么大的贴吧,想来您也始终处在风口浪尖的位置,遭受外界挖苦讽刺太多太多,慢慢地也开始遭受内部的分裂,真的太不容易了。人们常常看见您作为吧主的光环而心生嫉妒,却很少知道这背后需要面对的指责、诋毁,需要自己默默消化的各种负面念头。真的很感动于您一直以来的坚持,感动于您一次次耐心的回答,感恩!
\end{quote}

我坚持到现在,不仅要面对外界的各种诽谤和诋毁,有时也要面对戒色界内部的诽谤和诋毁,坚持到现在的确是很不容易的,已经有资深前辈劝我退出戒色界了,原因就是诽谤和诋毁太多,一般人真的会起退心的。看到我受到这么多的诽谤和诋毁,有的资深前辈都萌生退意了,但我决定尽量坚持下去,做好每一个答疑,尽量帮助更多的戒友,为了大家牺牲我自己也在所不惜,义无反顾!处在风口浪尖的位置肯定会遭到很多的攻击,这是必然的,古语云:“木秀于林,风必摧之;堆出于岸,水必湍之;行高于人,众必非之。”古往今来不知有多少智者、仁人,因其才能出众,技艺超群,招来别人的嫉妒、诽谤、诋毁、诬陷,甚至丢了性命。戒色吧之前就有前辈顶不住诽谤而退隐了,能够顶住诽谤而保持心态平衡,也是挺不容易的。枪打出头鸟,谁站出来成为那个最显著的戒色人物,谁就会遭到最多的攻击,这在以前我就料到了,也做好了心理准备,任何人处在我的位置都会知道高处不胜寒,坚持的不易,因为我发愿帮助更多的人,所以我会尽量坚持下去。

诽谤和诋毁是一种攻击,但从另外一种角度来讲,也是一种考验,也是一种“肥料”,就像大粪可以帮助蔬菜生长一样。大德说过:“受恶骂如饮甘露,遇横逆如获至宝。”不管遭遇什么攻击,自己心态一定要好,毛主席《西江月·井冈山》里的两句诗我很欣赏:“\textit{敌军围困万千重,我自岿然不动。}”面对误解、诽谤和诋毁,应该泰然处之,我也欢迎他们诽谤和诋毁,外界的诽谤和诋毁就不去说了,但我希望戒色界内部能够搞好团结,不要出现内斗的现象,大家应该团结起来,对于不同的戒色方法要互相支持与尊重,不可自赞毁他,不可贬低别人而抬高自己,应该求同存异,以和为贵。不可有名利心、私心和嫉妒心,不要争名、争第一,这样戒色公益事业才能更好地发展下去。一位戒友说:“戒色是积善之事,不要掺杂上功利,更不要说专利了,要众志成城,一起为戒色公益尽力而为!”还有一位戒友说:“看到飞翔的帖子,就想说一句话,最大的幸福就是,很久之后回来看到你还在。我想飞翔这么长时间的坚持,对于戒友们来说就是一种莫大的鼓舞。”另外一位戒友说:“每次看到您回答问题和坚持对于别人来说也是一种动容。”坚持不易,且行且珍惜!以初心坚持,更不易,珍惜每一次答疑的机会,是别人给了你行善的机会,要感恩别人,谦下地服务于众生。

如果拿失败案例和极端案例说事,这个世界上还真没有好的戒色方法了,因为每种戒色方法都有大量失败的案例,即使教材再好,也会有大量失败的案例,即使老师水平再高,也会教出不及格的学生,这是没有办法的事情。也有个别戒友因为存在思想误区而走极端,这是必然会出现的现象,就像有的学佛人也会走火入魔,不是佛法不好,而是自己的认识存在问题。任何戒色方法都可以挑出几十种刺,欲加之罪,何患无辞!这个时代色情的诱惑太猛烈,有的戒友看了很多戒色文章,学佛也很精进,但还是会破戒,一方面是因为习气太重,修心不到位,另外和大环境的诱惑多也有很大的关系,网络上、街上都有各种诱惑,相对古代而言,现代戒色的难度提升了太多太多,古人一辈子接收的色情讯息量也许还抵不上你一天所看的色情讯息,而且现代色情的内容更刺激、更变态、更直观,不像古代只是纸质传播。高科技是把双刃剑,现在已经把 VR 技术和人工智能用于邪淫了,诱惑变得更猛烈更夸张了,将来一普及,后果真的不堪设想,会导致更深的沉沦与堕落。在这个时代戒色是很有难度的,其实在哪个年代戒色都不容易,看看《寿康宝鉴》就知道古人戒色也不容易,只不过现代人要面对更多的诱惑,难度更大,这需要更高的戒色修为和定力。

这季前言把念起不随和斗争再补充说一下,有的大德说,不要斗争、不要抵抗、不要控制,只要看着念头,不要跟随。从不随的角度来讲,是可以这样说。而有的大德则说要时时刻刻与自己的妄念作斗争,所谓:“打得念头死,许汝法身活。”\textit{降伏其心。(《金刚经》)} \textit{降魔者先降自心,心伏则群魔退听。(《菜根谭》)} \textit{起心动念处,念念觉察,念念消灭。(憨山大师)} 如果看过很多大德的开示,一定会发现对待断念实战,有着两种截然不同的说法,有的大德主张不随,有的大德主张斗争,还有的大德既说念起不随,也说斗争和降伏,这让人感到费解和困惑,到底要不要斗争?其实总体而言,还是要斗争的,念起不随、不睬、不理,就像冷战一样,冷战难道不是在斗争吗?冷战本身就是一种斗争形式!有的大德之所以说不要斗争,那是因为怕你压念,走入误区,另外也是从不随这个角度来讲的,以示区分。

上季一位戒友给我反馈,那段反馈是他在 \ref{124} 发表前就写好的,放在 \ref{124} 的帖子回复里他也觉得不大合适,因为我已经澄清了大部分的误解。他就主张不睬念头,念头脱开,念头自灭,反对断掉、斗争、降伏、消灭等词汇表达,其实这两种表达都是可以的。就像围棋有多种流派,李昌镐是继吴清源之后又一绝世人物,其将防守型棋风发展到新高度。作为一个“力战派”棋手,古力所取得的成就令人惊叹,他曾十五次夺得围棋赛的冠军,古力的杀力非常强大,你觉得是防守好,还是力战好?两种风格各有巅峰人物,阿法狗防守与力战都厉害,打败所有人类高手,最好的防守是进攻,最好的进攻是防守,在最高层面,矛盾对立而统一,在一定条件下可以相互转化。

看住念头,念起即觉,觉之即无,或者看念头从哪里生起,这两种都是可以的,看念头从哪里生起,这要求更高的觉悟,念头从空而起,再消融于空。看住念头并没有错,很多大德都说要看住念头,而看念头从哪里生起,也没有错,两种不同的修心方法而已。看问题一定要完整而全面,否则就会像盲人摸象一样,以偏概全。这是我得出的经验,有的人看到不随,就反对斗争,这就会陷入狭隘的偏见。修行方面的认识有很多思想误区,要多看大德开示,对于各种不同的断念方法要有一个总体而深入的了解和认识,否则就会执于某一种方法而反对其他方法,这样就会陷入狭隘和偏激。对于各种不同的断念方法都要尊重,不可厚此薄彼,对于大德的表达词汇也要充分尊重,不要根据自己狭隘的理解来盲目反对。

我们看大德开示一定要圆融地理解,否则就会陷入矛盾和疑惑,我看过很多大德的开示,我发现他们对于修心的表述是可能存在矛盾之处,但目的都是为了不让妄念连续下去,不让妄念发展壮大,看似矛盾的表达方式其实都是为了同一个目的。之所以感觉矛盾,那是因为从不同角度来讲的,所谓:横看成岭侧成峰,远近高低各不同。如果你能圆融地理解,那就没有任何问题,对于相反的表达也能圆融地接受,不会盲目反对。

下面分享一些案例。

\begin{case}
    我相信飞翔大哥的话,因为我就是靠看《戒为良药》戒色的,亲身体会最具说服力。目前已超过三年多时间,期间没有破过一次戒,但至今我仍然不敢有丝毫松懈,因为《戒为良药》万般告诫,一旦放松警惕,那么离破戒就不远了。根据我自身戒色的经验,戒色的过程中一定得学习优秀的传统文化(陈大惠老师的视频很不错),这对戒色会有事半功倍的效果,这在《戒为良药》中也有讲到。

    \textbf{附评} 戒色吧有一批戒色两年、三年以上的资深戒友,他们是戒色吧这个学校的尖子生。教材是一样的,但总有优等生和差生之分,这是没办法的,因为每个人的悟性、勤奋度、吸收率、心态、决心等是存在差异的。我首先看重的是决心,戒色一定要下大决心,最好是破釜沉舟的决心,决心首先要狠!有什么样的决心就有什么样的行动!决心不够,那种热火朝天的干劲就燃不起来,其次我也很注重悟性,悟性实在是太重要了,悟性上乘的戒友学习戒色文章很快就能契入,而悟性较差的戒友可能要戒戒破破几年后才能有所契入,差距真的就有那么大。要戒色成功一定要系统学习戒色文章,注重积累,多做笔记,多复习笔记,不断提升吸收率,好的戒色文章每看一遍都有新的收获,有时真正悟懂一句话,觉悟就飞升了。戒色更要注重实战,真正懂得修心,强化观心断念,每天都要保持警惕。戒色刚开始可以靠热情和新鲜感,但要戒得长久,就要学会培养良好的学习习惯,持之以恒地坚持下去,真正融入自己的生活,以戒色为乐,真正热爱戒色,能做到这点是很不容易的。这位戒友戒得很不错,戒了三年多,一次未破,很棒的战绩,记得我戒到三年多时,自感已经相当稳定了,但心魔还时不时会来骚扰,所以还是要保持警惕和戒备,不可骄傲自满和疏忽大意。我戒到现在还经常写实战心得,还在不断总结实战的经验教训,虽然心魔现在很难攻破我,但我觉得我还有很大的提升空间,我也发现修心既简单又深奥,把念头断掉就是在修心,但是在实战中会出现各种情况,对每种情况的识别、了解和应对真的很考验一个人的觉悟和断力,要不断总结、领悟和修炼才能达到高层次。真正达到高层次的人也必定具备良好的德行,而德行的提升和学习优秀的传统文化是密不可分的,戒色一定要注重修德,多学习圣贤教育是很有必要的。为武之道,以德为本,习武首先要重视武德的学习和培养,戒色也是如此,有了良好的德行,自然会越戒越好。
\end{case}

\begin{case}
    请教一下吧友们,遗精后状态下滑很快,心态也变得很差,现在是高三,特别是写题的时候,不愿意去思考,有种力不从心的感觉,每次遗精恢复得好几天才会好一些,我想知道怎么快速调整情绪、心态。

    \textbf{附评} 遗精也属于能量的耗损,如果身体比较好,偶尔一次还扛得住,但如果之前手淫很频繁,身体损伤严重,这时一次遗精就会感觉身体很难受,脑力也会下滑,人也会变懒变迟钝。不管对于学生党还是工作族,脑力和精力都是极其关键的。脑力差,会很影响学习和工作状态,精力差,人就会变懒,不愿意动,肾虚就有嗜卧懒动的症状表现。戒到一定时候肯定会出现遗精的问题,刚开始有的戒友还说遗精怎么还没出现,其实是之前一直撸的缘故,继续戒下去,到时候就会出现遗精,这时候一定要学会控遗!如果出现频遗,那对身体的伤害是非常大的,前段时间一位戒友说自己一月频遗九次,即使是铁汉也难以招架这个遗精频率。一篇文章讲到频遗的危害:“影响精神状态容易精神萎靡不振、情绪不稳定、记忆力下降、遇事无精打采、对外界显得十分淡漠、多梦失眠等现象。那股青少年应有的血气方刚、朝气蓬勃、努力奋进和天天向上的气概,在他们身上一扫而光,必然会影响到学习与工作,于是上课提不起精神、学习成绩一落千丈、工作效果每况愈下。”一位戒友说:“两天遗两次,感觉整个人都暗了。”频繁遗精让整个人跟废了似的,漏精就像汽车漏油一样,加满油的跑车和加满精的身体,那是怎样的气概和冲劲?如果频繁漏精,身体是会垮下去的,有的人因为频遗也得了一身的病。一月遗精三次以上就要引起高度重视了,自己要学会避免各种导致遗精的因素,上次一位戒友就是因为过于劳累所致,自己要懂得分析原因,学会规避导致遗精的因素,平时要加强控遗意识和养生意识。遗精后要好好休养,调整好心态,遗精后邪念也容易变得活跃,这时候要提高警惕,加强修心。一般遗精后的几天身体会有所不适,毕竟能量漏失了,这时候我建议适量做做有氧运动,可以散步、慢跑、快走等,另外可以试试养生桩和静坐,这样可以加快身体恢复,睡眠质量一定要保证,良好的睡眠可以很好地促进身体恢复,饮食方面的营养也要跟上,情绪和心态方面也要注意调整,不要有太重的心理负担。高三对脑力和精力要求极高,希望高三的戒友能够做好控遗,调整好状态,好好备战高考,也祝他们高考成功!
\end{case}

\begin{case}
    从 2017 年 8 月 18 号我迎来了命运的转折!那一天我遇到了戒色吧!看到了《戒为良药》这本书。从此以后,由于每天坚持的学习,成功戒到了两百天!戒了这么久,最明显的就是我比以前爱笑了,敢直视别人眼睛,女生也主动和我说话,甚至有女生想追我,社交恐惧好了不少,但只要熬夜就又会复发,所以,戒色养生要两个一起抓!戒色心得:我也不和其他人说那么多,其实大道至简,戒色开始难,后来越来越容易。

    \begin{itemize}
        \item 坚持每天看戒色文章,落实到生活中去。
        \item 外避内断!在外诱惑立马避开,内在念头立马断掉!
    \end{itemize}

    \textbf{附评} 这位戒友总结的“外避内断”非常之好!是高度总结和概括的四个字,内不随念转,外不为境迁,避色如避箭,防淫如防火,念头一起,就要立刻断掉,一定要快!如果不断掉,就会被邪念附体和操控,就像一个木马病毒被成功植入一样,到时候就会身不由己,进入疯狂找黄、疯狂看黄、疯狂破戒的状态。现在这个色情泛滥的时代,只要上网基本都会看到这类内容,特别是擦边的内容更是满天飞,各大平台对你的眼球狂轰滥炸,一次次吊起你的好奇心,吸引你去点击。心魔也会突然偷袭,一次次将你残暴 KO,而我们必须强大起来,彻底摆脱被动挨打的局面,强者主宰内心,强者 KO 心魔!如果你的实力不行,肯定会再次破戒,再次陷入那个怪圈。一开始你也许不具备跑完整个马拉松的实力,但你可以慢慢培养和训练,让自己最终具备这个实力。修心功夫也是如此,一开始很多人的实战表现稀烂,是绝对的菜鸟,后来他们强大了,实战表现越来越强,这时轮到心魔颤抖了……这位戒友戒了两百天,说自己爱笑了,敢直视了,社恐也好了不少,这都是恢复的征兆,手淫对心理的摧残不亚于对身体的摧残,当心理恢复后,就有自信和底气了,人也变得开朗乐观,你有那股强大的正气和底气与任何人都可以直视,这时候你会发现很多人变得不敢看你了,以前是你不敢看他们眼睛,现在轮到他们不敢看你的眼睛了,一种气壮山河的气势油然而生,不断壮大的内气越发浑厚和磅礴,持续提升自己的灵性和正能量,让你的直视变得极具力量感。戒色后是该注意养生,不要熬夜,熬夜也很伤身体,这位戒友也提到了熬夜就会复发,所以要尽量避免熬夜,加强养生。能够戒两百天也是很不容易,这位戒友的悟性很不错,希望他好好戒下去。
\end{case}

\begin{case}
    飞翔老师,刚才鬼使神差地就破戒了,起床后欲望特别大,然后脑中冒出了找黄的想法,于是我打开了手机疯狂找黄,找着找着就渐渐没有了欲望,但是找黄的想法一直在脑中,我当时不知怎么想的,原本忍住就过去了,结果我仍然继续找,找到了黄,然后我当时还在犹豫到底要不要撸,结果看到了画面,实在受不了,就撸了,我也是醉了,我明知道撸过会后悔,但是……戒了一年了,第一次破戒。

    \textbf{附评} 戒色一年就因为一次断念实战没做好,就又回到了“过去”。当脑中冒出了找黄的想法,为何不立刻断掉?如何要听信那个念头?为何没有警觉?实战的教训很深刻啊!那个念头一旦安装成功,就会执行一连串的行为,即使你的欲望在疯狂找黄和看黄的过程中渐渐消退了,但那个“撸出来”的执行命令其实还在继续,不撸出来是不会罢休的!我们一定要时刻警惕,刚醒来时往往警惕性较差,这时候心魔很容易得逞,我们要在那个醒来的时段格外警惕。心魔很会挑选进攻的时机,当你有点迷糊时,它就开始进攻了,如果你当时听信了,就会重新掉入怪圈。我到现在看过无数的破戒案例,自己也有十几年的亲身体会,我深知断念实战的重要性,怎么强调都不过分,在邪念上来的那一刻,你没警惕,你没断掉,你听信了,你跟从了,结果就是一发不可收拾,一连串的破戒,就这么残酷,就因为那个念头没断掉,结果就沦为了撸管肉机!首先必须通过学习来提升戒色觉悟,这样当念头入侵时才能精确地识别和判断,不会认贼作子,然后要通过练习来不断提升断力。有的戒友能识别,但缺少断力,当他断力上去了,就能在实战中战胜心魔。两块砖头叠在一起,你一掌下去能劈开吗?估计没经过训练,很难,我见过高手一掌下去,五块都裂开了,人家天天在练习。王桐晶荣获挑战不可能第三季年度挑战王,那个节目我看了感到不可思议,你拿着计算器也赶不上那个速度,我一直在想王桐晶是怎么做到的,后来看了一篇文章,里面讲到她加入解放军珠心算队,在这个被称为珠心算“梦之队”的地方,人才济济,初来的她成为了全队的吊车尾。这对王桐晶是一个莫大的打击,从顶端又跌回底端,何况,军队的管理是相当严苛,对最后一名的惩罚无疑也是最重的。别人练,她也练,别人休息了,她还被罚着练。为了再次证明自己,王桐晶几乎魔障了,“那时候什么想法都没了,就是算、算、算”。即便后来摆脱了惩罚的噩梦,她还是一直坚持练到教室只剩她一个人。深夜里,偌大的房间,只听见“嗒嗒嗒”的算珠拨动之音。“别人一遍,我就练三遍”,这是王桐晶为自己立下的要求。王桐晶回忆道:“以前训练时,一天一支笔都是常事,练得连手指都有些变形,也几度想过放弃,但想起教练教导时的话语、家人临别时的目光,看着身旁同学飞快地运笔,想起过去,不由就又多了几分前进的动力。而每次看到自己的进步,看到收获,也还是挺开心的。”王桐晶的练习过程很值得我们学习,我们不一定要达到那种练习的强度,但至少每天要投入一定的时间,并且持之以恒,这样断念水平才能逐步提升上去。
\end{case}

\begin{case}
    飞翔哥,我永远支持你!不久前我终于真正能够断念了,达到这一步,我花了三年,从高一到如今高三。以前的我看文章不太认真,有点走马观花,一直把压念当做断念,练习了一年也没有什么效果,破罐子破摔。不久前终于明白你真正的意思了,邪念上来终于瞬间就灭了,感恩。以后我的一生一定会做一个像你一样正直的人。谢谢你,飞翔哥。

    \textbf{附评} 懂得观心断念,戒色才算真正入门,这个入门标准是金标准,如果仅仅是学会行善积德,还不行,一定要懂得对治邪念,懂得修心。行善更要去恶,特别是意恶,意恶就是各种邪念,一定要断掉。这位高三的戒友以前看文章不太认真,如果一开始就能认真对待和正确理解,也许就不需要三年了,我见过悟性好的戒友三个月左右就入门了,很快就能掌握观心断念。观心断念其实并不是很难,只是我们没习惯而已,我们总是盯着外面的事物看,很少真正观察过自己的念头。我看过很多大德的开示,几乎都提到了一点,那就是必须要学会观察自己的念头,必须要学会观心,观心就是观念头。比如你盯着一根蜡烛看,看十秒钟没分心,很专注,然后试着把这股观察力转向内在,观察自己的念头,刚开始很生疏,但只要坚持观察念头,就会发现随着觉察力的提升,在你观看到念头的一刹那,念头就消失了,这是最最奇妙的事情。断念不是压念,断念是觉而化之!这点一定要认清,一开始没正确理解和领会,结果就会导致练习没效果,指导思想一定要正确,否则很可能是无用功。能真正学会断念是很快乐的事情,这种快乐来自于主宰内心,这种主宰力和统治力的获得会给内心带来很大的自由和快乐,你不再是心魔的傀儡和奴隶,你不再是那个任心魔摆布、任心魔宰割的提线木偶了,你开始做回主人了。
\end{case}

\begin{case}
    奉劝各种还在撸的朋友们快戒掉吧!我真的害怕了,从毕业工作开始,可能是压力大,每天都一次有时候两次,爱喝饮料,可乐啥的经常喝,今年去体检,查出事了,尿蛋白 3+,肌酐 186,住院一个月,现在每天要吃十种药,慢性肾炎三期,最后会走向肾衰竭,这不是小事,真的很害怕!哎!大家一定要戒掉啊!

    \textbf{附评} 过来人的告诫总是发人深省,触目惊心,真的撸到怕了!半条命都撸掉了,弄不好还会搭上另外半条命,惨烈啊!住院吃药,人财两空啊!人生事业陷入绝对的灰暗和低谷,看不到任何希望,各种绝望和惶恐,那种心境只有亲历者才能真正明白。一位戒友感叹:“在一阵猥琐过后,却只是感觉生命更无意义了。”邪淫不会带来真正的快乐,那点快感只会让人更加空虚和无聊,让自己负能量缠身,性格乖张,脾气暴躁,伤得深了,各种症状也会找上门来,报应啊!出来撸迟早是要还的!天道祸淫最速!很多人都会把压力化为手淫行为,我以前也是,做学生党时感到学业的压力,就会通过手淫来发泄和逃避,暂时忘却一切,手淫完了感觉很累很困,直打哈欠,作业都不想做,第二天还起不来。手淫似乎能解压,但是能量的耗损会导致更大的压力袭来,这会让人陷入更糟糕的恶性循环。碳酸饮料也要少喝,过量饮用对身体是有害的,特别是冰冻的碳酸饮料更要少喝,太伤阳气了。现在这位戒友每天要吃十种药,他的处境可想而知,身体垮了,一切都指望不上,看病吃药要花太多的钱,生活也会陷入困顿。不生病时,感觉生活很美好,一旦生病,即使大晴天也感觉天空很灰暗,撸到一定时间就要算总账了,这笔账是赖不掉逃不掉的,最终全部报应在自己身上,到时就苦大了。慢性肾炎三期还不是肾衰竭,处于这个时期说明肾脏功能已经很差了,肾小球滤过率已经很低,如果不积极治疗的话会发展到肾衰竭期。不少人对手淫危害很无知,他们根本不知道继续撸下去的后果,如果知道这么严重的后果,真的会从椅子上跳起来,怎么能再继续摧残自己呢?手淫刚开始是喜剧片,好像天上掉馅饼一样,这么爽又是免费的,真的太好了,无知的少年被快感彻底征服了,后来撸下去就变成了灾难片,各种症状和不幸开始降临了,撸到最后就是恐怖片了,开始得大病了,惶惶不可终日,整日生活在恐慌和绝望之中,度日如年,被痛苦折磨时更是度秒如年!

    慢性肾脏病(CKD)分期:\begin{description}
        \item[CKD 1 期:炎症反应期] 这一期血肌酐没有明显变化,肾功能尚未受损,通常称之为肾炎。这一时期是最容易治愈的时期,临床指征多表现为蛋白尿、潜血、水肿、高血压。
        \item[CKD 2 期:肾功能代偿期] 这一期,血肌酐值处于 133 - 177 \unit{\micro\mole\per\litre} 之间。此时,肾脏还有一半的肾单位可以正常工作,可以靠自身的代偿功能来满足身体的日常需求。若及时接受正规治疗,是完全可以实现临床治愈的。
        \item[CKD 3 期:肾功能失代偿期] 这一期,血肌酐值处于 178 - 442 \unit{\micro\mole\per\litre} 之间。肾单位受损已经超过三分之二甚至更多,肾脏自身的代偿功能已经无法满足身体的日常需要。患者大多已经开始感觉到乏力,但症状仍然不明显。如果能得到及时、正规的治疗,仍有 50\% 的概率可以实现临床治愈。
        \item[CKD 4 期:肾衰竭期] 当血肌酐超过 443 \unit{\micro\mole\per\litre} 时,就已经进入了肾衰竭期,这时,肾功能已经下降到了一个不可逆的阶段,每时每刻都有肾细胞被杀死。这时许多症状开始明显出现,比如贫血、头晕、乏力、恶心等。治疗重点,已经变成阻止肾细胞继续坏死,保护剩余的肾功能了,有些医院也已经开始建议患者进行血液透析。
        \item[CKD 5 期:尿毒症期] 这一时期的血肌酐值通常超过了 707 \unit{\micro\mole\per\litre},患者的肾脏,有 90\% 的肾单位已经坏死,肾脏体积大多萎缩至 6 \unit{\centi\metre} 以下,且基本丧失了产生尿液的功能。患者大多会非常强烈症状表现,如头晕、乏力、恶心、呕吐、心衰、贫血等,且出现频繁。此时的治疗意义已经变成了尽量维持患者正常生活,减少患者痛苦。
    \end{description}
\end{case}

下面步入正文。

这季谈下戒色与生活的关系,戒色是一种理念、一种修为、一种历练,本身也是一种生活方式,比较理想的状态就是把戒色真正融入自己的生活之中,成为生活的一部分,能够和生活很好地平衡起来。上等的戒友戒到一定程度,不会觉得自己是在刻意戒色,而是感觉很自然,一点也不勉强和压抑,好像本该如此。真正有品质的生活是高度自律的生活,是负责任的生活,过去邪淫时其实是不正常的,那种状态身不由己,邪念一上来就把自己带跑,没有主宰权和控制权,很可悲的失控自毁的状态,沦为心魔的傀儡和奴隶,能有什么真正的自由和快乐可言?这是健康正常的生活状态吗?显然不是!戒色后我的内心有一种自由感和轻松感,还有一种终于恢复正常的感觉,谢天谢地,终于从那个怪圈中摆脱出来了。其实我们每一个人与生俱来都拥有天然喜乐与自在的权利,看看孩子就知道他们多么纯真与快乐,每天都很开心,而这种快乐的状态在邪淫后就消失了,我们开始在看黄手淫中寻找刺激和快感,背离了本初的纯净与美好,最后经受了巨大的身心折磨与痛苦。

很多人一开始误解戒色的含义,后来有了伤精症状的体验后才发现戒色是那么重要,也不仅仅是身体的症状,还会影响自己的容貌气质和运势,影响自己的学业、事业和家庭,邪淫对一个人的生活会造成全方位的负面影响,让人充满负能量,影响家庭和谐,影响人际关系,让人生彻底陷入低谷和恶性循环。真正的智者肯定懂得戒色的价值,戒色是生活中非常重要的一部分,甚至可以决定一个人一生的成败和命运的走向和结局。社会上很多人能成功一时,但真正能守得住的很少,很多人能赚几百万、几千万甚至上亿,但是后来开始吃喝嫖赌,各种乱来,结果生意亏损,濒临破产,甚至还有不少坐牢的。万恶淫为首,犯了邪淫,生活就会开始进入负面的轨道,那是很糟糕的生活状态,会干出很多荒唐的事情,会成为自己以前最看不起的那类人。台湾富翁黄任中邪淫一生耗尽 56 亿家产,穷困潦倒孤独离世,黄任中曾自嘲说:“人生像抛物线,当它要下来时,就下来,还真是拿它一点办法都没有!”这也许是他最好的墓志铭。

“人一思淫,心田即暗。中正之心已邪,则光明正大之气遂失。若人时时存邪念,积久而邪气蛊惑于身心,即小人矣。”中华儿女炎黄子孙要善养浩然正气,这样才对得起父母、对得起列祖列宗,这股浩然正气充塞在眉宇之间,如钢似铁,绝对正直刚强。\textit{人生欻翕云亡,好烈烈轰轰做一场。(文天祥《沁园春·题张许双庙》)} 男子汉大丈夫要干一番大事业,岂可沉迷于龌蹉猥琐的恶习?把身体精华全部射掉,在色情旋涡中不断沉沦,剩下一具千疮百孔、危如累卵的躯壳,所谓:满地荒唐精,一把辛酸泪,都云撸者痴,谁解症状苦?镜中人在悄然衰败,从纯真无邪的少年到灰暗颓废的猥琐大叔,从明亮喜悦的眼睛到无神暗淡的双眸,当心灵不再纯净,当身体的能量被恶习掏空,很多的坏变化都会出现,整个人、整个生活都会变质!沉迷手淫的生活不是你真正想要的生活,挣脱手淫的束缚,才知道什么是真正的幸福和快乐!

戒色和生活品质是密切相关的,而且戒色的潜在影响力实在太大,不戒色不会知道原来戒色这么好,身心的恢复,内心重新变得积极向上,整个人焕然一新,生活重新变得充满希望。邪淫是负面的影响力和破坏力,而戒色会带来正面的影响力,让自己的生命更有价值更有意义更充实,戒色可以重建一个人的身心状态和生活景象,会给一个人的生活带来不可思议的改变。\textit{世界上只有两种生活方式:腐烂和燃烧。胆小如鼠、贪得无厌之徒选择前者;见义勇为、慷慨无私之士选择后者。(高尔基)} 邪淫就是在腐烂,色不迷人人自迷,迷于色,贪于色,就像走入了一片雷区,指不定哪天就炸了,非常危险没有安全感。有的人前几年还行,但是疯撸一段时间后,神经症一爆发,那就痛苦无量了,或者得上其他重病,生活就彻底灰暗了,邪淫对生活质量的影响真的是摧毁性的,刚开始不觉得,因为刚开始伤得不深,感觉不是很明显,但是日积月累,恶果报应就渐渐显现了。邪淫是一个慢性耗损的过程,一点点废你,不知不觉间就让你滑入症状的深渊,当你猛然惊醒时,才发现自己的生活与处境已经今非昔比面目全非了,与别人的关系也严重失调,生活陷入了困境。

高品质的生活和一个人的经济基础有一定关系,但并不是开上好车住上豪宅才叫高品质的生活,真正高品质的生活是精神和心灵层面的,不是钱可以给予的,而是每一天都有纯净美好、单纯快乐、充满正能量的感觉。钱可以带来很多物质上的满足,但满足过后还是不满足,还是空虚,比如你买了一部新手机,刚开始爱不释手,但是过段时间就没感觉了,慢慢厌倦了。物质是如此,性也是如此,短暂的满足过后是更大的不满足,似乎永远无法彻底满足,你面对的是一个深不见底的欲望黑洞,不可能有彻底满足的那一天。而心灵层面却能因为戒色体验到一种持久的满足感和喜悦感,这种喜悦是常新的,是内心不断泉涌的喜悦。孩子没钱,也不追求物质,但是他们每天很快乐,而成人大多不快乐,绝大多数的成人都不知道真正的快乐在哪,他们把快感当作快乐来追求。在成人世界里几乎见不到那种单纯快乐的笑脸,成人世界里那种清澈明亮的眼睛更是少之又少,因为他们的心灵已经被污染了,已经不再单纯、不再纯净了,能量也被大幅耗损了。

生命的意义在于开启净化之旅,回到本初的纯净与美好,那种纯净纯善的纯真赤子的心灵状态,那种心灵状态就像光明璀璨的钻石一般。穿过灵魂的暗夜,再度做回纯真的赤子,纯净的心灵会带来纯净的愉悦,这是非常高端的体验。也许你来到这个世界上体验过很多事情,即使你开过几千万的豪车,住过七星级的酒店,但这些体验都不能和净化心灵所带来的美好感受相比。很多戒友都说戒色比手淫爽太多了,为什么会这样?很多撸者会说手淫真的好爽,可以幻想和无数女人发生关系,但其实戒色比手淫更爽,这种爽不是来自快感,而是来自纯净的心灵,来自战胜心魔主宰内心。高品质的生活源自内心的纯净与美好,不在于外面的物质世界。色情和手淫会毁了你的生活,国外戒色书籍提到:“色情的另一面是毁灭。”快感不是真的快乐,当一个人邪淫时,他的表情往往是严肃的,这种严肃里面带着某种贪婪和邪恶的感觉,两只眼睛充满了禽兽的感觉。邪淫是在开发自己内心的阴暗面,会导致负能量缠身,负能量最终会导向痛苦的体验。

现实生活中,我们应该很好地协调戒色与生活的关系,戒色稳定了会给生活带来积极正面的影响,戒色的好处是不言而喻的,亲身体会者都知道。戒色吧有一些思想存在误区的戒友和走极端的戒友,他们需要正确引导,其实戒色和生活是完全兼容的,是可以完全融合的,戒色是生活的一部分,很重要的一部分,但不是全部,其他时间该干嘛就该干嘛,自己要管理和安排好自己的生活。

\begin{itemize}
    \item 戒色后生活中不可过于紧张,应做到光明坦荡,问心无愧。
\end{itemize}

有些戒友读了一些戒色文章,知道女色的厉害,\textit{二八佳人体似酥,腰间仗剑斩愚夫,虽然不见人头落,暗里教君骨髓枯。(《金瓶梅》)} 他们知道沉迷邪淫的严重后果,但是搞得过于紧张了,影响到日常生活了,走在街上看到异性都紧张得不行,给人感觉就是神经兮兮的,不正常,戒色出现这种情况,是走偏了。戒色是光明正大的气场,不是那种紧张兮兮、躲躲闪闪的样子,戒色更像是雄狮的气场,而不是老鼠的气场。走在大街上是该管理好自己的视线,但也应该做到自然大方,不要让别人觉得你紧张和怪异。戒色应该是内心非常正的一种状态,具备深厚的定力,有相当强的观心断念能力,有较高的修养和修为。大家看比赛,教练一般都强调要专注,但不可过于紧张,太紧张会导致发挥失常,我强调的是要警惕而不要过于紧张。太极拳讲“松而不懈、紧而不僵。”这个自己要好好体会,戒色不能搞得紧张兮兮,这样正常生活都会受到很大影响,别人看到你就觉得不正常。

\begin{itemize}
    \item 把戒色融入自己的生活,但不要过于执着于戒色。
\end{itemize}

戒色本来就是生活的一部分,要懂得把戒色融入生活之中,慢慢习惯于戒色自律的生活,到时就像鱼在水中不会感觉到水一样。你不会感觉你在刻意戒色,因为你已经习惯于戒色,就像习惯于呼吸一样。记得我最初开始戒色时,我还有种我在戒色的念头,后来渐渐就没有了,因为戒色已经融入了我的生活,我每天都在修心,习惯成自然了。我发自内心热爱戒色,但我不会过于执着于戒色,过于执着也会导致内心的痛苦和焦虑。该做事时尽力去做,尽量把事情做好,做过之后内心就不要一直去想了,一直想这想那,担心这个担心那个,只会把自己搞得心神不宁。有的戒友非常担心自己破戒,很害怕哪天心魔进攻,把他再次拖入怪圈,这个可以理解,真正想戒色的人都不希望自己破戒,但也不必多想,关键是提升自己的觉悟和实战能力,这样具备实力后,就有成竹在胸的取胜把握。有时候执着是个好词,带着坚定执着的恒心是容易成功的,但看问题要具体情况具体分析,如果对于某些事情过于担心和执着,那只会适得其反。

\begin{itemize}
    \item 理性看待戒色,不要对戒色期望过高。
\end{itemize}

有些新人可能对戒色抱有不切实际的幻想,戒色的确会给很多方面带来很大的改善,但也要看具体情况而言,每个人的情况是有所不同的。有的戒友希望自己戒色一定要变帅,要帅到某个明星那样,以变帅为戒色目的会显得比较肤浅,一旦变帅了,就很难找到动力了,而且想变帅的背后目的往往是为了吸引异性,这样戒色动机就不纯了。有两句话说得很好:“只问耕耘,莫问收获。”我们对戒色不要有太高的期望,戒好每一天,加强养生恢复,改变的效果自然是显著的。

戒色一定要理性,我那时戒色是为了恢复身心健康,后来升华了,就是正己化人,我从来没想戒色后一定要怎么怎么样,很多事情是顺其自然的。有时期望越高,失望越大,达不到期望时往往会起退心。我们不应对戒色抱有太高的期望,好好坚持戒下去,一切水到渠成!生活中的很多事情都是如此,随着实力的增加,很多事情都会变得可能,就像有的人刚开始摸篮板都费劲,他就想着要扣篮,这就脱离实际了,如果他继续长高或者练习弹跳力,当抓框比较轻松了,这时候扣篮就开始变得可能了,很多运动员一开始也没想到要打破世界纪录,因为那个目标实在太高了,但随着成绩的提升,他的目标就自然变成了要打破世界纪录,因为他的实力已经渐渐达到了。

\begin{itemize}
    \item 婚前、婚后都要戒色,婚后戒色以寡欲为主。
\end{itemize}

婚前戒色非常重要,\textit{孔子曰:“……少之时,血气未定,戒之在色……”(《论语·季氏》)} 在发育期纵欲会影响生长发育,还会影响学业,婚前戒色也是一种历练,让你学会控制自己的欲望,提升自己的定力,面对诱惑时,自己能够把持得住,这是非常关键和重要的能力,青少年阶段如果能学会修心,对于以后的整个人生都有着莫大的正面影响。如今这个时代色情讯息非常泛滥,很多少年很早就接触了黄片,染上了看黄手淫的恶习。国家都在扫黄,黄片对于心灵的污染和腐蚀是非常严重的,几年时间就可以把纯真少年变成行尸走肉,一脸的无神、呆滞和颓废。戒色吧也有不少婚后的戒友,有的人就纳闷了,为何婚后也戒色?因为婚后也可能放纵,也可能继续看黄手淫,有老婆的人也会手淫的,手淫的对象可以经常换,但老婆是不能随便换的,人是容易喜新厌旧的。如果没有很强的责任感,已婚人士也可能出现一夜情、婚外情、嫖娼等现象,非常可怕。婚后戒色分两种情况,一种是身体健康的情况下,要做到寡欲,另外一种就是症状缠身的情况下,这时应该和老婆沟通好,自己先彻底戒色养生一段时间,等身体恢复了再节制性生活,古代名医对于那些重病患者一般都要求彻底禁欲一年以上,还要配合药物调理,如果带病纵欲,后果真的不堪设想。

\textit{男女居室,虽生人之大伦,为圣王所不能禁,然必行之有节,则阴阳和而孕育易。若淫欲无度,则精伤气馁,神散血枯。由是而潮热,而骨蒸,而枯槁,而赢瘦,而尪怯,变生种种,年寿促矣。若夫艳冶当前,姣娆在侧,情投意洽,顿起淫心,因而云雨绸缪,真精施泄,虽此身殆毙,有所勿顾。(名医沈芊绿云)} 养生之士,先宝其精,精满则气壮,气壮则神旺,神旺则身健,身健而少病,内则五脏敷华,外则肌肤润泽,容颜光彩,耳目聪明,老当益壮矣。此养生者以精气神为主,而尤以精为宝也。\textit{凡觉阳事辄盛,若一度制得,则一度火灭,一度增油。若不能制,纵欲施泻,即是膏火将灭,更去其油,不可不谨自防也。(《养生书》)} 精能生气,气能生神,荣卫一身,莫大于此。精与气相养,气聚则精盈,精盈则气盛,精盈气盛则神足。人年四十以下,多有放恣,四十以上,即顿觉气力衰退,衰退既至,众病蜂起,久而不治,遂至不救。\textit{人身非金铁铸成之身,乃气血团结之身。人于色欲不能自节,初谓无碍,偶尔任情,既而日损月伤,精髓亏,气血败,而身死矣。(孙真人)}

\textit{人生天地间,圣贤豪杰,唯其所为。然须有十分精神,方做得十分事业。苟不知节欲,以保守精神,虽有绝大志量,神昏力倦,未有不半途而废者。(周思敏)} 婚后寡欲是非常重要的,在身体健康的情况下一般推荐一月三次以内,婚后虽符合人伦,但也要注意节制,手淫恶习必须要戒掉,婚后对象固定,一般会慢慢厌倦和变淡,而手淫则会让你一直像打了鸡血一样疯撸,因为色情的内容是五花八门的,可以一直可以给你新鲜感,而且诱惑的内容已经变得越来越猛烈,越来越刺激,也越来越变态了。国外对色情的危害已经研究得比较透彻了,我们必须要远离色情的毒害,这样才能过上健康安定的生活,邪淫者其实是缺少安全感的,因为指不定哪天身体就会突然垮掉,别以为现在感觉还行,只要继续放纵下去,说不定哪天身体就会突然垮掉,当身体垮了,生活的方方面面都会受到很大的影响,家人也会受到连累。之前就有不少案例都是纵欲把身体搞垮了,婚也离了,老婆也跟别人跑了,工作也丢了,还要整日与药为伍,经常跑医院,甚至还要面对家人的不理解,亲戚的非议,活得不如一条狗。总之,婚前和婚后都要戒色,色字头上一把刀,戒色才能带来一种安全感,戒色的男人也显得更成熟、更镇定和更自律,也更有责任感,有一种正人君子的气质和风度。

\begin{itemize}
    \item 即使戒色不被人理解,也要保持淡定和坦然。
\end{itemize}

有的戒友会向家人朋友寻求一种认同与支持,但家人和朋友很可能不理解你戒色,因为他们还在认同无害论或者低估了色情和手淫的危害。无害论被那些砖家宣扬了几十年,记得上世纪九十年代杂志上就有无害论,砖家会把手淫后的症状归结为自己想出来的,非常荒谬的结论。这是一个砖家横行的年代,你以为那些砖家的名头很权威,其实他们很无知,是一本正经的胡说八道,打着科学的幌子诱导人邪淫。之前国外的性学家都是说手淫无害,甚至还强调手淫的各种“好处”,我相信中国的砖家也是受了国外的影响,以前无害论宣扬得实在太厉害了,以至于现在都进了教材。

在民国时期还有上世纪 50 年代左右,有害论还是主流,后来国外的性解放运动以及国外性学家的各种误导性的著作开始影响中国的知识分子,当某些知识分子成为了中国最早一批的性学家,他们的观点肯定是沿袭国外性学家的观点。知识分子有话语权,可以在杂志期刊报纸上发表文章,这样就会误导一大片,错误的观点会越传越广。好在国外现在已经认识到了色情与手淫的危害,国外也崛起了很多戒色网站,相信无害论会慢慢被认清,到时中国会有一批真正具备真知灼见的专家教授来澄清无害论并引导大家戒色,我希望他们在尊重传统戒色的基础上来提倡科学戒色,不要贬低传统戒色。有的人为了提倡自己的戒色方法而贬低其他所有的戒色方法,资深戒友应该都知道这类人的存在,这类人的心胸其实很狭隘,德行也有问题,我们一定要懂得尊重传统的戒色方法,这点很关键。

戒色不一定要告诉家人或者朋友,特别是在你戒得还不是很稳定时,最好先保密,因为告诉身边的人也会给你带来一定的压力,如果你的状态不是很稳定,反而会影响你戒色,甚至因为他们的误解会带给你一些不必要的困扰。最近有位戒友就向我反馈了这方面的问题,他把自己戒色的事情告诉了同学,而有的同学依然执迷不悟,知道他在戒色就故意说那些事情来影响他,这属于损友一类,应该是要远离的。有的戒友还告诉了自己的父母,有的父母还是支持的,但也可能出现相反的情况,会用无害论来反对你戒色,这也是比较多见的情况,所以最好先保密。如果你在朋友间很有威信,那可以试着劝他们戒色,也可以通过委婉的方式来劝,不要说得太直白,可以根据对方的一些伤精表现,特别是容貌气色的变化来提醒对方,比如双眼无神、精神萎靡、黑眼圈眼袋等,从这个角度来切入,比较容易使人生信,也可以从肾虚的其他症状表现来劝,自己心里对危害要清清楚楚,说出来也要具备说服力。如果自己戒得比较好,精气神也很不错,双眼炯炯有神,充满自信与底气,以这样的状态和气场来劝人戒色,别人也容易相信。

宣传戒色很可能会招致误解,在网上宣传甚至会招致谩骂,自己一定要做好心理准备,即使被误解也要保持淡定、从容和坦然。即使被人骂,其实你已经宣传成功了,他们已经知道戒色这回事了,种子已经种下了,假以时日,等到他们症状缠身就会想到你的规劝。之前很多人都是误解戒色吧的,后来都来戒色了,因为症状是最好的老师!会督促他们来戒色。当伤到一定程度,症状也会逼他们戒色,因为不戒不行了,身体实在吃不消了,甚至有一种大难临头的感觉,到时才发现戒色是多么英明和正确,之前对戒色的误解是多么愚蠢和幼稚!

\begin{itemize}
    \item 合理安排时间,不要让戒色占用过多的生活时间。
\end{itemize}

我们要正确把握戒色与生活的平衡点,要戒色成功需要投入大量的时间和精力来学习戒色文章和练习断念,但也不应占用生活中过多的时间,很多学生党和工作族平时也很忙,事情很多,自己一定要做好时间管理和精力管理。我了解过国外一些高水平的运动员,他们也不是整天在训练,但是他们的训练很系统很有计划很有针对性,每天训练的时间也不是很长,但训练质量很高,常年不间断,用坚持的力量来逐步提升实力。有的戒友也去健身房健身,大家都知道不可能 24 小时都呆在健身房,只需每天训练一个小时左右,持之以恒地坚持下去,力量和身材就会有很大的变化,关键是坚持,注重学习和训练的积累,水滴石穿,绳锯木断!当然如果个人时间比较多,也可以加大对戒色的投入,时间充裕的话可以加大学习的力度,如果能从戒色中找到乐趣,那就可以做到学习很长时间都不会感到疲倦。我比较欣赏一句话,那就是:“享受你的训练!”举起杠铃时很多人感到酸痛难忍,但是那些练家子却感觉很爽,他们享受那个举铁的过程。我们也要享受戒色的过程,找到其中的乐趣,这样更容易坚持下去。

有的戒友的思想误区是:生活中全部都是戒色,戒色占据了生活的一切。我要说的是不要让戒色荒废了生活,戒色虽好,但生活中还有很多其他重要的事情等待你去做。好好戒色是为了更好地生活,戒色不能走极端,思想偏激的戒友时常会出现,他们需要正确的引导。戒色应该和生活很好地融合起来,戒得好了,精神状态佳,脑力状态棒,精力状态旺,这样就可以更好地学习和工作。戒色固然重要,但一定要协调好自己的生活,戒色也不是万能的,但戒色可以给你一个好的状态去奋斗自己的人生。

要学会科学地制定、严格地实施自己的学习计划,比如每天看多少戒色文章,做多少笔记等,有时候不一定要做笔记,也可以复习笔记,温故而知新是非常重要的,我很注重复习笔记,我有很多顿悟都是在复习时出现的,看一遍和看十遍的感觉是截然不同的,看十遍和看五十遍的感觉又是完全不同,有的笔记我都看了上百遍,依然会有新的体会和心得。很多内容你看一遍几乎没有留下任何印象,只有结合实战的体会再回过头来反复看,这样才会有比较深的理解和认识。等你觉悟提升到一定程度就能悟出越来越深的意思,越来越能够把握戒色文章的重点,越来越能够在实战中落实和贯彻。在制定学习计划的时候,要考虑进度和完成的能力,并且能够保证学习计划雷打不动,要学会合理安排自己的时间,每天分出一部分时间给戒色,就像你每天会分出一部分时间来刷牙洗脸、吃饭、睡觉等。大家每天睡觉前可以回想一下今天做了什么,很可能不少琐事占据了大多数的时间,自己可以规划一下日程,这样就可以摆脱琐事的纠缠,从而节省很多时间和精力来干正事。

\begin{itemize}
    \item 认识到戒色对生活的巨大影响力。
\end{itemize}

当你可以呼吸到新鲜空气的时候,你感觉不到它的珍贵,当你快窒息时才知道空气是那么重要。同样地,当你拥有健康身体的时候,也感受不到它的价值,当你症状缠身时,才知道无病的日子是多么奢侈。多少撸者因为看黄手淫而身体垮掉,他们挣扎在症状的地狱里,暗无天日,苦苦煎熬,而戒色就像一道曙光降临,这是一道拯救的光芒。我到现在看过非常多的逆袭案例,很多戒友之前因为手淫而身心俱废,后来逆袭了,真的是焕然一新,整个人都不一样了。有的撸者会说撸完后整个人都不好了,而戒色是让整体感觉有一个飞跃,一种能量状态的飞跃,再次回到那种阳光健康、积极向上、开朗乐观的状态。

戒色对生活的影响力实在太巨大了,可以彻底扭转生命的颓势,让生命之花重新绽放。一位戒友说:“戒色后变化好大,脸上的痤疮好了一大半,身体与心灵都焕发着一种轻松与快乐,睡眠质量也逐渐变好,人际关系也有所改善。”另外一位戒友说:“到现在已经戒色有四年了,戒色的好处真是彻彻底底感觉到了,从猥琐自卑的小老头变成人见人爱的帅哥,真是特别感谢戒色吧的存在,是飞翔老师与众吧友给了我新生,尤其是去我们学校贴吧宣传戒色的吧友,也是你拯救了我。”还有一位戒友说:“坚持戒色养生,尤其是戒除 YY,神衰症状终于好多了,好几年都没这么精神过了。”戒色可以让你重新燃起对生活的希望,邪淫的阴霾正在逐步退去,一切似乎又回到了正轨,你变得爱笑了,内心感到自由而快乐,又有底气、自信与勇气去面对自己的生活了。

\paragraph*{总结}

戒色之后,我们要好好管理自己的生活,正确处理戒色与生活的关系,合理安排时间和学习计划,把戒色真正融入自己的生活中去。戒色本身就是一种生活方式,戒色也是一种能量管理,当能量往上流,滋润五脏六腑和大脑,获得纯粹的大快乐;能量往下流,是一种耗损和外泄,虽然能带来短暂的快感,但最终必将体验到痛苦,沉迷色情与手淫必将毁掉一个人的生活。经过多年灵魂谷底的挣扎、惶恐、穿越与淬炼,才发现在纯净的心灵之中藏有这么轻松喜悦的体验,生命中真正重要的东西往往安住于朴实无华,简单、单纯而纯粹!戒色可以重建内心的圣殿,让你过圣洁的生活,让你有一种光明而崇高的感觉,任何物质、任何黄片都无法给你这种感觉,邪淫只会让人堕落,让人迷失,让人背离本初的纯净与美好。戒色后可以过一种高品质的生活,体验不一样的身心感觉,邪淫的状态就像粪坑里的蛆,以屎尿为乐,戒色后就像翱翔天际的雄鹰,整座城市都在你的视野之下,你本可以翱翔于天际,为何要在粪坑里翻滚?现在是时候戒掉它了,让堕落的灵魂得到拯救,让生活拥有全新的品质,让生命焕发全新的光彩,让人生充满全新的希望!

下面分享一首戒色诗歌。

\begin{poem}[春风吹醒纯净的灵魂]
    \begin{multicols}{3}
        \centering~\\
        打开儿时记忆的封印 \\ 那个欢快奔跑的男孩 \\ 浮现在脑海里 \\ 戴着柳条编的小帽 \\ 奔跑在春光里 \\ 细腻温柔的春风 \\ 吹拂在男孩脸上 \\ 滋润着纯净的灵魂 \\ 凝眸望着一切 \\ 树木、小草、花朵和阳光 \\ 内心自然感到喜悦 \\ 那些单纯而美好的时光 \\ 是如此值得怀念 \\ 徜徉在纯净的王国 \\ 无忧无虑 \\ 一切是如此简单 \\ 却将你深深打动 \\ 不懂大人的世界 \\ 也不懂看黄手淫 \\ 只是纯真地存在 \\ 却感到特别快乐 \\ 又一年春天 \\ 我独自一人来到公园 \\ 树绿了,花开了,草青了,鸟鸣了 \\ 孩子们也欢快起来了 \\ 这是一个美丽的季节 \\ 我带来了纯净的灵魂 \\ 我已经彻底告别了邪淫的生活 \\ 春风吹拂在我的脸上 \\ 一如儿时那般细腻温柔 \\ 我闭上了双眼 \\ 用心感受此刻的纯粹与美好 \\ 当我再次睁开双眼 \\ 我看到了孩子们欢快奔跑的身影 \\ 我幸福地笑了
    \end{multicols}
\end{poem}

下面推荐一本书。

\begin{book}[《醒世导航》]
    犟牛居士对《道德经》的感悟,《道德经》是一部旷世奇书,短短的五千言,囊括了天文地理、政治经济、军事科学、文学艺术、人文自然等诸多领域。大至天体运行规律、宇宙自然法则;小至养生保健之法、为人处世之道,巨细靡遗,无所不包。自从两千多年前《道德经》问世以来,历朝历代方内人士从不同的角度诠释其意义,洋洋洒洒,不知凡几,仅收入《道藏》的译本就达五十余种。在浩如烟海的道教典籍中,老子的《道德经》始终具有不可撼动的崇高地位,被历代道教祖师奉为圭臬。无论《道德经》有多少译本,用佛教的观点去解读这本深奥的道家典籍却鲜见于世,因为佛道两家自古以来就存在着几乎不可调和的门派之见。本书作者犟牛居士自从接触道教之后,以自己亲身体悟的佛法去感受道教的玄妙,并用朴素的文字加以诠释,在破除门户之见和促进宗教融合的工作中,做了一番独到可喜并且有益的尝试。里面讲到:“众生之迷,主要迷在认妄识为心,此心是心、意、识三魔之一,非菩提自性大觉的真心。真心,空也,本体也。此即老子前章讲的无体无形、无生无灭,超于天地之神。认识此心,在当今社会的五大宗教和诸多门派的众多学子、修者中,亦凤毛麟角。”犟牛居士的开示非常殊胜,几句话就点到了修行的核心,从《道德经》的角度来诠释,别有一番古韵。犟牛居士也是我个人比较喜欢和敬仰的一位大德,希望有缘者能读到《醒世导航》,领悟自己的真心本体。
\end{book}
