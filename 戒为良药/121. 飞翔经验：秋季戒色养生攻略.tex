\subsection{秋季戒色养生攻略}

\paragraph{前言}

有的人会觉得家里经济困难,觉得这辈子结婚无望,所以就只能靠手淫、嫖娼、不负责的性关系来满足自己的欲望。这其实是一种放纵的借口而已,也是一种自暴自弃的心理,家里经济困难的人有很多,之前还有沦为躺床废人的戒友,但是他们后来逆袭了,找了稳定的工作,生活有了盼头,结婚也在计划之内。我发现当有了正能量后,更容易把工作做好,脑力和精力都处于非常棒的状态,这样做起事情来也很有效率,如果沉迷色情与手淫,工作状态也会大受影响,脑子里邪念很多,负能量很重,工作也干不长。当一个人进入正向循环后,很多事情都是水到渠成的,本来沦为废人,工作和婚姻都没指望了,每天与药为伍,后来戒色养生了,身心逐渐好起来了,到时生活又重新变得充满希望了。即使你现在很穷,但只要你具备正能量,你就有能力开创自己的美好人生,工资刚开始也许不高,但坚持干几年慢慢就上去了,你的职位也会提升,你已经有足够的正能量来坐稳那个属于你的位子,你能稳得住那个位子。本来是员工,后来当管理了,工资自然上去了,因为有了戒色觉悟,所以不会出去嫖娼乱来,把钱都用在正道上,这样就强化了正向循环,人生就会越走越好。

所以,不要把结婚无望作为放纵的借口,人生是可以打拼的,很多亿万富翁当初也是白手起家。觉得结婚无望就放任自己沉迷手淫、嫖娼等,这是完全错误的思想!其实他们内心还在贪恋快感,所以就会找出这么一个借口,把自己设定为不能结婚者,所以就有理由放纵了。退一万步讲,即使真的无法结婚,也不应该让自己沉迷于邪淫,邪淫会给自己带来很大的痛苦,甚至可以毁灭你的人生!当你的身体垮掉了,一切都会变得灰暗,生存质量会大幅下降。从更高的角度来讲,不结婚正好有利于修行,心里少了一份挂碍,从世俗角度来讲,则是不孝有三无后为大。这两种角度都是对的,看你自己从哪个角度来看问题。无论如何都不应把结婚无望当做放纵的借口,当你戒色成功了,有足够的精力去胜任工作了,到时候结婚就是可能的,关键不要消极思考,以为自己不行,当你爆发了强大的潜能后,你会发现当初的自我设限是完全荒谬的,你能成为非常优秀的成功人士,你能成为某一行业的佼佼者!1982 年,马云第一次参加高考,首次落榜,数学只得了 1 分。1983 年,马云第二次参加高考,再次落榜,数学提高到了 19 分。如果当时马云自暴自弃,也就没有现在的成就了,自己的人生要自己去奋斗,千万不要自我设限。

下面分享一些案例。

\begin{case}
    感恩飞翔哥,我现在戒色快一年半了,每天学习戒色文章两小时左右,极少有间断,现在戒色感觉像是在坐过山车,一会儿欲望很高,一会儿欲望相对较低,感觉还是欠稳定,每天战战兢兢,感觉破戒就在一念之间,很容易就被心魔反转过去,觉得戒色越来越难,而且嗔恨之心比较强烈,看了元音老人的开示,感觉还是化解不了,请问飞翔老师怎么办?
    \subparagraph{附评} 这位戒友戒得很不错,坚持学习极少间断,但他还需进一步完善自己。欲望是浮动的,和季节、饮食、锻炼等因素都密切相关,真正的稳定是做好断念,不怕念起,就怕觉迟,每天都应该保持警惕,《诗经》中说:“战战兢兢,如临深渊,如履薄冰。”戒色是该提高自己的警惕意识,破戒就在一念之间,一念防不住,就会被心魔攻破,戒色就是念战!不仅要克服意淫、怂恿念,也要克服嗔念等负面念头,和断念的原理一样,通过不断对治,嗔念会逐渐减少的。嗔念是低频负面念头,对身心对他人都很不好,必须下决心克服,多发感恩心、宽容心、慈悲心,学会忍辱柔和,真正的忍辱不是懦弱,而是一种高度自控、有力量有主宰的表现。憨山大师《醒世咏》:“红尘白浪两茫茫,忍辱柔和是妙方。到处随缘延岁月,终身安分度时光。”《佛遗教经》:“能行忍者,乃可名为有力大人。若其不能欢喜忍受恶骂之毒,如饮甘露者,不名入道智慧人也。”一念嗔心起,百万障门开,嗔心是很可怕的,这位戒友觉得戒色越来越难,和他无法克服嗔心有很大关系,嗔心不克服,就会觉得内心很不稳定,嗔心就像内心的地震一样,会产生巨大的破坏力。从中医的角度来讲,怒伤肝,生气嗔恨会导致气机紊乱,影响体内脏器功能,不少人都是生气后神经症复发或者加重的。为什么自己无法克服嗔恨的心态,一定要找到原因,大德的开示是很好的,关键自己要理解和吸收,按照大德的指示去做。我过去也是经常愤怒,一副愤青的模样,也很容易生气,后来通过学习我知道了嗔恨不仅对自己身心健康很不好,还会严重影响人际关系,而且我还认识到如果被别人诽谤或者辱骂了,自己能够忍辱柔和,那是可以消业的。广钦老和尚:“忍辱是最大福德之处,能行忍的人,福报最大,也增加定力且消业障、开启智慧。”星云大师:“能忍讥耐谤,就能成就大器;能不为世间忧悲苦恼动摇,就能完成大勇、大力、大无畏、大担当的人格。”寒山大师云:“嗔是心中火,能烧功德林,欲行菩萨道,忍辱护真心。”化解嗔心,首先要认识到嗔心的危害,其次可以用思维来对治,如何思维可以多看大德开示,再次,就是直接断念,嗔心一起,马上就断掉,断掉后也可以加上思维对治,这样就是双保险。
\end{case}

\begin{case}
    因为 SY 让我多疑、焦虑、记忆下降、嗜睡、身体素质下降、自卑、猥琐、脑子整天想一些龌龊的事情,跟女生交流我都会害怕脸红。让我失去了很多东西,脾气很暴躁,经常和父母吵架,别人都说是叛逆期,但只有我知道其实是因为 SY 导致的。现在我多看戒色文章,有空就听喜马拉雅 FM 飞翔的《戒为良药》,现在的我也不像以前那样自卑了,最近还参加了各种运动和歌唱比赛,整个人跟焕然一新一样,虽然我戒色不是很久还没完全恢复,但整个人很精神,跟以前完全不一样的状态,人也变得开朗大方起来。我用我的经历让大家知道 SY 真的会毁了一个人的一生,希望各位戒友不要再被心魔所支配,希望大家能早日戒除这个让人毁了一生的恶习。
    \subparagraph{附评} 戒色后能量得到了提升,整个人焕然一新,身心状态完全不一样了。SY 前的自己、SY 后的自己、戒色后的自己,这三个阶段是非常不同的,SY 前的自己不知道 SY 的危害,那时还是无知幼稚的孩子,心灵单纯、纯净,每天都很快乐,那时也不懂得纯净状态的可贵,不懂得珍惜。自从掉进了撸坑就一发不可收拾,从向着国旗敬礼的纯洁的少先队员堕落为龌龊猥琐见不得光的撸管男,真可谓一个天一个地,从天堂到地狱。SY 后的自己会感受到很多坏变化,主要就是身、心两方面,有的人可能身体的伤精症状还不算严重,但心理上却出现了很多不好的变化,从开朗阳光的男孩变成了沉默寡言、郁郁寡欢的人,很多撸者都出现了社交恐惧症,还有强迫症、疑病症等等。仔细对比一下 SY 前和 SY 后,就会发现变化是巨大的,那个纯真喜悦的少年在 SY 后消失了,取而代之的是一个吸黄毒的瘾君子,一个对快感极度贪婪的人,看的片子多了,心理还会发生变态。邪淫者处于烦躁易怒的状态,很容易和父母吵架,这和叛逆期有所不同,叛逆期与独立意识和自我意识日益增强有关,而邪淫后出现争吵是心里烦躁所致,一如之前某位戒友所言:“破戒后发现整个人的精气神立刻下来了,也莫名其妙地变得很容易烦躁。”邪淫者把自己变成了一个负能量的刺猬,脾气会变坏,很容易伤到别人,动不动就会和别人吵架。戒色后的自己那又是完全不同的状态了,可以感觉到自己的身心正在恢复,精神面貌正在逐步改善,又会重新变得积极开朗起来,感觉自己的正能量也跟着起来了,内心的幸福感也增加了,脸上的笑容也多了起来,似乎变回了 SY 前的自己,只不过这时已经懂得了珍惜,知道纯净状态的可贵,也很享受这种纯净美好的感觉。一位戒友说:“好久没有这么精神了,整个人都变得自信、阳光。”另外一位戒友说:“因为我不邪淫,光明正大,心中自然无所畏惧了。”戒掉这个毁人一生的恶习,不再受心魔支配,这样的人生才真正值得期待。
\end{case}

\begin{case}
    飞翔哥您好,我今年 19 岁,戒色一共三次,第一次 2014 年的时候戒色一年零两个月后来放松警惕破戒,后来又戒了一年零八个月也破戒了,这次是七个月,就在昨天破戒了,晚上连续破了三次今天早上破了两次,有种破罐子破摔的心态,我真的不知道该怎么办了,每次破戒前一二十天左右生活上的习惯就发生了变化,比如看擦边图、熬夜、赖床、不看戒色文章之类的,现在我很难受,不知道表达什么,只想让飞翔老师开导一下,怎么重拾信心。
    \subparagraph{附评} 这位戒友这三次其实戒得还可以,两次都超过一年,第三次七个月,19 岁能达到这个战绩已经很不容易了,不过他毕竟还小,心智还不太成熟,状态容易起伏,不够稳定。破戒总有原因,自己在破戒后一定要认真总结和反省,问题到底出在哪,一定要认识清楚。不管戒多久,都要保持警惕,戒色是持久战,要做好持久战的心理准备,不少人戒到半年或者一年以上,进入戒色稳定期了,就开始有骄傲轻敌的想法了,觉得自己已经成功了,可以离开戒色文章了,这类人其实很容易破戒的。前段时间一位戒色三年多的戒友回来了,他之前离开戒色吧了,为什么回来?因为破戒了!戒色三年很不容易,可以说很厉害了,但心魔还是会进攻他,平时也要面对各种诱惑的考验,如果放松警惕,那就很容易破戒。戒到一定程度觉得自己不用看戒色文章了,这种想法很危险,看戒色文章还有给新人答疑,这可以很好地保持自己的戒色状态,如果离开戒色吧,也不看戒色文章,也不帮助戒友,和戒色的一切都疏远了,那戒色的状态就容易出现严重下滑。不看戒色文章,到时自己也会忘记戒色,但是心魔不会忘记你,等你忘记戒色,放松警惕了,心魔就会进攻你,一举把你拿下,再次把你变成撸管肉机。我戒到现在依然要面临心魔的进攻,有时翻种子比较厉害时,一天当中图像要上头十几次,但每次都是一觉察就消灭了,我绝不会让心魔得逞。国家在和平时期依然不会忘记战争,不会忘记过去的教训,还是在不断练兵,不断发展自己的军事实力。戒色是终身修为,我们不要忘记心魔的存在,等你忘记了,失去警惕了,结果肯定会破戒,前辈深知警惕的重要性,也懂得保持良好的戒色状态,这样才能戒得比较稳定。这位戒友在破戒前就有了一些不好的变化,比如看擦边图、熬夜、赖床、不看戒色文章等,这时候就要引起高度警觉了,擦边图千万不能看,自己要学会及时调整戒色状态,不能放任自己重新滑入破戒的轨道。被心魔攻陷后遭到了屠城式的血洗,两天就破了五次!如果不及时忏悔、反省和总结,很可能会连破十几次乃至几十次,身体会再度爆发严重的伤精症状。话又说回来,19 岁能戒一年零八个月,已经相当不错了,我 19 岁时还没超过一个月,不过那时我完全是在强戒,没看戒色文章。现在的孩子有了戒色的学习交流平台,是非常好的机会,一定要好好珍惜。戒色真的是系统工程,越戒到后来要求就越高,必须坚持学习,不断提升自己的觉悟和境界,这样才能进入极稳定的戒色层次。
\end{case}

\begin{case}
    飞翔老师你好,我之前发过帖子,说我手机里有几十个 G 的黄片,舍不得删,很多戒友都劝我删掉,但我没听大家的劝告,还是迷恋其中不能自拔。于是,昨天晚上看片连撸了三次,第三次射完后,觉得心口一阵剧痛,濒死感突然袭来,这回心里是真的怕了,再撸就死定了!我觉得我最对不起父母,父母省吃俭用,给我买好吃的,而我却做出这种禽兽不如的事情来!于是,我毅然决然地删掉了所有黄片,远离黄源,避开色弹,认真学习戒色文章,提高觉悟。希望各位戒友鼓励我、监督我,我是真心想戒掉!
    \subparagraph{附评} 这位戒友这回真的撸到怕!撸到面临死亡了!很多人都是不见棺材不掉泪,不撞南墙不回头,前辈的告诫是非常重要的,所谓:不听老人言,吃亏在眼前。前辈有很深刻的亲身体会,而且看了非常多的案例,知道撸管后会发生什么!但不少新人是听不进去的,他们堕入了色情的迷魂阵,极度迷恋其中,用高音喇叭 24 小时不间断对他们劝告,也无济于事,只能让症状的大棒敲醒他们了,人尝到痛苦后就会幡然醒悟,包括那些诽谤戒色吧的人,当他们体验到症状的痛苦后,他们就认可戒色了。有的人在了解了手淫的危害后,还抱着侥幸心态,认为自己能控制,其实真正能控制的人是不会撸的。前段时间一位戒友在帖子里提到了“非榨干不可”,他是在形容破戒时的状态,一次不过瘾,过一会就来第二次乃至第三次,直到射不出了还不死心,真的是“非榨干不可”!被心魔附体后也不是自己做主,完全就是身不由己的状态。连续手淫的危害是加倍的,就像一个人刚跑完马拉松,接着让他继续跑第二个马拉松,真的可能会出现猝死,上季的一个案例也是这种濒死的情况,非常可怕!很多撸者都还很年轻,父母还指望着自己的儿子,以前计划生育,一般都生一个,如果撸挂了,后果真的不堪设想,谁给他们养老送终?父母给你买好吃的,并不是让你躲在房间里偷偷手淫的!把自己身体撸垮了,真的很对不起父母的养育之恩。那些邪淫资源真的要下狠心删除,千万不可有丝毫的贪恋之心,那是毒品、危险品、腐蚀品,是毁人、废人、毒害人的东西!认清色情的危害,真正拿出决心来戒,不要舍不得删,等你被症状虐爆,你就知道那些东西真的太害人了,你就不会觉得它们是“福利”了。一位戒友说:“我戒色有差不多十天,但是已经感觉到戒色的福利了,我有社交恐惧症,但是戒色十天左右,明显感觉到眼神有定光,很有神气,感觉不怕社交的感觉。”真正的福利不是 AV,而是戒色,即使你看了上亿部的 AV,也不会感到真正满足,内心只会越来越空虚,人也会越来越颓废,沉迷于色情与手淫会彻底毁掉你的人生,这不是耸人听闻,而是无数戒友血泪史的总结。对于邪淫资源只有一个字:删!删光好戒色,删光就是决心的体现,舍不得删的人还没戒就已经失败了。所以要下狠心,做一个了断,让那些邪淫资源见鬼去吧!再也不看那种害人的视频!!!
\end{case}

\begin{case}
    飞翔大哥,今天早上五点半下火车后回到寝室,一个人独处,本来想睡觉,但是一直躺床上玩手机,心魔不断进攻,最后没有抵挡住,连破两次,来到戒色吧一年多了,最长戒了 120 天,这次 45 天又破戒了,准备考研,看来还得加强对戒色文章的学习和观心断念的练习,不能放松!
    \subparagraph{附评} 躺床上玩手机容易“出事”,所以很多资深的戒友都说不要在床上玩手机,他们的告诫是对的。躺着很容易放松警惕,又是一个人独处,心魔很容易进攻,手机上擦边新闻很多,手机上网是很危险的,一定要提高警惕,看到诱惑内容视线不要停留,不要聚焦,不要回看,不要去点击,这些都是对境实战的纪律。每天都有很多戒友阵亡,其中不少都是玩手机破戒的,手机现在上网很方便,很容易接触到诱惑内容,自己一定要很小心,不要去看擦边内容,看到马上避开,避色如避箭,用好避字诀,实战中就能顺利过关。我有时也会遭遇诱惑图片,包括在戒色吧删除那些发黄图的楼层,即使在戒色吧也要面对诱惑的考验,谁叫我们生在一个诱惑满天飞的时代,这就需要我们自己具备很高的觉悟和定力,还有超强的实战意识,对境时能够做出正确的选择,而不是飞蛾扑火。面对心魔的进攻,不要慌乱,要沉着应战,各种邪念、各种图像、各种怂恿会冲上头脑,头脑是战略高地,占领头脑,心魔就可以为所欲为,所以我们必须击溃心魔的进攻,要像狼牙山五壮士一样拼死守住自己的头脑高地,不管多少邪念冲上来,都坚决打下去。磨刀不误砍柴工,工欲善其事,必先利其器,功在平时,用在一瞬,实战就是眨眼间的事情,自己平时要好好练习观心断念,做好慎独,时刻保持警惕,只要坚持学习和练习,肯定会不断进步的,一定要注重积累,多做笔记,多复习,每次实战过后自己也要认真总结,看看自己还有哪些方面做得不够好,这样就可以及时完善和强化。实战是最终的检验,真金不怕火炼,一定要强化自己的实战能力,这位戒友之前戒过 120 天,还是不错的,不过这次只有 45 天,经过这次实战,他意识到了还需加强学习戒色文章和练习观心断念。实战过后很容易发现自己的问题所在,实战会暴露出之前未发现的问题,要做到精于实战,就要不断学习、练习和总结,自己要仔细体会和深入研究前辈的断念实战理论,坚持一段时间就会有一个很大的提高,到时你的断念实战水平就会今非昔比,你不再是那个被心魔虐得团团转的菜鸟了,你已经可以瞬间制敌了。当一个歹徒闯进你的家里,如果你学过擒拿术或者格斗术,那就可以迅速制服歹徒。同样地,如果一个邪念闯进你的头脑,你的断念水平很高,你就可以瞬间制服它。两者的道理是一样的,一个闯进你的家里,一个闯进你的头脑,如果制不住,那就会被对手牵制。佛法讲“制心”、“降伏其心”,就是在讲念头实战,我们一定要强化观心断念,制服邪念,实战意识一定要强,每时每刻都准备着战斗!
\end{case}

\begin{case}
    自从戒色以后,自信心越来越多,工作也越来越能静下心来,甚至歌曲都比以前好听了,思维敏捷了好多,遇事不慌,今天遇到一个流氓行为的人在工厂里骂大街,影响我的工作,被我三句话说懵,不吭声了,戒色给我带来的好处实在太多,感谢飞翔老师,我会继续努力,积极向上。
    \subparagraph{附评} 戒色之后会逐渐产生一种神情,那就是——不怒自威!气场完全撑起来了,一股浩然正气如钢似铁,让别人心生敬佩,让别人心服口服。戒色的好处真的很多,心理方面可以让人自信、镇定,有底气,有斗志,有冲劲,变得开朗、坦然,光明磊落。邪淫者容易出现烦躁相,这是肾精耗损的一个表现,也是负能量增加的一个表征,很多撸者也会出现自卑,容貌气质变差,精神萎靡,很容易产生自卑心理。戒色之后自信会逐步恢复,敢于直视别人的眼睛,视线变得集中,不再游移不定,说话也变得自信有底气,似乎在谈吐之间多了一份特别的力量,甚至还没说话,只要看一眼对方就把对方给征服了。这是两种能量场的潜在交流,你的能量场强,甚至不需要说话,就可以让能量场弱的人感到敬畏,就像一只老虎走到一只猫面前,虽然都是猫科动物,但是能量场完全不一样,老虎是百兽之王,威武雄壮,威风凜凜,一声惊天动地的吼声彰显王者的风范。戒色后就是猛虎出笼、雄狮登场,有一股战胜心魔后的底气和正气,这种气场十分强大!即使坐在首富面前也不会慌乱,依然那么坦然和镇定,战胜心魔的人胜过战胜千军万马,戒色让你具备这种王者般的底气。这位戒友提到“工作也越来越能静下心来,甚至歌曲都比以前好听了。”当一个人内心变得宁静祥和,就能感受到邪淫状态下无法感受到的美好,之前有戒友说戒色后听儿歌都觉得很好听,能够感受到那种单纯的大快乐,而追逐快感则会屏蔽掉这种感受能力,脑子被邪念占据,内心变得蠢蠢欲动、烦躁不安,邪淫带给人的负面影响真的太大了。一位戒友说:“因为戒色我体会到了自律所带來的清净与自由。”戒色的价值不可估量,好处真的是太多了,根据我的体会和研究,戒色会产生一种力量感,不是卧推几百公斤的那种肌肉力量,而是内心的力量感,那种真正的主宰感。ERIC SPOTO 是卧推之王,他的纪录达到了惊人的 722 磅(约 327 公斤),虽然外形强壮如牛,但他不一定能降伏心魔,如果无法降伏心魔,内在就有一种虚弱感,不管外在多强壮也无法掩盖内在的虚弱感。真正降伏心魔的人,他的底气和力量感完全超过卧推之王、深蹲之王和硬拉之王,降伏心魔的人是王中王!!!他的眼神有王者的威严,凛然的目光具有震慑一切的气势!!!
\end{case}

\begin{case}
    我似乎找到了自己一直破戒的原因,没有意识到断念实战的重要性,欲念一上来就跟着跑,看了很多戒色文章,行了很多善,养生身体也恢复得很好,但某次独处警惕弱欲念上头时没及时断,一样被心魔附体变成撸管肉机,尽管过程中清醒地知道不该这样,但身不由己,曾经的毒誓豪言壮语也都抛到脑后,心魔这样强大,我更要勤修断念实战!
    \subparagraph{附评} 断念实战是戒色的核心,就像学生党平时努力学习,最终要面临考试一样,戒色之后也会迎来魔考,心魔会进攻你,邪念会上头,就看你的实战表现了。不少戒友很注重学习,戒色笔记也记了很多,但是实战却不太理想,原因就是:只注重学习而不重视练习断念,这样实战水平就难以得到质的提升,也容易沦为纸上谈兵,学习和练习是同等重要的,一定要坚持练习断念。一位戒友说:“戒到破戒高峰期,出现了一些念头不能断除,脑子里那个念头出现的时候,心里就咯噔一下,然后没过多久,就开始行动了,那个念头就像是一个指令,我不理念头它就是念头,我一旦认同它,就会马上着魔。”戒色之后,邪念肯定会入侵的,这正是刺刀见红的时刻,这个时候千万不能软弱怯战,否则就会一败涂地,实战时要像高手那样操作,稳健老练,快如闪电。很多发毒誓的人在发誓时雄心万丈,豪言壮语,甚至还自残表决心,但是当念头一上来,他们却没能断掉,结果就被念头附体了,变成了身不由己的撸管肉机,那个念头就是一个指令,一个侵入的木马程序,跟着念头跑就是在强化念头的力量,最终就会导致破戒。断念实战是我一直强调的重中之重,不管看多少戒色文章,最终的检验就是邪念袭脑时,你是否能做到斩立决?!你的断念速度有多快?你对念头的敏感度有多强?你的警惕性如何?戒色高手不跟念,而是断念,他们可以主宰自己的念头。还有一位戒友说:“戒色快二十天了,今天中午躺床上突然淫念袭来,内心挣扎了一下,鬼使神差地点开黄网,看了一下午,一下午就这样虚度了。”有些戒友说自己一躺床上邪念就会袭来,因为断念不力,所以很快就会沦陷,记得以前的我也是如此,早上醒来很容易意淫,躺着休息时也很容易陷入意淫,各种性幻想还有回忆,那个阶段的我不懂得断念,就是放任自己沉溺其中,后来我逐渐明白了断念的重要性,现在即使躺着休息也有很强的控制力,也不怕邪念入侵了。戒色的战斗是看不见硝烟的,而且是瞬间的事情,那一念没断掉,你就身不由己了,鬼使神差了,明知不对,但就是停不下来,像着魔一样,撸完才清醒过来。如果念头上来时你能及时断掉,那就不会破戒了,有时是非常细微的念头上来,是一种十分微妙的想看黄想撸的感觉,这就需要更强大更敏锐的觉察力来消灭之。不少戒友破戒很多次后才领悟到断念的重要性,被心魔虐了很多次后才明白前辈为何那么强调断念,前辈在实战中摸爬滚打,积累了丰富的实战经验,最后得出的结论就是:断念必须快,必须狠!高手断念,一瞬,就结束了,那一瞬,就是一记觉察,就是一看,看到念头,念头就消失了。高手的这一看,特见功夫,是长期练习才能达到的,可谓功力深厚,这一看太有力量了,也太有味道了。你怎么也想不到,这一看,威力那么大,不管多强的邪念都会灰飞烟灭,看的功夫是需要坚持练习的。你看过去,念头不见得会消失,因为你的觉察力弱,但前辈这一看,就可以让邪念瞬间消失,即使是强烈的恋物癖的念头也经不起这一看,前辈的这一看就像激光一样,可以瞬间摧毁念头怪,让邪念瞬间气化于无形。这个世间最值得修炼的就是断念的功夫,断念是一门很高深的功夫,要不断练习、思考、领悟,一层功夫一层体会,一成深一层,层层妙无穷,逐渐达到较高的境界,要练到出神入化、登峰造极的地步,真正主宰自己的内心!
\end{case}

下面步入正文。

之前写过春季、夏季和冬季的戒色养生文章,这次把秋季给写出来,也是之前一位戒友建议的,这样四季戒色就齐全了。每个季节戒色有相似之处,也各有特点,根据季节的特点来及时调整是很重要的,根据不同的季节来做出针对性的调整,这样就能戒得更稳定。

每年秋季是男人一年中雄性激素分泌最旺盛的时候,这个季节也是很容易破戒的,夏季容易破戒是因为穿得少,街上诱惑多,而秋季则和雄性激素分泌旺盛密切相关,秋天的太阳晒得人暖洋洋的,也容易导致勃起的频率增加,莫名其妙就会勃起,这时候要格外警惕,加强修心,转移注意力。夏季出汗比较多,身体无病三分虚,而秋季正好就是夏季过后的调整期,应该让自己尽快找回饱满的能量状态。秋天早晚温差大,是泌尿系统疾病高发的季节,很多戒友都反馈尿频的症状开始反复了,前列腺炎的康复需要一个过程,会经历多次症状反复,天冷时要注意保暖,加强养生。前段时间看到一位戒友发帖说自己睡凉席感冒了,进入秋季凉席应该撤下了,特别是北方,有的地方都下雪了,都有冬天的感觉了,这时一定要注意保暖防寒,要时刻注意天气变化,及时添加衣服,夜间睡觉时要盖好被子,以防着凉感冒。致病因素是很多的,自己要有很强的养生防护意识,这样才能平安度过这个“多事之秋”。

心魔的秋季攻势也很猛烈,借着雄性激素分泌旺盛时疯狂进攻,甚至有的老戒友都破戒了,如果对秋季戒色的特点有所了解,就会在秋季加倍警惕,以免被心魔攻破。雄性激素分泌旺盛时,人就会变得蠢蠢欲动,就像一个火药桶一点就炸,进入秋季会感觉很容易勃起,这时如果放任自己意淫,结果肯定会破戒。另外还要注意“秋乏”现象,有戒友反馈进入秋天戒色状态变得不太好,人很容易懈怠,学习的劲头也减弱不少。俗话说:“春困秋乏夏打盹儿”,在炎热的夏天,人体大量出汗造成了水盐代谢失调,肠胃功能减弱,心血管系统的负担加重,人的身体处于过度消耗阶段。夏去秋来,气候由炎热变得凉爽宜人,人体出汗也明显减少,人的机体进入到了一个周期性的休整阶段,水盐代谢开始恢复平衡,人的心血管系统的负担也得到缓解,消化系统功能也日渐正常,然而此时人们的身体却有一种说不出来的疲惫感,这就是人们常说的“秋乏”,其实这是不同季节人体的自然生理反应。经过一段时间的调整,秋乏现象会自然而然地消除。

知道秋乏现象后,就明白了进入秋季后为什么会感觉疲乏,提不起精神,自己注意休养,积极锻炼,适量晒晒太阳,这样身体很快就能调整过来,身体调整好了,学习戒色文章的效率也会跟着提升。这些都属于戒色的细节之处,有的人不懂秋乏,自己也不懂得调整,当出现疲乏无力时就会懒得看戒色文章,一天不看,两天不看,三天不看,慢慢就中断了,戒色状态随之一落千丈,没过多久就破戒了。戒色是系统工程,需要懂很多相关的知识,各种细节都要深入了解,到时碰到问题就会迎刃而解,这样就可以做到胸有成竹,从容应对,不会出现沮丧和慌乱。戒色状态是起伏的,就像运动员的竞技状态一样有高有低,关键自己要学会及时调整,出现问题时也要学会分析原因并找到对策。

秋天也要格外注意情绪管理,要防悲秋,《楚辞》:“悲哉!秋之为气也。萧瑟兮,草木摇落而变衰。”杜甫《登高》诗:“万里悲秋常作客,百年多病独登台。”自古逢秋悲寂寥,秋天万物枯败,一片昏黄,易使情绪受到消极影响。深秋,草枯花谢,冷风萧萧,常使人无端伤怀,容易导致情绪低落,心情抑郁不舒。预防悲秋可以从以下几个方面做起:

\begin{description}
    \item[心理调节] 保持情绪乐观,不要消极思考,须知草枯花谢乃是正常的自然现象;
    \item[生理调节] 天气晴好时,应多做户外运动,沐浴在温暖的阳光下,伸展四肢,会感觉浑身充满力量,生机无限,从而远离悲秋;
    \item[饮食调节] 饮食得当可以滋养好心情,莲藕、莲子、红枣、龙眼等食物有养心安神的作用,对焦虑、抑郁很有帮助。
\end{description}

戒色资深前辈都知道情绪管理的重要性,懂得情绪管理实在太重要了,要极力避免让自己陷入悲观消极的心态,一旦出现悲观的想法时,要及时调整,要用积极的富有建设性的想法来取代消极的念头。有的人遇见挫折了,就会产生很多消极的想法,乃至怨天尤人,而智者会从挫折中学习,挫折也是一种教育,可以加速一个人的成长,很多成功人士之前都经历过多次失败,但他们并没有消极,而是从挫折中学习,不断给自己鼓劲。戒色之后情绪应该保持乐观、积极向上,随着身体的恢复,也更容易保持乐观开朗的心态,然而其他一些因素比如天气季节因素或者生活中的一些挫折会动摇这个状态,自己一定要及时调整。生如夏花之绚烂,死如秋叶之静美,秋季很多草木都枯黄了,但其中也有蕴藏着一种美感,很多油画都是关于秋季的,你能感受到画面中的那种美好、宁静与祥和。

天凉好个秋,夏季太炎热,秋天给你开空调,凉爽的秋天,舒适的心情,会给人们带来新的希望。秋季的天空总是那么明朗,抬头仰望蔚蓝的天空,自己的内心也觉得开阔起来,听着鸟语,闻着桂花香,躺在草地上小憩,灵魂得到了很大的滋养。戒色之后要多亲近大自然,在大自然中一个人很容易放松下来,内心会变得快乐起来,你就像一片叶子、一朵小花、一颗果实一样融入秋的生命里,那些让你烦恼的俗事也会暂且离你而去。秋天是收获的季节,田野里一派丰收的景象,一望无际的稻田像铺了一地的金子,一阵风吹来,便掀起一阵阵金色的波浪,被稻风吹拂是一种莫大的享受,在那种轻柔美妙的吹拂中,你自动就进入了纯粹的真我,你会情不自禁地闭上双眼,享受这片刻的惬意。城里的孩子可能很难有这种享受,农村的孩子更贴近土地,看着农民开心幸福的笑容,不禁会想:幸福快乐是如此简单,何必沉迷于色情与手淫。

从凛冽萧瑟的景象来看,秋季是伤感的,从丰收喜悦的角度来看,秋季却是美好的。罗兰《秋颂》:“秋天的美,美在一份明澈。有人的眸子像秋,有人的风神像秋。”我喜欢静静地品味秋天的况味,静静地听一场秋雨,内心的情绪会变得很微妙,秋天,遐思无限,是回忆的季节,秋天,像是世界突然意识到自己该安静下来了,有着某种反思人生的意味。每逢走在路上,闻到秋天的桂花香,那淡郁的清香飘逸开来,总感觉周遭都安静了,呼吸着它只觉得幸福满溢。在林荫路上徘徊踱步,落叶纷飞,全部落在了回忆里,时过境迁,让人有一种超然的感觉,童年的伙伴们你们现在还好吗?还记得你们纯真无邪的笑脸,在阳光中欢快奔跑的身影,大家在课堂上郎朗的读书声,转眼间,你们已经为人父母,过上了另外一种生活。秋天的肃杀,秋天的悲凉,秋天的天空高远纯净而明朗,秋天的色彩缤纷而绚丽,秋游的快乐,秋雨的落寞,秋天是很值得回味的一个季节,戒色来到了秋天,人生进入了诗意。

下面是秋季养生的一些要点。

秋季,气温开始降低,雨量减少,空气湿度相对降低,气候偏于干燥。秋气应肺,肺为娇脏,喜润而恶燥,主气司呼吸,而秋季干燥的气候极易伤损肺阴,从而产生口干咽燥,干咳少痰,鼻燥出血、皮肤干燥、便秘等症状,所以秋季养生要防燥,宜多摄入一些水,多吃一些清润的食物。秋季,在燥气中还暗含秋凉,人们经夏季过多的发泄之后,机体各组织系统均处于水分相对贫乏的状态,如果这时再受风着凉,极易引发头痛、鼻塞、胃痛、关节痛等一系列症状,甚至使旧病复发或诱发新病。老年人和体质较弱者对这种变化适应性和耐受力较差,更应注意防凉,作为伤精患者,很多人对于天气变化是很敏感的,温度下降后,一些伤精症状就开始反复了,自己要注意保暖,不要贪凉,夏季的凉席撤了,空调也不要开了,电扇也收了,季节转换时要格外注意,秋季洗澡出来时要穿袜子,否则脚部着凉容易诱发严重感冒,养生要注意细节。

\begin{quote}\it
    秋三月,此谓容平。天气以急,地气以明,早卧早起,与鸡俱兴,使志安宁,以缓秋刑,收敛神气,使秋气平,无外其志,使肺气清,此秋气之应,养收之道也;逆之则伤肺,冬为飧泄,奉藏者少。(《黄帝内经》)
\end{quote}

译文:秋季的三个月,谓之容平,自然景象因万物成熟而平定收敛。此时,天高风急,地气清肃,人应早睡早起,和鸡的活动时间相仿,以保持神志的安宁,减缓秋季肃杀之气对人体的影响;收敛神气,以适应秋季容平的特征,不使神思外驰,以保持肺气的清肃功能,这就是适应秋令的特点而保养人体收敛之气的方法。若违逆了秋收之气,就会伤及肺脏,使提供给冬藏之气的条件不足,冬天就要发生飧泄病。

容平:从容不迫,平和,是从容平和不急不躁的状态。为什么秋天要突出容平呢?因为春生、夏长,到秋该收了,秋天就很从容地去等待收获了,是从容不迫、内心非常安定的状态,不是那种急迫的感觉,而是很淡定,很从容,之前的工作都完成了,就等着丰收了。

每一个季节都和下一个季节是紧密相关的,夏天的时候比较放纵,到了秋天身体也许就会爆发严重的症状,如果秋天放纵,到了冬天身体也许就会崩溃,如果冬天连续破戒,到了春天身体可能就会突然垮掉。你不会想到曾经健康的身体会病到这么严重的程度,失去健康后才懂得珍惜,真的不能放纵自己。每个季节都是一环扣一环,这个季节没做好,就会影响到下一个季节身体的健康。有的人夏季破戒时身体感觉还行,但是到了秋天突然就不行了,一下出来好多症状,自己都吓怕了,这就像还债一样,夏季欠的债,秋季来还。

秋季调养主要从以下四个方面来进行。

\paragraph{精神调养}

要做到内心宁静,神志安宁,心情舒畅,切忌悲忧伤感,中医理论认为,秋天对应的脏器是肺,悲伤容易伤肺。情绪要乐观,抛开一切烦恼,避免悲伤情绪。每个季节都要注意保持心平气和,而秋季容易出现悲秋之感,所以要及时调整情绪,避免悲观伤感的心态,培养兴趣爱好,可以听一些抒情柔缓的音乐来调节情绪,积极化解不良情绪的负面影响。应多参加有益的社交活动,多与别人交谈沟通,结交知心朋友,培养豁达乐观平和的心态,知足常乐,避免过度紧张思虑,以免忧思伤脾,注意培养幽默感,可以看一些喜剧小品等,听听轻音乐。行走时须慢而稳健,站立时须身定而恭,坐时须端正挺直,说话声音温和,一举一动都须端庄祥和,闲适安泰。遇事要冷静,正确对待顺境和逆境,顺不妄喜,逆不惶馁,自觉地养成冷静沉着的处事方式。平时多行善积德,语善、视善、行善,善能升阳,多发感恩之心,多说鼓励人、激励人、体贴柔和的话,为善可以调节自己的心理,也可以让内心幸福喜悦,对于精神调养很有帮助。

\paragraph{起居调养}

秋季起居应“早卧早起,与鸡俱兴”,早卧以顺应阳气之收敛,早起使肺气得以舒展,且防收敛之太过。秋季应做到早睡早起,以应秋候,注意添加衣物,防止因受凉而伤及肺部。早睡以敛肺气,正符合人体需求,又有安睡的条件——天气凉爽,经过一个少眠的夏天,正好借此补偿。秋季早睡对于人体保健是很有好处的,符合养收之道。戒色之后应该懂得养生,养生的知识对于身体的恢复是至关重要的,不少戒友就因为对养生很无知,所以导致恢复不太理想。懂得养生的道理,就可以按照养生的智慧来操作,把身体调整到一个比较好的状态。养功非常关键,前辈在戒色后也会遇见各种问题,但前辈能够及时调整过来,这种调整能力十分重要。

\paragraph{饮食调养}

《素问·脏气法时论》说:“肺主秋,肺收敛,急食酸以收之,用酸补之,辛泻之”。意思就是说,酸味食物能收敛肺气,而辛辣食物发散泻肺。秋天宜收不宜散,所以要尽量少吃葱、姜等辛味之品,适当多食酸味果蔬。所以,秋季常吃梨、番茄、柠檬、乌梅、葡萄、山楂、石榴、猕猴桃等酸味水果可养肺,还能达到止泻祛湿、生津解渴、健胃消食、增进食欲的作用。秋时肺金当令,肺金太旺则克肝木,秋季燥气当令,易伤津液,故饮食应以滋阴润肺为宜。《饮膳正要》说:“秋气燥,宜食麻以润其燥,禁寒饮。”秋季时节,可适当食用芝麻、糯米、粳米、银耳、豆腐、百合、莲藕、蜂蜜、菠萝等柔润食物,以益胃生津。秋天早晨多吃些粥,既可健脾养胃,又可带来一日清爽。秋天常食的粥有:山楂粳米粥、鸭梨粳米粥、白萝卜粳米粥、杏仁粳米粥等。我国民间素有“十月萝卜小人参”的说法,萝卜古称芦菔或莱菔,营养丰富、甜脆可口,所含的维生素 C、维生素 B2 及钙、磷、铁比苹果、梨还高,《本草纲目》记载,萝卜生吃可止渴消胀气,熟吃能治疗感冒、消化不良、胃酸腹胀、胸闷气短、咳嗽痰多等疾病。值得注意的一点是,秋季瓜果成熟,不可贪食过度,以免损伤脾胃。

《神农本草经》中说,芝麻能够“补五内、益气力、长肌肉、填精益髓”。芝麻中的维生素 E 能防止过氧化脂质对皮肤的危害,使皮肤更加白皙光泽,其养血的功效对皮肤粗糙、干燥也有很大的改善作用。中医认为,栗子营养丰富是补肾健脾的佳品,据测定,栗子里面含有丰富的蛋白质、脂肪、胡萝卜素、维生素、矿物质等多种营养物质,其中栗子中含碳水化物为 70\% 左右。我国历代医学家均把栗子看成是益气、健脾、补肾、强身的最佳滋补品。汉朝医学家陶弘景说:“栗益气、厚肠胃、补肾气,令人耐饥。”唐代医学家孙思邈说:“栗,肾之果也,肾病宜食之。”《本草纲目》中记载:“花生悦脾和胃、润肺化痰、滋养补气、清咽止痒。”《药性考》中也说:“食用花生养胃醒脾,清肠润燥。”可见其对脾胃失调、咳嗽气喘、贫血、便秘、肠燥等都有很好的治疗作用,是一种老少皆宜的食品。红薯被人们称为“土人参”,是一种味美价廉的长寿食品。其味甘性平微凉,能够生津止渴、润肺滑肠、补脾益胃、通利大便,还具有抗癌作用,但红薯不可一次食用过多,否则会出现肚胀等症状。

\paragraph{运动调养}

秋季是运动锻炼的大好时机,可根据个人情况选择不同的运动项目进行锻炼,如爬山、慢跑、快走、骑行、羽毛球、篮球、乒乓球、跳绳等,进行适量的运动锻炼以提高机体抗病能力,减少疾病的复发,促进身心健康。锻炼时不宜一下脱得太多,应待身体发热后,方可脱下过多的衣服;锻炼后切忌穿汗湿的衣服在冷风中逗留,以防身体着凉。记得以前有一次我打球间歇坐在场边,后背被冷风一吹,回家腰就不舒服了,有点直不起来的感觉,休息好几天才缓过来,天冷锻炼要格外注意保暖,特别是有冷风的天气,一定要多注意。由于人的肌肉韧带在气温下降环境下会反射性地引起血管收缩,关节活动度也会减小,因而极易造成肌肉、韧带及关节的运动损伤。因此,每次运动前一定要注意做好热身活动,避免造成运动损伤。

运动量也不宜过大,以防出汗过多,阳气耗损,自己要注意控制运动强度,并不是锻炼得越多就越好,关键是适量锻炼,锻炼时觉得自己的身体有些发热,微微出汗,锻炼后感到轻松舒适,身体恢复也很快,这就是效果好的标准。相反,如果锻炼后十分疲劳,休息后仍然身体不适,那么运动量可能过大了,下次运动时可以减少一下运动量。之前有的戒友拼命锻炼,每天出很多汗,结果适得其反,身体恢复很不理想,中医讲到:汗者,精气也。能量也会通过汗液的形式耗损,锻炼时微汗即可,要严格控制出汗量,如果出汗量过大,身体就很容易疲劳,对养生恢复很不利。严重的伤精患者应该以养生功法为主,比如八段锦、六字诀、养生桩或者静坐等,平时可以散散步,等身体恢复差不多了,再去做强度大一点的运动。身体虚弱时要注意静养,不可盲目运动,否则会影响身体恢复。

\paragraph{总结}

秋天是万物成熟收获的季节,也是人体阳消阴长的过渡期。春夏养阳,秋冬养阴,秋季天气干燥,秋季养生要注意养阴。秋天养阴,第一,要多喝水,以补充夏季丢失的水分。第二,多接地气,秋季我们要多走进大自然的怀抱,漫步田野、公园,让心情舒畅,这有助于养阴。秋季欲望比较强,和雄性激素分泌旺盛有关,一天当中欲望比较强的就是早晨,还有就是午睡醒来,哪些时间段欲望比较强,自己要做到心中有数,到时就可以提高警惕,有效化解欲望。过了秋季就是寒冷的冬季,戒好秋季就是在为冬季戒色做好充分的准备,冬藏精,冬季戒色也是很大的考验,就看你是否藏得住。四季戒色,每个季节都各有特点,根据我自己的体会,我个人比较喜欢春秋两季,春天生机勃勃,秋季象征着成熟和丰收,夏季过于炎热,出汗多消耗大,冬季寒冷,身体容易感到不适,恢复也会减慢。季节转化之间也易于出现症状反复,我们应该对四季戒色有较深的了解与认识,在出现情况时就能及时做出针对性的调整。

秋天是意境的写照,清冷、萧瑟、寂寥、悲凉,风萧萧兮易水寒,壮士一去兮不复还!秋天高远澄明,苍茫壮阔,层林尽染,色彩缤纷,大雁南飞,丹桂飘香,景色宜人,有一种繁华落尽见真淳的境界。我喜欢秋天的肃杀,这象征着断念的狠劲,断念的严厉果断,毫不妥协!我喜欢秋天的天空,云淡风轻,有一种高远的感觉,大雁排成“人”字形缓缓地飞向远方,那种意境特别美好,有种回到童年的穿越感。我喜欢秋天的稻田,一阵凉爽的秋风吹过,稻子随风摇摆,像金黄色的海洋掀起一道道波浪,那种吹拂可以让你瞬间临在,瞬间进入真我,幸福和喜悦在心中洋溢开来,纯真是永远不变的情怀。

下面分享一首戒色诗歌。

\begin{poem}[秋风中的对决]
    \begin{multicols}{4}
        \begin{center}~\\
            秋风肃杀 \\ 黄叶满地 \\ 飘散的落叶里 \\ 站着一位戒者 \\ 他久久不动 \\ 就像一座雕塑 \\ 被安在了那里 \\ 秋天特有的光影效果 \\ 为这场对决准备了 \\ 最好的场景 \\ 四周寂静无声 \\ 一切都仿佛凝固了 \\ 此番对决不知何时开始 \\ 何时结束 \\ 一切都在等待中 \\ 这种等待充满了爆发力 \\ 就等石破天惊的那一下 \\ 突然他笑了 \\ 一抹微笑挂在嘴角 \\ 对决已经结束 \\ 每次战胜心魔 \\ 他都会感到喜悦 \\ 就在刚才一幅图像 \\ 冲上了他的头脑 \\ 只一瞬 \\ 图像消散于无形 \\ 他胜了 \\ 秋风吹起一堆落叶 \\ 他迈步走向远方 \\ 留下一个坚定的背影
        \end{center}
    \end{multicols}
\end{poem}

下面推荐一本书。

\begin{book}[《禅不是修出来的》,丁愚仁]
    泰山禅院创办人丁愚仁老师,之前我文章里引用了丁师的几段话,大家应该都有印象,那几段话真的戳中了修行的要点,很有见地。丁师一九四六年出生,祖籍山东泰安。读过六年小学,电业局工人,现已退休,一九九四年在山东万德大灵岩寺皈依,法号悟达。在泰山脚下的的泰安市三合村开设“泰山禅院”,供人参禅闭关,讨论佛法人生。丁师是一位真正的高人,给人的印象十分质朴,亲切和蔼,和六祖慧能有几分相似。我初读丁师的开示就觉得非常好,书中的一句话:“一个自由人是心的自由。”我对这句话很有感触,以前我邪淫时,觉得想看黄就看黄,想撸就撸,这多自由啊!后来我发现我的内心其实很不自由,被邪念给束缚住了,沦为了心魔的傀儡。戒色之后我感觉我的内心很自由很轻松,这也是很多戒友的共同感受,戒色后内心真的自由了,如释重负,有一种摆脱了的感觉。禅就是本来面目,就是真我,是本具的,的确不是修出来的,所谓修心,就是去掉覆盖物,就像洗去珍珠表面的污泥。这本书内容很丰富,道理很有深度,对修行感兴趣的戒友,不妨读一读。
\end{book}
