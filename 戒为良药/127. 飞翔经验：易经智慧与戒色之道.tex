\subsection{易经智慧与戒色之道}

\paragraph{前言}

上季一位资深戒友说:“结合我自己戒色的体验,这十条经验真是好全面啊,真是一点不漏啊,做好这十方面,想不戒掉都难。”戒色十规每一条都是精华重点,每一条都讲到刀刃上了,希望大家好好落实。根据我的经验和戒友的反馈,难的不是学习,难的是吸收和落实,有的人也记了很多戒色笔记,但吸收率太低,不注重复习;有的人也看了不少戒色文章,但不能做到知行合一,有知无行,那还是等于零;有的人夸夸其谈,自诩懂了很多戒色理论,但当邪念上头时,他还是献上了膝盖和奴颜!我到现在依然在复习前几年做的笔记,我很注重复习笔记,温故而知新,指不定复习到哪条笔记就突然“炸”了,一下就顿悟了,这时才真正吸收到“髓”,之前吸收的都是皮毛,懂得很浅,甚至是假懂,似懂非懂,吸收到髓是很不容易的,需要不断复习、思考和总结。我们一定要把戒色十规的精华重点一条条去落实和执行,每天坚持,注重积累,持之以恒,必须立足于断念实战,不能纸上谈兵。不管戒色的过程是多么困难和坎坷,只要我们不放弃,坚持到底,一步一步坚实地向前走,量变终究会引起质变,胜利终将会属于我们。

下面分享几个案例。

\begin{case}
    飞翔老师,就是我现在在南方城市读大学,这边天气热,经常有些女的穿得非常清凉露腿,就非常把持不住,人一多白骨观就不见效,就连出门跑步都是一堆腿精召唤心魔来攻击我,飞翔哥我该怎么做呢?

    \textbf{附评} 对境实战很考验一个人的定力和实战意识,非礼勿视,一定要做好视线管理。实战四个字:外避内断!避开诱惑,管住视线,不聚焦,不盯着看,不停留,然后要加强修心,严格做到念起即断。色不迷人人自迷,大白腿本来不吸引人,关键是自己迷了,母猪的腿比人的更白,你看着有感觉吗?到了非洲,那不是大黑腿?母猴子岂不是大毛腿?不管是什么腿,本来不吸引人,因为你觉得它好,才会造成迷恋。不少戒友都有恋癖的问题,我以前也有,而且也很迷恋,后来戒色后悟到之所以迷恋,就是因为起了分别念!就是因为自己觉得那个部位好、性感,或者觉得女性衣物好、性感,就是这种分别念在作怪!一旦有了这种分别念,肯定喜欢自己认为好的,排斥或不喜欢自己认为坏的,对好的极度迷恋,对坏的一点兴趣都没有。一旦强化了这种分别念,到时遇见相关的对境,就会一点抵抗力都没有,一些恋癖的戒友都问该怎么办?首先要改变观点,不要认为它好,不净观就是让你觉得身体脏,事实的确如此,人人都有一包屎,毛孔里也藏污纳垢,几天没洗澡就能搓出泥来。多思维不净观就能有效对治贪恋,然后要及时断除那种分别念,因为过去一直迷恋,所以会形成强大的惯性,看到对境,内心就会觉得好,觉得性感,然后蠢蠢欲动,那种想法一旦冒出,必须马上断除,久而久之,就渐渐淡化了。我以前对人体的某些部位也非常迷恋,看到那些部位就欲罢不能,一点抵抗力都没有,两只眼睛一直盯着看,充满着贪婪的欲望。我潜意识中觉得它好,觉得它性感,而且还有其他一些分别念的词汇在不断强化这种倾向。上次聊过一位戒友,他对女性某个部位有恋癖,然后他说了一个词,那个词我这里就不提了,那个词就是一个分别念极强的词,他脑袋中有这个分别念,他就很难摆脱恋癖,他一次次在起那种分别念,就是在加重恋癖的倾向。大德说过:“分别易入魔。”一分出好、坏、高、低、美、丑等,就很容易陷进去,一定是喜欢自己认为好的,然而每个人的标准都不尽相同,也许你认为好的,在另外一个人眼中也就一般了。有的人见了丝袜就不行,如果劫匪套着丝袜闯进你家里,你什么感觉?你还会觉得性感吗?不管什么恋癖,其根源就在于你觉得它好,觉得某个部位好,或者女性衣物好,所以才会极度迷恋和贪求。反过来去思维,觉得它一般,没什么好,不过如此,也就能摆脱这种倾向了。就像收藏者对一件藏品爱不释手,对其有很多好的分别念,和别人聊起来,就说这件藏品怎么怎么好,然而在其他一些人眼中,所谓的藏品并没有多大吸引力,因为他们没那种分别念。遭遇诱惑时,不分别其实是一种很高的智慧,这是我后来才逐步悟到的,看到任何诱惑,都不要分别执著,就像没看到一样,不分别,不执著,视而不见,心如止水。
\end{case}

\begin{case}
    尊敬的飞翔大哥,您好,现在我终于体会到您说的“觉”了,看见即是消灭、看见即是降伏,您说得一点都不错。当会觉了,层次立马就上去了。前天早晨我一觉醒来,脑海中忽然想起水中月、镜中花、空,这三个名词。我下意识地问自己,佛说的水中月、镜中花、空到底是什么?此时忽然一个负面念头劫持我,我突然明白了,我问自己那个念头在哪里,我试着去找那个念头,我发现找不到,但是它不是不存在,那种感觉就是水里的月亮、镜子里的花朵,我开始知道佛陀说的是真的,佛陀没有骗人。从那一刻起,我就会觉了,邪念上来了,只需要一觉,它自动消失,根本不需要去拿它怎么样,它本来就是空的,当你发现它的时候,它就已经消失了。希望戒友们都可以尝试着去了解念头,去发现它,其实它不可怕,与心魔斗,其乐无穷!

    \textbf{附评} 这是 13 级老戒友的反馈,他这次领悟真的非常棒!当会觉了,断念实战就会变得简单,层次一下就上去了,会觉,就是登堂入室了,再继续不断深造,将来就有可能登峰造极。断念其实是一门技术或者技法,就像学钢琴、学二胡、学车、学打字、学游泳等,是可以学会的,然而每个人的悟性和勤奋度有所不同,所以掌握的速度就存在很大差异。有的人学了很久都没学会,但是某天突然开窍了,他也就会了,会了之后继续加强练习,对内心的统治力就会越来越强大。我以前也不会觉,也不懂觉,后来看了大德开示,很快就领会了觉,才发现觉的力量是这样强大,向内看,向内觉察念头,当看见念头时,念头就消失了,念头无法承受这一觉,这一觉威力十足,就像激光炮一样消灭念头怪!拉玛那·马哈希(Ramana Maharshi)说过:“真我是观察者。”观察可以消灭念,念头消失了,有的根器高的人就顿悟了真我,念头像幽灵一样出现和消失,一直在冒充你,而真正的你不是念头,而是观察本身。念头或图像有一种强大的特性,这个特性就是——“把你带跑”!稍微不觉知,就会把你带跑很远,等你回过神来,才发现自己已经跟着念头跑了那么久。念头来源于空,最后消融于空,念头会出现,而你可以让其消失,当你真正学会观心断念了,你就有这个主宰权。断念不是压念,压念是思想误区,压念就是有一种试图压制念头的想法,这样会适得其反,越压越反弹,而真正的断念只是一觉,很轻松,断念高手不怕念起,而压念的人很怕念头起来,所以尝试去压制,结果只有挫败感。《真心直说》里面说的十种功夫,第一种就是“觉察”!“谓做功夫时,平常绝念提防念起,一念才生便与觉破。”“故祖师云,不怕念起只恐觉迟。”觉察是修心最重要的功夫,首屈一指的功夫!希阿荣博堪布:“心的训练可以从觉察开始,尽量清晰地了知自己身心的活动,言语、举动、感受、情绪、心念的生灭变化,久之,心会变得安静而敏锐,并且对苦、无常等生出切身的领悟。”你不会想到,一切的奥秘就藏在“看”中!看消灭念!看就是觉,觉就是知,知就是照,照就是观,观就是看。丁愚仁老师强调的是“知”,知道就行,知道念头,念头就消失了。元音老人强调的是“觉”,一觉即灭,一觉即空。最怕的就是不“知”不“觉”,跟着念头跑。觉知是两个字,分开来讲,会让人有更深的理解和领悟。看、觉、知、照,这四个字都很好,每个字都各有特点,但指向的意思都是同一个。学会觉了,绝对是里程碑式的大事件,更进一步,认识到纯粹的觉知就是自己,那更是开天辟地的大事件。
\end{case}

\begin{case}
    飞翔老师,我今天差点破戒了,还好悬崖勒马,止住了堕落的行为。我已经戒了两年多了,以为不会再破戒了,所以就放松了警惕,甚至开始贪恋念头了,对诱惑擦边内容也是不以为然,甚至有的时候还点击进入看,总在自欺欺人地对自己说:“就看看而已”,就是这样地放任,所以今天下午躺在床上玩手机,打开了微信的看一看就遇到了诱惑的袭击,就鬼迷心窍般地多看了一眼擦边内容,真的是第二眼就着魔啊!就这样被它带走了,虽然自己的脑海里还有反抗之意,但是已经无法自拔了,一步一步地堕落,胃口越来越大,都能清晰地感觉到肾精在往下漏,睾丸已经开始难受了,但还是止不住,差点就去网站搜索那种视频了,还好悬崖勒马,马上从床上爬起去洗了个冷水澡,但是心魔还不肯就此罢休,还在怂恿我说:“既然都已经漏了,还不如爽一把……”还好我没有听它的,我记得您说过:“无论发生什么事情都不能破戒,不撸是最后的底线。”心魔的力量实在是太强大了,真的不能小看它!我以后一定要远离一切诱惑,哪怕擦边的内容都不会再去看了!

    \textbf{附评} 这是 14 级的老戒友“心维天际”的反馈。他这次实战教训很深刻,一般破戒前都有一些征兆,比如放松学习、放松警惕、沉迷网游、沉迷玩手机、总是有意无意接触黄源,特别是擦边图和擦边新闻、文字等,德行跟不上,怨天尤人,嗔恨心重,负面念头多,骄傲自满,情绪不佳,还有就是贪恋意淫不肯断,出现这类征兆时就要引起高度警惕了,一定要注意及时调整。现在这个时代色情低俗的内容极度泛滥,很多网站和平台都有这类擦边内容,吸引人去点击,非常之多,甚至可以说满天飞,铺天盖地。戒色十规第五条视线管理:对境实战时要做好视线管理,看到诱惑马上避开,不聚焦、不停留、不回看。戒色十规专门强调了视线管理,看到诱惑一定要注意避开,不要去看第二眼,不要试图看清,往往第二眼就陷进去了,第一眼没看清,然后第二眼聚焦一看,就陷进去了。古希腊神话中看美杜莎一眼就会变成石头,这个神话可以给对境实战很好的启示,诱惑图片有一股魔力,看第二眼就很容易着魔。我们对于擦边内容一定要有很强的警惕心,“心维天际”对诱惑擦边内容不以为然,有这种想法和态度就危险了,擦边内容的杀伤力也是非常大的,一定要警惕擦边内容,很多戒友都是倒在擦边图上。“心维天际”戒了两年多了,时间长了容易放松警惕,这个问题要引起足够重视,要有危机感,不管戒多久,都要注意保持警惕,警惕的螺丝不能松,否则戒色大厦会轰然倒塌!“心维天际”毕竟戒了两年多,戒色的觉悟还是有的,他最后悬崖勒马,并且没有听信心魔的怂恿,这点做得很好。不撸是最后的底线,有时做得不好,会沉迷意淫甚至去看黄,严格来说已经算破戒了,但如果这时悬崖勒马没有撸,就要按照“不撸是最后的底线”来要求自己,因为一旦悬崖勒马,心魔就会拼命怂恿你,说你漏了,说你已经破戒了,这时候就要祭出“不撸是最后的底线”,这样就能避免撸管。

    最近资深戒友“禅行僧”在帖子里说:“那天上午起床,欲望就出现了,忽然有个念头想看看网络上的帖子,当时心里就想到,网络上的帖子污染太严重,还是不看了。忽然心里又想到,就看看帖子,又不会破戒,如果真的有污染就关掉。然后就打开手机在网上看帖子,刚一看就发现这网络上污染太严重了,心里就感觉反感。之后要去关闭的时候,发现有个图片没看清,然后又去看了一下,发现是诱惑的擦边图。立刻心里就起了强迫念头,非常严重。那时心里还有欲望,又出现强迫的念头,心里就有点慌,手机也没有关,在伏强迫念头的时候,发现下面硬起来了,那时就想硬起来应该没什么,就没在乎,结果刚硬起来几秒钟就发现有要射出来的感觉,当时我非常惊讶,怎么会出现这种情况,于是我立刻就跳起来了,不过精子还是出来了。”禅行僧也戒了两年多了,他和心维天际一样,对心魔的某一类怂恿没能识别出来,导致听信!这类怂恿就是“就看看而已”、“就看看,又不会破戒”、“就看看,不撸”、“看看没事的”。这是非常狡猾的怂恿,虽然说就看看,但是看着看着就失控了,开始看越来越暴露的图片或视频,一步步失控。这两位资深戒友的教训很深刻,对境时没看清、多看了一眼,都是第二眼陷入。刚开始则是听信了心魔的怂恿“就看看”,他们两个人的经历如出一辙,即使是戒了两年多的资深戒友,也可能马失前蹄,就因为一次实战没做好,就可能沦陷。心魔实在很狡猾,要学会识破,不要听信,对境时更要加强警惕,严格管住视线。我们一定要吸取这两位资深戒友的教训,心魔一直在虎视眈眈,我们必须警惕,实战时要做正确的事!不看第二眼,避开!!!

    禅行僧还有一个问题,那就是强迫念头,他小时候就有强迫症,强迫的念头比较严重,难以控制,后来戒掉手淫之后,这个毛病还依然存在。首先要明确是什么强迫念头,这样才有利于解开心结。我那时也有强迫思维,后来慢慢就克服了,我那时强迫疑病很严重,不能看到绝症的字眼,看到那类字眼我就会很不舒服,很紧张,一直强迫焦虑,一会担心这,一会担心那,惶惶不可终日。妄想、强迫、疑病、恐惧、焦虑、抑郁我都有,而且在那个阶段还特别严重,每天都很绝望,深深的绝望,就像死刑犯眼中流露出的绝望。后来坚持戒色养生,身心恢复很多,心理失调也随之大大缓解了,那是脱胎换骨、焕然一新的感觉,真的是重生了。但有时还会出现强迫思维,我一般不去跟随,安住于纯粹的觉知,这样强迫的倾向就会越来越弱化。禅行僧的强迫表现就是看到诱惑图片就感觉很恐怖,这让他心里非常难受,因为他有思想洁癖,觉得诱惑图片弄脏了他的思想,这让他纠结和紧张。我们看到诱惑图片要马上避开,然后要保持心平气和,恐伤肾,经常处于恐惧的心理状态,对戒色很不利。断除邪念就能让内心恢复干净,修心即是净心,不要纠结于脏,否则就会出现过度紧张和敏感,内心也会陷入慌乱。其实不必怕脏,只需做好断念即可,念头断了,内心自然就清净了。禅行僧虽然有较高的觉悟,但自身的一些问题还有待克服,实战表现还需继续强化,希望他越戒越好。
\end{case}

\begin{case}
    戒色以后,考上了公务员,我从十三岁开始邪淫,一直到大二,中间七年简直是人间地狱,不堪回首。体弱多病、抑郁症、焦虑、强迫症,脸上都是痘痘,负能量爆棚,心性冷漠,自私自利,嫉妒,狭隘……心悸,最后撸到感觉自己随时会死!万幸自己在大二遇到了戒色吧,幡然悔悟,再也不想似以前一样地活着了,再也不想人不人鬼不鬼一样了。于是开始戒色,刚开始的时候也是不顺利,会经常破戒,但是我没有放弃,因为邪淫的痛苦太强烈了,因为邪淫我失去了太多太多。随着戒色的深入,我的身心健康也在很快地恢复,整个人身心疾病都在消失,变得有了自信,有了喜感,人生也变得平顺。考六级,计算机二级、三级,普通话测试都很顺利,搁以前都不敢想象的。大四那年专心准备了三个月的国考,顺利上岸,回到家乡,成了一名中直机关的干部,工资在当地也算很高的了。说了这么多就是想鼓励师兄们,只要好好戒色,是可以挽回生命的颓势的,是可以重遇全新的自己的。加油吧,师兄!

    \textbf{附评} 这是一个逆袭的案例,很给力!通过戒色,这位戒友恢复了身心健康,有了自信和喜感,运势也开始扭转了。邪淫的人之所以运势差,就是因为振动频率太低,负能量太重,邪淫会降低自身的振动频率,吸引不好的事物,宇宙中的每件事物都在振动,所以你的振动频率会吸引来相似的人、事、物与环境,要提高自己的振动频率而吸引美好的事物前来,人生的奥秘就在于提升自己的振动频率。怎么提升?戒色、修善、改过。负能量的事情不要去做,正能量的事情要多做!很多戒友在戒色后都有一种感觉,那就是轻松和轻盈,走路都能蹦起来,为什么会有这样的感觉?就是因为振动频率上去了,邪淫之人容易感到沉重,不开心,甚至愁眉苦脸,这是振频下降的表现。孩子内心纯净,自带喜感和纯真美好的气场,纯净之人的振频是非常高的,所以内心会很喜悦、轻盈,有一种幸福感。这位戒友戒色前负能量爆棚,负能量的状态肯定是自私自利的,充满着各种负面的想法,那是负能量的模式,一位戒友曾说:“记得在邪淫后,我开始感到自己变得更自私了,其实手淫本身就是一种自私,就是为了自己爽,满足自己自私的贪心。”他说得很有道理,邪淫会让人进入非常自私狭隘的状态,疯狂看片疯狂撸,就是为了满足自己自私的贪心,负能量会变得非常重,整个人就像一个“霉气罐”,散发着让人不舒服的负能量。戒色唤醒了深埋在我们心中那些美好和积极的力量,让我们重新看到了人生的意义和希望。多少次,在射掉那泡液体后就感觉人生变得毫无意义,仿佛整个世界都变得灰暗了,深深的空虚感、失落感和挫败感,还有深入骨髓的无力感。有时撸到射不出,还不死心,还撑着两条摇摇晃晃的双腿在做最后的冲刺,真可谓丧心病狂,那种状态完全是病态的。这位戒友戒色后逆袭了,考上了公务员,成了机关干部,试想如果他不戒色,也就没有那么好的身心状态来备战考试,那考运肯定会大受影响,戒色改变了他的命运,挽回了生命的颓势,真的非常给力。那些沉迷邪淫的人该醒醒了,不要亲手毁了自己的人生,让戒色开启人生最伟大的篇章!
\end{case}

\begin{case}
    老中医的话触目惊心,老中医 74 岁了,其中的一些话真的骂醒我了。老中医说:“手淫这个东西必须戒掉,它就是犯罪,如果你手淫了,那除非是你自己觉得自己寿命太长,你想早点上西天。那东西真的是要人命的,将来身体一定不会健康,没有了健康什么工作也做不了,将来怎么养自己的父母?”总之,老中医把我痛骂了一顿,大家真的要好好戒色啊!那是通往地狱的路。

    \textbf{附评} 74 岁的老中医,从医经验和阅历应该相当深厚了,他的这番话振聋发聩!手淫这个恶习伤身败德,高度成瘾,是该彻底戒掉的,虽然砖家说手淫无害,其实有害无害,事实会给出答案的,事实胜过雄辩,到时自然就知道了。之前我的文章分享过好几位老中医劝戒手淫的案例,我觉得老中医真的是一种很强大、很有正气的存在,和抗战老兵一样,有一股铁骨铮铮的正气,而他们说出来的话真的是掷地有声,铿锵有力,每个字都能砸出一个坑。老中医都是爷爷辈的,资历和级别都极高,他们的意见极具参考价值,他们的劝诫融合了他们一辈子的阅历和经验,我们应该认真听取他们的劝诫。手淫必须要戒掉,说它犯罪好像有点过,但要看从什么角度来理解,手淫是对自己身心健康的犯罪,“身体发肤,受之父母,不敢毁伤,孝之始也”。手淫伤肾,肾为五脏之根,把根本给伤了,何止发肤?父母做了那么多好吃的给我们吃,而我们却偷偷躲进房间里猥琐地掏空自己,这太不应该了。老中医说:“那东西真的是要人命的。”这句话一点不夸张,不少人都撸出了濒死感,上次一位戒友分享了一个视频,点开视频就是一位撸者猝死的场景,应该是抬到了殡仪馆,家人哭天抢地,孩子还很年轻,看上去也就二十出头,一只手还抄在内裤里,内裤上有一滩精斑,人已经发青了……有很多小动物在长时间交配后都会猝死,人其实也是一样的,耗损太厉害了,生命就可能戛然而止,零件都是好的,但是可供运行的能量却没了,大家都有手机,手机一直用,用到最后电没了就自动关机了。现在年轻人猝死的事情是很多的,原因有熬夜、纵欲、劳累、疾病等,纵欲是其中一个很重要的因素。中医很注重保精养生,肾精就是人体的核能,两个肾脏就像核反应堆,保住肾精才有核动力!反之,如果不懂保精,沉迷色情,滥撸滥泄,不断掏空自己,最后肯定沦为病夫、废人,健康没了,一切都陷入了困境。老中医的金刚棒喝真的把这位戒友震住了,骂醒了,也教育了我们,给这位老中医一万个点赞!手淫之陋习,必须痛下决心狠戒之!
\end{case}

下面步入正文。

这季谈下《易经》对戒色的启示,《易经》是我国最早的一部哲学著作,是经典之中的经典,《易经》是中国最古老的文献之一,是一部非常古老而深邃的经典,是华夏五千年智慧与文化的结晶,被誉为“群经之首,大道之源”。在中国古代思想史上占有非常重要的地位,它不仅对先秦诸子百家产生过巨大影响,而且对中国古代的哲学也产生了巨大的影响。可以这样说,《易经》是中华文化的根基,也是中国哲学的源头。《易经》含盖万有,纲纪群伦,是中国传统文化的杰出代表;《易经》广大精微,包罗万象,亦是中华文明的源头活水。其内容涉及哲学、政治、生活、文学、艺术、科学等诸多领域,是儒家、道家共同的经典,中华民族的智慧之源在《易经》。

孔子在晚年读《易经》爱不释手,如痴如醉,把编《易经》简册的牛皮线绳都磨断了三次,“韦编三绝”即出典于此。他还说“假我数年,五十而学《易》,可以无大过矣。”可见,孔子对《易经》推崇备至。孔子研究《易经》之后,发现了《易经》的重要价值,并把它列入“六经”,使《易经》升华为经典著作,从此登上大雅之堂,对《易经》的保存和传播起到了重大作用。

学习传统文化肯定要学习《易经》,《易经》文义古奥,被称为上古奇书,由伏羲、周文王、孔子等圣人所传,为中华民族古代的智慧结晶,揭示的是事物发展变化的自然规律,指明的是人生处事的智慧和法则,它上论天文,下讲地理,中谈人事,博大精深,真的非常深奥。研究《易经》可能需要几十年的积累和功夫,到一定年龄段就会被其深深吸引,《易经》蕴藏了宇宙、社会、人生的深刻哲理,活到一定年纪自然想弄懂这些道理,人生不是混吃等死就算完的,还有很多道理需要我们去学习和领悟,这样生命才能进入更高的境界。时至今日,《易经》仍具有强大的生命力和巨大的影响力,对《易经》的研究与学习,显得颇为重要,也许你在人生的某个阶段感到异常迷茫,这时不妨学习下《易经》的道理,也许就会有豁然开朗之感。

中国自古至今就有“医源于易,易医同源”的说法。在中医的医学史上,出现了许多中医名家,这些名家同时又是易学大家。如唐代药王孙思邈就提出了“不知易,不足以言太医”的论断,就是说不懂《易经》者不能成为一个高明的医生,明代大医家张景岳说;“易具医之理,医得易之用,医不可无易,易不可无医。”“易之为书,一言一字,皆藏医学之指南。”这就是说医易相通的道理。唐朝宰相虞世南说:“不读易,不可为将相。”中国的历代高人,没有一个不学习《易经》的,《易经》是一部非常古老、非常神奇的经典,“群经之首”这四个字就可以看出它的重大价值,中国有那么多的经典,但没有一个有这个美誉,可见其重要性之大。《易经》是讲道理的,讲规律的,弄懂这些道理和规律,对于人生真的很有指导意义。

每个人对《易经》的解读都有所不同,我这季是从戒色的角度来谈下《易经》,我只能谈点浅薄的心得体会,在圣贤教育面前我永远显得浅薄、渺小、无知,这也让我感到谦卑,能够接触到传统文化,聆听古圣先贤的教诲,是我一生的荣幸,愿把一点心得体会与各位戒友分享。

\begin{itemize}\it
    \item 君子以惩忿窒欲。(损卦)
\end{itemize}

忿即忿怒,遇不顺意时,轻则忿怒骂人,重则发生打人等不良行为,忿的状态持续到一定时候就上升为怒。换言之,忿是怒的初级状态,怒是忿的极端状态。欲是欲望,君子要克制忿怒,窒息嗜欲,嗜欲深者天机浅,沉溺于感官的刺激,这种人的智慧一定很浅薄,只有从声色繁华中超脱出来的人,才能具有真正的大智慧。《易经》讲的窒欲其实就是在强调戒色,当然欲的范围很广,物欲、食欲、名利欲等也算在里面,窒欲的窒字非常有深意,要把欲望窒息掉,这种魄力和手段是非常强的,窒欲有一种你死我活之感,就像把敌人死死按在水里,让其窒息而亡!我从窒欲这两个字中读出了一种强大的决心和执行力。一个嗔怒,一个嗜欲,君子要力戒之,嗔怒的人往往内心很虚弱,他控制不住自己,所以就会嗔怒,真正强大的人,内心往往很平静,很沉稳,处变不惊。不少戒友嗔恨心重,怨天尤人,这是一个很大的缺点,一定要改正过来,多发感恩心、孝顺心、慈悲心,不要怨天尤人,要从自身找原因,不断完善和提升自己。

\begin{itemize}\it
    \item 潜龙勿用,亢龙有悔。(乾卦)
\end{itemize}

潜龙勿用在戒色方面对应的就是杜绝婚前性行为,这点至为关键。在上世纪七八十年代时,婚前性行为还是较少的,那时的人比较单纯和保守,有戒友分享老照片,那个年代的年轻人的精气神的确很棒,那是一个没有色情和网游的年代。到了上世纪九十年代末婚前性开始渐渐多起来了,也是在九十年代末黄碟开始泛滥,进入了千家万户,很多戒友都有在自家找黄碟的经历,不管藏在哪里,他都能找到,所有的直觉都用在了找黄上,翻箱倒柜,总能找到。也是在九十年代性学家的性开放理论开始大肆传播,一方面是黄碟泛滥,另外一方面是性学家高调提倡,结果可想而知。《易经》告诉我们要潜龙勿用,\textit{孔子曰:“……少之时,血气未定,戒之在色……”(《论语·季氏》)} 在婚前这个阶段要学会保住自己的能量,用于自己的学业和事业,养精蓄锐,才能有一番大作为,反之,把身体掏空了,脑力、精力都会很差,这样何谈奋斗?还没开始多久就感觉精力不济,脑袋昏沉,直接就放弃了,之前的雄心壮志早就化为乌有,彻底瘫软了,再强的男人泄精后都像一滩烂泥,扶不上墙,本来想大干一场,有所作为,泄精后直接就放弃了,洗洗睡了,甚至都懒得洗了,倒头就睡,一天就稀里糊涂过去了。泄精前真的是雄赳赳气昂昂,活力四射,眼神有力,充满斗志,一泄精,人就萎掉了,双眼无神、呆滞、空洞,甚至都懒得转动眼球。拳王阿里就有赛前禁欲的习惯,禁欲让他在身体对抗中不疲软,眼神犀利,训练状态奇佳。1990 年,泰森在日本东京遭到道格拉斯 KO,随后泰森自述前一晚放纵了,纵欲让他脚都软了。泰森声称“我的整个职业生涯就是败在了女人身上”,一代拳王倒在了“粉拳”之下,实在令人唏嘘。真男人和软脚蟹其实就在一念之间,纵欲的特点就是人会软掉,活力、脑力、体能都会下降,软脚蟹的命运就是惨遭 KO!戒者自带强大的气场,眼神犀利、有神、有力,精气神爆棚,在真正的戒者眼中,有着雄狮般的威严!

亢龙有悔,盈不可久。意思是:龙飞到了过高的地方,必将会后悔,因为物极必反,事物发展到了尽头,必将走向自己的反面。在戒色方面对应的就是虚则亢,人在纵欲状态会有一个阶段感觉自己欲望很强,觉得自己身体很好,其实这是一种表象,不出几年,身体就会爆发症状。亢龙最后肯定会悔恨,因为症状会给予痛苦的体验。懂得了虚则亢的道理,才知道欲望强不是好事,欲不可强,越强越亏,亏到最后肯定症状爆发。亢龙有悔还有一个解释就是居高位的人要戒骄,否则会失败而后悔。戒色后也要戒骄,不管戒多久,都不能骄傲和放松警惕,随着戒色天数的增长,其实你就来到了一个“高位”,这时一定不可出现骄傲自满的念头,要始终谦虚低调,戒骄戒躁,保持警惕。德不配位,必有灾殃,到了高位再跌下来,往往会跌得很惨,摔得很重,如何能稳固在高位,那就必须让自己的德行跟上,德行上去了,位子就稳固了,德行不克,真的是危机四伏,祸来难料。

\begin{itemize}\it
    \item 君子终日乾乾,夕惕若,厉无咎。(乾卦)
\end{itemize}

君子修德,日日精进,无一息之间断,夜夜警惕,犹危险之在侧,如此则没有灾殃。古人强调慎独,强调警惕,为什么要如此强调警惕?因为有心魔在虎视眈眈,一不小心就会被附体,做出禽兽无耻之举。最近某些著名公益人士爆出性侵的丑闻,令人大跌眼镜,他们是做了很多的善事,但却不知对治自己的邪念,就像当年俞净意公一样,善事做了很多,但是却还在起各种邪念,导致命运不好。戒色界有一种理论,那就是“行善论”,不讲修心,只讲行善,这其实是一种思想误区。古人强调“断恶修善”,是双管齐下的,不是偏于一方的,断恶的最关键就是断“意恶”,也就是断邪念,这才是重中之重。否则虽大力行善,也可能像那些公益人士一样,做出非常不道德的邪淫之举,这是很可怕的事情,表面上好像做了很多善事,是励志、奋斗、正能量的榜样,但是暗地里却有着极其猥琐、不可告人的一面,宾馆的门一关上,就瞬间进入了禽兽模式,切换之快,令人咋舌,前一秒还道貌岸然,门一关上,色魔就出来了。公益人士是正能量的代表,怎能做出如此猥琐不堪之事?这完全是衣冠禽兽。《易经》强调警惕是对的,如果不对邪念警惕,就可能被邪念驱使,做出各种荒唐错误的事情,有的人被邪念附体后,就想着去嫖娼,买各种性具,在那种状态下最舍得花钱,甚至肯借钱去买,完全被欲望冲昏了头脑,两条腿也不听使唤,往那种场所跑。真的要警惕,时刻警惕,但不要过度紧张,要保持自然而警惕的状态,警惕自己的邪念,一念之差,就可能变成另一个自己,那个自己就是疯狂纵欲的自己。

\begin{itemize}\it
    \item 积善之家,必有余庆;积不善之家,必有余殃。(《易传·文言传·坤文言》)
\end{itemize}

修善的人家,必然有多的吉庆,作恶的人家,必多祸殃。所阐述的是一种事物由循序渐进、慢慢积累,最终量变引起质变的现象。“善不积,不足以成名;恶不积,不足以灭身。”古人很有积善意识,“君子以遏恶扬善,顺天休命”,君子应该遏制邪恶,宣扬善行,以顺从天命(自然规律)。现代很多人都没有积善意识,活得浑浑噩噩,没有任何意义,不知道每天在干嘛,我过去就是那种状态,不懂得感恩,也不知道去帮助别人,活得很自私,也很痛苦,自私的人往往内心是很痛苦的。后来我有了积善意识,内心感觉就完全不同了,坚持行善,内心变得祥和、喜悦、轻安。积善就是在增加正能量,内心会有一种崇高的感觉,也会觉得自己的人生变得很有意义。我们来到这个世界是要为天地之间增加正能量的,这样才能顶天立地,问心无愧!堂堂七尺男儿缩在角落里玩弄自己的生殖器,这是多么肮脏、猥琐和不堪的画面,实在愧对祖先,愧对列祖列宗,愧对父母的养育之恩,把自己的精气神都挥霍在邪淫上,这样糟蹋自己实在太不应该了。我们必须戒掉恶习,多做善事,做一个正气凛然之人,为天地之间增加正能量。我们要学会培养自己的积善意识,注重积累点滴的小善,点滴的小善可以像雨水一样汇聚成溪流,更多的溪流就可以汇聚成大江、大海,坚持小善最终可以汇聚成了不起的大善。

\begin{quotation}\it
    刘濠,字浚登,为刘基(字伯温)的曾祖父,南宋末年曾官任翰林掌书。刘濠是一位乐善好施、行善积德的人,他身上有着一种难得的人文素质与高贵品质。用现在的话说,就是具有大爱、大义、大德。宋朝灭亡之后,刘濠辞官隐居家乡武阳,虽不做官,平时他仍关心百姓疾苦,并十分同情贫困的百姓。据说,每逢梅雨季与寒冬腊月,他常登高看望村里的人家,如果发现谁家不见炊烟,他就把自己家的粮食送去救济他们。刘濠这种乐于助人、济贫救世之举,深受乡里百姓爱戴。除乐善好施之外,刘濠焚屋救人的义举在历史上也留下重重的一笔。宋朝灭亡后,阶级矛盾和民族矛盾不断升级,反抗元朝统治者便纷纷揭竿起义。元世祖至元二十四年(1287),刘濠的同县人林融也招募了一支义兵反抗元朝。林融原为宋代提刑,因感愤于宋朝灭亡,便打起兴复宋室的旗号,聚众起义,结果惨遭镇压而战死。林融死后,朝廷并没有就此罢休,而是派使者前往青田县九都一带严查林融余党,欲赶尽杀绝。当地的地主豪绅,乘机挟仇陷害百姓,把交不起田租的平民也当作余党,将姓名报给专使,专使据此搜罗了万余名无辜者登记入册,准备返京上报朝廷清除。刘濠见此事牵连人员如此之众,十分不忍。为了救无辜百姓,使者在青田武阳住宿时,刘濠特设酒筵宴请专使,并与孙子刘爚(刘伯温父亲)配合将其灌醉,待其酒醉沉睡后,刘濠从专使的行囊中取出被录的名册,将原有万余人的名单从中录下两百来名巨魁、恶绅的姓名后,放火将自己的房屋连同使者的名单一同烧掉。看着熊熊大火燃起,他又派人将使者救出。使者惊醒时,房屋已经倒塌。专使见行囊和籍册一概化为灰烬,大为惊恐。刘濠则趁机以记录数百恶德败行之人的名册作为替代的册籍给使者交差。这样一来,后来朝廷仅杀两百人,使无数无辜受牵连者得以保全,免于一死。所谓“积善之家,必有余庆。”刘濠救人的义举对刘基也产生了十分深刻的影响。刘基小时候很聪明,他的老师郑复初就曾对刘基的父亲说:“你的祖父道德深厚。这个孩子必定荣耀光大你的家门。”后来刘基帮朱元璋灭了元朝,平定了天下,得以封诚意伯的爵位。人家都说他是因为祖上积德,才有机会获得这样的成就。

    梅兰芳的开蒙老师吴菱仙,常把梅兰芳的祖父的事迹讲给梅兰芳听,他说:“你祖父待本班里的人实在太好。逢年逢节,要据每个人的生活情形,随时加以适当的照顾。我有一次家里遭到意外的事,让他知道了,他远远地扔过一个小纸团儿,口里说着:‘菱仙,给你个槟榔吃!’等我接到手里,打开来看,原来是一张银票。”梅兰芳的祖父好慈重义,厚待他人,因而又有“义伶”的美誉。他的事迹,足以给斤斤计较及刻薄寡恩的人作反省之资。古人有一种说法,为人所付出的,将从上天那里得到回报。梅兰芳能够成为富甲伶界的艺术大师,应该是梅家“施于人而得于天”的感应吧。

    陈赓的父亲陈绍纯,以乐善好施闻名乡里,行善不倦,从柳树铺通往湘乡的路就是他参与捐资修建的。陈绍纯以先辈的积蓄购田三百多亩,加之自身的劳动和俭朴,家道渐兴。陈家房屋原来有二十间,后来加配了三间,在里面安置床铺、被褥及衣物,专供孤寡老人、残疾人、孤儿和一些无家可归的乞讨者居住,并提供吃喝。陈家曾经为救济饥寒交迫的难民,慨然卖田施救,到最后仅剩下一亩三分田。可称得上是“大义”之家。陈赓后来能成为开国大将,或许是受到其父母的“躬行大义”的感染。

    赵朴初的祖父赵曾裕也是位大善人,光绪八年发大水,赵曾裕设法募集白银用于救灾,救活了不少人。赵朴初的叔父赵纶士曾创办安庆六邑联中,举家产用于办学,其母病故,将亲朋吊唁之款全部捐给学校。赵朴初的父亲赵炜如在家以教书为业,母亲陈仲宣是一个佛教徒,家中设有佛堂,每日早晨烧香拜佛。门前的水塘是她的放生池,里面放养着不少她买下的鱼、龟。赵朴初的父母为人和善,穷苦人向他家借钱,都能够借到,谁家发生天灾人祸,他们都能给予接济。一位当地名叫汪凡的老人曾回忆,他小时候到赵家去卖鱼,正碰上赵朴初在家做粑,朴老母亲买下了他的鱼,还送他好多粑,让他带回家孝敬父母。赵朴初一生从事佛教和慈善的事业,是和他的家世背景分不开的。朴老的祖荫,可谓深厚。
\end{quotation}

家道之兴,必有前人的广积栽培,而后才有如《易经》所说“自天佑之,吉无不利”的后代子孙生于其家。举世之人皆想要自己的子孙贤达,若唯以自私而求之,子孙反贫弱无智,而推公行义之人,子孙往往兴隆有德。故明代四大师之一的憨山大师,在《道德经》的注解中说:“施而不取,我既善矣。人不与而天必与之。”我们戒色后一定要深深懂得积累善行,多做善事,多行义举,这样不仅自己人生乃至家族命运都会有质的飞跃。

\begin{itemize}\it
    \item 劳而不伐,有功而不德,厚之至也。(《周易·系辞上》)
\end{itemize}

有功劳而不自我夸耀,有了功而不自以为有功德,这种表现敦厚之极。上德之士,不以为有德,不显其德,不彰己德。老子《道德经》讲:“上德若谷”,“谷”者,“虚”也。形容人十分谦虚,所谓“虚怀若谷”。有的人喜欢自我夸耀,有一点功德就经常挂在嘴上,逢人便说,这就是智慧浅薄德行不够的表现,真正有功有德有智慧的人都是沉默不言的,就像没做过一样,这样的人才值得我们敬佩和学习。虽然有极大的功劳,但还是甘居别人之下,甚至让功于人,让功于众,懂得谦退,不会傲慢。《易经》这段话我个人非常喜欢,这种品质也是我非常仰慕的,能做到这点也很不容易,有功劳不夸耀,不自以为有功德,非常低调和谦虚,这种人的道德水平和涵养真的很高,厚之至也,敦厚、忠厚,厚道!围棋有讲厚势,一般喜欢运用厚势的人,构思都非常宏大,换句话说就是大局观好。戒色到一定阶段,就会发现德行是至关重要的一个因素,德行上去了,戒色才能更稳定,自身的境界才能实现不断突破。有些戒友之所以破戒,就是德行有亏,德行没有跟上,内心还有很多负面的念头,负面的念头会导致内心不稳定、不祥和,这样就容易间接导致破戒。学习圣贤教育可以让我们了解自己的不足之处,可以学习圣贤的嘉言懿行,效仿圣贤,循天地之德,养浩然之气,慕正士之风,尚君子大义,胸怀万宇,戒镇乾坤!

\begin{itemize}\it
    \item 谦谦君子,卑以自牧也。劳谦君子,万民服也。(谦卦)
\end{itemize}

《说文》中对谦的解释是:“谦,敬也。”也就是恭敬、谦虚、谦逊的意思。对“牧”的解释是“养牛人也”,引申为放养牲口,比如放牧、牧羊等。“自牧”就是自我放养,不等别人来管理,不等外力来强迫,而自觉遵守规则,努力做好自己,一日三省,克己自重。自畜其德,卑以自牧,不断学习,不断反省,以成其谦。古人多以谦逊作为一种修身养性的基本准则,并且用“满招损,谦受益”来自我警醒。谦卦六爻皆吉,在我眼中,《易经》六十四卦中,谦卦是最强、最厉害的一个卦,也是我最喜欢的一个卦,最想研究的一个卦,能够真正参透这一卦,定会受益匪浅,受用终身!其他卦有好有坏,就谦卦全是好的。谦卦之中,包含两层含义,一是指谦虚,待人接物不狂傲,平易近人。另外一层是谦让,不与别人针锋相对,谦让容人。自然界的万物也遵守谦之道,人类的道德标准是厌恶骄傲自满而崇尚谦虚。地位高贵的人,如能具备谦让之德,他的道德会更显光辉;地位卑下之人,如能具备谦让之德,其行为必不会逾越礼制法度,具有谦德的君子一定会有好的结局。谦卦启示谦让与自律的美德,为人处世一定要懂得谦虚,一个缺乏谦德自命不凡过于狂妄之人,是很难与人和谐相处的,也很难在事业上有所成就。当一个成功者保持平和谦虚,就会有持续成功的状态,甚至会获得更大的成功。对于戒色而言,首重谦德,能够谦卑自守,以谦卑的姿态安住低处,大吉。“谦者,德之柄也。”因为谦虚才能执德,骄傲则必失德。谦卦教导人谦虚,唯有谦虚才能受到尊崇,而光明其德,人的德行越高,戒得也会越好。德才学识量,德是排在第一位的,要以德御才,德才兼备,德者,才之主;才者,德之奴。一个人的德量真的很重要,德量指的是道德涵养和气量。《菜根谭》:“德随量进,量由识长。故欲厚其德,不可不弘其量;欲弘其量,不可不大其识。”(译文:人的道德是随着气量而增长的,人的气量又是随着人的见识而增加的。所以要想使自己的道德更加完美,不能够不使自己的气量更宽宏;要使自己的气量更宽宏,不能不增加自己的见识。)

有句话说得好:“一个人若没有真正放低过自己,就不会懂得谦卑的力量,那是灵魂上的高贵。”谦卑者往往有一种高贵的气质,层次越高的人,往往越谦卑,越低调,正如谷穗越成熟会越低头,越干瘪会越仰起,空瘪的谷穗把头高扬向天空,饱满的谷穗则把头低垂下大地。真正有思想、有见识、有能力、有涵养、有德行的人,都愿意把自己放在较低的位置,服务众生,无私奉献,这才是真正的谦卑。“地低成海,人低成王”,真正有大智慧的人,都深谙谦卑之道,人越谦虚,把自己的位置摆得越低,别人就会更加亲近你、拥护你和支持你,人越傲慢,别人就会开始攻击你、排斥你和疏远你。

《彖》曰:谦,亨,天道下济而光明,地道卑而上行。天道亏盈而益谦,地道变盈而流谦,鬼神害盈而福谦,人道恶盈而好谦。谦,尊而光,卑而不可逾,君子之终也。《彖》这是告诉人们,具有谦让之德的君子,可畅行天下而无往不利。

谦卦的卦辞也简练,只有五个字——亨,君子有终。可是其内涵却极为丰富,谦虚使人受益,使人进步,所以会亨通。君子能够做到这一点,便会得到善终。

\begin{description}
    \item[谦卦第一爻,爻辞:初六:] 谦谦君子,用涉大川,吉。爻辞释义:谦虚而又谦虚的君子,可以涉过大河,安全吉祥。
    \item[谦卦第二爻,爻辞:六二:] 鸣谦,贞吉。爻辞释义:鸣:指宣扬,传播。鸣谦:使谦虚的名声远扬四方。本爻的意思是:谦虚的名声远扬四方,坚持正道可获吉祥。
    \item[谦卦第三爻,爻辞:九三:] 劳谦君子,有终,吉。爻辞释义:有功劳而谦卑的君子,有好结果,吉祥。
    \item[谦卦第四爻,爻辞:六四:] 无不利,挥谦。爻辞释义:挥,是发挥、施展。此爻的意思是:没有任何不吉利,只要发挥谦虚的美德。
\end{description}

\begin{itemize}\it
    \item 君子敬以直内,义以方外,敬义立而德不孤。(《坤·文言传》)
\end{itemize}

“直”是说人应品性纯正,“方”是指办事应合乎理义。君子以恭敬慎重的态度作为内心的正直准则;以合乎理义的行为处理外界事务,这样兼具“敬”和“义”的有德行的人是不会孤单的,他必定得到众人的亲近和支持。孔子说:“德不孤,必有邻。”有道德的人是不会孤单的,一定有志同道合的人来和他相伴。古人常说,同声相应,同气相求。天地之间的万事万物,都有一种朝着与自己相近的事物移动的倾向,相同或者相近的事物总会走到一起的,所谓“物以类聚,人以群分”。戒色者与戒色者会走到一起,画家与画家会走到一起,赌徒与赌徒会走到一起,频率相近的人会自然走到一起。君子要学会保持恭敬的心态,曾国藩:“主敬则身强。内而专静统一,外而整齐严肃,敬之工夫也;出门如见大宾,使民为承大祭,敬之气象也;修己以安百姓,笃恭而天下平,敬之效验也。聪明睿智,皆由此出。庄敬日强,安肆日偷。若人无众寡,事无大小,一一恭敬,不敢懈慢,则身体之强健,又何疑乎?”主敬是一种很深的涵养,恭敬了,内心自然就正直了,看看邪淫者的内心就知道他们是多么不恭敬,那些龌龊的想法简直太无耻、太肮脏、太下流、太不可告人了,有的人甚至意淫自己的同学和老师,看到异性就往那方面想,完全就是色魔附体。如果有对人恭敬的涵养和意识,那内心的邪念自然会大幅减少,对人恭敬也是一种原则,敬人者人恒敬之,恭敬的心态可以对治淫乱的心态,对人恭敬有礼,尊重对方,而不是意淫别人;恭敬心也可以对治傲慢心,让自己谦卑下来,不会趾高气扬看不起人。印光大师:“若贤若愚,通皆恭敬,不生傲慢。”欧阳修云:“君子之修身也,内正其心,外正其容。”内心要正直,首先要懂得恭敬,做事要合乎道义,敬义两字真的非常关键。

\begin{itemize}\it
    \item 泽无水,困。君子以致命遂志。(困卦)
\end{itemize}

意思是说:泽里没有水,象征困穷。君子当此之时宁可舍弃生命也要实现崇高的志向。君子观此卦象,以处境艰难自励,穷且益坚,舍身捐命,以行其夙志。困卦很有启示,处于困境之中,这时就要拼一把才能改变现状,有点破釜沉舟的意思。泽无水,困;人无精,亦困!精之于人,犹如油之于车,车再好,没油也开不了。现在有电动汽车了,但是道理是一样的,如果缺少能量,就会导致受困。邪淫者迟早会进入人生的困境,这是肯定的,有的人很早就进入困境了,有的人则稍微迟一些,每个人身体底子和福报有所不同,但最终都会因为邪淫而进入困境,这是必然的,有的人可能邪淫几年就进入了困境,而有的人可能在十几年后才进入困境,报应迟早会来的。有的人虽表面看似风光,其实他的内心非常空虚和惶恐,只能从邪淫的快感中找寻片刻的麻醉和忘却,有的亿万富翁自杀,他们并不缺钱,甚至有钱有势,呼风唤雨,但还是选择了自杀,原因就是处于了某种困境,也许是外在的压力,更多的是心理的困境,最后选择一死了之。如何突破困境真的是一门很深的学问,人人都可能被困住,不仅是因为邪淫,也可能因为其他的因素。关键自己要学会突破困境,要有一股杀出重围的架势,拼死一搏,抬棺死战!完成命运的大逆转。

我们一定要懂得积累自己的能量,邪淫必须要戒掉,邪淫是导致陷入困境的一个很大的因素。前段时间有个戒友专门提到了嫖娼后丢工作的事情,他说这是一种规律,嫖娼后极有可能会丢工作,被领导开除,然后生活陷入困境,他讲得很有道理,我之前的同事也是因为嫖娼丢了工作,嫖娼后气场变得很不好,然后领导看他不顺眼,开始找他茬,因为邪淫,他工作也不在状态。我之前的两个同事都是因为嫖娼被开除,真的很准!因为邪淫导致症状缠身,工作无法胜任,自己辞职的也有不少,加上看病吃药,积蓄慢慢就没了,人生陷入了困境,症状严重的人甚至都虚弱得无法下床,这是真事,我看过不少案例,都沦为了躺床废人。还有的人撸了十几年,性功能非常差,精子质量也差,弱精症,死精症,这时被家人逼婚,可谓焦头烂额,不知道怎么办才好,这也是一种困境,能量不断损耗,最后的结局必然是陷入困境。男人一定要懂得管理自己的能量,有了肾精,有了正能量,人马上就不一样了,这时冲破困境就相对简单很多,能量十足的人,他的冲劲是完全不同的,他是在“乘马冲杀”,那种冲杀的气势和强劲的动力非常之猛烈,摧枯拉朽一般。再强的人就怕犯邪淫,犯了邪淫后迟早会被困住,困住后就完全施展不开,所谓龙困浅滩遭虾戏,虎落平阳被犬欺,鲨鱼再厉害,离开了水,也只能等死而已。

\begin{itemize}\it
    \item 君子藏器于身,待时而动,何不利之有。(解卦)
\end{itemize}

君子有卓越的才能超群的技艺,不到处炫耀,而是在必要的时刻把才能或技艺施展出来。这话告诫我们,在默默无闻的时候,要加强自身修养,等到机会来时,就要充分展露自己的才华。在不利的情况下,要藏器待时,等待机会,这是《易经》中非常重要的思想。这是告诫人们“藏”与“动”的智慧,什么时候“藏”,什么时候“动”,这都要看“时”的条件。在时机到来的时候,就要抓住有利时机,在最大的程度上实现自己的人生价值。冯梦龙《东周列国志》第九十五回:“昭王深自韬晦,养兵恤民,待时而动。”越王勾践卧薪尝胆,励精图治,最终雪耻灭吴,这也是待时而动。这段话我有自己的解释,是从戒色实战角度来讲的,所谓器,就是断念之刃,平时不断练习,把这把刀磨锋利,藏于身;所谓待时而动,就是等到念头上来时,立刻斩杀之,要快、要狠、要严、要烈,切忌贪恋和犹豫,念头一冒,你的铜铡刀就下去了!包青天的三口铡刀,威风凛凛,庄严公正,罪犯看了都害怕。作为戒者,必须要具备一把无比锋利的断念铡刀,这把铡刀要斩尽所有的邪念,你就像铁面无私公正严明的包青天一样威严地坐在中堂,背后“明镜高悬”,殿前铡刀镇堂,念来即斩!绝不轻饶,对邪念零容忍,刚正不阿,刚严的典范,眉宇间一派凛然正气,泰山压顶般的浩然正气,如钢似铁的千古正气。君子最强之利器,就是断念之刃,这把利器藏于身,待时而动,心魔来犯,请试吾刀!!!

\begin{itemize}\it
    \item 蹇,君子以反身修德。(蹇卦)
\end{itemize}

意思是说君子在面对挫折、困难(蹇)的时候,应该反躬自问,修养品德。蹇,多指行动迟缓、困苦、不顺利的意思,当陷入困境时,这时就要反身修德,修身种德,事业之基,《菜根谭》:“德者事业之基,未有基不固而栋宇坚久者。”一个人的高尚品德就是他一生事业的基础,这就如同兴建高楼大厦一般,假如不事先把地基打得很稳固,就绝对不能建筑既坚固而又耐久的房屋。当遇见挫折和困难,陷入不顺了,这时候最明智的做法就是认真反省,改过迁善,多行善积德,这样就能改变现状和突破困境。《素书》:“先莫先于修德”。古人极其重视和强调修德,无论是为人、处世还是立业,修德永远是放在第一位的,这是地基,只有夯实了深厚的道德根基,才能广载万物,生生不息。《易经》讲到“天行健,君子以自强不息;地势坤,君子以厚德载物”,天(即自然)的运动刚强劲健,相应于此,君子处世,应像天一样,自我力求进步,刚毅坚卓,发奋图强,永不停息;大地的气势厚实和顺,君子应增厚美德,容载万物。一个拥有博大胸襟与高尚品德的人,不断进取,宽厚待人,自然会得到众人的敬佩与支持,进而取得巨大的成功。清华大学的校训即为:“自强不息,厚德载物。”乃是引用此处。自强不息和厚德载物,分别出自乾卦和坤卦,也概括了《易经》乾卦和坤卦的精神内涵,也是中国古人最看重的两种品质。修德是我们学习、修炼的首要内容,戒色一定要注重修德,地基稳固了,就能戒很久,反之,地基不稳,戒到一定时间就会垮塌下来。我最近看到一处工地施工,那些建筑工人把地基挖得很深,打得很牢固,这样就可以造高楼。德就是地基,遇到不顺了,应该反身修德,把地基彻底夯实,这样才能突破困境。

子曰:“德之不修,学之不讲,闻义不能徙,不善不能改,是吾忧也。”(孔子说:“不去培养品德,不研讨学问,听到了应当做的事,却不能马上去做;有错误却不能改正,这些都是我所担忧的。”)古圣先贤特别强调修德,《易经》里专门讲到“日新其德”,意思是天天增新美德。《少年进德录》:“淫秽一事,最能损德。”古人知道邪淫的危害,伤身败德取祸,所以告诫大家要力戒邪淫,一方面戒邪淫,另外要懂得培福培德,做盛德君子,日新其德,止于至善。我观察过很多成功人士,他们之所以能成功,和他们拥有的道德品质是密切相关的,他那个德和他那个位子是相配的,如果他犯了邪淫或者贪赃枉法,那个德就下降了,和位子就不配了,到时就会出现灾祸。我过去也不懂得修德的重要性,那时我的脑子被邪淫的内容所占据,满脑子全是那些龌龊的场景和画面,整个人就是一具被黄毒攻占的僵尸撸管肉机,不顾一切地疯撸,每天都是末日。回想过去十几年的邪淫经历,我完全活在了下半身的世界,我甚至觉得我不是在用头脑思考,而是在用睾丸思考,脑子里面被各种邪念占据,完全不可告人的邪念,极度肮脏下流。在一次次独处时疯狂看黄疯狂手淫,把身体精华掏空榨干,就像甘蔗过了榨汁机,彻底沦为废渣!为了找到那个让我有强烈冲动为之一射的镜头,我可以连续看十个小时的黄,甚至熬夜看黄,完全走火入魔的状态,撸到射不出了,还不死心,还想再来一次,精射掉了,就射水,到后来 JJ 都麻木了,没快感了,都不能充分勃起了,但手臂还在机械地运动,不撸立软,就靠撸维持着仅存的硬度,以求一射,真的是疲于奔命,丧心病狂,就是为了“完成任务”,最后总算射出了,人也快倒下了,两条腿都形销骨立了,戳在那,就像两条鹤腿一样,终于长吁一口气,任务完成了,心魔派的任务太繁重了,真是不掏到光净就不让停止,心魔要榨干我的每一滴精,我就是心魔的包身工,实在身不由己,这根本不是享受,而是奴役。疯狂找片累成狗,撸完空虚、厌倦和悔恨,删除片子发誓要戒掉,某天心魔袭脑没断掉,又开始疯狂找片,这就是一个死循环,一个怪圈。每次大撸特撸后,镜子里的人气色就像鬼一样,耷拉着眼皮挂着黑眼圈,眼睛特无神,整日浑浑噩噩,不知道自己在干嘛,负能量很重,经常起负面的念头,不仅伤害别人也伤害自己。那时“德”这个字真的离我好遥远,后来戒除了邪淫,学习了圣贤教育,才知道修德原来是那样重要,从此我立志要做一个有德行的人,不断完善和提升自己的德行,我感觉到了久违的喜悦感和幸福感,这是邪淫时怎么都无法获得的美好感受,这种生活状态才是我真正想要的。邪淫时的我是多么荒唐、愚蠢、无知和堕落,到后期已经有变态倾向了,很可怕,如果不戒掉,我都不知道自己会变态成啥样。谢天谢地,现在我总算告别了那个低频黑暗的状态,就像走出了地牢,重回阳光纯净的世界,这种感觉真好!我再也不想过那种邪淫放纵自毁的生活了!那种生活就像失控的、即将坠毁的飞机。

\begin{itemize}\it
    \item 君子以正位凝命。(鼎卦)
\end{itemize}

正位凝命,如鼎之镇。鼎,不偏不倚,有厚重、端正之相,国之重器也。君子行动有恒,事物感应可致亨通,其利在于守正,坚贞自守则获吉祥,心正则兴旺,心邪必衰败!心邪导致振频下降,必然感召霉运。守正这两个字太重要了,很多人穷的时候还能守正,等到富裕后就开始乱来了,穷的时候缺乏“造业活动资金”,圈子也很小,尚能安分守己,富裕后圈子大了,色胆也跟着大了,造起恶业来会很猛。福报大而又没有智慧,只会倚福造恶,这是非常危险的事情,后果很严重。穷的时候,造不了那么多的恶,富裕之后,造恶的能力翻倍猛增。富裕后一定要有智慧,不能邪淫,古人讲富贵不能淫,非常有道理,《老子》第五十八章:“祸兮福之所倚;福兮祸之所伏。”福与祸并不是绝对的,它们相互依存,可以互相转化,坏事有时可以引出好的结果,好事有时也可以引出坏的结果。有了福报还能够守正,这很不容易,现在这个社会诱惑太多,看到很多案例都是福报很大,造业很猛,很快就走向衰落,甚至身价几十亿的人都会因为邪淫走向衰败和毁灭。国外戒色书籍说,色情的另一面就是毁灭,一点没错。君子要让自己处于正位,只有在正位,能量才能凝聚起来,处于邪位,能量就处于耗散的状态。很多人都有这个感受,邪淫后钱财总是存不住,有点钱就用于放纵,邪淫不仅掏空身体,也掏空了钱包,症状爆发后进医院检查看病都要钱,有的人身体症状严重,工作都干不了。处于邪位,就是一种耗散的状态,能量无法有效凝聚起来。处于正位,我能感觉到能量凝聚后的充实感,浑身有劲,内心也自然感到喜悦,对人也懂得包容。李嘉诚香港办公室挂的唯一一幅 24 字长对联:“发上等愿,结中等缘,享下等福;择高处立,就平处坐,向宽处行。”这幅著名的楹联,语出清代湘军儒将左宗棠,这 24 个字浓缩了深刻的人生哲理,“发上等愿、结中等缘、享下等福”,就是胸怀远大抱负、只求中等缘分、过普通人生活;“向高处立、就平处坐、从宽处行”,则是看问题要高瞻远瞩、做人应低调处世、做事该留有余地。无论面对多么错综复杂的局面,都能够处变而不惊,遇险而不乱,既能创造一番事业,又能守住一番事业。《易经》里讲到:“闲邪存其诚,善世而不伐,德博而化。”(防止邪恶,保持内心的真诚,为善于世而不夸耀,德行广被而感化世人。)守住正位,谦虚低调,好好培养自己的德行,为善于世,正己化人,为天地之间增加正能量。

\begin{itemize}\it
    \item 《易》无思也,无为也,寂然不动,感而遂通天下之故。非天下之至神,其孰能与于此。(《周易·系辞上》)
\end{itemize}

我个人认为,这段话是《易经》最深奥、最有内涵、最值得反复体会的一段话。这段话其实就是在指明真心、真我,也就是一念不生、了了分明的状态,这个状态既简单又极其深奥,很多人都无法理解这段话,如果你还在认同身体和念头为自己,你对这段话就会百思不得其解,即使那些著名的易学家和教授也无法理解,即使绞尽脑汁想破脑袋都想不通。只有彻底认清身体和念头不是真正的自己,观察者才是真我,只有这样才能正确体会这段话。这段话可以说是《易经》的最核心,也就是核心中的核心,黄念祖老居士:“孔子说:‘无为也,无思也,寂而不动,感而遂通。’孔子是圣人,说的就是卦,还不是人的本性。寂而不动才能感,心中乱糟糟的那怎么能感?”记得以前我也无法理解这段话,因为那时的我还在顽固地认同身体和念头是自己,这点认同就是错误的根源所在,你去问别人,真我是什么?肯定一脸茫然,然后你问“你是谁?”对方回答肯定是“我叫某某或者某某某”,一个名字而已,一般人肯定会认同名字、身体、念头为自己,这种认同是根深蒂固的,从来没人告诉他们什么是他们的真正身份,要破除这种根深蒂固的错误认同是很难的,有的修行人也许一辈子都无法认清,都无法转过弯来,这是很可悲的事情。无思也,就是无念的状态,但并不是昏迷或者熟睡时那种无知觉的状态,而是了了分明,这就是纯粹的意识(pure consciousness)、纯粹的觉知(pure awareness)、纯粹的存在(pure being),名称有所不同,但指向的都是同一种状态。那个无念的“pure”状态,pure 这个英文单词我很喜欢,这个词的发音也很有意思。我以前的文章也专门阐述过这种状态,上次看到有位戒友看了我的文章,就对那段文字无法理解,一头雾水,这是很正常的反应,刚开始肯定无法理解,除非你悟性超卓,才能当下契入。一般都需要一个过程,反复看,反复悟,突然有一天就顿悟了,明白了前辈的真正意思。其实很简单,大道至简,人的倾向是喜欢复杂的东西,觉得这样才有挑战,而对于至简的东西反而很难相信。《终极自由之路》里讲到:“你之所以来到这个世界,是为了发现你的真我。”“当我们完全地回归真我时,我们才能彻底地满足。”“终极幸福就是真我。”我们此生最高的使命就是认识真我、活出真我!

\paragraph{总结}

《易经》堪称我国传统文化的源头活水,它的内容极其丰富,可谓博大精深,对中国几千年来的政治、经济、文化等各个领域都产生了极其深刻的影响。无论孔孟之道,老庄学说,还是《孙子兵法》,抑或是《黄帝内经》,无不和《易经》有着千丝万缕的联系。凝聚着中国古圣先贤古老智慧的《易经》,曾被误解为一本算命的书,其实算命、堪舆只是其很小的一个面向,我更看重的是《易经》讲述的为人处世的道理,还有阐明的事物发展变化的规律。《易经》是中国哲学思想的总源头,真是非常深奥,很值得研究,这季摘录了十二条,只是很少一部分,《易经》中发人深省的好句真的有很多,后人的注释文章也会带来很多启发。学习中国的传统文化,肯定要学习《易经》,很多大学教授都在研读和讲授《易经》的哲理和智慧,学习《易经》这门古老而神秘的智慧经典,可以让我们获得伟大而深刻的智慧,可以让我们趋吉避凶,更好地把握自己的人生。

最近又看了 94 版《三国演义》,的确很经典,三国里刘备、关羽、张飞、赵云、诸葛亮,我都很喜欢。刘备德行好,能做大哥,一般德行好的人都能做老大,德行要胜过才华,最厉害的是德,德比才厉害,任何才都比不上德,因为德直接在振频上进行操作,最终是看振频的高低,而不是才华的高低,有德之人振频高,振频高就可以坐很高的位子或者达到很高的境界,所谓德者居上。《东周列国志》中宁戚向齐桓公进言:“以威胜,不如以德胜。”德胜才能心服口服,刘备说过:“惟贤惟德,能服于人。”有德之人可以服众,可以聚集一大批才华横溢的人才,心甘情愿为其效劳,共创一番宏图伟业。关羽过五关斩六将,神威盖世,我们断念也要具备这种气魄,这种英雄气概,斩尽一切邪念!不管是意淫、怂恿念、图像、微妙感觉,一上头,一袭脑,马上斩立决!念起即是决战!绝不手软,绝不犹豫!单挑心魔,斩于马下!过色关斩万念!!!断念也要具备张飞之勇,万夫莫敌,一声断吼,一股爆炸性的力量展现,力道刚猛,让人看了有一种惊心动魄的气象,吓退心魔百万兵;常山虎将赵子龙,浑身是胆大将风,英气逼人豪气万丈的孤胆英雄,血染征袍透甲红,当阳谁敢与争锋!戒色也要有子龙的胆魄与锐气,杀出一条血路,虽千万人吾往矣!完成惊天的壮举!诸葛亮精通易学,悟性奇高,运筹帷幄之中,决胜千里之外,我很欣赏诸葛亮的气度,那份从容不迫、气定神闲的气度真的很有儒雅君子的风范,实乃修为深厚之士,气宇不凡,风度翩翩,神态高雅,举止自若,飘逸如仙。他对易学的理解很深刻,造诣颇高,神机妙算,是智慧的化身。要获得智慧,把握事物发展的规律,研读《易经》是第一选择,《易经》真的很有价值,很有内涵,很有智慧。

《易经》是非常深奥的,是关乎天、地、人根本规律与关系的智慧经典,那些道理需要花很长时间去领悟和参透。活到一定年纪,自然会被《易经》所吸引,十几岁时往往会被娱乐明星所吸引,后来就会知道追星是肤浅的,颜值是靠不住的,真正有价值的是圣贤教育,等待我们去发现它的价值。我这季从戒色的角度浅薄地谈下心得体会,权当抛砖引玉,建议有兴趣的戒友可以认真研读一下《易经》,定会受益匪浅。

下面分享两首戒色诗歌。

\begin{poem}[觉·刀]
    \begin{multicols}{3}
        \centering~\\
        戒色刀客 \\ 苍凉而悲壮的背影 \\ 仿佛屹立了千年 \\ 拔刀,狂风大作 \\ 眼神犀利,杀气凛冽 \\ 出刀前的气氛 \\ 就像炸药即将爆炸一样 \\ 邪念突然上头 \\ 刀在鞘中剧烈跳动 \\ 只有杀过念喝过血的刀才会这样 \\ 实战就在刹那 \\ 电光火石之间 \\ 刀已回鞘,微风拂面 \\ 眨眼的一瞬,战斗已经结束 \\ 刚才浓烈的杀气顿然消失 \\ 化作一派纯真无邪 \\ 一刀绝患! \\ 一战惊人! \\ 一刀之威,万念难挡 \\ 一刀下去,神鬼皆泣 \\ 这是觉察的刀 \\ 比闪电更快 \\ 一刀出,万念灭 \\ 刀光璀璨,划破长空 \\ 刀气纵横,威震天下 \\ 杀念的决心无比强烈 \\ 不是念死,便是我亡
    \end{multicols}
\end{poem}

\begin{poem}[那个看]
    \begin{multicols}{3}
        \centering~\\
        向内看 \\ 最奇怪的事情发生了! \\ 念头消失了! \\ 这时你突然顿悟 \\ 你不是念头,而是那个看! \\ 看消灭念!
    \end{multicols}
\end{poem}

下面推荐一本书。

\begin{book}[《纪文达公笔记摘要》]
    此书是从清朝乾隆年间大学士纪昀所作之《阅微草堂笔记》中摘录百余篇汇总而成。纪昀曾任《四库全书》总纂修官,再加上多年宦海生涯,阅历广泛。所见所闻所历的因果报应之事颇多,纪氏一一记录,并伴以评论,这就是现在流通于世的《阅微草堂笔记》。民国年间,陈荻洲居士从《阅微草堂笔记》中摘录百余篇因果故事,编辑成《纪文达公笔记摘要》,以期遍界流通,印光大师也曾为此《摘要》撰写序文,回溯往圣前贤,无不提倡因果,以期平治天下,淑世牖民。我在网上下了个电子版,一共 160 个因果故事,有白话文翻译,读来很受教育,里面很多因果故事都给我留下了深刻印象。这本书涉及到很多跨维度信息,对于这类信息我们还是要尊重,也许有的人不太能接受,但还是应该要尊重。这季把这本书推荐给大家,希望有缘人能读到它。
\end{book}
