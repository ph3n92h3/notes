\subsection{澄清对于断念口诀的误解}

\paragraph{前言}

最近参加了一个葬礼,是我奶奶的葬礼,奶奶从小对我特别好,她的质朴和慈爱让我感动不已,她走的时候 87 岁,能够活到这个年纪已经很好了。我十分感叹生命的无常和修行的重要性,每个人无论现在多么强健,多么健康,终有一天会死亡的,这一天终究会到来的。当我十几岁时,就已经经历了四场葬礼,前两场印象不是很深刻,毕竟年纪还小,是外公和外婆,第三场是我爷爷,73 岁走的,爷爷是抗战老兵,一身铁骨铮铮的正气让人肃然起敬,在当地很受人尊敬。第四场是我的父亲,47 岁走的,他的离世对我的打击很大,他虽然活得不长,但对这个家庭的贡献却非常大,可以说是家庭的顶梁柱,他是生病去世的,在外的烟酒应酬太多,把身体搞垮了。相信很多戒友都经历过亲属离世的事情,可能一些小戒友还经历得不多,到了快三十、四十岁,这种事情就开始多起来了,不一定是自己的亲属,也可能是朋友的家人去世,或者是朋友、同学、同事、领导等去世,死人的事情是时常发生的,奶奶火化那天,殡仪馆一上午就有差不多十个人火化,我奶奶排在第六位。小时候我对死亡很避讳,而且也不喜欢这个话题,还有很深的恐惧感,那时我认同的观点是人死如灯灭,什么都没有了,所以当我开始手淫后,内心会想着活着的时候一定要好好享受放纵下,不然死了就什么都没了,正是这种心理加速了我的堕落。后来接触了传统文化,接触了佛法和大德开示,也接触了频死体验,也读到了再生人的现象,知道轮回的现象确实是有,并不是人死如灯灭,这时才真正认识到修行的重要性,万般带不去,唯有业随身,《大般涅盘经》云:“不见后世,无恶不造。”《中阿含经》云:“不畏后世,无恶不作。”活着的时候应该诸恶莫作,众善奉行,平时多行善积德,多孝顺父母,多帮助别人。邪淫一定要戒掉,其他不良习气也要一起戒掉,我们不知道无常何时会到来,关键是做好自己,问心无愧。活着就是一场修行,应该止于至善,做纯净纯善的自己。

下面分享一些案例。

\begin{case}
    总算来了,感恩飞翔哥。虽然是老戒友,过去一年戒色状态不稳定,最近两个月每天花两个小时学习《戒为良药》,花一个小时学习《楞严经》,做笔记、复习笔记,戒色才起死回生,深深感受到学习传统文化和戒色知识让我快乐,练习断念让我快乐!我不想再回到过去的无助状态了,我要重生,我要崛起!过了年就 28 岁了,时不我待,要让戒色成为我的信仰,读书志在圣贤,做人势必戒色!我从一次次的断念中体会到了快乐,也找回了力量感,对《戒为良药》的体会更深刻了!

    \textbf{附评} 这位戒友真正下决心和下功夫了,开始认真对待戒色了,投入肯定会有回报,坚持学习戒色文章,多做笔记,练习断念,这样戒色就会渐渐进入正轨。所有的学习和练习都是为了实战的那一下,那一下必须足够强硬和果断,否则就会被心魔附体。不少戒友屡戒屡败,也不知道总结和反省,对于戒色文章也没真正下功夫学习,这样怎能戒色成功呢?真正戒色成功的前辈都是热爱学习之人,我也一直在坚持学习,几乎没有中断过,不是看文章,就是复习笔记,一直在坚持,我也乐在其中。当你能从戒色中找到乐趣了,就会更容易坚持下去,坚持两个字很不容易,刚开始很多人有很高的戒色热情,学习戒色文章也很精进,但是过了几个月进入戒色厌倦期了,到时就开始疏远戒色文章了,这样肯定是会破戒的。最近 28 岁的亚洲飞人苏炳添在男子 60 米比赛中以 6 秒 55 的优异成绩夺冠,苏炳添在收获 2018 赛季第一场比赛胜利的同时,也创造了新赛季亚洲该项目的最好成绩。苏炳添接受了十年以上系统的专业化训练,才有今天的成绩,这也是长期坚持的结果,没人能随随便便成功,成功的背后肯定有大量的付出,不知要花多少时间和精力,不知流了多少汗水,在成功的路上也会遭遇很多挫折乃至失败,但还是咬牙坚持下去,最终柳暗花明,迎来命运的转机。戒色也是如此,要成功,就要发长远心、坚固心、不退心,要坚持不懈地学习和练习,注重积累,提高训练质量,有的人也练习断念,但是训练质量很差,如何提升训练质量也是一个极其关键的问题。这位戒友现在的戒色状态很不错,希望他好好保持下去,真正的断念高手内心会有一种力量感,这种力量感不是来自举起 200 公斤的杠铃,而是来自于战胜心魔,彻底主宰内心!这种力量感显得更为强大!战胜心魔的人,他的气场就像一头威武雄壮的雄狮一样,他可能肌肉并不发达,但只要走进他,你就能感觉到他散发出来的那种强大的力量感!真正的强壮在于内心的统治力!真正的力量来自于降伏心魔!!!
\end{case}

\begin{case}
    感谢飞翔哥,感谢戒色吧,今年 29,估计断断续续地撸了八年,症状有容貌衰老、不自信、运气不行,单身。现在戒了快两个月,一直有看《戒为良药》,还有看戒色吧的分享,吃了一些中药调理,然后每周跑步锻炼,不熬夜,经过这段时间的恢复,感觉就像获得了新生,同事都说我这段时间越长越帅了,皮肤光泽很好,容貌也变好了,工作也顺利多了,自信十足地跟人交流了,感觉一股正气在胸中流淌,还有一个女孩子喜欢我了,戒撸者的快乐难以言喻,现在的我也会多做善事,虽然很小,但是能帮助到人也是很开心的。

    \textbf{附评} 为什么撸管会导致运气变差?其实是很好理解的,因为撸管会影响容貌、脑力、精力、自信等,而且还会导致身体出现各种伤精症状,整个气场都会变差,这样运势自然会减弱,我更喜欢用运势这个词,这样更客观些。比如原本你可以考好,但是脑力下降后,学习就开始变得吃力,学习成绩也会下降,这样就会影响考运。找工作的人如果带着一张猥琐颓废的撸管脸去面试,很难与面试官对视,目光就像做贼心虚一样躲躲闪闪,如果你精气神很足,内心光明正大,充满正气和底气,非常自信,以这样的状态去面试,成功率会高很多,也更容易找到好工作。如果你正气很足,看一眼面试官,面试官就被你的威严给慑服了,还没开口自我介绍,面试官就已经对你心生敬佩了。戒色之后,我们要懂得养生,懂得管理自己的能量,懂得行善积德增加正能量,这样坚持一段时间,精气神恢复后,到时就会焕然新生,周围人也会说你变得精神了,变得帅气了,一个男人可以长得不帅,但只要有了精气神也会有一种帅气的感觉,不仅帅气而且充满自信与底气。多少撸者都把帅气给撸掉了,前段时间一位戒友晒了对比照,之前长得像李易峰,沉迷撸管后精气神完全涣散了,帅气也不见了,一脸的无神和呆滞,五官和脸庞也走样了,实在很可惜啊!男神撸得惨不忍睹,不忍直视,颜值暴跌。男人一定要懂得戒色,真的太关键了,色情泛滥的时代诱惑非常猛烈,一不小心就会陷入色情的陷阱,我们必须严格自律,做一个顶天立地、浩然正气的戒色硬汉,所谓:七尺男儿戒邪淫,不愧祖先不愧天!
\end{case}

\begin{case}
    第 211 天,感觉最近过得好快,每天都很欢喜,很爱说话,说话也比以前大声多了,也爱开玩笑了,以前都不敢开玩笑,感觉社恐快恢复了,而且比之前更加的自信,时刻观照念头,不怕念起,只怕觉迟。感恩戒色吧,感恩飞翔老师,感恩帮助我的吧友,要是没遇到戒色吧,我想可能会痛苦地死去。

    \textbf{附评} 蜕变了、逆袭了、身心轻盈了、症状消失了、内心欢喜了、开心了、自信了、爱笑了,这是多么好的变化啊!遇到戒色吧就是一个蜕变与重生的机会,这个机会非常难得,很多人过去一直被无害论洗脑,很难接触到真相,等到身体垮掉后还不知道怎么回事,能够遇到戒色吧,并且接受戒色的理念,这是有福报、有善根的表现。戒色现在是国际大趋势,国外也有很多戒色网站和戒色书籍,人家起步很早,他们的理论和方法是比较科学和专业的,很值得我们学习与借鉴,当然我们也有自己的优势,那就是传统文化和圣贤教育,我们也有自己专业深入的研究,经过这些年的发展,也积累了丰富的戒色经验。戒掉恶习之后,能够亲身体会到各方面的好变化,这种身心状态真是太棒了!很久违的自己!那个开朗、自信、爱笑的自己又回来了,并且已经懂得修心,懂得观心断念,懂得控制自己的念头,开始学会主宰自己的内心了,这是很大的进步!在学校里很难学到这种知识,而戒色吧可以教给你,最高的知识是修心的知识,这是一切知识的顶点!一位戒友说:“昨晚撸了三次,一点钟才睡觉,我根本不想撸的,我就像一部机器,只管执行程序。”被心魔攻陷就是这种状态,身不由己,沦为撸管肉机,执行心魔的指令,掏空榨干才肯罢休,撸到后来其实已经不想撸了,感觉已经麻木了,只是机械地在那撸,疲于奔命,只是在完成心魔指定的任务,那就是彻底射空才算完,射到最后射出来的都是水了,射到双腿发软,站都站不稳。那个入侵的邪念就是木马程序,不断掉,就会被心魔操控!资深的戒友深知这一点,所以他们断念特别狠,绝对不能让心魔得逞!上季一位戒友说:“心魔很微妙地怂恿我,让我去搜黄图,长达七个小时的马拉松式搜黄后我破戒了,之后又破了一次,我这时才体会了那种完成任务的感觉(不是贪图快感,而是被控制了,能清楚地体会到)。”心魔的特点就是“入侵并控制”,入侵的方式有很多种,也很微妙,我们要学会识别并及时断除,千万不能被心魔控制!一旦被控制,就会进入疯狂看黄疯狂手淫的可怕状态。
\end{case}

\begin{case}
    本人 22 岁,学医的,有七年手淫史,上学期间一直没戒除,直到今年毕业进医院实习,感觉到生命的脆弱,有一个健康身体的重要性,决心戒除手淫,同时我也在半个多月前遇到了戒色吧,更进一步地清晰地了解到了手淫的危害,结合自身的身体状况,原来脸上的痘痘、黑眼圈和头顶旋处稀少的头发(不太明显)都是手淫造成的,我发誓,要同吧友们一起戒掉手淫,未来还有太多的美好在等着我们,为了含辛茹苦把我们养大的父母,为了将来能给他们创造一个幸福的晚年,为了自己美好的明天,加油!共勉。(戒色时间才 20 天左右就感觉到黑眼圈明显减轻,痘痘大幅度减少,愉悦感和自信心正在回归)加油吧,兄弟们!

    \textbf{附评} 学医要面对很多的生离死别,生命的确很脆弱,一个健康的身体也实在很重要,到医院看看那些处于病苦之中的人,就会感叹健康是多么宝贵。年轻人往往阅历比较浅,处于比较无知的状态,很多人都在疯狂追求感官刺激,压根儿没意识到这种放纵会导致什么后果,年轻时还有点底子,但是也经不起经年累月的耗损,迟早身体会垮掉的。父母给我们好吃的,也不希望看到我们如此放纵自己,疯狂糟蹋身体的精华,满地荒唐精,一把辛酸泪!痛苦啊!悔恨啊!先哲云:“人一思淫,心田即暗。中正之心已邪,则光明正大之气遂失。”木有根则荣,根绝则枯。鱼有水则活,水涸则死。灯有膏则明,膏尽则灭。人有真精保之则寿,戕之则夭,不异于此。\textit{孔子曰:“……少之时,血气未定,戒之在色……”(《论语·季氏》)} 圣人提醒少年,使其力制色心,悚然自爱,以保养柔嫩之躯。幼时能于色欲一关把得牢,截得断,他年元神不亏,气塞两间,达而立朝之日,精神得以运其经济,立掀天大事业,真人品真学问,皆由于此。即使不成大器,亦得以尽其天年,为祖宗似续之计。较死于非命者,霄壤之殊矣。莲蕊居士曰:断欲有十种利,反是有十害。一身心清净,毫无所污。二正念常存,异诸禽兽。三气足精满,寒暑不侵。四面目光华,举足轻便。五俯仰天地,无惭愧色。六省药饵费,可周贫乏。七屏绝邪缘,胸无牵恋。八读书作字,俱有精采。九脾胃强健,能消饮食。十本地风光,自有真乐。在青少年和青年阶段戒色是非常重要的,这位戒友说得很好,把身体撸垮了,怎么给父母一个幸福的晚年?身体废了,还会连累父母,真是大不孝。看看父母的皱纹、父母的白发,看看父母操劳的背影,再看看自己的手,你都干了什么?!还有什么理由继续撸下去?!坚决戒掉它!不仅为了自己,也为了父母!
\end{case}

\begin{case}
    不是不报,时辰未到,撸了 4 年的报应来了,今天早上起床脸部浮肿很严重,小腿大腿也肿了,然后去医院检查,高血压,当时吓我一跳,我才 17 岁啊!怎么可能高血压呢?后面医生又让我去做一个尿常规,检查出肾病综合征,当时觉得整个人生都没希望了!希望各位戒友以我为戒!

    \textbf{附评} 肾病综合征(nephrotic syndrome,NS)可由多种病因引起,以肾小球基膜通透性增加,表现为大量蛋白尿、低蛋白血症、高度水肿、高脂血症的一组临床症候群。大量蛋白尿是 NS 患者最主要的临床表现,也是肾病综合征的最基本的病理生理机制。NS 时低白蛋白血症、血浆胶体渗透压下降,使水分从血管腔内进入组织间隙,是造成 NS 水肿的基本原因。中医专门讲到肾虚会导致高血压,肾脏内藏真阴真阳,是人体阴阳平衡的根本,对维持人体气机的升降平衡,保持正常血压的稳定有重要作用。肾阴不足,不能滋养肝阴,阳亢火旺,气血上逆,亦可发为眩晕,络脉阻滞,壅滞脉道,鼓胀经脉则血压升高。同时肾阳是人体一身阳气之根本,肾阳虚,阳气不得外达,失却温煦作用而脉络拘急,气血壅滞于内不得外达,可见脉沉、血压升高。以前我做健身教练时,新会员进来会给他们量血压、测体脂和测柔韧性,我那时也经常给自己量血压,我发现我的血压有点偏高,不太正常,但那时我很年轻,身体也很强壮,所以并未多想,那个阶段我处于纵欲的状态,虽然肌肉强壮,实则外强中干,脸上气色也不算好,所幸那时没有得上严重的肾病。这位小戒友才 17 岁,但已经得了 NS,报应迟早会来的,也许一直没什么大碍,但积累到一定程度突然来个大病,那打击就很大了。总的来说,肾病综合征比慢性肾炎要轻,积极治疗还是有望恢复的。17 岁就得 NS,应该和体质、手淫恶习、生活习惯等因素都有关系,现在这个时代色情的诱惑实在太猛了,疯狂纵欲几年,加上熬夜、不好好吃饭,身体保不准就会得上大病,到时就苦大了,各位戒友,一定要警惕啊!
\end{case}

\begin{case}
    我今年十八岁,手淫史已经有三四年了,那时候什么也不懂,就沉溺在其中了,前段时间节食减肥,身高 \SI{192}{\centi\metre},体重 175,我觉得自己有点胖,所以减肥,两个月时间减到了 140 斤,当时很高兴,而且同时每天也手淫。直到一个月前,嘴里面口腔溃疡一大块,然后低烧了十天左右,那时候胆战心惊的,害怕会不会是白血病,在低烧了第六天,颈部肿起来了一个鸡蛋大小的淋巴结,紧接着全身都有花生米大小的淋巴结肿大,腋窝、腹股沟、颈部。打了四天消炎针吃了各种消炎药阿莫西林等等,没有作用。于是去了市级最好的医院做检查,血常规做了,CT 拍了,找的熟人说是没事估计就是炎症,结果还是害怕,就做了穿刺,从我的颈部淋巴结取出来了三条肉条,去做化验,结果出来了,说是不排除肿瘤可能,这一下给我妈吓得,医生说要做免疫组化才能看出来是肿瘤或者白血病等等,结果我说是不是能确诊,医生却说不一定,如果检查不出来,得做基因重排,我的天呐,还是我妈又找了个熟人打听了一下,说是免疫组化百分之九十五都能确诊,这才放心做了免疫组化。于是等待了三天后,我的身体有所好转了,从 \SI{38}{\degreeCelsius} 到 \SI{37.5}{\degreeCelsius} 又到 \SI{37.2}{\degreeCelsius} 在好转,我妈这三天也没跟我闲着,去找了一个六十多岁的老中医,还是比较有名的,几十年了,治好了很多人。他把完我脉后说,是不是腰疼,我说没有啊,他说以后不要手淫了,我惊了,真是一语点醒梦中人!他给我开了抗病毒的中药顺气血的中药,败火的中药,还有活血化瘀的散结的药。我吃了几天,身体越来越好,最开始那会等个电梯我都气喘头晕要蹲着才行,后来可以散步了可以做做运动,身体正在向好的方面发展。最终呢,拿着各种化验单子去肿瘤科,医生说我是 EB 病毒引起的单核细胞增多症,这个就会引起我持续低烧,医生说我最近是不是做了什么让免疫力降低的事,我说减肥,但是我隐隐知道,手淫也会导致慢性免疫力下降。唉,我的病还没有完全好,眼看着就要过年了,给家里人添了这么多乱,很羞愧,真的打算戒撸,以后再也不碰这些东西,每天安排得满满当当,不给自己留空闲时间,不去想不去看,心静人自净。希望看到的人可以引以为戒,手淫导致的免疫力下降是确确实实存在的,一旦肾虚就会阳虚,体内阳气不足,病毒自然容易侵蚀体内。

    \textbf{附评} 这个案例节选自一位戒友的帖子,手淫会导致免疫力下降,我在初中和高中时就深有体会了,初中沉迷手淫后,变得容易感冒,鼻炎也加重许多,经常不通气,为此还做了下鼻甲切除手术。高中有次手淫完猛做俯卧撑,其实身体已虚,不宜剧烈运动,结果第二天颌下腺就肿大了,两边都肿起来很大一块,记得那时颌下腺肿大了好几年,一直没有确诊,每次手淫后症状都会有所反复,转头都有点别扭,给人感觉脖子梗着,不自然,后来上了大学去看医生才最终确诊,通过热敷就好很多了。后来淋巴结也肿大过,很硬的一粒,当时以为是不治之症,很恐慌,检查报告出来,确诊为炎症,我长舒一口气,好像死刑犯获释一般。这位戒友也经历了恐慌的体验,搞得胆战心惊的,他就是因为节食加上每天手淫,因为节食,所以营养没以前足了,而且每天还手淫,耗损却一直在持续,这样身体怎能吃得消?即使是十八岁的身体也经不起这样的摧残啊!老中医一语道破,点醒梦中人!手淫是容易出现腰痛的症状,因为:腰为肾之府!我以前也出现过多次,但也不是每次手淫后都会腰痛。老中医把完脉后肯定知道是肾虚的脉象,这么年轻肯定是手淫恶习导致的,所以叫他不要手淫了,“真是一语点醒梦中人!”之前浑浑噩噩地撸,还以为手淫无害,等到症状出来了也没和手淫联系在一起,最后还是老中医点醒了他。中医认为,肾精充足则身体强健,五脏六腑功能正常,精力充沛,神采飞扬;肾虚则免疫力下降,会导致各种疾病缠身。手淫就是让免疫力降低的事,是伤身败德的恶习,一定要力戒!狠戒!痛下决心来戒!一位戒友说:“我以前几乎一天一次,依然身高 182,考上了国内排名前五的研究生,工作签得也很好,但工作第二年,也就是 SY 的第 16 年,症状全面爆发,就一个神经衰弱就折磨得我生不如死,这东西早晚出问题。”有的人基因好、热爱运动、营养也好,所以虽然手淫,但也能长到理想的身高,然而骨质却不行,容易骨折,就像豆腐渣工程一样。肾主骨,手淫是会影响骨骼发育的,的确会影响身高,不过是因人而异的,毕竟影响身高的因素有很多。戒色后可以感觉到骨头在变硬,好像骨头里面正在生出一股内力,浑身都充满了力量。撸者则会感觉到腿软无力,骨头关节嘎嘎作响,好像骨头变脆了,也像机器缺少润滑油一样。上面那位戒友说“这东西早晚出问题。”说得一针见血!有些撸者因为底子厚、智商高,所以即使在撸管的情况下都能考上名校,甚至都有考上博士的,人生也曾风光过,但是邪淫迟早会拖累他的,恶果是逐步显现的,量变产生质变,一旦超过了临界点,那就生不如死了,身陷症状地狱,痛苦啊!到时真是度日如年,感觉人生极度灰暗,没有任何希望。戒色要趁早,真的刻不容缓!必须下大决心来戒掉它!这是一场硬仗!像个男人一样去战斗!拼死捍卫你的肾精、你的健康、你的正能量!
\end{case}

下面步入正文。

鉴于某些人的歪曲理解,这季对断念口诀再做一个澄清和详细的说明,有些人智慧有限,对断念口诀产生了严重的误解,有必要澄清一下,通过这季的论述也可以加深大家对这个口诀的认识和理解,这个口诀对于戒色和修心是至关重要的。我戒到现在差不多有八年了,一次未破,和我熟练运用断念口诀是密不可分的,这个口诀的威力非常强大,很多资深戒色前辈在分享戒色经验时都提到了这个口诀,大家看过精品帖应该都知道的,他们都是断念口诀的受益者。

戒色要面对一个终极课题,那就是心魔来袭怎么办?不管何种戒色方法都绕不开这个课题,而断念口诀正是为了降伏心魔,主宰内心!

\begin{quote}\bf
    断念口诀为:念起即断、念起不随、念起即觉、觉之即无。
\end{quote}

\paragraph{念起即断}

修行的核心是修心,修心就是要斩断妄念的相续,虚云法师:“古来祖师作为,如何直截了当,无非都是教人断除妄想。”元音老人:“不令妄念相续,把妄念斩断。”大家戒色后,邪念、怂恿念会不断入侵,这时候就要斩断相续,要严格做到念起即断,不能令其发展壮大。

\subparagraph{误解之一}

把断念误解为压念,压念是不让念头起来,有一种主观压制的想法,就像搬石头压草,这样是不行的,事实上也压不住,因为念头会自动冒出来,有压念的想法本来就是一种思想误区。有的人心里会想:“既然邪念不好,那就要压住它,不让它起来。”这就是压念,真的能压住吗?其实根本压不住,邪念还是会一个个冒上来,袭脑的速度非常之快。压念会导致内心过度紧张和挫败感,越想压制越压不住,而断念则是念头尽管起来,一起就断掉,不被念转。这就是断念和压念的区别,是断掉它而不是刻意压制念头,某些人会把断念误解为压念,这是严重的思想误区。

\paragraph{念起不随}

念头起来了,不理,不随,不跟着它跑,它自然就化于无形。

有的大德会说,不要抵抗妄念,不要与妄念斗争,只要念起不随,做到不要理睬妄念即可,而有的大德则说要时时刻刻与自己的妄念作斗争。印光大师:“当杂念初起时,如一人与万人敌,不可稍有宽纵之心。否则彼作我主,我受彼害矣。若拌命抵抗,彼当随我所转,即所谓转烦恼为菩提也。”又云:“非将死字挂在额颅上,决难令妄想投降。妄想既不能投降,则妄想成主,本心成奴。是以多少出格英豪,被妄想驱逐于三恶道中,永无出期。可不哀哉。”为什么有的大德会说不要抵抗,不要斗争,而有的大德则说要抵抗、要斗争呢?这不矛盾吗?到底要不要抵抗和斗争?后来我悟明白了,其实“不随就是断”,不随就是斩断妄念的相续!这其实就是一种抵抗和斗争,只不过大德针对“不随”这个角度,会说不要抵抗和斗争,实则也是为了斩断妄念。

\subparagraph{误解之二}

看到不要抵抗、不要斗争的开示,就以为抵抗和斗争是错的,这其实是智慧不够,理解片面,如果看过很多大德的开示就知道,这两种看似矛盾的说法都是经常出现的。有的大德偏重于说斗争,有的大德则偏重于说不随,两者其实都是对的。严格来讲,不随其实就是一种斗争,也是为了斩断妄念。

\paragraph{念即起觉,觉之即无}

这八个字实在太厉害了!圭峰禅师云:“念起即觉,觉之即无;修行妙门,唯在此也。”

\begin{quote}\it
    瞥然一念生起不要理它,念起即觉,觉之即无,不起相续心,不要让妄想持续不断地盘旋在心头,要时时有觉知,时时有观照,时时保持专注、清明、绵密的状态。(净慧长老)
\end{quote}

\begin{quotation}\it
    问:当念头起来的时候,要如何控制它?怎么样不跟坏念头跑?

    宣化上人:念起即觉,觉之即无。
\end{quotation}

这八个字就是实战最根本的指示,要比“不随”更直接、更主动,不随是不去理睬,而念起即觉,觉之即无,则是主动消灭。

这十六个字的口诀其实是一个整体,念起即断是总原则,做到不随即是断,念起即觉,觉之即无,也是为了断念。你做到觉之即无了,其实也就做到了念起不随,所以这十六个字是一个整体。

《了凡四训》:“大抵最上治心,当下清净;才动即觉,觉之即无。”说的其实就是断念口诀,念头才动,立刻觉察到,一觉察到,念头就灭了。古德云:“不怕念起,只怕觉迟”,要“觉”得快,不要让邪念相续,不要让邪念发展壮大,从小火星烧成森林大火就难以扑灭了,断念贵早,断念贵快。觉迟是因为缺少观照,缺少警惕,觉而不断则是因为断力不够,缺少练习,断念是需要坚持练习的,所谓久练自化,熟极自神!刚开始很多人都做不到,但是后来他们坚持练习观心断念,慢慢就能做到了。一旦做到了,再继续精进练习,就会越来越强,心魔越来越难以攻破。元音老人开示过:“念一来,就觉而化之。”最厉害的一个字就是“觉”,一记觉察,念头就被消灭了。

《菜根谭》云:“一起便觉,一觉便转,此是转祸为福、起死回生的关头,切莫当面错过。”《菜根谭》讲的降伏心魔的方法,原理和断念口诀也是一样的。

王阳明在《传习录》中说:“随他多少邪思枉念,这里一觉,都自消融。真个是灵丹一粒,点铁成金。”说的还是断念口诀,断念实战就是这么干净利落,一觉,即胜!

断念口诀的十六个字,字字千钧,字字重如泰山!当初我戒色没多久,看了一些大德的开示,他们在讲到断念实战时,大多都提到了这个口诀,要么是念起即断,要么是念起不随,要么就是念起即觉,觉之即无,当时我敏锐地感觉到这些关于断念实战的开示极端重要,我也亲身体会了断念口诀的巨大威力,于是我把它们组合在一起,就是这十六个字,这就像修心的武林秘籍一样,非常珍贵的口诀。参透这个口诀,练习这个口诀,达到高水平,最终就能立于不败之地。其实断念口诀已经流传了上千年,有很多古今的大德都提到了这个口诀,我看他们的开示,特别注重断念实战的部分,因为一切的一切最终都要在实战上检验,来不得半点虚假。

\begin{quotation}\it
    妄起即觉,觉即妄离。(虚云法师)

    念起即觉,觉即照破。(憨山大师《观心铭》)
\end{quotation}

这些开示和断念口诀的字有所不同,但含义如出一辙,我看过很多开示,也不一定是四个字那种,而是用一段话来描述断念口诀的意思。不知多少大德的开示都在指向这个断念口诀,这个口诀极具威力,是处理念头的金口诀!!!这个口诀的价值超过万亿兆!!!真正识货的人一定会把这个口诀奉为至宝!

《寿康宝鉴》里的一则典故:
\begin{quote}\it
    明朝邝子元,有心神昏滞、精神衰弱恍惚的病症。向一老僧请求治病药方,老僧说:“人之所以心神得病,大多起因于妄想杂念,必须要学习念起即觉,斩断妄念,才能祛除病源。有你现今的病,乃因淫欲过度所致。身体沉溺于淫色而耗损元气,这是外感之欲,深夜睡眠前后,思念美色,乃至于犯淫梦中,这是内生之欲。这两种淫欲纠缠并作,必然会耗竭元精,增重疾病,伤害性命。要想治病,必须身体戒除淫行,内心断除淫念,如此则能肾水不败,心神清明,久而久之,病症自然消失。”邝子元听了老僧的教示,回去后远离淫色,清静自心,扫除欲念。如此,经过一段时间持续不断,终于心病不药而愈,身心康健,实在归功于戒淫色、除妄念的良好功效啊!
\end{quote}

这个典故是非常经典的,专门讲到了断念,“必须要学习念起即觉,斩断妄念”,戒色肯定要断念,国外戒色文章也讲到了要断念,真正有智慧的人肯定懂得断念口诀的价值,一些无知之辈只会误解和贬低断念口诀。有的人虽然承认口诀很好,但认为大家做不到,其实是他自己做不到!事实上,很多人都做到了,这个口诀讲的是断念的原理,并不是很难,大德开示里经常提及这个口诀,一开始也许会觉得难,那是因为你从来没有向内看,还不习惯反观内心,当你学习和理解了断念的理论,然后勤加练习,慢慢就能从初学乍练到登堂入室、渐入佳境。

这季不仅会澄清误解,也会结合一些案例对断念实战做进一步的分析,比之以往的断念内容有所补充和完善,特别是一些细节,至为关键!

\begin{case}[断念不能晚]
    今天非常突然心魔起念,我断念晚了几秒就是这几秒让我忏悔不已,心魔起念的对象是我姐姐……整个下午到现在觉得自己禽兽不如!
    \subparagraph{解析} 断念实战中,很多戒友都有这样的体会,断意淫慢了点,就感觉身体的能量在往下走,同时睾丸也松了,那种感觉十分微妙,就因为慢那么一点,就会感觉身体隐约有些不适,能量已经被耗损了,再发展下去直接就漏液体了,到时耗损就更严重了。心魔会起各种变态的念头,这类念头会让自己感觉惭愧,甚至也会让自己感到害怕,觉得自己怎么会起这种念头呢?断念如果晚了,意淫就开始进入各种龌龊的情节了,像小电影一样在脑海中播放。断念一定要快、狠!不能晚!时刻保持警惕,必须严格做到念起即断。
\end{case}

\begin{case}[加过壳的念头]
    最近经历了好几次惊心动魄的实战经历,脑海中浮现看似正常的场景或句子,于是跟了,突然间就变成了意淫,真是防不胜防,还好我及时警觉,否则后果就不堪设想了。
    \subparagraph{解析} 邪念入侵头脑就像木马病毒入侵电脑一样,两者具有高度的相似性,一般的病毒都会被杀毒软件查杀,黑客为了突破杀毒软件,会用免杀技术,就是为了避免杀毒软件的查杀,给木马文件加了一个壳,举例来说,如果说程序是一张烙饼,那壳就是包装袋,可以让你发现不了包装袋里的东西是什么。黑客将一个木马文件加上壳,这样杀毒软件就识别不出,心魔黑客也的确掌握了免杀技术,一个看似很正常的回忆场景或者念头,只要你跟了,后面紧跟着就是意淫!心魔给那个意淫加了一个看似正常的“壳”!!!你不知有诈,跟了那个念头,结果就是一连串的意淫!这也是我最近结合实战体验和戒友反馈,研究出的心魔的一个套路。知道心魔这个套路后,你就会对脑海中看似正常的场景和念头,也会提高警惕,你知道不能随便跟,一跟就容易出问题。对于那些自动冒出来的场景或念头,一定要格外警惕,很可能来者不善!识别念头的能力要增强,一双火眼金睛实在太重要了!
\end{case}

\begin{case}[压缩过的念头]
    飞翔哥,我发现有时邪念来得非常之快,就是一个非常微妙非常微妙的念头,一下子就一掠而过了,而且是一个细微的念头里面包含了巨量的信息!那种感觉体验过才能明白,不太容易表达,就是一个很短很短的念头里面包含了非常多的信息,好像那个念头是极度压缩过的,那个念头经过之后,我隐约地察觉到,然后用语言把它口述翻译出来,才发现那个念头是心魔伪装而成的。当那个念头经过而我第一时间没有察觉到的时候,我就感觉大脑好像被输入了一段指令,然后那段指令解析后包含了很多的执行动作,当那个念头一经过我就感觉里面包含了很多信息,好像一大段话被压缩成了一个几毫秒短的念头似的,心魔把怂恿的内容以微妙感觉的方式向我发送,真的阴险!好在我比较警惕,直接大声地把心中那股不自然的微妙感觉口述出来(这是我有时自省过错,发掘藏得很深的心中邪念、负念的方式),把它从我的脑海里揪了出来,才发现、识破它,心魔真可怕!
    \subparagraph{解析} 这位戒友的体会很细腻,我有时的实战体验也和他一样,的确是一个很细微的念头里包含了巨量的信息,从脑海里一掠而过,速度非常快,你只要一跟那个念头,就会弹出一连串的念头,就像解压了一个压缩文件,一下就弹出非常多的信息。念头入侵时,应该要及时断掉,你不跟它就是断,你觉之即无也是断,念头就是指令,包含了破戒的执行动作。显然心魔是在向你输入破戒指令,如果你有觉照,有断念能力,那就可以断掉那个指令。为什么很多破戒的戒友都说身不由己,一旦指令输入成功,就感觉身体不是自己的,心魔已经获得了操控权,这时什么大道理和危害都抛之脑后了,不管不顾了,一心只想着撸,那种状态很疯狂,可以连续看黄一下午乃至通宵,简直就是走火入魔了。我们平时的每一天都要保持高度警惕,牢牢看住自己的念头,一定要警觉和小心,破戒其实就在一念之间,那一念没断掉,就会被心魔附体。这位戒友的实战意识还是很强的,警惕性很高,他的这段反馈应该会给大家很大的警醒和启发。
\end{case}

\begin{case}[狡猾阴险毒辣的心魔]
    飞翔哥,心魔好阴险,之前有个怂恿念头在心里有反复(破戒与不能破戒的反复),但最后被我击败了,过了几天,心魔又说:“你之前答应我破戒,后来又反悔了,你不讲信用。”由于我平时是个很讲信用的人,差点被攻破,好在最后我明白了心魔就如同二战时阴险的日本鬼子,不应该和心魔讲信用,讲仁义道德,应该立断,没破但真的好险,同时希望飞翔哥再点拨点拨。
    \subparagraph{解析} 心魔是非常狡猾阴险的,无疑是一个强大的对手,不仅会怂恿你破戒,还会向你发送诽谤的念头,诽谤佛菩萨,诽谤大德,诽谤戒色前辈,非常非常阴险毒辣!千方百计动摇你的决心和立场,无所不用其极!而且心魔的劝说很有针对性,这位戒友是一个很讲信用的人,所以心魔用“你不讲信用”来针对性地劝说,差点就攻破了,好在这位戒友最后明白了心魔的套路,没有上当。对于心魔的怂恿,应该立断!不应该陷入辩论,心魔特别擅长诡辩,想着法儿劝你破戒。
\end{case}

\begin{case}[不要听信心魔的怂恿]
    戒了大半年多了,这一个月来破了四五次,根本控制不住自己,每次都是周末,破戒前心魔告诉我,“就看看,不撸”,看着看着就破戒了,现在真是欲哭无泪,二百多天的成果泡汤了,真的感觉自己又要被拽回原来那个怪圈,太恐怖了!我真不想再受那样的折磨了!现在学习很紧张,上了高中,也没什么时间看文章了,心魔狂轰滥炸般地袭击我,求各位出个招啊!
    \subparagraph{解析} 这位戒友本来戒得不错的,可惜没有识破心魔的怂恿,心魔惯常的怂恿就是:“就看看,不撸”、“最后一次”、“偶尔一次没事的”、“戒这么久了,也该放松放松了”、“只是这一次而已,下次再戒”,其他各种怂恿还有很多很多,有时很具有针对性,关键自己要识破怂恿,不要上当。自己也要坚持学习提高觉悟,如果有正知正见,心魔很多的怂恿就会自动失效,它之所以怂恿你,有很多情况就是你对某些问题不确定,缺少正确的见解,从而导致你犹豫不决。有时你并未破戒,但心魔会怂恿:“你破戒了!你失败了!”这是非常狡猾的怂恿,让你觉得自己失败了,然后灰心丧气,放弃抵抗,自暴自弃。这位戒友上高中,学习紧张,但再紧张也要抽空学习戒色文章,不少高三戒友也在坚持学习戒色文章,自己要做好时间管理,挤时间学习,利用各种零碎时间来学习。二百多天的成果不会一下归零,但应该避免连续破戒,重新被拽回原来那个怪圈,就像获得自由后再次被关进了暗无天日的地牢!实战教训的确很深刻,当那个怂恿的声音在内心响起时,我们每个人都应该格外警惕,立刻断掉,坚决不听信。
\end{case}

\begin{case}[不要听信心魔的怂恿]
    当时的自己戒色 169 天了,精气神十分充足,当时自己受了心魔的怂恿,认为自己戒了这么久了偶尔撸一次没关系的,于是搜黄看了,看着一半刚准备撸管妈妈突然叫我去吃饭了,吃了饭我醒悟了一些,回想 169 天差点毁于一旦。但心魔没放过我,它疯狂为我找借口疯狂怂恿,一排排炮弹落向我的阵地,我还是没有守住。终于,我还是破了,这一破,就是一年五个月。有时候真的是一念天堂,一念地狱。
    \subparagraph{解析} 我又想起了《四十二章经》那两句经典的话:“慎勿信汝意,汝意不可信。”心魔会冒充你,以第一人称向你说话,如果没有这方面的警惕和觉悟,那就很可能会上当,以为是自己的想法,然后就认同了,于是开始疯狂找黄疯狂手淫。我们对于心魔惯常的怂恿一定要熟悉,前辈的文章里之前都讲过的。为什么前辈那么强调学习,有很大一部分原因,就是戒到一定程度你会遭遇心魔的进攻,而那些套路在前辈文章里基本都有讲到,学习可以让你知己知彼,让你百战不殆,坚持学习也可以让你保持良好的戒色状态。这位戒友戒色 169 天,心魔一个常规的怂恿套路就把他攻破了,觉悟还是不行,而且一破就是一年五个月,之前的好状态彻底找不见了,一旦出现破戒,必须认真反省和总结,加强学习补强觉悟,这样才有望东山再起。
\end{case}

\begin{case}[不是口诀没用]
    飞翔哥,我已经破戒很多次了,每次都是在戒色过程中脑子里经常出现以前见过的诱惑图片,在梦里也经常能梦见,所以经常导致破戒,虽然这次比以前强了,但是这几天脑海中从早到晚一直有邪淫的画面,导致我没挡住心魔的进攻,到时候口诀也没用,遇到这种情况我真不知道该怎么办,以前都是很快破戒,这次进步了,和心魔抗争了好几天,但我不可能 24 小时一直看戒色文章吧。
    \subparagraph{解析} 图片式进攻是很厉害的,一般是近期看过的诱惑图片会反复进攻你,不一定是主动去看的,也可能是不慎看到,当时能避开,但是过后心魔还是会用图片进攻你。还有就是回忆中印象比较深刻的图片也会经常进攻你,企图把你攻陷!夺取身体的操控权!图片袭脑也是相当快的,一不留神,你就跟了,你就被带跑了,然后一帧一帧在脑海中播放,你开始意淫了。为什么总是失败?自己要好好总结经验教训,最根本的一个原因就是比心魔弱,降不住它!就像格斗比赛一样,一方压着另一方打,猛揍对方,如果另一方很强大,就能摆脱被动挨打的局面。我一直很强调练习断念,平时熟背断念口诀就像在磨刀一样,你可以试试普通的念头能不能立刻断掉,如果普通的念头都断不掉,何况那种诱惑性很强的念头或图像呢?有的戒友会说口诀没用,其实并不是口诀没用,而是缺少练习,临阵磨刀太晚了,在平时就要把断念之刃磨得吹毛即断,无比锋利,到时念头怪一上来,立刻叫它身首异处!这位戒友和心魔抗争了好几天,虽然有进步,但实力还是不行,断念高手都是压倒性的胜利,刹那间解决战斗,不会出现抗争好几天的局面,不会陷入拉锯战。另外戒色并不是要 24 小时一直看戒色文章,戒色只是生活的一部分,每天拿出一些时间看看戒色文章,做做戒色笔记,复习一下笔记,有心得体会也可以写一下,其他时间保持警惕即可。

    一位戒友说自己:“每天背断念口诀一个小时,最后还是破戒了。”这就是练习不得法,要从自己身上找原因,不能怪断念口诀,有些戒友鹦鹉学舌,并未真正领悟断念口诀的含义,实战时还和过去一样,虽然每天在练习,但是练习的效果和质量如何?自己要认真反思。如果每天练习一个小时,其他时间跟着念头跑,这样肯定是不行的,其他时间也要保持观心断念,就像杀毒软件要一直开着。练习断念口诀是为了提升实战水平,一定要严格按照口诀的意思去做,邪念入侵时一定要及时断掉,不能跟着跑。国外的戒色文章也讲到要断念,要控制念头,如何提升断念水平是一个核心的问题。有的戒友练习口诀也是马马虎虎,一点不用心,这样怎能保证效果?上季一位戒友也每天练习断念口诀,但还是反复破戒,原因就是沉迷游戏,他所谓每天练习口诀也是马虎了事,一点不认真。练习是为了强大,要强大到一定程度才能降伏心魔,所以要坚持练习,持之以恒。有的人也练了很久,但水平难以取得突破,进入了瓶颈期,这也是一个问题,他们缺少一次顿悟,对断念缺少更深刻的理解和领悟,有了这次顿悟,练习的效果就会突飞猛进,实战能力就会与日俱增。如何获得顿悟?那就是加强学习和复习笔记,注重积累,这样很快就会有顿悟。

    上季一位戒友在我帖子里分享了一段话:“林志炫年轻的时候,接受节目采访时,当时接受高人指点刻苦练习后,就有很厉害的唱功,但他说自己还有功课要练习,练习,不断地练习,永无止境,现在他还在练习他的功课,还在突破他自己。唱歌功夫尚且如此,更何况我们修心的内在功夫呢,不断地练习,观心断念,觉察消失的功夫,永无止境,修行就是跟心魔在磨,一次次地磨。”林志炫在节目中说:“我觉得这个练习是永无休止的,当时我认为是我的极限,可是过了几年之后,我发现那是我的瓶颈,所以后来就再突破。”在节目中林志炫说自己出道 18 年,还在练习,真的是练无止境!精益求精!不断在突破!断念也是这样,所谓“功夫一日,技进一日,功夫无息法自修。”最近有天早上我刚醒,处于有点迷糊的状态,心魔进攻了,一幅邪淫回忆的图像上来了,我立刻觉察,图像就消失了。有戒友也向我反馈过半夜或者刚醒处于迷糊的状态,这时邪念一上来,把他带跑了,心魔趁你处于警觉不高的状态来进攻你,心魔很会挑选进攻的时机,我们一定要善于总结实战的经验教训,即使在迷糊的状态下也能保持一定的警觉和快速反应、快速断念的能力。
\end{case}

\begin{case}[在起心动念处修,对境时要格外警惕]
    各位戒友兄弟前辈,本人 28 岁,手淫十几年,2017 年来到戒色吧,戒了两个多月,但是昨天不小心破戒了,真的很苦恼,之前的努力都白费了,接下来的戒色状态又得很努力调整了。当时破戒的情况是,以前断淫用口诀断得好好的,一有意淫念头来了,还没发展下去就被我断掉了,但是不知怎么搞的,这次我因为一眼瞥了下美女跳舞视频,这次就像着了魔一样,明明知道任由意淫念头发展下去就很难控制,但是这次就是不想用断念口诀去断了,潜意识里就是想看黄,这时就连我自己也搞不清怎么突然促使我像色鬼一样那么想去看黄。果不其然,看黄破戒了,而且还是连续两次,后悔了!
    \subparagraph{解析} 对境时就是检验一个人定力和功夫的时候,我们所处的时代色情诱惑很猛烈,网上更是色情泛滥,这要求我们具备更高的定力,要做到不为诱惑所动!再强的诱惑袭来,心里都能保持淡定,这是相当不容易的。美女跳舞很诱惑,遭遇这种对境,应该马上避开,避色如避箭!这个实战意识一定要反复强化,有时做得不够好,自己要反省和总结,争取下次要做到位。然后最好是断念口诀配合不净观、白骨观来对治,这样效果会更好,不净观、白骨观可以有效对治贪恋的心理,这点非常关键,为什么瞥了一眼就像着魔一样不想断念了?其实他的心里还是很贪恋女色,所以舍不得断意淫。不净观、白骨观一思维,就可以破掉那种贪恋,再加上断念口诀,就可以顺利过关了。《禅要诃欲经》:“汝身如行厕,薄皮以自覆,智者所弃远,如人舍厕去。若人知汝身,如我所厌恶,一切皆远离,如人避屎坑。”避屎坑三字太给力了!振聋发聩!《受十善戒经》:“如粪虫乐屎,贪淫者亦然。”粪虫以屎为美,实则龌龊不堪。《增广贤文》:“芙蓉白面,不过带肉骷髅;美艳红妆,尽是杀人利刃。”这句贤文告诫我们不能贪恋美色、放纵欲望。大丈夫应该有一身浩然正气,岂可把一身的精华泄在屎尿堆上?有的人会误解不净观和白骨观,其实不净观和白骨观是为了对治贪恋和邪念,矛头对准的是自己,重病沉疴要下猛药,并非不尊重女性,当你断除那种邪念了,才是最大程度地尊重女性。
\end{case}

\begin{case}[在起心动念处修,对境时要格外警惕]
    187 天功亏一篑,之前戒色都好好的,每天学习飞翔老师的戒色文章,牢记口诀,每天都做固肾功,身体恢复得非常不错,浑身充满了力量,念头管理得特别好。直到 160 天的时候,我跑步遇到了一个姑娘,聊得比较投机便相约一起跑步。开始的时候很纯洁地一起跑步,也没有出现什么异常。直到七天前她跟说我要不要在一起,因为对方也不错嘛,就在一起了。结果,之后在一起跑步就会流前列腺液,打电话聊天也会流前列腺液,就是条件反射一样,就算脑子里没想淫念也会流。并且,这几天明显看《戒为良药》少了,断了学习,导致今天接触擦边黄破戒。
    \subparagraph{解析} 能戒 187 天也很不容易,各方面都很好,可惜生活中遇见了一个姑娘,本来戒得好好的,身体恢复也很不错,出现这个事情就方寸大乱了。戒色后精气神、自信、底气等方面都会逐步恢复,也显得阳光开朗,这时候异性缘是会变好的,但是这也是一种考验,当你的定力还不是很稳固时,谈恋爱是很容易漏的,只要有一点微妙的感觉就会漏,打电话内容很正常,但只要对方的语气比较亲密或者撒个娇,这边就不对了。谈恋爱对定力的要求非常高,而且也会打乱你的生活,脑子里想的很多事情都和女友有关,很多时间和精力都要分给女友。戒色后不是不可以谈恋爱,但一定要等到时机成熟,自己也一定要把握好分寸,谈恋爱后一定要加强修心,修心功夫很深厚,警惕性很高,那就不容易漏,如果有那么一点点微妙的感觉没及时制止,那就很容易漏。
\end{case}

\begin{case}[稳住你的戒色状态]
    有一种细微的感觉,不是具体某张图片或者视频,或者文字,就是一种奇怪的感觉,这种感觉来的时候就是突然打破你的戒色状态,感觉可能会破戒,但是却找不出来心魔从哪里进攻的,就算察觉了,也有种无从下手的感觉。这种感觉来的时候以前都没有认出它,就是心态莫名其妙失衡了,最近才发现这种奇怪的感觉的出现,很莫名其妙出现,莫名奇妙地控制你,不是邪念那种心魔直接的进攻,师兄是怎么应对的呢?
    \subparagraph{解析} 被微妙的感觉攻击是比较常见的,很多戒友都有反馈,我在之前的文章里也多有提到,对付微妙的感觉,需要更强的觉察力和更高的警惕性,因为它很细微,很微妙,它不是戒色战场的子弹或炮弹,而更像是毒气,一点点渗透进来,突然打破你的戒色状态,把你控制住。我自己在实战中也经历过很多次的微妙感觉,我能感觉到它是念头形成前的一种状态,难以说清,但确实存在,像雾一样萦绕在脑海。我一旦感觉到它的出现,就会变得格外警惕,看着它,不要认同它,一般很快就会消失,有时一记觉察就消失了。我现在很警惕微妙的感觉,因为这种感觉的渗透非常厉害,它刚开始出现时是一团模糊的信息,你能感觉到它的不善,这时你已经感觉不妙了,随后等它变强了,就会展现为某个明确的邪念,速度非常之快,所以必须很警觉,慢一点就会陷入被动,要电光火石间把它灭掉,特别考验你的反应和断力。
\end{case}

\begin{case}[战胜心魔,主宰内心]
    我感觉能遇到戒色吧这个平台就是我最大的福报,记得去年的我还是个撸管肉机,我发现我真的是痛恨极了 SY,可是每当心魔来临的时候我总是可以把所有的痛苦通通忘掉,变成了心魔的傀儡,每当心魔吸光我宝贵的肾精大笑而去,我却只剩下一具躯壳深深地被罪恶感笼罩……我的 SY 史大概有九年,次数我都记不清了,最严重的时候 SY 完站都站不住,呼吸困难,好几次我都感觉一口气上不来我就死了,这是真的不夸张,几乎失去了生活的能力,有好几次躺在床上想这是我吗?这是那个曾经发誓要让家人亲人过上好日子的我吗?就这样我哭了很多次,可是我不甘心这样堕落下去,但是我拖着不听使唤的身躯不知该何去何从,也上网查过但总是被无害论洗脑找不到原因。记得有一次无意中看到一位前辈宣传戒色吧,我很好奇就搜索了一下,在此我要感谢这位前辈,当我进来的时候真的是恍然大悟!突然感觉到我不再孤单,我的心有了方向有了目标。刚开始的时候就像一块海绵疯狂吸收着戒色知识并且落实,但是哪那么容易就摆脱多年的恶习,就在我第五次破戒的时候,借助着飞翔大哥的断念口诀,我猛然地发现了心魔的存在,并开始跟心魔战斗,一次次地将心魔打败,那种感觉真的比 SY 几秒钟的快感爽太多太多。戒到现在差不多半年左右了,加上积极锻炼坚持行善很多症状都自动消失了,没有吃一粒药,当然最感谢的还是飞翔大哥和戒色吧的前辈们,戒色吧因你们而更美!
    \subparagraph{解析} 这位戒友戒色之前就是撸管肉机,行尸走肉,被心魔虐得体无完肤。当他遇见戒色吧后恍然大悟,开始疯狂学习戒色知识,一开始也破戒,但他的悟性很不错,意识到了断念口诀的价值,很快就进入了良好的戒色状态。他发现了心魔的存在,邪念一次次冒出来,一次次占领你的头脑,一次次操控你!很多人都误以为那是自己的想法,其实不是,有的人昨天还发誓戒色,没过几天一个念头上来就把他附体了,又沦为了撸管肉机。心魔就是邪念,一切负面的念头、怂恿的念头等,孔子曰:“我战则克。”《尚书》里专门提到了“克念作圣”,克在古文中有“战胜”之意。一定要战胜自己的邪念,真正的敌人在里面,征服千军万马不如征服自己的心魔!征服心魔的人胜过征服整个地球!这位戒友说:“一次次地将心魔打败,那种感觉真的比 SY 几秒钟的快感爽太多太多。”他说得真好,与心魔斗,其乐无穷,过去被心魔虐成狗,现在剧情彻底反转了,开始完爆心魔了。这种逆袭真的太给力了!绝对是荡气回肠史诗级的逆袭!过去他是心魔的傀儡,心魔一次次在他身上得逞,一次次奴役他,榨干他的肾精,现在他开始战胜心魔,主宰内心了,这才是我们应该过的生活!!!
\end{case}

\begin{case}[戒怒戒怨,保持祥和稳定]
    飞翔哥,昨天是我戒色的 26 天,是我的戒色的最高天数了,因为工作的一些原因,从早上起嗔恨心、怨心一直起到了下午,在这个过程中我忘记了保持警惕,下午的时候心魔怂恿我去买邪淫器具,第一个念头出现了我没有警觉,很快第二念第三念出现了,一个接着一个地出现就像在我脑海里面爆炸了一样,我被搞动摇了我没招架住它的疯狂进攻,回家后挣扎了一会就去买了,买完就 SY 破戒了,射出后一切都变得没意思,回想刚刚那个过程,我完全失去了理智,我被欲望蒙蔽了,好绝望,破完后就取消订单申请退款了。
    \subparagraph{解析} 心态失衡就容易被心魔攻破!嗔恨心和怨恨心是必须要克服的,德行不够就会生出许多障碍,在生活中我们肯定会遇见各种问题,人生不如意事十之八九,关键自己要有临危不乱、坦然面对的气度,苏轼《留侯论》:“古之所谓豪杰之士,必有过人之节。人情有所不能忍者,匹夫见辱,拔剑而起,挺身而斗,此不足为勇也。天下有大勇者,卒然临之而不惊,无故加之而不怒;此其所挟持者甚大,而其志甚远也。”泰山崩于前而色不变,麋鹿兴于左而目不瞬,遇事镇定自若,不受外界影响。当然说起来容易,做起来难,在一次次实战中不断反省和总结,这样就可以优化自己的实战表现,做到心平气和是非常关键的,内心一定要尽量保持祥和稳定。《菜根谭》云:“养喜神以为招福之本。”喜神多瑞,和气致祥,我们在生活中一定要善于养喜神、养和气。不生气,其实是一种修养,不怨恨,其实是一种德行,戒色要不断完善自己的修养和德行。这位戒友因为工作原因生气了,然后忘记保持警惕,心魔就趁火打劫了,开始怂恿他,他没有警觉,没有立刻断念,继而动摇了,最后被心魔攻破了。生活中有很多情况会让我们生气,让我们失控,我们一定要及时调整,避免陷入负面的心态,这点非常重要。
\end{case}

\paragraph{总结}

元音老人:“经过成千上万次的斗争,由一开始的念起不觉,到念起能觉,由觉而不能转,到一觉即转。”刚开始断力薄弱,经过练习和实战的磨练,断力就会逐步增长。元音老人是我极为敬仰的一位大德,是真正泰斗级的大德,犹如云中黄山般气势磅礴而雄伟,同时又是那么亲切和蔼,平易近人,慈悲恳切,古道热肠,感人至深。我尊元音老人为根本上师,得遇恩师实乃我无上福报,内心无比感恩元音老人的大恩大德。老人的开示谈断念实战谈得比较多,念起即觉,不压不随,觉而化之,就像太极高手战胜来敌一样,能够把对方的力给化掉,《太极拳论》云:“由招熟而渐悟懂劲,由懂劲而阶及神明。然非用力之久,不能豁然贯通焉。”在不断练习的过程中,不断地体会老师讲的内容,不断地感悟自身的内在变化,渐渐地感受到越来越多的东西,而这些东西是你以前从来没有领会到的!上乘的断念功夫需要不断练习和实战的磨练,一羽不能加,蝇虫不能落,实战反应极快,经过成千上万次的斗争,功夫渐入化境,变得越来越精于实战,战胜心魔不在话下!《功守道》里马师傅两目闭合,脑内开始了与各大高手对决的画面,在我看来,马师傅代表觉察力,而那些高手代表着各种念头,最后全部被马师傅打败了。据说马云练了三十年的太极拳,一招一式的确有练家子的风范,打斗时很镇定很沉稳,面对强敌的进攻,一点不慌乱,这非常难得。这部短片让我想起了断念,当马师傅闭上眼睛,他就进入了脑海内的实战场景,这和断念很相似,都是发生在脑海,只不过断念显得更快,不会陷入缠斗。秋风扫落叶,马师傅闭眼的那个画面拍得很有意境,巅峰对决就发生在脑海中。

俞净意公遇灶神记,相信很多戒友都看过,俞公与十余人结文昌社,还每月放生,但是命运却不好,生五子四子病夭,其第三子甚聪秀,左足底有双痣,夫妇宝之,八岁戏于里中,遂失踪,不知所之。生四女仅存其一,妻以哭儿女故,二目皆盲。俞公的命运十分凄惨,表面上貌似做了很多善事,但是问题在于他内心的——意恶!也就是各种邪念!“于私居独处中,见君之贪念、淫念、嫉妒念、褊急念、高己卑人念、忆往期来念、恩仇报复念,憧憧于胸,不可纪极。”可谓“满腔意恶,起伏缠绵”。后来俞公改了,开始发自内心来行善,善念真纯,善力精进,并且懂得了修心断念,于是他的命运改变了,中了进士,失踪的儿子也找回来了。行善一定要去恶,最关键的是去意恶!意恶是最根本的。有的戒友在提倡行善积德,学习圣贤教育,这很好,但是他们对修心断念认识不够,重视程度也不够,其实修心断念和行善积德一样重要,而且修心断念更为根本,俞公自别号“净意”,就是净化意念,断除邪念,修行是在起心动念上修的,是在念头上下功夫的,一定要懂得修心断念的重要性。

不管哪种戒色方法都有人破戒,这个世界上不存在哪种戒色方法能保证百分百成功,即使戒色方法再好,也会有人破戒的,就像学校里的老师再好,班级里还是会有差生,同样的教学内容,有的学生能考 100 分、90 分,而有的学生只能考 40 分、30 分,甚至更低。行有不得,反求诸己,要学会从自己身上找原因,自己一定要深刻反省,为什么别人断念那么强,警惕那么高,能戒几年都不破,为什么自己做不到,一定要认真反省。戒色吧就像一个大班级,优等生和差生都是有的,优等生永远属于少数,这是肯定的,比如一个班级,前面十几个都是比较优秀的,悟性高,勤奋,学习能力强。我们要向优等生学习,看看他们是怎么练习断念的。不管何种戒色方法,终归都是要断念的,邪念上来时,就看你的实战表现了。一位戒色两年的资深戒友说:“每天断念口诀的练习,只有把断邪念、断意淫的能力,练强!练快!才能断掉一切负面念头!如果你会断念了,戒色可以说已成功大半了!”他说得很好,断念是戒色实战的核心,一定要强化断念实战,理论可以很厚,几千页、几万页,但最后就看断念!就像军事书籍可以堆满一个图书馆,但最后还是要真刀真枪地干!不能纸上谈兵!平时就要实战化训练!

我强调断念口诀就是因为实战那一下子实在太重要了,可以说无比重要!很多人见了心魔稍微抵抗一下,就跪了!又沦为了心魔的傀儡,任其摆布和操控。迎接心魔的不应该是你的膝盖,而是你的断念利刃!来即斩,斩立决!必须强悍,实战那一下必须够快、够狠!练习使人强大!高手都是练出来的,希望大家好好练习断念口诀,断念水平上去后,面对心魔就有战胜的把握了,内心会有一种主宰感和力量感,不再是任心魔宰割的羔羊了。一位 6 级的戒友说:“戒色文章也有学,可为什么邪念一上来,一击即溃,没有一点抵抗力,没有还手之力。”光学习戒色文章还不行,必须练习断念,就像士兵平时要进行训练一样,不能空谈理论,空谈理论的人最后会沦为戒油子,实战表现稀烂,戒色态度也会变得很差。上季一位戒友说:“实战的时候就看那一瞬,哪怕平时 99.99\% 的时间都是正人君子,只要实战的那一刹那没有断除邪念,就会被邪念附体,瞬间变成纵欲的禽兽。”实战就在刹那,眨眼间的事情,所有的训练就是为了把握那一刹那!那一刹那你的表现要极强、极狠、极快!一位 18 岁戒色 248 天的学生戒友在帖子里分享经验说:“不怕念起,只怕觉迟,要时时刻刻警惕,时时看住自己念头,观心断念是戒色的灵魂,只要意淫关过去,戒色的天数可以说是突飞猛进,时时练习断念口诀,念起即断、念起不随、念起即觉、觉之即无,要勤加练习,要形成条件反射。意淫一出来,第一件事不是跟着淫念走,而是立刻想起断念口诀,跟着淫念走了时间越长,你就越被动,意淫刚起来的时候力量很小,你可以不费吹灰之力把它斩杀,时间一长就不容易对付了,所以对待意淫要做到不犹豫、不纵容,要立刻斩杀!”18 岁能戒 248 天,实属不易,从他分享的经验来看,他对断念的理解很深刻,也注重练习,能戒 248 天也在情理之中,相信他会戒得更好。苏炳添是练出来的,刘翔也是练出来的,金牌背后不知流了多少汗水,希望大家好好练习断念,刚开始也许感觉很难,但久而久之,坚持不懈,观心断念就会越来越强。

有的戒色文章会提倡持咒念佛,而我的戒色文章主要提倡断念口诀,这个口诀其实来自于大德开示,我只是把它们组合在一起。持咒念佛很殊胜,有佛菩萨加持,而断念口诀也很殊胜,两者都很好。口诀还有一点好处就是,它的覆盖面更广,即使不信佛也可以用这个口诀,所以覆盖面更广,接受度更高。断念的方法有很多,断念口诀、思维对治、持咒念佛等都是很好的,不可厚此薄彼,有的人提倡持咒念佛,然后误解和贬低断念口诀,这是不妥的,应该充分尊重各种断念方法,不可抬高自己,贬低别人。还有的人写文章反对断念口诀,所言完全是胡说八道,实不足信,完全是在误导大家。这季澄清了这个问题,相信大家都能认清了,对于误解断念口诀之人所写的文章,也能一眼勘破了。不过话又说回来,断念口诀固然重要,但也要注重整体思想觉悟的提升,毕竟戒色是系统工程,戒色也不仅仅是戒手淫,而是生命的全面改造、净化和提升,树立正确的价值观,戒除一切不良习气,多孝顺父母,多行善积德,多学习圣贤教育,过一种高度自律的充满正能量的生活,内心有一种纯净美好、光明崇高的感觉。

下面分享一首戒色诗歌。

\begin{poem}[一刀流]
    \begin{multicols}{3}
        \centering~\\
        刀气凝聚成形 \\ 瞬间化作一道寒光 \\ 斩灭入侵的邪念 \\ 那一刹那极快 \\ 看不到出刀 \\ 只看到倒地的心魔 \\ 和回鞘的动作 \\ 干净利落的对决 \\ 杀念的刀,觉察的刀 \\ 潜心练习,不断磨刀 \\ 这是一种修炼 \\ 应用于实战之中 \\ 反应快,断念狠 \\ 精钢缎造的断念之刃 \\ 刀身生出凛冽的白光 \\ 刀的精神,刀的杀气 \\ 刀的气魄,刀的勇猛 \\ 刀光一闪,刀已回鞘 \\ 比闪电更快 \\ 他成千上万次地练习拔刀 \\ 只为了实战的那一下 \\ 没有人能形容这一刀 \\ 他一出刀,结局已定 \\ 心魔就是来送死的! \\ 好快的—刀! \\ 这一刀威力无俦! \\ 这一刀有惊天裂地之威! \\ 他实战只出一刀 \\ 因为再无第二刀的必要
    \end{multicols}
\end{poem}

下面推荐一本书。

\begin{book}[《般若花》,徐恒志]
    本书汇集了徐恒志居士历年来的佛学著作、讲稿和书简十五篇。其中“学佛是怎么一回事”和“怎样实践佛法”系作者根据自己多年的实践,概括而精细地对这两个学佛的基本问题,作了具体说明,从而为初学者提供了必要的答案。此外本书还就学习经论的方法、般若观照的要义、明心见性、佛教的人生价值观以及念佛往生等问题作了阐述和探讨,有助于读者对佛法的理解和实践。《般若花》是非常殊胜的法宝,徐恒志老居士也是我非常敬仰的大德,1915 年出生,原籍浙江镇海,二十五岁正式学习佛法,到能海上师处去受三归五戒,法名定真。于 1953 年受心中心密法阿阇黎灌顶位后,开始应邀为上海佛教青年会讲授佛法,多年来在国内各道场、学府宣讲。2007 年 3 月 5 日安详示寂,享年九十二岁。之前看过老居士写的《我的学佛因缘》一文,看了之后很是感慨,老居士的身世颇为坎坷,也得过严重的神经衰弱,后来修学佛法完成蜕变。老人家一生“以般若为导、以总持为法,以净土为归”,显密圆通、潜修密证,严谨治学、悲心广大。其淡泊名利、简朴平易、谦和待人的独特人格魅力和高尚的精神境界,已饮誉海内外,德望所归,四众同仰。原中国佛教协会赵朴初会长赞徐老是“当代维摩诘”。卧龙山普净寺智正老法师赠联:现居士身虚怀若谷照大千当今维摩诘,示般若花实相明灯悬万古一代人天师。《般若花》里讲到:“千万个修,抵不过我一觉,觉则心空,此是最上福德。”“只要一觉,顿然光净。”“前念起时一返照,前念便空,后念起时,再返照,后念又空,这样念念生起,念念返照,便得念念空净,这实是正本清源的调心方法。”讲的就是断念实战,原理和断念口诀如出一辙。这本书网上有 \texttt{.pdf},有志学佛的戒友可以看看徐老的开示,定会受益匪浅。
\end{book}
