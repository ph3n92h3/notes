\subsection{戒色厌倦期、多汗问题、视力问题}\label{26}

\subsubsection{戒色厌倦期}

戒色厌倦期是一个不得不面对的问题,也是一个必须克服的问题,很多戒友都曾向我反映过这个问题。

如果大家常驻戒色吧,肯定会发现一个现象,那就是来来去去的人很多,很多人刚开始戒色决心极大,戒色热情高涨,无奈“一鼓作气,再而衰,三而竭”,没几周就消失了,又回到了过去,特别是破戒后的戒友,一旦破戒,戒色决心和热情都会大幅下滑,戒着戒着,就不想戒了,看不到希望,只有绝望和无奈。其实这就是戒色厌倦期,不管你是什么人,都会有厌倦情绪,这种厌倦情绪在其他领域也很普遍,比如你听一首歌,刚开始感觉非常好听,听几周你可能就厌倦了,再也找不到当初的那种感觉,剩下的就是厌倦和不耐烦。食欲方面也是一样的,刚开始吃某样美食很喜欢,叫你天天吃,可以吃得你想吐。手机也一样,刚开始特别想拥有一款手机,用久了就没那种感觉了,就会觉得这款手机也不过如此,再也没有当时那种激动的感觉。

应该这样说,人是一种容易产生厌倦的动物。

在戒色方面也不例外,很多戒友厌倦了学习,厌倦了戒色文章,懒得看、懒得学,不耐烦。一旦出现戒色厌倦情绪,是极容易导致失败的,这就像一支部队的士气不振,很多人不想打仗,想做逃兵,这样的部队根本没有战斗力,如果你出现厌倦情绪了,要学会马上克服,及时调整情绪,让自己的戒色斗志重新饱满起来。一支部队没士气,就是一盘散沙,一支部队有士气,凝聚力就强,个个如猛虎下山。佛教修行也讲究:勇猛精进,不能有懈怠心,要保持恒心。但是人难免有厌倦情绪,因为人是有惰性的,所以一旦出现厌倦情绪,我们要及时调整,就像一只表走慢了,我们要把它调整到正确时间。

下面就谈谈我的调整办法。

说实话,我也出现过厌倦情绪,但我很快就调整过来了,这里面涉及到一个概念,就是情商(EQ),情商就是情绪智商,情商包含了自制、热忱、坚持,以及自我驱动、自我鞭策的能力。情商和智商不同,情商是可以后天习得的,就是可以通过学习来获得的。当你通过学习知道该如何调整自己的情绪,那么你就能完全掌控自己的情绪,而不是被情绪控制。一个人的情绪是可以主宰一个人的行为的,很多戒友破戒都是情绪破戒,一出现无聊情绪和压抑情绪,马上转变成 SY 行为来发泄。如果你情商高,懂得调整情绪,就不会出现这种情况了,所以我们必须学会调整情绪,这在戒色的道路上是异常关键的,出现无聊情绪和压抑情绪,我们要及时调整,出现厌倦情绪时,我们也要及时调整,让自己的戒色斗志重新振作起来,戒好每一天。

不仅学习戒色文章会厌倦,解答问题也会产生厌倦,家兴有向我反映,说他对解答也厌倦了,出现了退隐的想法。这时候我建议他要注意休养,不要把自己搞得太累,在状态不好的时候就减少解答,否则带着心理疲劳去解答,只会感觉更累,这种累其实就是心累,心累是很容易起退心的,说白了,就是不想干了。

学习戒色文章也是如此,刚开始戒色热情高涨,你可以选择大量摄入戒色知识,但学到一定程度,你就有可能出现厌倦学习的情绪。这时候,你可以调整一下,把摄入量减少些,每天看一篇即可,或者每天看几个警示案例也可以。不学习的时候要注意保持高度警惕,不要让心魔钻了空子。等过了这段低谷期,你就能重新找回良好的戒色状态了。

具体的调整办法总结如下:(以跑步作比喻,希望大家能更好地理解)

\begin{itemize}
    \item 一旦出现厌倦情绪,要学会分配体力,不要把自己搞得太累。就像跑 1500 米,跑不动了,不要不跑,而是跑慢点,继续前进。
    \item 要学会激励自己,激励自己就是在调整情绪,让自己的情绪重新振作起来。好比跑步时,你告诉自己:我能行,我能跑完,我能做到。
    \item 出现负面情绪时,要让自己保持积极乐观,不要沉迷于压抑颓废的情绪。好比跑不动时,你想放弃,有个声音告诉你坚持就是胜利。
\end{itemize}

我说过,戒色后 YY 是一关,频繁遗精是一关,这是戒色道路上的两只拦路虎,如果你克服了这两关,就胜利了一半,如果你再能克服情绪关,那么你就能戒得更稳定,更长久。

\subsubsection{多汗问题}

下面再来谈下多汗的问题。

出汗过多的问题,近来提问的戒友比较多,现在虽已立秋,但天气还是很炎热,稍微一运动,还是很容易出汗的。出汗一定要分清是生理性出汗还是病理性出汗,病理性出汗也就是多汗症,原因就是中枢神经系统功能失调的表现,中医认为多汗是阴阳失调引起的。现在多汗是季节因素为主,如果你其他季节也这样容易出汗,那就不正常了,说明你身体需要调整了,最好去看下中医配合中药调理,然后坚持戒色养生,身体就能慢慢调整过来。

其实,伤到一定程度容易出现两种情况:

\begin{multicols}{2}
    \begin{itemize}
        \item 多汗
        \item 无汗
    \end{itemize}
\end{multicols}

一般以多汗比较常见,无汗相对少一些。SY 导致的症状一般都两个极端,有人出油多,有人皮肤发干,一般以出油多比较常见。医学上对于出汗是有明确分类的,有自汗、盗汗、头汗、半身汗、手足心汗等,最常见的还是自汗和盗汗。所谓自汗就是无缘无故、不自主地出汗,一般都是在白天并不炎热也没有运动的情况下发生的。盗汗医学上认为就是在夜间睡着了时候出汗,而睡醒了后汗就止了。而 SY 导致的肾虚,不管是肾阳虚还是肾阴虚都有多汗的症状,很多人出汗量极大,像全身洗过一样,而中医有讲到:汗者,精气也。大汗伤阳,如果出汗量太大,对于身体的恢复是极其不利的,所以有多汗症状的戒友最好能及时就医,配合中药调理,这样解决了多汗问题,相信你身体恢复也会上一个台阶的,会比以前恢复得更好,更稳定。

也有戒友说,为何运动出汗后很舒服,没有感觉任何不适。这只能说你阳气还是没伤到那种程度,等你伤到那种程度,再这样出大汗,对于健康是很不利的,所以身体虚的时候一般都建议静养,运动以散步为主。这个道理和遗精也有点类似,很多人身体好,遗精一次并没感觉到任何不适,但有的人以前放纵过度,现在一次遗精就马上会出症状,马上感觉不舒服。就是这个道理,伤到一定的程度,很多事情都和原来想象的不一样了。当然,适量的出汗对身体也是有利的,可以排出体内的邪毒,但是出汗量过大则会伤身体。很多人运动量一大,晚上就遗精了,有的人运动量一大,症状出现了反复,所以,当身体虚时,尽量以静养为主。当阳气慢慢养足了,可以适量做些锻炼,这样对于恢复才比较有利。我建议大家可以根据运动后身体的反应来调整,有的戒友运动后身体无任何不适,这种情况可以继续坚持锻炼,但一定要注意适量,并不是越多越好。而有的戒友运动过后有不适感,或者出现了症状反复,这种情况就说明你现在还不适宜这种运动量,可以先以静养为主,运动量先降下来,做做八段锦,还有散步,等养足肾气,再慢慢把运动量提起来,但也要注意运动适量。

对于 SY 导致的很多症状,我都是建议积极治疗的,并不是一味地要求你戒色养生,而不治疗。该治疗的治疗,再配合戒色养生,这样恢复才比较快。当然,也有不吃药,专靠戒色养生就能恢复的戒友,这类戒友一般在养生上都很有心得。这样养生功夫到位,对于恢复也是很有利的。很多戒友虽然积极治疗,但不注重戒色养生,这样的治疗效果就不会理想,因为要痊愈是三分治,七分养,养生才是重点,古代很多名医看病,对患者都有医嘱,比如:远房帏,节制饮食,按时起居,不要劳累等。很多戒友之所以慢前久治不愈,就是不懂得戒色养生,这样花费上万都看不好,如果你真正意识到了戒色养生的重要性,那结果就会完全不一样。我十几年的慢前,现在就痊愈了,化验正常,也没有症状了。所以,有症状,要治更要养,否则万难真正痊愈。很多人靠药暂时痊愈了,一放纵,马上就又复发了。

\subsubsection{视力问题}

最后谈下视力问题。

\begin{quote}\it
    眼乃一身之精华,不宜不秀。(《太素脉》)
\end{quote}

中医还讲到:五脏六腑之精气皆上注于目!所以一个人的眼睛就是自己五脏六腑功能的反映。而肾藏精,藏五脏六腑精华之气。所以 SY 伤了肾气以后,人容易出现以下几种眼部变化:

\begin{multicols}{3}
    \begin{itemize}
        \item 眼睛无神
        \item 眼睛无光
        \item 眼球混浊
        \item 眼睛血丝
        \item 眼睛变小
        \item 眼皮变化
        \item 黑眼圈眼袋
        \item 眼球呆滞,转动不灵
        \item 眼窝凹陷
        \item 视力下降
        \item 飞蚊症
        \item 结膜炎
    \end{itemize}
\end{multicols}

其中视力问题是广大戒友普遍比较关心的问题,特别是对于学生党而言,视力下降的确是一个很大的困扰。视力下降的原因很多,具体如下:

\begin{multicols}{3}
    \begin{itemize}
        \item 用眼习惯不好
        \item 用眼过度
        \item 遗传基因
        \item 沉迷 SY
        \item 眼部疾病等
    \end{itemize}
\end{multicols}

SY 应该不是视力下降的直接原因,应该是间接原因,很多人虽然不 SY,比如小学生,但他也会视力下降戴眼镜,这种情况就是遗传或者用眼习惯不好导致的。而有的人本来视力极佳,但自从沉迷 SY 后,眼睛视力下降就很厉害了,像这种情况就考虑是 SY 的间接原因,再加上用眼习惯不好,用眼过度,就比较容易出现视力问题了。通过戒色养生,视力状况是有望改善的,眼睛的不适感也会有所缓解。如果视力问题对你来说是个大困扰,那么将来也可以考虑激光矫正。

一般坚持戒色养生,人的眼睛会重新变得有神,有定力,有光彩。光彩是个很微妙的感觉,就像一个电灯泡的亮度一样,有的灯泡亮,有的灯泡暗。而人体阳气足,反映到眼睛上,就是眼睛明亮有光彩。反之,就是无神暗淡。你去观察下小孩,你就会发现小孩不仅肤质极好,另外一个很明显的特点就是眼睛明亮有神,给人一种清澈可爱的感觉。而成人的眼睛就差了很多,一方面身体漏过了,另外就是生活习惯,比如抽烟熬夜久坐等,这样身体里面伤掉了,自然会反映到眼睛上来。我那时沉迷 SY,眼睛无神,缺少定力,一种很没自信的感觉。现在完全不一样了,通过戒色养生,我的眼睛又重新变得有神起来,比起以前自信了许多,也敢和人直视了,有底气了。

我建议有视力困扰的戒友一定要建立起良好的用眼习惯,上网要控制时间,看书学习要保持良好的姿势和距离,注意休息。然后一定要戒掉 SY,学会养生,可以再加强些营养。这样视力下降的问题就会得到有效的控制。
