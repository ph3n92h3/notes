\subsection{对境实战之五大原则}

\paragraph*{前言}

戒色吧现在已经突破四十万了!记得我刚来戒色吧时,人数还不足五百人,现在发展非常迅猛,这和广大戒友的努力是分不开的,很多戒友都在帮助宣传戒色吧,不少还自掏腰包印宣传品。向这些默默奉献的戒友致以崇高的敬意!今年戒色吧应该可以冲到六十万,到时候就可以帮助更多的人了。戒色吧的不少前辈都离开了,新鲜血液正在不断注入,戒色吧现在已经是一个非常活跃的贴吧了,将来的发展不可限量。这股强大的正能量注定会帮助更多的人走出撸管的泥沼,回归阳光健康的生活。

这个时代的孩子是不幸的,因为这是个邪淫泛滥的时代;这个时代的孩子又是幸运的,因为还有戒色吧这样的网络净土。好好珍惜吧!在这里,你可以完成人生的蜕变,甚至可以获得脱胎换骨般的变化。

关于个人崇拜,我以前的文章也澄清过,请大家不要崇拜我,我只是一介具缚凡夫,很惭愧!能够帮到大家是我无比的荣幸。我希望以自己一点浅薄的经验,能够带给大家一些有益的启示,仅此而已。我希望大家都能超过我,我希望每一位戒友都能超过我!大家都能戒除手淫恶习是我最愿意看到的。你们的快乐就是我的快乐,你们的痛苦我愿意帮助分担。大家加油!好好坚持。

下面分享几个案例。

\begin{case}
    手淫有十几年了,以前没感觉手淫对身体的变化,现在三十岁了,强烈感觉身体不行了:抑郁症、神经衰弱、心理自卑(企图自杀)、大便困难、阴囊潮湿,有时还会发热,去好多顶级医院中西检查治疗,都没用。

    \textbf{附评} 症状的延迟并不代表手淫是无害的,开始手淫迟早是要还的,这位戒友的经历就是很好的明证。有的人沉迷手淫,自我感觉还行,就以为手淫是无害的,其实手淫的恶果正在逐渐累积,达到一定临界值,就会爆发出来。量变产生质变,在这个累积的过程中,很多人是“没有感觉”的,就像你感觉不到树在生长,但是树的确是在悄悄生长,过个几年就会一下变得很明显。扁鹊见蔡桓公的典故大家应该都学习过,扁鹊通过望诊,已经断定蔡桓公身体出问题了,但是蔡桓公一点感觉都没有,他说“寡人无疾”!作为患病的个体,在疾病的前期发展阶段,很可能完全没有感觉,生活中也经常有这样的例子,就是平时感觉还不错,但是一检查很可能已经患病了,甚至是比较严重的疾病,所以患病个体的主观感觉很可能是错的。很多撸者就是这样,还以为自己身体很好啊,其实已经被蛀空了,只是他自己不知道,之前就有戒友自我感觉很强壮,但是中医一把脉,身体已经失调了,就是年轻在撑着,否则早废了。在十几岁和二十出头,身体恢复比较快,所以即使有症状,很多人还是扛得住。但是当撸龄达到十年以上,神经症就很可能会找上门来,到时候就生不如死了。

    过了三十岁,身体慢慢走下坡路了,各种问题就开始暴露出来了,到时候就积重难返了,恢复的难度加大了很多。撸管的人,身体就像豆腐渣工程,说倒就倒,说不行就不行。这位戒友去了很多顶级医院都没用,不戒色养生,光靠药,是难以恢复的。必须明白戒色养生才是恢复的基础和前提,就像地基一样,切忌边治疗边纵欲,那样病情甚至会加重。上次有个戒友遇见一位老大爷,老大爷看他气色很差,就说:“小伙子,身体不行啊,跟我年轻时差远了,库存很重要,要不到老你会后悔的!”这位老大爷是位明眼人,弄不好就是位中医,“库存”一词实在精辟!很多人原来身体底子还可以,但是沉迷撸管,越撸越薄,体质迅速下降,最后症状缠身。库存真的太重要了!精库不能随便泄啊!生命的能量不能提前透支啊!\textit{肾精人之宝,不可轻放炮;惜精即惜命,固精人难老。(116 岁老中医罗明山)}
\end{case}

\begin{case}
    飞翔大哥一直在强调觉悟二字,我推荐大家下一个熊猫看书,把飞翔大哥和一些精品贴下载到手机,然后熊猫看书有个本地听书,我就是每天跑步的时候听,反复地听,感觉每次听的效果都不一样,我觉得那些看不进去的朋友可以用听书的办法,效果很好。

    \textbf{附评} 重复是学习之母,对于一件事具有兴趣,你就成功了一半。很多资深戒友都深深明白温故而知新的道理,不断重复地学习,不断思考和领悟,真正把戒色知识刻入自己的大脑,要真正达到“深刻”的程度,不能停留在表面,随便看看,走马观花,这样受持的深度就不行。很多戒友看过就忘了,也懒于看第二遍,以为自己懂了,其实根本没懂。“说时似悟,对境生迷”,真懂还是假懂,遇到魔考,马上见分晓,真金不怕火炼!一定要用心学,多做笔记,多思考多领悟,戒色文章和戒色笔记应该反复看,反复研究,反复思维戒色的道理,这样对于戒色的认识和理解慢慢就能达到“入木三分”的程度,到时候遇见魔考,就有定力了,就可以顺利过关了!一定要注意复习,要提高学习的吸收率,营养都在戒色文章里,就看你能吸收多少了。

    这位戒友的推荐很不错,下一个熊猫看书,反复听戒色文章,这也是很好的学习方式。有时候真正听懂一句,真正理解了那一句,觉悟就可以飞升,一下就领悟了,之前就是在瞎看,没有真正悟进去。真正悟进去,觉悟的提升就是突飞猛进!一定要注重“重复学习”,因为每重复一遍,就会有新的领悟和收获,看一遍和看十遍,感觉是不同的,看十遍和看三十遍感觉又是不同的。只要你对戒色文章真正具有浓厚的兴趣,那么重复学习时就不会感到厌烦,如果没兴趣的话,看一遍就会不耐烦,再也不愿看第二遍。我们必须懂得重复学习的重要性,真正钻研进去,这样觉悟的提升才会日新月异,有时候看戒色文章,往往就是第一遍没有感觉,多看几遍感觉就出来了,等到觉悟提升到一定层次,再回头复习,又会获得新的领悟,就是这么奇妙。以前你没看懂,是因为你觉悟尚浅,未能充分领悟,等觉悟上去后,再回头看,就能突然看明白。不少戒友都向我反馈过,说以前看某篇戒色文章,忽略了很多内容,过段时间再看,马上就领悟了,事实就是如此。破戒前看和破戒后看,又会很不同,破戒后再反省再看,往往比之前会更深刻。
\end{case}

\begin{case}
    飞翔老师,我今年二十四岁,撸龄十年,由脾肾两虚到现在五脏都虚,现在的天气全身都怕冷,浑身疲乏无力、头昏眼花、听力下降、失眠,身体一直在消瘦,身高 175 \unit{\cm},体重 100 \unit{\kg},阳痿早泄、尿频尿急、左侧精索静脉曲张,已手术。精神上焦虑、抑郁,根本没办法正常工作,已经治疗两个多月了,身体素质反而变差了,现在连锻炼的力气都没有了,不知道还能不能活下去,每天活得生不如死!

    \textbf{附评} 这也是一位生不如死的戒友,我看过不少受害者的案例,到现在至少看过几万例了。其中很多人的生活质量都直线下降,症状缠身,学业和事业都受到很大影响。身心遭到严重摧残,各种症状百般折磨,这的确是生不如死的生活。撸出神经症,那就更难熬了,十年撸龄,最后把自己逼到崩溃的边缘。医药费也是一笔很大的花销,因为经常看病,很多戒友和家人的关系也变得很紧张,甚至和父母经常吵架,觉得父母不理解自己,而父母则觉得你在装病。撸到后来,真的是一种很苦恼很悲催的生活状态。

    人生会因为撸管而变得无望和颓废,身体垮掉了,脑力泄掉了,人生还有什么指望?很多人还没结婚就废掉了,还没结婚就一身的症状,他根本不敢想将来。在这种人生的低谷,一定要咬紧牙关,下定决心来戒色,另外应该配合积极治疗,多在养生方面下功夫,这样身体会慢慢恢复正常的。我那时也是连锻炼的力气都没有,走一段路就觉得很累,睡醒了还是感觉很累,我是坚持戒色养生一年多才缓过来的。我是从症状地狱爬回来的人,我太清楚那是怎样的痛苦和折磨,生不如死一点不为过。撸管的人生,就是苦逼的人生,众苦相逼,生不如死!希望撸龄较短的戒友能够吸取前辈的惨痛教训,早日戒除手淫恶习,早戒是绝对明智的!撸管是人生的暗礁,撸管也是人生道路上的地雷,总之,撸管恶习实在碰不得!
\end{case}

\begin{case}
    有人说戒色初始决心很重要,会直接影响戒色天数,而且能在想撸的时候控制住自己的心魔,而且我就是刚开始下了很大决心,真的能克服住破戒的念头;可又有人说如果开始下决心很大,破戒后会造成自暴自弃,反而会适得其反。我想问一下,飞翔哥赞成那个观点?

    \textbf{答} 嗯,这两种观点都有道理的。初心对于戒色是非常重要的,初心的表现就是你的戒色态度和决心,是不是想彻底戒掉,有多想?决心有多大?这对于戒色是非常关键的,态度决定高度,态度也决定你可以走多远。但是,有一句话叫希望越大,失望越大,我们也应该充分认识到戒色的道路是曲折的,不是一帆风顺的。破戒是非常有可能出现的情况,如何正确应对破戒,破戒后心态如何调整,破戒的原因分析等等,这些都异常重要。如果破戒后自暴自弃,那就会越戒越差。我们要认识到,破戒不仅仅是一次打击,更是一次难得的学习机会,好好总结破戒的经验教训,不断补强觉悟,下次就能越戒越好。加油!

    \textbf{附评} 戒色高手基本都经历过破戒,就像打游戏的高手,刚开始基本都被虐过,大家都是从菜鸟开始的,通过不断学习提高觉悟,才慢慢进阶到高手行列的,谁都不是一生下来就是高手的。有的高手会这样说,自己之所以成为高手,就是吃亏吃出来的,吃一堑长一智,犯错使人聪明!关键是你不能总犯同一个错误,那就不是聪明了,那是不开窍!失败并不可怕,关键是要学会总结失败的经验教训,这样才能越戒越好,比如很多炒股大师都经历过亏钱,有的甚至经历过破产,但是最后他们依然获得了成功,正是亏钱的经历让他们不断升华,直到扭亏为盈!失败可以打败一些人,但失败也可以让另外一些人变得更强。这就是人与人之间的区别。破戒后,你是否认真反省、你是否愿意总结、你是否愿意不断纠正错误、你是否能够调整好心态?从对待破戒的态度,就可以看出这个人可以戒多久。破罐破摔的人,只会越戒越差;善于总结和思考的人,就会越戒越好。

    \begin{quote}
        There's a phrase in Buddhism, 'Beginner's mind.' It's wonderful to have a beginner's mind.(佛教中有一句话,初学者的心态,拥有初学者的心态是件了不起的事情。)(乔布斯)
    \end{quote}

    初学者的心态就是愿意学习的心态,比较谦虚的心态,认真总结的心态。初学者的心态正如一个新生儿面对这个世界一样,永远充满好奇、热情、求知欲和进取心!

    跌倒七次,就站起来八次!婴儿学走路,都有一个跌倒的过程,你是否依然敢于尝试?你是否能够越戒越勇?不忘初心,方得始终,和大家共勉。
\end{case}

下面进入正文。我们做功夫就是两方面,一、内不随念转,二、外不为境迁。这是一内一外两层功夫,前几季我讲念头实战比较多,这季主要就是讲遭遇外境,我们应该如何应对,分为五个部分详细论述,具体如下。

\subsubsection{对境第一原则:视线不要有所停留}

我们每天都有可能遭遇诱惑的外境,比如上网,看到诱惑文字和图片的可能性非常大,特别是诱惑的图片,真可谓防不胜防,还有就是擦边新闻,夏季走在马路上也容易遭遇诱惑。我们接受外部信息的主要器官就是眼睛,如何管理自己的视线就显得格外重要了。很多戒友可能没想过要管理自己的视线,其实视线控制是极其重要的。佛言:慎勿视女色。子曰:非礼勿视。圣人给后人的告诫就是:不要去看!因为美色乱人心,诱惑会让你的心理产生波动,继而容易产生冲动的行为。古代没有网络,邪淫的资讯没那么发达和泛滥,古代妇女的装束也比较严谨,很讲究礼节,所以视觉上的诱惑比之现代要清淡很多。而现代社会,网络的出现让撸管的资料变得随手可得,手机上网更是提供了这种便捷。

现在的年轻人基本都会上网,上网时很容易看到诱惑的图片,这个时候怎么控制自己的视线呢?我忠告的第一原则就是:不要停留!所谓第二眼着魔,很多人第一眼感觉没看清,然后想仔细看看,这种念头让心魔有机可乘,非常容易破戒。看到诱惑的第一反应:避开!菩萨见欲,如避火坑。避色如避箭,这个戒色意识一定要在脑中不断强化,如果不避开,就会中花箭!菜鸟的典型表现就是粘上去,而高手的反应就是避开!必须避开,就像避开冲向自己的大卡车!这种意识一定要强烈!高手强就强在意识!一个避字诀要好好参透,如果不避开,就会深陷进去。视线一旦停留,破戒的危险就会陡然增加。你停留,就意味着你被诱惑吸引了,所以千万不可停留,避开是最好的选择。没有第二眼,这是我对广大戒友的忠告!有第二眼,就危险了,如果你盯着看,那危险系数就很高了。

\subsubsection{对境第二原则:保持散视的状态}

遭遇诱惑时,我们要学会让自己的视线保持在散视的状态,而不是聚焦的状态。聚焦是注意力集中的状态,散视则是注意力分散的状态,注意力一旦集中到诱惑的图片上,就容易产生邪念,乃至出现破戒。散视的状态很微妙,一眼望过去,注意力就是散的,没有明确的集中点。有点像上课不专心听讲,虽然看着黑板,貌似是在听老师讲课,其实等于“视而不见,听而不闻”,老师讲的什么内容完全不知道,是一种“心不在焉”的状态。虽然这种状态在学习方面不是好的表现,但是放在诱惑的对境方面,可谓功效特异。遭遇诱惑对境,让自己保持散视,让自己保持不集中的状态,那就可以获得对诱惑的免疫力。你去看好了,一般的普通人遭遇诱惑时,很容易把注意力完全集中上去,他会瞬间变得很敏感,两眼盯着看,甚至两眼放绿光;而我们和普通人恰恰相反,让自己保持在“心不在焉”的状态,让自己的视线保持分散,这样诱惑对我们的吸引力就会大大降低,通过这种方法完全可以做到“视而不见”,这个方法我一直在用,也是久经考验的一个方法,因为我做戒色图片需要大量搜图,往往会遇到一些暴露的图片,但我一次都没意淫,就是因为我让自己保持在散视的状态。因为散视,因为“心不在焉”,所以诱惑的图片无法对我起作用,也就是对我无效!

\subsubsection{对境第三原则:不要追求新奇的感觉}

“新奇,新奇,更多的新奇!多巴胺会因新颖的事物而急升。一辆新车、一部新电影、最新的电子产品……都会让我们对多巴胺上瘾。所有新事物的刺激感都会随着多巴胺下降而消失。”最初用大白鼠进行的实验程序如下:把一只公鼠和四五只处在发情期的母鼠一起放到一个封闭的盒子里。公鼠马上会和所有的母鼠开始交配直至最终精疲力竭。虽然母鼠会继续触碰向公鼠求欢,但公鼠不会有响应。然而,如果新的母鼠放入了盒中,公鼠又会变得警醒,再次焕发能力与新的母鼠交配。柯立芝效应是这样运作的:雄鼠的大脑对旧的雌鼠所释放的多巴胺越来越少,但是当新雌鼠出现,它就会急升。

我们沉迷邪淫,其实也是如此,是一个不断寻找、又不断厌倦的过程。获得新的刺激后,多巴胺急升,当逐渐厌倦后,就进入了“不应期”,这时候就需要新的刺激来重新激发。男性的本能就是需要不断新鲜的刺激,特别是看到新奇的诱惑图片,很多戒友就无法自控了,新鲜感具有一种魔力,能够让人深陷其中,其实我们只是被多巴胺愚弄了!有个戒友原来戒得还不错,后来听说黄漫又出新的了,他马上就蠢蠢欲动了,不久就破戒了。新奇是一种很大的吸引,人又往往有好奇心,好奇害死猫啊!我们应该充分认识到:反正都是要厌倦的,何必疲于奔命地追求?你追求好的内容也就罢了,你这样追求邪淫的内容,最后就会把自己害惨!所以,我们不要去追求新奇的感觉,一定要保持高度警惕,警惕自己去点击。

\subsubsection{对境第四原则:不要去分别}

分别易入魔!修行方面就有这句话。很多戒友就是喜欢分别,这个身材好啊,那个皮肤白啊,关于女色的分别念太多太多了,涉及到邪淫方面的则更多,你在这方面分别越多,懂得越多,就越容易走火入魔。其实根本不用去分别,都是臭秽不净的,不要被薄皮所蒙,再好看的女人,揭开那层皮,都惨不忍睹,令人作呕!人就是一具臭皮囊,造粪的机器,其实没什么值得贪恋的,色不迷人人自迷,万丈深渊亦自掘。你迷在里面,就觉得这个好啊,那个好啊,喜欢啊,其实说白了,就是一具骷髅、一堆肉、一坨内脏加一张皮!你喜欢的女色不过如此!一分别,就容易迷在里面,我们对境时应该告诉自己,不要去分别,都一样的,再美再丑都是臭秽不净的。不净观可以很好地对治分别念,我们应该好好修习。

\subsubsection{对境第五原则:牢牢看住自己的念头}

面对诱惑的境界时,最终还是会表现在念头上,因为念头会动!一般有三种情况:

\begin{description}
    \item[见色心动,追逐念头] 第一种情况最差,就是没有修行的凡夫,一碰到诱惑的境界,心马上就粘上去了,几乎没有任何定力可言;
    \item[见色心动,念起即断] 第二种情况还不错,已经学会看住自己的念头,对境时虽然会起念头,但是能够做到念起即断,已经具有相当的定力;
    \item[见色不动,心如止水] 第三种情况最好,就是对境无心,\textit{二六时中,对五欲八风,如盲人视物(《续古》)},视而不见!相当深厚的定力。
\end{description}

\textit{降魔者先降自心,心伏则群魔退听。(《菜根谭》)} 佛教也讲降伏其心,戒色就是修心,说到最最根本处,就是修心,就是修念头!任何一篇戒色文章都无法做到让你彻底不起邪念,因为每个人的八识田里都有淫欲种子,平时就会翻种子,遇缘更会起现行。所以,我们一定要学会看住自己的念头,对念头实行最强监控!邪念就是贼!一定要牢牢看住它。严格做好念起即断的思想工作,控制念头的关键就是:使念不连续!比如对境时,脑海中出现了一个邪念,你要马上发现,不能跟着邪念跑,念起不随就是断!思维对治也是断!总之,就是不能使邪念连续下去。你跟着邪念跑就是你的失职,你跟着邪念跑了一段才意识到,其实已经有点晚了,因为邪念已经起势了,邪念一旦起势,就会控制你的行为,到时候破戒的可能就会很大。到了欲火中烧的阶段,不破也得破,到时就很煎熬了。所以断意淫贵早!千万不能让邪念起势。

蒋介石的戒色日记上写到:“见艳心动,记大过一次!”戒色就是要学会看住自己的念头,你如果放任邪念,就像放任小偷在你家随便偷东西一样,最后把你最宝贵的肾精给偷走了,而你作为主人非但不加以阻止,反而还帮着小偷一起搬东西,真是荒唐至极!戒色一定要学会控制自己的念头!越早介入越好,越晚介入越被动。断意淫口诀每天应该重复几百遍,要达到条件反射的程度,必须熟练,所谓熟极自神!平时就要磨刀,临阵磨刀太晚了,用起来太钝!根本斩不断意淫!平时就要不断重复这个口诀,磨锋利了!到时候就能做到念起即断,手起刀落!大家看看魔术师的手法练习,有时候一个手法要练上几个月乃至几年才敢拿出去表演,要练到无破绽的程度,要达到精纯化境的地步。

\paragraph*{总结}

这季和大家分享了对境时的五大原则,希望大家能够在境界中不断磨练自己,这就是戒色实战!磨练到最后就能做到对境无心,任何诱惑的境界都无法动摇你,都无法对你产生影响。刚开始做功夫,难免会对境动心,但千万不要跟着邪念跑,应该严格做到念起即断,这样持之以恒地做下去,慢慢就能做到对境无心了,到最后就会越来越稳定。

推荐书籍:

\begin{book}[《菜根谭》]
    本书论述了修身、处事、待人、接物等人生处世哲学,揉合了儒家的中庸思想,道家的无为思想和释家的出世思想以及作者的生活体验为一体,是中华民族传统文化的结晶。其中的格言警句文词优美,对仗工整,短小精粹,耐人寻味,蕴含着深刻的哲理。该书问世四百多年来,广泛流传于民间并远播海外。
\end{book}

\begin{book}[《冰鉴》,曾国藩]
    曾国藩不为人所知的另一套学问。有人说,清代中兴名臣曾国藩十三套学问,流传下来的只有一套,那就是《曾国藩家书》。其实传下来的有两套,另一套是曾国藩识人的学问——《冰鉴》!《冰鉴》,取以冰为镜,能察秋毫之义,是曾国藩一生重要的杰作之一,其鉴人的深邃思想和实用方法全存于该书中。虽然该书只有短短的两千多字,却包含了无穷的智慧,融会了中国几千年来察人、鉴人的精华。
\end{book}

\begin{book}[《不离》]
    法王晋美彭措著。法王是世界上最大佛学院——喇荣五明佛学院的创始人,当今众多极具影响力的高僧大德的根本上师。本书是法王毕生的智慧流露,字字句句难得一见,非亲近弟子不能得闻。其亲传弟子索达吉堪布为利益更多人,多方搜集几十年的法王教言,费尽心血翻译、编辑而成。是法王迄今唯一的汉文传世之作。
\end{book}

这季分享三首戒色诗歌:

\begin{poem}[一个戒者的呐喊]
    \begin{multicols}{3}
        \begin{center}~\\
            这是一个肉弹横飞的时代 \\ 无数的青少年和青年 \\ 就这样不断地阵亡 \\ 从纯真的孩子变成撸民 \\ 只有一撸之遥 \\ 一旦打开潘多拉的魔盒 \\ 就会一发不可收拾 \\ 戒色的战场 \\ 每天都很危险 \\ 每天都有人阵亡 \\ 诱惑实在太猛烈 \\ 我们必须小心谨慎 \\ 昨天还信誓旦旦的人 \\ 今天就可能连破三次 \\ 不要相信砖家 \\ 不要相信无害论 \\ 他们只会把我们变成废人 \\ 他们的歪理邪说 \\ 只会把我们送进医院 \\ 我们把自己掏空 \\ 医院把我们的钱掏空 \\ 所谓人财两空 \\ 面对诱惑,活着回来! \\ 我一遍遍地告诫自己 \\ 如果不保持警惕 \\ 下一个阵亡的就是我
        \end{center}
    \end{multicols}
\end{poem}

\begin{poem}[戒撸重生]
    \begin{multicols}{3}
        \begin{center}~\\
            你的纯真消失在撸管的岁月里 \\ 你的猥琐和自卑与日俱增 \\ 你的瘾一天天强化 \\ 而你的身体却一天天垮掉 \\~\\
            当你症状缠身,当你容貌尽毁 \\ 你就会彻底明白 \\ 撸管是要付出惨重代价的 \\ 这是多么痛的领悟 \\~\\
            人之所以痛苦 \\ 在于追求错误的东西 \\ 你追求邪淫的快感 \\ 必然会引爆人生的大苦 \\~\\
            撸到最后,都是无奈的血泪史 \\ 撸管注定丑陋 \\ 这是一种丑陋龌龊的行为 \\ 任何冠冕堂皇的理论 \\ 都无法改变其丑陋的本质 \\~\\
            拿什么拯救你 \\ 我邪淫堕落的灵魂 \\ 唯有坚持戒撸 \\ 才能让我们浴火重生
        \end{center}
    \end{multicols}
\end{poem}

\begin{poem}[从天堂到地狱,你撸过人间]
    \begin{multicols}{2}
        \begin{center}~\\
            还记得第一次撸管在什么时候吗 \\ 你肯定记得 \\ 因为第一次意味着印象深刻 \\ 每个人都有第一次 \\ 第一次撸管就是从天堂滑向地狱的开始 \\ 撸管前的生活是那么纯净美好 \\ 没有邪淫的污染,每天都很单纯 \\ 这种单纯的大快乐 \\ 在失去后才会懂得它是多么可贵 \\ 这份无撸的纯美 \\ 只有在彻底戒撸后才会倍加珍惜 \\ 它,真的来之不易! \\ 请好好珍惜这份纯美的感觉 \\ 它可以带给你无限的愉悦 \\ 远远胜过邪淫短暂的快感 \\ 无撸就是天堂! \\ 撸管迟早会把你打入症状地狱 \\ 充满痛苦与烦恼的症状地狱 \\ 各种症状都会找上门来 \\ 你的生活将会陷入惶恐与不安 \\ 与撸管前的纯美不同 \\ 这种生活简直就是一种折磨 \\ 天堂地狱,只在一念之间 \\ 从天堂到地狱,你撸过人间 \\ 从地狱到天堂,你戒出生天 \\ 让我们向往无撸的自由 \\ 那才是真正的大快乐
        \end{center}
    \end{multicols}
\end{poem}
