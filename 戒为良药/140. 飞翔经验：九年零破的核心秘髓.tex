\subsection{九年零破的核心秘髓}

这季将会详细分享一下我戒色九年多的核心秘髓,我戒到现在一次未破,绝对真实,不含一丝水分。九年多的时间看似很长,但一年一年地过,一晃就九年了,这几年感觉过得特别快,不知大家是否有这样的感觉。曾经我的最高纪录是 28 天,那是强戒的状态,没有任何的学习,就靠一时的决心和毅力,破了就怪自己毅力不行,那个阶段的我是十足的菜鸟,心魔攻破我就跟玩儿似的,一点难度都没有,一个怂恿,一幅图像,我就会失控,开始疯狂找黄,那个猥琐的身影又开始忙碌起来了……后来我是通过坚持学习和练习才蜕变的,不管何种方法都很强调日课的积累,贵在坚持,有了坚持的力量,自己的实力就会逐步提升,到时即可降伏其心,突破怪圈。

今年一月份我分享了《戒者笔记》,这是我戒色九年多整理的笔记,很多戒友都很喜欢,这本笔记是精华之作,分为九大部分,一万五千余条精华笔记,值得反复研读。我戒色后记了很多笔记,觉悟就是在记笔记、复习笔记、总结笔记、归纳笔记中得到了大幅的提升,不能光记笔记,要懂得吸收笔记,对于重点笔记要多复习、多归纳、多提炼、多记忆,不能看过就忘。我现在每天依然会复习笔记,笔记就是精华所在,看一遍笔记就能得到很大的加持,很多认识都会变得深刻而清晰。经常复习笔记对于保持良好的状态也是非常重要的,有时复习笔记,顿悟就突然来了,一下就明白了,那种兴奋真的比中了大奖还高兴。

无知的人会觉得戒色偏激,症状迟早会告诉他们真相!真正有智慧的人才懂得戒色的意义与价值。戒色是一种修为,古圣先贤都提倡戒色,戒色就是戒邪淫,很多撸者脑子里装的都是无害论,对邪淫一点概念都没有,他们真的很无知,也很可怜。\textit{淫欲为害,伤身丧志,虽属夫妻,亦当节制。若是邪淫,更非所宜,古今志士,无一犯之。(《德育启蒙》)} 邪淫伤身败德,天道祸淫最速,实乃真理也!\textit{多有少年情欲念起,遂致手淫,此事伤身极大,切不可犯。犯则戕贼自身,污浊自心。将有用之身体,作少亡,或孱弱无所树立之废人。又要日日省察身心过愆,庶不至自害自戕。否则父母不说,师长不说,燕朋相诲以成其恶,其危也,甚于临深履薄。(印光大师)} 色是少年第一关,此关打不过,任他高才绝学,都无受用。

一位戒友的反馈:“中医院的中医这样对我说:‘如果不戒手淫,频遗、前列腺炎、腰酸腹痛、四肢无力等等症状,就不会好。’这是我第三次就诊,赶上人少,中医整整和我谈了半个小时,他说问题就在于手淫,手淫不戒,吃多少药也没用,戒了手淫,甚至不吃药都行,都能好。”这位中医的医德很好,经验也很深厚,手淫不戒,那些伤精症状真的很难好,如果能戒掉,很多症状都是会缓解的。在这个色情泛滥的时代,很多人之所以症状缠身,就是因为看黄手淫或者其他形式的纵欲,把身子骨掏空了,症状就开始爆发了,到时就要受苦了,惶恐地奔波在去医院的路上,为自己的放纵买单!

九年零破,我是怎么做到的?下面我将为大家详细讲述。

\subsubsection{要有大愿}

人生的生命价值是什么?不只是为自己活着,也要为别人奉献,行善积德,把光明与美好带给这个世界的所有人。戒色很重要的一点就是动机,刚开始为了恢复身体健康而戒,为了恢复容貌气质而戒,为了恢复脑力而戒,这是比较常见的动机,刚开始这样的动机无可厚非,但随着戒色觉悟的提升,就要建立更高的动机了,要正己化人,要有度众生的大愿,动机要崇高,不是为了自己,而是为了众生。我能坚持到现在和发愿是有很大关系的,当然也和广大戒友的支持是分不开的,无比感恩大家的支持!当初学佛时,请了楞严咒的法本,是宣化上人的版本,上面有宣化上人的十八大愿,是非常崇高而伟大的大愿,我也效仿发了一个度众生的大愿,现在每天还在发。佛法的两句话:“\textit{不为自己求安乐,但愿众生得离苦。}”这种崇高的发愿很感动我。戒色是人生的整体改造,涵盖着生活的方方面面,是让你由内向外成为一个光明、善良、自律又优秀的人,成为一个无私利他、为世界带来正能量的人。戒色不仅仅是戒色,我们生而为人,应该为天地之间增加正能量,帮助更多的人!

一开始的动机是为了自己,随着觉悟的提高,就要懂得无私利他了,这点很重要,要有正己化人、行善积德的长远发心。为了自己的话,很难长久。比如身体恢复了,生活状况也改善了,这时候就可能丧失动力,因为之前的目标基本实现了,这时候就可能会松懈下来。所以,戒色动机一定要升华,刚开始不可能那么崇高,因为觉悟不够,通过学习圣贤教育,觉悟提升后,就要建立崇高的动机了,要发愿帮助更多的人。如果你有崇高的动机,你自然就有了责任感、使命感和担当。

一位戒友说:“感谢飞翔哥的坚持,看到您的坚持我就有了一股力量。”另外一位戒友说:“六年前我还是一个高中生的时候,飞翔哥在,如今您还在,还在坚持发帖子教导吧友,时隔一年多再回来看到您的帖子,刹那间甚是惭愧,又是敬佩,祝您身体健康,新年快乐!”坚持是不容易的,为了帮助大家,我个人牺牲了很多,也遭受了很多诋毁和诽谤,不过我觉得这一切都是值得的,衷心希望每一位戒友都能恢复健康、回归正轨、生活幸福,也衷心希望每一位戒友都能超过我,甚至那些诋毁我的人也能超过我,我垫底也没事。诋毁我的人是我的逆增上缘,还得感谢他们。不忘初心,牢记使命,正己化人,无私奉献!希望大家一起传递正能量,一起帮助更多的人。坚持不易,且行且珍惜!

\subsubsection{紧抓两个实战}

我写了很多戒色文章,后期关于实战的文章和案例越来越多,因为我深刻认识到实战强,才能立于败之地!脱离实战,肯定一败涂地!两个实战就是断念实战和对境实战,戒色九年多一次未破,和我注重实战是密不可分的,这九年我经历了太多的实战,断念和对境,不知经历了多少次。如果我实战弱,肯定会破戒,戒色一定要紧抓实战,以实战为核心。实战不行,肯定失败,念头导致行为,断念不行,一切白搭。对境是很大的考验,如果没有足够的实战意识,很可能会陷进去,这个时代诱惑太猛烈,一定要强调对境。手机上网,就是对境;上电脑也是对境;生活中看见诱惑也是对境。对境的考验层出不穷,也可以说防不胜防,但只要有足够强的实战意识,还是可以顺利过关的。

\begin{case}
    是念头触发了手淫的行为,所以很多戒色方法为什么治标不治本,因为脱离了实战,所以最终导向的是失败,所以戒色最最最最最重要的课,就是学会控制念头。

    \textbf{分析} 这位戒友用了五个“最”来强调这个核心,要学会控制念头,要学会观心断念,外在的修身治标不治本,本是心,一定要抓住根本,修身方面可以说得高大上,但如果脱离修心,那就是空中楼阁。自古以来,大德都是把修心放在核心的,我们一定要学会修心,\textit{欲戒淫行,必自戒淫念始,淫念起,则淫行随之矣。(《寿康宝鉴》)} 一念稍疏,陷溺难返。

    \begin{quote}\it
        淫念一生,诸念皆起。邪缘未凑,生幻妄心。勾引无计,生机械心。少有阻碍,生瞋恨心。欲情颠倒,生贪著心。羡人之有,生妒毒心。夺人之爱,生杀害心。种种恶业,从此而起,故曰:“万恶淫为首。”今欲断除此病,当自起念始,截断病根。(印光大师)
    \end{quote}

    印光大师直指核心——当自起念始,截断病根。就是断除念头,断念是实战的根本。我的核心思想和大德是一脉相承的,完全符合大德的开示。也正是因为我抓住了这个核心,才能戒到现在九年多,一次未破。
\end{case}

\subsubsection{观照是正行}

\textit{观照是正行(元音老人)},也就是看住自己念头,返观内照,返观自心。观心永远是强调的重点,邪念肯定会上头,必须要时刻警惕,但不要过于紧张,自己调节好。因为保持在观心的状态,所以就能在邪念上头时,及时做出反应!不观心,肯定是不行的,观心就像内在开了监控一样,你就像保安一样看着这些念头,对于邪念要及时断除。如果保安睡着了,那坏人就可以大摇大摆进去了,同理,如果你失去观照,肯定会陷进念里,跟着念头跑,所以,观照是正行!

\subsubsection{断念一把刀}

戒色的路是用断念的刀砍出来的!这是我戒色九年多的深刻体会,刚开始几年没那么深的实战体会,经过这九年,经历了那么多次的实战,我才得出了这么一句话。一位戒友说:“光看戒色理论,不练习断念,就会成为戒油子。”的确是这样的,断念这把刀平时就要磨锋利了,到时实战时才能快如闪电,削念如泥!瞬间解决战斗!

\begin{case}
    独处时心魔一出来,我立马缴械投降。

    \textbf{分析} 这个案例让我想起了自己的曾经,也和他一样,独处时心魔一来,马上垮掉,那时我是菜鸟,现在我强大了,不会再让心魔得逞!
\end{case}

\begin{case}
    飞翔哥,自从我学会觉察断念,我已经轻松戒了两百多天了。

    \textbf{分析} 觉察断念并不难,关键是认真研读文章,正确理解,坚持练习,真正学会观心断念,就算真正入门了,戒色天数就会突飞猛进。
\end{case}

\begin{case}
    毫无疑问,戒色只有一个核心——观心断念!断念强大了,戒色就能够简单无数倍!大家可以回想一下,是不是每次破戒都是本来好好的,结果一幅图像或者微妙的感觉来袭,之后就是一连串的邪念攻击,图像、怂恿、邪淫回忆通通冒上来了,虽然极力抵抗,但还是有一种强大的破戒的冲动让你失去了理智,至少我在每天坚持背诵五百遍断念口诀之前是一直在重复这样的流程的。但当我认识到“觉悟完善、断念强大就不会破戒”这个道理后,我开始坚持每天学习和练习断念口诀,其他时候也一直保持观心,这才成功突破怪圈,身体再次充满了力量。因为背诵断念口诀,锻炼出了强大的觉察力,我能够在第一时间发现悄然出现的邪念,在发现它之后,它便被激光炮般的觉察力轰得粉碎,完全不会有破戒的冲动了。观心断念才是核心,外在的修身也是为了更好地观心断念。大家一定要背诵断念口诀,请一定要相信我,拿出半个多小时背上五百遍,当你再次面对心魔时,你就会发现会断念的戒色简单得多了,做到才能得到,当你把前辈的教诲彻底落实了,戒色不可能不成功。兄弟们加油!

    \textbf{分析} 这位戒友真正去练了,他的这段反馈很好,会观心断念,戒色真的会变得简单很多,想想每次破戒的心理过程,就是因为断念没做好,导致被攻破,沦为心魔的提线木偶。因为那时不会断念,虽然极力抵抗,也无济于事,往往稍微一抵抗就不行了,稍微一抵抗就被攻破了,根本不是心魔的对手。通过学习和练习断念口诀,这位戒友强大起来了,这才突破怪圈。觉察力强大了,心魔就不是你的对手了,它无法奴役你了。当你真正去练,真正强大起来了,到时戒色必然成功。
\end{case}

\begin{case}
    断念口诀实在过于强大,在过去的三年里,我失败了无数次,无数次立下雄心,无数次自己纵容自己,身体状态一直处于低谷。直到今天,我才忽然醒悟,体会到断念口诀的强大威力。在之前的三年里,我虽不是在强戒,但是却实在算不上认真。回首无数个日夜,我竟然找不出一天我的断念口诀念过五百遍以上。我一直很消沉,认为断念口诀也就那样,心魔来了一点用都没有。但是我错了,昨天,我下定决心一定要念够五百遍才睡觉,并且拿着计数器计算着。而且躺下又念了几十遍,那是我戒色以来练习断念最多的一天。结果,今天早上,奇迹发生了。欲望与以往一样悄无声息地袭来,然后是心魔的怂恿。这个时候心魔是要怂恿我去看黄的,以往的我都要挣扎一下才能离开床。如果在破戒高峰期的话,直接就失败了。但是今天,奇迹发生了。当心魔对我说到“要不然看看吧”,它刚说到第二个看字,瞬间断念口诀就自己涌了上来,瞬杀心魔。根本不需要挣扎,也不难受。这真的是太神奇了,仅仅是第一次念诵达到五百次左右,口诀的力量已经初见威力。可以想象,如果我坚持每天都五百遍,长此以往,那心魔可能根本就不是我的对手了。一直困扰我的,戒色的核心难题,意淫和被怂恿,竟然就这样被我发现了破解的法门。原来方法一直都在,早在《戒为良药》才有几十季时,它就已经在了,而我熟视无睹,从来没有“发现”它。戒友们一定要注意了,现在我再次向大家推荐和强调,每天一定要诵念五百遍左右的戒色口诀。念起即断,念起不随,念起即觉,觉之即无。猛练它,猛练基本功,就是各位戒友突破怪圈的秘诀!三年来,我终于第一次体会到瞬杀怂恿的感觉,感觉自己终于可以再向前迈一步了。

    \textbf{分析} 这个反馈也非常好,戒色三年一直没成功,为什么?没有执行和落实,一切都等于零。相对于念佛持咒动辄几千上万的数量,断念口诀五百遍已经算少的了,但也是要坚持去落实,保证每天的日课,注重练习的积累,这样才能出断力。这位戒友说自己三年竟然找不出一天练习口诀五百遍以上的,这样肯定是屡戒屡败。大家应该都知道健身,只有坚持练习,肌肉才能长大,力量才能提升,不练,肯定不行,关键是专业系统地练习。关于断念口诀,我很早就告诉大家了,这个价值无量的修心诀,真正去练了,你就能知道它的威力了。这位戒友认真练习一天,第二天早上效果就出来了,心魔怂恿他看黄,断念口诀立刻就上来了,就像条件反射一样,瞬杀心魔!昨天他练习的成果一下就体现出来了,他这才明白这个口诀的威力,仅仅认真练习一天,就初见威力了,如果认真练习几个月,那会强到何种程度!练习使人强大,不练就强不起来,一直处于弱小挨打的状态。很多戒友之所以一直失败,就是因为没有认真练习,如果认真练习了,绝对是另外一种状态。谁练谁得,不练不得!不管是断念口诀还是念佛持咒,都需要练习的,都要坚持日课,都要念熟。有的戒友问:口诀和佛号可以一起练吗?可以专攻口诀,也可以专门念佛,也可以分清主次,比如念佛为主,口诀为辅,或者口诀为主,念佛为辅。自己灵活安排。这个修心诀的练习价值极高,念佛持咒也很殊胜,两者都很好。
\end{case}

\begin{case}
    飞翔哥,看见你的文章突然顿悟了!这种顿悟是之前从来没有过的,我觉得这次顿悟是之前不断学习积累的,我终于学会了该如何断念,之前看飞翔哥的文章说背断意淫口诀,于是我就机械性地背,可是断念还是很费力,后来慢慢学习戒色文章,终于明白了其中的含义,断念确实是在不断地深入理解,加上大量的理论和练习而逐渐强大的,我也突然明白了那十六字的真正含义,现在我可以第一时间觉察到念头已经上来了,甚至可以预判到念头快要上来了,比如在某种场合,例如无聊情绪上来的时候,我就知道邪念快要上来了,于是我特别警惕,念头上来的一刹那,我可以毫不费力地断掉邪念,就像头脑里什么也没发生过一样!

    \textbf{分析} 正确理解 + 坚持练习。这两点做到了,断念水平就会突飞猛进!不断研读断念的文章,对断念的原理深入理解,在练习和实战中深入体会,这样就能逐步强大起来。觉察力强了,局面就一下转变了,可以做到降伏其心了。这位戒友刚开始断念很费力,后来强大起来后,毫不费力!他做到了!就是通过学习戒色文章和练习断念口诀达到的成果。
\end{case}

\begin{case}
    我给大家说一下断念实战吧,断念口诀我相信大家都应该很熟悉了,但是我们要会用,这句口诀真的是有无穷无尽的力量,价值连城呐,我平时有空就会默念或者读出来,只要有空余的时候,我就会苦练断念,我现在能做到一觉即胜,就是当我一记觉察看到念头,念头就已经消失了,这就是实战,只有苦练,还有明白这句口诀的含义。我每天苦练断念,每天学习戒色文章,到现在我没断过一天,只有时刻保持警惕,你才能不败,我短短的 46 天有如此的领悟与成绩,离不开每天的苦练和学习。

    \textbf{分析} \textit{虽有高授,然非自己功夫久者,无能贯通焉!(《太极拳谱》)} 短短的 46 天,这位戒友已经做到了一觉即胜!看看他对断念口诀的重视,他下的功夫!其实这个口诀并不难,真正掌握的戒友肯定知道的,就是为了提升觉察力。念到条件反射的程度,邪念一上头,马上口诀就出来了。继续练下去,当发现邪念上头时,邪念就消失了,这就是觉之即无。觉察力强了,自然就做到了,很轻松的一觉,实战就结束了。如果不去练,那谁也没办法,即使念佛持咒也需要每天保证几千乃至上万的日课。要真正去下功夫,平时要研读断念的文章,真正明白断念的原理,加上勤加练习,一定会实现突破的。不去练,一切等于零,戒十年也不行,真正去练,也许几个月就突飞猛进、战力爆棚了。
\end{case}

\begin{case}
    今天第 167 天,不论戒多久,你离破戒永远只在一念之间!一朝学会观心断念,才彻底重视断念口诀的练习,学会观心断念,你就知道这才是戒色真正入门的标准。记住,戒色最重要的就是学会观心断念,不会观心断念,不会从根本上着手,其他方法都只是锦上添花。

    \textbf{分析} 这位戒友讲得很好,其实很多大德都在讲观心法门,它是一个总的法门,大总持,“观心”二字,可以概括一切修行法门,是一个根本的法门。大家看过《当下的力量》,应该知道托利也是在讲观察者,必须学会观心,也就是观察自己的思维,思维就是念头。真正学会观心断念了,也就真正入门了,如果对观心断念没有很深刻的理解,没有熟练掌握,那就很难戒除。真正掌握了观心断念,也就真正抓住了根本,从根本上下手,事半功倍,很快就能突破怪圈。其他方法也许可以锦上添花,但没锦哪来的花,观心断念才是真正的核心。
\end{case}

\begin{case}
    我每天看一个多小时的《戒为良药》,现在戒了一百五十多天,飞翔哥的《戒为良药》讲了很多戒色知识,而且戒色的核心就在《戒为良药》当中,就是断念。当然断念也是需要一定觉悟的,不然的话,脑海中没有很深刻的认识,就会当断不断,觉悟是一定要每天反复学习才能稳固,再配合断念口诀,戒色的方法其实就是这么简单,能做到就可以戒除。

    \textbf{分析} 综合觉悟是很重要的,断念则是实战的那一下!这位戒友这段总结很好,戒色方法其实很简单,提高综合觉悟,加上练习断念口诀,就能决胜实战!关键还是严格落实,保证日课,高手都注重积累,菜鸟则一曝十寒,没有坚持积累,没有坚持学习和练习,这样实战时就会一塌糊涂、一败涂地、一触即溃!
\end{case}

\begin{case}
    飞翔哥,我今天早上躺床上练习断念口诀,好像学会觉察消灭念头了!我是念头起来了,从念头中拉出来,一知一觉念头就消失了!请问这是觉察消灭念头吗?

    \textbf{分析} 这位戒友也学会了,坚持练习,迟早会学会的,不要认同念头,不要跟着念头跑,作为观察者去观察念头,一知一觉念头就消失了!这就是觉察消灭!这个操作我已经在戒色文章里讲了很多次了,只有真正去练习,才有望掌握。关于断念实战的操作,第一种,发现念头上来了,马上念口诀或者念佛,速度要快,平时练到条件反射的程度。第二种,当觉察力很强了,直接就消灭了。看见念头,念头就消失了,很多大德都在讲这个操作,我现在用得最多的就是这个操作。
\end{case}

\subsubsection{对境一张皮}

断念一把刀,对境一张皮!不要被那张画皮迷惑,其实就是那层皮在迷惑我们,皮下面都是不干净的,那层皮的毛孔里也是不干净的,不值得贪恋。\textit{不净者,美貌动人只外面一层薄皮耳,若揭去此皮则不忍见矣。骨肉脓血,屎尿毛发,淋漓狼藉,了无一物可令人爱。但以薄皮所蒙,则妄生爱恋。(印光大师)} 我对境这么多次,也经受住了很多的考验,我的实战总结就是对境一张皮,身材千千万,其实就是一张皮,没有那层皮,没有一个人会去贪恋。所以,要学会看破那张皮,不要贪恋那张皮。多思维不净观,可以有效对治贪恋和邪念。

\begin{case}
    飞翔老师,你说的王炸组合太给力了!对境与断念越来越轻松坦然,以前断得很慢很犹豫很吃力,现在轻松秒杀,它也再也不敢回击。以前没有意识到两者同时使用的重要性,现在平时也都是思维对治 + 飞速断念,平时练得多了,对境好不轻松。而且我感觉到思维对治可能是一个大前提,否则很多人即使知道断念,但就是舍不得,真正看透看破,就不会执着,这对断力的提升有很大帮助!

    \textbf{分析} 思维对治非常重要,上季讲了转贪六思维,就是要把贪恋的心理转变过来,这样断念时才会果断坚决,一点都不贪恋。思维对治不仅可以对治贪恋,本身也有断念的效果,加上断念口诀,那真的是王炸组合。对于恋癖倾向,可以这样思维,那些物品很普通,一点不性感,要觉得它们脏乱差臭,不要觉得它们好。这样来思维对治,转变观念,坚持一段时间,那种恋癖的倾向就会淡化减弱,直至恢复正常。
\end{case}

\subsubsection{关于念佛持咒}

有些戒色方法是走念佛持咒路线的,念佛持咒很殊胜,我念过楞严咒、大悲咒、往生咒等,现在日课有念佛和念往生咒,如果有佛缘或者一直屡戒屡败,也可以试试念佛持咒,这的确是一个很好的方法,关键还是要有信心,要坚持日课,通过念佛持咒,邪念是会变少的。我记得刚开始念佛的那半年,妄念很多,就像一束光照进屋里,会看见很多灰尘,刚开始念佛时会有这个现象,坚持念下去,心里会变得清净。不过这种清净是暂时的,继续戒下去,又会遭遇很多次的翻种子,到时很考验断念能力。念断念口诀也是如此,刚开始会发现念头变多了,其实不是念头变多了,而是开启了觉察力,一束光照进心里,发现原来有那么多的念头。坚持练习,坚持对治,慢慢念头就会变少的。念佛持咒有佛菩萨加持,但也要自己有信心和恒心,念佛持咒也要用于断念,这点很关键。

\begin{case}
    以前一直靠念佛号戒色,因不注重实战,念佛号四百二十万声,但照样破戒、看黄,导致我对念佛产生怀疑,对佛法产生怀疑,不想念佛,懈怠念佛,没有信心念佛。看了这一季才明白,我只是把念佛当成日课,而没有拿佛号来实战断念,导致屡戒屡败。断念要提升觉察力,如何提升呢?在念诵口诀或佛号的过程中,发现念头,觉察念头,发现走神,觉察走神,立马拉回来,继续自听自己念口诀念佛号的声音。不断练习这个觉察的过程,觉察力提升了,断念就厉害了。

    \textbf{分析} 这个戒友的反馈很典型,为什么有不少人也念佛,但照样破戒,就是日课脱离实战,虽然念佛了,但是邪念上头时,还是跟随邪念,看见诱惑,还是盯着看,这样肯定会破戒的。念佛后邪念还是会上头的,这点要切记,不是念佛后,邪念就再也没有了,不是的。念佛持咒是要用于断念的,大德也多次提到过,邪念一起,马上转成佛号,这样就是断念。平时念佛持咒就是在训练觉察力,觉察力是提升的核心,觉察力强了,断念就厉害了,就能真正主宰内心了。
\end{case}

\subsubsection{关于心魔这个概念}

\begin{quote}\it
    最厉害的魔,就是心魔。他能令你不修行。你想修行,他就来捣乱,令你生妄想。他会这样对你说:“不要修行啦!修行是一种很苦的事。你要吃多一点,要穿多一点,要睡多一点,把身体保养好一点。那才是对的,不要太愚痴!”这个自心魔常来破坏自己的修道心。(宣化上人)
\end{quote}

心魔这个概念不是我的首创,是修行方面的概念,很多大德都提到过心魔,相信大家看大德开示,会经常看见这个词。什么是心魔?其实就是负面的念头、邪念、不良的怂恿念。心魔这个概念并不玄乎,其实就是负面的念头而已。它是出现在脑海中的一种客观的可以亲身感受到的现象,的确会怂恿你,还会出现意淫、诱惑的图像、微妙的感觉等。大家过去肯定破戒过很多次,对于心魔应该有比较切身的体会,的确是有这种现象。对于心魔不要害怕,因为它只是一种念头而已,我们应该学会识破它、降伏它。

\begin{quote}\it
    若人善巧解战斗,独自伏得百万人,今若能伏自己心,是名世间真斗士。(《佛本行集经》)
\end{quote}

\begin{case}
    戒色到 195 天,刚才看新闻时心魔突然冒出来了,心魔怂恿我去看黄,心魔怂恿我:“光是看看,不撸不意淫,没事的。”就这样我被心魔所蛊惑去看以前看过的黄网,看了有一分钟后我才想起了断念口诀,背诵了几十次后才打败了心魔,退出了黄网!赶紧出去外面散心了,不敢再一个人呆在房子了!

    \textbf{分析} 心魔最常的怂恿就是怂恿看黄,它这个套路曾经攻破我很多次,因为那时我不知道这是心魔的怂恿,还以为是自己的想法。它会说:“就看看,不撸。……撸一次,没事的。……最后一次!”它的怂恿很有针对性,后来我研究和熟知了心魔的套路,就再也没上过当。
\end{case}

\subsubsection{落实戒色十规}

戒色十规是最实战的专业戒色体系,很多戒友都在反复研读 \ref{126},出到 \ref{126} 时,我在戒色方面的研究、阅历和答疑量已经积累到相当深厚的程度,所以才有了 \ref{126}。戒色十规是一个集大成的成熟戒色体系,是一个高度专业化的戒色体系,专业戒色是最普适的,不需要信佛也可以戒,覆盖面最广。我能戒到现在就是因为在落实戒色十规,远离黄源是第一规,这条非常重要,因为对境的机会太多,所以必须时刻提醒自己远离黄源,避开诱惑!学习戒色文章是第二规,不学习的强戒肯定不行,所以必须坚持学习提高觉悟,真正领悟戒色的原理和规律,认识危害,掌握戒色的方法。练习断念是第三规,观心断念是实战核心,断念不行,必破无疑!做好慎独是第四规,破戒基本都是发生在独处时,独处时心魔易进攻,一定要做好慎独,严防心魔进攻。视线管理是第五规,这条是讲对境实战的,看见诱惑第一反应要避开!第一反应必须要正确!看第二眼、盯着看、回看,这些都是错误的操作。第六规是养生恢复,戒色是系统工程,养生也是系统工程,养生方面也极为重要。第七规是情绪管理,一个稳定乐观、积极向上的情绪是非常重要的,出现生气、颓废消极等不良情绪时,一定要及时调整。第八规是改过迁善,戒色一定要改过,改正不良习气,整体改造,断除一切负面心态,多做善事。第九规是培养德行,德行是更高的修炼,断念是技术层面的,德行则是地基,地基不稳,戒色大厦必然倒塌。第十规是学习圣贤教育,可以把人生的境界提升到更高的层次,真正学到圣贤的智慧。

这九年多我就是按照戒色十规来做的,慢慢完善和提升自己。戒色十规的每一规都说到了刀刃上,资深戒友一看就知道这十规意味着什么,戒色成功之道就在戒色十规里面。知道戒色十规,就要严格落实,真正去做,再好的方法不去落实,也帮不上你。只有真正去落实了,才能得到益处,才能渐入佳境。抓落实,抓执行,抓日课,坚持下去,恒久力行,必然能突破怪圈,彻底逆袭。

\subsubsection{必须夯实德行的地基}

德之不修,戒之不稳!德有多高,戒有多稳!最后肯定要强调德行,德行有亏就会出现这样一种现象:虽然戒色了,但负能量还是很重,还是自私自利,嗔恨心和傲慢心还是很强。戒色后应该培养自己的德行,对治所有的负面念头,建立自身的高频能量场,戒色是一种高度自律、自控、纯净、纯粹、高尚、有品质、美好的、正能量的生活方式。我们必须夯实德行的地基,这点会随着戒色时间的延长而变得格外重要,就像刚开始造楼房,地基的影响不是很大,越造越高后,就对地基的要求非常高,地基不行,就会垮塌下来。希望大家好好完善和培养自己的德行,改过迁善,不断对治负面的习气,克服负面的念头。

\begin{case}
    飞翔老师,看了你写的“俞净意公给我们的启示”那一篇文章,我才知道我为什么戒了两年半破戒,后又戒了八九个月又破戒的原因了,就是因为意恶太重!反思了一下,在破戒之前,都是满脑子的恶念,各种嗔恨、嫉妒、抱怨、消极、骄傲、自大、心胸狭隘同时又悲观等念头总是充斥着整个脑海,因为那时候还意识不到这类念头会导致戒色防线的失守,只注重断意淫和心魔怂恿还有远离诱惑方面的觉悟,却忽略了恶念带来的后果。

    \textbf{分析} 恶念就是负能量,负能量重的人很容易破戒,因为心里容易失衡,很容易出现情绪破戒。俞净意公给我们的启示就是意恶的问题,虽然俞公也行善,但是心里的意恶太重了,意恶重,负能量就重,负能量会感召负面的人事物,命运怎么可能好?我们不仅要断意淫,也要学会断除其他各种负面的念头,断念的原理都是一样的,思维对治 + 断念口诀,也可以念佛持咒来转。有一个词叫“佛口蛇心”,虽然念佛,但是心里却有很多恶念,就是不知道用念佛来断念,只知道完成日课。一个人负面念头多,他的德行肯定很差,戾气太重,浮躁易怒,这种心理状态很容易导致破戒。戒到最后拼的就是德!德行有亏,就很危险了;德行深厚,就会戒得很稳定。我的戒色体系很强调实战,也很强调德行,这两个是最关键的,也是最重要的。德是地基,地基是根本;断念是实战根本。抓住这两个根本,就能越戒越好,越戒越强!
\end{case}

\subsubsection{切忌消极}

关于消极的问题也要引起重视,有的戒友看到某些开示,就容易产生误解,变得消极,这是陷入思想误区了。关于无常的开示是为了激励修行,并不是为了让人消极。因为无常,所以要更勇猛精进地修行,更好地珍惜和把握现在所拥有的机会。梦里明明有六趣,觉后空空无大千。大德是说过这个世界是一个梦,但在梦醒之前,痛苦还是那么真切,所以还是要努力行善积德,断除恶行。我没见过一个大德是消极的,都在努力地度众生,都是积极向上、充满正能量的。在生活中,我们也会遭遇很多挫折,这时也不能消极,消极是低频的状态,消极的人看上去那么无力和颓废。我们一定不能消极思考,消极思考会拉低振动频率,让自己感觉无力,我们要积极思考,让自己的人生变得积极向上,努力去奋斗自己的人生,即使遭遇挫折,也要积极思考,也要多发善念,感恩挫折的教育。

\textit{真正学佛,只要见地和方法正确,益处绝对多多,若是学佛反招不顺,除了因果报应或魔考的理由外,也要检查自己的观念行为是否偏差。(南怀瑾)} 很多戒友都在学佛,佛法是很好的,但一定要有正见!没有正见,很容易走入误区,不是变得消极,就是变得神经兮兮,甚至傲慢自大,这都是走偏了,对佛法产生了误解。学佛也不能追求神通,关键还是修心,净化自己的内心。善根差的人甚至会诽谤佛法,这类人很可怜,对于佛法的一些道理,即使暂时不能接受,也要秉持求同存异的原则,充分尊重。千万不要去看诽谤的文章,以免自己被误导,一同陷入诋毁诽谤之深坑,搞得自己充满负能量。

\subsubsection{培养清净观和信心}

所谓清净观,也就是看待一切事物要从正面出发,不要去观察和思考它的负面。我们尽量不要去观察善知识和大德的缺点,这点很重要,在我自己的修行历程中,有时会遭遇这类念头,就是对别人负面评判,相信大家都应该经历过。对善知识和大德进行负面评判,一方面会导致自己失去信心,另外就会导致自己积累负能量,导致不好的果报。所以培养清净观和信心是十分重要的,对善知识和大德要有十足的信心、坚定的立场,绝不动摇。很多时候都是自己的负面投射,其实善知识并没有那个问题,是自己的负面妄想,即使善知识真有那个问题,也要注意修清净观,人无完人,要懂得体谅和包容善知识,要看整体,顾全大局。这点很重要,这其实和做人是完全相通的,做人不行,修行也会充满障碍。

\subsubsection{高级的纯净感}

孩童般的纯净感是那么美好,我们曾经活在那个世界里,后来失落在黄片里,黄片就是毒品,后来才知道纯净的大快乐是多么宝贵,当初是拿钻石换糖吃。不仅失去了纯净的大快乐,也失去了宝贵的肾精,搞得症状缠身,负能量爆棚,活得异常惶恐。多少孩子明亮的眼睛在看黄手淫后暗淡下来,纯真没了,猥琐和憔悴浮现在脸上,一脸的死气沉沉,宛如行尸走肉。看多少黄片,撸多少次,都无法真正快乐起来,只会变得越来越空虚和痛苦,在色情的沼泽中越陷越深。

记得之前看过一个戒友的发帖,他说自己“表面老实,背后邪淫”,他说的这八个字引起了我的共鸣,我曾经也是这样,表面看着还算老实,其实内心很龌龊,有很多邪淫的念头,我无法做光明磊落的自己,无法做到表里如一。后来我戒掉了,我可以做光明磊落的自己了,我学会了修心,学会了主宰自己的内心,过去一个邪念就会让我陷入疯狂,现在我可以断除它,从而真正主宰自己,戒色九年我蜕变了、升华了,摆脱了邪淫的生活,做回了纯净的自己。没有什么可以阻挡,对纯净的向往!不能做纯净的自己,其实就是一种莫大的痛苦。

一位戒友发帖说:“不知不觉戒色两千六百天了,一个人走过了春夏秋冬,走过了严寒酷暑,两千六百天,一个曾经想都不敢想的数字,我做到了,我真的做到了。感谢飞翔老师,感谢戒色吧相互鼓励的兄弟们,没有你们的支持和力量,我不会有自己的今天。昨天面试完,在回来的路上我坐到公园里,看着明媚的阳光,树梢上闪闪发光的影子,湛蓝的天空,不时还有鸟儿飞过,泪水又打湿了我的眼眶。感恩。”这位戒友戒得很棒,纯净的感觉是如此宝贵,当心灵恢复纯净,就能再次感受到童年时的蓝天白云,那种神奇美妙的感觉,内心很单纯,很美好,看着大自然,看着鸟儿飞过,都会开心喜悦地笑起来,就像纯真的孩子一样。

我摆脱了色情,恢复正常了,我不用再像疯子一样找黄了,也不用像马拉松一样看黄了,也不想再丧心病狂地疯撸了!仅仅是享受内心的纯净感,都让人感觉那么幸福和愉悦。高级的纯净感是这个世界上最奢侈的东西,因为在你第一次看黄、第一次磨床,或者第一次手淫后,它就消失了,感觉整个世界都灰暗下来了,满脑子的意淫,内心很难真正开心快乐起来。彻底戒掉后,才能感受到那种久违的纯净感,那么单纯,那么美好。清澈纯净的感觉,真的很高级,这是真正值得拥有的感觉,让人感动流泪,来到这个世界上,那个本初的自己,就是如此纯净,如此美好。

\subsubsection{敬,德之聚也}

印光大师强调的就是诚与恭敬,唯恭敬至诚者,能得其全。有一分恭敬,则消一分罪业,增一分福慧;有十分恭敬,则消十分罪业,增十分福慧。学习善知识的开示,一定要恭敬,曾国藩也强调敬,一个敬字,非常关键,曾国藩说:“\textit{主敬则身强。}”要有敬之气象。孔子:“\textit{修己以敬。}”恭敬善知识,就能真正学到东西,如果自己不敬,肯定浮躁,学不进去。敬,德之聚也,能敬必有德。能敬,德业自然增厚,能谦敬的人一定具有德行。敬人者,人恒敬之,这也是为人处世方面的道理,君子敬而无失,与人恭而有礼。恭敬太重要了,不仅是一个人德行的体现,也直接影响一个人最终的造诣。一个高段位的戒者,必须注重培养自己的恭敬心。

\subsubsection{让内在的神性开花}

每个人的内在都有神性,也有兽性,神性是纯粹的觉知,是真我,是无私的爱,包含一切正面的品质;兽性是邪恶,是自私,是贪婪,是一切负面的品质。我们要学会开发神性,而不要去开发兽性,开发神性是圣贤,开发兽性是禽兽,不为圣贤,便为禽兽。过去邪淫十几年,我一直在开发兽性,最后心理已经有变态倾向了,那十几年我过得浑浑噩噩,几乎没有做什么善事,内心充满了负能量,那不是我真正想要的状态,那种状态让我痛苦和惶恐,也让我绝望,特别是神经症爆发后,我就像惊弓之鸟一样,惶惶不可终日,那种紧张、疑病的状态我到现在还记忆犹新。我们应该认识真我,安住纯粹的觉知,让内在的神性开花!感受临在之美,活出神性的自己。

\subsubsection{以净土为归}

这部分内容适合有佛缘的戒友,即使不信佛也是可以看看的。吧里有不少十几岁的孩子,还是学生党,可能他们修持的机缘还没成熟,不过可以先做一下了解,将来如果机缘成熟了,就可以修持了,我只是做一下推荐和介绍。这部分内容是极重要的,希望大家认真看一下。

戒色后几乎每一个戒友都会学习传统文化、圣贤教育,因为几乎每一位成功戒除的人都有在学习和推荐圣贤教育,圣贤教育的确可以给我们带来很大的启发,可以学到很高的智慧。大家都知道人生是一场修行,而修行最高的目的是什么?是做一个善人,为社会、国家做出自己的贡献,这对不对?可以说对,但不是最高的目的,最高的目的是——了生死、出轮回。我个人认为,了生死、超越轮回是戒者最高的理想和目标,而净土法门提供了这种可能。首先我们要了解六道轮回,大家可以看看大足石刻六道轮回图,那个石刻真的震撼到我了,当所有的扭曲和误解全部消散后,直面那个古老的真相,内心只剩下了深深的震撼!古圣先贤在向我们传递怎样的信息?众生轮回于生死海中,无始至今,头出头没,轮回六道,了无出期。我们一直在六道里转来转去,一直出不去!如果我们有宿命通,知道过去世在六道里转来转去,吃了那么多苦,还是出不去,那该多悲哀啊!六道就像一座大监狱,有六个牢房,六种待遇,具体每个人的待遇又有所不同,因为每个人福报和业力是不同的,大德称六道为“生死牢狱”!真正有最高觉悟的人,肯定想出去!这是古来多少修行人的最终目标,就是修出去!

不少戒友可能看过《肖申克的救赎》,他花了很长时间在墙上挖了一个洞,出去了!首先他在监狱里想出去,有这个出离心,其次就是想办法出去,我们在六道轮回里也类似,也是一个牢狱,最高觉悟者知道这是一个牢狱,要出去,所以才修行。一般的人不知道这是一个牢狱,甚至会觉得在这里生活很好,但这种好是暂时的,无常的,迟早会经历痛苦,轮回路险,危机四伏。即使再有钱,死的时候一分也带不走,所造的善恶业却带走了。大家可能听过三世怨,第一世修福,第二世享福,一享福就迷惑了,享福的时候忘记修福,造罪业,福享尽了,第三世就进三恶道。福报大,造业大,福满祸生,倚福造业,很危险。

\begin{quotation}
    若今生尚有修持,来生定有世福可享,但一享福,必定要造恶业,既造恶业,则后来之苦报,不忍言说矣!……生西方,即了生脱死,超凡入圣;求来生,则因福造业,因业堕落三恶道。(印光大师)

    当年,我的先师夏莲居老师,听到净土法门,回来后在屋子里就这么乐了好几天,太欢喜了!他说:‘这回我可有办法出去了!’除了这个法门,要出这六道,那就太难了!那要断尽了见、思惑才能出得去。要断见惑、思惑,不光是指咱们人世间的贪、瞋、痴、慢,还要把欲界天的、色界天的、无色界天的贪嗔痴慢全都去掉不可,这样你才能出六道。所以,要出六道是这么难呐!如今,咱们就是凭这一句佛号,有信、有愿,还不管你念多念少、念好念坏,只要你有信有愿,就决定可以往生!因此,咱们出娑婆世界往生极乐,是人人都有份的!希望大家都能跟夏老师一样,都欢喜呀!……活着就是为解决生死,也就是抓住这一生机会永超生死。……我们要下定决心,誓于今生出离生死,往生极乐。(黄念祖老居士)

    我就奉劝大家还是念阿弥陀佛,求生西方极乐世界。这比较方便,你们诸位能相信,就照这实行。(元音老人)
\end{quotation}

那些具有最高觉悟的修行人,都是为了修出去,靠自力修持很难修出去,而修净土法门则容易很多,在此生很有可能修出去。我最喜欢的是竹子的譬喻,\textit{唯有净土法门是横出三界,其余是竖出三界。例如,一个虫子长在竹子里头了,要从竹子里头出来,有两个办法:一个办法,在竹子里一节一节地咬,咬来咬去,出去了,这是竖出三界,咬很多节;净土法门是横出三界,横着咬,当然费点劲,竹子皮硬一点,但就在一个地方钉住,咬一个窟窿,就出来了,这是易行道。(黄念祖老居士)}

\begin{quotation}
    一切法门皆须自力修持到业尽情空时方可了生死,否则任汝功夫深、见地高、功德大,倘有一丝一毫烦恼未尽,则仍旧是轮回中人。既在轮回中,则从迷入悟者甚少,从迷入迷者甚多;又不知还能遇佛法否,即遇佛法,不遇净土之法,则仍旧出苦无期。仗自力则举世难得一二,仗佛力则万不漏一。净土法门以自己之信愿持名感佛,佛则以誓愿摄受,譬如乘轮渡海,非己力之所可比也。(《印光法师文钞》)

    往生西方,如出粪坑监牢,到清净安乐逍遥自在之家乡。……福大则造业大,既造大业,必受大苦。若生西方,则永离众苦、但受诸乐矣!……吾人在生死轮回中久经长劫,所造恶业,无量无边。若仗自己修持之力,欲得灭尽烦恼惑业,以了生脱死,其难愈于登天。若能信佛所说之净土法门,以真信切愿,念阿弥陀佛名号,求生西方,无论业力大,业力小,皆可仗佛慈力,往生西方。(印光大师)
\end{quotation}

有句话叫“谁的青春不迷茫”,其实很多人是迷茫地过一生,不知生从何来,死往何去,非常迷茫地过一生,非常可悲的一生。当遇见善知识了,开示六道轮回的真相,并且告诉我们修出去的方法,那真的就像抓住了救命稻草!净土法门实在太宝贵了!太殊胜了!能得遇净土法门,真的太庆幸了!因为这个法门可以让我们修出去!我们应该抓住此生的机会,超越轮回,往生极乐世界。

《安士全书》被印光法师称为“善世第一奇书”,包括“文昌帝君阴骘文广义节录”、“万善先资”、“欲海回狂”、“西归直指”四部。其中包括行善,也包括戒色,最后是以净土为指归,这样就比较圆满了。我的戒色文章也是以净土为指归,也是受了圣贤教育的启发,就是为了给众生最大的利益。\textit{今天我也以此奉劝,你修什么都好,研究什么都好,但真的要知道生死可怕。你要愿意真正要去度众生、要自觉觉他的话,愿意尽早实现这个愿望的话,你不求生净土是为失大利。(黄念祖老居士)} \textit{末法亿亿人修行,罕一得道。唯依念佛,得度生死。(《大集经》)} 修行就是为了了生死、出轮回。人身难得,得人身是很难的,概率很小,如果这一生修不出去,再想获得这样的机会,就太难了。

\begin{quote}\it
    归去来,魔乡不可停,旷劫来流转,六道尽皆经,到处无余乐,唯闻愁叹声,毕此平生后,入彼涅槃城。(善导大师的偈颂)
\end{quote}

当年我第一次看到这个偈颂时,就被震撼到了!这就像大悲慈父阿弥陀佛在召唤我们回家,善导大师就是阿弥陀佛化身。这个偈颂真的写得太好了,内心一下就被触动到了,这个偈颂我经常会念的。“旷劫来流转,六道尽皆经”,这一生得遇净土法门,轮回流浪汉终于有望回家了,有时念这个偈颂时,眼泪都会流下来,真的深受感召,深受感动。看到阿弥陀佛的画像,也感觉很亲切,很喜欢,的确是大悲慈父,阿弥陀佛和中国的因缘那么深,即使小孩子也会念上几句,谁知道这里面竟然藏着了生死的法门啊!

净土方面我推荐大家可以看印光大师文钞、善导大师的净土思想、黄念祖老居士的大经解、刘素云老师讲无量寿经(共七十集)、大安法师的开示、犟牛居士的开示。这里提一下黄念祖老居士的口语开示,真的是法味极浓、法益极大、法恩极深,非常有亲和力,蜻蜓 FM 和喜马拉雅上可以收听,搜名字就可以找到。恩师爽朗的笑声犹在耳畔,让人如沐春风,内心感觉很欢喜。黄念祖老居士的两本书《大乘无量寿经白话解》《佛说大乘无量寿庄严清净平等觉经解》大家可以看一下,网上能搜到。

我当年刚开始学佛,也不是马上就修净土宗,刚开始听《金刚经》,念楞严咒,大概有一年半的时间,之后机缘成熟就开始念佛了,我也很喜欢念佛。得遇净土法门,真的很殊胜,期待这一生横出三界,往生极乐。也希望有缘的戒友也能修持净土宗,一起求生极乐世界,这是我的一个愿望。在这个世界修行,难度是很大的,因为退缘很多,诱惑太大,大德说过“进一退九”,但一定要有百折不挠之信心,坚持到最后,《无量寿经》里十八大愿:至心信乐,十念必生!在生命走到尽头时,还有十念必生!!!在临终时,应该拿出最大的决心来求生极乐,平时就要每天发愿,具备信愿行三资粮。\textit{净土法门,以信愿行三法为宗。信,谓深信依佛言念佛求生西方必能遂愿,有信自(自心有佛性)、信他(信阿弥陀佛实有),信因信果等义。愿,谓发愿命终往生净土,至彼国得不退转回入此界度化众生。行,分正行与助行。正行指修念佛,助行指兼修礼诵供养、六度万行等。(印光大师)} \textit{念佛法门别无奇特,只是深信、切愿、力行为要耳。(藕益大师)} 往生极乐世界是为了将来更好地度众生,要有度众生的大愿才能相应。最后以彻悟大师的诗句作为结尾:

\begin{multicols}{2}
    \begin{center}
        故乡一别久经秋,切切归心不暂留。 \\ 我念弥陀佛念我,天真父子两相投。 \\ 说着莲邦两泪垂,阎浮苦趣实堪悲。 \\ 世间出世思惟遍,不念弥陀更念谁。 \\ 愿生西方净土中, \\ 九品莲花为父母。 \\ 花开见佛悟无生, \\ 不退菩萨为伴侣。
    \end{center}
\end{multicols}
