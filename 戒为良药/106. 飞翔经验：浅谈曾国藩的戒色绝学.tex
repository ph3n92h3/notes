\subsection{浅谈曾国藩的戒色绝学}

前言:

最近看到一位戒友戒了 70 天左右,然后又开始想适度无害了,他从一开始戒色就没有认清适度无害论,心里还在认同适度无害论,这样心魔就会用适度无害来拼命怂恿他,到时破戒也就势在必然了。还有一位戒友戒了 300 天破戒了,原因就是看了禁欲有害的文章,他对禁欲是否有害认识不清,脑中还存有不少思想误区,当看了那类文章,戒色立场马上就开始动摇了,于是就破戒了。这两位戒友之所以破戒,就是思想认识上还存在不正确的地方,这样戒到一定程度自然就会破戒,这是必然的。关于适度无害论我之前的文章已经驳斥过多次,要戒色成功,必须认清适度无害论,如果在这点上认识不清,那肯定无法戒色成功,因为戒到一定程度就会想搞适度,心存侥幸,对于手淫的高度成瘾性认识不足,对于邪淫的危害认识不足。懂得修心的禁欲是完全无害的,反而还有很多益处,戒色的益处相信很多戒友都体会到了。禁欲是否有害的问题我很早的文章就专门论述过,这方面的认识一定要正确,只有认识正确了,戒色的立场才会像钢铁般坚定,毫不动摇。禁欲有害的文章不要去看,那类文章的观点是完全错误的,既然戒色了,就不要看那类文章了,以免自己动摇,很多资深戒友都懂得远离那类文章,因为有时明知道对方的观点是错误的,但也可能被它洗脑,最后搞得自己不坚定,这方面一定要高度警惕,不要去看。

上季一位戒友反馈说夏季街上的诱惑很大,虽然他尽最大努力做到非礼勿视和断念,但还是感觉很费力,做得还是不够好。他在面对诱惑的对境时,感受到了肉弹强大的冲击力,对他的心理造成了一定的影响。能做到对境不动心,方为大力人,刚开始对境,是可能做得不够好,这很正常,通过不断的实战磨炼,并且自己加强修心和视线管理,慢慢就能做到无动于衷了,从对境心乱到对境无心,这是需要一个过程的,不是一蹴而就的,只有自己加强警惕,及时转移视线,不要聚焦,这几个要点必须反复强化,同时要牢牢看住自己的心,防止内心出现波动,对境时应该要做到心如止水,视而不见。诱惑再大,与我无关,肉弹再猛,对我无效。你诱惑你的,我走我的路,井水不犯河水。《父子合集经》中云:“若人于欲境,心不生放逸,则能越魔网,是为大智者。”对境时就是在检验一个人的定力,而定力来自戒色的觉悟,学习提高觉悟,觉悟产生定力,《宗镜录》:“迷时境摄心,悟时心摄境。”当你迷于女色,那就会被诱惑的境界所夺,把你给摄住了,把你的魂给勾走了,当你真正认识了女色的不净本质,也就不会那么冲动了。色不迷人人自迷,对你而言是很强的诱惑,也许在另外一个人看来,也不过如此。当你的定力强到一定程度,到时任何诱惑都动不了你,就像龙卷风再猛烈,也无法动摇喜马拉雅山。

下面分享一些案例。

\begin{case}
    飞翔老师你好,我是大一新生,在一所全国有名的 985 院校,大学以后有了手机,结果撸得更加频繁了,现在身体很差,学习全部都耽误了,记忆力特别差,原来我特别聪明,什么东西都一学就会,一直是三好学生,现在脑力不行了,连大学的课程感觉都已经跟不上了,自己气质变得很猥琐,还经常熬夜,所以容貌变化很大,我已经撸了四年啦!我现在上嘴唇变得很厚,下巴往回缩,看起来丑哭了,这是因为手淫导致的吗?
    \subparagraph{附评} 能考上 985 院校很不容易,这位戒友是个人才,他撸了四年了,也就是说在撸管的情况下,他还考上了 985,不得不说,他先天的脑力和智商应该是很高的,但是恶果却在四年后出现了,在大学期间充分表现出来了。经常熬夜 + 频繁撸管,纵是吕布泰森也架不住啊!大学熬夜是很普遍的,大多都是贪玩游戏、卧谈会、谈恋爱等。大学是相对自由的,但是大学也是危险的,特别是对于缺少自律的人而言,很可能会导致灾难性的后果。记得我小学时的一位秦姓同学,他在数学方面有天才,得过奥数大奖,但是上大学后却被退学了,原因就是沉迷网游,荒废学业。上大学后住集体宿舍,撸管应该会有所收敛,但有的人会打时间差,趁宿舍没人时来上一发,还有就是躲被窝里看黄撸管,我上大学那会,手机还不能看黄,而现在手机已经能看黄了,所以躲被窝看黄的情况已经变得很常见了。手机用得不好就成了“手雷”,手机给人们的生活带来了极大的便利,但是手机也为邪淫提供了资源,手机上网搜个诱惑的图片,非常容易,撸的资源随便就能搞到,这样就大大增加了废掉的可能。这位戒友刚上大一,身体已经变得很差,脑力也不行了,他一直是三好学生,现在却连大学课程都感觉跟不上了,气质也变得很猥琐。三好学生一旦沉迷撸管,到时就会变成“三差学生”:身体差、脑力差、形象差!丑哭了!邪淫后脸部会变得不对称,国外的戒色文章也讲到了这一点,相由心生,一点不错,邪淫导致嘴唇变厚的情况很多见,我以前也是嘴唇变厚,中医:脾开窍于口,其华在唇。中医认为肾为先天之本,脾为后天之本,肾与脾是相互资助、相互依存的关系。手淫耗损肾精必然会影响到脾,到时嘴唇就容易变厚,而且是不规则、不对称的厚,显得很难看。之前有位戒友发过对比照,手淫后他的嘴唇变得很厚,不规则、不对称、非常难看,再配合他无神的眼神和极差的气色,看上去真的很糟糕、很颓废、很无力,和手淫前判若两人!在相学上也讲到嘴唇对应色欲,嘴唇厚代表色欲重。大家可以看看自己手淫前的照片和手淫后的照片,差别其实是很大的,不仅是精气神的下降,还有诸多微妙的形态方面的坏变化,严重者真的是面目全非。考上大学后更应该要懂得自律,很多人都远离父母,在其他城市上大学,少了家人的管束,加之大学玩乐的氛围比较重,人很容易就会迷失自己。本来上大学是好事,但是一放纵就成了灾难,在大学这个阶段应该要好好把握自己,把主要的精力用在学业上,为将来的工作打好基础。
\end{case}

\begin{case}
    飞翔哥,这两天不知道怎么了,意淫好重啊!我在看书学习的时候,看着看着突然脑子里面就冒出了很多意淫的念头,挡都挡不住,而且那些意淫的念头都是我之前邪淫的时候看过的相当刺激的东西,一下子突然就冲到了脑子里面,好可怕!我知道您说的断意淫非常重要,但是念头一来的时候(很多时候就是画面)根本就挡不住,一下子就让我有了破戒的冲动。我幸好是在教室,不然自己一个人可能一下就破戒了,我该如何让这些念头不再出现,我知道这是因为我之前邪淫太多,脑子里面的刺激记忆太多了,现在应该就像您说的是在翻种子,我特别害怕经受不住翻种子的诱惑而破戒。
    \subparagraph{附评} 这个案例又提到了念头实战,念头实战是我一直强调的部分。当心魔入侵时,就看你的表现了,看你怎么接敌,看你怎么消灭念头。不少新人都是接敌就溃败,没有任何的抵抗力,一到念头实战就立马败下阵来,沦为了心魔的傀儡。这位戒友的思想还存在误区,他说“我该如何让这些念头不再出现”,念头肯定会上来的,不可能不出现,关键就是要立刻消灭之,不能被念头附体。八识田里的淫欲种子是很难消除的,古德修行了几十年才能消除种子,即使禅宗破初关后,八识田里的种子还在,还需要不断精进修行来消除种子。我戒到现在超过五年了,有时念头还是会上来,以图像画面上来的居多,而且是那种印象深刻的、刺激性强的情景在脑海中浮现出来,有时是走路时冒出来,有时是坐着时冒出来,有时是躺着时冒出来,这都不一定的。这位戒友是在教室看书学习时,念头就入侵了,心魔的表现就是念头袭脑,袭脑的速度极快,念头就像隐形的八脚怪一样瞬间就爬上了头脑,企图占据头脑,如果断念不得力,那身体很容易就会被心魔接管!本来身体是在你的控制之下的,但是心魔打败你之后,它就拥有了身体的使用权,而你则变得身不由己,就像被操控的提线木偶。这位戒友说他“挡不住”,为何挡不住,因为他的断念太弱了,当图像插入头脑时,必须立刻消灭之。实战意识一定要强,断念要快、要狠!切忌贪恋和犹豫,当断不断,反受其害!不少人都会问,为何自己断念不行,其实就是两个字:欠练!断念就像一门超杀技,需要不断反观内心、不断发展自己的觉察力,这样才能最终战胜自己的心魔。作为新人,刚开始应该要熟背断 YY 口诀,要背到条件反射的程度,并且要对断 YY 口诀的含义进行正确领会与理解,此外还需加强学习戒色文章提高综合觉悟,对念头的入侵方式有更深入、更全面的认识与了解,增强自己的识别能力。当你的断念变得越来越精深、越来越娴熟,到时被虐的就是心魔!我们必须比心魔强大,如果比心魔弱,就会被心魔虐,这是毫无疑问的,弱小的一方注定被蹂躏,注定会失去自主权。戒色就像打仗一样,是一场内心的战争,断念就是拼刺刀,断念就是白刃战!你的刺刀叫“觉察”!念起即觉,觉之即无!《修心八颂》云:“愿我恒时观此心,烦恼妄念初生时,毁坏自己他众故,立即强行而断除。”面对心魔的入侵时,必须够强硬、够狠决!印光大师云:“当杂念初起时,如一人与万人敌,不可稍有宽纵之心,否则彼作我主,我受彼害矣。”当邪念初生、初起之时,我们就要立刻断掉,断念贵早,千万不可让邪念起势。《修心利刃轮》云:“践踏祸根妄念头,刺中我敌凶手心。”这两句开示非常强悍、非常给力,实战意识极强!断念就是你死我活的战斗,必须降伏自己的心魔!记得前段时间,有一天午后我经历了剧烈的翻种子,大概半小时内,心魔入侵了七次,有怂恿也有图像,我7记觉察就把心魔干下去了,七记觉察就像七道激光,把入侵的念头怪给消灭了。当你有战胜心魔的把握了,就不怕心魔入侵了,因为你的实力是压倒性的,你是稳操胜券、万无一失的。
\end{case}

\begin{case}
    飞翔哥,是这样的,大街上遇到诱惑,但因需要看路,不能够马上转移视线,我虽然没起欲望,但心魔告诉我:你没转移好视线,你失败了,你输了,破戒吧!
    \subparagraph{附评} 知道是心魔怂恿,就应该立刻断掉,千万不可听信,心魔有时会让你觉得自己失败了,这样你就会放弃抵抗,甘心沦为心魔的奴隶,一定要学会识别并消灭之。心魔是非常非常狡猾的,非常非常贼!只要你心中有那么一点点的犹疑和不确定,心魔就会疯狂怂恿你。在戒色实战中,有时是会出现转移视线不及时的情况,自己要认真反省和总结,争取在下次实战中能够有上佳的表现,这位戒友说:“因需要看路,不能够马上转移视线。”路是需要看的,但我们可以转换为“散视模式”,不要去聚焦。我们戒色后要对破戒的定义有清楚的认知,很多新人都会问,遗精算破戒吗?他们对破戒的定义没有基本的了解,所以才会这样问。遗精不算破戒,但要注意控制频率,破戒最根本的定义就是你发生了邪淫的性行为,主要是指手淫、婚前性、嫖娼、婚外情、一夜情等,严格来说沉迷意淫也算破戒,因为意淫会导致暗漏。视线没管理好,这并不能算破戒,但如果放任自己盯着看,那就很容易起邪念,所以这方面要很警惕。很多人都架不住心魔的怂恿,心魔是怂恿专家,怂恿的内容也是五花八门,一旦你在认识上存在误区,那就很容易着了心魔的道。对于心魔的怂恿,我们一定要学会识别,然后要下定决心:不管发生什么,都不破戒!这是最后的底线,必须坚决守住!
\end{case}

\begin{case}
    飞翔哥,你好!请您帮帮我好吗?我真心希望得到您的帮助。我 12 岁开始 SY,手淫九年后,上高三时久坐,熬夜,上网吧通宵看黄 SY,到高三下学期就与同学发生一点矛盾,然后就怀疑他派人暗杀我,觉得我家周围都有便衣警察要逮捕我,走在马路上就觉得所有的汽车都是受我检阅的,去饭馆吃饭就认为饭馆里其他吃饭的人都是那个同学派来要暗杀我的。21 岁(也就是 2008 年)时被医院精神科诊断为精神分裂症,是偏执型。从那时起就开始吃西药。吃着西药时,什么也不能干,不能上学,不能工作,无比痛苦绝望,多次想死,想自杀,于是就继续在家看黄 SY,从 SY 中寻求解脱。(当时还认同手淫无害论,现在坚定地不认同了,已戒色三个月,但一减药就有症状。)到今年吃西药八年了,西药不能停,一停精神病就复发,请问飞翔哥,我如何才能恢复?
    \subparagraph{附评} 大家对精神分裂症应该都有所耳闻,此病最大的特点就是妄想,其中最多的就是被害妄想、关系妄想等,别人很正常的举动,在他眼中就是针对他的谋害意图。大家都听说过精神病人伤人乃至杀人的事情,因为在他脑子里,别人就是来杀他的,为了避免被杀,所以要先杀掉别人。精分我在第 103 季的第 9 个案例里也讲到过,那个戒友也是因为手淫得了精分,有一定戒色觉悟的戒友都应该知道手淫会导致身心的双重失调,而身心又是互相影响的,身体失调了也会影响到心理,精分的发病和生理的失调是密切相关的。国外在 19 世纪就知道手淫会导致精分,无知的人会以为这是危言耸听,我当初也不知道手淫会导致精分,当我深入学习中医医理和了解手淫的危害后,再结合那么多的真实案例,现在我很确信手淫是诱发精分的一个潜在因素。在手淫的最初几年,出现的心理失调主要是情绪变差、烦躁易怒,这样就会影响家庭和睦,随着伤精程度的加深,神经症就会找上门来了,到时强迫症、焦虑症、神衰、社恐、疑病症等就会出现,神经症已经让患者很痛苦了,但这还不算最极端的心理失调,最极端的心理失调就是精神分裂。精分属于重型精神疾病,而神经症相对属于轻型精神疾病,其实神经症已经让人生不如死了,只不过精分的强度更大一些。据新闻报道,目前我国精神疾病患者在一亿以上,72.3\% 不知自己已经患病,重度精神病患者超过 1600 万,致残率高达 60\%,自杀率达 30\%。现代社会压力大,很多人经常熬夜,加之邪淫泛滥、无害论误导,于是就疯狂手淫,把五脏的精华都当垃圾一样射掉,最后弄不好真会得上精分。这位戒友现在觉醒了,不过才戒三个月,时间还太短,他这种情况起码要戒一年以上,真正的恢复期应该在三年以上,好好坚持戒色养生,再配合积极治疗,这样心理的失调就会逐步恢复正常。
\end{case}

\begin{case}
    飞翔哥,我截止今天已经戒了 126 天,126 天前的我根本没想过什么是目标,就在我戒了一百天的时候我辞掉年薪十多万的工作开始准备考研,现在已经复习了接近一个月,我现在每天都是运动养生加学习,身体智力都在逐步恢复,我真的不敢想象 126 天前的我是什么鬼模样,我以前从来不敢想的理想,我现在有胆去想了,我有信心做好任何事,这是戒色于我最大的改变!
    \subparagraph{附评} 辞掉年薪十多万的工作,有的人可能会觉得不理解,我相信这位戒友是经过深思熟虑的,戒到一定程度,心里自然会充满底气,到时就有能量去为人生更高的目标而奋斗了,到时出来也许就不是年薪十万了,也许是几十万乃至上百万了。之前有位戒友,原来他的年薪是四万多,后来戒色两年多,精力、底气、斗志、容貌气质全面提升,结果升任为公司副总,年薪一下变成了几十万。很多人命里是很有福报的,但是因为邪淫,福报无法全部发挥出来,当他戒掉邪淫后,马上就不一样了。这位戒友说他不敢想象 126 天前的自己是什么鬼模样,说明那时的他虽然拿着高薪,但是身体状态的确很差,当身体差到一定程度,工作就无法胜任了,到时就会被动辞职,而这位戒友是戒色后主动辞职的,当养足了肾精的核能,他就有能量去实现更高的理想了,我相信这位戒友会成功的,因为他的底气摆在那里了。两个肾脏就像安装在体内的两个核反应堆,肾精足,能量就足,到时就可以用最强大的动力去拼搏和奋斗自己的人生。很多人撸到后来,动力尽丧,底气尽失,理想这两个字早就抛之九霄云外了,每天就是混日子,得过且过。大家应该都看过两个算式,1.01 的 365 次方等于 37.8,而 0.99 的 365 次方等于 0.03,有时真的不能差那么一点点,肾精泄掉后,就会影响一个人的精气神、精力状态和脑力状态,这方面差了,事业也会受到很大影响,差之毫厘谬以千里。这位戒友说自己“有胆去想了”、“有信心做好任何事”,这种心态正是成功者的心态,相信他将来会一飞冲天、大展宏图。
\end{case}

\begin{case}
    飞翔大哥,我已经发愿:把戒色作为自己终身的事业不断为之奋斗,而戒色吧就是末学永远的办公室与战场。近来体会到一件事:当自己热爱戒色的时候,不管是戒色学习,还是帮助其他的戒友,都给自己带来了一种无法言说的快乐与满足,这种快乐与满足根本不是手淫邪淫所能相比的。这种快乐和满足是发自内心的,持久的,而且好像还能够开发自己智慧,我将继续不断努力实践下去。
    \subparagraph{附评} 这位戒友所发的愿非常好,我也曾经发过类似的愿,我发愿帮助更多的人戒色,把帮助他人戒色作为自己终生的公益事业来坚决执行下去。有了愿力,行动起来就会充满动力,要不断发愿,不断给自己鼓劲,让自己坚决、坚定、坚持地做下去,永不退转。当你真正热爱戒色和无私奉献了,就会感受到真正的大快乐,真正的大快乐来自纯净的心灵,也来自于无私地利他,这是一种不求回报、无丝毫后悔的付出,付出的人其实是最快乐的、最满足的,付出比索取要快乐很多倍。帮助别人要真正发自内心,而且要坚持做下去,当做自己一生的公益事业来做,很多明星和富商都在做慈善,做慈善就是在增加自己的正能量与福报,他们很懂得这个道理,利他其实就是利己,当你没有求回报、求福报的心,这样去利他就会更纯粹,收获的快乐也会成倍地增长。《周易》云:“积善之家,必有余庆;积不善之家,必有余殃。”积善就像学生修学分一样,不要放过小善,积小善成大善,必须注重积累,一点点坚持去做,不要中断!《太上感应篇》:“夫心起于善,善虽未为,而吉神已随之;或心起于恶,恶虽未为,而凶神已随之。”行善如春园之草,不见其长,但日有所增;行恶如磨刀之石,不见其消,但日有所损。人来到这个世上,一定要学会行善积德,尽己之人帮助更多的人,这是人生的必修课。
\end{case}

\begin{case}
    我 25,手淫七年,尿毒症晚期了!大家千万不要学我,给你们一个警示,为自己多做一点福报。手淫真的害人啊!大家如果发现自己小便有泡沫,要赶紧去医院检查。健康无小事,我就是不当回事,血泪的教训。本人的教训就是不按时检查身体,小便有泡沫三四年不当回事,希望大家要引起重视,现在临床上尿毒症是无法治愈的,痛苦一生。西医检测尿毒症的唯一标准就是肌酐,肾小球滤过率,我年前检查的时候已经 476 了,我 2011 年底体检的时候还没有问题,这个病一般病变的时间为一年或者更长,我在医院活体肾组织穿刺结果显示巨灶增生性 iga 肾病,医生说已经几年了。我得了这病,确实反思了很多,包括以前自己的思考行为方式,我以前不管生活还是上班总是喜欢与人争强斗狠,什么事都只想占便宜,一点事就喜欢放心里,我想这个对身体很不好,加上熬夜,抽烟很多。我血压高的时候高压到 220,低压 180,医生说要是上了年纪的人早就脑溢血死了。
    \subparagraph{附评} 网上很多人会问手淫会不会得尿毒症,因为他们担心自己会得尿毒症,根据我的深入研究,手淫是可能会导致尿毒症的,但有一个前提,那就是伤到一定程度才可能,而且还需综合其他致病因素。撸久了,伤深了,什么病都可能得上的,中医认为尿毒症主要有五个病因:一、情志所伤,长期情志不舒,忧思恚怒,肝失疏泄,肝气郁结,三焦气机不畅;二、失治误治,如在原发病治疗中,治不得当,迁延日久;三、外邪侵袭;四、烦劳过度,酒色过度,损伤肾气,肾阳不足,命门火衰,脾肾两虚。脾虚则呕恶不食,肾虚则不能化气行水,湿浊弥漫,波及他脏,最后五脏俱败;五、饮食不节,饮食不节包括暴饮暴食、长期饥饿、偏食、饮酒过度等,都会损伤脾胃,使生养气血功能受到影响。若脾胃长期受损,必致气血来源不足,后天无以充养先天,故肾脏亦损。中医根据辨证把尿毒症具体分为热毒内蕴型、湿热互结型、肝胃不和型、脾虚痰浊型、脾虚血瘀型、肾虚血瘀型、阳虚血瘀型、脾肾阳虚型八大类型。我那时就是脾肾阳虚,但还没有伤到尿毒症的程度,尿毒症这个病和纵欲是密切相关的。有位戒友的爸爸得了尿毒症,已经去世了,对他的影响很大,他爸爸就是因为放纵男女之事得的尿毒症。在他照顾爸爸的期间认识一个小伙子,大概 25 岁,透析一年了,听说家里乱七八糟地欠了一屁股的债,活得很悲惨。我之前的案例也分享过罹患尿毒症的戒友,当真正得上尿毒症了,人生将会陷入更浓重的灰暗。这位戒友得上尿毒症,主要就是因为手淫、熬夜、抽烟,加之争强斗狠导致情志不舒,可怜他年纪轻轻却已经是尿毒症晚期了,纵欲戕生,古今同慨,《遵生八笺》云:“若耗散真精不已,疾病随生,死亡随至。”古德提倡保精、惜精、聚精,因为他们深知肾精与健康的关系,《西游记》第五十五回里唐僧道:“我的真阳为至宝,怎肯轻与你这粉骷髅。”至宝岂可轻泄,那些女优根本不值得一撸,都是粉骷髅而已,为的就是榨干你身体的精华,让你沦为废人,乃至趋于棺材。
\end{case}

\begin{case}
    很荣幸进入戒色吧,是戒色吧在黑暗中给了我光明的指引,让我有了重生的机会,我是 1998 年开始手淫的,到现在十几年了,手淫让我从纯真少年变成了地狱的丑鬼,病症缠身,摧毁了我的一生,后悔当初年幼无知啊!
    \subparagraph{附评} 在开始手淫后,最美好的东西就失去了,最纯粹的大快乐也失去了,被困在了邪淫的低频牢笼中。唯有纯净的心灵才能带给你真正的富足感,沉迷于邪淫的生活只会加强你的匮乏感和空虚感,不论你看了多少黄片,不管你撸了多少次,你都不会感到真正满足,手淫就是一种饮鸩止渴的行为。即使你看了上亿 G 的黄片,即使 100 个屏幕同时在你面前播放,你最终还是会感到深深的厌倦,当多巴胺断崖式下跌后,你就会觉得任何黄片都味同嚼蜡,索然无味,再也提不起一丁点的兴趣。快感是用宝贵的肾精兑现的,为了几秒快感而把五脏六腑掏空,实在是愚痴至极,就像用钻石换糖吃一样,而且还是毒糖!孩子总是处于一种没来由的快乐中,因为他们的心灵纯度极高,还没有被黄毒污染,所以能感受到纯粹的大快乐,而且那种快乐的感觉就像泉水一样不断往外冒,并且伴随着不断爆发的新鲜感与神奇感。到了发育期,心里开始起邪念后,心灵纯度就急剧下降了,从 100\% 下降到 30\% - 10\% 以内,下降得越多,就越痛苦。圣贤教育之所以提倡“思无邪”,就是为了帮助我们提升自己的心灵纯度,当心灵纯度恢复了,就能再次感受到纯粹的大快乐和极致的美好。撸前是纯真少年,宛如生活在天堂,每一天都是奇迹,撸后渐渐变成了地狱的丑鬼,疾病缠身,惶恐而颓废。戒色吧就是人生的灯塔,指引迷途的孩子重返纯净王国,只有戒除了邪淫,才能再次感受到纯净与美好,否则,即使一天 24 小时不停地撸,也是极度不幸的,因为他正在自毁的道路上一路狂奔,疯撸只是在加速自己的毁灭!
\end{case}

\begin{case}
    我分享下自己的经历。最早我一直纵欲,成绩很惨,全年级 650,我 629,现在戒色,慢慢升到我们理科 89 名,班级名次也是一次比一次高,这次期末考试更是考进了重点班!邪门的是我一个朋友经常纵欲,然后他一次考得比一次差(劝不住),我戒色确实一次比一次好!
    \subparagraph{附评} 这位戒友从差生逆袭进了重点班,真的很励志,纵欲泄脑力,因为肾通脑,肾精会化生脑髓,真正懂得医理之人,肯定会懂得戒色意味着什么。大家都喜欢用配置好的电脑,如果让你退回到十年前的那种电脑配置,你一定会觉得很慢、很卡、很落后,当升级到高配置的电脑后,你就会觉得处理信息的速度加快了,整体性能有了质的飞跃,和过去不能同日而语。撸管就是在降低人脑的配置,让大脑处理信息的速度变慢、变差,在沉迷撸管后,会感觉脑力严重下降,记忆力、注意力、思维力都会不同程度下降,到时就会感觉自己变笨了,原来一下就看懂的题目,现在看很多遍都搞不明白了,就像一把刀变钝变锈一样。在戒色一段时间后,人脑的整体性能就会得到提升,到时就会感觉脑力上来了,精力上来了,人也会变得沉得住气,不会心浮气躁,看不进书。戒色给大脑带来了更快、更棒的运行速度,处理题目时也得心应手,势如破竹,非常流畅。从最差撸脑蜕变成最强大脑,戒色可以帮你!戒色升级你的配置,升级你的处理器,处理那些题目就像砍瓜切菜一般。很多人都喜欢给电脑跑分测性能,有的跑分可以达到十万、二十万乃至近三十万,如果可以给人脑跑分,你一定会发现在沉迷撸管后,人脑的跑分会大幅下降。戒色就是在给人脑提升跑分,让你的大脑性能卓越,到时不管是学习还是工作,都能大幅提升效率。这位戒友的逆袭非常给力,可惜他的朋友不听劝,考得一次比一次差,人与人之间的差距就在戒撸上拉开了,当你深知这个提升脑力的核心秘密后,你就会异常坚定地戒下去,坚持戒色,不仅让脑力飞,也让精力飞!让你拥有飞一般的畅快体验!
\end{case}

\begin{case}
    没有破的那两个月,感觉无比轻松,过年回家,家里人都说我变帅了,身体变好了。惭愧的是,戒色依然没有成功。这次又是六十天,第三个。然而最近一周破了两次。我毕竟太年轻,文字读懂了,其实道理并没有懂。两个月前,我觉得我的人生已经没有办法,只能忍着痛苦、贫穷、孤独,两个月后,自己相貌变帅,体力变好,然而自己又想变得更好,这个时候,反而破戒了。我估计我是个下下根人,只能在被生活一次次虐的过程中,略微领悟点道理。
    \subparagraph{附评} 戒色可以让一个人恢复身心健康,可以让一个人重新变帅,但是要守住戒色成果实属不易,因为心魔环伺,随时都准备进行激烈的反扑,企图再次控制你。这位戒友说自己毕竟太年轻,对戒色文章的内容还没有真正吸收内化,虽然文字读懂了,但是道理却没有更深刻的认识与理解,当觉悟存在缺陷,心魔就会专攻觉悟的短板,戒到一定程度自然就会破戒。要想变得更好,有这种想法很正常,但千万不能出现急躁的情绪,所谓欲速则不达,戒色一定要注意“戒骄戒躁”,对变得更好的期待也要合情合理,不能要求过高,否则达不到就容易灰心丧气。吃一堑,长一智,要学会从失败中学习,从失败中崛起。年轻有年轻的好处,身体恢复快,有热情,有冲劲,年轻也有年轻的缺点,那就是心智还不是很成熟,内心还不够稳重,取得一点戒色成绩就可能会出现骄傲的想法。只有在一次次被虐的过程中,自己深刻反省和总结,不断加强学习补强和完善觉悟,这样才能越戒越好。能戒六十天,已经算是略有小成了,这位戒友戒色两个月产生的蜕变是很大的,但可惜没守住,很多道理在当时阅读时貌似懂了,但是在真正破戒后才明白自己根本没懂,这时再回过头来研读前辈的文章,才会有更深入的切身体会。这位戒友说没破的两个月,感觉无比轻松,我觉得他说得很好,戒掉恶习后,就像重新“放出来”一样,就像卸下了千斤重担,人会有一种超脱感和自由感,内心会感觉更快乐、更祥和,与家人相处也会更和睦,戒色会带来很多美好的感觉,戒色可以让你有足够的能量来面对以后的人生。对于暂时的失败,不要气馁,通过对失败的总结和反省,可以让自己戒得更好,一定要积极思考,多鼓励自己,多激励自己,拿出勇猛精进的势头,杀翻心魔贼!杀出一条血路,杀出戒色的生天!!!
\end{case}

下面步入正文。

这季是关于曾国藩的,很多戒色前辈都提到了曾国藩,曾国藩的修身之道的确很值得学习和借鉴,他“立德、立言、立功”三不朽,有“千古完人”之美誉。曾国藩(1811 - 1872),汉族,宗圣曾子七十世孙。中国近代政治家、战略家、理学家、文学家,湘军的创立者和统帅,与李鸿章、左宗棠、张之洞并称“晚清四大名臣”,官至两江总督、直隶总督、武英殿大学士,封一等毅勇侯,谥曰文正。曾国藩的军事思想影响了好几代人,很多历史名人都对曾国藩推崇备至,民国军事家蒋方震在他的《国防论》中赞赏曾国藩是近代史上“一个军事天才家”,还说凡领军者都应该效法曾国藩。毛泽东曾经说:“吾于近人,独服曾国藩。观其收拾洪杨一役,完美无缺,设使易以他人,岂能若是?”陈毅元帅也说过:“曾国藩用兵很有一套,在军事上很值得研究。”曾国藩的军事才华可见一斑。蒋介石也推崇曾国藩,蒋多次告诫他的子弟僚属:“应多看曾文正,胡林翼等书版及书礼”,他审订《曾胡治兵语录注释》时说:“曾氏已足为吾人之师资矣。”在黄浦军校,他以曾国藩的《爱民歌》训导学生,蒋说曾、左之所以能打败洪、杨,是因为他们的道德学问、精神与信心胜过敌人。

1811 年(嘉庆十六年),曾国藩出生于湖南长沙的一个普通耕读家庭。兄妹九人,曾国藩为长子。1832 年(道光十二年),曾国藩考取了秀才,并与欧阳沧溟之女成婚。连考两次会试不中,随后又努力复习一年。1838 年(道光十八年),曾国藩再次参加会试,终于中试,殿试位列三甲第四十二名,赐同进士出身,自此,他一步一步地踏上仕途之路,在京十多年间,曾国藩就是这样坚韧不拔地沿着这条仕途之道,步步升迁到二品官位。十年七迁,连跃十级。曾国藩出生并不显赫,祖辈皆务农,其父亲是一位落第秀才。曾国藩为什么能集众多辉煌于一身,取得巨大成就呢?在曾国藩遗留下来写给家人的书信及日记中,可以得出结论,他能取得如此成就的关键就是在修身上有着不同凡响的造诣,而修身的根本在于修心,也就是说,曾国藩在修心方面的造诣已经达到了相当高的程度。

曾国藩可以说是一位“励志帝”,一生相当坎坷,在与太平军交战的十年中,曾国藩数次身陷绝境,也曾经跳水自杀过,幸好被部下救起。曾国藩从来不认为自己很聪明,他多次讲自己很驽钝。他考进士,考了三次,28 岁才考上,可谓屡考屡败,屡败屡考,终于被他考上,这种韧性很值得我们学习,曾国藩是从失败中成长起来的,不管是考试还是打仗,他都失败过很多次,经过不断总结经验教训,才取得了最后的成功。年轻时期的曾国藩也和我们一样,有着诸多的缺点,但是曾国藩最大的优点就是有着极强的反省精神,懂得从错误中吸取教训。曾国藩的脾气曾经也很火爆,刚到北京头几年与朋友吵过两次大架。第一次是与同乡、刑部主事郑小珊因一言不合,恶言相向,“肆口谩骂,忿戾不顾,几于忘身及亲”。另一次与同乡金藻起了口角,“大发忿不可遏,虽经友人理谕,犹复肆口谩骂,比时绝无忌惮。”这几句形象地描绘了曾国藩性格中暴烈冲动的一面,有人说年轻时期的曾国藩是“愤青”,说得也挺有道理。

曾国藩也好色,曾国藩日记中多次记载,比如在朋友家看到主妇,“注视数次,大无礼”。在另一家“目屡邪视”,并且批评自己“直不是人,耻心丧尽,更问其他?”纵欲是戕身伐命的危险之举,曾国藩深知这一点,他说自己“明知体气羸弱,而不知节制,不孝莫此为大。”道光二十二年十一月初四,他为此大骂了自己一次。那一天他早起读了读书,没有所得,而“午初,人欲横炽,不复能制”,做了“不应该做”的事,遂骂自己“真禽兽矣!”不为圣贤,便为禽兽,这是曾国藩的人生哲学,曾国藩对自己的要求是极为严格的,他知道自己必须做到“制欲”,否则就会被心魔操控,耗泄掉宝贵肾精,身不由己。曾国藩这么要强的人,绝对不允许自己屡次犯这种错误。

之所以提曾国藩,就是因为曾国藩一开始也是修养不佳,缺点很多,但是后来他改了,他把愤青的能量转化到学习圣贤教育、活出圣贤教诲上面来了,本来那股暴烈冲动的能量是很可怕的,很可能导致自毁,但是他这么一转化,就变成了推动他成功的强劲动力。曾国藩深知戒色的重要性,孔子讲君子有三戒:“少之时,血气未定,戒之在色;及其壮也,血气方刚,戒之在斗;及其老也,血气既衰,戒之在得。”君子的第一戒就是戒色,戒色也就是戒邪淫,儒释道都是要求戒邪淫的,有的人会说佛教要求戒邪淫,其实不然,圣贤教育都要求戒除邪淫,不仅中国的传统文化,即使国外圣人也是如此要求的,国外的某些经典甚至对性的禁忌和规范更严格、更系统。圣人深谙戒邪淫的重要性,在这一点上,古今中外的圣贤教育都是高度一致的,肾精能量是不能随便耗泄的,因为肾精关乎一个人的智力、精神、容貌气质、身心健康各个方面,必须懂得珍惜这股能量。

\paragraph{曾三戒,君子无戒不立}

曾国藩早年有“三大戒”,即“戒多言、戒忿怒、戒忮求”。多言必失,所以要戒;之前曾国藩脾气火爆,这样不仅伤身,而且还很容易得罪人,曾国藩深知“惩忿窒欲”的道理,所以忿怒要戒;忮求是嫉害贪求的意思,这在官场是一大忌讳,是必须要戒除的。除了这三戒,曾国藩还戒烟、戒色,应该叫“曾五戒”才是。因为抽烟太滥,曾国藩受到了师长的训斥。曾国藩是个自尊心很强的人,知道抽烟有百害而无一益,于是决心戒烟。为了表示自己的决心,他还将自己原来的名字“子城”改为“涤生”,并发誓从此悔过自新,重新做人。在日记中他这样解释“涤生”二字,“涤者,取涤其旧染之污也;生者,取明袁了凡之言:从前种种,譬如昨日死;以后种种,譬如今日生。”关于戒烟,他对其弟说:“自戒潮烟以来,心神彷徨几若无主。遏欲之难,类如此矣!不挟破釜沉舟之势,讵有济哉?”戒烟和戒色有相通之处,但戒色的难度更大一些,毕竟烟是身外之物,而色是在身上的,是根深蒂固的一种欲望。曾国藩拜理学大师唐鉴为师,在唐鉴的指导下,他痛改往日恶习,坚持每天写日记反省,并养成慎独和静坐的习惯。曾国藩针对自己的不良习惯痛下功夫,绝不姑息苟且,不管是戒烟还是戒色都取得了成功。君子无戒不立,有了戒,能量就守住了,底气也会随之而来,心地也会变得光明磊落,一派正大光明的气象,气宇轩昂,气度非凡。

\paragraph{曾国藩的修心功夫}

\subparagraph{语录一:才觉私意起,便克去,此是大勇。}

曾国藩带兵打仗厉害,但他却没说攻城拔寨的人是大勇,反而他说“克去私意”才是大勇。克去私意其实就是在修心,就是在断念,能够断念的人才是真正的大勇!戒色的核心就是观心断念,一定要在起心动念上修,戒色是内心的战斗,是正念觉察与邪念的斗争,外在的战场可以看得到硝烟,但是内心的战场只有自己一个人知道,从这个角度来讲,戒者是真正的孤胆英雄!就像一个人干掉一支庞大的军队一样,如一人与万人敌,要有一夫当关万夫莫开的气势,必须镇得住心魔。谁敢横刀立马,唯我戒色猛将!要戒出大将之风,念来即斩,来多少杀多少,在邪念的尸山上屹立不倒!

\subparagraph{语录二:去好色之私。}

曾国藩的《治心经》:“所知在好德,而所私在好色,不能去好色之私,则不能不欺其好德之知矣。”好色从某种角度来讲,也可以说是一种自私,好色会导致伤身败德,一个人沉迷于色情,必将影响他的道德修为。子曰:“吾未见好德如好色者也。”好色是人性的弱点,一个人要想有一番作为,必须懂得“去好色之私”。很多官员都因为色关过不去而最终落马,一个人如果开始邪淫了,报应必将是惨烈的,天道祸淫福善,这是规律,和规律对着干,下场可想而知。每个人都知道电的厉害,所以不会去摸电门,其实邪淫就像高压线一样,犯邪淫的后果是非常严重的,刚开始也许看不出来,但时间长了迟早会出事。很多人在症状爆发时都会被吓到,之前沉迷撸管时完全没料到会这样。曾国藩《治心经》的第一句:“治心之道,先去其毒,阳恶曰忿,阴恶曰欲。”人都是有欲望的,本来欲望无好无坏,但如果控制不住欲望,那就会导致灾难性的后果,就像洪水一样把人吞没。

\subparagraph{语录三:善观己者观心。}

观心这两个字的力量太大了,修行的最核心就是观心,曾国藩当然深知观心的重要性。我的戒色文章一直在强调观心断念,当你向内看了,你就真正入门了,否则戒来戒去,都不知道观察自己的内心,那就等于是门外汉。观即观察、觉察,觉察本身就具有消灭念头的功能,但有一个前提,那就是“念起即觉”,念头刚起来时你就要马上觉察到,否则等念头起势了,再觉察,这时候已经晚了。娴熟于观心的戒者,在念头刚起来时就能消灭之,而不懂观心的人,肯定会跟着念头跑,他们会认同念头为自己,这种跟着念头跑其实是一种无意识的行为,念头一起,就跟着跑,自己一点也做不得主,当你开始观心了,你就开启了更高的意识层面,你成了一个观察者,你可以看着念头,你可以使之消失。刚开始练习观心,很多戒友都感觉到念头势能的强大,那股力量起来后,会把你拖入一个漩涡,一个念头接着一个念头生出来,很难停下来。当继续练习观心,就会发现念头的力量开始减弱了,因为你发展了自己的觉察力,慢慢地,就能做到在念头生起的一刹那,就立刻消灭之。印光大师云:“非将死字挂在额颅上,决难令妄想投降,妄想既不能投降,则妄想成主,本心成奴,是以多少出格英豪,被妄想驱逐于三恶道中,永无出期,可不哀哉!”印光大师所说的妄想,其范围更广,但道理都是一样的,一开始戒,就要有破釜沉舟之决心与勇气,这是你死我活的战斗!曾国藩:“知此事万非疲软人所能胜,须是刚猛,用血战功夫,断不可弱。”曾国藩把打仗的狠劲用在了修心上来,其效果绝对是突飞猛进的。血战功夫这四个字极为给力,唯有立志坚定,行动才能斩钉截铁,必须要有刚烈威猛之作风,以血战到底的气概来与心魔作斗争!要拿出拼命的架势来戒,往死里戒,戒到死,死磕到底,死硬不破!气势上必须要盖过心魔。

\subparagraph{语录四:唯灭动心,不灭照心;但凝空心,不凝住心。}

这是曾国藩日记里的十六个字,出自《洞玄灵宝定观经》,被曾国藩引用在日记里了。这其实就是在讲修心的功夫,“唯灭动心”,很好理解,当念头动起来了,要立刻灭除,不能被念头附体,不能被念头奴役,否则就会导致身不由己而破戒。“不灭照心”,照即观,观即觉察,就是要一直保持对内心的觉察,这个觉察的功夫不能中断。“但凝空心”,就是要保持内心的清净无染,空心也就是清净心,也就是印光大师所说的本心。“不凝住心”,就是不要住在念头和外在的事物上,和佛家所讲的不执著、不住著是一个意思,因为你一有住著,内心就不清净了,就无法保持在空位。这十六个字的含义,看似简单,实则非常深奥。有的戒友可能会说,念头不能动,那岂不是不能学习和工作了,变成了木头和石头了?这十六个字不是让你变成无思想的木头人,而是让你学会做念头的主人,该起念头学习和工作时,就用念头,用完了再回到空位,空位就像你的家,就像你的大本营一样,而念头只是工具,用到时就用,很多人不是在用念头,而是成了念头的奴隶,就像一个人无法驯服野马,结果被野马带着疯跑,做不得主。在学习和工作时,你可以有意识地回到空位,因为当你什么也不想时,灵感往往会自动冒出来。这十六个字大家要好好思考和体会,意思很深。曾国藩在修心方面很有体悟,他能位极人臣,的确是修为所致。

\subparagraph{语录五:欲心一萌,当思礼义以胜之。}

曾国藩:“防身当若御虏,一跌则全军败没。爱身当如处子,一失则万事瓦裂。”曾国藩形容得很好,如果被心魔攻破了,就像全军覆没一样,心魔冲进城内,烧杀抢掠,为所欲为,然后留下满目疮痍,扬长而去。当邪念初萌时,就要立刻对治,曾国藩这里给出的方法就是思维对治,通过思维礼义廉耻来对治邪念。断念有好几种方式,觉察断念是一种,也就是念起即觉,觉之即无;思维对治也是一种,白骨观不净观都属思维对治;念佛号也是一种,邪念一起,马上念佛号,一句“阿弥陀佛”,犹如金刚王宝剑,瞬间斩断邪念;还有转移注意力也算一种,只不过相对比较初级,转移注意力不如前三种来得直接。前三种我用得比较多,转移注意力我作为辅助,断念需要一个精进练习的过程,就像练习出刀一样,必须快如闪电,手起刀落。

《论语·季氏》:“礼之所兴,众之所治也;礼之所废,众之所乱也。”中国是礼仪之邦,但近些年来道德急剧滑坡,礼仪之邦已渐行渐远,时代呼唤正能量,时代也需要圣贤教育的再次启蒙。“礼义廉耻,国之四维;四维不张,国乃灭亡。”这是两千七百年前振兴齐国,成就伟业的一代英才管仲的千古名言,孔子也专门讲到:“不学礼,无以立。”当你思维礼义廉耻的道理时,你就会觉得邪淫这种事情是万万不能做的。荀子:“礼者,人道之极也。”《晏子春秋》:“凡人之所以贵于禽兽者,以有礼也。”禽兽可以在马路上交配,因为它们不懂得礼义廉耻,它们完全没有概念,而人如果不懂得礼义廉耻,那就不仅仅是禽兽了,可以说是连禽兽都不如了,《礼记》:“人有礼则安,无礼则危。”一个人懂得礼义廉耻了,他就会主动约束自己的行为,心里也会很安定,如果不懂得礼义廉耻,那么很可能就会乱来,乱来的结果必然会导致身心健康的危机。邪淫伤身败德,这种事情为君子所不齿,是应该力戒的!孔子:“非礼勿视,非礼勿听,非礼勿言,非礼勿动。”我们要懂得用礼来约束自己。《四十二章经》:“想其老者如母,长者如姊,少者如妹,稚者如子。生度脱心,息灭恶念。”邪淫的念头和礼义廉耻是相违背的,很多人见到女人就意淫,不仅意淫同学,还意淫老师,甚至还意淫家人,完全丧失礼义廉耻了,胸中没有半点正气,基本就是半兽人了,每天活着就想着如何宣泄自己的兽欲,彻底沦为了下半身动物,邪淫是一种退化,让人退化成禽兽不如的东西,很多人也知道这样不好,但克制不住,曾国藩给出了对治的方法,那就是思维礼义廉耻,我们要做正人君子,不要做邪淫畜生。

\paragraph{内断于心,自为主持}

曾国藩:“内持定见而六辔在手。”心中具备坚定的信念,且对关键问题有着正确的看法,这样方能稳健而不动摇,就像手握六辔一样很好地驾驭住车马的行驶,非常稳当。曾国藩是一位极爱读书之人,很懂得学习的重要性,他曾说过,读书有多方面好处:一是能树立正确信仰(人生观、价值观);二是能“自卫其生”,意思是能让人学到生存技巧、本领;三是能保养身体。曾国藩自己更是无一日不读书,终生手不释卷,直到临死前一天还在写日记、读书。曾国藩:“盖世人读书,第一要有志,第二要有识,第三要有恒。有志则断不甘为下流;有识则知学问无尽,不敢以一得自足,如河伯之观海,如井蛙之窥天,皆无识者也;有恒则断无不成之事。此三者缺一不可。”曾国藩 22 岁考中秀才,随后参加过两次会考都落榜。在第二次落榜之后,他取道南下,准备游历天下增长见识,实现“行万里路,读万卷书”的宏愿。到南京时,身无分文,于是在一位朋友那里借了银子一百两。然而,他却用一百两银子买了一部《二十三史》,当时一个七品县官一年俸禄才是四十两银子左右,可见曾国藩对读书的钟爱。曾国藩发愤攻读一年,这部二十三史全部阅读完毕,此后便形成了每天评点史书十页的习惯,一生从未间断,一部二十三史烂熟于心。在其随后身居高位时,他更是每日坚持不懈学习诸子百家,并在家信中时常告诫兄弟及儿子,要求他们每天都要读书学习。戒色方面,我一直在强调学习提高觉悟,觉悟降伏心魔,戒色神力来源于学习,只有通过不断学习才能让觉悟不断提升,越悟越深,进入真正精深化境的层次,到时对很多问题的看法就会产生定见,定见这两个字的威力非同小可,不少戒友戒到一定程度就会自动破戒,为什么会这样?因为他们对于某些问题的看法存在思想误区,缺少定见,心里总是在犹疑不定,心魔就是利用他们觉悟上的缺陷而进行的针对性的怂恿。所以,如果缺少定见,那就很容易被心魔攻破,心中真正有了定见,就不会动摇了,因为定见犹如定海神针一般,有了定见就十分稳当。

\paragraph{君子不重则不威}

《论语》里有“君子不重则不威”之说,观人先取威仪,如虎下山,百兽自惊。如鹰升腾,狐兔自战,不怒而威,正气凛然。曾国藩讲“稳重从容,可当大事”,曾国藩最大的特点就是“行步极厚重,言语迟缓”。曾国藩之重源于祖父,他给儿子写信说,说他曾经仔细观察,“祖父仪表绝人,全在一重字。”在军事上,“带兵之人,一定要是智深勇沉,文经武纬之才”。邪淫会让一个人显得轻浮,这种轻浮不仅表现在言语方面,而且还体现在身体上,泄精之后脚底无根,走路也显得轻浮不稳重,相信很多戒友都深有体会,特别是在连续泄精后,两条腿就软掉了,走路发飘,在这种状态下进行体育运动,很容易就会骨折受伤。曾国藩:“古来豪杰,吾家祖父教人,以‘懦弱无刚’四字为大耻,故男儿自立必须有倔强之气。”倔强有刚强不屈之意,曾国藩讲的倔强,并不是任性和不听话,也不是蛮干,而是一种男人的血性,一种奋斗的精神。曾国藩:“立身之道,内刚外柔。”做人要刚柔并济,刚是内心坚定的原则和立场,柔是外在谦卑柔和的待人处事之道,刚有百万雄师过大江的气势,柔有水滴石穿的韧性与坚持。过刚易折,太柔显弱,所以刚柔之间要把握好分寸。曾国藩还说:“促迫褊窄,浅率浮躁,非有德之气象,只观其气象,便知涵养之浅深。”“眉宇间大有清气,志趣亦不庸鄙,将来或终有成就。”“端庄厚重是贵相,谦卑含容是贵相。”“天下古今之才人,皆以一傲字致败。”“千古有道自得之士,不外一谦字。”曾国藩的这几句话,显示了他在这方面的深刻认识,我们一方面要戒邪淫,另外就是要多学习圣贤教育,不断提升自己的涵养与精神境界,培养自己的浩然正气与威仪。

\paragraph{功名看器宇,事业看精神}

曾国藩深谙相术,他很会识人,著有《冰鉴》一书。《冰鉴》是曾国藩总结自身识人、用人心得而成的一部传世奇书,是曾国藩体察入微、洞悉人心的心法要诀。在手淫前,很多人真可谓器宇轩昂,朝气蓬勃,但是在沉迷手淫几年后,慢慢就颓败下来了,那种光明正大的气场完全消失了,取而代之的是猥琐颓废的感觉。当一个人外在的形象气质下降后,他的功名也会受到很大的影响,邪淫会严重影响一个人的运势,一个男人可以不帅,但一定要有精气神,这点至关重要。很多戒友都反馈沉迷手淫后,胆子变小了,志气也没了,肾精耗损到一定程度,就会莫名其妙地害怕,因为肾主恐,中医讲到肾水充则肝血足而胆壮,肾水虚则肝血不足而胆弱易恐。曾国藩自己也承认,“有用之岁月,半消磨于妻子”,并且认识到“多声色者,残性命以斤斧。”他深知肾精与一个人的身心健康以及气质精神的密切关系,所以他狠下决心来戒色,消磨这两个字非常准确,英雄就怕美色磨,即使婚后的性生活也是要保持节制,古代养生家特别强调婚后寡欲,如果症状缠身了,应该先禁欲一段时间,等身体恢复了再节制性生活。曾国藩特别强调“事业看精神“,而泄精最伤一个人的精神,精神萎靡了,脑力下降了,很多事情就做不好了,而且大家发现没有,泄精后人会变懒,喜欢拖延,做事缺乏长性,总是处于心浮气躁的状态,这和肾精丢失有着直接的关系。一分精神一分事业,十分精神十分事业。还有两句话叫“一分精神一分财,人不精神财不来”。曾国藩也说过“富贵看精神”,一个人如果整天无精打采,垂头丧气,眉宇间充满邪气和戾气,大家看到这种人,都会感觉不舒服。人有三宝“精、气、神”,如果你精神饱满,斗志昂扬,这种气场也会影响到别人,大家都会愿意和你交往,因为从你身上他们能感觉到一股积极向上的正能量。

\paragraph{慎独则心安}

关于慎独的问题,我在第 103 季专门讲到过,康熙皇帝曾将“慎独”概括为“暗室不欺”,并告诫子孙:“《大学》、《中庸》俱以慎独为训。”林则徐在居所悬挂一幅醒目的中堂,上书“慎独”二字,以警醒、勉励自己。曾国藩《诫子书》:“慎独则心安。自修之道,莫难于养心,养心之难,又在慎独。能慎独,则内省不疚,可以对天地质鬼神。人无一内愧之事,则天君泰然,此心常快足宽平,是人生第一自强之道,第一寻乐之方,守身之先务也。”圣贤教育极为重视慎独,因为独处时最容易犯错,很多人都是在独处时破戒的,道德修养应该达到这种境界,即在无人监督、无人知道的情况下,还是表里如一,在独处时应该高度自律,如十目所视,十手所指,不敢轻举妄动。曾国藩:“圣贤成大事者,皆从战战兢兢之心来。”独处时应该开启“橙色警报”,如履薄冰,如临深渊,不要给心魔可乘之机,不要让心魔得逞,心魔就喜欢在你独处时入侵,所以独处即实战,必须很小心才是。

\paragraph{猛火煮,慢火温}

曾国藩:“学问之道无穷,而总以有恒为主。”又说:“无恒者,见异思迁也,欲求长进难矣。”开始要猛火煮,因为刚开始的时候往往热情高涨,冲劲十足,所以宜提起全力,猛火痛烧一段,待到热情有所下降后,就要学会慢火温了。有的戒友刚开始初心猛烈,干劲冲天,然而过了三个月热情就消退了,提不起劲了,这时候应该养成良好的学习习惯,不可中断,用慢火一点点炖,用习惯战胜戒色厌倦期,只有具备恒心的人方能成功。我很强调做戒色笔记,其实我更强调复习戒色笔记,因为复习是一个深入吸收消化的过程,仅仅记一遍笔记,很难彻底吸收消化,必须几十遍乃至几百遍地复习、复习、再复习,这样才能吸收到真正的“神髓”部分。有的戒友可能会问,是不是只能在开始的时候猛火煮,这个是不一定的,猛火煮也可以是周期循环的,就像跑步一样,刚开始跑快点,然后中途跑慢些,到了冲刺阶段,又可以猛冲。莲花生大士曾经讲到:“要像燃起大火般精进。”古德曾开示“如救头燃!”大家想象一下自己的头发烧起来了,这时候你一定会拼命去扑灭,我们要用强劲的动力去戒色,就像打仗时鼓足士气一样,当士气低落了,自己要学会调整。我现在就是猛火煮和慢火温交替进行,状态好的时候就猛火煮,状态差的时候就慢火温,我发现戒色的状态和球员一样,有时候状态神勇,火力全开,有时则比较低迷,恍如梦游,自己一定要学会调整。最近看 NBA 勇士队的比赛,大家会发现勇士队也会有低迷惨败的时候,但是他们有着极其强悍的调整能力,往往在下一场就会迎来很强的反弹和爆发。有的戒友一到戒色厌倦期,他就越戒越差了,就是自己不懂得调整,一定要多激励自己,要学会养成良好的学习习惯,不要中了中断魔!中断一天,就可能中断几周乃至几个月,到时就很难找回良好的戒色状态了。曾国藩的文章多次强调恒心,欲求长进,必须要有恒心,如绳锯木断般的恒心与耐心。

\paragraph{坚忍功夫}

至于坚忍功夫,曾国藩可谓修炼到了相当高的程度。他说:“第一贵在忍辱耐烦。”“好汉打脱牙和血吞。”又提及“岳州之败,靖港之败,湖口之败,盖打脱牙之时多矣,无一次不和血吞之。”不管是考试还是打仗,曾国藩都失败过多次,但他并没有灰心丧气,反而越挫越勇,越挫越奋,堪称励志楷模。曾国藩极为崇尚坚忍实干,不仅在得意时埋头苦干,而且是在失意时也绝不气馁,他在安慰其弟曾国荃连吃两次败仗的信中说:“另起炉灶,重开世界,安知此两番之大败,非天之磨炼英雄,使弟大有长进乎?谚云:‘吃一堑,长一智。’吾生平长进,全在受挫辱之时。务须咬牙励志,费其气而长其智,切不可徒然自馁也。”曾国藩的逆商极高,逆商(Adversity Quotient,简称 AQ)全称逆境商数,一般被译为挫折商或逆境商。它是指人们面对逆境时的反应方式,即面对挫折、摆脱困境和超越困难的能力。真正的成功人士,也许智商并不是很高,就像曾国藩虽然不够聪明,但他踏实勤勉,遭遇挫折时,能够强势崛起,打开局面,冲破罗网。梁启超评曾国藩:“曾文正者,岂惟近代,盖有史以来不一二睹之大人也已;岂惟我国,抑全世界不一二睹之大人也已。然而文正固非有超群绝伦之天才,在并时诸贤杰中,称最钝拙;其所遭值事会,亦终身在拂逆之中;然乃立德、立功、立言,三不朽,所成就震古铄今而莫与京者,其一生得力在立志自拔于流俗,而困而知,而勉而行,历百千艰阻而不挫屈,不求近效,铢积寸累,受之以虚,将之以勤,植之以刚,贞之以恒,帅之以诚,勇猛精进,坚苦卓绝,如斯而已,如斯而已。”“莫与京者”,就是没有比他更大的了。梁启超对曾国藩的评价相当之高,从 30 岁前的平庸到 30 岁后的脱胎换骨,曾国藩的一生就是一个励志传奇和一部逆袭大片,他之所以能够成功,就是从圣贤教育中汲取了超强的营养与能量,曾国藩是真正懂得修心的人,而且他的修心功夫已然达到了极高的境界。曾国藩不是以智取胜,而是重剑无锋,大巧若拙,看上去很钝拙,其实实力异常雄厚,内在的修养极深。曾国藩说过一句话很经典,那就是“天道忌巧”,这里的巧是投机取巧,而不是真正的大巧,真正的大巧是抱朴守拙,守住本分。清代傅山提出:“宁拙毋巧”。韩非子则说:“巧诈不如拙诚。”老子说:“养成大拙方为巧,学到如愚始见奇。”养拙守拙是一门很高深的修养,曾国藩在这方面可谓集大成者。

\paragraph{“诚敬”里面的大奥秘、大玄机}

曾国藩:“一念不生谓之诚。”“诚”是儒家传统文化中反复强调的一个字。《中庸》:“诚者,天之道也,诚之者,人之道也。诚者,不勉而中,不思而得,从容中道,圣人也。”《中庸》里还有四个字给我印象很深刻,那就是“至诚如神”。说到诚,大家脑子里可能会跳出诸如诚实、忠诚、真诚、诚信、诚恳、坦诚、虔诚、热诚、赤诚等词语,这些词语的意思大家都很容易理解,但诚还有更深层的含义,王阳明云:“诚是心之本体。”我们的本体也就是纯粹的觉知,本体又叫本心、本觉、心性、本性、真心、法身、佛性等等,本体有无数个名字,但都是同体异名。曾国藩:“有所谓一阳初动,万物资始者,庶可谓之静极,可谓未发之中,寂然不动之体也。”静极为至静之境,心之本体即在此境之中。当你处于纯粹的觉知时,你的内心是绝对寂静的,没有任何念头,当你起念头了,念头来来去去,而纯粹的觉知则作为大背景而始终存在。《当下的力量》的作者托利还写过一本书《Stillness Speaks》(寂静之音),里面讲到:“寂静是你的本质。什么是寂静?你内在的空间或意识。”纯粹的觉知(Pure Consciousness)有时也翻作“纯意识”,它是内在无限的空间,具有觉察的能力。虚云老和尚云:“但能放下万缘,善恶都莫思,一念不生,即真心现前,此心一时现前,时时现前,永远现前,不为尘劳污染,即我是现成之佛。”曾国藩应该已经领悟到了本体,只是他没有完全说破,黄念祖老居士有一段开示:“孔子说:‘无为也,无思也,寂而不动,感而遂通。’孔子是圣人,说的就是卦,还不是人的本性?!”圣贤教育有很多一脉相承的地方,其中最核心的部分就是直指本心,就看你是否能够直下承当了。

印光大师云:“印光实有人所不得而己所独得之诀,不妨由汝之请,以普为天下之诸佛子告。其诀唯何,曰诚,曰恭敬。此语举世咸知,此道举世咸昧。印光由罪业深重,企消除罪业,以报佛恩。每寻求古德之修持懿范。由是而知诚与恭敬,实为超凡入圣了生脱死之极妙秘诀。”印光大师又开示:“有一秘诀,剀切相告:竭诚尽敬,妙妙妙妙!”所谓敬就是恭敬、敬畏,表现在内心就是不存邪念,表现在外就是持身端庄厚重而有威仪,对人很有礼貌很恭敬。古德云:“一切罪从忏悔灭,一切福从恭敬生。”曾国藩的敬字功夫十分了得,《诫子书》第二条:“主敬则身强。内而专静纯一,外而整齐严肃。敬之工夫也;出门如见大宾,使民如承大祭,敬之气象也;修己以安百姓,笃恭而天下平,敬之效验也。聪明睿智,皆由此出。庄敬日强,安肆日偷。若人无众寡,事无大小,一一恭敬,不敢怠慢。则身强之强健,又何疑乎?”其中“庄敬日强,安肆日偷”的意思为:庄重恭敬,就会一天比一天强大起来,如果肆意妄为,就会一天比一天衰败下去,“偷”这个字用得很好,就像财富被偷掉一般,所以一个人千万不能肆意妄为,特别不能犯邪淫,邪淫就是大不敬,邪淫就是在肆意妄为,完全就是在自掘坟墓。不管是诚还是敬,如果能做到极致,都会自然契入纯粹的觉知,所以印光大师把“竭诚尽敬”作为极妙秘诀来告知后学。诚敬既是为人处世的基础,也是领悟宇宙真理的一把金钥匙,我们应该像曾国藩一样,好好在诚敬这两个字上下功夫。

\paragraph{正位凝命,如鼎之镇}

曾国藩的这八个字非常给力,《易经》鼎卦云:“君子以正位凝命。”正其言,正其行,正其思,三门皆正,则成正位也。戒色修善,诸恶莫作,就是在正位,邪淫就是在邪位,正位方可凝命,凝有凝聚能量之意,反之“邪位耗命”,邪淫会导致能量的不断耗损,最后必然会导致灾难性的后果。当你正能量起来后,你会发现自己的眉宇之间会凝聚着一股强大的正气,非常纯正威严,戒到一定程度,这股正气自然会在脸上出现,你的神情和气质也会变得充满正能量,举手投足间也会表现出正气的风范。《颐》卦:“养正则吉也。”《大壮》卦:“贞吉”(坚守正道而获得吉祥。)不断培养浩然正气,即可获得吉祥、吉利,如果沉迷于邪淫,必然会招致恶报,养正则吉,狂撸为灾。坚守正道才能获得正能量的加持,坚守正道才能让自己的生活吉祥如意。只有在正位,才是吉利的,如果处在邪位,必然是凶险的,危机四伏。鼎是中国古代的一种青铜器,非常厚重,中国历史博物馆收藏的“司母戊”大方鼎重 875 公斤,可以说重量非常惊人。鼎在古代被视为立国重器,镇国之宝,是国家和权力的象征,而在我的理解里,鼎就象征着一个人的正气,当一个人的浩然正气培养起来后,就会有身如鼎镇的感觉。只有彻底告别了邪淫的生活,才能重新找回自己的正气,正气聚集即可凝聚生命的能量,让自己进入更高的境界。肾气充足之人,精力特别旺盛,而且体力充沛、头脑清晰,做事效率高而有条理,行动力极强,事业非常容易成功。人如果贪色纵欲,就会造成肾气外泄,肾气不足就会导致精神不振、脑力减退、腰膝酸软、体虚乏力、头晕耳鸣、抵抗力下降等。肾气一旦亏损,则五脏六腑、精神气血都会受到很大影响,从而致使百病丛生。曾国藩深知戒色的重要性,他说:“一国有一国之气,一家有一家之气,一身有一身之气,元气者,生气也。能养生气,则日趋于盛矣。”还说:“养生之事,莫大于惩忿窒欲。”只有在正位了,才能把自己的能量发挥到最大,只有在正位了,能量才能守得住,否则邪淫耗精,到最后身心俱废,别说干事业,就连正常生活都成了问题。我们必须戒除邪淫,凝九鼎之正气,镇八极之乾坤!

最后总结:

美国国家精神卫生研究院(National Institute of Mental Health )大脑研究和行为实验室主任麦克林(Paul MacLean)提出了“三合一脑”(Triune brain)理论。此理论根据演化阶段,分成爬虫脑(Reptilian brain)、哺乳脑和皮质脑,把前两个脑(最古老的爬虫脑和哺乳脑)合在一起,并称其为潜意识。人的爬虫脑非常古老,和所有爬虫类的大脑在本质上并无二致,爬虫脑的特点就是冲动、贪婪、自私,天生就渴望更多的交配权和更多的性放纵。人类既有爬虫脑,又兼具内在的灵性,如果你根据爬虫脑来生活,那就会堕落为禽兽,很多人在邪淫后都会感觉自己像畜生一样,做出来的事情甚至连畜生都不如。国外“爬虫脑”这个概念很好,和我们讲的心魔类似,“爬虫脑”就像安装在我们头脑中的一个木马程序,我们的目标就是战胜并清除这个木马程序,否则只能活在禽兽的层面,圣贤超越了爬虫脑,他们提升了自己的灵性。

从“禽兽”到“圣贤”,曾国藩战胜了自己的“爬虫脑”,曾国藩在很多方面值得后人学习,但也存在一定的争议,比如说他杀人太多,有“曾剃头”之称,然而战时杀人,乱世重典,有迫不得已而为之的苦衷,很多历史人物都是有功有过。曾国藩曾对儿子曾纪泽说:“我这辈子打了不少仗,打仗是件最害人的事,造孽,我曾家后世再也不要出带兵打仗的人了。”作为一个复杂的历史人物,后人对其的评价也经历了大起大落,从建国后的大肆贬低到如今渐渐回归理性客观,大家发现曾国藩在修身和做人方面的确有很高的造诣,这点是不容否认的,的确有很多值得后人学习的地方,曾国藩受到推崇也理所当然。曾国藩跟虚云老和尚是亲戚,曾国藩也是信佛之人,在云南拜过师父。曾国藩的老师很多,既有理学大师,还有佛门的师父,的确受到了高人的指点,否则也不可能达到如此的高度。

古德云:“断欲有十种利,一,身心清净,毫无所污;二,正念常存,异诸禽兽;三,气足精满,寒暑不侵;四,面目光华,举足轻便;五,俯仰天地,无惭愧色;六,省药饵费,可周贫乏;七,屏绝邪缘,胸无牵恋;八,读书作字,俱有精采;九,脾胃强健,能消饮食;十,本地风光,自有真乐。”戒色是人生一项很重要的历练,色关不过,心魔不斩,人生真的会危机四伏,曾国藩的逆袭和戒色是分不开的,戒色给了他极大的能量来奋斗自己的事业。在婚前是应该禁欲的,婚前放纵的危害实在太大了,很多人已经未婚先废了,而且随着脑力和精力的下降,学业和事业也会受到很大的影响。婚前潜龙勿用,婚后正淫则要做到节制,曾国藩在戒色修身方面给我们做出了很好的榜样,他的一句“真禽兽矣”,可谓振聋发聩,人与禽兽之间的距离真的很短,一念之差就可能沦为禽兽,曾国藩骂自己禽兽也是为了最大程度地警醒自己。他曾经说过:“财与色之地须当远避,近则有污。”贪财贪色对一个人的心灵污染是极为严重的,内心的污染多了,本有的真乐就尝不到了,只有当心灵净化后,才能再次尝到本有的真乐。学习曾国藩的文章,给了我很多启示,特别是在戒色这方面,感觉更亲切一些,在戒色方面,曾国藩是绝对坦诚的,因为他的好色,感觉他是性情中人,因为他的戒色,感觉到他内心的强大。作为戒者的曾国藩,是我最看重,也是最佩服的,最后以曾国藩的“惟正己可以化人”来作为此篇的结束,与大家共勉!

下面分享五首戒色诗歌。

\begin{poem}[刀尖上的撸者]
    \begin{multicols}{2}
        \begin{center}~\\
            他还没结婚 \\ 身体已经废了 \\ 两个睾丸 \\ 荒诞地挂着 \\ 空空如也 \\ 没有生育功能 \\ 神经也衰弱了 \\ 丑陋扭曲的嘴脸 \\ 他自己看着都恶心 \\ 曾几何时 \\ 也是纯真的少先队员 \\ 唱着《让我们荡起双桨》 \\ 那时是祖国的花朵 \\ 纯洁而富有朝气 \\ 而现在几近毁容 \\ 撸成了残花败柳 \\ 他总是关上门来 \\ 一个人默默哭泣 \\ 在岁月的深处 \\ 婴儿哭成了大人 \\ 他不再是纯洁的孩子 \\ 而是龌龊的撸者 \\ 镜子都被他摔碎了 \\ 刀尖上的撸者 \\ 随时都会被色刀生剖 \\ 再也不能这样撸 \\ 再也不能这样废 \\ 请做回纯洁的孩子
        \end{center}
    \end{multicols}
\end{poem}

\begin{poem}[报销之后]
    \begin{multicols}{2}
        \begin{center}~\\
            之前撸得很频繁 \\ 压根没想到 \\ 自己会有报销的一天 \\ 曾经身体给出过警告 \\ 但都被他忽略了 \\ 看病、吃药、住院、开刀 \\ 躺在病床上对着天花板发呆 \\ 他终于明白了一个道理 \\ 撸管是要还的 \\ 痛苦迟早会把撸者虐爆 \\ 报销之后,他不想看黄了 \\ 因为身体已经无法承受那种消耗 \\ 他已经撸不起了 \\ 但是他还是害怕自己 \\ 好了伤疤忘了痛 \\ 到时再被心魔攻破 \\ 后果不堪设想 \\ 报销之后,他开始反省人生 \\ 过去完全就是一个无知的傻逼 \\ 活生生地把自己撸进了医院 \\ 人财两空,生不如死,苟延残喘 \\ 与其苟延残喘,不如勇猛戒色 \\ 杀翻心魔贼!推翻心魔的暴政! \\ 拒绝再被心魔奴役!!!
        \end{center}
    \end{multicols}
\end{poem}

\begin{poem}[雨中的背影]
    \begin{multicols}{2}
        \begin{center}~\\
            那泡东西射掉后 \\ 还剩下些什么 \\ 更深的空虚与无聊 \\ 没意思,一点都没意思 \\ 他删掉了所有的黄片 \\ 反复拷问自己 \\ 为什么要这样糟蹋自己 \\ 他走进了滂沱的大雨 \\ 背影是那样的决绝 \\ 是该好好洗洗了 \\ 洗去邪淫的心垢 \\ 恢复纯净的心灵 \\ 他为自己这么无力 \\ 而感到愤怒和绝望 \\ 他不想再这样下去了 \\ 是该做个了断了 \\ 他在大雨中继续前行 \\ 泪水伴随雨滴 \\ 从脸庞滑落 \\ 回想邪淫的十几年 \\ 真的是太苦了 \\ 他明白既然选择自赎之路 \\ 那就应该坚持到底 \\ 即使跪着也要走完 \\ 就是爬也要爬到终点 \\ 他的脚步变得异常坚定 \\ 只留下默默远去的背影 \\ 消失在远方滂沱的大雨中
        \end{center}
    \end{multicols}
\end{poem}

\begin{poem}[心泉]
    \begin{multicols}{2}
        \begin{center}~\\
            眼睛清澈而明亮 \\ 透着一股天真的喜悦 \\ 整个身心 \\ 沉浸在纯净的愉悦里 \\ 感觉自己是世界上 \\ 最幸福、最快乐的人 \\ 这是很久违的感觉 \\ 可以追溯到撸管前 \\ 那时你是一个孩子 \\ 活在纯净的世界里 \\ 不知撸管为何物 \\ 真正的美好 \\ 来源于心灵的纯净 \\ 当你开始撸管了 \\ 就会遗失这种美好 \\ 你的内心有一眼泉水 \\ 当心灵纯净时 \\ 这眼泉水 \\ 就会自动喷涌快乐的感觉 \\ 心灵受到邪淫的污染后 \\ 就等于封住了心泉 \\ 你再也感受不到纯净的快乐 \\ 只有戒除一切邪淫的行为 \\ 才能重新开启自己的心泉 \\ 真正持久的满足与快乐 \\ 只有纯净的心灵才能给你 \\ 请恢复生命的庄严与圣洁
        \end{center}
    \end{multicols}
\end{poem}

\begin{poem}[大主宰]
    \begin{multicols}{2}
        \begin{center}~\\
            人生最高的课程 \\ 就是学会降伏心魔 \\ 学会主宰自己的内心 \\ 如果你无法主宰自己的念头 \\ 即使你身价亿万,你也是不幸的 \\ 因为你只是心魔的奴隶 \\ 活在内心的奴隶社会 \\ 被心魔驱使,做不得主 \\ 在一次次射出后 \\ 陷入更深的空虚 \\ 在一次次的堕落后 \\ 把自己推入了更黑暗的深渊 \\ 一个人只有能真正主宰自己的心 \\ 他才能感受到自由与喜悦 \\ 这是发自内心的大快乐 \\ 可以带来一整天的愉悦感受 \\ 只有当心灵恢复纯净了 \\ 你才是快乐的,否则不管撸多少次 \\ 你都是惶恐而痛苦的 \\ 因为身体被邪淫掏空了 \\ 症状迟早会爆发出来 \\ 短暂的快感换来的是持久的苦报 \\ 努力学习戒色文章吧 \\ 学会降伏自己的心魔 \\ 降伏心魔的人 \\ 才是最强的存在
        \end{center}
    \end{multicols}
\end{poem}

下面推荐一本书。

\begin{book}[《这个世界的真相》,阿姜查]
    作为风靡全球的伟大禅师,阿姜查被称为我们时代的伟大智者,他在世界各地受到广泛的欢迎,他用最通俗的语言对困扰我们的最深奥的问题进行了解答。我们活着为了什么?生活的真正归宿是什么?怎样才能快乐地生活?这些近乎终极的问题,在《这个世界的真相》中都能找到答案。他的书超越了时间与空间的距离,能以一种格外简单的方式自然地融入我们的心灵。与生活在都市中奔波、劳碌的人们能轻易地发生对接与碰撞。他的书,是献给都市心灵的田园牧歌,是进入疲惫心灵的一捧甘泉。阿姜查是著名的泰国高僧,1948 年,他在森林中与 20 世纪伟大的森林禅师阿姜曼相遇,获得重要启发,改变了他的修行方法。1954 年回到家乡吴汶省巴蓬森林,追随者日多,于是有了著名的巴蓬寺。这本书我是在网上下载的 PDF 版本,我记了 103 条笔记,阿姜查教导人们观察自己的心,这与其他大师的核心教导并无不同,只不过阿姜查用自己的语言风格进行了简明而深远的表述,给人很大的启示。比如:\begin{itemize}
        \item “智慧的唯一来源是去观察自己的心。”
        \item “去除心中之贼。”
        \item “心就像一只猴子无意识地到处乱跳,学习成为心的主人。”
        \item “有一天你会发现,观察自己的心才是修行的根本。”
        \item “如果你能保持觉知,当烦恼生起时,它会消失。”
    \end{itemize} 阿姜查说:“我只是一个观察者,一个觉醒的人。”这句话说得非常之好,这本书推荐给大家阅读,希望大家能从中获益。
\end{book}
