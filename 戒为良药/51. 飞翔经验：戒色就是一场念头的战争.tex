\subsection{戒色就是一场念头的战争}

\paragraph*{前言}

如果你向一个普通人说戒色,他肯定会觉得你在走极端,这是普通人的误解。戒色吧的主旨是婚前禁欲,婚后节制。尽量避免婚前性行为,把最好的自己留到结婚后,为优生优育创造良好的身体条件。不少人闯入戒色吧,没有深入的了解和认识,他一看到这么多人在戒色,就马上说,戒什么色,我又不是要做和尚。他以为戒色吧的人都想做和尚,这是莫大的误解,也是自以为是的表现,没有深入的了解和认识,就妄下结论。他还以为手淫很正常很对,他的想法就是人有欲望为什么要压抑啊,有欲望就该撸嘛。有的人还搬出“食色性也”,古人说“食色性也”并不是叫你纵欲,古人是反对邪淫的,手淫就属于邪淫的范畴,“食色性也”更多的是传宗接代的需求,并不是叫你邪淫,很多人都误解了食色性也,把食色性也当作自己放纵的借口,实在太无知了。手淫消耗的正是人体最宝贵的肾精,医院迟早会和撸者“算总账”的,逃不掉的!精少则病是绝对的真理,经过了几千年的验证。以身试法,只会导致悲剧的结果。

这是过来人的忠告,当然某些“温水青蛙”很难听进去,他觉得手淫很舒服,为什么要戒啊,根本就听不进你的劝告,简直是对牛弹琴。对于这种人,只能让症状教训他了,症状才是最好的老师,当然有些无善根之人,即使症状出来了,也不思戒色,一错到底,到死也无法觉悟。执迷不悟,迷得太深,撞了南墙也不肯回头,可悲啊!还有些人自感手淫十几年,身体还行,除了一些可以忍受的身体不适,并无其他大问题。殊不知,大问题正是从小问题积累而来,生活中大家应该都听到过,有的人要么不生病,要么就生大病,有的人平时看看还好,但一检查却已经患上了严重的疾病。还有的人表面看着挺好,但突然有一天就猝死了,央视《新闻 1 + 1》就关注过我国中青年猝死比例不断上升的现状。有数据显示,我国每年死于心脏性猝死的人数多达 55 万。也就是说,每天至少一千多人猝死。

3 月 10 日是世界肾脏日,有一年的主题是“保护肾脏,挽救心脏”。肾病专家几年前注意到一个奇怪的现象,许多肾病患者在病情远未发展到肾衰竭之前,已死于心脏疾病。但是与心脏相比,人们对肾脏往往不太重视。最新研究发现,心脏病与肾脏好比一对“难兄难弟”,为什么这么说?因为肾病的发生会加速心脏病恶化,而心脏隐患很可能以肾病的形式表现出来,即肾病与心脏病关系密切。这是西医的研究结果。而我国中医早就阐述过肾虚是会导致心脏问题的,心肾相交这种动态的平衡一旦失调就会出现心肾失交。根据“心肾相关”理论,肾虚日久必将累及于心脏,使心脏受累。心脏受累又可以反过来加重肾脏疾病,以致最后出现心肾同病的病理状态。中医:肾为五脏之根。撸管伤肾后,五脏都可能会出现问题。如果你喜欢熬夜,或者过度劳累,又或者体重超标,那弄不好真有猝死的可能,这并非危言耸听,一定要提高警惕。我聊过的戒友中,不少人都有心脏不适,我曾经也出现过心脏疼痛心悸的表现,后来戒掉半年以上,自己就慢慢好了。

有时候,人的自我感觉是具有欺骗性的,你以为你还行,但其实正在江河日下,正在滑向出症状的临界点。所以,我们千万不要心存侥幸,彻底戒撸才是明智的选择。清除你大脑里的无害论,这是最关键的第一步。必须深刻认识到撸管的危害,必须纠正戒色的思想误区。还有的人一天撸管几次,自感性欲旺盛身体强,殊不知是虚则亢的表现,现在如此亢奋,其实为将来的早泄阳痿埋下了伏笔。就像股市的泡沫经济一样,完全是虚假繁荣,虚假繁荣之后是什么呢?告诉你!是暴跌,是经济危机。而虚则亢之后,就是身体出现危机,症状开始找你了。

\begin{quote}\it
    上士闻道,勤而行之;中士闻道,若存若亡;下士闻道,大笑之,不笑不足以为道。(《道德经》)
\end{quote}

现实情况是,下士不仅会大笑你,还可能会骂你,下士往往也是无素质之人,开口闭口都带脏字,很自以为是。这样的人与戒色吧是无缘的,毕竟悲心再强烈,也无法度无缘之人。在戒色吧,你会非常认同一句话,那就是:真理掌握在少数人手里,只有少数的人能彻底明白撸管的真相。中国十几亿人,而戒色吧目前只有十几万,即使发展到几百万人,相对于整个群体,还是属于少数。除非媒体能开始宣传有害论,戒色思想能得到普及,撸管有害能得到广泛的认同,撸管有害能进入学校的生理卫生手册。现在的中国还是无害论的天下,希望撸管有害早日成为全民共识,就像香烟盒上所写的一样:吸烟有害健康!

还有的戒友和父母说戒色,父母一听戒色,大多会产生误解,害怕儿子想做和尚了,那传宗接代怎么办啊,这是父母合理的担心。如果你说只是戒撸而已,因为上一代人在性教育方面也是缺失的,或者是被误导的,那也很可能无法给你正确的引导,有的父亲则把撸管当做正常的习惯,并没有认识到撸管的深刻危害,所以他对儿子的教育也是这样,并不能给出正确的引导。所以,戒色这件事最好不要和父母讲,特别是初中和高中的戒友,你们的思想还不是很成熟,在家长眼里还是孩子,说出来的话没有很强的说服力。本想获得家长的支持,没想到说出来反而被家长所误解,那就会影响到你戒撸了。戒撸其实还是一件相对隐私的事情,特别是在无害论泛滥的年代里,说出来很容易遭到误解。我们默默坚持戒撸即可,戒色吧有那么多志同道合的戒友,他们可以给你鼓励,也可以给你足够的支持。

下面分享几个答疑案例:

\begin{case}
    飞翔老师,我是老油条了,很多戒色知识都看过无数遍了,但是感觉对戒色知识都麻木了,自从今年三月中旬戒了 109 天破戒之后,一直反复破戒到现在,期间没有超过二十天的,我知道我真的没脸再来问问题了,但是我真的是不知道该怎么办了,我感觉现在的自己连一年前开始戒色时的自己都不如,我看过无数的伤身案例,自己身体也有很多症状,但是我居然对自己都漠不关心了,我不知道该怎么办了!就比如今天,戒的第五天,但是欲望真的是突然莫名其妙的很重,我发现它真的是客观存在的,而且我这几个月开始慢慢看了好多部 H 片,而且学会了好多种找 H 片的方法,我真的是想忘都忘不掉啊!每当欲望来临的时候,就无法克制自己去找 H 片,而且现在路上的诱惑太严重了。总之,太难了!

    \textbf{答} 嗯,你这种经历很多人都有过的。就是第一次失败后,然后就完全找不到戒色感觉了,起了退心,对戒色文章也麻木了,失去戒色信心与动力,又被心魔劫持了,又重新做回了心魔的傀儡。出现这种情况不要害怕,好好忏悔下,忏悔的力量可以救你,重新下大决心,重新找回戒色状态。学习一定要养成习惯,很多戒友看戒色文章浮于表面,只是走马观花,没有深入的理解和认识,没有融入自己的戒色意识里,这样觉悟提高就很难了。最好有自己的戒色笔记本,每天看些文章做些戒色笔记,然后多复习,坚持一段时间,真正学进去悟进去,这样觉悟就能扎实地提高。加油!不要灰心。好好坚持,很多戒友都经历过破戒,从破戒中总结经验教训,重新东山再起,你行的。

    \textbf{附评} 要说到戒色老油条,其实每个人都经历过这个过程的,我那时强戒很多年,可以说是老油条中的老油条了,从初中开始,戒过无数次,没一次成功的,最多戒过 28 天,然后就是一连串的破戒,直到身体再次发出严重警告,才开始新一轮的戒色。那时我反复尝试,就是无法战胜心魔,当然我那时完全是不学习,就是靠强戒,结果必然是失败。后来我通过不断学习,觉悟像坐电梯一样上去后,再次面对心魔,已经今非昔比了,我已经可以完全降伏心魔了,心魔出的考题,我可以全部答对。我曾经把戒色比作练级,心魔老怪一百级,你只有十几级,实力这样悬殊,只有被心魔虐的份,心魔一出来,马上就投降了。只有通过不断学习提高觉悟,才能最终降伏心魔。

    有些戒友的确在学习,但是他们无法持续提高觉悟,学习一段时间,就遇见了觉悟提高的瓶颈期,还有就是戒色厌倦期,开始对戒色文章麻木,不想做笔记,对于戒色文章只是随便看看,看过了就全部忘记了,他们的觉悟一开始是在提高,比如一开始只有十几级,通过学习达到了五十级,但和心魔老怪的一百级来比较,还差很远。想战胜心魔老怪,必须不断学习提高觉悟,不能放松对自己的要求。这位戒友能戒 109 天,已经实属不易,但是后来他出现了戒色厌倦期,我前面文章专门讲到过戒色厌倦期的心理调整,很多戒友都会在心生厌倦时出现破戒,戒色厌倦期几乎每个戒友都会经历的,我也经历过,当我出现戒色厌倦期了,我会调整学习量,但不会中断学习,我有很多戒色笔记本,我的觉悟是一条条做笔记得来的,还有就是通过自己的思考和领悟得来的,在做笔记的同时,其实已经在吸收、思考和领悟了。所以,我特别强调做笔记的重要性,每个戒友都应该有自己的戒色笔记本。一定要时常做笔记,时常复习笔记的内容。养成良好的学习习惯,即可克服戒色厌倦期。

    很多戒友在第一次失败后,就会出现一连串的破戒,一鼓作气,再而衰,三而竭!完全没有当初的戒色动力和热情了,破戒的确会掉经验,也会掉士气。但是强者,会通过破戒来提高自己,因为破戒可以让其看到自己觉悟存在的缺陷,看到自己认识上存在的不足,正是认识上存在误区和盲点,才会导致遭遇心魔时被心魔打败。心魔就是一位严格的考官,你只要答错一题,结局就是破戒。

    大家都是学生党过来的,一个班级几十位同学,都在学一本语文课本,然后期末考试,就见分晓了。到底是真懂,还是一知半解的懂,或者只是浅懂,而不是深刻理解和认识。戒色也是如此,心魔一考验,就能看出你觉悟的高低了,觉悟高者即可过关,觉悟存在缺陷者则极易破戒。真金不怕火炼,当你的觉悟货真价实了,就不怕心魔的考验了。很多戒友被心魔打败后,又重新做回了心魔的傀儡,又重新变成了心魔的提线木偶,完全受心魔摆布,不得自由。那种被心魔劫持的日子真的很悲哀!也很无奈!

    要找回良好的戒色状态,一,就是要等到再次被症状吓醒,出于恐惧再次开始戒色,这比较被动。二,就是要多发忏悔之心,真心忏悔的力量是很大的,然后要发大决心去戒色。真正落实学习,扎实提高觉悟,这样就能重新找回良好的戒色状态。失败不可怕,可怕的是放弃学习和厌倦学习,就像龟兔赛跑一样,兔子中途完全放弃了,而乌龟很聪明,虽然缓慢但是始终保持前进的势头,到最后,即可到达终点。我们应该学习乌龟,让觉悟持续提高,提高慢点没事,关键是持之以恒,恒则成。

    戒色厌倦期,几乎每个戒友都会经历,戒色厌倦期也会刷下去一批人。能克服戒色厌倦期的人,可以进入更稳定的戒色状态。被刷下去的戒友,也不必灰心,谁没破戒过?完全可以东山再起!不用太自责,只要把学习捡起来,重新找到良好的学习状态,扎实提高觉悟,彻底戒掉就不再遥远。如果放弃学习或者厌倦学习,那就很难了。

    看戒色文章,只看一遍,没兴趣看第二遍的人,很难戒掉。因为只看一遍,其实看得还很浅,就像走马观花一样,看过就忘,似乎明白了,但一句也说不出来。孔子读书就主张温故而知新,好的戒色文章值得反复学习,要不断做笔记,不断复习笔记的内容,提高吸收率。好的戒色文章,每次看都会有新的领悟,新的收获。看一遍的戒友,和看二遍的戒友,他们的觉悟是有差别的,看二遍的戒友和看五遍的戒友,他们的觉悟又会有所不同。做笔记的戒友比走马观花的戒友,觉悟提升得更快。我的建议就是,一,要多看几遍,乃至百看不厌。二,要多做笔记,多复习。真正落实了这两点,你的觉悟增长会很快。而没兴趣看第二遍的戒友,一定要改正走马观花的阅读习惯,对于好的戒色文章,要反复看,多做笔记,进行更深入更细致的吸收和消化,真正把学到的内容转化成自己的戒色意识。

    孔子韦编三绝的故事,很值得大家学习。春秋时期孔子十分好学,晚年还坚持研究《易经》,他反复钻研该书,把该书的捆竹简的牛皮带都磨断了三次,终于把研究的心得写成十篇文章,即《十翼》。后人把《十翼》与《易经》附在一起,作为《易经》的补充部分。

    反复看,反复研究,多做笔记,自己不断思考和领悟,这样觉悟提升就会很快。只看一遍,太浅了!希望大家能够深入戒色文章,真正深刻理解,并且吸收转化成自己的戒色意识,这样坚持下去,离彻底戒撸就不再遥远了,能够这样坚持三个月,觉悟就可以扎扎实实地提高一大截,马上就可以和别人拉开差距,再次面对心魔,就有把握战胜心魔了,至少可以比过去强很多,当你不断强化自己的觉悟,终有一天可以彻底降伏心魔。学习提高觉悟,觉悟降伏心魔,这就是戒色成功的铁律。
\end{case}

\begin{case}
    持续戒了很久了,但我 SY 是和在网吧上网一起开始的,有时就想可能不是 SY 的问题,而是去上网的问题,老是把问题往长时间上网这上面推,认为和 SY 没关系,很苦恼!但还是坚守住,可是经常为了这事失眠,一直在想 ,而且可能自己开始 SY 的时间不长,戒了差不多两年了,可能属于那种症状还没出来就开始戒的,没有真正体会过症状的感受。综上,就是感觉受到了无害论的动摇,老是把问题往长时间上网推,而且认为是久坐导致的。现在天天失眠,很痛苦,希望得到帮助。

    \textbf{答} 嗯,上网也伤身体的,比如久坐伤肾伤脾,久视伤肝,但纵欲也伤肾,而且非常厉害,这是中医常识,建议学一下中医养生理论,为人子弟不可不知医,否则就会盲目乱来,这也是无知的可怕。加油!

    \textbf{附评} 症状还没出来就开始戒,当然下决心就不会很大,因为人都是逼出来的,大多数戒友都是症状出来才开始戒的。当然,真正有善根之人,不需要很严重的症状也是可以戒色成功的,这也是有先例的,他们的想法就是,不想等到被症状虐惨了再戒,因为那样会很被动,恢复难度也更大,这是比较明智的想法。有的戒友可能会觉得,症状越重,戒色成功的可能性就越大,这种想法是不成立的,因为我聊过不少戒友,其中不乏动过大手术的戒友,居然在手术后没多久又开始撸管了,虽然知道是撸管导致的大病,但就是无法控制自己。人有一个弱点,那就是好了伤疤忘了疼,另外,如果还是不学习的强戒,那要戒色成功真的是太难太难了,强扭的瓜不甜,强戒的下场注定失败。所以我们一定要走专业戒色的道路,多学习提高觉悟,这样戒色成功的可能性就比较大了。走专业戒色的道路,不需要伤得很深,也可以戒色成功。如果你强戒和盲戒,即使伤得再深,也难以戒色成功。大家请记住这一点。
\end{case}

\begin{case}
    我手淫十几年,未婚,现在 25 岁。最近一次撸管好像是在今年二月底,某一天出现了一种莫名其妙的恐惧念头,大概是恐惧症,对自己的体味,比如说别人和自己的口气、汗味特别敏感,一靠近人就有一种莫名其妙难以抑制的恐惧感。请问这是戒断的正常反应吗?我自认为自己戒色不是特别彻底,自己仅仅是戒撸,但是脑袋里面经常出现淫念(每次出现淫念我都会去做一些其他的事转移注意力)。虽然单位的人都对我很友好,自己也有一份稳定体面地工作,可是我一点都不快乐,尤其站在别人面前就会恐惧,这种滋味确实有点难受,请飞翔大哥解惑,不胜感激之至!

    \textbf{答} 中医:肾主恐。肾虚到一定程度,即可出现恐惧症的倾向,以社恐为常见,很多戒友伤到后来,就莫名其妙恐惧了。当然伤精方式很多,撸管是一方面,还有就是和女友纵欲、熬夜久坐、暴怒等,所以一定要学会养生之道,加上彻底戒撸,恐惧症会慢慢改善的。加油!

    \textbf{附评} 撸管导致的生理症状很多,导致的心理症状也很多。撸管是对身心的双重摧残,撸管的人很容易出现自卑、恐惧、焦虑、偏执、强迫、急躁易怒等心理表现,一句话,撸管会导致心理环境的不健康和失调。不少戒友都反映过,撸管后脾气就变坏了,容易和人吵架,特别是和父母吵架,沉不住气。这是伤精患者的典型表现之一,就是人会变得急躁易怒。出现自卑的戒友,那就太多了,太普遍了,一抓一大把,撸管变丑也更容易加重自卑心理。恐惧症也比较常见,肾气不足,人就会莫名其妙地恐惧,这种恐惧一般正常人还理解不了,其实就是伤精导致的恶果。在西医上会归为心理疾病,其实心理疾病和生理是有密切关系的,肾主恐,伤肾到一定程度,即可出现恐惧的心理状态。当然,某些人先天就胆小,再加上频繁撸管,那就更容易出现恐惧症了。这个案例还透露一条信息,那就是一定要彻底戒撸,何谓彻底?就是连意淫的念头也要克服,有的人虽然戒撸了,但是他的意淫还是很多,这样就是暗漏,久而久之也会导致症状爆发的,沉迷意淫会导致伤身,而念起即断就没有任何伤害,关键是要断得快!
\end{case}

下面步入正题。这季就“戒色就是一场念头的战争”这一主题和大家做一个详尽的分享,从念头这个角度来诠释戒色,相信大家会把戒色看得更清楚,也更到位。

戒色,如果说到底,其实就是要学会观察自己的念头,然后控制自己的念头。

戒色就是两种念头在打仗,一方面是想戒色的念头,另外一方则是邪念,也就是想破戒想放纵的念头。很多戒友在破戒前都很挣扎,很矛盾,因为两种念头在他头脑里拔河,两种念头在他头脑里打仗,就看哪种念头占据上风,很多戒友最后屈服于了邪念,从而导致破戒。还有的戒友,头脑里面根本没打仗,邪念一上来,马上就投降了,马上就开撸了,根本就没抵抗的过程。就像打仗时没有守城的士兵,马上就放敌人进来了,马上就沦陷了!我们学习戒色文章,就是在不断加固我们的城池,让邪念攻不进来,一旦发现邪念在爬城墙,马上警觉到,就把它灭掉。

\begin{multicols}{2}
    \begin{description}
        \item[敌方] 心魔(意淫、怂恿的念头、想试的念头等)
        \item[我方] 正念(通过学习提高觉悟而产生的戒色意识)
    \end{description}
\end{multicols}

心魔是敌方的主帅,邪念就是心魔发出的,以意淫为主,还有就是怂恿类的念头也要格外警惕。佛教讲“降伏其心”,其实就是要降伏邪念,降伏心魔。当然佛教要降伏的范围更大,贪嗔痴慢疑都要降伏,而我们要降伏的就是贪色的念头。不过,戒到更高的层次,其实贪嗔痴慢疑都是要降伏的,因为戒色也是要注重修德和修道的,这样才能达到更高的境界。

有的戒友破戒后,掉士气掉热情,就像打败仗一样,感觉很气馁,有的干脆做了心魔的俘虏,连续破戒,越戒越差。搞得自己也很苦恼,其实要战胜心魔,必须通过不断学习提高觉悟,用戒色知识武装我们的头脑,这次面对心魔失败了,不用害怕,加强学习,多总结经验教训,下次继续挑战心魔老怪。只要你持续进步,总有战胜心魔的那一天,要越挫越勇,不能被暂时的失败所吓倒,似乎感觉心魔不可战胜,心魔的确很强,心魔也很会钻空子,但并非不可战胜。心魔强,你要比心魔更强,才能降得住它!

\textit{修道就是修心,修心修什么?就是修念头!(大安法师)} 我们戒色也是在修心,也是在修念头。很多戒友对于修心搞不明白,不知什么是修心,其实“心”就是你的念头,就是心念。发育前,你每天的念头里没有黄毒,比较纯净,就像纯净的山泉,没有受到严重的污染。但是发育后,开始接触黄色的内容或者被邪友带坏,心相续就会变污浊了,我们的念头连接起来就像一条河流,如果一直想黄色的内容,就会污染这条河流,本来纯净的河流就会变得污浊发臭,而相由心生,心相续污浊了,也会反映到身体上,也会反映到脸上的,因为身心是合一的。经常撸管的人,会有一张撸管脸,猥琐自卑没有精气神,甚至会出现脸部变形,身体也会出现各种不适,暗疾缠身苦不堪言。

因为心变污浊了,人的身体也容易发臭,身心是对应的!我搜集的案例中,不少戒友的身体都变臭了,狐臭、汗臭、脚臭、口臭、鼻臭比较多见。

\begin{case}[发臭]
    我有四五年过度手淫史了。几乎一天一次,现在腰疼,浑身无力,总感觉累使不上劲,早先没狐臭,现在出点汗有狐臭味。消化不好,吃点东西就感觉涨涨的。勃起快射精快。请问是不是早泄或者肾虚。是肾虚的话是阳虚还是阴虚。另外检查精子是无精子状态请问是怎么引起的?
\end{case}

\begin{case}[发臭]
    我从十三岁开始手淫!已经两年了,频繁时每天都有,或者两天一次,自从手淫一年后,我出现了不适应症状。先是感到腰酸,慢慢地,时间长了,开始一跑步就腰疼,挺严重的。之后毛病越来越多,有焦汗症。还有狐臭。可是我没有家族狐臭史啊,怎么会有狐臭?估计和手淫有关。
\end{case}

\begin{case}[发臭]
    腰酸背痛脑子一片空白,做点事出一身汗又有狐臭口臭脚臭,为什么上天把这些不幸全降临在身上,前几天已经有半个月没手淫了,前天晚上又破了,现在又染上了股癣。我他妈的是不是上辈子做了什么孽了,今身又要还,戒个手淫都戒不了,我这辈子完蛋了!也请大家不要在手淫了,不要跟我一样。
\end{case}

手淫导致内分泌紊乱,导致五脏功能紊乱,原先没狐臭的人也会出现狐臭,手淫是导致狐臭的一个潜在的因素。我之前也有汗臭,然后脚臭很严重,现在彻底戒撸后,一点都不臭了,即使出汗也没臭味,有一句话叫戒定真香,坚持戒色养生,控制遗精频率,身体是会恢复清香的,那是一种很微妙的感觉,就像小孩的体香一样,那正是灵魂的清香。

我们学习戒色文章,其实就在加强正念的力量,心魔的力量实在很强大,特别是黄毒泛滥的年代里,心魔虽然很强大,就像一百级的老怪,但并非不可战胜,只要不断练级(练觉悟等级),终有一天是可以战胜心魔的,这和网游的道理是完全相通的。不断练级,不断长觉悟,从开始的菜鸟,到和心魔平起平坐,再到最后彻底降伏心魔,基本就是这样一个过程。如果你放弃学习或者厌倦学习,那你的觉悟等级是无法和心魔抗衡的,是无法抵御心魔的攻击的。戒色就是练级,练觉悟等级,每天学习戒色文章,多做笔记多复习,就是在练觉悟等级。还是那句话,学习提高觉悟,觉悟降伏心魔!戒色就是一场念头的战争,不断加强正念,让正念坚固,让正念无坚不摧,就可以彻底降伏邪念,就可以彻底降伏心魔。心魔像弹簧,你弱他就强,你强他就弱!你必须通过学习让自己强大起来,没有别的选择!

\begin{quote}\it
    念佛的作用就是斩断妄念,佛的圣号就是一柄慧剑。(元音老人)
\end{quote}

念佛其实就是在斩断妄念,以一念代万念,手中的武器就是一句佛号。就像一把锋利无比的宝剑一样,妄念一起,马上发觉,马上就斩断,即做到“念起即断”。其实“念起不随”就是断!就是不理睬妄念,不跟随妄念,很多人一开始是做不到的,要不断练习观心,增加对内心的觉察,这样才能慢慢做到。念起即断,手起刀落,断意淫必须干脆利落!断意淫就是要快和狠!可以执持佛号斩断妄念!也可以用断意淫口诀“\textbf{念起即断、念起不随、念起即觉、觉之即无}”,能用好这个口诀,也是可以降伏意淫的。

{\it 当你觉悟很低时,面对心魔,就像小孩在和大人拔河。一拔就输!

当你觉悟加强后,面对心魔,就像青少年和大人拔河,可以对抗下,但还是输!

当你觉悟更强后,面对心魔,就像大人和大人拔河,可以对抗更久,但还可能输!

当你觉悟变成大力士后,面对心魔,不再害怕,相反,心魔会开始怕你!你可以轻松战胜心魔。

变成大力士后,是不能放松警惕的,一放松警惕,心魔还是可能会战胜你的!}

戒色就是一场念头的战争,戒色也是一场拔河,这两种比喻都很形象,你必须加强正念的力量,这样才能降伏心里的邪念。

\paragraph{审查每一个念头}

禅宗有“看念”的功夫,也可以说是“观心”,因为“心”就是你的念头!就是看自己的念头,观察自己每一个念头,反观自心。最后戒色成功的人,都是念头审查者,对于心里生起的每一个念头,都能及时察觉到,对于不好的念头可以马上进行干预,使之消灭。这种“看念”的功夫对于戒色是异常重要的,很多戒友不会看自己的念头,也不懂得看自己的念头,导致无法掌控自己的念头,被不良的念头牵着跑,自己做不了主。比如,头脑里出现一个意淫的念头或者想看 H 的念头,你能第一时间“看”到,第一时间觉察到,那就可以马上灭掉它。如果你不灭掉它,这个不良念头就会控制你的行为。念头控制行为,脑控制手,如果你不审查坏的念头,那就会导致不良的行为。所以,一定要“看清”自己的每一个念头,对于不良念头要马上觉察到,要引起高度警觉,立刻消灭之,不能让其得势,一旦念头得势,就会牢牢控制你的行为。切记!

\paragraph*{总结}

戒色是向内的战争,是念头之间的战争,是对自己心魔的宣战!战胜自己其实比战胜别人更伟大。\textit{战胜自己的人比战胜一座城池的人更有勇气。(《塔木德》)} \textit{胜人者有力,自胜者强。(《道德经》)} 真正的强者是战胜自己的人,你必须战胜自己的邪念,而不是被邪念所控制,被邪念所控制,你就会过着一种衣冠禽兽的生活,甚至连畜生都不如。畜生有发情期限制,而人随时随地都可以搞,最后把自己的身体搞垮,而且邪淫也会把自己的福报消掉,最后真的是害惨了自己!所以,我们一定要战胜自己,做一个真正的强者!加油!
