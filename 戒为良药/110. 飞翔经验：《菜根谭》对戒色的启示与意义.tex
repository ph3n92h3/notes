\subsection{《菜根谭》对戒色的启示与意义}

前言:

关于戒断反应和症状反复期的区分,戒断反应一般在一个月内,刚开始戒色有可能会出现戒断反应,坚持戒下去,慢慢就会缓解消失。如果你戒了两个多月或者半年、一年出现的症状,那就考虑为症状反复期。大家要学会分辨戒断反应和症状反复期,我看到很多人答疑,一律都说是戒断反应,这让戒了几个月的人很难信服。恢复的过程不是一帆风顺的,肯定会经历多次的症状反复,就像刚学走路的孩子,会有一个跌跌撞撞的过程,到后来才会逐步稳定。导致症状反复的原因比较多,比如最近天气转凉,很多人开始出现慢前的症状反复,这时候不要慌张,要注意休养,慢慢症状又会缓解消失,自己一定要加强养生之道。其他导致症状反复的原因还有劳累、遗精、熬夜久坐、吃冷辣、喝酒、生气等等,如果是比较严重的症状反复,那可以考虑就医治疗,如果比较轻微,一般休养几天症状即可缓解。我那时也经历过多次的症状反复,包括慢前的症状反复、耳鸣的症状反复、神经症的症状反复等,出现症状反复时,我一点也不担心,我知道痊愈的过程肯定会经历症状反复,只要加强养生,症状反复就会越来越少,最后就会进入较为稳定的痊愈层次。

我在 105 季的一个案例附评里讲到:“随着年纪的上升,身体的恢复速度会大大减慢,年轻的时候撸一次,也许七天即可恢复,但是如果是频繁撸,而且伤得越来越深,那就不是七天了,可能七个月都恢复不过来了。”有一位戒友看到七天可以恢复,于是就以为可以搞适度,这完全是误解,能恢复不代表就能搞适度,我一直在反对适度无害论,因为 SY 极易成瘾,很容易一发不可收拾,适度无害论完全就是伪命题。只要撸就有害,就像上网需要流量一样,SY 和沉迷 YY 都会耗损宝贵的肾精流量。所谓七天恢复是来自于中医的七日来复,七天是气血运行的一个周期,SY 后身体是会慢慢恢复的,就像皮肤上的伤口,过一段时间它就会自动愈合一样,身体有自动恢复功能。但是 SY 具有高度成瘾性,随着伤害的加深,恢复速度就会减慢,特别是症状爆发后,那恢复速度就更慢了。恢复如滴水,而下面在狂漏,这样身体怎能吃得消?之前有的人按照一周一撸,最后也症状缠身了,YY 和看黄都属于暗漏,就像在偷你肾精流量一样,你以为自己是在一周一次,其实你不知不觉间漏掉了太多太多,而且破戒时间也大有讲究,在有的时段身体处于虚弱状态,而你撸了,结果伤害加倍。伤得严重了,可能一次泄精几个月都缓不过来,最后就会陷入恶性循环,积重难返。手淫恶习必须彻底戒掉,不要幻想搞适度,不要心存侥幸,前辈的立场绝对坚定而明确,那就是:彻底戒撸!所谓节制是对婚后讲的,婚前应该做到彻底戒色,把最好的自己留到结婚后,在婚前降伏心魔,获得高度的自控力,这样在婚后才有望做到节制。不少人婚前疯狂放纵,身体垮了,未婚先废,家里还催着结婚,陷入了进退两难的境地。婚前戒色就是一项历练,也是一种能量管理,很多人都图一时之快,最后把自己撸废了,到时真的追悔莫及。

有的无知的新人可能会觉得戒色可以解决一切问题,甚至可以包治百病,这种想法就是“戒色万能论”,这种迷信戒色的倾向是完全错误的。戒色是有很多好处,但即使戒色了,生活中依然可能遭遇各种挫折,也可能因为其他原因而染上疾病,这都是有可能的,因为致病因素是很多的,不止手淫一个。生活也不可能总是一帆风顺,肯定会有一些挫折和不顺心的事情,如果你处于邪淫颓废的状态,在面对生活的挫折和打击时,也许你就会变得消沉乃至一蹶不振,如果你是一个戒者,身心、脑力、精力都处于一个比较好的状态,这样即使挫折来了,你也可以坦然面对,很快就能克服,戒色可以让你保持在一个积极向上的状态,对于克服困难是很有帮助的。对于戒色的好处要客观公正,不要盲目夸大戒色的作用,前辈从来没说过戒色可以包治百病,该治疗还是要积极治疗,戒色养生只是有助于身体的恢复,是有不少症状在戒色后逐步缓解消失,但这也不能说包治百病,只能说有助于身体恢复。症状严重的话,还是应该积极治疗的,不要延误病情、耽搁治疗。戒色也不可能解决一切问题,因为即使你戒色了,生活中的各种问题还是会冒出来,戒色只是给你一个良好的状态去应对这些问题,从而圆满妥善地处理好这些问题。

下面分享一些案例。

\begin{case}
    戒色 177 天,此时的心情真的太激动了!真诚感恩戒色吧!从没体验过这样的生活,发自内心的快乐,在这戒色的几个月中我懂得了很多,幸福的根本其实是心。想起以前的色情生活真的太痛苦了,几秒钟的快感换来了长久的痛苦,每天就像活在地狱,没有目标,不知道自己要干什么,无所事事,还整天骂父母,甚至打父母,邪淫让我脾气暴躁,畜生不如!是戒色吧拯救了我,如果不是戒色吧,没有吧友们的帮助,没有前辈们的教导,我恐怕早已废了。是戒色吧让我明白什么叫做人,是戒色吧给了我一个崭新的自己,给了我一个无比大的希望!在此非常感谢你们!也祝吧友们早日戒除邪淫,回归美好生活。
    \subparagraph{附评} 撸者追求的是快感,但他们并不是“发自内心的快乐”,古德虽然也说“淫乐”,但其实淫乐是快感而不是真正的大快乐,两者还是有很大区别的。真正的大快乐根本不需要外在的条件,只需恢复纯净的心灵,这样即可再次体验到纯净的大快乐,这种大快乐是“心泉”自然涌现出来的,而邪淫就像一块厚重的窨井盖,会封住你的心泉,让你无法再体验到纯净的大快乐。久而久之,你就会遗忘纯净的大快乐,转而追求短暂的快感。快感和痛苦是绑定的,追求快感迟早会感受到痛苦!不仅痛苦,还有深深的空虚、深深的惶恐乃至深深的绝望。这位戒友的体悟很好——“幸福的根本其实是心”,看黄手淫只能给你一时的快感,爽一时,苦一世,小快感,大报应!沉迷手淫,你真的是亏大了,把血本都亏进去了。愚痴者是很难认识到这一点的,他们还以为自己爽到了,其实他们已经开启了定时炸弹,症状迟早会大爆发!也许这几年感觉还行,但是正在滑向那个临界点,而很多温水青蛙还浑然不知,直到症状爆发后才知道手淫是要还的,欠下的邪淫巨债,最终自食恶果,到时人生就会陷入很深的困境。很多人都是撸到早泄阳痿才戒,这时候才意识到问题的严重性,之前他们还在认同无害论,最后事实给出了真相。前段时间一位国家一级运动员的戒友也来戒色,他的身体素质很棒,但是邪淫对他容貌气质的摧残已经很明显了,所以他也有了戒色的想法。不管你的身体素质有多厉害,如果继续沉迷于手淫恶习,迟早也会出问题的,我见过一身肌肉疙瘩的健美运动员,脸上却有着明显的衰败相,外强中干的人有很多,伤到一定程度,外强也保证不了了,整个人都会垮下来,有的人甚至要坐轮椅,更有甚者直接就猝死了。邪淫的人就像活在了地狱,心理变得烦躁易怒,这样家庭就容易失和,很多撸者和父母吵架,甚至打骂父母,简直丧尽天良、畜生不如,这还是人干的事吗?!这位戒友戒色 177 天,可以说是拯救灵魂之旅,终于从地狱出来了,发自内心的快乐,找回了久违的美好。内心纯净祥和了,家庭也会变得和睦,家和万事兴,戒除邪淫,人生才会步入正轨。邪淫的危害太大了,邪淫真的可以把人变成畜生,畜生都不会打骂父母,邪淫真的可以把人变成恶鬼,面带鬼气,行尸走肉。邪淫就是在开发一个人的恶性与兽性,万恶淫为首,一点也不错。停止邪淫,恢复阳光纯净,多做善事,多孝顺父母,这就是在开发一个人善良淳厚的天性,只有在这种状态,你才会感觉到美好,感觉到快乐,感觉到一种深深的和谐与圆满。
\end{case}

\begin{case}
    邪淫之惨烈(自己的经历)神经症:焦虑症、抑郁症。我的焦虑症最让我受不了的是头痛,头有紧箍感,疼到什么程度呢,我都抬不起头来,每时每刻都在痛。去脑科检查,各项功能正常。但是就是痛,撕扯感,痛得人不能忍受,痛得我想捶墙。为此,我休学了一年。头痛使我从奥赛班掉到普通班最后几名,父母为此操碎了心,我不孝。第二,天天害怕,无时无刻不在恐惧,坐在教室,一直出冷汗,吓得看见同学老师都害怕,整天惶惶不可终日,那滋味,生不如死。第三,相貌,曾经的我活力无穷,皮肤白皙,五官端正。现在的我皮肤黑炭,眼睛一个大,一个小,因为一个单眼皮,一个双眼皮。大夏天手脚冰凉,我一个大男人,手还没有我长辈温暖,邪淫的果报惨烈啊!在我生病这几年,我的妈妈因为担心我,也患上轻度焦虑症和抑郁症,我不孝。我每个月吃的药,都要花一千多元,我家不是很富裕,我爸爸是干体力活,我高昂的医药费加重了我爸爸的负担。
    \subparagraph{附评} 这位戒友的经历真的挺惨的,每时每刻都头痛,痛到想捶墙。神经症就是一个症状地狱,里面充满了折磨,充满了哀嚎,充满了惶恐与绝望。神经症是伤精患者的一道分水岭,一旦跨过了这条界线,那痛苦就会暴增。根据我的调查和研究,现在得神经症的人,年纪都大大提前了,网络时代很多人都在熬夜久坐,甚至在熬夜时看黄手淫,这样就大大加快了症状爆发的速度。本来是撸龄达到十五年左右才会爆发神经症,而现五年左右即可爆发神经症,快的话三年左右就会爆发神经症,十几岁的少年得上神经症的越来越多,在花样年华撸出神经症,这是绝对悲催的事情,实在太痛苦了。现在手机可以随便上网,接触黄源太容易了,孩子本来自制力就弱,加之无害论的误导,结果就会疯狂沉迷手淫恶习,很多人已经达到了走火入魔、丧心病狂的地步了,后果实在不堪设想。这位戒友本来是奥赛班的,学习成绩应该很不错,后来掉到普通班最后几名,落差可谓不啻天渊。中医讲肾主恐,撸到一定程度,就会莫名其妙害怕,严重的会出现恐惧症,这种心理的失调也非常折磨人。这位戒友也提到了邪淫对容貌的影响,肾精亏耗容易导致面色黧黑,常伴有耳轮焦干、腰膝酸软、头晕耳鸣等,大家观察下邪淫的人,他的整体色调是偏灰暗的,灰头土脸的感觉,眼圈也是黑的,脸上气色晦暗,而精气神充足之人,他的整体色调是明亮的,双眼明亮而清澈,富有灵气,脸上容光焕发,肤质也清透,看上去很清爽的感觉,一点也不油腻。《黄帝内经》:“女子五七,阳明脉衰,面始焦,发始堕。”五七为三十五岁,女子三十五岁面始焦,“焦”这个字非常微妙,能量不足了,脸部就开始出现“焦”的感觉了,大家应该都烧过纸钱,纸钱烧完后就是一层焦的薄片,邪淫者在能量耗泄后也会出现“面焦”的感觉,那种感觉很微妙,远看的话很明显,脸部有一种“焦”的质感,耳轮也会焦,这都是能量不足的一种外在的征兆。邪淫后也容易出现大小眼,这就是邪淫导致的不对称,眼皮和眼睑都会浮肿变形,眼睛会变得难看,失去神采。大家应该都看过戒前戒后的对比照,很多人在戒色后眼睛突然变得很有神采,眼睛的确变漂亮了。眼睛本来就是一身精华的外显,观察眼睛就可以知道一个人的精气神如何。邪淫真的是大不孝,《孝经》云:“身体发肤,受之父母,不敢毁伤,孝之始也。”邪淫者不是毁伤发肤的问题了,完全是在掏空五脏六腑的精华,绝对是不孝之至。前段时间一位得神经症的戒友说他看神经症花了二十万,这可是一笔很大的数目啊,身体垮了,人财两空,生活陷入困窘。印光大师云:“近世少年,多由情欲过重,或纵心冶游,或昵情妻妾,或意淫而暗伤精神,或手淫而泄弃至宝。由是体弱心怯,未老先衰。学问事业,皆无成就。甚至所生子女,皆属孱弱,或难成立。而自己寿命,亦不能如命长存,可不哀哉?汝恐亦犯如上诸病,有则改之,无则加勉。”大师谆谆教导,犹如耳提面命,我辈后生,须当敬听。
\end{case}

\begin{case}
    飞翔哥,汇报一下我的破戒总结。就在昨晚,我失眠了,对此本来我是很淡定地保持观心的(因为有过多次失眠的经验了,一般保持观心,一段时间就能睡着),结果,竟然过了一个多小时还没睡着,就想着与其这样不如去答疑,结果在答了几个问题后,突然来了一个极细微的念头,我没有及时断除,最终导致了我的破戒。我反思,首先放假在家,我坐时间太久了,每天没有多少运动,这可能就是失眠的原因。最关键的就是,在用手机的时候我犯了几个错误,首先“在独处时应该把警惕开到最大”,结果我并没有做到,其次,实在不应该躺在床上用手机,因为躺着邪念容易冒出来,并且对心的掌控力也会下降!也因此,我总结了一个经验:床是破戒高发地点!在床上一定要把警惕提到最大,尤其是在床上玩手机。
    \subparagraph{附评} 这是一位资深戒友的破戒总结,很多戒友都看过他的精品帖,虽然他的戒色觉悟已经比较高了,断念水平也比较强,但是一着不慎满盘皆输。高手有时也会犯低级错误,在某些特定的情景下,对内心的观照会减弱,这时心魔就可能趁虚而入,从而一举攻破防线。当一个人睡不着时,心里就容易烦躁,这时候心魔就会进攻,心魔很会挑时间,而且专攻你的弱点,当你心态不稳时,心魔就开始下手了。独处时一定要保持高度警惕,牢牢看住自己的念头,这位戒友就是“突然来了一个极细微的念头”,没有及时断除,结果就破戒了。有时心魔入侵时非常微妙,非常细微,但还是能感觉到的,你知道它来了!如果你觉察晚了,那就会很被动,因为它已经发展壮大,到时就很难降伏了。躺在床上用手机,必须格外警惕,因为在那种无聊的状态下,很容易浏览擦边图或擦边新闻,然后看着看着就陷进去了。不管是手机上网还是电脑上网,你都要保持高度的警惕,擦边的内容不要去点,不要去看。躺床上时人比较放松,这时对内心的观照会减弱,如果再用手机看擦边内容,那就很容易被心魔攻破。在戒色后,我们一定要学会养生之道,不要久坐,久坐会导致气血瘀滞,久坐也会导致痔疮问题,久坐伤肾也伤脾,一直久坐也可能导致睡眠障碍。养生做好了,可以促进戒色,反之,养生做不好,就可能会间接导致破戒。如果这位戒友那天不失眠,也许就不会破戒了,那为何会出现失眠呢,就是因为坐的时间太久了,身体出现失调了,导致睡眠障碍,这是一环扣一环的。久坐问题比较普通,一般建议每 40 分钟起来活动一下,走动走动,做下八段锦。在我戒色的过程中,也有那么几次出现过睡眠障碍,导致睡眠问题的原因有很多,即使正常人偶尔也会出现睡眠障碍,特别是在季节转换时,很容易出现短暂的睡眠障碍,这时自己要注意调整,睡不着时要提高警惕,加强观心,不要浏览擦边内容,要有这个实战意识,你要时刻记住:心魔正在伺机而动!!!心魔正在等你放松警惕!!!攻下菜鸟,对于心魔而言很容易,但是攻下戒色高手,对于心魔而言就相当困难了,除非高手自己犯了低级错误,一旦被心魔抓住,那就会身不由己,重新沦为心魔的傀儡。戒色高手应该具备超强的实战意识,每一秒都是实战,每一秒都要警惕,心魔就在你的内心深处,你必须时刻小心,当心魔入侵了,你必须斩立决,你必须够强硬、够狠、够快!不要让心魔附体,不要让心魔奴役你,你必须降住心魔!做心魔的克星,不要被心魔克住。心魔强,你比它更强!强过心魔才不会破戒!这位戒友的悟性还是很高的,相信他会东山再起的,希望他好好吸取这次破戒的经验教训,希望他越戒越好。
\end{case}

\begin{case}
    《戒为良药》是我的救命稻草,强戒十余年,屡戒屡破,两个月前,我已经投降了,是的,真的投降了,当时想,没人能做到吧,后来接触到《戒为良药》,真是如抓到了救命稻草,现在每天至少看两百页的《戒为良药》,很多文章都看了不下五遍,这得益于我喜欢看书,家里堆满了书,我是那种喜欢看书的人,所以阅读能力比较强,这对于我学习《戒为良药》,如虎添翼。从接触戒色吧和《戒为良药》以来,每天签到,每天学习,从来没有再破戒过,以前总是周末破戒,飞翔大哥有一季专门讲周末破戒和无聊破戒的,那一季反复看了很多遍,很有效果,又到周末了,有种战战兢兢、如履薄冰的感觉,心魔进攻的频率明显加快,一天甚至十几次,但是,我已经不是那个被心魔随便虐的菜鸟了,更不愿意做心魔的提线木偶,现在有很多武器可以和心魔对抗了,这都得益于飞翔大哥的《戒为良药》,感恩飞翔大哥。
    \subparagraph{附评} 这位戒友的学习能力比较强,我了解过很多戒色成功的前辈,几乎都是善于学习之人,戒色的神力就来自于学习!学习提高觉悟,觉悟降伏心魔!要想戒色成功,必须持续学习戒色文章提高觉悟,要养成良好的习惯,持之以恒。猛火煮、慢火炖,熟读戒色文章,多做笔记,多复习。在刚开始的阶段,初心猛利,这时候应该猛火煮,拼命学习,如饥似渴,经历了“猛火煮”这一阶段,之后应该慢火炖,进一步学习、消化、研究,达到深入理解和认识的程度。慢火炖一段时间,再猛火煮,猛火煮和慢火炖要交替进行,要不厌其烦地学习,反复深入地学习,拿出最猛烈的决心来学习,干劲冲天,勇猛精进。古德云:“人一能之,己百之,人十能之,己千之。果能此道矣,虽愚必明,虽柔必强。”不要离开戒色文章,不要离开戒色笔记,上午复习三十条戒色笔记,可以管一上午,下午再复习三十条戒色笔记,可以管一下午。到了晚上再复习三十条戒色笔记,又可以管一个晚上。你手里应该有几千条的戒色笔记,不断复习这些笔记,达到滚瓜烂熟的程度,到时自然会迎来一次次顿悟。前段时间有位戒友分享了一些戒色笔记,大家看了都觉得获益匪浅,戒色笔记就是戒色文章的重点所在,必须经常复习,真正内化吸收。这位戒友每天看两百页的《戒为良药》,估计很多人都做不到,有的人可以每天看十万字的小说,但看戒色文章,一篇都看不进去。学习戒色文章就像练级一样,被心魔虐,就是因为你的觉悟等级低,你的断念水平弱,一次次被心魔虐爆,根本不是心魔的对手,心魔是大 BOSS,而你只是一个戒色菜鸟,根本没力量和心魔抗衡,很多人都是一触即溃,见了心魔根本没有抵抗,甚至都不知道什么是心魔,根本没概念。曾经的我也屡戒屡败,那时的我对心魔也没概念,但我知道自己心中有股异常强大的拉力把我拉入手淫的怪圈,当它出现时,我几乎无法与之抗衡,所以那时我也曾认为这个世界上没人能够戒除。后来我的觉悟提升后,才发现手淫是可以彻底戒掉的,关键就是要降伏自己的心魔。不管你看了多少戒色文章,也不管你明白了什么戒色道理,最后的那一下才是最具决定性的,那就是断念实战!心魔肯定会发动一波波的进攻,各种邪念都会冲上来,企图占据你的大脑、操控你的身体,所有的学习、领悟与练习都是为了提升你的断念实战水平,你必须强大起来,不顾一切地强大起来,精深学习、精深领悟、精深练习,三者缺一不可,只有理解没有练习,那是纸上谈兵,光说不练,假把式。只有练习而没有精深的领悟,那练习的水平很可能会陷入停滞,只有对理论有深透的理解,这样练习就会更有针对性、更有成效。断念要斩钉截铁,要有霹雳手段,不能犹豫和心存贪恋。你只有变强,不断地变强!强到降伏心魔!否则只有被虐的份!很多人都被心魔虐得团团转,被心魔虐得体无完肤,只有当你真正强大起来后,你才能终结这种被虐的局面。
\end{case}

\begin{case}
    我今年三十岁了,手淫十六年了,可以说手淫葬送了我整个青春,到现在悔得肠子都青了。手淫意淫导致我乏力、无精打采、尿频、阴囊潮湿、早泄、畏寒肢冷等。浑身的病生不如死,尤其是尿频半夜去好几次厕所,十多年都没有睡过好觉,那是一种煎熬、折磨,都是手淫导致的。现在每天晚上睡觉总感觉有凉气往自己身体里进入,然后就有尿意,每天晚上都会让尿憋醒两三回,这是我最苦恼的事情。然后还早泄,现在还没有女朋友,整个人没有信心了,可以说手淫意淫几乎毁了我。我现在郑重提醒那些青春期冲动纵欲手淫的年轻人,停止你们罪恶的手吧,这样是对自己好,千万别让手淫意淫毁了自己。如果真的可以穿越的话,现在的我穿越到十四岁那年,让那年的我看看现在手淫浑身是病的我,我想我会放手吧,可惜没有如果。
    \subparagraph{附评} 三十而立,然而16年的手淫恶习已经把身子掏空了,即使立起来也是豆腐渣工程,说倒就倒,没有任何抗震能力。手淫葬送整个青春,这种悲剧在多少人身上重复上演着,无法计算。十六年的手淫意淫,就是耗损了十六年的身体精华,最后怎能不垮掉?有的人二十岁时觉得手淫的危害不大,等到他三十岁,他就不会这样认为了,当达到一定的伤精年限,身体就会突然崩溃,那种崩溃的架势是很吓人的,身体突然就不行了,一旦不行,就会牢牢把你控制在那种颓势下,这种崩溃会形成一种向下的走势,要扭转这种颓败的趋势是很难的。十几岁、二十出头,毕竟还很年轻,很多问题都看得不全面,等到一定年纪自然会发现以前的自己好傻好无知。等到伤精症状爆发了,才会感叹,为什么以前身体好时不戒手淫呢?为什么要等到彻底垮下来才开始戒呢?到时悔恨就晚了。手淫后渐渐从天鹅变成了癞蛤蟆,当你处于纯真无邪的状态时,你就像白天鹅一样纯洁、优美、高贵和庄严,当你开始手淫后,你的心灵被邪念污染了,慢慢开始变得丑陋、龌龊、肮脏和猥琐。少年无知啊,疯狂手淫,葬送青春,直到三十岁还没有爬出撸坑,还没有摆脱撸囚的身份,悲催啊!三十岁的大叔郑重提醒年轻人,他是以自己的血泪教训来提醒的,那些无知瞎撸的孩子是该醒悟了,不要重复前辈的惨痛经历。如果可以穿越时光,十四岁的少年和三十岁的大叔面对面,告诉那个无知的孩子:“不要撸!知道站在你面前猥琐衰败的大叔是谁吗?——就是撸了十几年的你!”仰望苍天,泪流满面,向天再借十六年!这十六的青春年华都被手淫恶习葬送了,精华泄去后,只剩下一具疾病缠身的猥琐躯壳蜷缩在时光的角落里苟延残喘。多么希望这一切都是一场噩梦,醒来还是那个十四岁的纯净少年……
\end{case}

\begin{case}
    我破戒了。我戒三百多天了,这也会破,真是戒色来不得半点假啊!那几天频繁遗精,自己欲望很大,自己在路上控制了很多次,但因为必须要去办事,在街上总是去看异性,所以欲望不断袭来,晚上睡前更加严重。自己的内心说,“撸吧,YY 这么长时间不在乎撸一次”、“SY 一次再戒,反正已经 YY 了,危害都一样。”等等这些想法,最后我破了,我知道那是心魔,但我还是听从了它,我 SY 了两次,看了快一晚上的黄,我是麻木了,SY 完之后感觉生不如死。
    \subparagraph{附评} 心魔套路很深,你必须很小心。遗精后邪念会变得活跃,这时必须保持高度警惕,对待邪念应该用“严打”这两个字,保持高压态势,冒出一个灭一个,对邪念零容忍,只要冒出,必将消灭。印光大师云:“战之一字,关系甚深,人欲、天理之际,若不以力战,则理被欲蔽,俾理必隐而欲必著矣。”只有力战,才能打赢这场内在的战争!这位戒友在遗精后控制得不好,虽然他控制了很多次,但最后还是失控了。戒色三百多天,还是会遭遇魔考,即使戒色五年以上,依然会遭遇各种魔考,觉悟高、断念快,就能顺利过关,反之肯定会被心魔附体。心魔玩的就是套路,怂恿是最狡猾、最阴险的套路,心魔不仅会冒充你,还会不断地劝你破戒,对你一顿猛烈的怂恿,掀起疯狂的心理攻势。这位戒友已经开始主动 YY 了,然后心魔就利用已经 YY 来怂恿他,其实这时还是可以悬崖勒马的,但最后还是没把持住。虽然知道那是心魔,但架不住心魔反复地怂恿、反复地劝,最后还是听从了它,做了心魔的傀儡。这位戒友的破戒经历很典型,不仅在他身上,也在很多戒友身上不断上演着。戒色三百多天已经很不容易,但根据我的研究,一年左右还不算稳定,两年左右也不算稳定,达到三年以上才算基本稳定,真正的戒者必须久经考验而不破,不怕心魔套路深,就怕你不识套路,更怕你无法斩断套路。在一开始时,就应该做好断念,遗精后欲望很大,欲望的表现是什么?就是意淫,还有就是想看黄、想撸,当这种“想”的微妙念头出现时,就应该立刻断掉。它一次次冲上来,你一次次把它干下去,心魔不断组织冲锋,企图占领你的头脑高地,而你要像烈士一样和心魔斗争到底,子弹打光,手榴弹拼光,石头扔光,就是不破戒,气死心魔!真正的戒色高手根本不怕念头上来,上来多少灭多少,真正的戒色猛将根本不怵心魔,不怕贼强,只要将猛!正所谓:手握屠魔刀,杀尽邪淫念!!!
\end{case}

\begin{case}
    戒了之后一切美好,戒了两个月了,自己的鼻炎不治而愈了,感觉人越来越精神了,每天基本上都有跑步,跑四十分钟,真的很感谢戒色吧,让我这个十三岁的孩子这么幸运。对了明天开学了,我从来开学前一天都在补作业,自从来了戒色吧后,作业就早早做起了啊,不知道为什么好开心,所以发个帖来庆祝一下,希望每位哥哥叔叔都戒色成功。
    \subparagraph{附评} 这是一个 00 后的小戒友,记得前几年 00 后几乎没有,这几年 00 后开始发育了,戒色吧现在能经常看到十三岁、十四岁、十五岁的小戒友了,这些孩子很可怜,让我想起了以前的自己,我也差不多是在十四岁左右染上手淫恶习的,然后就一发不可收拾,越陷越深。无知的孩子打开了潘多拉的魔盒,放出了自己的心魔,于是就开始过一种两面的生活,在父母老师面前努力装成纯真的好孩子,其实暗地里已经不纯洁了,内在空间已经被黄毒侵蚀污染了。著名哲学家康德在文章里讲过:“手淫是对道德主体的彻底毁灭,用手淫来亵渎自己的行为是一种纯粹的兽性。”手淫属于邪淫的一种,具有高度成瘾性,手淫就像毒品一样令人沉迷上瘾,无法自拔。和海洛因以及其他具有致命诱惑力的毒品一样,只要沾染上一次,手淫就会令无数的孩子一步步堕入深渊。砖家的无害论会给撸者安慰,但是手淫之后身体的确变得不好了,各种症状开始爆发,这时才知道砖家扯淡扯大了!打着科学的名义在扯淡!不知误导了多少孩子。现在的孩子是不幸的,生活在一个极易接触到黄源的年代,在上世纪九十年代还需购买,而现在不用出门就可以获得,所以堕落就变得更加容易了。在手淫后你会发现自己的脾气变差了,戾气变重了,因为手淫就是在积累负能量,不仅心理开始失调,身体上的症状也开始爆发,到时就苦大了。现在的孩子也是幸运的,因为他们还能遇到戒色吧,可以在很早的年纪就认识到手淫的危害,回想十几年前的中国,几乎都被无害论给屏蔽了,在那个年代几乎接触不到手淫的真相。很多三十、四十岁的大叔戒友都比较羡慕现在的孩子能够这么早接触到戒色吧,真乃有善根、有福报的表现。这位十三岁的小戒友戒了两个月,改变很大,美好的感觉又回来了,他说“不知道为什么好开心”,其实孩子之所以那么开心,就是因为他们心灵纯净,心灵被黄毒污染后,就渐渐不开心了,当心灵恢复纯净了,自然就会变得开心起来,看这个世界也会觉得特别美好,纯净的孩子眼中有一种奇异的光芒,有一种天然的喜悦,有一种目睹奇迹的惊喜。纯净的孩子整个都是由开心组成的,根本不知道什么叫烦恼和忧愁,一旦进入发育期开始手淫了,渐渐就迷失在快感中了,被快感冲昏了头脑,忘记了纯净的大快乐。撸得越多,越不幸,到时负能量缠身,变得越来越不开心,内心被邪念污染,整个人开始散发腐烂的气息。愿普天下的孩子都能保有纯真美好的灵魂,失去纯真,失去一切,纯真是一个人最宝贵的财富!
\end{case}

\begin{case}
    去看了老中医,把脉医生冷笑了一下,说你这脉搏和你的年龄不符啊!我问医生我还有救吗,医生说能调好,放心,就是最近要禁止房事。我已经打算这辈子再也不撸了,虽然不知道能不能恢复,但是我想至少不会变得更差。今年二十四了。对不起我的爸爸妈妈。从今以后我会变得更加孝顺,更加懂事,珍惜一切。现在配合中药调理中,尽人事听天命。
    \subparagraph{附评} 这位老中医的冷笑很有威力,冷笑一般是贬义,但用得好,也可以救人于水火。之前有的人去看中医,二十多岁的人,他的脉象呈现的是六十多岁老头的脉象,里面已经严重衰败了,要不是还年轻,否则早就垮下来了。有的人撸了几年,觉得自己身体还行,其实他的身体已经出现“裂缝”了,随着伤精年限的延长,这条裂缝就会变得越来越宽,最后身体就会彻底垮掉。大家应该有亲身体会,那就是从开始撸管到最后症状爆发,这中间有一段时间身体感觉基本还行,这时候就会麻痹大意,以为撸管无害,其实危害正在暗中累积,只是暂时感觉不出来而已,就像一块肥皂在不知不觉间就被消耗了,达到一定的程度就会突然发现自己的身体已经大不如前了。看过很多 NBA 的扣将,在某个时期他们真的生龙活虎,膝盖一点问题都没有,但是后来膝盖就开始废掉了,不少扣将两个膝盖都做过手术。以前看过一位民间扣将的帖子,他说之前能飞善扣时不会想到会有这么一天,因为那时膝盖是好的,但是半月板正在不断磨损,最后膝盖就废了,连走路都疼,更别说跳了。伤害是暗中累积的,而当事人可能浑然不知,当达到临界点就会突然废掉。这位老中医也提到了要禁止房事,很多人边吃中药边手淫,这样治疗的效果就大打折扣了,弄不好身体还会变得更差,上补下漏,永难痊愈,上补下止,这样能量才能慢慢蓄起来。很多撸者都未婚先废了,之前疯狂放纵时,他们是想不到会有这样的结果的,因为年轻无知啊,被砖家洗脑后变得更加盲目冲动,只有症状的大棒才能把他们敲醒。古德云:“精存自生,其外安荣,内脏以为泉源,浩然和平,以为气渊。渊之不涸,四体乃固,泉之不竭,九窍遂通,乃能穷天地,被四海。”肾精就是身体的核能,不可随便耗泄,这是养命的能量,脑力、精力、精神都需要肾精的支持,古人懂得“保精惜精”,今人多被无害论洗脑,对于肾精的价值完全无知,把自己身体撸垮了,那人生就彻底悲剧了。想想自己的父母吧,这样疯狂撸管对得起含辛茹苦养育你的父母吗?一个邪淫的人有何颜面去面对列祖列宗?祖宗为子孙积下的福德就在你的一次次邪淫中消耗殆尽、化为乌有,作为子孙的你怎能不感到惭愧、汗颜与内疚?让我们做顶天立地的正气男儿,不做邪淫龌龊的无耻之徒。戒邪淫,行众善,尽人事,听天命!坚持戒色修善,达到一定的境界,头顶就像突然炸开一道纯白色的光柱,直冲云霄,一股正能量的气场立刻把你撑了起来。戒出一身浩然正气,堂堂正正做人,光明磊落戒色!!!
\end{case}

\begin{case}
    我刚才有了一次顿悟,突然对飞翔大哥的一句话有很深的理解,飞翔大哥说:“如果你能把戒色作为自己的兴趣爱好,你的进步会很快的。”刚看到这句话时觉得平淡无奇,现在才知道非常重要。兴趣是最好的老师,我们一旦对什么有了兴趣,就会毫不犹豫地去做,根本用不着逼自己。培养兴趣爱好的好处是:1. 动力十足,进步超快。2. 转移注意力,生活充实,不容易犯邪淫。
    \subparagraph{附评} 这位戒友的顿悟很不错,前辈的很多话都是自己有了很深刻的体会后才写出来的,看似简单,实则有很深的含义。当你把戒色当做自己的兴趣爱好来培养,这样看戒色文章就不会觉得枯燥乏味,不少资深戒友的戒色状态保持得非常好,为什么?因为他享受阅读戒色文章的过程,他享受戒色带给他的美好与愉悦,真正的戒者应该享受戒色,乐在其中。对于真正享受戒色文章的人而言,一万字并不算多,甚至还会觉得意犹未尽。但是对于很多新人而言,看几行字就看不下去了,新人多浮躁,心里很混乱,这时候一定要下大决心来学习戒色文章,养成良好的学习习惯,真正看进去,悟进去。学习戒色文章,复习戒色笔记,这应该是很享受的事情。当你看一本你很喜欢的书籍时,你不会感到疲倦,你会越看越有兴致,甚至拍案叫绝。学习戒色文章也需要找到这样的状态,当你真正契入进去后,简直其乐无穷,悟道的快乐是无可比拟的,往往在悟懂一个道理后,戒色觉悟就飞升了,一下就上去了。前段时间一位新人看不进戒色文章,就想通过自残来戒色,这是完全错误的,不看戒色文章,你的觉悟怎么提升呢?靠自残是不行的,真正的戒是心戒,从心上去戒,心就是念头,也就是要学会控制自己的念头。不少新人不注重学习,就只会提问,比如“要破了,怎么办?”为什么不做好断念?等到欲火中烧了才说怎么办?这时候已经晚了。来到戒色吧一定要学会专业戒色,不看戒色文章,那就无法入门,当你勇猛精进地学几天后,很多问题自然就得到答案了,前辈的文章里基本都有讲到的。新人一上来就强戒,最后注定会失败,新人需要前辈的引导,前辈极为重视学习,学习戒色文章是非常关键的。新人自己也要注意调整心态,消除浮躁,安静下来认真阅读和领悟戒色文章。戒色文章是需要自己用心去悟,用心去品的,好的戒色文章百看不厌,每看一遍都有收获。很多人看黄兴致颇高,看戒色文章兴致寥寥,这样就难戒了,看黄可以看一个下午,一个通宵,几乎废寝忘食,当你学习戒色文章也有这样的勇猛劲头,何愁不能戒除手淫恶习?应该把戒色当做自己的兴趣爱好,当做人生的一项历练、一种修为。要学会享受戒色文章,有了极高的兴趣后,就会迫不及待地打开戒色文章,看的时候也很容易进入状态。有的戒色前辈看起戒色文章来气吞万里如虎,学习的劲头很足,不愿错过每一句话,很多句子都是反复品读、反复研究。看一遍和看十遍的感觉是不同的,看到一定的遍数,突然就会顿悟,一下就和句子发生了奇妙的化学反应,一下就悟到了,真是“一句了然超百亿”。那种感觉真的太爽了!有一种醍醐灌顶、恍然大悟、豁然开朗的感觉。有的人的兴趣爱好是旅游或者体育运动,而我现在主要的兴趣爱好就是戒色和悟道,当你做自己喜欢的事情时,那自然就会很投入、很享受。戒色不是一种煎熬的过程,戒色应该是一种高度享受的过程,享受纯净的大快乐,享受戒色的单纯与美好,享受悟道带来的喜悦与兴奋,这种享受简单而纯粹,会带来巨大的满足感,真的是妙不可言。
\end{case}

\begin{case}
    飞翔大哥您好!这段时间我一方面坚持不断地学习戒色文章,主要是以您的《戒为良药》为主,另一方面就是每天至少花一个小时在吧里,回答其他戒友的问题,给其他戒友鼓励。坚持了快一个星期,我竟然获得了意想不到的收获:第一个是自己每天都感到非常快乐,内心里有一种“从内向外”的愉悦感和充实感。本来我是轻度抑郁症和重度焦虑症患者(以前您应该看过我的帖子,我当时还处于调养期),现在我竟然感到生活原来可以这样积极,这样充满阳光。之前有一段时间其实我也在戒色吧帮助其他戒友,但由于某些原因,中途就放弃了,疏远了戒色吧,但没多长时间,就又重新回到抑郁焦虑的状态。现在我重新回到吧里,与广大戒友一块学习,一块前进,恐惧焦虑的情绪竟然不知不觉地消失了,换来的是积极阳光的心态与行动。这可以说是戒色吧赐予我的第二次生命。(没得过抑郁症和焦虑症的人是不会理解一个抑郁症或焦虑症患者是生活在怎样的水深火热之中的)。第二个收获是,我竟然发现自己的脑袋变得比以前灵光了。之前看您的《戒为良药》,里面很多内容,看过也就是看过了,但这两天我突然发现自己能够“看明白了”。我知道文章里面的核心是什么,应该怎样运用到实际戒色中。不光是这样,在生活中,我竟然突然找到未来的方向了。以前自己虽然也算有理想,有目标,但这些目标既没有实现,也很飘渺。现在,我突然一下子看清未来的路应该怎样走了。我个人的理解是,自己坚持帮助其他戒友戒邪淫,同时自己坚持戒色,增加了自己的福报,脑力恢复得就很快。再一次感谢飞翔大哥,感谢戒色吧里所有的戒友!
    \subparagraph{附评} 这是“黄金眼”的反馈,是一位老戒友了。这次他终于找到了戒色的感觉,开始步入正轨了。有句话叫“为善最乐”,当你真正无私地去帮助别人了,你的收获肯定是巨大的。每一次善行,每一句鼓励的话,每一句感恩的话,都是在积累正能量。邪淫会让你充满负能量,一方面要戒邪淫,另外就是要有意识地积累正能量,这样身心恢复才会比较快,几乎所有戒色前辈都在强调行善积德,有的戒友听进去了,一直在坚持答疑或宣传,当做自己的一份事业在坚持,有的已经坚持了好几年,收获非常之大,很直观的感受就是心态变得积极正面了,内心充满阳光和正气,一种微妙的愉悦感也开始洋溢开来,和家人朋友的关系变得和睦了,人变得积极向上了,颓废消极的感觉完全消失了。神经症患者的确生活在水深火热之中,普通人很难理解,最近有位得抑郁症的明星自杀了,闹得沸沸扬扬,得了这类疾病的确很痛苦,身陷症状地狱,即使你再有名再有钱,也无法驱逐这种痛苦,很多患者都选择了自杀,自杀是不对的,但被逼急了,人就很容易走极端。好好坚持戒色养生,多行善积德,这样神经症就有望恢复正常,症状严重的话建议配合中药调理。黄金眼第二个收获就是脑力上来了,脑力上来后,理解力就会变强很多,过去看戒色文章很多地方看不懂,无法领会前辈的意思,现在脑力提升后,终于能够“看明白了”,很快就能抓住核心。很多新人看不进戒色文章,一方面是心态浮躁,另外也有脑力下降的原因,脑力下降后很多内容都无法正确理解了,甚至还会误解。我那时也经历过脑力严重下降的困扰,后来坚持戒色养生,脑力逐步提升后,感觉就大不相同了,以前看大德开示,很多地方都难以理解,而现在很快就能契入进去,对于开示有很好的理解和领会。戒色可以让人恢复底气,对未来充满希望,即使你现在是穷光蛋也没关系,这种底气才是最可贵的,很多创业大佬在之前也很穷,甚至还失败过多次,但是人家的底气和动力尚在,这样就可以东山再起。我很看重戒邪淫后心态的正向改变,相比于身体症状的缓解消失,心态的改变也显得非常重要,你会发现心里开始变得祥和了,你的脾气开始变好了,对别人的容忍度也开始增加了,你开始变得积极向上,变得阳光开朗,你的人生开始步入正轨了,之前一直在“邪轨”上,邪轨是自毁之路,邪轨是有很强惯性的,很多人一直处于那个怪圈中,难以自拔。要突破这个怪圈,必须下大决心戒色,并且要通过学习不断提高觉悟,这样才有望回到正轨。回到正轨后你就会发现这样的生活才是正常的,过去邪淫放纵完全就是不正常、失控、病态的生活,是活在心魔的奴役之下,根本没有自由可言。《周易》云:“君子以遏恶扬善,顺天休命。”这里的“休”是美好、美善之意,君子应当顺应天道的规律,遏制恶的事物,发扬善的事物,这样才会有美好的命运。古圣先贤是真正懂得天地法则的人,他们具有高度智慧,深谙背后的运行规律,他们的开示极具分量。《尚书》云:“作善降之百祥,作不善降之百殃。”一个人积善还是积恶完全取决于自己,积善会导致好的结果,积恶最终会导致自食恶果!君子的第一戒就是戒色,是孔圣人提出的,戒色就是戒邪淫,戒邪淫对于一个人太关键了,可以说是关键到了极点。邪淫放纵,后患无穷,将来的苦报很惨烈啊!大家开车都懂得遵守交通规则,而我们生活在天地间也是要遵守道德准则的,是绝对不能乱来的,古圣先贤已经把背后的规则告诉我们了,也可以说是把天机泄露给我们了,你真正听进去了,真正去实践,你就会发现圣贤所言非虚,他们的开示有着极其深刻的道理。《礼记·大学》:“大学之道,在明明德,在亲民,在止于至善。”当你把戒色修善融入自己的生活,你的生命就会升华到全新的境界,到时再看看过去邪淫的日子,简直是白活了。
\end{case}

下面步入正文。

这季我将阐述一下《菜根谭》对戒色的启示与意义,《菜根谭》是我比较喜欢的一部国学经典,是明朝还初道人洪应明收集编著的一部论述修养、人生、处世、出世的语录集,为旷古稀世的奇珍宝训。《菜根谭》成书于明万历年间,距今已有近四百年的历史,内容共分五部分:修省、应酬、评议、闲适、概论,是一部融合儒、释、道三家人生哲学的修身养性、为人处世的经典之作,对于正心修身、养性育德,有不可思议的潜移默化的力量。市面上有很多关于《菜根谭》的书,我也买了一本,《菜根谭》从为人处世到修道的内容都包含在内了,采用的是语录体,揉合了儒家的中庸思想、道家的无为思想和佛家的出世思想的人生处世哲学。《菜根谭》文辞优美隽永,含义深远,十分耐人寻味,开卷有益,百读不厌,《菜根谭》以心学、禅学为核心,里面也专门讲到了念头实战,我特别重视这部分的内容。古人的精神境界的确很高,有很多思想都值得现代人好好学习和品味,对于陶冶情操、提高觉悟非常有帮助。

咬得菜根,百事可为,《菜根谭》需要细细地品,就像品茶一样,这样才能品出其中的韵味与深意,“菜根”也需要反复地咀嚼,这样才能充分吸收,智慧才会越来越高。“欲做精金美玉的人品,定从烈火中锻来;思立掀天揭地的事功,须向薄冰上履过。”这四句开场白阐明了人生需要磨炼,更需要警惕和小心。烈火出精钢,只有经受住磨炼和考验,才能达到更高的境界。“横逆困穷,是煅炼豪杰的一副炉锤。能受其煅炼者,则身心交益;不受其煅炼者,则身心交损。”当你身处困境时,你要知道这正是一次“煅炼”的机会,所谓百炼成钢,经过淬火煅炼,你才能绽放出全新的光彩。戒色豪杰、戒色英雄、戒色的仁人志士必须经得起千锤百炼,必须经得起实战的反复检验。《菜根谭》给了我很多启发,特别和戒色的道理是完全相通的,戒色要进入更高的境界,那就应该深入学习圣贤教育和传统文化,这样才能进入极为稳定的戒色层次与境界。不仅要学会戒色,也要学会做人,学会修德,见过不少十几岁的戒友,学习的冲劲很足,但是缺少修德的意识,他不知谦虚、宽容、恭敬、感恩、孝顺等等美德品质和戒色的关系,戒到一定程度往往骄傲自满,这样就很容易破戒。要走好人生路,那就必须学会做人,学会修德,这里面的学问是很深的,需要自己一点点去学,一点点去悟。

古德云:“心安茅屋稳,性定菜根香。”邪淫的人他的心是不安的,往往处于惶恐的状态,邪淫后有一种不对劲的感觉,不仅身体的精华泄掉了,整个人的正气也跟着泄掉了,戾气开始增加了。当你处于心安性定的状态下,你才能发现整个世界的美好,当你处于邪淫浮躁的心态下,你看别人都会觉得不顺眼,很容易引发争吵。当你心安性定了,内心就会感觉很祥和,处于一种微妙的平衡与和谐中。仰不愧天,俯不怍人,内心无限光明和坦然,因为没做什么亏心事和见不得人的龌龊事,所以内心就会很安定,很祥和。《论语》中,孔子赞颜回:“贤哉,回也!一箪食,一瓢饮,在陋巷,人不堪其忧,回也不改其乐。”颜回过着粗茶淡饭的清苦生活,住在简陋的房子里,但他却能自得其乐,丝毫不受外界物欲的困扰,“君子忧道不忧贫”,悟道的快乐岂是看黄手淫所能比的,古人就是有气节,有着很高的精神追求。你觉得他清苦,其实他的内心快乐得很!他已经超脱了欲望的束缚,他正在享受纯净的大快乐、大喜悦,悟道者的快乐指数和幸福指数都是非常之高的,他的快乐是那么纯粹,那么发自内心,他悟到了纯净真我,他活出了自己的本心,这种快乐的程度真的是太强烈了,普通人根本无法理解。

很多人都以为有钱了就会快乐,其实不然,据我所知,很多富商虽然家财万贯,但是内心还是很空虚,还是觉得不快乐,不满足,甚至感觉很烦恼。即使吃遍了星级酒店的山珍海味,最后还是觉得粗茶淡饭吃得香。当你心安性定了,你才能品出生命的真滋味,真味只是淡,淡中有不可思议的味道、不可思议的深度、不可思议的境界。亿万富翁的快乐指数可能还不及纯净孩子的千分之一,大人与孩子的快乐,很大的区别在于,大人的快乐需要外在的刺激,比如需要做成某件事,或者买新手机、新房、新车或者奢侈品、谈恋爱等等。外在刺激导致的快乐很短暂,而且这种快乐会逐渐演变成空虚和不满足,而小孩无法理解大人的世界,小孩处于纯净的次元,他们的快乐并不需要外在的刺激,心灵纯净的孩子自然就是快乐的,就像一朵鲜花,自然就有芬芳。大人已经被物欲和性欲所缠缚,就像被两条巨蟒紧紧缠绕和勒住一样,他们早已忘记了纯净的大快乐,而这正是最可悲的事情。曾经有位著名的亿万富翁这样说过:“为什么我有钱,却不快乐,还得了抑郁症?”真正的大快乐与钱无关,很小的孩子,他们没有什么钱,但是他们可以开心快乐一整天。大人的可悲之处,就在于他们的心灵已经被各种欲望的想法给污染了,这样他们的内心深处就无法涌现纯净的大快乐了,只能通过外在的刺激来获得短暂的快乐,而那种快乐是极不稳定的,很容易助长人的贪心,最后给人带来莫大的痛苦。

仔细研读《菜根谭》,听取圣贤的教诲,真正明白之前完全无知的道理。在粗茶淡饭中,体会淡泊的美妙,在精神生活中,获得极大的满足,真正开始享受纯净的生活。钱可以买很多东西,但是买不来纯净的大快乐,纯净的大快乐只能通过净化自己的心灵才能获得。纯净的大快乐可以持续一整天,而靠外在刺激的快乐只能持续一个短暂的片刻,纯净的大快乐是内在的开花,是内在的涌现,是内在的爆发,而外在刺激的快乐根本无法与之相比。一个人虽然穷,但如果他知道如何获得纯净的大快乐,他就根本不会在意自己穷,因为在精神世界里他就是最富有的人、最快乐的人。为何要像一个乞丐一样在外乞讨快乐?只要恢复纯净的心灵,你就是最快乐的人,真正认清这个道理,谁还会那么在乎自己是否有钱呢?在小的时候,我看到很多大人总是板着一张脸,觉得他们很严肃,觉得他们的世界很奇怪,小伙伴们在一起总是开开心心的,自己一个人时也会感到莫名的开心与快乐,没有任何理由,没有任何原因,就是快乐,泉涌般的快乐从内心深处不断喷涌而出,不得不快乐,完全地沉浸在快乐的感觉里,那么单纯,那么美好。等到了发育期,我变了,我开始从看黄手淫中寻找快乐,从各种物质追求中寻找快乐,从那时起,我发现自己很难有发自内心的笑容了,我无法再经历到纯净的大快乐,我终于明白了大人们为何总是板着一张脸,显得那么严肃和怪异。很多大人的笑容都是僵的,不是发自内心的笑容,因为大人的心灵被污染了,非常可悲的状态。小时候还想着快快长大,其实长大了才知道什么叫“金色的童年”和“灰色的成年”!丢失了纯净的大快乐,整个人生都是灰色的,不管你多么有钱,在内心深处你都是穷人,一个向外乞讨快乐的穷人。

《菜根谭》是一位悟道高人所写,思想境界很高,从为人处世到体悟大道都有讲到,很多人都喜欢《菜根谭》,各行各业的都有,但很少有人把它和戒色联系在一起,其实《菜根谭》里有很多语录对于戒色都很有启示,甚至有直接指导的意义。下面是我精心挑选的 20 条语录,我将从戒色的角度来做逐一的讲解和分析,如下:

\begin{quote}\it
    一念错,便觉百行皆非,防之当如渡海浮囊,勿容一针之罅漏;万善全,始得一生无愧。
\end{quote}

\subparagraph{解析} 古德云:“防意不严,走尽邪蹊。”防意必须要严,意就是念头,在起心动念上必须要严防,心魔伺机而动,你不严防,就会被心魔钻了空子。很多人破戒后回忆,就是某个邪念入侵时,自己没有及时断掉,也就是没有防住,结果就被心魔附体了。有时邪念入侵时很微妙,有一种想看黄、想撸的微妙感觉出来了,这时就要保持高度警惕,立刻断除这种微妙的感觉,这种微妙的感觉就属于比较细微的念头。还有的戒友戒到一定程度,内心出现了这样一种声音:“撸一次吧,就一次,没事的。”这就是心魔在怂恿他了,出现这种念头时也必须立刻断掉,心魔的怂恿五花八门,都要学会识别和断除。一旦被心魔攻破,你不是你,到时就会干出很多龌龊的事情来。“一念错,便觉百行皆非”,那个念头你没看住,后果就会不堪设想,一定要严防死守,《法苑珠林》:“防意如城,慧与魔战,胜则无患。”对起心动念保持密切的监视和观察,必须戒备森严,无懈可击。“万善全,始得一生无愧”,人这辈子一定要学会行善积德,日行一善或三善,持之以恒地坚持做下去,达到千善、万善的程度。

\begin{quote}\it
    一念常惺,才避去神弓鬼矢;纤尘不染,方解开地网天罗。
\end{quote}

\subparagraph{解析} 惺,意为“清醒”。用在这里也有“警觉、警惕”之意。戒色战场,必须很警惕、很小心。即使你完全懂得断念的原理,并且把断念水平练到很高的程度,但如果你一不警惕,那就会再次破戒,见过一年、两年破戒的,即使已经具备了很高的觉悟了,还是要保持充分的警惕,否则一放松警惕,心魔就笑了,因为机会来了。对于内心的每一个念头都要很警惕,要学会识别能够导致破戒的各种念头,一旦出现必须立刻消灭。印光法师用的是“消灭”这个词,在念头实战时,必须要消灭入侵的邪念,实战随时都可能发生,也许在你上课时,或者在你走路时,或者在你坐着时、躺着时,那种念头随时都可能会冒出来,所以你必须时刻保持警惕,开启“实时监控”,高手皆警惕,如果警惕意识不行,那迟早会再次破戒的。

\begin{quote}\it
    夜深人静独坐观心,始知妄穷而真独露,每于此中得大机趣。
\end{quote}

\subparagraph{解析} 这段其实就是在讲禅修了,独坐观心,也就是看着自己的念头,当妄念消融了,纯粹的觉知(亦即妙明真心)就显露出来了。通过觉察进入纯粹的觉知,然后安住一会,没多久另外一个妄念又会冒出来,再觉察消灭,这时又会出现一个无念的空当,继续安住。一次次觉察,一次次安住,到时就会变得极为娴熟和稳固,对心的掌控就会变得越来越强。安住于纯粹的觉知是非常快乐的事情,因为在那个状态,心灵是非常干净的,一个妄念都没有,当心灵干净了,自然而然就会体验到纯粹的大快乐。晚上相对比较安静,在那种环境下更容易进入纯粹的觉知,纯粹的觉知才是真正的你,身体不是你,念头不是你,你的真实身份是纯粹的觉知,之前关于这点我也讲过好几次,一旦你真正认清了,你的觉悟水平就进入了更高的悟道层次。本来你习惯性地跟着念头跑,因为你认同念头是你自己,然而现在你可以看着念头,突然你变成了观察者,正在看着自己的念头。你不再认同念头,因为你知道自己是纯粹的觉知,而念头只是工具而已。要学会成为念头的主人,不要跟着念头跑,牢牢安住于纯粹的觉知上。《涅盘经》云:“愿作心师,不师于心。”也就是要学会主宰自己的心,心就是念头,一定要做念头的主人!

\begin{quote}\it
    饱后思味,则浓淡之境都消;色后思淫,则男女之见尽绝。故人当以事后之悔,悟破临事之痴迷,则性定而动无不正。
\end{quote}

\subparagraph{解析} 等你吃饱了,再看之前的食物,就会觉得提不起兴趣了,连续吃几天同样的食物,可能会吃到吐。类似的现象,当你撸完后,你就会觉得没意思,真的很没意思,射完了就会有一种很深的空虚感和疲乏感,有的人还会后悔,因为他本来想戒撸的,现在破戒了就会感到后悔。再大的诱惑,在你射完后都会变得“不过如此”,所以多思维一下射完后的心境,即可克服对女色的痴迷。摩擦你的手指,并不会感到兴奋,反而会觉得枯燥乏味,因为摩擦手指并不会导致多巴胺分泌,然而摩擦 JJ,你就会感到兴奋,因为撸管会导致多巴胺分泌。如果去掉多巴胺这个因素,撸管就会变得枯燥乏味,甚至是难受,大家应该都有这个体会,那就是在射完后再去碰 JJ,就会感觉很难受,一点都不舒服。平时应该多这样思维:撸管一点都没意思,我不需要撸管。多这样思维就会让痴迷由浓转淡,最后恢复正常,一个真正有正气的男人是不会痴迷女色的。很多撸者都觉得撸管很爽,其实真正的大爽是戒撸,当你体会到了戒撸的大爽,你就会觉得撸管真的很没意思,一个大男人干点什么不好,为什么非要躲起来偷偷撸管呢?真是猥琐至极!一位新人发帖说自己被人撞见了,想死的心都有了,是被不认识的人看到的,他说:“裤子脱了,正在看片,还没开始打呢,对视一秒后想死的心都有了!”人最猥琐的时刻被陌生人撞见了,然后对视一秒,那种眼神太复杂了,内心那个纠结啊!那种无地自容的心情,痛苦啊!还有一个十五岁的孩子在看片,被他妈妈从背后撞见,当时他的内心真的是翻江倒海,五味杂陈。一个人做这种见不得人的事情,还被人发现了,那种心情可想而知,所有纯真的表象、所有的道貌岸然都在瞬间轰然倒塌了……“原来他是这样一个人,原来他有这么猥琐的一面。”一个人最阴暗、最猥琐的一面被人发现后,他的内心是绝对纠结的,只有戒色君子才能做到光明磊落、表里如一、问心无愧!

\begin{quote}\it
    降魔者先降其心,心伏则群魔退听;驭横者先驭其气,气平则外横不侵。
\end{quote}

\subparagraph{解析} 这句语录直指戒色的核心,也就是降伏自己的心魔,也就是修心,也就是观心断念。从第一次撸管时,你就放出了自己的心魔,然后你一次次地喂养自己的心魔,这个心魔变得越来越变态,你自己都感到害怕,就像在你心里养了一头可怕的怪兽一样,这头怪兽的口味越来越重。刚开始你的性取向还基本正常,随着撸龄的增加,你会发现自己开始变态了,逐渐喜欢看变态的片子,普通的片子已经无法让你兴奋了,最后你就变得畜生不如了。上次一位新人说他都有吃屎的倾向了,太可怕了,一个人居然可以堕落到如此不堪的地步。当心魔主导你的人生,最后的下场必定是悲剧的,所以降伏心魔就成了人生最重要的一课,但这极为关键的课程在学校里是学不到的,学校里并没有这种课程的学习,这种极为特殊的知识只有在圣贤教育里才能找到,虽然国外的戒色文章也讲控制念头,但远不如圣贤教育来得系统和深入。不少人戒了很久都不知道如何断念,在面对心魔时,一次次被打败,一次次全面沦陷,心魔的表现就是邪念袭脑,如果你能做好断念,心魔就不会得逞,所以“降魔者先降其心”,《金刚经》也讲“降伏其心”,如果你降伏不了自己的心魔,你迟早会被心魔拖入地狱。

\begin{quote}\it
    欲路上事,毋乐其便而姑为染指,一染指便深入万仞。
\end{quote}

\subparagraph{解析} 世上无如人欲险,几人到此误平生。一旦你尝到了手淫的快感,就会疯狂地追求这种短暂的快感,不顾一切地摄取这种快感,很多人刚开始时都一天一次或者多次,身体不行了才改为一周一次,但有时还会回到一天多次,次数是不固定的,但最后身体必定会垮掉,早点晚点,也许今年你感觉还行,但是明年身体就不干了,开始爆发各种伤精症状,到时就痛苦不堪了。当事人之前很可能以为不会这么严重,但是当伤害积累到一定程度,那就是一笔“巨债”了,手淫的巨债最后还是要自己偿还,到时必将自食恶果。“一染指便深入万仞”,这句话和“一发不可收拾”颇为相似,一旦开始很快就会上瘾,你觉得你没上瘾,其实你已经上瘾了,只是你不知道而已,你已经处于失控的状态了。不上瘾的人,立刻就可以彻底戒掉,永不复撸,但很多号称没上瘾的人,一个月都坚持不下去,很快就会再次复撸,而且是变本加厉地疯狂破戒。“一次上瘾”可以说是手淫这个恶习的标签,一旦尝到了那种快感,撸者的眼神里就会充满了贪婪,总是想要更多,永不满足,撸死方休,那种状态完全走火入魔了,被心魔附体后,那双眼睛的眼神特别怪异和可怕,有一种禽兽的感觉,已经没有人性了。在那种状态下,之前的戒色誓言全部都抛之脑后,唯一想做的就是撸出来。一旦染指手淫恶习,那真是掉入了万仞之深渊,整个青春乃至整个人生都可能被这个恶习毁掉,手淫的恶果迟早会显现,短期内可能不明显,伤精时间长了,慢慢就开始给你颜色了。

\begin{quote}\it
    斩绝萌芽,才见本来真体。
\end{quote}

\subparagraph{解析} “萌芽”这两个字很到位,已经告诉你断念的最佳时机了,断念要在念头初萌之时,要斩绝萌芽,一定要快!念头刚起来时,还很弱,这时断念正是时候,如果晚个几秒或者十几秒,到时念头就会变得强大许多,到时断念就会较为困难。所以绝对不能让念头起势,在最开始的零点几秒就要斩断邪念,准备好你的快刀利刃,念头一起,马上斩断,冒出多少,斩多少。真正的高手根本不怕念头冒出来,因为邪念就是来送死的,所要做的就是快斩!出手一定要快!这是你死我活的战斗,把邪念赶尽杀绝!彻底消灭,不留余地,零容忍!一夫当关,万念莫开!!!当你这一刀斩下去,邪念顿时消失,这时出现了一个无念且警觉的空当,这正是本来真体——纯粹的觉知。一次次断念,一次次体验纯粹的觉知,慢慢你就会变得越来越强大,邪念再也难以撼动你。

\begin{quote}\it
    一段不为的气节,是撑天撑地之柱石。
\end{quote}

\subparagraph{解析} 《欲海回狂》云:“不可两字之中,大有力量。”君子有所为而有所不为,邪淫的事情君子坚决不为,行善积德的事情君子尽力而为!一段不为的气节,一身的铮铮铁骨,他们看黄,他们手淫,而我们拒绝看黄,拒绝手淫,这就是君子的气节,拒绝同流合污。这股正气必须要强悍,君子绝对不能逾越心中的底线,一个人有了气节,有了骨气,他整个人就会被正气撑起来,气场会变得格外强大。他们邪淫堕落,我们正己化人,我们拒绝一切邪淫的行为,这就是钢铁般的立场,这就是戒色烈丈夫的气节。男儿顶天立地,岂可邪淫堕落?要有关二爷那种威震乾坤的正气,手握青龙偃月刀,邪念来犯斩立决!戒色的气节和骨气必须要硬,必须要狠!

\begin{quote}\it
    心地干净,方可读书学古。
\end{quote}

\subparagraph{解析} 当你的心灵被邪念污染了,每天意淫纷飞,这时候你读圣贤的书籍,是很难真正契入的。当你戒除邪淫,学会修心后,心地慢慢恢复干净,这时候学习圣贤教育就会很快契入进去,越看越觉得妙,越看越觉得有味道。《论语》的文章小学时就学过,到现在我还在看还在学,越学越有收获,圣贤教育历久弥新,底蕴特别深厚,有着极高的智慧。不仅要学会专业戒色,也应该多学习圣贤教育,只有圣贤教育才能把你的生命提升到更高的境界与层次。古人的智慧真的不可思议,真理经过一万年都不会改变,关键就是要打扫心地,这样才能读懂圣贤教育,到时一定会受益匪浅的。当你的心地真正干净了,你看这个世界也会觉得很美好,当你心地变得龌龊了,你看人、看世界都会觉得不爽,美好的层面一直存在,你之所以感知不到,就是因为心灵变脏了,你的心灵堆满了邪念的垃圾,必须净化你的心,当心灵恢复纯净后,到时你的整个感觉就会完全不同,再看花花草草,就会突然感觉特别美好,有一种特别久违的感觉深深地震撼着你的灵魂。

\begin{quote}\it
    心体光明,暗室中有青天;念头暗昧,白日下有厉鬼。
\end{quote}

\subparagraph{解析} 古德云:“不欺暗室!”即使在没有人看见的地方,也不做见不得人的事。手淫就是一个隐秘的恶习,是一个专欺暗室、见不得人的恶习。再高大的形象,一旦开始手淫,就会显得异常猥琐,虎躯一泄,顿成鼠躯,脸上开始出现猥琐的表情,笑起来也特别猥琐。举头三尺有神明,暗室之中有青天,所以即使在暗室也要保持光明正大,仿佛有十目所视、十手所指,一个人坐在家里依然很有规矩。当一个人独处时,往往会遭遇心魔入侵,所以古人特别告诫要慎独,手淫基本都是独自一人时所犯的恶习,所以慎独就显得特别关键,当邪念起来时,必须立刻断掉。念头不能暗昧,念头一暗昧,真的就会鬼气缠身,很多撸者的脸上都有鬼气,那是一张下塌变形的脸,整张脸的走势是向下塌的,仿佛就像蜡烛一样向下融化,双眼无神而空洞,一股邪气充溢在整张脸庞上。当你泄精后,走到青天白日下,就会感觉很不自在,因为你身上的阴气太重了,真的是白日下有厉鬼,而且还是个色鬼!

\begin{quote}\it
    情欲意识,尽属妄心,消杀得妄心尽而后真心现。
\end{quote}

\subparagraph{解析} 经常会看到有的戒友说:“最近欲望好重,好想撸。”欲望是一种要求,当你出现“好想看黄、好想撸”的微妙感觉时,应该立刻断除,这样就等于化解了欲望。戒色是化解欲望,不是硬憋着,化解的关键就是在念头上断除,我戒到现在从来没有憋着压抑的感觉,因为已经通过断念把那种微妙的感觉给彻底化解了,这样它就无法作用在我身上。“消杀”两字很给力,及时消灭和杀掉那种妄心,这样就不会出现煎熬感和破戒了。为什么会感觉憋着,因为你断念晚了,断念早是不会有憋着的感觉的。普通人出现那种欲念时,他是不会断掉的,因为砖家告诉他欲望是正常的,有欲望就要发泄,砖家貌似很正确的理论其实套路很深,砖家只谈发泄欲望,而不谈控制欲望,也不谈化解欲望,只是一味地让你耗泄身体的精华,最后肯定症状缠身。戒色的人应该知道四种套路深,第一种心魔套路深、第二种砖家套路深、第三种男科医院套路深、第四种网上卖药的套路深。心魔的套路是为了让你破戒,砖家的套路是为了让你邪淫堕落,男科医院的套路是瞄准你的钱包,网上卖药的套路,先假装帮助你,然后让你买他的药。不管套路有多深,请记住:不要上套!这条语录的后半段和第七条有点类似,断除妄念,真心就显现了,真心即纯粹的觉知,也就是真正的你。

\begin{quote}\it
    念头起处,才觉向欲路上去,便挽从理路上来。一起便觉,一觉便转,此是转祸为福、起死回生的关头,切莫当面错过。
\end{quote}

\subparagraph{解析} 这条语录就是专门讲念头实战的,可以说是对念头实战最直接的指导,“一起便觉,一觉便转。”这八个字就相当于断意淫口诀的“念起即觉,觉之即无。”两者的原理是完全相同的,就是要你在念起时,觉察消灭之。戒色的实战就在邪念入侵时,不管你看了多少戒色文章,即使看了几千篇,最后的检验就是邪念入侵时,你所给出的表现。觉察消灭是一种,思维对治也是一种,上次一位戒友戒了188天,用的就是思维对治,总体而言,觉察消灭更快一些,当然思维对治也很不错,关键就是要熟练。学佛的戒友也可以念佛号来斩断邪念,不管何种方法,都是为了降伏邪念。憨山大师云:“不许妄想萌芽,潜滋暗长。若能于妄想起处一念斩断,则旧积业根,当下消除。所谓‘不怕念起,只怕觉迟’。觉照稍迟,则被它转矣。若能于日用起心动念处,念念觉察,念念消灭,此所谓众罪如霜露,慧日能消除。”觉察即消灭,觉察即降伏,憨山大师这段话强调的就是“念念觉察,念念消灭”。觉察力需要一个练习的过程,刚开始觉察力比较弱,坚持练习观心断念,最后就会越来越强,很轻松就能做到念起即断。憨山大师在《观心铭》里讲到:“念起即觉,觉即照破。”说的也是觉察,觉察太厉害了,觉察是最快的刀,斩断所有入侵的邪念。《菜根谭》几百条语录中,这一条是我格外看重的,因为这一条直接讲念头实战,可以说是直指最关键的要点。

\begin{quote}\it
    老来疾病都是壮时招得;衰时罪孽都是盛时作得。故持盈履满,君子尤兢兢焉。
\end{quote}

\subparagraph{解析} 这一条语录揭示了疾病发生的规律,很多病的病根都是在年轻时埋下的,因为年轻时身体底子还行,感觉不是很明显,但是随着年纪的上升、身体恢复速度的下降,到时各种伤精症状就开始爆发了,也许你撸了十几年没有大毛病,然而到了30岁或者40岁以上,突然给你爆发一个严重的疾病,到时身体就会彻底垮下来。年轻时往往比较盲目无知,很多事情没经历过,也没人告诉他,仅凭自己有限的体验而得出的结论往往是错误的,一次次撸就是在一次次掏空自己,从撸到垮,中间是有一个过程的,最后的结局肯定是垮下来,肯定会有那么一天的,那一天就是跟你算总账的时候,也可以说是秋后算账。看到一个二十岁出头的新人,他说自己撸了十年左右,还没有严重的症状,其实伤害已经造成了,再撸下去,身体迟早会出大问题的。二十出头毕竟还很年轻,很多问题都没有完全暴露出来。一位戒友劝他朋友时是这样说的:“别看你现在还行,再过几年你就知道了。”很多新人也比较无知,其实伤害已经表现在他身上了,只是他不知道和手淫有关,比如脑力下降,气色变差,双眼无神、腿软腰痛等,他因为无知所以不知道是手淫导致的。有的人会说:“我也撸,怎么没你们说的症状。”不少人之前都有这种想法,自我感觉还行,其实有时自我感觉是有欺骗性的,别看你自我感觉还行,其实你的内在已经开始衰败了,量变产生质变,到时症状肯定会爆发。前辈是过来人,也曾经年轻过,在某个阶段的确自我感觉良好,也正是那个阶段的盲目无知为以后的大衰败埋下了伏笔。有的戒友在身体垮掉前,他的自我感觉非常之好,自感身体很强壮,根本不会料到自己的身体会垮到那种程度,最后垮下来时,真的把他吓个半死。这绝对不是危言耸听,等你到了那一天,你就知道了!君子即使在身体好时,他也不会放纵,他非常小心谨慎,因为他知道千万不可放纵,他已经看到了整个过程,从撸到废的整个过程,全部看清了,所以他极为小心谨慎。

\begin{quote}\it
    胜私制欲之功,有曰识不早、力不易者,有曰识得破、忍不过者。盖识是一颗照魔的明珠,力是一把斩魔的慧剑,两不可少也。
\end{quote}

\subparagraph{解析} 这一条也讲念头实战,关键就是“识得早、识得破”,然后还要“断得快”,要具备这种强悍的“断力”。我之前的文章也讲过“识别 - 断除”,首先你要学会识别心魔的进攻方式和套路,知己知彼百战不殆,你必须很深刻全面地了解你的对手,也就是心魔。早识别,早断除,这两个早一定要做到,现在做不到,也要通过练习让自己尽快做到,这直接关系到念头实战的成败!很多新人不会识别心魔,认贼为己,还以为是自己的想法,于是跟随强化邪念,最后欲火中烧,不得不破。在识别的环节,很多人都出了大问题,识别不出,就会导致敌我不分。通过不断学习戒色文章,你就会慢慢了解心魔的各种进攻方式和套路,这样心里就有底了,什么念头上来,是正是邪,你心里都清清楚楚了。有些人识别能力尚可,他能够识别出心魔,但是“断力”比较欠缺,还需要加强练习观心断念,“力是一把斩魔的慧剑”,断力是通过一次次练习练出来的,平时就可以拿普通的念头来练,看看自己是否能立刻斩断它、消灭它。真正的断念高手基本都练习了几万次,从业余断念水平练到了断念九段,在断念道场日复一日地练习,就像练一门绝世武功一样,拼命练,最后练到了出神入化的程度,到时对付心魔就有十足的把握了。不少人戒到一定程度会害怕自己再次破戒,再次掉入那个怪圈,他之所以害怕,因为他缺少十足的把握,戒色高手有十足的把握,所以他不怕,他已经胜券在握了。

\begin{quote}\it
    德者事业之基,未有基不固而栋宇坚久者;心者修行之根,未有根不植而枝叶荣茂者。
\end{quote}

\subparagraph{解析} 厚德载物,德是基础,要夯实这个基础,房子能造多高和地基的稳固程度是密切相关的。戒色必须注重提升自己的道德修养,之前有的戒友就是不注重德的培养,结果戒到一定天数,他骄傲了,结果就破戒了,究其原因,就是谦德没修好,谦德可以有效对治骄傲自满的心理。骄兵必败,必须要注重谦德的培养,不能骄傲。练武之人会把武德放在很重要的位置,我看过一本武术的书籍,那位练武高人说,以德服人才是最厉害的。德是最高的武功,不战而屈人之兵,崇高的品德让别人心服口服。如果你用武力把对方打败了,对方很可能怀恨在心,这就停留在争勇斗狠的低级层面,要进入武术的最高境界,必须要注重修德。我们戒色也一样,只有你的德跟上了,才会让你进入更高的戒色境界,小戒靠忍,大戒靠悟,至戒靠德!德建名立,行端表正,积善成德,德厚流光。古有明训,传子以金不如传子以德,这份遗产,丰厚之至!这条语录的后半段讲到了修心,心是修行的根本,修心也就是修念头,要懂得控制自己的念头,修行不能停留在表面,有些戒友看到抄经好,就去抄经,看到放生好就去放生,抄经和放生都很好,但是你自己一定要懂得修心啊,不能停留在表面。

\begin{quote}\it
    色欲火炽,而一念及病时,便兴似寒灰。
\end{quote}

\subparagraph{解析} 这条也是讲念头实战,不过属于思维对治,这里思维的是病,想到那些痛苦的病,邪淫的兴致就下去了,这种思维对治平时就要极其熟练,否则临阵磨枪,这样效果往往不太理想。你平时就要深入地思维邪淫所导致的各种疾病,要有很深刻的认识,自己也要多思维症状爆发时的痛苦。曾国藩也用思维对治,他说:“欲心一萌,当思礼义以胜之。”狄仁杰用的是不净观和白骨观,也属于思维对治。思维对治是有很大效果的,但是平时就要多思维,产生高度的理解和认同,这样念头实战时才能发挥出威力。这条提到了“色欲火炽”,火势炽盛时其实已经晚了,最好的时机是小火星时就灭掉它,到了火炽时,断念的难度就比较大了。我都是小火星时就灭掉了,我不给它发展壮大的机会,古德特别强调念起即断,念起时的零点几秒就是断念的最佳时机,一定要懂得把握这一黄金时机。

\begin{quote}\it
    猛兽易伏,人心难降。溪壑易填,人心难满。
\end{quote}

\subparagraph{解析} 这条让我想到了王阳明先生的“破山中贼易,破心中贼难。”两者有异曲同工之妙。魔有内魔,有外魔,外魔易退,内魔难降,如不能降,必要着魔,降伏心魔,外魔自败。最关键的还是要降伏自己的心魔,内不动心,一切外在的诱惑都无法动摇你。武松打虎,算是极猛了,但是降伏猛兽还不算厉害,最厉害的就是降伏心魔,心魔才是真正的怪兽,比哥斯拉还哥斯拉,它就隐藏在你的内心深处,它一直企图操控你。很多人都没意识到这个怪兽的存在,因为他们不知道什么是心魔,但是他们知道内心时不时会冒出可怕的邪念,其实这就是心魔的表现。一直看黄手淫,就是在喂养这头怪兽,最后这头怪兽就会越来越变态,你内心邪恶的势力会主导你的人生走向,会摧毁所有一切的美好与希望,最终让你活在彻底的绝望之中。降伏了心魔,才有平安可言,降伏心魔的人才是真正的王者,“猛兽易伏,人心难降”,难降能降,才显王者风范!

\begin{quote}\it
    人生祸区福境,皆念想造成。故释氏云:利欲炽然,即是火坑。贪爱沉溺,便为苦海。一念清净,烈焰成池。一念惊觉,航登彼岸。念头稍异,境界顿殊。可不慎哉!绳锯材断,水滴石穿,学道者须要努索;水到渠成,瓜熟蒂落,得道者一任天机。
\end{quote}

\subparagraph{解析} 一念天堂,一念地狱,一念佛,一念魔。一切都在念头上分了高下,你起邪念,疯狂邪淫,最后其实是害了自己,也连累了自己的家人。佛法方面有讲到:多欲为苦,欲为苦本。凡夫以苦为乐,彻底搞颠倒了。凡夫迷于快感而无知于放纵欲望带来的巨大过患,等到报应现前,才知道邪淫的危害是如此巨大,有的善根比较差的,至死都不悔悟,真是太可悲了。戒色不是叫你做和尚,而是让你学会控制自己的欲望,学会控制自己的念头。“念头稍异,境界顿殊。可不慎哉!”有时候起了一个邪念,就会感觉不对劲,沉迷意淫后,更是感觉不对劲,所谓境随心转,心里起邪念,自己所感受到的境界就会朝着不好的方面发展,邪念会感召吸引负面的事物,走路都可能踩到狗屎,平地都可能跌倒。最近看了一个帖子,那个人上午手淫,下午就摔伤了,你觉得很邪乎,其实一点不邪乎,手淫完了身体和精神都不在状态,注意力不集中,腿也软掉了,这样就容易摔倒,有的还骨折了,开车也容易出车祸,开车时注意力不集中,反应迟钝,然后就撞了。邪淫的确会影响运势,因为你整个人都不对劲了,还谈什么运势?有的人面试了十几家,都不成功,为什么?别人看他那猥琐样,就觉得不舒服,而且脑力下降后往往答非所问,面试的表现实在太差。戒色就是要学会修心,主宰你的每一个起心动念,邪念肯定会入侵,邪念肯定会闯入,最近还看了一个网上的视频,给了我很大的启发,美国乔治亚州一华裔女子,面对入室的盗贼,以一敌三击毙一人,这位华裔女子面对持枪的盗贼果断开枪,真的很彪悍!上来就是几枪,打得盗贼抱头鼠窜,相信很多戒友都看过这个视频。我们戒色其实也在面对同样的事情,只不过是发生在你的头脑里,邪念突然闯入了你的头脑,你必须立刻击毙闯入的邪念,必须快准狠!必须果断!不可认同邪念,不可跟随邪念!必须拿出你的狠!邪念入侵,Fire!!!

\begin{quote}\it
    宠辱不惊,闲看庭前花开花落;去留无意,漫随天外云卷云舒。
\end{quote}

\subparagraph{解析} 这条语录可能是《菜根谭》里最美、最具有诗意的句子,一份淡定、一份超然,一份从容,一份豁达,一种看破放下的心境,一种超脱欲望束缚的自由。在戒邪淫后,你会再次感受到大自然的美好与和谐,就像恢复了嗅觉一样,你能再次体会到那种芬芳与美感,那种感觉真的是太美好了。因为沉迷邪淫,所以你就丧失了对这种美好的感知,只有不断净化自己的心灵,才能再度被这种美好所深深震撼与感动。纯净的孩子就活在这种极致而纯粹的美好里,孩子睁着亮晶晶的大眼睛打量着这个世界,仅仅是观察而没有任何抓取的欲望,在这种纯粹的观察里就能感受到极致的美好与和谐,孩子的那份纯净是最宝贵的,这是多少钱都买不来的,最高的享受就是享受纯净美好的感觉。你本是纯净的小孩,只是后来被邪淫污染了,邪淫的你并不是真的快乐,不管你看了多少黄,不管你撸了多少次,你都不会真正快乐起来,你感受到的只是短暂的快感,而快感过后就是空虚乃至痛苦。真正的大快乐,不需要看黄,也不需要手淫,记住,你本是快乐的,是邪淫污染了你,让你变得不快乐。当你不撸了,你就会渐渐懂得快乐的真义。

\begin{quote}\it
    一念慈祥,可以酝酿两间和气;寸心洁白,可以昭垂百代清芬。
\end{quote}

\subparagraph{解析} 这条语录也是我个人比较喜欢的,《道德经》中老子所言的第一宝就为慈,慈祥、慈悲、仁慈,没有伤害别人的想法与念头,只有无私利他的愿望与动机。《太上感应篇》也说“慈心于物。”慈者万善之本,即仁心也。慈祥是一种很高的境界,一个慈祥和蔼的老人往往具有很强的亲和力,散发出来的磁场非常之好。一行禅师说:“只有慈悲不可战胜。”我觉得他说得极好,慈悲的能量是最强大的,振动频率极高。心怀慈悲之人会在他的周围建立起一个超强的能量场,自然会感召吸引正面美好的事物。慈祥的人看上去一团和气,与世无争,非常和谐的感觉,让人如沐春风。《壶天录》:“和气致祥,乖气致戾,处家固然也,即涉世亦何不莫然!”大家都知道以和为贵,家庭和睦非常重要,待人也应该要和气,一个家庭中如果有一个人在犯邪淫,这个家庭就容易失和,如果父子都在犯邪淫,那这个家庭就可能严重失和,会出现经常吵架乃至离婚等事件,还容易出一些意外事故,都是能量场不和谐感召的。秦东魁老师讲的犯邪淫和亏孝这两个问题在现代社会中特别突出,尤其是犯邪淫,犯邪淫其实就是大不孝,一直犯邪淫,后果真的很严重。“寸心洁白”讲的就是保持纯净的心灵,古圣先贤有各种表述方式,但都是指向一个目的,那就是净化你的心灵,你的心灵本来是纯洁的圣殿,而现在堆满了黄毒的垃圾,散发着恶臭。当心地恢复干净了,自然就会散发清香,一种特别美好的感觉也会随之而来,守身如莲,香远益清,本真妙韵,和润仁心。

最后总结:

这季分享了《菜根谭》的二十条语录解析,《菜根谭》的确是一本很有品味、很有深度的书,对人生的各个方面都很有启示,值得反复品读,当初我读《菜根谭》,是被它的文字意境所吸引,后来又被念头实战的语录所吸引,从中获益很多。《菜根谭》的其他一些语录也是很不错的,因为篇幅限制,我只精选了这二十条,主要侧重于从戒色方面进行解读。先哲说过:“人的灵魂来自一个完美的家园,那里没有我们这个世界上任何的污秽和丑陋,只有纯净和美丽。”曾经我们都是纯净美丽的孩子,来自于完美的家园,后来在这个尘世里开始堕落,纯净不再,美丽不再,渐渐沦为了邪淫无耻之徒,从纯净的孩子撸成了猥琐大叔,看着那张扭曲变形的脸,真的无语泪先流。心为神灵之台,庄子云,万恶不可内于灵台。让我们戒除手淫恶习,回到那个纯净美好的家园。

下面分享三首戒色诗歌。

\begin{poem}[记得你最纯真的样子]
    \begin{multicols}{2}
        \begin{center}~\\
            再次体验到了 \\ 纯净的大快乐 \\ 泪水在他脸颊上 \\ 肆意地流淌 \\ 这一刻等了太久 \\ 撸囚终于出狱了 \\ 掉进撸坑十年了 \\ 就像坐了十年的牢 \\ 已然颓废了太多太多 \\ 拿出曾经的照片 \\ 那个纯净的孩子 \\ 笑得那么灿烂 \\ 过去的自己 \\ 是那么的美好快乐 \\ 渐渐长大后 \\ 开始邪淫放纵 \\ 越来越偏离 \\ 当初那个纯真的自己 \\ 满脑子的邪念 \\ 活在了惶恐之中 \\ 记得你最纯真的样子 \\ 记得你爱笑的眼睛 \\ 请戒掉手淫恶习 \\ 请做回最纯真的自己 \\ 恢复干净清澈的灵魂
        \end{center}
    \end{multicols}
\end{poem}

\begin{poem}[戒色喜峰口]
    \begin{multicols}{2}
        \begin{center}~\\
            戒色大刀队 \\ 挥舞着大刀砍向心魔 \\ 坚决抗击心魔的入侵 \\ 揭开戒色逆袭战的序幕 \\ 与邪念白刃相接 \\ 勇士们身背闪亮的大刀 \\ 背影是那样决绝 \\ 断念实战中 \\ 抡起大刀奋力劈杀 \\ 刀光闪耀在敌群中 \\ 手起刀落处 \\ 瞬时倒下一大片 \\ 心魔遭遇了最顽强的抵抗 \\ 挥舞大刀向心魔头上砍去 \\ 火山爆发般的强大力量 \\ 不顾一切地勇猛冲杀 \\ 身体就是你的国 \\ 不做亡国奴 \\ 英勇的大刀队 \\ 血战的悲壮 \\ 百万戒色将士 \\ 惊天地、泣鬼神 \\ 荡气回肠的史诗对决 \\ 拼死抵抗,绝不让心魔得逞 \\ 坚决守住戒色阵地 \\ 誓与阵地共存亡 \\ 大刀大刀,雪舞风飘 \\ 杀敌头颅,壮我英豪
        \end{center}
    \end{multicols}
\end{poem}

\begin{poem}[不撸的传奇]
    \begin{multicols}{2}
        \begin{center}~\\
            猥琐灰暗的躯壳 \\ 正在迸裂 \\ 里面透出了万丈光芒 \\ 彻底冲破了围绕身体四周的黑雾 \\ 所有的丑恶和污秽全部消散 \\ 你重生了 \\ 一个纯净美好的笑容 \\ 在你脸上绽放开来 \\ 内心感受着不断爆炸的新鲜喜悦 \\ 那种爆炸实在太过强烈 \\ 使你不得不开心,不得不快乐 \\ 没有任何理由的开心 \\ 没有任何原因的快乐 \\ 你体验到了某种极乐的感觉 \\ 仿佛你是世界上最幸福最快乐的人 \\ 在密集的内在爆炸后 \\ 你渐渐平静下来 \\ 另外一种微妙的愉悦开始洋溢开来 \\ 一种特别美好的氛围 \\ 将你整个笼罩 \\ 一种完美和谐的感觉 \\ 开始浮上心头 \\ 这种感觉太久违了 \\ 当你还是一个孩子时 \\ 你就活在这种纯粹美好的感觉中 \\ 像孩子一样纯真无邪 \\ 像孩子一样开心快乐 \\ 你就是不撸的传奇!
        \end{center}
    \end{multicols}
\end{poem}

下面推荐一本书。

\begin{book}[《直指法身》]
    2009 年菩提文化出版社出版的图书,网上有卖,我读的是 PDF 版本,做了 145 条笔记。这本书的作者是第九世大宝法王(1556 - 1603),创古仁波切给予的解读和开示。这是一本重量级的开示,对修心有非常深入系统的阐述。所谓法身,指的就是佛性、觉性、妙明真心,也就是纯粹的觉知,名字有无数个,但是指向都是同一个。能够读到这本书,对我来说也是一种莫大的幸运,这本书深化了我对观心的理解,这本书里面特别强调了“直观自心”,我们总是向外看,从来没有好好向内看,当我们真正向内看了,一种真正的转化才会到来。这本书里面讲到:“念头生起时,一观看就消失了。”这句话可以说是念头实战的核心要旨,也就是:念起即觉,觉之即无。当你一次次返观自心,你的觉察力就会越来越强,你的领悟也会越来越深,到时自然可以降伏心魔。戒色是需要相当悟性的,当你真正悟进去了,真正理解前辈所强调的那些重点,到时你的实战表现就会大幅提升。这本书是关于大手印的教法,元音老人也讲过恒河大手印,大手印和禅宗如出一辙,在最高层面都是一致的,都在强调观心,通过观心来降伏心魔。经云:“能观心者,究竟解脱,不能观者,究竟沉沦。”当你真正悟明白了,你会发现一切都是那么的简单,如果悟不明白,那就很难降伏心魔了,因为心魔的实力高出你一大截,你怎么弄都弄不过心魔。一般都需经过好几次的顿悟才能真正明白前辈到底在说什么,顿悟是需要前期积累的,就像达到沸点前有一个持续加热的过程,不断坚持学习戒色文章,当积累到一定程度,突然就顿悟了,突然就开窍了,突然就明白了,明白之后再不断强化练习,实战表现就会变得更加出色。这本书相对比较系统,是一个完整的教学,讲述的重点也比较多,希望有缘的戒友可以读到它。戒色和修道是完全相通的,当你进入了修道的层次,你才能达到更高的境界。直观你的自心,把精髓的教导付诸于实修!
\end{book}
