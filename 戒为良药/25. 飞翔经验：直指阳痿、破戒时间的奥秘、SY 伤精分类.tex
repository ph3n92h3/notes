\subsection{直指阳痿、破戒时间的奥秘、SY 伤精分类}\label{25}

\paragraph*{前言}

戒色吧现在很多戒友都有了学习意识,不再一味地强戒,开始步入了戒色正轨,戒得越来越专业了,这是非常好的现象,一旦养成每天学习戒色文章的习惯,看进去了,真正学进去了,戒色觉悟和戒色意识就会突飞猛进。这个道理其实和打网游一样,你每天打网游,你的等级很快就能上去,如果你不修炼自己的等级,就不会有进步,始终是菜鸟。在游戏方面,就是被对手虐,在戒色方面,就是被心魔虐,因为心魔等级比你高,你等级太低,见一次失败一次,所以,必须通过不断学习提高觉悟等级,觉悟和意识有了,心魔就很难动你了。

我希望戒色吧有更多的戒友分享自己的经验,新人需要前辈的经验指导,新人最大的特点就是思想误区多,很多问题认识不清,而经验帖可以帮助他们更好地提高觉悟,把他们引导到一条正确的戒色道路上,坚定他们的信心和决心,避免走更多的弯路,从而提高戒色的成功率。

研究 SY 行为好几年,我发现成功的戒色模式只有一种:那就是通过不断学习提高觉悟。觉悟修到了,自然就戒掉了,等你觉悟非常高了,只要保持警惕,就可以彻底戒掉。而修炼觉悟就像爬山一样,很多人爬到一半,就不想爬了,就下去了,很多人虽然爬得慢,但一直坚持爬,一直在坚持学习戒色文章,不断提高觉悟,虽然进步缓慢,但一直在不断提高,这样他最终依然可以达到觉悟的顶峰,到时再强化一下警惕意识,就可以做到彻底戒掉。

我能有现在的觉悟,也是这样一步步上来的,我当初学习戒色文章是这样的:每天摘抄好的戒色句子,当然也包括养生类的内容,因为养生和戒色是相通的。我那时一年多时间摘抄了六大本,我对自己的要求是,每天进步一点点,这样一个月后,就是一个大进步。从开始摘抄戒色文章后,我的觉悟就有了飞升,很多以前没搞懂的问题一下全想明白了,头脑越来越清晰。后来我又看了不少中医医案的书籍,觉悟又有了进一步的提高,现在每天我都会抽空看看中医和佛法,还在不断提高自己的思想境界,学无止境,提高也是无止境的。

很多戒友会说自己看不进去,一看到长篇大论就烦,看一行就不想看了。这种戒友其实很多,因为肾精一亏,人的脑力就会下降,记忆力和注意力乃至理解力都会下降,更糟糕的是,纵欲后人的情绪会变得急躁易怒,出现“烦相”,做事学习都没有足够的定力和耐心。这种状态我以前也经历过,那时我看到满屏幕的字就烦,不想看。后来戒色一段时间,脑力有所恢复后,心又能重新静下来了,慢慢就看进去了,看进去后,就是越看越想看,并且把自己认为好的句子摘抄下来,不知不觉我的觉悟就有了飞升。把看过的文章和摘抄的句子时常看看,又能做到温故而知新,这样觉悟又会有提高。觉悟提高到一定程度,你就会发现,你对黄毒有免疫力了,而且能降伏心魔了,这其实就是觉悟提高的结果。

对于每天的学习量,我是这样安排的,根据自己的状态来定,如果这天我精神状态好,兴趣比较浓,就多看点,有时是十几页,有时是一章,有时兴趣浓,可以看大半本书。而有时我状态不是很好,兴趣匮乏,这种情况我安排的学习量,就是每天至少一页,看一页只需要几分钟,即使再没耐心,忍一下也能看完,这样我就能做到每天学习不间断,这样我的觉悟就在不断提高,最终就能迎来觉悟的大飞升。我建议各位戒友,每天尽量看一篇戒色文章,不一定要新的文章,温故知新也很好,不要离开戒色文章,你也可以把自己摘抄的内容反复看,这样你觉悟提高也会很快的。

有戒友会说,我喜欢看图片,不喜欢看长篇的文字,因为看不进去,烦!我的建议就是等你戒色一段时间,脑力有所恢复后,就要让自己学着看文字了,因为我做的图片只是戒色的辅助,最关键的还是文章,要专业系统地提高觉悟,必须看大量的文章,图片只是一个助力,文章是主,图片是次,这点一定要明确。

还是那句话:戒色神力来自学习!不学习不知道,不学习觉悟没提高。觉悟的持续提高,最终会让你战胜 SY,降伏心魔。

下面步入正题。这季就阳痿问题、破戒时间的奥秘、SY 伤精分类三个方面展开详细的论述,具体如下。

\subsubsection{阳痿问题}

我 \ref{6} 的文章讲到过阳痿早泄的问题,现在 \ref{25} 了,可以写得再详细全面些。

我回答了这么多的问题,关于早泄、勃起不坚、阳痿的占了不少,伤精伤到一定程度,就会出现性功能障碍,我最早出现的是早泄,大概在频繁 SY 后两年就出现了,然后就有了勃起不坚,勃起不坚的现象大概持续了十几年,因为那时我热爱运动,作息饮食还算规律,无网瘾不抽烟,所以十几年都还能勃起,就是硬度不行。后来开始熬夜久坐后,阳痿就出现了,我那时阳痿还是很严重的,费了九牛二虎之力,并且要强刺激,才能勉强勃起,而且硬度也不行,很快就软下去了,后劲明显不足。

出现早泄和阳痿,男人就会很恐慌,因为性功能对于男人太重要了,应该说恐慌程度和脱发的恐慌程度不相上下,性功能不行在女人那里就会抬不起头,没自信,还会担心老婆出轨给自己戴绿帽子,所以早泄阳痿的男人,心理压力也非常大,焦虑情绪很严重。

最糟糕的是,很多人没有中医养生常识,一出现早泄阳痿,第一反应就是要吃要补,要壮阳,让自己重新坚挺,重新雄起,千万不能让自己软下去,在这种思想指导下,就是到处吃补药。有些补药刚开始吃效果还行,但补药也是会吃疲掉的,经常吃效果就不明显了,到时候就要换药吃,最后的结果就是吃遍很多名贵的补肾壮阳药,身体还是不行,并且很有可能,身体除了阳痿,还会出现很多其他症状,苦不堪言。其实,这类人因为没中医养生常识,从一开始就误读了身体的信号,古语有云:为人子弟不可不知医。如果你不懂医理,在很多方面的认识和选择上就会步入误区,最终害了自己。一个男人阳痿了,这个身体信号其实就是身体在求救,意思叫你不能再放纵自己了,原理类似于毛孔受寒自动关闭一样,很多人不懂这个道理,阳痿了反而吃补药放纵自己,根本就没有戒色养生意识,他们不懂最好的补药其实就是:不泄为补。因为无知,他们选择了错误的道路,靠补药来更好地纵欲,结果可想而知,将来真的有可能彻底阳痿,甚至会染上糖尿病或者中风。

因为肾气不足,身体选择阳痿来自保,你误读身体信号,反而靠吃药纵欲,必然招致恶果,这就是无知的代价。

戒友一旦出现早泄、勃起不坚、阳痿倾向,就要学会戒色养生了,如果症状严重建议去看中医调理,吃中药对身体恢复有利,但一定要注意修心,否则中药吃下去很有可能会助长欲望,结果就是一边吃中药一边纵欲,这种上补下漏的行为,对身体的恢复极其不利。如果你不能做到彻底戒色,而是边吃边漏,我敢肯定你的治疗效果不理想,没多久你就会发现自己又早泄阳痿了。这样就陷入了一种困境,其实就是他自己思想认识上有误区,如果这种思想误区得不到纠正,结果就是花了很多钱,吃了很多药,早泄阳痿还是依旧。所以要治好一个人的早泄阳痿,必须先让他开悟,给他讲道理,让他明白道理,开悟其实就能治病,不开悟,他这辈子就废了。不少戒友结婚前就把自己玩废了,出现了早泄阳痿,家里又催着结婚,这样搞得他很害怕结婚,早知今日何必当初。

不过大家也不用绝望,早泄,勃起不坚,阳痿都是可以恢复的,我就完全恢复了,因为我认识到位,思想认识上没有误区,养生功夫做得很好。最关键一点,我不会去试,因为一试就又掉进 SY 陷阱里去了。大家肯定会问,那怎么知道恢复了呢?其实是否恢复从晨勃的质量是完全可以看出来的,晨勃坚挺持久,硬度强,这都是身体恢复的表现。如果你去试,就非常有可能一而再地破戒,前功尽弃。很多人都喜欢试,结果就是一试无法自拔,又重新开始纵欲。

根据我的经验,早泄和勃起不坚的恢复时间大概需要半年以上,阳痿则需要一年以上的戒色养生,要恢复是各方面都要做好,最关键的前提就是严格控制遗精次数和杜绝 YY,这两点做到后,再在养生上下功夫,这样性功能是可以慢慢恢复的。我是过来人,只要你有信心和恒心,不断学习提高觉悟,做到彻底戒色,提高养生意识,性功能完全可以恢复。就怕你步入思想误区,那就很难恢复了。早泄阳痿都分轻度、中度和重度,如果是轻度的,恢复时间相对会快些,如果是重度的,那恢复时间会更长,恢复难度也更大,但只要坚持戒色养生,各方面做到位,依然是可以恢复的。

另外,有人是紧张性的早泄阳痿,自己 SY 没事,这种情况西医会归咎为心理问题,其实紧张心理的出现和肾气的亏损是密不可分的,因为身心是合一的,因为肾气的亏损,人的情绪心理都会出现相应的变化,有的人原来不紧张的,SY 后变得紧张烦躁易怒,这其实就是肾精亏损的表现,甚至小便都紧张,无人时才可以尿出。这类紧张性的障碍,通过坚持戒色养生也是可以恢复的,肾气养足,身心都会恢复到正常状态。

\subsubsection{破戒时间的奥秘}

下面谈下破戒时间的奥秘。

这部分知识很少有人知道,一方面要熟谙中医医理,另一方面要有咨询反馈的案例,有了深入研究后才能体会到这部分知识。

在我咨询的案例中,有好几位戒友都反映过,为何每次破戒身体恢复的速度都不一样,有时破戒后几天就能恢复,有时破戒后一个月甚至几个月还有症状,为什么会这样?要知道这个问题的答案,就要了解人体的阳气水平,人体的阳气在每天的不同时间段都不一样,在每个月的不同时间段又不一样,在每个季节又不一样,如果你在阳气水平低的时候放纵,结果就比较难恢复,这个道理其实很好明白,比如你在手机费很少时打电话,很可能打一个电话你就欠费停机了,但在你还有很多话费时打电话,就没事。

人体的阳气水平也是在不断变化着,忽高忽低,总的变化规律如下。

\paragraph{每天的变化}

子时一阳生,午时一阴生。故养生之道在子时不行房事,以免戕伐阳气。因为子时阳气刚刚生发起来,好比一个小火星,此时行房事就会很伤阳气,更容易出症状。阳气从子时生发起来,然后逐渐壮大,至午时盛极而衰,然后午时一阴生,然后阴气逐渐壮大,至 23 点子时盛极而衰,然后子时一阳生,这就是一个循环。而锻炼最好选择在上午,因为上午阳气处于上升阶段,这时候锻炼就是顺应天时,效果会更好,一般晚上入夜适合静养,因为晚上阴气重,这时候锻炼出汗,也很伤阳气。这个道理大家懂了以后,就可以更好地选择锻炼的时机。

古代十二个时辰:

\begin{multicols}{3}
    \begin{description}
        \item[\xpinyin{子}{zi3}] 晚上 11 时正至凌晨 1 时正
        \item[\xpinyin*{丑}] 凌晨 1 时正至凌晨 3 时正
        \item[\xpinyin*{寅}] 凌晨 3 时正至早上 5 时正
        \item[\xpinyin*{卯}] 早上 5 时正至早上 7 时正
        \item[\xpinyin*{辰}] 早上 7 时正至上午 9 时正
        \item[\xpinyin*{巳}] 上午 9 时正至上午 11 时正
        \item[\xpinyin*{午}] 上午 11 时正至下午 1 时正
        \item[\xpinyin*{未}] 下午 1 时正至下午 3 时正
        \item[\xpinyin*{申}] 下午 3 时正至下午 5 时正
        \item[\xpinyin*{酉}] 下午 5 时正至晚上 7 时正
        \item[\xpinyin*{戌}] 晚上 7 时正至晚上 9 时正
        \item[\xpinyin*{亥}] 晚上 9 时正至晚上 11 时正
    \end{description}
\end{multicols}

\paragraph{每月的变化}

每个月的月亮都有阴晴圆缺,我们可以根据月亮的变化来判断一个月的阳气变化。月初之时,月牙微露,阳气开始渐渐生发释放,月相也慢慢由缺变圆,也就是上弦月,在上弦月慢慢变为满月的过程中,阳气是生发释放最为旺盛的时候,人体内的生命能量也最活跃,月满一过,重阳必阴,阳气逐渐地转入收藏状态,月相也渐渐由满变缺。到了二十二,二十三即成为下弦月。下弦月以后,月的亮区进一步缩小,直至三十,光亮皆无,只能看见月亮的影子,这个时候就叫晦。整个月象的变化,实际就是阳气变化的一个例证。所以,在阳气微弱或者阴气极重时,尽量不要进行性生活,否则是比较容易出症状的。

\paragraph{每年的变化}

夏至一阴生,冬至一阳生。

春生、夏长、秋收、冬藏。养生这个词,其实只是说了保养之道的四分之一,还有夏养长、秋养收、冬养藏。所以,冬天尽量不要过性生活,冬天是一个养精蓄锐为来年做准备的这么一个季节,如果在冬天放纵,来年拿出来生发的能量就少了,所以今年冬天你放纵了,明年你身体就很可能会出症状,特别不能在冬至日放纵,因为冬至一阳生,阳气正微弱,此时破戒更容易导致身体出症状,也很不利于恢复。冬藏精,藏好了,来年身体健康才有保障。

在古代专门有一门择日择时的学问,现在普及较少,很多年轻人都不知道,只有一些高人能真正了解那门学问,什么时候干什么事很重要,因为背后都有深刻的道理,戒友在不同季节破戒后,身体的恢复进度是不同的,因为每个季节人体的阳气水平都不同,你在阳气微弱时破戒,必然犯了忌讳,身体就较难恢复。如果你在阳气正旺时破戒,那还相对容易恢复些。

关于破戒时间的忌讳,相信很多戒友看过九毒日,九毒日,农历五月初五、初六、初七、十五、十六、十七以及二五、二六、二七,此九天为“天地交泰九毒日”。五月份有「九毒日」,为纵欲大忌!所以古代有习俗,五月让妇女回娘家住一个月,九毒日更要慎重。九毒日背后肯定有其道理,我们宁可信其有不可信其无,因为据我研究,破戒后的报应有延迟现象,有时并不是马上就会出症状,可能会在几十天以后,甚至上百天才出症状,因为这里面有一个蝴蝶效应,而蝴蝶效应发生的过程也是需要时间的。

我们要懂得忌讳,尽量避免犯忌,这样对于我们身体的健康才比较有利,知之则强,不知则老。如果你懂得避免忌讳,这样就可以避免很多危机。

这方面的详细内容,我建议大家可以参阅《寿康宝鉴》,里面专门有“保身立命戒期及天地人忌”的内容,懂得这方面的知识真的很重要。

\subsubsection{SY 伤精分类}

最后谈下 SY 伤精分类。

这个题目是一个戒友建议的,因为他症状比较多,也不知道自己伤到什么程度了,所以建议我能把伤精表现大致划分下,他自己也可以做到心中有数。如果你广泛地研究过伤精患者的案例,你一定会发现,很多人都是身兼多种伤精表现,身跨多种伤精分类,症状并不是单一的,以下是我对伤精表现的大致划分:

\begin{multicols}{2}
    \begin{description}
        \item[泌尿系统疾病] 前列腺炎、尿道炎、精索等
        \item[呼吸系统疾病] 肺部疾病,也包括鼻炎,哮喘等
        \item[脑力严重下降] 注意力、记忆力、理解力不同程度下降
        \item[结石类] 肾结石、尿路结石等
        \item[皮肤症状] 各类皮肤疾病。包括痤疮,荨麻疹等
        \item[头发问题] 以脱发和白发为主
        \item[消化系统疾病] 肠炎,便秘腹泻,消化吸收弱等
        \item[频繁遗精问题] 频繁遗精也是一种病
        \item[变丑问题] 变丑的太多
        \item[性功能障碍] 早泄阳痿,勃起不坚
        \item[心脏问题] 心悸、早搏等
        \item[腰痛腿软] 腰痛腰酸,腰膝酸软
        \item[颈椎问题] 颈椎病的各种表现
        \item[耳鸣问题] 肾虚导致的耳鸣
        \item[多汗多油] 这两多比较常见
        \item[视力问题] 视力下降,飞蚊症等
        \item[其他疾病] 大概有上百种,肾虚百病丛生
    \end{description}
\end{multicols}

很多症状即使不是 SY 直接导致的,也和 SY 密不可分,因为肾精是人体的健康货币。

1 - 17 类不涉及心理方面的疾病,虽然对戒友会产生困扰,但还不至于那么生不如死,当然这 17 类比较笼统,只能算大致分类。接下去主要是心理类的问题,如下:

\begin{multicols}{3}
    \begin{itemize}
        \item 焦虑症
        \item 抑郁症
        \item 神衰
        \item 植物神经紊乱
        \item 强迫症
        \item 偏执
        \item 自闭症
        \item 恐惧症(包括恐艾、社恐等)
        \item 疑病症
    \end{itemize}
\end{multicols}

当然,我这里讲的心理类问题,并不是纯心理的,是有一定躯体症状的,比如焦虑情绪并不等于焦虑症,焦虑症是有一系列躯体症状的。一旦染上这些问题,想死的非常多,去焦虑症群或者抑郁症群,经常有要死要活的,这些疾病的症状真是千奇百怪,无所不有,如果你不是这类患者,你和这类患者交谈,你也许会认为他说的是天方夜谭,但的确是这样,只有体验过才能理解患者的感受。我那时焦虑症神衰,天天感到绝望崩溃,多次有自杀的想法,因为那种状态简直就是生不如死,你会觉得死亡也许还好受些,就是那种感受,普通人是无法理解的。有的患者不敢出门不敢坐车,普通人就会觉得怎么那么胆小啊,不会吧。其实伤到一定程度,人的胆子真的会变得胆小如鼠,因为肾主恐,恐伤肾,肾虚到一定程度,人就会变得很胆小,和以前判若两人。我以前生病时就不敢出门,现在我恢复后,觉得不可思议,我那时怎么那么胆小呢?现在全明白了,就是肾精严重亏损,纵欲熬夜伤了神经。

所以,如果你是 1 - 17 类,那还算好,你不会那么想死,如果你还有严重的心理类疾病,问题就比较严重了,要恢复就更慢了,难度也更大,很多焦虑症患者五年甚至十几年都出不来,常年靠药维持着。其实焦虑症等心理疾病,是必须开悟才能出来的,靠药不行,药能缓解但无法真正治愈,必须要学会养生之道,必须懂得珍惜肾精,否则即使你暂时靠药控制住了,依然会很容易复发,因为你没认识到发病的真正原因。

\paragraph*{结语}

偶尔有戒友会提问 SY 是否真的会导致运气的变差,表面看,运气和 SY 没啥必然关系,其实是有一定关系的,因为 SY 会导致脑力下降,脑力正常时,很多事情可以做得很好,而脑力下降时,很多本来可以做好的事情就做不好了,表面上看,好像自己运气变差了。比如一次考试,有的人会说,我当时怎么那么健忘啊,我当时怎么没想到啊,我运气也太差了,其实这种运气差的根源就在于脑力的下降,因为脑力下降,所以你不在状态,这时候人的运气自然就变差了。

再比如面试时,给面试官的第一印象是非常重要的,而一个沉迷 SY 的人眼睛无神,气色像鬼一样,甚至有黑眼圈和眼袋,一副衰样,没自信很自卑,这样的人去面试,成功率自然不会高,不少人面试了十几家,一次都没成功,就怪自己运气差,其实就是 SY 导致的变丑,精气神不行,如果你精神饱满,容光焕发,自信满满,这样给面试官的印象就会完全不同,你这种饱满的精神状态是可以感染到面试官的,他会觉得你很有朝气,而公司正是需要这样有朝气有活力的年轻人。
