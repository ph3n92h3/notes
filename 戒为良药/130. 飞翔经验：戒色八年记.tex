\subsection{戒色八年记}

\setcounter{subsubsection}{-1}
\subsubsection{引子}

我戒色八年多了,一次未破!不含任何水分,真正意义上的一次未破,货真价实的一次未破,坚持写戒色文章和答疑,也坚持了八年多,每年基本全勤,我牺牲了很多,但我心甘情愿,无怨无悔。八年了,这股帮助戒友的热血,仍未冷,当初完成第一个答疑的热情还在,初心还在,我还是尽力做好每一个答疑,帮助每一位需要帮助的戒友。我一直保持纯粹的态度来做这件事,未曾收过戒友一分钱,这点我问心无愧,曾经有位戒友要给我一些钱,因为我帮他做了很多答疑,他有点过意不去,但我还是婉拒了他的好意,我希望自己能够保持纯粹,不求任何回报,不收一分钱,以绝对纯粹的态度来帮助大家,我容不得一丁点的名利追求掺杂在里面。我相信真正的力量来自于纯粹,能做到纯粹会带来极大的感染力和感召力,纯粹者才能真正崇高无私!对于我而言,这是荡气回肠的八年,感慨万千的八年,在我陷入痛苦的症状地狱,生不如死,人生进入读秒时间和至暗时刻时,我奋起戒色,绝杀了心魔,完成了人生的蜕变和逆袭。八年抗战,一路走来实为不易,一年一年地坚持,不知不觉已经八年多了,无比感谢广大戒友的支持与鼓励,如果不是你们的支持,我是不会坚持到现在的,也无比感恩大家给我这样的宝贵机会来为大家做无私服务。虽然我们未曾谋面,但这种志同道合的兄弟情义是绝对感人肺腑的,在这样一个色情泛滥的时代,还有这样一群人在坚守内心的纯净,这是很难得的。

回首凝望这段戒色的生命历程,记录着一个生命对另一个生命的启示,也见证着一个又一个生命的蜕变,戒色是一部真正的英雄史诗,征服全天下全宇宙,不如征服自己的心魔,你只有主宰自己的内心,才能真正主宰自己的人生!即使全世界都处于黑暗之中,即使所有人都对你嗤之以鼻,也不要让自己心中的戒色之光熄灭,坚定信念,正己化人,恒久力行,微弱的光芒迟早会爆炸整个黑夜。让辉煌的光芒照亮生命之路,照亮身边的人,照亮整个世界,神州尚有壮士在,浩然正气战邪淫!这股传承了几千年的刚正之气依然没有失传,依然活在每个真正的戒者身上!不管这个世界的色情多么泛滥,我们也有权选择做纯净的自己,拒绝色情的诱惑,坚守到生命的最后一刻。戒色吧是黄毒浊世的一股清流,戒色灯塔照亮了万千沉沦欲海的苍生,为他们指引了净化重生之路。

一位戒友评价《戒为良药》:“这是一本帮助你戒除色情、提高觉悟、重回真我的书,试问哪有比这样的金玉良言更珍贵?文章写得很走心,像是智者劝诫少年的肺腑之言。作者根据自己的亲身经历以及海量研究,总结出了科学、系统的戒色知识,并用流畅的语言表达出来,阅读就好像在跟作者对话,读起来毫不费劲。我很少花时间写评价,这好像是第一次,就是因为受作者的善行所打动,在此向作者飞翔致敬!”我也向广大戒友致敬,你们的支持是我最大的动力,我从内心深深感恩大家,我愿每一位戒友都超过我,希望你们的人生幸福美满。看到大家逆袭是我最开心的事情,我愿意把自己放在最低的位置来成就大家,看到你们再次绽放出纯真无邪的笑脸,我由衷感到幸福,看到你们的眼睛再次变得明亮、清澈、喜悦而富有灵气,我由衷感到欣慰。我希望大家一起来传递这股正能量,让更多的人认识邪淫的危害,认识戒色自律的重要性,这股正能量正在传遍华夏大地,越来越多的人已经认识到了邪淫的危害,并且自觉戒除恶习。在色情泛滥的时代,我们强势崛起了,我们发出了时代最强音:我们拒绝邪淫!我们要做正能量的自己!!!

戒色八年多,我会对自己的邪淫经历和蜕变过程做一个大致的回顾与梳理,也会把戒色重点写入其中。这是一篇半自传体的文章,既理性也感性,我会把自己独特的感受和理解分享给大家。戒色八年多的成绩,对于戒色新人来说是一个难以企及的高度,就是对于资深戒友来说,也是一个非常高的成就,甚至可以用难以置信来形容。如果把时光拉回到十几年前,我也不相信自己能戒掉,因为那时我最高的戒色天数为 28 天,就是一味靠毅力强戒,根本不学习戒色文章,那时转移注意力、充实生活、积极锻炼、早睡早起,这四项我都做到了,但是快到一个月时,心魔一来,我就破了。戒过很多次都无法突破一个月,有时连一周都无法突破,一到周末就疯狂破戒,我那时觉得手淫恶习非常难戒掉,那个怪圈实在太强大了,就像黑洞一样把我吸进去,我甚至不相信这个世界上有谁能戒掉,在那时我觉得这是不可能的事情。而现在我做到了,并且已经彻底戒掉八年多,一次未破,真的是恍如隔世。大家看了我之前的文章肯定已经知道我是怎么戒掉的,其实掌握了方法和原理,每个人都是有望戒掉的,戒到一定程度,天数就不再重要了,到那时你不再关注天数,而天数却在突飞猛进,一晃几年就过去了。

好了,我要开始回顾了,希望大家认真阅读或聆听,我讲述的经历一定会引起大家深深的共鸣,让我们一起开始吧。

\subsubsection{那个磨床的少年}

我还记得我第一次磨床大概是在十三四岁的年纪,就是刚开始发育那会,个子还不是很高,大概是某天下午,鬼使神差就做了那个举动,从此一发不可收拾,很快就发展成了手淫恶习。那种舒爽的感觉让我很难拒绝,让我欲罢不能,让我深深沉迷其中,不能自拔。在手淫时我似乎忘记了世界上的一切,忘记了所有烦恼,忘记了学业的压力,但那个过程是短暂的,当一切结束后我必须回到残酷的现实中,之后就开始陷入后悔和自责,觉得不应该这样,还会感受到巨大的空虚和失落。记得要出来时,感觉像尿尿,但出来的却是一泡有浓度的液体,这让我有点惶恐不安,之后几天有点提心吊胆,后来渐渐就不怕了,当时也不敢和父母说,毕竟隐隐知道这不是什么光彩的事情,可能会被父母责备,所以我一直都是保密的。所谓食髓知味,有了第一次就渴望第二次、第三次……似乎是一个无底洞,永远也无法彻底满足。后来我通过学习才知道为什么会这样,背后是有一定科学原理的。

1953 年,奥尔兹和米尔纳在一次实验中,将电极深深植入小白鼠的脑中,通过电击来刺激大脑的某个区域。这个区域能让老鼠产生恐惧的反应,照预期,老鼠应该四处逃跑躲避电击,然而结果与之前的预计却完全相反,小白鼠竟然不顾一切去迎接电击,跑来跑去只是为了享受过电的感觉,多么奇怪的自虐行为。而这一切的发生竟然是因为失误,把电极安在了错误的位置,从而意外找到了一个全新的区域,他们尝试改变实验程序,让小白鼠自主选择是否按压杠杆电击自己,更加有趣的事情发生了:小白鼠会一直不停地按压杠杆,到死方休!这个实验貌似找到了大脑中一个能产生强烈快感的区域,可以让小老鼠体验到一种无法抗拒的美妙感觉,否则怎么会如此疯狂、如此着魔、如此上瘾呢?其实他们发现的是动物体内的奖励系统,当大脑发现获得奖励的机会时,它就会释放多巴胺神经递质。小白鼠之所以有这样的行为,是因为每当这个区域受到刺激时,大脑就会告诉它:“再来一次!这次准会让你感觉更好!”本质上,每次刺激都让小白鼠寻求更多的刺激,但是结果却不会带来彻底的满足感,彻底的满足感永远都得不到,永远都在下一次。

人就像高级小白鼠,上瘾的原理是一样的,那个杠杆就是 JJ,疯狂按压,疯狂摄取快感,到死方休!就是那种走火入魔的感觉。刚开始我也尝试控制,毕竟撸多了明显感觉身体不好,但有时看了黄,受了强烈的诱惑刺激,就很容易连续手淫,一定要彻底掏空才罢休,根本停不下来,有时一天甚至三四次,撸到射不出还不死心,一定射出才算完成任务,可以说疯狂至极,就算撸完马上死,也要坚持撸掉,已经达到丧心病狂的程度了,很多戒友都提及“丧心病狂”这四个字,形容那个状态的确很贴切。我后来才知道色情无异于毒品,手淫就像吸毒一样会让人上瘾,是一种高度成瘾的恶习,必须要彻底戒掉。

\subsubsection{第一次看黄}

第一次看黄的片子我到现在还模糊记得,但不会去回忆细节,以免陷入意淫。第一次看的是黄带,那个年代还是录像带,VCD 和 DVD 还没出来,相信很多 80 后对录像带都有比较深刻的印象,那个年代录像机还未普及,不是家家有,黄带也不多,在那个年代不是很容易看到黄,属于稀缺资源,那个年代看黄真的要被抓起来的。现在手机、电脑、DVD 几乎每个家庭都有了,看黄太方便了,网络上的污染也极其严重,上次看到一个帖子,里面的孩子也就十岁左右,就拿着手机看黄了,色情已经开始荼毒小学生了,真的非常可怕。记得我第一次看黄时,心里是非常紧张的,都能听到自己的心跳,内心也非常期待,当血脉喷张的画面出现时,我彻底震惊了,原来是这样直接和刺激,和我原来的想象有所不同。现在想来,黄片的确就是毒品,会带来非常强烈的刺激,会诱发出最大的贪心和最深的沉迷,有的人可以连续看黄几小时,甚至看十几个小时,开始看时还是白天,再次抬头时天已经黑了,看黄时的钟表走得飞快,太浪费时间和精力了。色情本身就是一种毒品,这是国外最新的研究结论,是一种新型毒品(new drug),它不会带来任何好处,反而会让我们疯狂沉迷其中,一旦看了就相当于吸毒,那种吸毒的机制已经开始运作了。真的一点都不能看,一旦沾上你就会沉迷其中,它会毁掉你的人生,我们不应该沉迷于色情之中,生活中有很多有意义的事在等待我们去做,切不可在看黄手淫中蹉跎宝贵的青春岁月。

国外的最新研究:“表面上看,可卡因和色情没有太多共同之处,但越来越多的研究表明:化学毒品会诱发大脑释放出让人兴奋的化学物质,而观看色情也有同样的效果。就像毒品一样,当这些使你兴奋的化学物质(例如多巴胺和催产素)传递到大脑时,他们会在大脑中构建一条新的通路。这条通路能从根本上诱导色情使用者再去浏览色情信息。在大脑通路被色情信息激活的过程中,大脑能释放出和第一次浏览色情信息时释放出的水平相当的化学物质,这种过程与毒品成瘾的过程相似。色情是一种通过眼睛注射进入大脑的毒品,尽管戒除它就像戒毒一样艰难,但网络社区的支持和帮助会让戒断色情成为现实,而且当你戒除后,你会变得无比自由!”可卡因是上瘾性极强的毒品,而黄片的上瘾性也不亚于可卡因,基本都是一次上瘾,可卡因等毒品需要花钱买,而网络上的色情毒品不需要买,这对于戒除就加大了难度。生活中你以为对方是正人君子,打开他的手机或电脑往往就能找到黄片,那是极为隐秘的一面,很多人都在偷偷使用色情的毒品。国外的文章还讲到:“当吸毒者吸食更多的毒品或是色情用户观看更多的色情内容,他们大脑中的神经回路就会变得更加粗壮,使他们自己无论是否想要都更容易重新去吸毒或观看色情内容。随着时间的推移,长时间超量的化学物质释放导致大脑的其他部分也发生变化。就像瘾君子最后会需要越来越多的毒品来获得快感,甚至仅仅是为了让自己感到正常一样,性瘾者在他们的大脑习惯了观看色情释放的大量多巴胺后,会快速地产生耐受性。换句话说,即使色情内容仍然能使大脑释放多巴胺,人们也不能感觉到和以往相同的影响。这是因为大脑正在试图通过摆脱一些多巴胺化学受体(这些受体像接球手的手套一样能接收释放出的多巴胺)来保护自身免受多巴胺超量的影响。因为少了一些受体,大脑就会以为多巴胺释放得不够,那么人们自己就感觉不到过去那种强烈的快感。结果是许多色情用户需要通过更大量、更频繁的搜索,寻找更极端类型亦或三者兼有的方式寻找色情以产生更多的多巴胺来感受到兴奋。”普林斯顿大学的 Jeffrey Satinover 教授如是将色情的影响描述给美国参议院委员会:“这就像我们发明的一种海洛因,用户可在自己家中秘密使用并通过眼睛直接注射进大脑。”

国外另外一篇文章讲到:“之所以网络色情是新型的毒品,是因为大脑对毒品和性唤起产生反应的部位是相同的。作为一种因性唤起或者高潮分泌的化学物质,多巴胺会触发大脑内的上瘾机制。如 Donald L. Hilton Jr.MD——一位执业医师和德克萨斯大学的神经外科临床副教授所观察到的:色情信息是一种视觉信息素。作为一个千亿美元级别的神经毒品行业,色情正在通过互联网的加速发展更迅速地改变人们对性行为的观念。色情信息在网络上无孔不入,它抑制了人对性取向的正常看法。想象一下大脑就是一片森林,徒步者日复一日从同一个地方经过,渐渐就踩出了路。浏览色情信息同样会产生神经回路,随着人们一次次浏览色情内容,这些神经回路就会在大脑的‘森林’中不断地被强化。这些神经回路最终变成大脑‘森林’里的小径。因此,邪淫者创造了不为人知的神经回路,这会使他们对性的看法被色情信息中的观点所支配。色情是一种混合型毒品,它通过引起兴奋(多巴胺造成的“嗨”的感觉)和制造高潮(毒品的“放松”效应),一下子触发大脑里的两种成瘾化学物质。这种混合机制令色情更容易成瘾,也更容易产生耐受性。色情的耐受性所要求的并不仅仅是更大剂量,而且是内容上的更多新奇,比如更多的禁忌行为,儿童色情,或是虐待。”

色情的本质就是一种毒品,黄片就是毒片,毒化心灵,让人越来越变态,现在最新的科学研究已经充分证实了这一点,在我第一次看黄时,其实就是在使用这种毒品,只是那个年代并不知道这是毒品,没有这种认识。

\subsubsection{上世纪九十年代末}

现代色情泛滥的开始就是上世纪九十年代末,国外的戒色视频也提及了这一点,大概是在 1997 年,录像机还未完全普及,影碟机就强势登场了,家家户户开始有了,黄碟也开始泛滥了。我记得我初中一位同学家里就有很多不良碟片,是他爸买的,我也到他家里去看过,租碟片的地方也能租到不良碟片。我那时还是初中生,但已经被黄片诱惑得完全不能自控,黄片就是我的毒品,我就是彻头彻尾的手淫瘾君子,到处租不良影片,真的就像吸毒一样,还不敢让家里人知道,还经常把家人支走,然后自己在家里疯狂堕落。影碟机的出现绝对是色情泛滥的开始,之前不管是书籍还是录像带,都没达到那个普及度,后来进入新千年,网络开始渐渐普及了,这就导致了更严重的泛滥,之前碟片还需买,网络的出现直接都不用买了,成了免费的了,比在地上找石子还容易,这就很可怕了,很多孩子从小学时就开始看黄了,家长往往还不知道,孩子都是偷偷看黄,偷偷手淫,过着隐秘而堕落的生活。孩子的眼睛本该清澈、明亮、有神,纯真喜悦且富有灵气,而现在很多孩子的眼睛却显得空洞而迷茫,精神也萎靡不振,就像花朵枯萎了一样,有经验的人一看就知道发生了什么。

\subsubsection{穿过一座城市去买黄}

有个少年顶着太阳,把自行车骑得像风火轮,从城市的一头骑往城市的另一头,几乎穿越了一座城市的距离,这个少年要去干嘛?去买黄!这个少年是谁?就是 20 年前的我,我在去买“毒品”的路上。在那个年代,几乎接触不到手淫的危害,虽然国家在打击黄赌毒,但普通人对于色情有什么危害,都是比较茫然的,父母也不知道色情和手淫的危害,在孩子青春期时也无法给予正确的引导。我那时知道自己在做一件见不得光的事情,其实那时我自己已经尝试戒过了,但无奈心魔太强,一次次把我拖入怪圈,一次次失败让我灰心丧气,我那时戒到没信心了,就自甘堕落,等到身体不行了,就下决心戒一段时间,再次失败后,就继续堕落,那是一个怪圈,一个死循环。我到现在还记得那个无知的少年,那个缺少正确引导的少年,在去买黄的路上,多么无知,也多么可怜,在孩子最需要正确引导的年纪,却无人能给予他帮助,无人来告诉他色情和手淫的危害,无人来告诉他如何戒除恶习。

看到一位新人发的帖子:“救救我吧,我对黄片越来越无法自拔,我才 16 岁,对黄片非常上瘾,几乎每周都看黄手淫两次,我好痛苦,但又戒不掉,我快死了,我好崩溃。”当生命力被手淫恶习一次次抽空,渐渐沦为行尸走肉,那种感觉真的感觉快死了,半死不活,半人半鬼,死气沉沉,萎靡不振,没有生气,没有活力,本是花季雨季的年纪,却被色情与手淫恶习无情摧残,未老先衰,撸后感觉自己越来越不快乐,心情抑郁,渐渐就会自卑、自闭。作为一个青少年,也很难面对如此龌龊下流恶心的自己,虽然砖家说手淫是正常的,但内心的良知会说这是不对的,出现伤精症状后更觉得手淫是不对的。我那时也差不多 16 岁,也是对黄片极度上瘾,根本无法自控,我买的黄片在手淫后被我剪掉了,开始我是藏在家里,后来觉得那无法让我安心学习,就像一颗定时炸弹,隔几天我就想看黄,根本控制不住自己,所以就用剪刀剪成碎片。等到心魔再次附体后,我又去买黄,就像现在很多人下载后撸完就删,下次想撸就又去下载。那个时代找黄很不容易,撸过的片子缺乏新鲜感,而现在网上的新鲜片子唾手可得,也不用买,这就大大提升了堕落和沉迷的可能。在染上手淫恶习后,我感觉那个纯真无邪的自己已经渐行渐远,我很难体会到单纯的快乐了,我本以为看黄手淫会让我快乐起来,但最后感受到的只是无限的苦楚,一身的症状,真的苦不堪言。

\subsubsection{手淫狂魔}

我那时的状态真的可以用疯狂来形容,走火入魔,疲于奔命,丧心病狂,很多次都撸到身体接近崩溃死机,面如死灰,双眼无神,站立不稳,气色像鬼一样,随时都会倒下,虽然身体已经发出严重警告了,但我还是贪恋那种快感。翻看那时的相片,有好几张都是哭丧着脸,一脸的呆滞和晦气。在青春期我的性瘾已经很严重了,那时我就得上了前列腺炎,谁能相信十几岁的孩子就得上了慢前,其实在色情泛滥的时代,这种现象非常普遍。前段时间看见一位戒友发帖说:“想死的心都有了,早上办公室打飞机被同事开门撞见了。怎么办?他进来打印东西的,这次被同事撞见了,太尴尬了,好想下定决心戒掉啊!”连上班时都在看黄手淫,这种恶习的确是无孔不入,手淫可以说是最隐秘的恶习,被人看到这极端龌龊猥琐阴暗的一面,真的是非常尴尬,在被撞见时,往往会有一个定格,仿佛时空都凝固了,还会有一个对视,在对视的一刹那,真的是想死的心都有了,那个对视含义很深,看尽了人性中龌龊丑陋不可告人的一面,在人前把自己装作正人君子,突然被人看到这么猥琐不堪的一面,的确让人大跌眼镜,反差真的是太大了。就我个人的深度体会,手淫时其实是一个人很虚弱的时刻,不管勃起多么硬,多么持久,都无法掩饰那份虚弱,那种无力掌控内心,无力战胜心魔,无力主宰自己的虚弱。

国外一位戒友说:“当事情结束之后,你又感到深深的无力。你擦了擦纸巾,又浪费了一个令人悲叹的一天。”手淫过后内心会进入一种悲哀、荒凉的境地,甚至可以用悲惨来形容,对自己失望,感到空虚和悔恨,还有深入骨髓的深深的无力感。本来想干点正事,邪念上头后就把所有的精力和时间都花在找黄、看黄和手淫上去了,浪费时间,浪费精力,浪费生命,糟蹋自己的精气神,最后就是一个失败者(loser)!在放纵中虚度一天又一天,活得浑浑噩噩,非常消极和颓废,看着父母忙碌的背影感到很愧疚,觉得对不起他们。父母对我寄予厚望,烧好吃的给我吃,而我却这样糟蹋自己,实在太不应该了。在初中和高中时,只要学习压力一大,我就会手淫,遭受一点挫折,我也会手淫,情绪不佳时,我就用手淫来发泄,睡前手淫,睡醒后还是手淫,手淫成了我逃避生活、逃避现实的一个方法,然而每次撸完后我又不得不硬着头皮去面对更加糟糕、更加难以收拾的现实,对我越来越不利的局面。

\subsubsection{永远不会射够的!}

一位戒友说:“射过那么多次,对那种感觉还是不明不白,而且我射过那么多次都还没射够,刚射过都会想撸第二发。这次算是体会到了瘾的力量。”《灵宝通微经》:“淫欲放纵,如饮咸水,多饮多渴,除死方休。”不管射多少次,永远不会射够的,永远不会彻底满足,永远还有下一次,心魔也会怂恿“最后一次”,最后一次过后还有下一次。快感是虚无缥缈的,那种感觉的确让人不明不白,似乎体验到了,但根本留不住,过后也很难说清那种感觉,怎么也回忆不出来,就像什么也没发生,极想再次体验它。很多人都在贪恋那几秒的快感,极度地贪恋,我以前也是疯狂贪恋,觉得很爽很舒服,生活中其他事情都没这么爽,这其实就是多巴胺在作怪,当普通的片子无法刺激多巴胺大量分泌了,这时就会去看变态的片子,最后性取向肯定越来越变态。那种舒服其实是一种诱饵,诱饵里面藏着痛苦的钩子,鱼儿只见诱饵而不见钩子,撸者也是如此,只看到快感舒服,却不知道舒服的代价是巨大的,以掏空五脏六腑精华作为那几秒舒服的代价,真的是亏大了。我那时就像一具“射精机器”,总也射不够,我身体根本无法应付那日渐庞大、无底而变态的欲望,高频率的手淫让我身体吃不消,就是在 16、17、18 岁,这样花样的年纪,我的脸庞和身体已经开始出现衰败的征兆了,在最美的年华我撸出了最猥琐的自己……

\subsubsection{镜中丑陋猥琐的男人}

青春期的我很喜欢照镜子,那是一个极其注重颜值的年纪,而手淫恶习让我变得猥琐而颓废,失去精气神,本来五官还算端正,人也朝气蓬勃,沉迷色情与手淫后,就渐渐走样了,眼皮开始下塌,双眼特无神,脸上灰扑扑的,痤疮粉刺一堆,严重时脸上还会起一层油脂样的东西,看上去很脏,怎么洗也洗不干净,毛孔也变得粗大,脸上很容易出油。记得小时候站在镜子前,我是那么清纯,皮肤光滑细嫩,表情神态也是纯纯的样子,一脸的纯真无邪,根本不知道手淫是何物,后来掉进撸坑后,就越来越丑了,每次疯狂手淫后都能感觉到自己变丑了,一丑就没自信了,渐渐就自卑了,不敢正眼看人。一位戒友说:“昨晚梦见自己的小学时代,自己背着小书包告别自己的父母和自己的几个发小一路走到学校,进了教室看着熟悉的一个个面孔,心底里很高兴,和男女同学都无忧无虑地嬉戏,每一个笑脸都是那么阳光,那时候的我是多么纯粹呀,自从初中被一个比我大一岁的孩子带着看黄碟之后就一发不可收拾了。”内心纯净时就像活在天堂一样,单纯而快乐,无忧无虑,内心龌龊后,整个人的容貌气质也跟着猥琐起来,小学时意气风发,感觉那时的天空很蓝,白云很白,微风也很轻,温柔地吹拂着我们纯真的脸庞和发梢,纯净的灵魂是如此美好,那时的我们无忧无虑,积极向上,真的好单纯,好快乐,那是一个纯真梦幻的年代。一旦开始撸了,就像霜打的茄子,彻底焉了!手淫后每次照镜子,总是感觉很失望,镜中浮现的人真的是我吗?怎么那么丑陋、猥琐和颓废,我真的不想看到这样的自己。

\subsubsection{化身博士}

我初中时看过一本书——《化身博士》,给我留下了深刻的印象,我觉得我就像里面的主人公,过着双面的人生,在学校里和家长面前我努力装作一个好孩子,然而在面对黄片时我却是一个十足的淫魔。《化身博士》讲述亨利·杰基尔医生喝了一种药水,在晚上化身成邪恶的海德先生四处作恶,他终日徘徊在善恶之间,其内心的内疚和犯罪的快感不断冲突,令他饱受折磨。杰基尔医生发明的药水可以将平时被压抑在虚伪表相下的邪恶心性,毫无保留地展露出来,随着人格心性的转变,身材样貌也会随之改变。一旦喝下药水,即摇身一变,成为邪恶、毫无人性且人人憎恶的猥鄙男子——海德。每当邪念袭脑,我就会摇身一变,变成邪淫版的“海德”,丧心病狂,不顾一切。我们人性中都有善恶的成分,关键是要去除恶的部分,发扬善的部分,在沉迷邪淫时,我们是在发展恶的部分,恶代表负能量,负能量最终会导向痛苦。

在青春期时,还有一部小说也给我留下了深刻的印象,那就是《道林格雷的画像》,故事是这样的:道林格雷是一名长在伦敦的贵族少年,相貌极其俊美,心地纯真,一位画家为道林画了一幅画像,而道林在画家的朋友亨利勋爵的蛊惑下,向画像许下心愿:美少年青春永葆,所有的岁月的沧桑和少年的罪恶则由画像来承担。道林开始不以为然,没想到自己玩弄了一位纯情的女演员的感情后,女孩自杀,此后竟然真的发现画像上的道林的样子发生了邪恶的变化。此后恐惧的道林反而更加放纵自己的欲望,而画像一日日变得丑恶不堪,道林仍然青春永葆。十八年后,那位画家又见到道林,道林出于对画家作品的憎恨和对自己丑恶灵魂的厌恶,谋杀了画家,当年那位女演员的弟弟也前来寻仇,被道林巧言欺骗,不久也死于意外。道林因为女演员弟弟的意外死亡而良心发现,举刀刺向丑恶的画像,那把曾经杀死过画家的刀,却刺进了道林自己的胸膛,横尸地上的道林形容枯槁,面目可憎,挂在墙上的画像却丝毫未改,光彩依旧。这个故事里的少年因为放纵自己的欲望而一步步走向邪恶,我那时就在疯狂纵欲,少年时代的我也比较帅气,我亲自见证了自己的面容在疯狂手淫后的丑陋变化,和道林格雷的画像一天天变得丑恶不堪,高度相似。这个故事当时震撼了我,我觉得我就是道林格雷,疯狂放纵自己,最终自食恶果。

\subsubsection{整个荒唐、龌龊与无耻}

手淫是一个非常荒唐、龌龊与无耻的恶习,一个大男人干点什么不好,何必非要玩弄自己的 JJ,这是一个糟蹋自己精气神的恶习,一位戒友回忆说:“自从我 2009 年 12 月染上恶习以来,我生命的颜色就逐渐由彩色转成了灰色。这种变化不是一蹴而就的,而是一种缓慢的但是又很明显的变化。染上恶习的这几年,我的性格变得多疑、敏感、自大、自卑、阴暗和自私。我看着自己的面容由之前的洁净干爽变为毛孔粗大、满脸油腻和皮肤暗淡,也看着自己由纯净单纯变成恶心肮脏猥琐。”这个恶习会导致自己积累太多的负能量,整个生命的色彩都会变成灰色,那个荒唐的人和那个荒唐的动作,一次次把自己掏空,起各种龌龊的邪念,在那种状态下,整个人显得极其无耻,到处意淫别人,还以此为乐。手淫这个恶习实在太掉价,太掉颜值,太掉正能量,最终会把自己困住,陷入深深的绝望和惶恐。

满地荒唐精,一把辛酸泪,一个无比憔悴的撸者,一副被掏得危如累卵虚得要命的身躯,这就是那时我的真实写照,站在镜子前,我对镜子里的自己感到失望,人也变得非常自卑,自卑后自然就自闭了,不爱说话,不爱交流了,说话时也不敢看对方眼睛,好像做贼一样心虚,每次照镜子只有深深的失望和无奈,感觉自己那么颓废、萎靡,真的丑了很多,都没信心面对以后的人生了。发育前我的内心还是比较干净的,那时很纯真,是朝气蓬勃的少先队员,戴着鲜艳的红领巾,还记得老队员用心为新队员们佩戴上红领巾,庄严的队礼互相敬起。队员们脸上的笑容是那么纯真、欢乐和幸福,敬队礼时,显得那么崇高而庄严,那时是祖国的花朵,努力成为有理想、有道德、有文化、有纪律的社会主义事业建设者和接班人。还记得唱《少先队之歌》时,那种踌躇满志的神情,那种积极向上的感觉,还有《让我们荡起双桨》,那种纯真与美好,让人刻骨铭心。染上手淫恶习后,满脑子都是污秽不堪的东西,和之前的自己大相径庭,感觉内心比下水道还脏还龌龊,我的脑子就像一个垃圾场一样散发着邪淫的恶臭。相由心生,一个人满脑子都是龌龊下流的念头,是不可能长得相貌堂堂一脸正气的,手淫几年后明显感觉自己猥琐了,丑陋了,脸部也开始变形了,好像某种气被抽走了,脸庞一下就走样了,戒色后脸部会收紧,变得紧致有型,手淫后则会浮肿变形,松弛走样,会变得不对称,显得怪异,气色也会变得晦暗,明显有一种衰败的感觉,就像过期水果一样。

一位戒友说:“长期因手淫造成的孤僻性格,只好以不断地手淫来减轻压力,何等愚痴和荒唐!!!短短四年下来,由于手淫过度,自己几乎脱成了秃头,学习也没有完成。”这个恶习真的太荒唐了,疯狂跑皮,几乎达到了摩擦起火的速度,死死盯着屏幕,那个节奏,那个手速,那个贪婪,那个佝偻而猥琐的身影在忙碌……当一切结束后,好像完成任务一样,瞬间变得一点意思都没了,看之前的片子也没感觉了,就像一场欺骗一样,骗走了最宝贵的肾精,心魔得逞后大笑而去,留下一脸沮丧和悔恨的撸者,身体精华被榨干,完全废渣一个。

\subsubsection{无害论的蛊惑}

我在二十年前就看过无害论,当时就被无害论蛊惑了,那时是在杂志上看到这类内容,书中的观点其实就来自于国外的性学书籍,完全是错误的观点,性学书籍迎合的是性解放,对于性当然全部是赞成的观点。专家认为手淫是合理的、正常的,甚至还有很多好处,看到这类观点的确会让人心安理得一段时间,那时我才十几岁,也比较迷信权威,对于专家也比较信服,后来才知道是砖头的砖。有些砖家知道的并不比你多,这是事实,他们中的很多人并无深入的研究,只是照搬了性学家的观点。我看了国外 19 世纪的医学书籍摘录,那个年代是反对手淫恶习的,认为这是一个自渎伤身的恶习,对身心健康有着极其严重的负面影响,在 18 世纪和 19 世纪,无论是牧师、教师、文学家,还是医生、精神病学家,他们都普遍认为手淫是一种有害的行为方式,无论从生理还是从心理上,都是如此。后来到了 20 世纪,国外的性学家崛起了,之前关于手淫危害的文章都被他们批判和否定了,事实真相被掩盖了。现在到了网络时代,事实真相又开始回归了,国外有很多戒色网站和各类戒色书籍,对色情和手淫的危害做了充分的揭示,将来无害论肯定会被淘汰,这是大势所趋。

国外戒色书籍写道:“色情给人虚幻的快感,前一刻它让人飘飘欲仙,后一刻它就牢牢地将人拖下无底深渊。”“色情的另一面是毁灭。我丢了饭碗,还差点跟老婆离婚。如果你一直看色情,那么总有一天,色情会毁了你全部的生活。我认为,现在很多人都还没有意识到色情的毁灭性。”“这种双面人生最终毁了我,也影响到了他人。”“色情成瘾者每一次观看色情之时,和生活中最可怕的变故不过一纸之隔。”“后来我才意识到,只有一个真正强大的男人才能戒除色情瘾这种顽固的恶习,而且,只有一个真正的男人、一个不看色情的男人,才能真心去爱一个女人。”国外的戒色书籍说得很好,我们要认清色情的毁灭性,浏览色情基本都伴随着手淫恶习,这两个行为是连在一起的。尤其在今天这个互联网高度发达的时代里,你可以轻松获得取之不尽、用之不竭的色情视频,无论多么变态和新奇的内容都能搜到,你的右手食指点击、点击、不断点击,你的整个身体像被钉在了椅子上,一动不动,你的两颗眼珠目不转睛死死盯着屏幕,发出贪婪幽暗邪恶的光芒,你沉迷其中根本停不下来。随着手淫次数的增加,这种不满足感就会日益膨胀,瘾会变得越来越强,心理也会越来越变态。一位戒友说:“我看了二十年的黄片,我也有想喝尿的想法……因为越看越变态!”看到最后普通的片子已经很难刺激多巴胺分泌了,到时就会看变态的,越看越变态!到最后就是严重的恋癖、虐待、吃屎喝尿,越来越变态,丧失人性,泯灭良知,越来越可怕,有的人还和动物发生关系,简直骇人听闻,实在太可怕了,已经堕落到如此不堪、如此下劣的地步了,连畜生都不如了。

我们这几代人都被无害论误导过,现在是该认清真相了,不能再被无害论误导和蛊惑了,色情与手淫的危害需要被社会公众重新认识。一位戒友说:“你拿出一堆医学论据去包装这个行为,把这个行为合理化,你心安理得地手淫,但是,你内心的猥琐是无法掩盖的。”他说得很好,不仅内心的猥琐无法掩盖,容貌气质的猥琐更加无法掩盖,伤精症状迟早也是会爆发的,也许现在感觉还行,但只要继续撸下去,迟早会进医院的。撸者基本都在看黄,而国家在扫黄,黄赌毒,黄排在第一位,黄对心灵的毒害也最大,看黄手淫肯定是不对的,无害论的伪科学迟早会被淘汰。彭鑫博士说手淫无害论是祸国殃民的说法,无害论是该被澄清了,很多人都在手淫后拿无害论安慰自己,然而症状还是爆发,最终肯定会知道无害论是完全错误的。戒色吧翻译团队翻译的一篇国外文章:美国人控诉手淫 22 害!其中讲到:手淫让你虚弱,它会使体内的蛋白质和钙质流失(精液是人体的精华,可不是吃几个鸡蛋就能补回来的);手淫使人产生紧张和神经系统问题(自卑、口吃、胆小怕事、神经衰弱);手淫是勃起障碍的一个主要原因(阳痿早泄);手淫一次即可上瘾,想要控制住几乎没可能(手淫极具成瘾性);手淫让人昏昏欲睡,射精后大部分时间你都会在睡觉,真的会非常疲惫(这时候斗志丧失,耽误本来计划好的事情);手淫也能影响心理,在射精后你会感到沮丧和自责;手淫导致无限的欲望,它让你无视地点、人物和文化,让你尽最大的可能去满足自己越来越重口味的欲望;过多的手淫导致精子数量减少、质量下降;手淫是不道德的,你幻想某个异性来手淫,这扭曲你的价值观和你对他人的尊重感;手淫浪费时间,让你变得没有价值;手淫的快感短短几秒而已(带来的痛苦却是长久的);手淫会导致失忆和思考能力下降;手淫绝不是你欲望的终点,你已经被它所欺骗,它绝不会解决你的问题,更不会让你满足。另一篇翻译的文章中,Frederick Humphrey 博士说:“那些抽干自己生命力的过度行为,比如过度病态的放纵,过多次数的滑精,过长时间、过频繁地刺激性器官,特别是因为这种放纵与生理和心理的过度工作紧密联系,最终都只能导致同样的结果。年轻人神经衰弱的主要原因就是手淫,如果持续下去,将会在不久的将来导致严重问题,并且产生久远的负面影响。可以负责地说,每年无数的人就因为手淫这一项恶习而被带入神经衰弱的地狱。”

国外目前对色情和手淫危害的认识已经比较到位了,当然我国也有自己的优势,那就是从中医的角度来解释手淫的危害,而国外则更多地从西医的角度来解释手淫的危害。两者都很好,过去我们一直被手淫无害论蒙蔽和欺骗,看了无害论之后,就会肆无忌惮地放纵自己,现在是该认清真相,下决心戒除恶习了。

\subsubsection{身心爆发症状}

当初染上手淫恶习,我并不知道手淫是有害的,我被快感冲昏了头脑,后来发现身体逐渐出现了各种不适,那时我就隐隐感觉手淫是有害的,记得最早出现的症状就是尿频和腰痛,还有就是容易感冒,鼻炎加重,腿软也很明显,变得爱睡懒觉,早上起不来,眼睛浮肿,没精神,脑力下降,精力下降,体能下降,痘痘增多,撸的时间越来越短,出现勃起不坚,内心也变得很难快乐起来,总是郁郁寡欢,人也变丑了,丑了之后就开始自卑了。当症状严重时,我就戒一段时间,当症状缓解了,我就又开始了,好了伤疤忘了疼。手淫前几年爆发的症状虽然很明显,但基本还可以忍受,那时年纪轻,恢复也很快,戒几周就能感觉到明显的改善,可惜好景不长,被心魔再次拖入怪圈后,身心的状态又大幅下降了。学生党最要命的就是脑力下降和精力下降,这两个一下降,学习成绩肯定会下降,我那时就感觉沉迷手淫后脑子变迟钝了,很多题目当时做不出来,等到老师报答案了才恍然大悟,就是做题目时脑子转不过来太迟钝了,想不到那个知识点。手淫后记忆力和注意力都大幅下降,有时听课时都在意淫,听课效率可想而知,有时被老师点名叫起来,我才从黄片的场景中回过神来回到课堂,然后一脸懵逼地看着老师,因为学习成绩下降,被老师叫了好几次家长,那几年我活得心惊胆战,就怕老师叫家长,也怕家长问成绩,实在焦头烂额,脑力下降真的很难应付繁重的学业,后来我通过中医认识到:肾上通于脑!我被这条医理彻底震惊了,原来手淫会导致脑力下降,这么多年的脑子迟钝总算找到原因了。我欲哭无泪啊,知道得太晚,在学生时代彻底结束后才知道这条医理,《素灵微蕴》曰:“肾为髓之下源。”肾主骨生髓,髓充盈于脊骨,上达于脑。肾精——脊髓——脑髓,这是连着的,肾精是可以化生脑髓的,也可以化为浊精,这是一股能量,取决于你怎么用它。后来我学了中医知识,知道在十几岁时我已经肾虚了,就是手淫害的,可惜那时的我太无知。

\subsubsection{第一次戒色}

在手淫后没多久,我就开始了第一次戒色,因为我觉得这个恶习在控制我,让我上瘾,让我身不由己,虽然那时年纪还小,但内心也是不愿过这种放纵的生活,本能地知道这种事情不好,内心的良知也在督促我戒色。在没有任何戒色觉悟,并且还有那么多思想误区的情况下,第一次戒色没有任何悬念地失败了,当时就靠下决心、转移注意力、充实生活、积极锻炼,平时还好,一到周末,独处时间增多,就容易破戒,在强戒状态下能挺过三周已经很不容易了,所谓强戒,就是不学习,一味靠毅力,失败了就怪自己毅力不行。其实根本不是毅力的事儿,不学习哪来的戒色觉悟,断念能力为零,对心魔也没认识,肯定失败!我那时是十足的菜鸟,第一次戒色失败后,之后又尝试了很多次,毫无例外都失败了,从没超过一个月。我是磨床怪客,也是手淫狂魔,磨床一直磨到了大学,手淫也是,这两个行为伴随着我的成长。邪念一起,我就会失控,我就会疯狂找黄,我就会干龌龊下流见不得人的事情。之所以戒色,一方面是因为症状开始出现,另外我还是想做回纯净的自己,我知道纯净的状态才是我真正向往的,但是那种念头一来,我就被强拽到那个怪圈,那股拉力实在太大了,我那时并不知道是心魔,只知道戒到一定时间,内心会突然有股莫名而强大的力量把我拖入那个怪圈,我对那股力量没有任何招架之力,就像小孩和大人拔河,一拔就过去了。心魔非常阴险狡猾,套路也很深,经常冒充我,怂恿我,我那时没有任何戒色觉悟,还以为是自己的想法,于是听信、跟随,最后欲火中烧,疯狂破戒。破戒后那种戒色积累的底气和自信一下子消失了,就像王子变回了癞蛤蟆。说实话,泄与不泄对人健康、精力、自信、底气、意志、思想、眼神、气质、气色、精神方面的影响的确是不一样的,完全是两种状态。戒色后眼神都会变得犀利起来,那股自信和底气,那种光明磊落,完全是邪淫时不能比的,如云壤之别。

\subsubsection{那个超强的怪圈}

那个怪圈有股超强的魔力,把人牢牢控制在里面,陷入死循环,很多人都试图挣脱出来,但基本都失败了。因为他们要面对的是心魔大 BOSS,这个老怪已经修炼了无数年,对付一个无任何觉悟的小孩,真的是小菜一碟。我无数次想戒,无数次下决心,发毒誓,甚至还自残过,但是面对心魔的进攻时,简直不堪一击,真的是一触即溃。我初中自残过,大学也自残过,还是依然失败,我痛恨那个无法自控、疯狂放纵的自己,然而面对心魔的进攻时,却显得那么无能为力。我是反对自残的,因为自残戒色必然失败,用自残惩罚自己也极为不明智。大学时我还在破戒后跑 50 圈(大概 250 米的操场),一方面有惩罚自己的意思,另外也想通过这种方式把心魔跑掉,那时缺少中医养生的知识,不知道大汗伤阳,结果过度运动导致颈椎病加重,心魔也没被我跑掉,现在知道锻炼只能作为辅助,核心还是修心,只是那时我不懂修心,只能在外在修身方面努力,结果肯定失败。现在戒色吧有很多新人的破戒帖,我看了之后就联想到曾经那个自己,和他们完全一样,根本不是心魔的对手,心魔一来,立马垮掉,立马沦为心魔的傀儡,那个实战的过程不到一秒,一秒之内我就会被心魔攻陷,真的就这么快。一位戒友说:“念头来,附体——破戒的死循环。”另外一位戒友说:“心魔一上来,又完犊子了。”心魔真的很强大,那个怪圈会牢牢控制你,防止你逃脱,要冲破它是很不容易的,当你达到一定的断念水平,又会发现这个怪圈其实是可以冲破的,断念水平差时真的太难冲破了,那股拉力实在太强大了。

\subsubsection{trouble!!!}

拳击比赛一方快被击倒了,解说员经常会情绪激动地大喊:某人陷入“trouble!!!”了,trouble 就是麻烦,吃了对方重拳,步伐踉跄,陷入困难,快要被 KO 了。记得以前我每次破戒,都是陷入 trouble,心魔就像一个混世拳王,那一个个邪念、怂恿、图像就像组合拳,要把我干倒,之前我实力弱小,每次都撑不过几秒,很快就陷入跟念带来的欲火中烧,开始疯狂找黄,那个猥琐的身影又开始忙碌起来,撸起来也是爆体的节奏,不彻底掏空是不会善罢甘休的,很多时候都是连续两次,一次结束,没过瘾,休息会又来第二次,第二次很难射出,疲于奔命,撸到手累,JJ 都麻木了,还在动作,JJ 都发软了,还在动作,绝对丧心病狂。一位戒友说:“今天我已经疯了,连破十次,喘不过气了,四次的时候已经射不出来,后面还接着撸,现在人已经站不稳了,我已经完蛋了,症状马上就会严重复发。”被心魔攻破后,就是疯狂!不顾一切地撸!不要命地撸!陷入可怕的纵欲状态!心魔一次次让我陷入 trouble,让我进入疯撸的状态,渐渐地,我在生活和学习上也陷入了 trouble,各种问题纷至沓来,让我难以应付。在弱小时,我真没有那个能力 KO 心魔,只有挨揍的份,那时的我并不具备让心魔一拳挺尸的强大终结能力,那时的我是十足的菜鸟,没有任何戒色觉悟,也没有断念能力。在内心的八角笼中,被心魔一次次完虐,KO 是最热血沸腾的高光时刻,而每次倒下的都是我。心魔太强,而我太弱,这种情形注定被心魔 KO,被心魔打得满地找牙,遍体鳞伤,心魔统治了我的内心,我只是它的一个傀儡。

\subsubsection{屡戒屡败,屡败屡戒}

和戒色吧很多人一样,我也经历了屡戒屡败,之前的自己没任何把握,戒了十几天就开始担心哪天“那股力量”来了,就会再次掉入怪圈,很没有安全感,我之前破过很多次,也失去过信心,也放弃过戒色,但后来症状严重爆发,还是把我逼回了戒色这条路上来,因为不戒实在不行,身体即将面临大奔溃。屡败屡戒这种精神是可贵的,但一次次失败会让信心和决心逐渐消失,一味靠毅力强戒是看不到任何希望的,在那个年代也学不到戒色知识,也只能靠毅力强戒,不知道戒到哪天就会突然破戒,自己一点把握都没有,我失败过多少次,我已经记不清了,但我心中还是希望自己能戒掉这个恶习,做回纯净的自己,每当戒到十几天、二十几天,这种渴望就越发强烈。我不想让自己永远困锁在幽暗的撸茧中,我想逆袭,我想蜕变,我想张开自己的灵魂之翼,带着强劲的生命力,向充满自由和光辉的新世界飞去。前几天看到一位戒友的求助帖,他说:“这两天一直连续破戒,真的是一发不可收拾,真不知道该怎么办了,明明知道手淫的危害,明明知道手淫后的痛苦,每次手淫后就感觉掏空了一样一天无精打采,浑身无力,双腿发软,没有一点正常人的样子,特别痛苦,但还是控制不住撸,我真的不想这样下去了,前辈们帮帮我,怎样才能摆脱手淫的控制,我真的不想这样下去了。”他的感受和我那时极为相似,真的很想摆脱手淫的控制,可惜那时我无法做到,当念头图像上来了,我就跟着跑,心魔稍微一怂恿,我马上就听信了,欲望起来后根本控制不住自己,那时的我不懂修心,失败了就怪自己毅力不行,这是非常低级的戒色状态。

\subsubsection{在黑暗撸坑苦苦挣扎}

我一直想冲破那个怪圈,但是一直失败,最后只能陷在撸坑里苦苦挣扎,我想挣脱那个束缚,但一直做不到,这个撸坑是一个亿人坑,遍布痛苦哀嚎的撸者,虽然有的撸者外表看上去还算正常,但其实他的内心正经历着某种挣扎和痛苦,这是表面上很难看出来的,他是无法自控的,他的内心其实很虚弱很无力。一位戒友说:“现在已经 27,大家都知道,这个时候本来是人一生奋斗的黄金时期,父母期盼自己能出人头地,有妻儿相伴,正是努力去建设未来,让父母妻儿未来能过上幸福美好生活的时候。可是现在呢?一身是病,带着千疮百孔的身体还拿什么资本去拼搏、去奋斗!辜负了父母的期盼,也没有妻儿相伴,一无所有,搞得一切本该按正常发展的人生轨迹严重偏航。当初还自以为是觉得 SY 是很享受的事,谁知道这其实就是一个糖衣炮弹的陷阱,是一个无底洞,等深深陷进去之后,才发现自己把自己给活埋了,真的可笑可怜。SY 慢慢地吞噬了我的身体,现在已经是百病缠身,人不人鬼不鬼的。有多少个无眠的夜,都是在孤独中无由地醒来,悔不当初,心就像被无数把刀在一刀一刀地割。”这位戒友的处境我很理解,我曾经也经历过,能够爬出这个黑暗撸坑,重见光明美好的世界,自此永远摆脱恶习,这是我那时的奢望,以我当时的实力,根本无法做到。我也深深懂得撸者的痛苦,他们的人生正经历困境,不管表面上多么风光,其实内心都是痛苦和绝望的,还有一种难以言说的虚弱感和无力感正笼罩着他们,他们会在一瞬间变得极为恐慌和无助,外表看似坚强,实则内心却异常脆弱,他们一直都无法真正主宰自己。

\subsubsection{三次痛苦的手术}

我到现在经历了三次手术,都和鼻子有关,一次鼻中隔偏曲纠正,两次下鼻甲切除术,这三次手术虽然打了麻药,但还是能感受到很大的痛苦,想象一下医生拿着剪刀伸进你的鼻子里去狠命地剪,还拿榔头敲,敲得整个手术台都在震动,想想都恐怖,但我还是熬过来了,记得在手术台上度秒如年,背部紧张得全部汗湿了。那个医生就像一个木匠一样敲敲打打,不像在动手术,而像在干木工活,麻药是局部的,而且只是麻了一部分,至少还有四分之一的痛苦要自己来承受。医生拿剪刀把鼻子里的骨头剪下来,剪不下来时就要用榔头和凿子,用力敲,狠命砸!那就像地狱里的景象一样,我现在回想起来都不寒而栗,做手术时眼睛是用布蒙着的,否则患者看到剪刀伸过来,是会极度恐惧的。在沉迷手淫后,我很容易感冒,鼻炎一年年加重,最后只能去做手术,看了因果方面的书籍和文章,我很确信这三次手术的痛苦和我邪淫还有杀生是密切相关的,年少无知造了一些杀业,加上疯狂邪淫掏空了身子,最后报应现前,苦不堪言啊!邪淫加杀生,那真是地狱搭档,果报现前时,真的是痛苦无量啊!我以过来人的身份告诫大家一定要戒邪淫和戒杀生,这两个是绝对的大头,这两个的果报也最惨。再说三个我身边的真事,我的一个亲戚杀了一条狗,还吃狗肉,当年就得了癌症,杀生的罪业非常可怕,还有就是一个杀猪的屠夫,到了四十多岁时得了喉癌,死前几天嚎啕大哭,很是凄惨。还有家附近的一个菜场路边摊卖鱼的贩子,人很高大魁梧,年纪也不大,大概三十多岁,天天在那杀鱼,搞得很血腥,不知他之前干这行多久了,总之,前年去问他怎么不来了,旁边的小贩说那个卖鱼的得了重病已经死了,我听到这个消息非常震惊,因为之前我经常看见他,感觉他很健康,身材也很魁梧壮实,怎么说没就没了呢?我想到了杀生!杀生短命报!可怕啊!生命被宰杀时,其苦无量,其悲无涯,苦苦挣扎,恐怖万分,其痛苦不可言说。如果能把它从生死线上解救出来,使其远离死亡的恐怖,从无助的苦痛哀鸣中解脱,这是何等的善举!戒色后应该适当放生,好好培养自己的慈悲心,这点很重要,放生时自己的心情也很愉悦舒畅,这也是很好的调心方式,不过要如法放生,注意放生的事项。

\subsubsection{邪淫浪子终回头}

记得我曾经疯狂放纵了近一个月,因为一个人住,更加助长了放纵的强度,作息饮食不规律,经常熬夜看黄,像马拉松一样看黄手淫,疯狂的状态以一次恐慌的大出血作为结束,记得当时的情形,整个马桶里的水都被鲜血染红了,我当时很恐慌,以为自己得了大病,后来就回到了家人身边,做了检查没什么大问题,只是裂开了一条口子,但出血量真的很大,把我吓到了。一个人住是很容易放纵的,这点我深有体会,因为没了潜在的监督,就很容易走向极端的疯狂。后来我得了神经症,被几十种症状轮番折磨,过得惶惶不可终日,那种生活太恐慌了,如惊弓之鸟,奔波在家和医院之间,当身体健康时,我不会想到还有这种痛苦,真的无法理解,但是当你真正进入了那种疾病的状态,就知道有多么恐慌和多么绝望了。为什么有的人会自杀,普通人想不通,得上了就知道那种痛苦了,那是身心的双重折磨。正是神经症的痛苦让我再次回头,再次开始戒色,之前我很早就想戒掉,但一直都失败,神经症的折磨激发了我下最大的决心,做破釜沉舟的最后拼死一搏!当然光有决心是不行的,我开始学习修行方面的文章和养生知识,正是学习带来了莫大的力量,戒色神力真的是来自于学习,当然也来自于练习观心断念,学习和练习永远是戒色的重点。通过学习中医我真正了解了手淫的危害,这个恶习真的是在自戕,中医完美解释了手淫导致的各种症状表现,这让我醍醐灌顶、豁然开朗,深深明白之前手淫就是在伤害自己,之前虽然也隐隐知道手淫不好,但没有那么明确而深刻的认识。

我以前就是一个邪淫的浪子,好几种邪淫的行为我都犯过,那是一种鬼混的状态,色鬼在混!完全无耻,不负责任,非常垃圾和混蛋的状态,小时候我最鄙视的就是鬼混的人,长大了我却成为了曾经我最鄙视的人,我深深忏悔往昔所造的邪淫罪业。邪淫时整个人浑浑噩噩的,被各种邪念占据着,驱使着,整个生活也是围绕着邪淫展开,在那个负能量的状态下,感觉人生最大的乐趣就是邪淫,那个状态的我充满负能量,就是一个标准的淫棍和愤青,现在想来应该是大粪的粪,那是一种马桶套头、屎尿灌顶的恶臭状态。在邪淫中醉生梦死,宛如一头勃起的蛮荒野兽,饥渴焦躁,盲目无知,十分荒唐和愚蠢,一股令人作呕的气息传来,让人退避三舍。这种堕落的生活方式最终毁了我,神经症彻底敲醒了我,我不想再堕落下去了,我想找回本初的纯净。八年多前的那个戒色的决定是我蜕变的开始,这次戒色与以往不同,以往我是一味靠毅力强戒,根本不学习,而这次我突然开窍了,懂得学习了,大量地学习,大量地做笔记,以前是疯狂找黄看黄,之后是疯狂学习戒色文章,如饥似渴,我把找黄的疯劲全部用到了学习上,笔芯用完一根又一根,不知疲倦地记笔记,复习笔记,乐在其中,很快我的戒色和养生觉悟就上来了。我感觉这次戒色好像摸到边了,好像能成功了,我当时有那么一种微妙的预感,特别是戒到三个月时,那种感觉很明显,因为之前我从来没突破过一个月,而这次达到了三个月,而且内心很平稳,这给了我信心,但我也知道不可掉以轻心,不可骄傲自满,还是要谨慎,还是要加强学习。

邪淫的浪子终于回头了,堕落的十几年终于结束了,一路走来,我真的过得好惶恐,内心也很痛苦,生活也是各种不顺,一直不知道为什么,后来才知道和邪淫密切相关,邪淫让我变得充满负能量,负能量感召的肯定是不好的事情。

\subsubsection{色是少年第一关}

\begin{quote}\it
    色,少年第一关,此关打不过,任他高才绝学,都无受用,盖万事以身为本。血肉之躯,所以能长有者,曰精曰气曰血。血为阴,气为阳,阴阳之凝结者为精,精合乎骨髓,上通髓海,下贯尾闾,人身之至宝也。故天一之水不竭,则耳目聪明肢体强健,如水之润物,而百物皆毓;又如油之养灯,油不竭则灯不灭。(《寿康宝鉴》)
\end{quote}

来到这个世界上,我们是要闯关的,就像一个游戏,第一关就是色关!色是少年第一关,这关必须要过!很多人一直没有闯过这一关,到了身居高位,最后还是栽在了这一关。家庭、学业、事业都需要一个好身体来作为经营奋斗的基础,滥撸滥泄,真的会把身体搞垮的。《少年进德录》云:“人之精力有限,淫欲无穷。以有限之精力,供无穷之色欲,无怪乎年方少而遽夭,人未老而先衰也。”“少年如已损伤,急宜断欲一年或二年,以补其陷。中年体已觉衰,急宜断欲三年,以充其体。从此永无色事,自可得臻上寿。”我国古医籍云:“夫阴阳之道,精液为珍,即能爱之,性命可保。”精液是非常珍贵的,美国医学家在一次生物试验中发现,“男性的精液中含有大量由脊髓原液构成的生化物质,除了含有蛋白质外其余维生素、黄色素、胆固醇、磷脂类、锌及柠檬酸等含量与脊髓液的含量等同。而脊髓原液所含的成分在供应神经系统营养上有着不可替代的作用,它可以减少由于劳累产生的疲倦,使人的反应力、精力一直保证在一个很高的水平。现代科学研究表明,精液中含有的蛋白质使神经键产生了电离作用,与脊髓原液对神经系统的作用相同。精子中含有非常丰富的珍稀的营养成份。这种成分是神经系统、骨髓干细胞系统和男性生殖细胞系统必需的成分,对这三个系统而言,这种成分的多寡决定了它们功能的强弱。”经常性的射精会引发反应迟钝以及疲倦症状,也会导致其他各种伤精症状。把精液说成垃圾、营养价值低的文章存在非常明显的误导,那类文章会让人沉迷于纵欲,还以为这样干没事,等到症状爆发就知道真相了。《养性延命录》:“保精则神明,神明则长生。精者,血脉之川流,守骨之灵神也。精去则骨枯,骨枯则死矣。是以为道务宝其精。”明代医学家张景岳:“善养生者,必宝其精。精盈则气盛,气盛则神全,神全则身健,身健则病少,神气坚强,老当益壮,皆本乎精也。”陈继儒在《养生肤语》中指出:“精能生气,气能生神,则精气又生神之本也,保精以储气,储气以养神,此长生之要耳。”古人很有保精和宝精意识,第一个保是保护肾精,第二个宝是把肾精当作宝贝一样呵护,绝对不能随便耗泄。

\begin{quote}\it
    子曰:“君子有三戒:少之时,血气未定,戒之在色;及其壮也,血气方刚,戒之在斗;及其老也,血气既衰,戒之在得。”
\end{quote}

\begin{quote}\it
    人一思淫,心田即暗。中正之心已邪,则光明正大之气遂失。若人时时存邪念,积久而邪气蛊惑于身心,即小人矣。
\end{quote}

君子第一修为就是戒色,戒色就是戒邪淫,一个人要懂得控制自己的欲望,不能放纵自己的欲望,放纵会导致自毁,会让自己受困,会让运势变差,整个人会陷入黑气笼罩,灰头土脸,心神昏昧浊恶,家庭、朋友、学业、事业、身体、情绪、财运、感情等,都会受到很大的负面影响,大家自己应该都能体会到。那些邪淫的人总是说“食色性也”。“食色性也”是告子所说,古圣先贤绝对不会叫人纵欲的,他们的观点是万恶淫为首!怎么可能叫人纵欲?所谓“食色性也”,是让人认识到食、色是人的两大基本欲望,但都要严格管理,是有禁忌的。食物方面不能暴饮暴食,也有很多饮食方面的规矩和讲究,而性方面则要戒除邪淫,是绝对不能乱来的,古人是反对婚前性行为和手淫的,两者都属于邪淫。把“食色性也”当作自己放纵的借口,实在是很无知,想想古圣先贤也不可能害子孙后代,他们都是一致反对邪淫的。

《少年进德录》是丁福保所作,丁福保先后编译出版了近八十种国内外医学书籍,合称《丁氏医学丛书》。《少年进德录》成书是在民国年间,论手淫之害较为详尽。其实在那个年代,很多有识之士对于手淫的危害已经认识得很到位了,著名的药理学家张昌绍先生也写过戒手淫的文章。

\begin{quote}\it
    手淫之害,较大于交接。犯此恶习者,多在少年,往往旦旦伐之,以短促其生命。其发现之病状,为脑神经衰弱,记忆缺乏,作事易倦,屡呼头痛,动辄忿怒悲泣。阴茎软弱无力,精液中无精虫,或全失交接之力,而成阴萎症。梦中漏泄精液,或时有精液之漏泄,而成滑精症。四肢乏力,躯体踉跄,不良于行,立足不稳,不能支持其躯体。手指震颤,眼中无光,视力衰减,眼窝陷没,耳鸣重听头重,时发眩晕。面如土色,皮肤苍白,全呈病态。筋肉弛缓无力,睡眠终夜不安,心跳惊悸,腰部酸痛。身体及精神,均起障害,终日昏懵,如在五里雾中。思考力渐渐减退,而归于消灭。关节疼痛,消化力障碍,胃腑痉挛。血液衰减,胸部充塞,皮肤肿溃。全身枯槁羸惫,神气黯然,如蜡人院之偶像,毫无生气。或成痴愚,或成肺痨癫痫,或致自杀,或卒倒夭死,或幸免早殇,而长为病夫以终身焉。夫无论何事,皆可防患于未然,独至手淫之恶习,暗室亏心,负惭衾影,为父兄不及知,为师长不及觉。欲防之而不胜其防,故其为害有如是之剧烈也。(《少年进德录》)
\end{quote}

\begin{quote}\it
    若耽于荒淫,则渐渐志识昏迷,心神衰耗,即使年少气盛,不即觉露,日复一日,终于不振,而百病随之。(《家庭宝筏》)
\end{quote}

手淫的危害是很多的,并不是无害的,当性学家崛起后,无害论就渐渐成了主流,但真相迟早会回归,人们会再次认识到手淫的危害,这是必定的。在色情泛滥的时代,更需要普及色情与手淫的危害,让更多的人了解真相,从而自觉戒除恶习。手淫恶习真的可以毁掉一个人的一生,当身体差了,很多事情自然就做不好了,甚至都不想去做了,身体宝贵的精华被肆意耗损,全身都会出现各种伤精症状,最后沦为废人一个,整日与药为伍。古人云:“人生天地间,圣贤豪杰,惟其所为。然须有十分精神,方做得十分事业。苟不知节欲以保守精神,虽有绝大志量,神昏力倦,未有不半途而废者。欲火焚烧,精髓易竭,遂至窒其聪明,短其思虑。有用之人,不数年而废为无用,而且渐成痨瘵之疾。盖不必常近女色,只此独居时展转一念,遂足丧其生而有余。故孙真人云:莫教引动虚阳发,精竭容枯百病侵。盖谓此也。人生欲念不兴,则精气舒布五脏,荣卫百脉。及欲念一起,欲火炽然,翕撮五脏,精髓流溢,从命门宣泄而出。即尚未泄出,而欲心既动,如以烈火烧锅内之水,立见消竭,未几则水干而锅炸矣。”

\textit{普劝青年志士,黄卷名流,发觉悟之心,破色魔之障。芙蓉白面,须知带肉骷髅;美貌红妆,不过蒙衣漏厕。纵对如玉如花之貌,皆存若姊若母之心。未犯淫邪者,宜防失足;曾行恶事者,务即回头。更祈展转流通,迭相化导。必使在在齐归觉路,人人共出迷津。(《欲海回狂》)} 欲海是十分危险的,大家可以想象一下无数的人漂在欲海上,而海平面下就是大白鲨在转悠,指不定哪天就把人拖进深渊。黄孝直曰:“少年时能于此色欲一关,把得牢,截得断。他年元神不亏,气塞两间。立朝之日。精神得以运其经济,作掀天事业。真人品真学问,皆由于此。即使不成大器,亦必克尽其天年,不致死于非命,此少年所当猛省也。”男儿欲遂青云志,须信人间红粉空!男子汉大丈夫应该光明磊落,顶天立地,岂可沉迷邪淫,毁于色情之祸与手淫之陋习?!\textit{彼无智慧人,行于畜生法;驰趣于女色,犹猪乐粪秽。(《大宝积经》)} \textit{如粪虫乐屎,贪淫者亦然。(《受十善戒经》)} 真正有智慧的人是不会贪淫的,他们深知贪淫的过患,一位戒友曾说:“为一层皮所惑的淫虫,活得真真可怜。”淫虫两字用得很好,从高楼望下去,人就像一只只小虫一样,邪淫者是恋皮的淫虫,看不破那层皮,迷恋那层皮,如粪虫乐屎,就像蛆虫在粪坑里翻滚,以屎为乐。

前段时间看见有戒友发帖说:“戒色难还是考清华难?”回复中有几个说考清华难,考清华北大的确很难,我查了下数据,2017 年清华北大共录取 6558 人,而高考总报名人数为 923 万,录取率仅为 0.07\%,也就是平均 1 万人中录取7个,1400 多人才出一个清华北大。我做学生党时,根本不敢想清华北大,那是真正尖子生去的地方,而我读书时只能算中等,文科有拔尖过,但其他都很一般,沉迷手淫加上严重鼻炎导致我脑子迟钝,我觉得自己就是想破脑袋也考不上清华北大,能考上清华北大的都是天赋极其出众、学习又勤奋努力的学生。从录取率这个角度来看,考上清华的确比戒色难很多,但戒色也有自身的难点,我想说的是,能考上清华北大的不一定能戒色成功,因为考清华北大用的是思维,思维就是念头,而戒色是控制念头,断除邪念,一个是用,一个是断!能用念头的人不一定能断除念头。清华博士偷拍偷窥,这类事件也上过新闻,即使上了名校也可能过不了色关,甚至因为邪淫被开除。色关是人生中必须要过的大关,如果过不了这关,即使将来事业取得了成功,也可能因为放纵自己而导致身败名裂,最后一败涂地。

\subsubsection{君子必戒邪淫}

邪淫范畴很广,手淫、婚前性、婚外情、一夜情、嫖娼、约炮等都属于邪淫的范畴,这是大致的分类,具体细分还有很多。

\textit{淫欲为害,伤身丧志,虽属夫妻,亦当节制。若是邪淫,更非所宜,古今志士,无一犯之。(《德育启蒙》)} 邪淫这个概念真的太重要、太关键了,现在电视里也讲传统文化圣贤教育,但几乎没见哪位教授讲到邪淫的定义和邪淫的危害,这部分内容好像被他们忽略了。其实戒邪淫是传统文化中非常重要非常关键的一个部分,君子一定要懂得戒邪淫,否则就可能沦为伪君子,表面一套,背地里一套,表面道貌岸然,背地里疯狂邪淫,反差非常之大。现在很多年轻人都不知道邪淫的定义,他们脑子里塞满了无害论和纵欲享乐主义思想,我以前和他们并无二致,也相信过无害论,也觉得人生得意须尽欢,要尽可能地满足和放纵自己的欲望,因为大部分人都是这样,我也觉得自己没什么不正常。什么是正常?难道邪淫放纵的生活就是正常?当然不是!后来我才知道戒色自律的生活才是正常人的生活,以前活颠倒了,活在了纵欲的鸡粪层,一个非常可悲的层面。真正的君子懂得邪淫的危害,知道戒邪淫的重要性,戒邪淫可谓安身立命之本,邪淫是个大窟窿,这个窟窿一定要堵上。戒色就是戒邪淫,戒色的好处是非常多的,戒色是很美好的体验,并不像某些文章误导的那样,说禁欲压抑和有害,相反,懂得修心的禁欲是非常健康的,不会感到压抑,并且会释放大量的快乐和美好的体验。

下面分享一些戒色案例。

\begin{case}
    我这个戒了一个月的人,都能感觉到特别明显的变化,比如以前一天睡七八个小时还整天哈欠连连,现在睡四五个小时一天精神百倍。以前想都不敢想的东西,现在居然也敢于去努力挑战奋斗一把。
\end{case}

\begin{case}
    戒色这么久,真的是脱胎换骨,见了我的好多人都说我像换了个人似的,不管是性格、成绩、外貌、懂事方面与以前都是天壤之别。
\end{case}

\begin{case}
    就在今年 8 月份,终于醒悟了,开始戒色,认认真真,现在效果终于明显了,脸上的痘没了,尿沥也少了,感觉生活又重新充满阳光。以前朋友都很讨厌我,现在大家也都感觉我性格变得非常好,人缘也好了起来。
\end{case}

\begin{case}
    戒色满一年,精气神归来。很感恩遇上戒色吧,感恩戒色吧的前辈们,没戒色前各种身心扭曲痛苦都伴随着戒色、学习戒色文章随之远去,现在的我感觉浑身充满精力,眼睛也像撑开了一片天,纯净的正能量集于一身!满满的正能量让我冲破心魔回归当下!
\end{case}

\begin{case}
    戒色 101 天,社恐好了很多,胆气壮了!吃了五年的西药,状态也没这么好过,五官重新有了坚固之相,眉毛比以前浓密许多也很黑,皮肤也好了很多,照镜子是一方面,洗脸的时候很快能感觉出来,总之是人变帅了很多。整整尿了一年的泡沫,现在基本上没什么泡沫了。现在阳刚之气恢复了不少,已经开始有了阳刚之美,眼睛从以前的浑浊暗淡无神变得很亮!
\end{case}

\begin{case}
    我戒色 20 天了,已经感觉到戒色的好处。有一股能量进入全身,自信多了,敢跟人说话了而且很自然,很积极乐观,敢跟人直视了。
\end{case}

\begin{case}
    戒色 30 天,小目标完成,说下我的感受,首先,社恐好多了,敢说话了,脑子很清晰,说话不语无伦次了,不紧张了,不自卑了,然后,很精神,一天到晚不那么累了,仅仅一个月,变化就不小。
\end{case}

\begin{case}
    戒色 198 天,今天是我戒色以来感觉最好的一天,戒色真的会让你变快乐。
\end{case}

\begin{case}
    飞翔哥我又重新找回纯净愉悦的感觉了,这种感觉真的让我泪流满面。
\end{case}

\begin{case}
    其实我觉得戒色给我最大的好处,就是心态改变了,有信心有能力做好很多事,并且我坚信纯净的人散发出的气场绝对是最强的!
\end{case}

\begin{case}
    随着戒色天数和自己的调整,心态豁然开朗,变得开朗外向了,喜欢与外界交流,内心的阴暗也慢慢消失了,非常喜欢这种感觉。
\end{case}

\begin{case}
    戒色 506 天,精神上,我从一个郁郁寡欢、行将垂暮的人到现在精神振奋,充满智慧。身体上从一个朽木到现在似乎年轻了十岁的感觉,真的是天上地下的差别。
\end{case}

\begin{case}
    那种自由纯净的感觉真的太美好,状态最好的时候是戒到三个月的时候,感觉看见什么都是美好的,我们一定都会找回那种感觉,那是 SY 没法比的。
\end{case}

\begin{case}
    戒了 283 天自身的一些变化,以前社恐严重整个人气场很低,走在大马路上不敢抬头走路,不知道为什么就是怕跟别人对视,戒色期间社恐慢慢好了很多,戒到 200 天的时候社恐已经完全消失,整个人阳光了很多,还有,戒色期间的那种成就感是真的很让人安心、满足,最重要的是戒色期间个头还长高了。
\end{case}

\begin{case}
    完全戒色已经两年了,最大的感受除了健康带来的无穷快乐,还有一种宽广的自由感,当一个人离开色欲的束缚,你就会感受到天地自然如此广大、如此广阔的爱。
\end{case}

\begin{case}
    戒色期间的感觉真的很好,即使一无所有,我也自在干净,纯净的感觉真美好。
\end{case}

\begin{case}
    现在享受着每天不撸管带来的快乐,这种快乐可以持续一整天,我的容貌已经开启自动恢复状态,一天一个样,原来戒撸真的可以给人带来很多快乐,以前想不明白的事情也慢慢感悟了。
\end{case}

\begin{case}
    我真的好怀念戒色三个月时那种底气自信,对人生未来充满希望的感觉。
\end{case}

\begin{case}
    戒色第 22 天,整个人变化真的大,眼睛更有神,更好看了,头皮基本不痒了,头皮屑也变少了,洗头发的时候脱发也没之前多了,反正就是越戒越开心,以前撸管的时候每天都感觉开心不起来,现在心情也有所改变。
\end{case}

\begin{case}
    第一次戒色超过一个月,那一个月真的很快乐,那个月我的痘痘全部消失,皮肤更好,身体很有精神,每天打篮球,身体恢复很好,脑力充足,仿佛回到了小时候那种自信浑身都是劲的状态。
\end{case}

\begin{case}
    戒到 100 天时,痘痘消了,脸变光滑了,变白,精气神很好。
\end{case}

\begin{case}
    戒了几个月,我开始感觉到重生的幸福感,这是这么多年以来未曾感受到的。
\end{case}

\begin{case}
    戒色一年多我痘痘少多了,脸部光滑了,大家都说我变白了,自己也变得自信起来了,都说我气质变好了,以前慢跑的时候肾特疼,现在运动会跑步还能得个亚军。
\end{case}

\begin{case}
    现在已经戒色 135 天,内心有一种愉悦感,可以伴随我一整天。
\end{case}

\begin{case}
    戒到九十多天的时候我发现有许多症状减轻了许多,甚至有的完全消失了。比如社恐,现在的我人缘莫名其妙地好了起来,也不怕和异性对视了。工作的时候脑力也开始变得充沛,学什么做什么都快了很多,处理问题也全面了。现在的我感觉是病树春抽芽,重新焕发生机。
\end{case}

\begin{case}
    今天是我戒色第 385 天,感觉很不错,现在有学习中国传统文化,让我懂得很多做人的道理,最大的变化就是身心的变化,戒色前整个人很消极,充满负能量,戒色 385 天后积极了很多,乐观了很多,人也充满正能量,喜欢与人交际。
\end{case}

\begin{case}
    我来戒色吧多年,2016 我才真正顿悟戒除邪淫,身体基本恢复,比如尿频尿急痘痘眼袋,从 14 岁长痘痘的我一直到 25 岁满脸前胸后背全是痘痘,我从没想过皮肤变得现在这样好,没吃任何药物,记住戒为良药!这四个字!
\end{case}

\begin{case}
    戒色 110 天了,感觉恢复很多,首先精力真的变好了,出油改善,发质略有好转,最惊喜的是额头两侧开始长出少量的新头发,精索疼痛好转了,继续坚持,相信后面还会有更大的惊喜。
\end{case}

\begin{case}
    戒色两个多月了,脑子不再迷糊了,神经症也恢复了不少。现在终于有了以前纯真的感觉,真的很美好!
\end{case}

\begin{case}
    戒色 174 天,期末考试成绩下来了,年级名次进步了 130 多名,感谢戒色让我的脑力恢复,在我有生之年遇到戒色吧是我的头等喜事!
\end{case}

这三十个戒色案例充分说明了戒色的好处,戒色后真的可以焕然一新,充满精力,充满精气神,如果伤得不深,戒色一个月,配合积极锻炼,就能恢复很多。有了精力,有了精气神,做起事情来就很有动力,很有斗志,很有效率,不容易犯低级错误。一个好状态真的太重要了,手淫恶习削弱了一个人的能量,让人陷入低能量的状态,到时一切都会变得灰暗颓废。即使再大的雄心壮志,也会在几发过后,化为乌有,到时什么也不想干,只想倒头睡觉,都懒得动弹。戒色之后能量上来了,心态也跟着好了,变得积极乐观,懂得包容了,人缘也好了,有自信了,敢于直视了。那种正能量、积极向上的状态真的很棒,身体症状逐步缓解消失,也让自己有一种解脱的轻松感。戒色后一股强大的能量把你整个人撑了起来,你开始顶天立地了,有一种浩然正气的感觉。戒色的好处真的很多,可以让人感受到纯净的大快乐,这是更高级的快乐,不是邪淫的低级趣味所能比的,能量得到提升的经验是非常圣洁的,你内心的圣殿开始重新光辉起来。戒色的灵魂是轻盈与自由的,内心是祥和与欢乐的,那是非常纯粹美好的感觉,你会感受到强烈的喜悦感和幸福感。一位戒友说:“不撸的日子比金子还珍贵!”那种纯净美好的感受,那种积极向上的斗志,那种底气和自信,那种一整天精力充沛的感觉,那种持续的愉悦感,岂是手淫几秒快感可比的?另外一位戒友说:“戒色太爽了,比撸管滥泄爽千万倍!积满精气神,干一番轰轰烈烈的大事业!戒色是飞龙在天的大爽,邪淫就是蛆拱屎的爽。”他说得很精辟,正所谓:金鳞岂是池中物,一戒邪淫便化龙!龙飞九天壮神州,戒色大爽振华夏!

一位国外戒友说:“在手淫过后我一度很消沉,我根本就没法和他人保持眼神接触,在社交场所中我也觉得自己笨拙、别扭。因为我失去了平衡情绪的能力,因为我一直在滥泄着作为男人最有力、最重要的能量:我的肾精。你可以感受到一种如同液体般真实的、自然的力量在你身体中流淌!在你的身体里有一种神奇的液体,这种东西能创造生命!想一想这种东西有多么疯狂、多么有力啊!那是在你体内的一股纯粹力量。当你持续地倾泻它,当然你就会开始感到难受,这就好比把你身体中所有的力量全部泄去。我花了很多很多年的时间才明白看黄是多么有害!戒掉它能使你加深对自己、所处环境以及整个世界的理解。这是一次难得的极棒的机会,等待你采取行动去抓住它。看黄手淫对于你和处于色情行业的人来说是灾难性的!我想把这个简单却又极其重要的一句话深深刻在此刻你那腐朽不堪的灵魂上,你准备好了吗?看起来简单容易的选择(邪淫),会给你更艰难的生活,而那些看起来艰难费力的选择(戒色),反而让你的生活变得轻松!”

国外戒友这段话说得很好,手淫后人容易消沉,也会变得颓废,社交能力会下降,容易变得紧张不自然,很难与人对视。肾精太宝贵了,就是人体的核能,这股能量如果能留得住,那精力、脑力、体能都能保持在上佳的状态,一旦泄掉,就会感觉整个人软掉了,一股巨大的力量被抽走了,人容易感觉疲倦,反应也变得迟钝,还有各种伤精症状会陆续出现。看黄手淫对自己的确是灾难性的,刚开始还以为是喜剧片,天下居然有这么爽的事情,后来这部喜剧片逐渐演变成了灾难片,最后是恐怖片。对于色情行业的人也是灾难性的,因为他们造作了极大的恶业,最后必定会受到惨烈的恶报。邪淫很容易,堕落就像坐滑滑梯一样,戒色貌似很难,但只要掌握正确的方法和原理,是完全可以戒掉的,当你戒掉了,你就超越了邪淫的束缚,生活也会变得轻松许多,你开始活在更高的灵魂维度,感受到更高级的快乐。全身神气饱满,双目光明,内力浑厚,对生活和未来充满希望。

一位戒除 415 天的戒友说:“回首自己邪淫的日子,我最大的感受就是总有一种东西在束缚着自己,就是开心不起来,总感觉难过和痛苦,同时对于生活没有热爱,不懂得爱自己,也不懂得爱别人,每天脑海中的想法都是负面和消极的,喜欢对于别人妄加评论,同时对于女孩子有严重的意淫邪淫的想法,身体也出现虚胖、神经衰弱等多种症状,总之自己的状态就像一个泄气的皮球,永远没有出头之日,我想戒,可我就是做不到。后来,我通过学习提高觉悟以后,反转了心魔的统治,做回了自己,自己的心变得宽容很多,同时邪淫的想法几乎消失,身体通过锻炼也变得匀称,有一种内壮的感觉,精力也比从前好很多。神经症恢复,身体在向好的方向发展,好久没这样火力全开了,我才意识到,自己从前在邪淫上浪费了多少时间和精力,自己荒废了多少青春和美好,有时,只为了一两秒钟的快感,会通宵熬夜,甚至花一下午、一天的时间找黄看黄,最后自己沦为废渣,这样的生活终于终止了,我终于不用再偷偷摸摸地干坏事情了,因为我做回了身体的主人。”

这位戒友表述的感受很细腻很到位,邪淫的日子其实是不快乐的,总有一种被束缚的感觉,开心不起来,快感过后就是抑郁,就是空虚和悔恨,就是失落、颓废与更大的不满足。症状爆发后,还要受到症状的反复折磨,还要去医院看病,人财两空。因为起了很多邪念,导致自己充满负能量,生活中很多事情都会变得糟糕不顺,见了异性就意淫,实在太猥琐不堪了,一点正能量都没有。这位戒友后来逆袭了,反转了心魔的统治,主宰了自己的内心,做回了身体的主人。在邪淫上,我们浪费了太多的时间和精力,荒废和蹉跎了青春,我之前为了找黄看黄甚至废寝忘食,为了那几秒的快感而走火入魔,根本停不下来,太疯狂了,找黄看黄时,分针仿佛在快速旋转,一会天就黑了,一会几小时就过去了。这位戒友最后两句说得很好,再也不用偷偷摸摸了,可以光明正大做自己了,可以主宰自己的身体了,这种感觉才是主人的感觉,而不是心魔的傀儡、欲望的奴隶。

\subsubsection{圣贤教育的召唤}

小学时的课本就有圣贤教育,还记得那时同学们早读时朗诵“子曰”,大家一起读,感觉很有氛围,小时候的确受到了圣贤教育的良好熏陶,然而到了初中开始堕落后,内心就渐渐远离了圣贤教育,脑子里被邪淫的内容占据着,活得就像一具行尸走肉,黄片的冲击实在太大了,那种诱惑也实在太强了,我能切身感受到黄片那种如毒品般的强烈吸引力,我根本无法抵御那种诱惑,别说青少年无法抵御,就算成年人都很难抵御。黄片似乎是“硬通货”,各个国家各个阶层各个行业的人都在看,看过报道,有的不良网站的月浏览量高达几十亿,非常惊人的数字,那种点击、点击、再点击的节奏,相信每位戒友之前都有经历过,这种点击可以持续几个小时,让人深度沉迷其中。一位戒友说:“短暂的堕落过后我又再次猛然惊醒!我心想,不能再这么堕落下去了啊,再这样下去,人生还是依然无尽的痛苦,我的人生还是无边的黑暗!我不想再这么痛苦下去了,我想从痛苦的人生中解脱出来!所以我又再次振作起来戒色。”这位戒友的想法和我过去一样,每次沉沦每次堕落都伴随着空虚和失落,前一秒似乎还很兴奋,一旦射掉,仿佛整个世界突然就灰暗了,整个生活都变得无意义,没意思了,很多人都厌恶堕落的自己,但心魔附体后,又开始找黄了,我发现找黄看黄时激发的那种专注是非常强的,把所有潜力都给激发出来了,如果把这种专注用在其他正确的方面,应该早已成就了一番伟业,那种投入和专注真的很强,全身都铆足了劲,可惜用在了错误的方面。

另外一位戒友说:“你想从深渊里逃出来,可是却总有一股无形的力量再把你拉回去!!那时候的瘾很大,稍微一丁点刺激都会想撸,那时候欲望说来就来,说撸就撸,整个人像个僵尸一样,脑子里一片浆糊,没有感情,没有人性,只知道撸,我的世界里除了欲望仿佛什么也没有了。”他这段描述很细致,那种疯狂放纵的状态的确如此,撸到一定程度,人性和良知都没了,只知道撸,看到异性就意淫,想到什么都会联想到邪淫,浑身上下一点正能量都没有。记得我那邪淫的十几年,我做的善事真的少得可怜,十几年都拿不出一件像样的善事,记得印尼海啸那年,我捐了一天工资,那也是单位领导要求的,我也是随声应和,自己并无主动行善的想法和意识。那十几年,圣贤教育离我真的好远好远,在那种疯狂邪淫的状态下,我也不可能学得进去,邪淫的状态往往很浮躁,看不进去,我那时还是一个愤青,经常有偏激和愤恨的想法,总之负能量实在很重,我的振动频率已经跌落到非常低级的层面。我在等待一次触底反弹,一次改变整个命运走向的绝杀!这是真正的压哨绝杀,就在我陷入神经症的症状地狱,人生进入了至暗时刻,每天吃药,经常跑医院,每天活在深深的恐慌和绝望中,几十种症状轮番折磨我,我活得好惶恐,好痛苦,都想自杀了,就在这最后时刻,我奋起了,发出了最后的怒吼,我表明了无比强硬的立场,我要逆袭!我再也不要过这样堕落的生活!!!

我下了死决心,这次是抬棺出战!绝对是破釜沉舟的勇气,就像全身绑满炸药的勇士,以必死的决心来戒色,我知道这是殊死一战最后一搏,这次戒不了,也许我这辈子就彻底完了。我又开始戒色了,这次戒色注定与以往不同,我开始在网上学习圣贤教育,学习大德开示,看儒释道的文章,看传统文化的经典,看中医养生知识。以往我是一味靠毅力强戒,每次都挺不过一个月,而这次我学会用知识武装头脑,开始提升觉悟水平了,也是从那时我开始总结以往失败的经验教训,开始深入研究戒色的原理和规律。我相信这是圣贤教育的召唤,圣贤教育就像一位充满智慧的老人,他在等我回头,等我聆听他的教诲,等我学会修心,等我蜕变,等我带动更多的人一起戒色,来完成崇高的使命。

儒释道的文章和书籍都是不错的,我做了很多的笔记,记了很多本,我最近翻看了那时的笔记,还能感受到当时的冲劲,那种疯狂学习的热情与兴致,那种热火朝天的干劲。儒释道都属于传统文化圣贤教育,但真正让我懂得修心学会修心的则是佛法,佛法是最高层的教诲。遇见佛法真的很幸运,当初听佛乐时,我流下了眼泪,似乎被某种慈悲、善良和圣洁的频率所深深触动了,触动了心底最纯真最柔软的部分。我以前对佛法是完全不懂,对佛教也存在一些误解和偏见,我认为是受了刺激后才会去出家,也觉得佛教好像很消极,也有迷信的成分,同时也觉得佛法太高深,不是凡夫俗子可以理解的,那些经文实在太深奥,根本读不懂。那时对佛教的理解就是这样矛盾,一方面觉得迷信,另外一方面又觉得太高深,让我有所敬畏。后来明理后才知道,佛法并不消极,佛菩萨在积极地度人,在积极地弘扬正能量,佛法也不迷信,是某些人停留在迷信的层次,把佛教降低为求财、求平安、求福报。真正的佛法是要明理,是要懂得修心的,很多人刚学佛时就觉得学佛要穿得破烂,吃得差,以为这就是学佛,其实真正要改变的是心,修行的根本在于修心,而不在于外相。外相可以穿得像乞丐一样,但内心烦恼还是很重,那就是表里不一。

能够遇见佛法,悟明佛法的道理,这也是我以前不敢想的,如果放在以前,我都不敢相信自己会和佛法能有什么联系,自己怎么可能会弄懂佛法的道理,根本不可能,放在那时,我也不相信自己可以做到,那时的我沉沦在欲海中不能自拔,佛法离我简直十万八千里,后来我开始学习佛法了,刚开始觉得特难懂,如堕五里雾中,完全摸不着头脑,那些名词概念就像天书一样,根本看不懂,即使看了解释也理解不了,那个阶段的我,悟性真的很普通,很一般。那时我会经常听佛经,比如《金刚经》,感觉朗读起来很优美,很有古韵,所以去公园时经常会在耳机里听《金刚经》。其实很多社会的高层人士都在学佛,他们不一定会去寺院皈依,但至少会学习下佛法的智慧,对指导人生还是很有帮助的,国外戒色 NOFAP 的坛主也有学习佛法,看到他的文章也引用了好几段佛陀的教诲。对于圣贤的教诲,我们要放下成见和偏见,以一颗包容的心态来学习,这样就可以获得很大的益处。

学佛后我看了很多大德开示,对很多大德都比较崇敬,在学佛几年后我渐渐确立了四位根本上师,分别是:元音老人、黄念祖老居士、宣化上人和印光大师。这四位大德对我影响很深,对我的法益也很大。能够得遇恩师,实在是最大的福报,不是世间任何财富地位可以比的。最早顶礼和皈依的是宣化上人,刚开始我在读宣化上人版的《楞严咒》,那时在群里看到有人推荐念佛持咒,说是对身体好,有各种好处,那时我自己了解了一下,最后选了《楞严咒》,《楞严咒》是最重要的一部咒,是咒中之王,也是咒里边最长的一个咒。那时差不多跪念了一年半的《楞严咒》,的确是有加持,感觉自己的悟性渐渐上来了,《楞严咒》很长,能够全部学会,整个读下来,都需要一定的时间,现在我主要念阿弥陀佛和往生咒。用现代科学来解释,原理就是念佛持咒可以提升你的振动频率,净化你的心灵,提升你的能量场。有的戒友在念准提咒,有的念大悲咒,都是可以的,可以根据自己的因缘和喜好来选择。宣化上人在美国旧金山创立了万佛圣城,他是将佛教传入西方世界的先驱者之一,是在美国建立三宝第一人。宣化上人的十八大愿,震撼了我,特别是“若有一未成佛时,我誓不取正觉。”“愿将法界众生所有一切苦难,悉皆与我一人代受。”这两句实在太伟大了,真的是高风亮节无伦比。宣化上人的很多开示都一针见血,非常直接和犀利,这是我非常喜欢的风格,我也非常尊崇宣化上人崇高的德行和功业。

大概在我戒到一年多的时候,那时我对修心已经有了一定的理解和掌握,戒色也相对稳定了,但缺少进一步的提高,就在这时,元音老人出现了!还记得那个下午,我点开了视频,那个亲切而慈悲的声音传来,我瞬间就被吸引住了,那是一个我等待许久的声音,似乎我整个生命都在等待这样一个声音,非常慈悲,非常恳切,感人至深。那是一段直指本来面目的开示,一下把我给震住了!在我听完那段开示之后,还在久久回味,听了一遍又一遍。那一年,生命中指点我认识真我的恩师出现了,他就是元音老人,“音容笑貌师宛在,一睹慈颜心神往”,虽然未曾谋面,但我对恩师却感觉格外亲切,有一种似曾相识的感觉,他是我一直在寻找的人,一直在等待的人,我多次面对恩师相片热泪盈眶,真的是感激不已,无以言表。元音老人朴实和蔼的作风就像邻家老爷爷,非常平易近人,非常慈悲,恩师在 2000 年弥勒菩萨圣诞那天从容站立往生了,在我心中,元音老人就是弥勒菩萨的化身。那个下午听到了那个直指的开示,我当时对于真我的体会就是四个字——平淡无奇!非常平常,就像白开水一样,无味乃有至味,真味是无味。我当时就承当了!感觉元音老人说得很对,我知道那就是我的真我,虽然看似平淡简单,但深入那个状态后却感觉极其深奥。当时毕竟是第一次听到,还不是很稳固,很多问题还不是很明了,后来又研读了元音老人的开示,真的是豁开正眼,受益匪浅。我在下载包里也专门分享了元音老人的文集,有缘的戒友可以好好学习一下,元音老人的开示不仅指点了真我,也格外强调断念保护,对断念实战极为重视,我继承了元音老人的开示重点,把断念实战放在了一个极其重要和关键的位置。从研读元音老人的著作中,我对修心,对观心断念有了更深刻、更全面的认识,元音老人对我的法恩实在太大了,真的就是我的法身父母!我愿生生世世永远报答根本上师元音老人的大恩大德!元音老人那句“甘作春泥群芳护”是我极为欣赏、敬仰和赞叹的,我愿继承和发扬恩师这种无私奉献的崇高精神。

我把一种强硬的气质注入到了我的戒色文章中,而这种强硬勇猛的气质很大程度来源于黄念祖老居士的宝贵开示,他说:“拼死念、念到死!”说的是勇猛念佛。我用到戒色方面,就是“拼死戒、戒到死!”大家读到这六个字,决心和勇气一下就上来了。黄念祖老居士还有一段开示也极为给力:“我们今生要奋发大志,决定在这一生之中,拔除多劫以来生死根本。这是冲天的大志,是多劫以来空前的壮举,是真要自觉觉他的大心。所以要排除万难奋不顾身,就像在敌人重重包围之中杀出一条血路。这是你(业力)死我(真心)活的战斗。要用真刀真枪,不再是表演戏台上的花拳绣腿。”杀出一条血路,这是你死我活的战斗,这段开示真的是太给力了!我对白胡子睿智慈祥亲切的老爷爷天生就有一种好感,而黄念祖老居士就符合这个形象,元音老人在上海,黄念祖老居士在北京,一南一北,都是泰斗级的大德,那股能量场实在太磅礴、太震撼、太不可思议了,就像云中黄山一样,我彻底被折服了。黄念祖老居士法号龙尊,亦号心示、乐生,别号老念、不退翁,老居士的书法和画作也非常好,修养真的很深,解放后任北洋大学、天津大学、北京邮电学院教授,是具有教授气质的一位稀有大德。网上有一张黄念祖老居士的照片,老居士端坐在那,满面红光,表情是那么神圣庄严,眼睛是那么睿智慈悲,那种慈爱的注视,眼神中流露出的慈悲,真的感人至深,震撼了很多人。感觉整张照片都在放光一样,黄老嘴角挂着一抹微笑,原来慈悲可以这样帅气,真的太罕见了,显得那么睿智、慈祥和亲切,黄老年轻时的照片我也看过,真的是一表人才。黄老一九四三年皈依当代禅宗大德虚云老法师,深受法要,直达禅宗巅峰,得无上妙谛。文革动乱中,黄老历经磨炼艰危,而修持无片刻松懈,相反愈加勇猛精进,所获真实利益不可胜记,正如大师悬记“唯艰难困苦备尝之矣,方可成就。”数次遇死,均安定持诵,将生死置之度外,完全放下,安然渡过,尤其一次遭遇龙卷风,周围物件房屋全部扫光,而黄老泰然无损,仍直立于原地,获大进展。黄老素怀传灯之志,弘扬净土之愿,拯救群生之望,也为报佛恩、师恩,遍观众经论,苦心参研、构思酝酿。自一九七九年经二年,闭门谢客,专心注释大经,于一九八一年完成初稿。一九八二年完成二稿,在严重疾病折磨下悲心更切,依然矢志不渝,奋力完成三稿,时为一九八四年、历时六载,竣稿付印,于一九八七年《大经解》流通于海外。黄念祖老居士的开示非常殊胜,建议有佛缘的戒友多学习,读黄念祖老居士的开示我有一种特别的感觉,那就是他的很多话都来自于他的亲身经历,完全是第一手经验,口语化的表述风格更加强化了这种特别的感觉,总之很令人信服。

印光大师,法名圣量,别号常惭愧僧。大师振兴佛教尤其是净土宗居功至伟,是对中国近代佛教影响最深远的人物之一。大师在佛教徒中威望极高,与近代高僧虚云、太虚、谛闲等大师均为好友,弘一大师更是拜其为师,其在当代净土宗信众中的地位至今无人能及。民国二十九年(八十岁),农历十一月初四,大师预知时至,端坐念佛,安祥生西,大师一生严持毗尼,一丝不苟,六时念佛,三业清净,护教重道,勤奋修学,言传身教,为人师表,弘扬净土,不遗余力,皈依弟子,众星拱辰,后人尊大师为净土宗第十三祖。印老自奉极薄,食则唯求充饥,不求适口;衣则唯求御寒,厌弃华丽。如果有人供养他珍美的衣食,他却而不受,不得已受下,就拿来转赠别人。如果是普通物品,就交到库房,由大众共享。他虽薄以待己,却厚以待人,凡善男信女供养的香敬,他都拿来印佛书流通,为人种福田。印光大师的德行非常深厚,平淡朴实的家风也是我极为仰慕的,大师教会了我做人的道理,敦伦尽分,闲邪存诚,诸恶莫作,众善奉行,也教会了我戒邪淫,也让我对净土宗有了更深入的认识,我现在修的就是净土宗,弥陀圣号 + 《阿弥陀经》 + 往生咒,每日有定课。印光大师对我的法益非常之深,特别是戒邪淫这块,印光大师几乎是每一位戒色者的导师,印光大师亲自整理修订流通《寿康宝鉴》,为《寿康宝鉴》作序,刻印《欲海回狂》,这两本书籍已经成为传统戒色的经典之作,几乎每位戒色者都有学习过。我很尊重传统戒色,也尊重国外的戒色文章,对于好的戒色文章和书籍,我都是本着一颗包容的心态来对待的。建议有缘的戒友能够好好学习《印光大师文钞》,此文钞是非常殊胜的法宝,文钞中印光大师说:“光老矣,不能常训示汝。”这句话曾经看得我流泪了,大师真的是非常慈悲的人,就像祖辈在语重心长地谆谆教诲自己的孙子。

印光大师关于戒色的开示摘录:

\begin{itemize}\it
    \item 世间聪明子弟,于情窦开时,其父母兄师不为详示利害,以致由手淫与邪淫送命者居大半。能不即死,也成残废,无可成立。
    \item 汝年尚幼,须极力注意于保身。当详看安士书中欲海回狂,及寿康宝鉴。多有少年情欲念起,遂致手淫,此事伤身极大,切不可犯。犯则戕贼自身,污浊自心。将有用之身体,作少亡,或孱弱无所树立之废人。又要日日省察身心过愆,庶不至自害自戕。
    \item 后世子弟愈聪明,则欲心愈重,情窦未开,不可告。情窦已开,不为说保身寡欲之道,或致手淫邪淫,及已娶忘身徇欲,均所难免。
    \item 近世少年,多由情欲过重,或纵心冶游,或昵情妻妾,或意淫而暗伤精神,或手淫而泄弃至宝。由是体弱心怯,未老先衰。学问事业,皆无成就。甚至所生子女,皆属孱弱,或难成立。而自己寿命,亦不能如命长存,可不哀哉。汝恐亦犯如上诸病,有则改之,无则加勉。
    \item 汝云五六年来,自出校后,病骨支离,已同半死。得非燕朋相聚,共看小说。以致真精遗失,手淫相继,因兹有此现相乎。此现在学生中十有八九之通病也。以父母师友均不肯道及,故病者日见其多,而莫之能止也。光以此事排印寿康宝鉴印八百本,凡后生见光,必明与彼说其利害,令其保身勿犯也。纵手淫邪淫,均能守正不犯。而夫妇居室,亦须有节,兼知忌讳。庶可不致误送性命也。否则极好之人,或因此死。群归于命,而不知其自送性命也。
    \item 今之聪明子弟,多犯手淫之病。令看寿康宝鉴,及了凡四训,庶不至致成残疾,及短命而死之苦祸。
    \item 人之少年,最难制者为情欲。今之世道,专以导欲诲淫为目的。汝等虽有祖上阴德,不至大有逾越,然须战兢自守,庶可无愧先人。倘不着力立品,受淫欲之戕贼,后来决定无所成就,或致短命而死。
    \item 寿康宝鉴,常看,则不至犯邪淫与手淫等,自戕其生,自折其福寿,而即取残废与死亡也。
    \item 现在后生,已知人事,即当为彼说葆精保身之道。若知好歹,自不至以手淫为乐,以致或送性命,或成残废,并永贻弱种等诸祸。未省人事不可说,已省人事,若不说,则十有九犯此病,可怕之至。孟武伯问孝,子曰,父母唯其疾之忧。他疾,均无甚关系,冶游,手淫,贪房事,实最关紧要之事,故孔子以此告之。
    \item 后世人业重,情窦早开。十一二岁,便有欲念。欲念既起,无法制止。又不知保身之义,遂用手淫。如草木方生芽,而即去其甲,必致干枯。聪明子弟,由此送命者,不知凡几。即不至死,而身体孱弱,无所成立。及长而娶妻,父母师长绝不与说保身节欲之道。故多半病死,皆是由手淫及贪房事所致。故孔子答孟武伯问孝曰,父母唯其疾之忧,乃令戒房事。不戒房事,则百病丛生。能戒房事,则病少多矣。
    \item 然人从淫欲而生,故淫心最难制伏。如来令贪欲重者,作不净观,观之久久,则见色生厌矣。又若将所见一切女人,作母女姊妹想,生孝顺心、恭敬心,则淫欲恶念,无由而生矣。
    \item 至如夫妇相交,原非所禁,然须相敬如宾,为承宗祀,极当撙节,不可徒贪快乐,致丧身命。
    \item 不但外色不可淫,即夫妻正淫,亦当有限制,否则不是夭折,就是残废。贪房事者,儿女反不易生,即生亦难成人,即成人亦孱弱无所成就。
    \item 绮语者,说风流邪僻之话,令人心念淫荡。无知少年听久,必至邪淫以丧人格,或手淫以戕身命。
    \item 色欲一事,乃举世人之通病。不特中下之人,被色所迷。即上根之人,若不战兢自持,乾惕在念,则亦难免不被所迷。试观古今多少出格豪杰,固足为圣为贤,只由打不破此关,反为下愚不肖,兼复永堕恶道者,盖难胜数。
    \item 光常谓世人十分之中,四分由色欲而死四分虽不由色欲直接而死,因贪色欲亏损,受别种感触间接而死,其本乎命而死者,不过十分之一二而已。茫茫世界,芸芸人民,十有八九,由色欲死,可不哀哉。
    \item 今之少年,多半犯手淫病,此真杀身之一大利刃也,宜痛戒之。
\end{itemize}

\subsubsection{顿悟修心诀}

有的新人会问戒色可以不信佛吗?我回答当然可以,戒色吧不勉强任何人信佛,不走信仰戒色的路线,也可以走专业戒色的路线,戒色吧是一个包容的平台,信仰和专业都是可以的,但主要还是以专业为主,因为专业戒色的覆盖面最广,接受度也最大。刚开始很多新人是不太能接受佛法的,他们对佛法有误解和成见,很多人后来就慢慢接受了,学佛不是叫你去做和尚,也不是马上叫你皈依,而是先了解下,学习一下佛法的智慧,对自己的人生是很有好处的。当你深入学习佛法后,就知道佛法讲的是真理,佛法不是迷信,然后慢慢自然就接受了。就我个人的体会而言,佛法看似内容很多,但最核心的东西就是真我,而真我又是那么简单,就是最纯粹的那个意识状态。那个最核心的东西就是最单纯的状态,不是东西的东西,只是一种状态而已。专业 + 信仰,如虎添翼,专业戒色研究的是戒色的原理和规律,很多方面研究得很细致,已经掌握了戒色成功的规律,而信仰戒色有佛菩萨加持,有大德的开示,也非常殊胜。大家看了我的戒色经历,应该知道是专业 + 信仰,主要研究专业戒色,但也接受信仰的力量,这样戒得会更稳固,很多戒色前辈都是专业 + 信仰。这里的信仰不是叫你马上去皈依,而是学习佛法的智慧,有些内容不能马上接受,可以先放在一边,应该以包容的心态来学习。

我能戒到现在一次未破,和我顿悟修心诀是密切相关的,戒色的根本是修心,念头是行为的先导,如果能控制自己的念头,就能控制自己的行为,所以必须在起心动念上下功夫。这点认识至为关键,如果你把重心放在充实生活、转移注意力上,那是不究竟的,是治标不治本的方法,虽然也能戒除一段时间,但最终心魔猛烈进攻时,肯定会败下阵来。断念实战不行,迟早还是会破戒的。修心诀就是那十六个字:念起即断、念起不随、念起即觉、觉之即无!这个口诀想必大家都已经很熟悉了,这十六个字是这样划分的,4 + 4 + 8,前面四个字是总原则,也就是念起即断,不管什么断念方法,都是为了做到念起即断。紧接着四个字,念起不随,是念起即断的补充说明,念头起了,不要跟随,不要跟着跑,不随即是断!而最后八个字则是实战的最高法则,就是觉察消灭,用觉察力来消灭念头!这十六个字其实是一个整体,总原则就是为了断念。这个断念口诀,不用信佛也可以用,所以覆盖面最广,弄懂了原理,然后勤加练习,就能登堂入室,渐入佳境,最后出神入化,登峰造极。

我最早戒色时,通过不断学习大德开示,就发现了这个口诀,这个口诀很多大德都提到过的,这引起了我高度重视,我知道这个口诀不一般,是修心的绝对重点。我领悟了这个口诀,坚持练习观心断念,所以能越戒越好,我对断念实战极端重视,不管学了多少戒色文章,做了多少戒色笔记,最后就看念头上来时,实战的那一下。那一下到底行不行?到底快不快?到底狠不狠?如果那一下不行,那就注定会失败。就像一个运动员在比赛前刻苦训练,最后要到赛场上去实战检验,戒色最后的检验,就是念头袭脑的瞬间,就看你的实战表现。没有断念的那一下,说什么都是白扯!说什么都是纸上谈兵!脱离断念实战,必将失败!

\textit{实际上,修一切的法门都是在观心,都离不开观心,只不过表现的方式有所不同而已。‘观心’二字,可以概括佛教的一切修行法门,因此,这可以说是个大总持,是一个总的法门。(净慧长老)} 戒色首先要学会观心,观心就是看住自己的念头,就像在家里装了监控摄像头,时刻监视着大门,如果有贼进来了,马上赶出去。邪念就是贼,怂恿的念头也是贼,图像也是贼,微妙感觉也是贼,你要马上发现,立刻断除。

\textit{妄念人人皆有,然妄念起时,我自知之。知而不随,是谓不相续,不相续则我不为妄转。(虚云法师)} 这里的知,指的就是觉察,知道念头来了,发现念头来了,不要跟随,不要让念头相续,这样就不会随妄念转,知之一字,众妙之门,真正领会了这个知,这个觉,那就相当于掌握了激光武器一样,一知、一觉、一看,念头就灰飞烟灭了,瞬间实战就结束了。\textit{人若知这良知诀窍,随他多少邪思妄念,这里一觉,都自消融。(王阳明先生)} 最近一天早上吃早饭时,回忆的诱惑图像入侵了,上脑极快,非常微妙,极度诱惑,我一记觉察炮,一下就把它轰没影了,再敢上来试试?!心魔不敢动了,实力强大就会对心魔形成威慑力,有了威慑力,心魔就不敢轻举妄动,那股杀力一定要狠!狠到心魔颤抖!我在断念的时候,甚至眼睛也要瞪一下,好像周围有一群敌人要上来,我狠狠地瞪着对方,这种充满杀气的眼神,让对方不敢上来,让心魔知道我不是好惹的,我是断念的狠角色!心魔可以在张三李四头上作威作福,拉屎撒尿,在我这里,只有横尸的份!我强悍的断力也是逐步修炼出来的,刚开始也是断得不干净,断了之后,心魔还会发动更猛烈的进攻,邪念和图像上得更猛了,企图攻破我,把我拿下。后来我强大起来了,它们再上来试试?!等待它们的是觉察的激光炮,来多少念头怪,都能在瞬间统统杀灭!它们就是来送死的!!!戒到这种程度就开始享受杀念的快感了,这种快感比手淫的快感爽多了,这种快感来自于真正主宰自己的内心!

一位戒友说:“这些天默念断念口诀几百遍,再有念头冒出来的时候,一意识到念头来了,断念口诀立即映入脑海,就像菩萨的金刚杵一般殊胜,念头还没站稳就灰飞烟灭。”这位戒友在断念实战方面已经大有长进,已经有了强悍的断念表现,但他还需想起断念口诀,这就多了一步,最快的实战是一意识到念头来了,念头就没了,在知道的刹那,发现的刹那,念头就被消灭了,这是最快最强的实战表现。当然他现在的实战表现也还可以,继续练习,继续进步,最后就可以做到一觉即断、一知即灭的程度。刚开始要熟背这个口诀,尽快熟悉起来,正确理解和坚持练习,两者缺一不可,坚持练下去,自然会有长进的,但也不可急躁,要保持耐心,就像开水烧开之前会有一段时间没动静,但的确在加热。我们练习断念也是如此,练到一定程度就会感觉自己没进步,甚至有所退步,这时不要灰心,要坚持练下去。苏炳添的成绩进步史,也有一两年是不进步甚至退步的,但只要继续坚持下去,继续完成每日的积累,最后肯定会实现突破,到时就会有一个大的飞跃。断念口诀练到后来就不需要背诵了,只需觉察,到时自然能做到一觉即断,对实战的体会和领悟也会越来越深,实战的表现也会越来越强悍。

\begin{quote}\it
    此五根者,心为其主。是故汝等,当好制心。心之可畏,甚于毒蛇、恶兽、怨贼;大火越逸,未足喻也。……譬如狂象无钩,猿猴得树,腾跃踔踯,难可禁制。当急挫之,无令放逸。纵此心者,丧人善事。制之一处,无事不办。是故比丘,当勤精进,折伏汝心。……譬如牧牛之人,执杖视之,不令纵逸,犯人苗稼。(《佛遗教经》)
\end{quote}

佛陀在涅槃前说的最后一部经典《佛遗教经》里还专门强调了修心,心即是念头,要“制心”,要知道心的可畏,邪念就像毒蛇、恶兽、怨贼、大火、狂象、猿猴,要及时制服,要急挫之,要折伏之,不能让其起势。“执杖视之”,是非常给力的四个字,就像一个人拿着棍棒,警觉地看着,只要敢动,就给它来上狠狠的一下,让它知道你的厉害,这样它就不敢轻举妄动了。修行的根本在于修心,把握了这个核心,就能无往不胜。《金刚经》讲“降伏其心”,和“制心”一样,都是很强势的词汇,并不是叫你逃跑,也不是叫你转移注意力,而是明显的主战派!就是要制服它,降伏它!就是要够狠!要主宰自己的内心,要有强大的统治力,就像奥尼尔站在内线,谁敢上去扣篮?!

\begin{quote}\it
    胜人者有力,自胜者强。(《道德经》第三十三章)
\end{quote}

能战胜别人的人,是有力的人,能战胜自己的人,才是真正的强者。孔子讲“克己复礼”,克在文言文中有战胜之意,要战胜自己,使言行符合于礼。我们戒色就是要战胜自己的心魔,心魔就是负面的念头,各种会导致破戒的念头,各种邪念,能征服宇宙的人,不一定能征服自己的心魔,能主宰内心的人才是真正的强者。我之前戒色为什么总是失败,我现在回忆起来,根本原因就是缺少降伏心魔、断除邪念的能力,也没有戒色觉悟和实战意识,念头上来时,不能制心,不能降伏它,结果就是被其附体,任其摆布,就像一个木马病毒入侵你的电脑,杀毒软件没杀灭,于是这个木马病毒就开始操控你的电脑。我那时是十足的菜鸟,一次次中招,一次次沦陷,心魔的进攻套路一次次在我身上奏效,攻破我简直太容易了,几秒就搞定了,那时心魔攻破我是没有任何难度的,稍微一怂恿,我就听信了,记得那时心魔总是说“最后一次”、“下次再戒”、“偶尔一次没事的”、“只看看,不撸”,我总是听信,一旦开始看黄我就会撸,就会变得失控疯狂,撸到最后有一种幻灭的感觉,还有一种丧尸的即视感,鬼气浮面,双腿发软,人都站不稳了。之前看过一篇文章,说国外的网络安全专家一堂课要三万元,就是教你如何防御攻击,而戒色前辈的修心课程其实是无价的,何止三万元?修心的知识比黄金还宝贵,当你学会修心了,你就可以主宰自己了,不仅省下了医药费,你还能以良好的精神状态来奋斗自己的人生,赚取人生的财富,收获人生的幸福。迎接心魔的不应该是你的膝盖,而应该是你的断念利刃!强硬的立场,强悍地断念!做主宰内心的强者!

一位戒色两年多的戒友的体会:“戒色的核心是什么?这里简单说下,我本人现在信佛,当然不可否认行善修德的威力和帮助,可我更想强调的是,光靠行善提升正气是远远不够的,我刚开始也有这样的误解,觉得正气足了就没有邪念了,以为精满就不思淫了,但没有那么简单,真正要想戒得好,还是要学会断念,断念才是王道,断念是正行,其他的都是助行,正行助行都不能少,离了谁都不行,但最要紧的还是要学会断念。这是一切的根本。不学会断念,正气再足,也有马失前蹄的一天,断念才是根本的根本的根本,不戒到一定程度无法理解到这一步,不是说行善不重要,已经有太多前辈强调行善的重要性了,可我得强调比行善更重要的是断念,断念不难,但是得靠努力练习才行。我们必须有断念的实力,就像剑客要有剑,士兵要有枪一样,不能只靠福报,福报得有,但还是得有自己的实力,光有福报没有保护自己的功力,迟早遇到强大的欲望而破戒。”

这位戒友的体会很好,光靠行善提升正气是远远不够的,即使正气再足,邪念也会再次入侵,邪淫的回忆还会再次浮现,不注重断念,必将失败。观心断念是根本,是核心,其他是辅助,断念是正行,其他的是助行。很多人看了提升正气的文章,会很认同,提升正能量是对的,但一定要认识到断念才是实战的根本。最后那一下,是要看断念水平的!戒色界一直有一种思想误区,那就是认为光靠行善就能戒掉,其实不然,要进入更稳定的戒色层次,必须学会观心断念,学会对治自己的邪念。有了相当的实战体会和经验后,就会深知这一点。

另外一位戒友的反馈:“飞翔老师好,最近我真正的体会到了‘戒色就是控制念头’,直到最近我感觉自己才真正地戒色入门了,就是我终于意识到观心断念对于戒色的重要性。之前我也练过断念,但是一直没能契入,没有真正地做到过觉,只是单纯地练练口诀而已,并不懂得觉察。以前我依靠各种正能量的因素,比如放生、学习正能量文章、吃素、念佛经、抄写经书,也去寺庙做过义工,包括出去工作,充实自己的生活,确实对我来说,有一定的效果,那个时候正能量满满,就是这种状态,我戒到了八十多天,后来破戒了,后来我又戒了一百天左右,还是依靠正气正能量这些方面,各种因素。但是,越戒到后面我发现不对劲了,我发现邪念还是有,而且似乎一天天地增多,那个时候每当我有邪念就会变得极其紧张、不安,特别惶恐,有那种煎熬感,许多时候和人吃饭、说话的时候邪念都会出现,弄得我心神不宁,我发现只要有邪念,不会断,就会煎熬,而且不敢见人了,觉得很惭愧,内心不能很坦然。其实之前社恐已经消失了,但是只要邪念一出现,我不懂得断,那肯定不敢见人了,交流不顺畅,就这么快。那个时候我还是不认同修心,因为那个时候我深深地认同了正气戒色的理念,记得当时我就是想“正气足,邪气消,我现在邪念多,是因为我正气还不够强,正气再强一些就很容易断念,不会煎熬了,邪念也少”,那个时候我隐隐约约感觉到了不对劲,我感觉这不是真正一个戒色者应有的状态,后来破戒了,就是因为念头一次比一次猛啊!一次比一次强,我败了,就是念头。那一百天破戒先是接触了一个擦边图,然后点击看了,我仔细回想,当我遇到那个图的时候,有一种拉力一样让我去点击,其实就是一种微细的念头,点击之后呢,我感觉欲望被点燃起来了,然后是一个微细的搜擦边的念头,一瞬间就过去了,我真的是身不由己一般,然后去搜了那个东西,之后百般纠结,将近三四天不断厌倦又不断寻找,找黄看黄,看完不过瘾继续找,又寻找又删除又自责一会又开始继续找,我真的就像换了一个人一般,那种状态真的是恐怖,就跟疯了一般,最后破戒了。破戒之后连续破了一次。我现在写这些的时候我还回忆起,在我破戒那一抖擞,都是有一个念头指令一般,这个念头我回忆起来就是感觉到,‘时间到了,该破了’。那种无助,那种痛苦与失落感实在是太沉重了。正是这次破戒,让我真正放弃掉了正气戒色,但也不是完全放弃,其中有一些观点,比如‘充实生活、培养兴趣爱好、提升自己、树立目标理想、多孝顺行善、多做正能量的事情、看正能量的书、远离负能量等等’,这些理念还是可以的,是很好的戒色辅助。这次一百多天破戒让我重新捡起《戒为良药》,重新开始练习修心,我发现这次我练习很有效果,很快就契入了,真的神奇。我想就是我相信了修心,我之前不相信,因为我的信念不足,我那个时候很不认同修心,我没依靠修心,只是正能量地戒,戒了一百天,但是破了,心魔一次比一些猛。”

这位戒友这段反馈真的说得一针见血,他之前做了很多正能量的事情,虽然也能戒除一段时间,但最终还是破戒了,为什么会这样?因为邪念上来后,他无法断除,结果被附体,然后就身不由己了,进入非常疯狂的状态。很多正气戒色的理论光强调行善而忽视了修心,这是这类文章的缺陷,但这类文章毕竟是劝善的正能量的文章,也不能说完全不好,但应该认识到,光靠行善是不行的,迟早要破,有的文章说自己靠行善和充实生活戒了多久多久,其实是有水分的,明眼人一看就知道是假的,或者有夸大的成分,光靠行善和充实生活不可能戒那么久,之所以作者要这样写,就是为了推广自己的戒色方法,虽然他们的初发心是好的,但这类戒色方法明显存在认识的缺陷。一般看到这类文章,我也不会去说破,毕竟是正能量的文章,我还是表示支持的。这位戒友虽然做了很多善事,甚至去寺庙做了义工,但是当邪念上来时,他断不掉,就会有煎熬感,最后被邪念控制。大家可以回忆一下过去破戒的整个过程,就是当邪念、图像、怂恿、微妙感觉等上来时,断不掉,就会被控制,跟随念头一段时间,就会出现煎熬感。对于“正气足,邪气消”这六个字,要正确理解,正气足了,内心的负能量的确少了,但不要误解为正气足了,邪念就没有了,不是这样,即使正气很足的人,他也可能起邪念,因为邪念会自动冒出,会主动入侵。一位戒友说:“昨晚单位一人值班忽然念头一起就又看黄撸了,真的没办法了。”念头一来,断不掉,肯定被控制,心魔的特点就是“入侵并控制”,我之前的文章专门强调过的,我的戒色文章是基于我十几年的亲身体会和对成千上万例的实战案例的深入研究、总结和分析,那个破戒的过程我经历了太多太多,也看了太多太多,我深知,戒色实战的那一下就看断念,断念不行,就会被控制,陷入疯狂的身不由己的状态。圣贤教育强调“断恶修善”,有的人就偏重于修善而忽略断恶,有的人虽知道断恶,却不知道断意恶!意恶才是罪魁祸首,念头导致行为,必须学会断除邪念!我戒到现在八年多一次未破,真正一次未破,怎么做到的?因为我把握了核心,修心才是核心,充实生活、行善等为辅助,这点认识一定要正确。

\subsubsection{踢断棒球棍!}

最近看了几个视频,里面的人表演了踢断棒球棍,棒球棍还是挺结实的,但还是被他生生踢断了,那个腿功十分了得,有的人可以一次踢断好几根棒球棍,腿力非常惊人。看了这种视频,我就在想他们是怎么做到的,很快就得出答案了,长期的专业训练!开始他们踢沙袋,然后踢木桩,最后踢铁柱,经常练习,越来越强化。开始踢时腿感觉疼,时间长了渐渐就不疼了,腿也变得越来越硬。最近还看了一个铁砂掌的视频,六块红砖叠在地上,不用手扶,直接劈下去,全碎了,铁砂掌师傅的手掌超厚,都是练出来的,一天要练几千次,日积月累,最后功夫真的不可思议。我们断念也是要坚持练习,刚开始断念感觉很吃力,也不习惯观心,只要坚持练习,慢慢就能越来越熟练,熟练之后继续强化,就能越来越精深,达到的水平就会越来越高,练习使人强大!掌握方法后就要勤加练习,时刻保持观心和警觉,但不要过度紧张,过度紧张是走偏了,很容易累,适当警觉即可,就像过马路时保持适当的警觉但不会过度紧张,时刻监视和注意内心的活动。一位戒友在帖子里说:“需要强大的断力来为自己开路。坚持练习,才是通往成功的捷径。”他说得很好,久练自化,熟极自神,要坚持练习,持之以恒,渐渐就能做到了,当第一次做到断念时,甚至会很兴奋,其实并不是很难,关键是要弄懂原理,坚持练习。你的每滴汗水都不会白流,你的每次练习都在朝向变得更加强大迈进,一分耕耘,一分收获,注重积累,终究会迎来质的飞跃。

\paragraph{断念寸劲}

寸劲是指在近距离攻击对手,动作完成时瞬间突然加速收缩肌肉而发出短促、刚脆的爆发力量,即比喻发劲距离之短促。练即最短的距离内,爆发出最大威力。寸拳的威力以至简、至威、至猛为主,更重要的是要提高反应速度。

断念的那一刹那需要爆发力,要提高自己的反应速度,对念头的出现要极其敏感,如果暂时做不到,也不要灰心,坚持练下去,对念头的敏感度自然会提升,断速也会加快。断得刚,断得脆,很干净,一点也不拖泥带水,如果断得晚了,就容易出现煎熬感,也会残留一些微妙感觉。

库里平时是怎么训练的?听他自己解读吧:“当我看到绿色光束,我知道我要做的动作是一次胯下运球,所以我必须去做,先按绿灯,然后去完成动作。”听起来很新颖吧?“这就是一次神经训练,”库里总结说,他一直在利用 FitLight 训练(敏捷反应测试训练系统),这种训练在运动员、运动队之间越来越流行,这套训练系统是利用移动光源、多种颜色来对球员进行训练,提升球员的反应能力、决策能力以及集中注意力的能力。库里的身体素质在 NBA 中不算突出,只能算普通,肌肉没那么强,弹跳也一般,但他能独步 NBA,是有他的长处,他练的项目是反应速度、球感、实战意识和决策能力等,库里的出手速度极快,美专家统计了库里的出手时间,震惊的是库里场均出手时间不足 0.4秒,一般球员的平均出手时间在 0.6 秒左右,也就是说库里整整比他们快了 0.2 秒,再根据防守球员的距离和反应速度来推断,库里的投篮几乎不会给防守者封盖的动作时间,也就是说库里的出手实在太快了,只要姿势正确,防守者几乎干预不了。

断念就是练的那个觉、那个知,那个反应速度,只要坚持练习是有望做到 0.1 秒的,苏炳添 60 米起跑反应达到惊人的 0.134 秒,起跑要做动作,而觉察的反应可以更快,很多资深戒友很注重反应速度的练习,这是一个很重要的指标。网上有测反应速度的页面,手机上也有,大家可以试试,看看自己的反应速度如何。戒色菜鸟的表现是:念头上来,跟念,缺少觉察,跟了十几秒乃至几分钟后才发现自己在意淫,这时候已经晚了,开始欲火中烧,有煎熬感了,要失控了。为什么古德讲不怕念起,就怕觉迟,因为念头起势后,就很难断掉,也会出现煎熬感,那种煎熬感去操场猛跑几圈也不一定能消掉。一位戒友说:“当念头起来的时候,你意识到念头起来就熄灭它,这非常关键,这就好比欲火,欲火刚出现是一个火星,很容易熄灭,这时候不熄灭,再后来就很难控制了,一旦形成烧山之势,根本是无法控制的,所以必须在它还很弱小,还没构成威胁的时候熄灭它。”这段总结得很好,最近美国加州森林大火导致七十余人死亡,烧成森林大火就很难扑灭了,消防车来了也难以控制,而大火是从小火星小火苗开始的,如果刚开始就发现,马上灭掉,那是不费力的,很容易就灭掉了。断念贵早,要懂得把握断念的黄金时机,反应一定要快,最好零点几秒就断掉它。

\paragraph{断念听劲}

“听劲”是太极拳推手劲法术语,指推手时感知对手劲力变化的能力。《陈式太极拳》注:“所谓听劲,乃是由皮肤的触觉和内体感觉来探测对方劲的大小、长短和动向的意思。”至于感知灵敏度的高低,是由练拳和推手功夫的深浅所决定的。

听劲对于断念实战是很有启发的,断念的听劲是一种感知,是对内心的高度敏感,身上要放松,心里要安静,即“体松心静”,这样才能出敏感,才能感知到内心更细微的变化。说起来很简单,但做起来却是有很深的层次,这是实战更高层的重点,身上真的够松,心里又足够安静,那功夫将会很厉害,静到极处能出炸力,反应速度极快,念头一上来,瞬间就炸飞了。身体一松,内心一静,人的感知就会变得超级敏感,要看松的程度和静的程度来看功夫的深浅。这里的松,不是松懈,而是适当放松且警觉,这样感知力才会变强。在那种状态下,内心的一点点波动,你都能敏锐地感知到,并且能快速做出反应。实战就看反应,看刹那的实战表现,你跟念,就错了,你快断,就能立于不败之地!唯快不破!“彼不动,己不动,彼微动,己先动”,要达到“微动即知,一知即灭”的程度。断念就在刹那,强者断念,弱者跟念!强者主宰自己,弱者欲火中烧!

练一层功夫,悟一层道理,得一层奥妙。有的人看了不少戒色文章,也知道不少戒色理论,说起来头头是道,好像很懂的样子,但是邪念一上来,他就不行了。这是因为他并没有认真练习断念,只是懂了理论,理论没有结合实践,这样的懂也是浅懂、假懂,所以说,坚持练习是非常重要的。有些戒友经常抱怨,为什么道理都懂,危害都懂,怎么就一直破戒呢?到底为什么?因为他断念水平未到,看了戒色文章,虽然懂了一些道理,但没有真正去练习,当心魔来了,还和过去一样,那肯定还是失败。缺少练习,即使看一千篇戒色文章,也对实战的提升毫无意义,戒油子总是夸夸其谈,实战表现却很垃圾。心浮气躁、急于求成的人也会失败,要静下心来,保持耐心,持之以恒地练习,即使暂时退步也不灰心,还能继续坚持下去。

\textit{由着熟而渐悟懂劲,由懂劲而阶及神明。然非用功之久,不能豁然贯通焉。(《太极拳论》)} 懂劲后,愈练愈精,出敏感了,就能做到“一羽不能加,蝇虫不能落。”一位戒友说:“最近对断念上了一个台阶,对念头有了更高的敏感。”不断练习观心断念,可以增强自己的感知能力,对念头有更强、更快的反应,坚持练习下去就能看到自己的进步。一位太极高手说了自己的一段故事:“当年亲眼看到师爷把身强力壮的对手把玩于股掌之间。少年的我,急切地问这是什么神奇的功夫?师爷捋着胡子,揪着我的耳朵笑着悄悄地说了三个字:基、本、功!看见我不解的神情,又在我的耳边轻轻地说:小子,一辈子记住这句话‘基本功赢人’。从此,‘基本功’三个字始终牢记在心。时至今日,几十年过去了。我依然练基本功、悟基本功、用基本功。一句话:离开基本功,到老一场空!”戒色也有基本功,戒色的基本功就是观心断念,首先要学会观心,学会观察自己的念头,要坚持练习断念口诀,在练习口诀的过程中你的觉察力会得到很大的提升,最终就能决胜实战。不管何种戒色方法,最终都万流归海,最后都要看断念实战!观心断念既是基本功,也是最终极的功夫。

三个阶段:着熟、懂劲、阶及神明。

\begin{description}
    \item[着熟] 正确理解断念的理论,坚持练习断念口诀或者持咒念佛、思维对治等,不管哪种断念的方式,都要做到熟练,生疏缺少练习肯定是不行的。
    \item[懂劲] 随着练习,以及结合实战体会,加上不断反复研读断念的文章,这样就能对断念理解得越来越深,逐渐产生克敌制胜的能力和把握。
    \item[阶及神明] 对断念越来越熟练,实战时越来越强,对心魔的套路和诡计也了然于胸,实战的表现很强悍,很稳定,进入出神入化的层次。
\end{description}

高手总结的实战经验和深入的研究体会,是很值得学习的,高层的戒色方法是紧密围绕断念实战的,你若能按照要领坚持练习,就能练出功夫,登堂入室,愈练愈精,成为高手。当还没有达到那个台阶,往往是模糊懵懂的,很多地方不太明白,一旦达到了那个台阶,一下就豁然开朗,突然就明白了,哦!原来是这么回事。明白后,就会越练越强。我最开始也没有十足的把握,后来才慢慢具备的,断念是要不断精进、不断进步的,你以为自己很强了,其实还有很多细节有待完善,还能更精进,还能变得更强,就像跑进 10 秒了,还能继续进步,每提升一点,你就更有实战的把握。没有把握时,心里会很慌,没有安全感,虽然暂时没破戒,但就怕哪天心魔一来,马上就会破戒,再次掉入怪圈。有把握是不太容易的,只要坚持学习戒色文章和练习断念,慢慢就能获得这种把握。高手的内心很笃定,笃定是有把握、从容不迫的意思,当你具备较强的实战能力后,你的内心自然就会很笃定,根本不怕心魔进攻。

\paragraph{断念狠劲}

一位戒友说:“念头上来了,也没有那种一定杀死念头的狠劲,还会犹豫贪恋,导致现在的戒色状态很不稳定,今年破戒几次,都是断念不利。”断念要培养一股狠劲,这股狠劲是有实力的狠劲,就像踢断棒球棍,先要练出相当的实力,这样实战时才能狠得出来,否则缺少练习,就算实战时想狠一点,也会发现自己实力不济,确实不行,想断也断不掉,面对心魔疯狂的进攻,很快就被攻破。断念一定要快、严、烈、狠,功夫是在平时练出来的,实战时只是一次检验,看你到底行不行?能不能做到念起即断?运动员在平时刻苦训练,到比赛时就是检验平时的训练成果。有些戒友发帖说自己怎么那么没用,怎么又破戒了,这时应该扪心自问,自己平时是否坚持练习断念?具体练到何种程度了?是否能做到念起即断?如果平时疏于练习,不肯下功夫,那么实战时的表现肯定稀烂,肯定会被心魔攻破。

男人对自己要狠一点,立足实战狠抓训练落实,严格要求自己,实战时的狠,在于那股气势,那股瞬间清屏杀光一切邪念的气势!断念的那一下要极具爆发力,极具张力和狠劲,反应速度要极快!不犹豫、不贪恋,邪念来犯,斩立决!在内心的八角笼中彻底碾压心魔,记住:每一次训练都没有白做,每一次训练都在变得更加强大,也许暂时感觉不太明显,但只要坚持下去,你就会变得越来越强大,越来越能瞬间终结心魔!那种实战能力是可以通过训练得到不断强化的。面对心魔狂风暴雨般的攻击,你可以做到沉着应战,毫无惧色,一点都不慌乱,用最强势的表现 KO 心魔!平时认真训练、认真备战,有句话说得好:唯有辛勤地训练才是成功最坚固的磐石!不断积累,不断提升自己的训练水平,最终的胜利必将属于你!练为战!战为胜!做戒色的硬汉,狠一点!要有一种极其强烈的特种兵狠劲和血性,眼神要犀利,断念要干净利落,做戒色的特种兵,干死心魔!做戒色的大将,横刀立马,正气冲天,斩杀一切邪念、图像、怂恿、微妙感觉!上来多少杀多少!杀光杀净,一个不留!!!杀力要强大无比,杀气要让心魔颤抖!

眉宇间有一股惊天地泣鬼神的浩然正气,如钢似铁,坚不可摧。浑身上下散发出的正能量气息异常强大,这是一种碾压心魔成齑粉的强悍气场,那雄浑澎湃的雷霆能量像一片片波纹涟漪荡漾开来,白光万丈,璀璨如钻,闪耀着纯净的光芒。一双炯炯有神的眼眸就像烈日当空,摄人心魄,充满刚烈不屈的战斗意志和精神。一波邪念、图像就像打了鸡血一般冲上来,只要他凛然一觉,瞬间就灰飞烟灭,犀利的目光扫视着内心的战场,一股强大无俦的威慑力镇压全场,仿佛空气都凝固了。整个人逸散出一股无法形容的凛冽杀气,经过第一轮的战斗,没有邪念和图像再敢上来,因为它们知道上来就是送死!强悍到这种地步,心魔不敢轻举妄动,这是真正的断念实战强者,携带着强大无敌的气势。他的气场在慈悲柔和与杀气冲天之间切换,当邪念袭脑时,他就瞬间变身为大杀器,有一股杀光一切邪念的狠劲!强大的战力如同暴风般激昂,惊雷般迅猛,往那一站,仿佛一头雄狮立在那里,浑身充满着爆炸性的力量,一股异常强悍的威压散发出来,让人看了有一种惊心动魄的气象,这一幕,落在所有人眼中,只剩下无尽的震撼!这气势实在惊人!一位大成境高手真正要出手之时,那般力量,绝对是万分恐怖!如泰山压顶一般,摧枯拉朽,势不可挡!一股超强烈的狠劲要把心魔撕成碎片!这是毫无悬念的碾压!在断念大成境的修炼者身上,你能感受到那股气势,那股狠劲,那股威严,那股震慑力,这代表着他的级别,他的境界,他的修为,他的实力!

\subsubsection{立刀旁!}

\begin{quotation}\it
    立刀旁的字就像一个人提着一把刀。

    立刀旁的字似乎有一股杀气,好像随时准备砍杀。

    立刀旁的字是刀客,是剑客,是真正的豪侠。

    立刀旁,刀已立,就等心魔前来喂刀!

    立刀旁,一夫当关,万念莫开!

    立刀旁,邪念入侵,手起刀落!

    快如闪电,心魔倒地!

    那个刀客的名字就叫“立刀旁”!

    专杀邪念,专屠心魔!

    他是真正的杀念者——立刀旁!

    随时准备出刀,随时准备杀念!
\end{quotation}

\textit{就是一种觉知力,就像一个勇者随时拿着宝剑一样,随时保持警觉,贪心等烦恼一生起时,就能立刻用正知、正念的宝剑,将它斩断。(大宝法王)} 断念需要警觉,要时刻保持觉知,随时准备战斗,念头一起,马上斩断。警惕和警觉也是我一直强调的,稍不警觉,就可能会跟着念头跑了,那是瞬间的事情,稍不注意,就会被念头带跑,所以要锻炼出强大的觉知力,随时保持警惕,随时准备战斗!实战意识极强!

一位戒友说:“今天下午我终于明白了念头有多狠!下午一个念头,跟了一小会,马上觉察到了赶快出来,但我仍然能感觉到它的拉力很强,好像一个人使劲把我往坑里拉一样,太狠了!”跟了一小会,已经晚了,实战时不能跟念,跟念就是在强化念头,念头的拉力是非常强的,一跟念头就是一连串的念头,心魔太擅长编故事了,一跟图像就是一部小电影在脑海中播放。实战时反应要快,断念要狠!另外一位戒友说:“有时图片和怂恿会一起进攻,心魔还会怂恿你回忆以前那种让你欲罢不能的场景。”怂恿是心魔常见的套路,就像一个人在脑海中劝你一样,对于这种套路要学会识破,坚决不听信。怂恿你回忆,这是非常阴险的套路,一回忆那种诱惑的场景,就很容易陷进去,心魔太狡猾了,一定要严防这种套路。

断念这门实战技法需要用心学、用心练,用心悟,通过不断学习总结断念的文章以及口诀的练习,慢慢就能渐入佳境,不是说你学了几篇文章或者练了几天马上就会变得很强,而是需要恒久力行,持之以恒地练习,需要的是认真的态度,坚持下去的决心,天天学、天天练、天天悟,做断念的练家子,到时你一定能够做到觉察即消灭,觉察即降伏,一旦你掌握了断念这个功夫,你才真正摸到了戒色实战的边。对断念要极其重视,平时的训练就是为了实战的那一下,脱离实战必将失败。学会修心,学会断念,才真正戒色入门,这是比较严格的入门标准,学会从起心动念上修,才是真正的究竟,有的戒色方法不注重修心,而是提倡充实生活、转移注意力,这种方法是不究竟的,迟早还是会破戒。不管你用什么戒色方法,最后念头上来时,断不掉,还是会破戒。有的戒色方法的确可以让人在几个月内感觉欲念很淡,好像戒色很轻松,然而继续戒下去,就会迎来翻种子,到时邪念会疯狂地进攻,在欲念很淡时往往会有些大意,不太注重修心,等到邪念猛烈来袭时,就顶不住了,到时就会疯狂破戒。

一位戒色三年的资深戒友,他在帖子里说:“修心断念,这是所有的戒色体系中最最根本的一条,最根本的一条,最重要的一条,重要的事情说三遍!为什么我这么强调这一点,因为俗话说得好,养兵千日,用兵一时。不管你平常看了多少戒色文章,做了多少笔记,最后的成败都看在邪念涌上头脑的那一刹那,你能不能察觉,能不能消灭。就是那一刹那,零点一秒钟,那一瞬间消灭了邪念,那你就是胜利者,如果心里有一丝贪念,那就糟糕了,星星之火可以燎原,等你发觉的时候,那时候邪念已经成为哥斯拉级别的了,想要断除,堪比上青天。所以断念贵在速度!”另外一位资深戒友说:“大概是在两百多天之后了,我忽然发现我不需要在对治邪念的时候搬出那句思维对治的话了,我一觉察到邪念,邪念马上就消失了,这个时候,我才渐渐明白飞翔老师说的‘不需要你思考,自动就断掉’的内涵。其实背诵口诀五百遍的价值就在于尽量不要散心,学会那一刹那‘念起即觉,觉之即无’的觉察力,念头一起,凛然一觉,马上就把邪念斩杀于无形,这才是这句口诀的真意。”刚开始练习口诀,很容易被念头带跑,拉回来继续练习,渐渐被带跑的次数在减少,觉察力在逐步增强,刚开始会觉得练习的成效不大,因为自己尚未熟练和精通,就像学打字、学弹钢琴,刚开始会显得很笨拙,等到练习一段时间后,就会娴熟起来,继续加强练习,就会越来越精通。还有一位戒友在帖子里说自己屡戒屡败四年:“现在已经戒色五个月,突破口就是断念!断念是最核心最终极的方法!”他一语中的,有的人说自己总是失败,很懊恼,都快失去信心了,我那时也深深体验过那种戒不掉的无奈和绝望,也曾经放弃过戒色,后来我学会修心后,就戒到现在八年多一次未破,为什么我能做到?因为我把握了根本和核心,那就是观心断念。那些总是失败的人,看看他们的实战表现有多差,就知道他们为什么戒不掉了!念头、图像、微妙感觉一上来,他们不是断不掉就是舍不得断,跟着念头跑,最后欲罢不能而破戒。这个破戒的心理过程已经重复上千次了,一次又一次,心魔每次都得逞,要改变这种被动挨打的局面,必须强化断念水平。弱小注定被虐,只有强大起来才能战胜心魔,主宰内心。

除了断念口诀、持咒念佛、思维对治,断念的方法还有断字诀、呸字诀、咄字诀,这三个都属于“一字诀”,就是靠一个字来断念,比如念头起来了,大喝一声断,从而斩断念头;呸字诀来自于藏传佛教,是很有名的一字诀,呸字本意有斥责、唾弃、否定之意;咄字诀来自于禅宗,咄字表示呵斥,念头来了,要连续下去,这时大喝一声“咄”,就能把念流斩断。这三个一字诀需要坚持练习才能做到熟练,一般在独处时可以用一用,有人在时就不大合适了,因为你突然大喝一声,别人还以为你精神不正常,自己也会感到尴尬。任何方法都是以观心为基础的,观力强大,直接觉察消灭。思维对治也很重要,思维邪淫危害可以充分警醒自己,思维不净观则可以对治贪恋,不净观 + 断念口诀是黄金搭档,不净观 + 持咒念佛也是。对治贪恋是很重要的,如果看不破那层皮,还在贪恋女色,那断念也不会坚决果断,因为舍不得断。关于持咒念佛我想补充说一下,我每天也持咒念佛的,也坚持了大概七年多了,刚开始持咒念佛心里妄念会很多,你会惊讶自己怎么那么多妄念,就像一束光照进来会发现很多灰尘,坚持念下去,会发现自己的内心渐渐清净了,邪淫的念头的确变淡了,变少了,有的人这时就会觉得戒色念念佛就可以了,不用断念了,其实念佛也是要断念的,戒到一定程度,邪念会自动冒出来,图像也会自动浮现,那个上头的速度非常快,念头一起,马上要念佛来抵抗!\textit{当杂念初起时,如一人与万人敌,不可稍有宽纵之心,否则彼作我主,我受彼害矣。若拼命抵抗,彼当随我所转,即所谓转烦恼为菩提也。汝能常以如来万德洪名极力抵抗,久而久之,心自清净。心清净已,仍旧念不放松,则业障消而智慧开矣。(印光大师)}戒到一定时间会经历各种翻种子,巨大的考验还在后头,有的修行人修行十几年还破戒,为什么?因为之后会经历剧烈的翻种子,降伏不了,就会破戒!很多人都有这样的体会,就是一边念佛一边还在起邪淫的念头,我也有过这种经历,这时应该提高嗓门,专注在佛号上,不要跟随邪淫的念头、图像等。念佛也是要实战的!这点我深有体会,不是说念佛了心里就再也不起邪念了,邪念肯定还会来。有的人很长时间都没一个邪念或者邪念非常少,这时他就放松警惕了,以为成功了,以为没事了,殊不知心魔一直在虎视眈眈,就等他放松警惕,到时就会突然进攻。不管戒多久,都不能放松警惕,一定要小心谨慎,做好观心断念。

\subsubsection{对境时必须做到百毒不侵!}

不少人对于戒色的道理和邪淫的危害都有所认识,但是诱惑的境界一来,就迷惑了,这就是“说时似悟,对境生迷”。这个时代戒色的难度颇大,因为这个时代太容易接触到色情的内容了,小孩子十岁左右就开始偷偷看黄了,网络时代也不用买黄,直接上网就能看,生活中女性的穿着也是比较暴露,不像古代都捂得严严实实,古人的衣服很少露的。时代不同了,戒色的难度自然就增大了,如果没有定力,随便看到一个诱惑的对境,邪念就会丛生。阿拉伯女性的穿着一般是身着大长袍,外加一条披风,而且头上还要包着头巾带着头箍,女人都要蒙面,这是阿拉伯人的传统形象。按照沙特的法律规定,沙特的女性出席公共场合必须要身穿宽松的黑色长袍,而且要佩戴纱巾,包裹住头发和脸,只能把眼睛露出来。以前看到阿拉伯的女性身穿黑长袍把全身上下都裹得严严实实,不太能理解,后来戒色后就理解了,她们那样穿可以避免异性起邪念,可以杜绝色情的诱惑,这是她们国家和宗教信仰的规定。到了那些国家,异性都穿黑长袍,你对境时会起邪念吗?肯定不会,除非你自己胡思乱想,她们的穿着已经把起邪念的机会降到了最低。

而在我们国家,穿着方面已经比过去开放很多了,夏天的诱惑很猛烈,每到夏季很多戒友都会把持不住自己,对境时邪念就冒出来了,能够做到对境无心,是需要相当高的定力的。现在这个网络时代,上网时很容易遭遇各种黄雷,各种擦边图和擦边新闻,真可谓防不胜防,这需要更高的定力才能把持得住,否则一张擦边图就会让人沦陷,这种破戒类型非常多,就是看了擦边图,不停地点击,最后开始搜黄看黄,一步步沦陷。我个人的深切体会,来自于成千上万次的对境实战经历,那就是:对境时必须做到百毒不侵!这句话要反复告诫自己,看到任何诱惑图片都不能动心,都不要去看第二眼,实战时常见的错误就是:没看清,再看一下,结果第二眼着魔,一聚焦,一盯着看就容易陷入。对境实战的教训是非常深刻的,这点我深有体会,即使那些戒色好几年的资深戒友也可能栽在对境实战上,戒色十规专门强调了视线管理,这点太重要了,重要程度不亚于断念口诀。对境时是最大的考验,就看你动不动心!一个人的戒色修为到了何种程度,对境时就能验得出来,就看他的实战表现,看他的视线是否粘上去,看他的内心是否起邪念。在色情泛滥、黄毒横行的年代,我们必须做到百毒不侵,这点至关重要,有无数的机会让你堕落,如果自己对境时没定力,那就会深陷色情的漩涡,我的实战经验就是对境时,不管看到什么图片,都不要分别好坏美丑,一分别,就容易陷进去,觉得图片好,就会盯着看,那个对境过程真的很微妙,只要你有一点喜欢,那就会反复看,欲罢不能,停不下来。

戒色十规的戒色要点每天都要看,每天都要提醒自己,告诫自己,一条条去严格落实。戒色十规是紧密围绕实战的一个高度专业化的戒色体系,而不是以充实生活、转移注意力为主的体系,我更注重的是实战,实战强,才是真的强!实战主要是两方面,一是断念实战,二就是对境实战。要做到“内不随念转,外不为境迁”,更加高度的概括就是四个字:外避内断!要学会避开诱惑,不要去看,避色如避箭,高手都懂得躲子弹,这样才能幸存下来。内断,就是起了邪念要立刻断除,外避内断是之前一位戒友总结的,他的总结非常到位,他也戒得非常好。

\subsubsection{原始的纯真}

每个人小时候都是那么纯真,那么无邪,是一种纯净纯善的状态,正是在这种状态下我们感到很快乐,不懂大人的世界,我小时候看很多大人,都觉得他们好严肃,不苟言笑,长大后我也变得严肃起来,因为大人的脑子开始有意淫了,很难感受到纯净的快乐,经常泄精会导致内心沉重、抑郁、焦虑,快乐不起来。我沉迷手淫后,有时会有一种强烈的感觉,那就是我离开了那个纯净的自己,真的越来越远了,我都快忘记那种纯净美好的感受了,我在邪淫的歧途上渐行渐远,在邪淫的粪坑里越陷越深。后来我戒掉了,才真正明白纯真是一个人最宝贵的品质,纯真年代是人生最幸福的时光,失去纯真便是堕落的开始,失去纯真也就失去了纯净的快乐,开始在欲望里找快乐,最后肯定会感受到莫大的痛苦,欲望是一个无底洞,不断把自己掏空,在邪淫放纵中迷失和幻灭,亲手把自己毁了。

当少年面对极度诱惑的满屏肉弹,那幼小的心灵受到的冲击可想而知,开始了隐秘而堕落的生涯,脑子被龌龊的邪念占据,从纯净次元跌落,这是非常可悲的事情。回忆儿时的岁月,感觉那段时光就像活在了纯净的奇迹里,和邪淫没有任何关系,内心是格外纯净而单纯的状态,在那种状态下大家的脸上都能经常看到喜悦开心的笑容,非常纯真的笑容,发自内心的快乐。而现在大家长大了,再也看不到儿时那种笑脸,很多人都陷在了邪淫的深坑里,以邪淫为乐,以放纵为荣,脸上的气色也渐渐不好了,皮肤也油腻了,眼睛也黯淡了,失去了纯净的光辉。我清楚记得我那个表弟,小时候多么开心快乐,那种纯真无邪的眼神和表情,真的太感人了,太纯粹了,也太可爱了,可惜进入发育期以后,他变了,他的变化和我进入发育期一样,脸上变得灰暗,眼睛不再清澈明亮,纯真的感觉彻底消失了,再也看不到那种纯真无邪阳光灿烂的笑脸了,我到现在还记得儿时他那种欢快而小小的身影,以一种特别纯真的语气来叫我“阿哥”,他儿时的一次次纯真流露,真的让我颇为感动,和他在一起我也感觉很开心很快乐。后来他从奇迹的世界里跌落了,一如当年我的跌落,我觉得这是非常可悲的事情。邪淫后消失最快的就是纯真,当一个人失去纯真,就像失去最宝贵的财富一样,特别悲哀,特别可悲。成长的痛苦即是纯真的丧失,失去了纯真,就很难真正开心快乐起来,儿时的神奇感觉也会消失不见。一个人最可贵的就是心灵深处唯美而毫无伪装的纯真,这是最打动人的品质,比钻石都珍贵!那种纯净美好的质感真的太宝贵了,纯白的灵魂,轻盈而愉悦,天真而纯粹!因为纯净,所以美好。

戒色可以再次点亮眼睛,点亮那一双双迷失在色情与邪淫中的灰暗的眼睛,整个人也会变得光明起来,纯真的笑容会再次浮现在脸上,可以找回那份失去很久的纯真,有了这种纯真,才拥有真正的幸福和快乐。纯净的大快乐其实是本自具足的,是盲目追逐欲望让我们变得不快乐,变得沉重,变得抑郁,变得空虚和绝望。戒色后那种特别纯粹、特别开心、特别轻盈的感觉会再次回来,走路都带风,真的就像重生了一般,看到花花草草都有一种特别美好特别喜悦的感觉。有的撸者看儿时的照片会默默流泪,因为儿时是那么纯真,那么快乐,而撸后是那么龌龊,那么恶心,那么惶恐,那么不开心,症状爆发还要忍受症状带来的痛苦。

小学毕业照里,每个孩子都那么纯真,春游的照片里,每个孩子都那么开心喜悦,那些阳光灿烂的笑脸实在太感人了,孩子眼睛清澈明亮有神而富有灵气,皆有一种纯净无暇的美!非常震撼人心。世界上最宝贵的东西就是纯真的童心,与纯真的孩子对视,感受那份来自纯真的力量,看着纯真的孩子玩耍,内心也会感到幸福快乐。走出邪淫的阴霾,走出那段灰暗堕落的日子,我此刻站在镜子前,露出了孩子般纯真灿烂的笑容,我内心是快乐的,我的眼睛是喜悦的,这种感觉太美好,愿每一个人都能找回最初的美好纯真。愿世上的每个人都能回归童心,拥有纯真,做一个真正幸福的人,当自己处于纯真的境界,内心是轻灵的,清澈的,一尘不染的,摈弃了所有猥琐的、龌龊的、低级的念头,只剩下纯粹美好喜悦的感受。纯真不仅可以使自己感到愉悦,而且还可以为他人带来快乐,虽然见过了各种人和经历了各种事,依然还能保持着最初的纯真,这是最难得的。每个人灵魂中都有那最简单最珍贵的纯真,在邪淫后失去,在戒色后复得,即使世俗污浊,我们也能保持洁白无瑕的纯真,那将是最可贵的品质。

\subsubsection{崇高的使命}

《太上感应篇》里有“正己化人”这四个字,当初看到这四个字就很有感觉,这是我真正想做的事情,先正己,再化人。\textit{惟正己可以化人,惟尽己可以服人。(曾国藩)} 我那时最早是在神经症的病友群宣传戒色,告诉他们手淫的危害,劝他们戒除手淫恶习,学会养生之道,这样身体才会逐步痊愈。我那时已经认识到手淫的危害了,自己也戒了一段时间,当时就有股冲动想把手淫的危害告诉他们。神经症患者是很苦的,身心的双重折磨,还不被家人理解,度日如年,每天都在煎熬,每天都活得很绝望,这方面我的感触很深,因为我曾经就是神经症患者。这个病一定要戒除手淫恶习,也要学会养生之道,不要熬夜久坐,三分治疗,七分养生,自己也要学会情绪管理,不要生气发怒。这个病要康复,应该彻底禁欲一段时间,结婚的人应该和妻子好好沟通下,先把身体养好再说。后来我来到了戒色吧,在这个平台可以帮助更多的人,我就常驻戒色吧了,刚开始就是帮助新人答疑,然后机缘成熟就开始写戒色文章了,我觉得自己肩负使命来拯救深陷色情陷阱的孩子,我曾经就是一个沉迷手淫的孩子,那时我很想戒掉,但没人指导我,也看不到戒色文章,一直失败,一直深陷于那个怪圈中不能自拔,浑浑噩噩十几年,那是不堪回首的十几年。现在我戒掉了,我想帮助更多的人,让他们也回归纯净的自己,这是崇高的使命,我会倾尽全力来帮助每一个需要帮助的戒友,即使牺牲我一个,也在所不惜。

某位戒友写道:“我从一个小学班里的班草、大家爱慕的男孩变成了一个初中班里成绩低下、相貌猥琐的男孩,我有时候问自己为什么变成这样了?!变得恶心、丑陋,殊不知就是 SY 让我变成这样的!可我却还在继续 SY!我真恨死我自己了。”这个孩子的经历很可悲,从班草到猥琐、恶心、丑陋,手淫这个恶习毁人无数,手淫会渲染出一个异常丑陋和龌龊的自己,充满戾气充满负能量。一位戒友感叹地说:“谁也不愿意做一个猥琐龌龊看着一脸死气沉沉连自己看着都恶心的人啊!”另外一位 17 岁的戒友发帖子说:“我看到我 7 岁时妈妈给我拍的相片,和现在的我比,那时候真可谓是灵气十足,充满了聪颖和纯净啊,看着自己小时候的照片哪敢想我多年后脑子里会有这些肮脏龌龊的信息啊,跟现在对比起来那时候萌得不得了,再看看现在满脸猥琐下流之气,简称猥琐男,我现在才 17 岁,我看到我小时候的照片我欣赏了足足半个小时,我最后真的是一阵阵苦笑和寒酸啊!自己因为邪淫收到过多少的不安,多少的挫折,多少的欺辱。我对不起自己啊!”在青春期就要懂得戒色,就要知道手淫的危害,很多孩子在手淫后都变得灰暗了,就像掉线的 QQ,头像一下就灰了,整天郁郁寡欢,开心不起来,就像变了一个人。纯净孩子的眼中尽是温柔、善良与纯真,邪淫者的眼中尽是空洞、颓废与无神。色情与手淫正在荼毒中国的年轻一代,我们有责任有义务站出来告诉他们真相,不能再被无害论蒙蔽了。戒色也不仅仅是戒除手淫,更重要的是在戒色的过程中养成自律自强的品质,让自己拥有一个好的身体,好的生活,懂得感恩,懂得孝顺父母,懂得行善积德,学会培养自己的正能量,积极奋斗自己的人生。

\subsubsection{与汝安心竟}

\begin{quotation}\it
    慧可见达摩,乞与安心法。

    慧可曰:诸佛法印,可得闻乎?

    达摩曰:诸佛法印,匪从人得。

    慧可曰:我心未宁,乞师与安。

    达摩曰:将心来与汝安!

    慧可良久曰:觅心了不可得。

    达摩曰:我与汝安心竟。

    慧可大悟。
\end{quotation}

这个公案是禅宗最有名的一个公案,我很早就看过这个公案,刚开始的几年都没能真正理解和领会,这个公案有点让人摸不着头脑,不知道达摩祖师到底在指向什么,完全不懂。

这个公案我是过了好几年后才真正明白的,慧可禅师说:“乞师与安。”让师傅帮他安心,达摩祖师让他把心拿出来,慧可良久,这个良久的过程就是觅心的过程。心一般有两个意思,一个是念头,一个就是真心、真我。达摩祖师让慧可找,慧可当然要向内看,去看自己的念头,去找自己的念头,当他去看时,念头就消失不见了。这是最奇怪的现象,当你去看念头时,念头凭空消失了,消失得无影无踪,像幽灵一样消失不见了!慧可说:“觅心了不可得。”找不到念头,达摩祖师说:“我与汝安心竟。”这句话一下让慧可大悟,当念头消失了,这时剩下的是什么?剩下的就是真心,“与汝安心竟”,这个心指的就是真心、真我,也就是纯粹的觉知。就是元音老人讲的“一念不生、了了分明”,黄念祖老居士讲的“离念的灵知”,丁愚仁老师讲的“念与念之间”,就在那个电光火石的刹那,慧可大悟,终于明白了。之前一直认念头和身体为自己,形成了非常顽固的错误认同,现在终于悟明白了,那个顿悟的刹那实在太宝贵了,很多人一辈子都悟不明白。

还有另外一个公案也很有名。

\begin{quotation}\it
    达摩祖师将要圆寂时,把四个得意的子找来,说:“我要离开的时间到了,你们说说各自的悟境吧!”

    第一位是道副说:如我所见,不执文字,不离文字,而为道用。曰:汝得吾皮。

    第二总持尼师:我今所解,如庆喜见阿閦佛国,一见更不再见。曰:汝得吾肉。

    第三位道育说:四大本空,五蕴非有,而我见处,无一法可得。曰:汝得吾骨。

    最后是慧可,却一句话也不说,只是向达摩顶礼,依位而立,默然不语。

    达摩却应许说:汝得吾髓。
\end{quotation}

前面三位都起了念头,只有慧可禅师没起念头,顶礼后站着不说话,达摩祖师知道慧可已经领悟真我了,于是说:“汝得吾髓。”

这两个公案真的非常好,我个人很喜欢这两个公案,有较高悟性的戒友应该已经能领会了,真我是最简单、最单纯、最质朴的一个状态,一开始会觉得平淡无奇,甚至有点无聊,一点也不吸引人,丝毫没有戏剧性,它是那么平淡和普通,然而不断安住它,就能体会到妙不可言的感受,那种安住状态会变得越来越坚固,你对内心的主宰会变得越来越强。

\subsubsection{老北京鸽哨}

“豆汁油条钟鼓楼,蓝天白云鸽子哨”,每个地方都有属于自己的声音,鸽哨,就是老北京原汁原味的声音,当四合院的上空一群鸽子飞过,便会响起一阵嗡嗡的鸽哨声,这是老北京电影中常出现的镜头,我虽没去过北京,但在 80 年代、90 年代有关北京的电视剧和电影中,有多次听到过这个掠过天空的声音,当时就给我留下了深刻的印象,在我印象中,鸽哨就是回忆童年的声音,悠远而动听,会引发某种纯真的情感与质朴的情怀,老北京的鸽哨回响在我的记忆深处。有人说,鸽哨是天空的音乐,一群鸽子振翅飞过天空,音乐就悠然而起,随着远近、角度的不同而变化万千,看鸽子在天空中飞翔,也能感受到那种自由,这是让我很神往的一种状态。后来渐渐发现,鸽哨是可以让人契入真我的一种声音,听到这个声音,自然就处于无念的状态,这声音会随着鸽群的飞翔回旋而变化,清脆悦耳,感觉很奇特,有某种特定的回忆感,很空灵的声音,让人魂牵梦绕。据说鸽哨自北宋时就有记载,至今已有近千年的历史,这是一种很古老很怀旧的声音,印象中的老北京,就是飞过城墙、飞过四合院、飞过天空的鸽哨声。小时候看《小龙人》就有鸽哨的声音,一晃我都这么大了,奔四的年纪了。北京这座城市前面加一个老字,才有那个味道,老北京的天空下,有根本上师黄念祖老居士在度众生,这也是我喜欢老北京的一个原因。听到鸽哨,闭上双眼,让漂泊在轮回中的疲惫的灵魂小憩一会,享受临在的片刻,享受真我的安逸。

\subsubsection{活出真我}

每个人来到这个世界上都是来寻找真我的,也许自己不知道,但那个本能就是朝向真我,有意识或者无意识地寻找真我,寻找那一份无念的体验。每年春季,生活在太平洋海域中的五亿多条大马哈鱼将开始 3000 英里的旅程,回到它们出生的地方产卵繁殖。这是地球上最壮观的洄游之旅,它们要历尽千辛万苦,经过数月才能回到家乡。这就是回归本源的一种本能,我们真正的本源就是真我——那个无形的源头。我现在知道了,为何之前我那么疯狂追求性,追求快感,追求射出的那一刹那,因为那一刹那我脑子空了,瞥见了真我,只是自己不知道罢了。我真正追逐的其实不是快感,而是真我!这是一种本源的回归,只不过通过性来回归是有很大弊端的,因为这个过程中你会起很多邪念,而且能量也会被大幅耗损,很容易导致症状缠身。真正明智的做法就是通过觉察进入真我,觉察消灭念头,念头没了,剩下的就是真我,那种体验看似平淡平常,无任何奇特之处,但只要不断安住就会发现它强烈的质感。\textit{了悟真我是最终的目标。[拉玛那·马哈希(Ramana Maharshi)]}那个最终极的问题就是“我是谁?”如果你认为自己是念头,是身体,那是错误的认同,念头是工具,身体是载体,本质是真我!真我是纯粹的觉知。“我”字打头的念头会让人产生一种虚假的自我感,你会把念头认作自己,这就像一个魔咒一样。当你彻底认清这一点时,那绝对是一个了不起的顿悟。念头会限制真我、遮蔽真我,你必须亲自打破这个限制,念头的正确身份是仆人,而不是主人,主人是觉知。你一次次被念头、图像带跑,无法做主人,只有发展出强大的觉知力、观照力,你才有望控制自己的念头,主宰自己的内心。发现真我、认识真我、活出真我是生命最高的使命,每一个轮回个体的最终使命,我希望我的戒色文章能把大家带到真我的层面,活出真正的自己!那个简单、纯粹、充满力量的真我!The true self!

\begin{quote}\it
    寂天菩萨说:‘在念头与念头之间,那里就有佛。’然而我们不相信就这么简单。我们到处去寻找佛,我们要找那发光的、伟大的、灿烂的佛。我们不知道,因为佛离我们太近了,所以我们根本看不到他,就像我们看不到自己的眼睫毛一样。(宗萨仁波切)
\end{quote}

\begin{quotation}\it
唯一的父亲,

以庞大之爱,已让我看见己之财富;

曾为乞丐之我,

持续在内心深处感受到师之存在。(顶果钦哲仁波切)
\end{quotation}

(这两段法语是我由衷喜欢的,分享给大家,这里的财富指的就是真我,从最究竟的层面来讲,真我也就是最根本的上师,它就在每个人的内在。)

\subsubsection{后记}

戒色八年多,凛然刚正之气再次回到了我的眼睛和眉宇之间,之前因为邪淫导致的戾气和猥琐气已经彻底消失了。我真正蜕变了,我不再看黄,不再手淫,我已经彻底告别了邪淫堕落的生活,在面对诱惑的对境时,我也具备了相当的定力,不会陷入了,戒色修心给了我莫大的力量,让我可以在色情泛滥的时代保持纯净的心灵,从容面对自己的人生。我比较喜欢“有德自安”这四个字,有德行的人自然有一种安定祥和的感觉,内心很稳定,眼睛很有威严和定力。具备戒德之人,心灵会散发出美好的芬芳,令人崇仰,令人愉悦,给人带来美好的感受。戒色开启内在力量之旅,你开始学会修心,学会主宰内心,你不再是过去那个被心魔随便虐的菜鸟了。

坚持到现在,我不仅遭受了外界的诽谤,也遭受了戒色界内部的诋毁,这些诽谤和诋毁就像身上的灰尘,轻轻拍去,继续坚定地前行。这些诽谤和诋毁也是逆增上缘,使得我的立场、决心和信心更加坚定,更加不可动摇,从这个角度来讲,我还得感谢这些诽谤和诋毁。我也衷心感谢大家的支持,没有你们的支持,我也不会坚持到今天,看到你们蜕变是我最高兴的事情。下面是四位戒友的留言:

\begin{quotation}\it
    飞翔老师,其实最开心的事情就是看到你还在更新《戒为良药》了!

    成长的路上,和飞翔哥一样,我也是才发育就接触了邪淫,最终导致了人生的第一场磨难。感恩戒色吧,感恩飞翔哥,把我从万丈深渊中解救了出来。也曾许多次流过眼泪,但还是得坚强面对。看着飞翔哥的文章,才接触到传统文化和修身养性。真正一身正气,人生路才能越走越好。飞翔哥的文章,我基本上都打印下来,反复阅读了很多遍。《戒为良药》不仅是戒色,很多方面都能给人带来启发。再次感恩飞翔哥,感谢一路上有您的陪伴与坚守,真美、真好!

    飞翔老师,虽然不知道您能不能看到我的评论,但是我想在这里表达对您深深的感谢,感谢您为戒色吧付出了那么多,为那些素未谋面的人指路,即使遭到恶人的诋毁,也坚持为我们这些深陷苦海的人点燃一盏明灯,感谢您。

    原来,能读飞翔大哥的文章,是人生多么重大的一种享受啊!永远支持飞翔大哥,要紧紧跟随大哥的脚步,将戒色进行到底!
\end{quotation}

这八年多,我一直在尽心尽力尽责地坚持,我不是为了自己,而是为了大家,为了每一位需要帮助的戒友,你们就像我的兄弟,就像我的家人,我愿意一直真诚地帮助大家。在我的动机里,不掺杂一丝一毫的名和利,我一直坚持的都是纯公益。戒色吧有一种纯粹无私的崇高精神在支撑,希望大家一起来传递这种精神,传递这种正能量。对于大家,我只有深深的感恩,真的是无比感恩,除了感恩还是感恩,感恩大家给我这么一个机会,我感到无比荣幸,我愿每一位戒友都超过我,我垫底也没事。我也深深感恩圣贤教育,感恩大德的开示,感恩根本上师澎湃而深厚的恩慈,感恩佛菩萨的加持,是圣贤教育点化了我,让我从黑暗撸坑回到了光明美好的世界,让我的振动频率得到了极大的提升。

从最高戒色天数 28 天到八年多一次未破,我做到了,只要掌握正确的戒色方法和原理,相信大家都可以做到。曾经的我真的是十足的菜鸟,屡戒屡破,被心魔虐得体无完肤,每次被心魔附体后,那个猥琐的身影又开始疯狂起来,疯狂找黄,疯狂看黄,疯狂手淫,进入“三疯状态”!后来我开窍了,懂得学习了,学会修心了,才有了后来的蜕变。在人生的至暗时刻我绝杀了心魔,就像篮球场上的压哨绝杀,是绝对振奋、绝对励志、绝对荡气回肠的经历,之前我的人生已经因为邪淫进入了垃圾时间,特别绝望,没有任何希望,而正在这时,我拼了!拿出了最大勇气和决心,最终战胜了心魔,冲出了怪圈。看到很多破戒帖,那种无奈、无望和颓废,我经历了无数次,我感同身受,我要说的是,只要你坚持学习和练习,你迟早会迎来转机的,你迟早会顿悟的,你迟早会战胜心魔的!要对自己有信心,只要坚持下去,注重积累,肯定能做到的。一定要严格落实戒色十规,戒色十规是我戒色思想的最高总结和提炼,背后有上百万实战案例的支撑,戒色十规是以实战为核心的高度专业化的成熟戒色体系,大家一定要重视起来,认真研读、贯彻和落实。戒色十规可以让你最快地掌握戒色的精髓和重点,第 126 季要反复学习,真正吃透。

国外戒色文章讲到戒色后:“浑身散发正能量,真实,纯净,善良,你会吸引正能量的人和事,灵魂更加轻盈,感觉就像是新生儿一般,天真无邪,能够直视他人的眼睛,开始真的活着,而不是行尸走肉。”戒色后一连串奇妙的体验在我身上发生了,我第一次感觉到自己是作为一个真实的、活生生的生命存在着,不再浑浑噩噩,不再行尸走肉,不再充满戾气,第一次感受到生命那无与伦比的美。我开始变得鲜活、明亮、喜悦、清透、纯粹,生命闪耀着纯净的光彩,发自内心地快乐,内心轻松了,脸上爱笑了,对周围人开始有善意了,懂得感恩、包容和无私奉献,也有了责任感和担当。我抬头仰望着星空,被它的静美深深吸引,那是一种无法用语言形容的美,是彻底的静谧、安详,是无比的浩瀚和宏伟。那一刻,我契入了真我,感到无比的幸福和快乐。在我们每一个人的内心深处,都藏着一个纯净的孩子,让我们做回纯净纯善的自己,让我们活出真我!如清泉般清澈柔软,如莲花般宁静芬芳,如钻石般坚固永恒,戒色是生命中永不褪色的光芒!照亮自己,照亮身边的人,照亮全世界!加油!奋进!!!

附《德育启蒙》,为印光大师著。

\begin{quotation}\it
    凡二十八偈。所言多正心、诚意、修身、齐家之要事。措词浅显、简便易行。

    但人常忽之,盖知之非艰。如能逐项做好,自能淑世育人,稳定社会,如能实践躬行,于人于己均得受用。

    \begin{adjustwidth}{-1em}{-1em}
        \begin{description}
            \item[孝亲] 身体发肤,受之父母,父母与我,实为一体。我爱自身,应孝父母,能不辱身,便是荣亲。
            \item[友爱] 兄弟姊妹,手足骨肉,痛痒相关,休戚与共。兄爱弟敬,和和睦睦,相推相爱,家庭之福。
            \item[敬师] 师严道专,人伦表率,道德学问,是效是则。养我蒙正,教我嘉谟,不敬其师,何能受益。
            \item[择友] 近朱者赤,近墨者黑,朋友相处,有损有益。益者近之,损者远之,劝善规过,端赖乎兹。
            \item[布衣] 衣取遮体,兼以御寒,大布之衣,惜福养廉。莫羡绸缎,锦绣华美,折了福寿,自暴自弃。
            \item[蔬食] 蔬食卫生,肉食伤生,杀时恨心,其毒非轻。勿贪吃肉,吃了须还,还的时候,真个可怜。
            \item[惜字] 字为至宝,远胜金珠,人由字智,否则愚痴。世若无字,一事莫成,人与禽兽,所异唯名。
            \item[惜谷] 田中五谷,以养人民,爱惜五谷,即是善心。修善者存,不善者亡,惜谷获福,殄谷遭殃。
            \item[惜阴] 七十古稀,弹指即过,过则已无,何敢懈惰。努力勤学,立德立业,自利利他,为世作则。
            \item[仗义] 一举一动,唯义是取,义之所在,无往不利。小人见利,即忘其义,虽得小利,究竟吃亏。
            \item[清廉] 人生福泽,前世所修,非义而取,是食毒物。清而不污,廉而不贪,世所崇敬,荣无加焉。
            \item[知耻] 耻之一字,其利无穷,有与圣近,无与兽同。惭耻之服,无得暂卸,我佛训诲,庄严第一。
            \item[尽忠] 一秉真诚,不被妄侵,事亲接物,了无二心。祗期尽分,不计人知,如是之人,堪为世仪。
            \item[守信] 守信之人,言不妄发,说到做到,不矜不伐。无信之人,事事皆假,人所厌弃,不如牛马。
            \item[仁慈] 仁爱慈悲,心之生机,此心愈真,福泽愈深。若无此心,势必残刻,纵有宿福,折尽受厄。
            \item[不杀生] 凡属动物,皆有知觉,贪生怕死,唯命是惜。若戏顽杀,及杀而食,现生后世,决定报复。
            \item[不偷窃] 凡有主物,不可偷取,偷小丧品,偷大招祸。偷人之物,折己之福,欲得便宜,反吃大亏。
            \item[不邪淫] 淫欲为害,伤身丧志,虽属夫妻,亦当节制。若是邪淫,更非所宜,古今志士,无一犯之。
            \item[不说谎] 言为行表,是本心术,心既不真,行何能正。望尔后生,切勿妄语,口是心非,终无结局。
            \item[不吸烟] 烟俱勿吸,以伤卫生,口气常臭,熏天熏人。鸦片香烟,其毒极烈,花钱买害,痴人可怜。
            \item[不饮酒] 酒是狂药,饮必乱性,醉则反常,越礼犯分。最好勿吃,免致大喝,聪明智慧,常保清白。
            \item[不赌博] 赌钱博奕,丧志失时,专心于此,正事弃遗。有限光阴,送之儿嬉,破家荡产,罪无了期。
            \item[不奢侈] 奢侈夸富,买祸买贱,君子下看,盗贼来劫。布衣蔬食,圣贤仪式,现生后世,人各取则。
            \item[不傲慢] 傲慢轻人,实自呈短,明人知伊,学养俱罕。纵到圣位,犹不轻人,绝无凡圣,念存于心。
            \item[不嫉妒] 人有才德,我当赞叹,彼于社会,必有贡献,若生嫉妒,是谓愚痴,业报夺汝,宿世慧思。
            \item[不偏见] 人有小智,未闻大道,每执己见,以为最妙。坐井观天,所见者小,若登高山,前见自了。
            \item[不迁怒] 有富贵人,气量或小,每因拂意,忿怒牢骚。迁怒无益,自他烦恼,海涵宽恕,是无价宝。
            \item[不耻问] 能问不能,多问于寡,冀人从己,故先自下。若是无知,尤当问人,博学审问,造诣方真。
        \end{description}
    \end{adjustwidth}
\end{quotation}

戒者应该具备的七十项品质:

\begin{table}[h]
    \centering
    \begin{tabular}{ccccccc}
        谦虚 & 恭敬 & 刚正 & 真诚 & 孝顺 & 敬师 & 忠诚 \\
        知恩 & 感恩 & 慈悲 & 温厚 & 和善 & 包容 & 仗义 \\
        和谐 & 团结 & 勤俭 & 公正 & 仁爱 & 宽恕 & 守信 \\
        自律 & 自信 & 积极 & 乐观 & 豁达 & 大度 & 知耻 \\
        尊老 & 友爱 & 忍让 & 知礼 & 守礼 & 重德 & 慎独 \\
        自强 & 坚定 & 精进 & 恒心 & 果断 & 好学 & 无怨 \\
        勇猛 & 坚毅 & 坚韧 & 耐心 & 沉稳 & 自省 & 负责 \\
        泰然 & 从容 & 镇定 & 严己 & 无私 & 利他 & 让功 \\
        守浅 & 守愚 & 谦卑 & 低调 & 谨慎 & 平和 & 惜福 \\
        热忱 & 高尚 & 纯粹 & 纯真 & 庄严 & 至诚 & 至善 \\
    \end{tabular}
\end{table}

\begin{poem}[崇高的灵魂]
    \begin{multicols}{3}
        \begin{center}~\\
            那个无知的少年 \\ 在磨床 \\ 满脑子的龌龊邪念 \\ 磨光了所有的纯真与美好 \\ 彻底的放纵与堕落 \\ 站在电视前 \\ 面对黄片的超强诱惑 \\ 少年疯狂动作 \\ 最后把整个灵魂都给射掉了 \\ 双腿发软发抖站立不稳 \\ 镜子前 \\ 少年耷拉着脑袋 \\ 眼皮都抬不起来 \\ 一脸的灰暗鬼气 \\ 好无神好颓废的感觉 \\ 还有一种恶心感 \\ 像行尸走肉一样 \\ 手术台上 \\ 少年在痛苦挣扎 \\ 在向医生求饶 \\ 度秒如年 \\ 背部全部汗湿 \\ 一台手术结束 \\ 整个人都快虚脱了 \\ 躺在医院病床上 \\ 呆呆地望着天花板 \\ 少年长大了 \\ 变成了青年 \\ 邪淫更加变本加厉 \\ 心理也开始变态 \\ 熬夜纵欲,作死的节奏 \\ 终于神经症爆发 \\ 生不如死 \\ 每天活在恐慌与绝望中 \\ 人生进入了至暗时刻 \\ 就在这时青年爆发了最后的斗志 \\ 死地则战!破釜沉舟! \\ 终于绝杀了心魔!冲破了怪圈! \\ 终结了心魔十几年的奴役! \\ 绝杀的那一刻 \\ 仿佛整个灰暗的天空都明亮起来 \\ 仿佛整个身心都轻盈起来 \\ 仿佛整个世界都美好起来 \\ 绝对荡气回肠惊心动魄的绝杀! \\ 这个少年、青年就是我 \\ 我的脸庞褪去了猥琐与颓废 \\ 一股刚正之气充斥于眉宇间 \\ 如钢似铁,威严不可侵犯 \\ 人生进入了光明而崇高的境地 \\ 戒除邪淫,归来仍少年! \\ 看着鸽群飞过天空 \\ 我闭上了双眼,张开了双臂 \\ 脸上露出了喜悦而幸福的笑容 \\ 就像一个纯真无邪的孩子 \\ 活出纯净的自己 \\ 活出崇高的灵魂 \\ 是每个人的天命 \\ 正己化人,帮助更多的人 \\ 是每个人的天职 \\ 让我们一起努力,加油!
        \end{center}
    \end{multicols}
\end{poem}
