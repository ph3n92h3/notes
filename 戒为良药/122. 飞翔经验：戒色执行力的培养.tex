\subsection{戒色执行力的培养}

\paragraph{前言}

上季一位戒友问:“心魔是真实存在的吗?我们身体里真的有心魔吗?还是只是一种比喻呢?”首先要弄清楚什么是心魔?心魔这个概念明确的指向就是邪念,负面的念头,很多人都会提及心魔,心魔这两个字带有神秘主义不可捉摸的感觉,好像是内心一股神秘的邪恶力量,很难说清楚。虽然他们会提及心魔这个词,但很少有人知道心魔确切的指向,你问某个普通人什么是心魔?他很可能一脸茫然,说不上来,也许只有一点非常模糊的认识,说不清楚。心魔这个词来自于修行,在修行方面是一个很常见的词,很多大德都用到过这个词。从念头袭脑来看,心魔是真的,这不是比喻,因为邪念的确会上来,入侵并控制你;从另外的角度来讲,心魔就是邪念魔,把邪念比作魔,魔代表负面力量,从这个角度来看,可以说是一种比喻,并不是你的身体内有一个实在的魔,如果你觉得你身体内有个实在的魔,那只会增加你恐惧的心理,而且弄不好还会陷入迷信。对于心魔不要害怕,其实说白了就是邪念而已,《菜根谭》云:“降魔者先降自心,心伏则群魔退听。”心生种种魔生,心灭种种魔灭,这里的心指的就是邪念。有时大德也会用“贼”这个字来形容心魔,王阳明先生:“破山中贼易,破心中贼难。”黄念祖老居士说过:“由认识贼到打倒贼、清除贼,最后贼也归顺了。”又说:“在这个贼当家的时候,什么荒唐的事情都可能干得出来,它是贼嘛,贼当家,不可能出好主意。”凡夫都是认贼作子,不懂得对治邪念,往往把念头当做自己,跟着念头跑。能够认识到这个心魔贼,就是了不起的发现,可以说是里程碑式的大事件,当你发现是邪念在作怪,在捣鬼,在奴役你,你就知道整件事到底是怎么回事了。最近看了一个关于作家杀人犯刘永彪的法制节目,他已经加入中国作家协会,曾获得“安徽文学奖”,发表作品 200 多万字。1995 年 11 月 29 日晚,浙江省湖州市织里镇发生了一起抢劫杀人案件,闵记旅店的老板、老板娘、孙子及一名来自山东的商人被人杀害。两名犯罪嫌疑人逃亡 22 年后,终于被警方控制,其中一名杀人犯就是刘永彪,后来成为了知名作家。他在监狱里接受采访时就说自己:“心里有一个恶魔,时不时地跳出来。”还提到了“一念之间”,很显然他说的恶魔其实就是邪念,他听从了心魔,犯下了大罪!他是一个有才华的人,可惜一念之间没把握好,就沦为了杀人犯。每个人都有心魔,只有战胜自己的心魔,才能真正主宰自己!被心魔奴役,整个人生都会变成一场灾难。看过一篇反腐的文章,里面讲到:“纵观贪官们堕落的前车之鉴,无非都是贪财恋色的心魔在作祟。”另外一篇反腐文章里提到:“美人关本来就是一道魔坎,比金钱关更难跨越,金弹、肉弹,一旦爆炸,就会让你粉身碎骨。”说得可谓一针见血,人来到这个世界上有一个终极挑战,那就是战胜自己的心魔!面对各种诱惑时,要经得起考验。

下面分享一些案例。

\begin{case}
    中考结束,分数以及分数线都出来了,我也终于考上了我市的重点高中。感谢各位吧友以及飞翔哥一直在背后默默地支持着我,让我戒色有点起色,目前已经稳定快两个月了,虽然不是很长,但是我感觉我的变化真的很大。记得是去年 11 月份来到戒色吧的,刚开始看到吧友们的惨痛经历我也感觉到有点害怕了,于是下定决心想要戒色,但是每次都戒到一个星期左右就破戒了,我又放弃了戒色一段时间,大约堕落了一个月左右吧,看着我的气色越来越不好又害怕了,又打开了戒色吧,真的是更加坚定信念了。但那个时候我还是在盲目戒色,以为只要忍住就可以,但是我真的抵挡不住邪淫的诱惑不断破戒,看到吧友们都说多学习戒色文章很有帮助,试着试着就开始走向正确的方向了,有点名堂了。就这样我戒色了两个多星期就感觉变化很大,上课的时候注意力更加集中,跑步也更加有力气了,同学们也对我的变化感到惊讶。说实话我以前在班级里也就是半个小混混而已了,没人在乎我。那段时间,中考到来的压力也是特别的大,不过我的成绩也在稳步上升,老师也觉得我进步真的很大,毕业之时给我留个优秀毕业生的名额,我真的是高兴坏了。寒窗苦读加上一心戒色让我考上了我理想的高中,感谢戒色给我带来的好处,感谢戒色吧,感谢各位吧友,再接再厉!
    \subparagraph{附评} 戒色吧有很多初高中戒友,他们要面对中考和高考,是非常重要的人生阶段。备战中考和高考需要很好的学习状态,然而手淫恶习会让人脑力下降,精神萎靡,身体抵抗力也会下降,会影响一个人的学习状态。记得我是初中时染上了手淫恶习,打开了潘多拉的魔盒,一发不可收拾,那个时代也就是上世纪 90 年代末,根本接触不到戒色文章,在那个怪圈里我挣扎过,彷徨过,自己也尝试戒过,但一直失败,从没超过一个月。我为什么初中就想戒撸呢?因为我明显感到手淫后身体变差了,容易感冒,出现尿频,鼻炎加重,痤疮加重,气色差,腰痛腿软,眼睛浮肿,脑子变钝了,上课注意力不集中,理解力下降,这些伤精症状都是真真切切的亲身体会,那时我就隐隐约约觉得手淫不好,但心里还不是很确定,因为比较无知,脑子里有思想误区,也没人告诉我手淫的危害,在那个年代也没人指导我如何戒色,完全就是一味靠毅力强戒,结果注定失败!上次看到一位戒友说:“戒到一个月左右,心魔一来,就马上破戒。”看到他的这句话,把我的思绪拉回了我过去被心魔反复虐的岁月,那时我真的很想戒掉这个害人的恶习,做回纯净自信的自己,但是我做不到,因为“心魔一来,就马上破戒。”那时的我也不知道什么是心魔,只知道心里有一股异常强大的力量把我一次次拉入那个怪圈,我根本无法抗衡那股力量,心魔就像一个掰手腕的绝对强者,弱小的我就是想抗衡也只是稍微抵抗一会就屈服了,没办法,实力不行,只有被虐。我那时很需要实战的指导,但苦于那个时代没有戒色文章,戒色多次失败后自己也放弃过,最后还是被症状一次次逼回戒色这条路上来,因为不戒不行了,不戒身体就完蛋了,伤到一定程度自然会选择戒色,身体根本吃不消那种疯狂的耗泄,欲望越来越重,身体已经无法承受那种耗损了,撸到后来真的一次都伤不起了,射过之后症状就给你颜色看!不少人已经撸到站不稳了,随时都有倒下的可能,生命力已经被这个恶习无情地抽走了。现在初高中的孩子,他们很幸运,虽然网络色情很泛滥,但是却能接触到戒色文章,可谓危机与机遇并存,在这个时代戒色,最终要达到的境界就是出淤泥而不染,即使色情诱惑满天飞,内心依然能做到如如不动,这需要很高的戒色觉悟和更强大的定力。这位戒友考上了重点高中,很不容易,从班上的小混混逆袭重点高中,这是多么大的蜕变啊!从他的表述来看,可以感受到那种积极向上的状态,戒色给了他良好的精力和脑力状态去应对自己的学业。通过学习戒色文章他开始步入戒色正轨了,过去他也是盲目地强戒,后来意识到学习的重要性,就渐入佳境了。我真替这位戒友感到高兴,初中的孩子还是很单纯的,身体恢复也快,希望他们好好把握自己的人生,远离邪淫堕落的生活,健康快乐地成长。
\end{case}

\begin{case}
    飞翔大哥,在这里向你表示最真诚的感谢,戒色养生 230 天加吃中药,整个人完全刷新蜕变了,大脑像刷新了一般,工作起来脑子不再像团浆糊,脑子很清晰,整个人身体状态不是好,是相当好,正气爆棚,常常生出一份与日月争辉的豪情,整个人对幸福的感受也更细腻了,下班公家车上看着窗外的灯光都觉得充满美好安宁祥和。对不起说得有点多,实在是想与大家分享一下现在的美好感受,想起 230 天前破戒之后,整个人完全像是要死了一样,生活在地狱里,惶恐不安,头脑昏昏沉沉,头紧有压迫感,抑郁悲观,现在想起来都觉得……唉!还好老天没有放弃我,这所有的所有,现在在 230 天后都一扫而空,不敢骄傲,更不敢炫耀,还必须精进勤勉,谦虚谨慎,也希望更多的戒友可以走出邪淫,戒色养生,症状严重应该去吃点中药。希望所有的戒友重新开始阳光健康的生活!
    \subparagraph{附评} 230 天前,这位戒友的状态“像是要死了一样”,他生活在地狱里,应该已经有神经症了,还好后来他开始戒色了,230 天后整个人焕然一新,他用到了“刷新”这个词,就像一个系统瘫痪了,被各种病毒占据,这时重装系统,一下完全刷新了,又回到了良好的身心状态。要达到“刷新”的效果,需要能量的不断积累,积累到一定层级,整个人的身心感受就会截然不同,那种感觉很微妙、很美好,脸上会有一种微妙的喜悦感,眼睛也有喜悦的感觉,变得爱笑了,清澈明亮的眼睛可以点亮整座城市,感觉生活又重新充满了希望。自己真正能主宰内心了,不再是心魔的傀儡了,这是一种莫大的解放,心里自由了,整个人轻松了,各种美好感受都回来了。一位戒友说:“说实在的,戒色真的改观很大,首先眼神是最明显的,戒色后一段时间,眼睛不会那么无神呆滞,而且在未戒色之前,不知道是怎么回事,不管别人知不知道你有没有手淫,你能感觉到他们有些反感你。戒色可以使得脸部不再浮肿,可以使得眼睛变得有神,不再像枯萎的植物一样,萎靡不振,毫无生气。”戒掉手淫后,眼睛的亮度会提升,明亮程度是一个很重要的观察参数,孩子的眼睛基本都很明亮,成年人的眼睛明亮的很少,大多看上去比较暗淡,我现在很注意观察一个人眼睛的明亮度,看一个人我会先看他眼睛的明亮度,这可以说是观察一个人的金指标,你可以注意观察你遇到的每个人,你会发现每个人的眼睛的亮度都存在差异,亮度可以划分为:极暗、暗、微暗、微亮、亮、明亮、特亮、亮彩!你可以从这八个划分来观察人的眼睛的亮度,亮度代表的就是能量值!能量越足,眼睛越明亮,而且会显得很清澈,很有神,有一层净光膜,就像抛光过的大理石一样。能量越弱,眼睛亮度就越暗,撸者的眼睛大多显得很灰暗,那种眼睛看上去死气沉沉的,缺少生气。孩子一旦开始手淫,他们的眼睛就会像星星一样暗淡下去。亮度最后一个划分是“亮彩”,不仅亮到极致,而且还有一层很微妙的光彩,让人一看就觉得非同凡响,亮彩是能量极足时才会出现。我专门研究过相术,凡是昏暗的气色就是不好的象征,各种部位看上去昏暗就代表能量弱,明亮则代表能量强。之前看过一篇专业运动的文章,里面提到了教练员怎么判断运动员的运动状态,很重要的一点就是观察运动员的眼睛,特别是眼神,如果精气神很足,那就代表运动状态不错,能保证训练质量,如果眼神涣散、呆滞无神,那就代表运动员不在状态,需要作出调整,训练强度不能太大,否则极易受伤。这个案例的戒友说自己“生出一份与日月争辉的豪情”,这其实就是有底气的表现,戒色之后正气和底气都会起来,正气和底气是密切相关的,正气足的人往往很有底气。戒色后的幸福感也会满溢,那种对幸福的感受的确更细腻了,你的关注点和思想已经从邪淫上移开了,这时你开始注意到那些曾经被自己忽视的生活中的美好,戒色的确可以给人一种微妙的幸福感和特别美好的感受,这是看多少黄都无法给你的,实际情况是——你看的黄越多,你就越背离那种单纯幸福美好祥和的状态。这位戒友最后提到了“不敢骄傲”,这点说得很好,人就怕骄傲,一骄傲就会被心魔钻空子,所以不管戒多久都不能骄傲,一定要修好谦德!唯有谦光自牧,勤修善行,方能步入戒色大成之境。
\end{case}

\begin{case}
    昨晚破戒了,戒了 21 天,以前最多戒过 25 天,我有种感觉,每到这种时候欲望真的会非常非常强烈,就那一个图像上脑就能把我抓得死死的,我能觉察到,知道我在意淫,但断不掉,太诱惑了,那种感觉就像小动物看到天敌,连挣扎和反抗的力气都没有直接就僵住了,很无奈,只有任凭图像一帧一帧播成小电影。
    \subparagraph{附评} 这是一个破戒的案例,一般戒到一个月左右就会进入破戒高峰期,到时心魔进攻会很猛烈,戒色有非常实战的部分,那就是观心断念!资深戒友都知道实战才是重中之重,脱离实战的空谈必定会失败,有的人自以为看过很多戒色文章,以为自己觉悟很高了,但是当心魔入侵时,他的实战表现和菜鸟没有区别,这就是纸上谈兵!被心魔一次次攻破,一次次沦为撸管肉机,像着了魔一样疯狂找黄,为什么会这样?就是因为那个念头或者图像上头时,没有立刻断掉,导致木马程序安装成功,然后执行疯狂找黄、疯狂看黄、疯狂手淫。那个念头或者图像就是入侵的木马程序,特别是图像上头极快,一位戒友反馈说:“我今天在车上,突然一个图片念头闪进我头脑,图片非常快,我马上警觉。”实战就在刹那,当图像闪进头脑,也就是你亮剑的时刻,有的人虽然能觉察到自己在意淫,但是断不掉,原因就是断力不行!就像劈砖一样,我看过功夫深厚者五块红砖叠在一起,一掌下去就全裂了,功夫的确很厉害,之所以那么厉害也是平时练出来的,没练过的人可能一块都劈不开,不仅要掌握技巧,也需要持之以恒地练习,人家练铁砂掌的师傅真的把手练得跟熊掌一样厚实,劈砖就像切豆腐一样容易。断念也是如此,刚开始和心魔交战,往往屡战屡败,这时候自己就要找原因了,为什么断不掉?为什么前辈可以断掉?破戒后要认真总结和分析,然后一定要注重练习断念,关于断念的理论也需要深入学习和领悟。从屡战屡败到百战百胜,这需要一个过程,我当戒色菜鸟时也是屡战屡败,怎么弄也弄不过心魔,对于那个阶段的我而言,心魔实在太强大了,我根本无力招架。后来我强大起来后,自然就能降伏心魔了,我已经具备战胜心魔的把握了。我之所以能做到就是因为坚持学习和练习,每天都在坚持,坚持到一定程度,实力就会突飞猛进,到时就有绝对的实力降伏心魔。从初学乍练、初窥门径到登堂入室、出神入化,这里面要下很大的功夫,也要看自己的悟性,刚开始很多新人根本不懂得如何戒色,也不懂得观心断念,一点概念都没有,心魔一来,他们马上就会破戒,当学习了前辈的戒色文章后,开始有点摸到边了,知道要断念,不能跟念,这时也只是一个粗浅的明白,还缺少深入的领悟和练习。当他们开始每天练习断念了,进步就会很快,每天练习加上每天学习前辈经验,学习和练习互相交融互相促进,到时断力的提升就会日新月异。这位戒友提到图像把他抓得死死的,就像小动物遇到天敌,一下就被心魔克住了,没有那个实力肯定会被克住,因为心魔强过你,所以它能克住你,如果你强过心魔,心魔就无法克住你!反而你把心魔给克住了!古人讲克己功夫,克在文言文中有战胜之意,要战胜自己的心魔,需要极其强大的断力!好好修炼自己的断力吧,成为心魔的克星!前几天看到一位戒友在帖子里中说自己:“只会纸上谈兵,实战简直弱爆了!”他已经知道自己的问题了,就是空谈理论,脱离实战,心魔一来,立刻跪地屈膝,尽现奴颜,迎接心魔的不应该是你的膝盖,而应该是你的断念利刃!那种决一死战、宁死不屈的气概!有的戒友还提到了口诀无效,其实并不是口诀无效,而是缺少练习,工欲善其事必先利其器,磨刀在平时,用刀在刹那,功在平时,平时疏于练习,实战时就会表现糟糕。大家想想,奥运冠军练了多少年?如果明天比赛,今天才开始练习,这样行吗?必须平时持之以恒地练习,这样实战水平才能稳步提升,这种提升是一种持续渐进的过程,不是一蹴而就的。等你真正强大了,心魔再上来试试!来一次,灭一次!心魔就是来送死的!
\end{case}

\begin{case}
    14 岁在省级青少年篮球队担任队长,教练说我是一个不可多得的天才,因为染上手淫之后就一发不可收拾,然后逐渐地不想运动,2015 下半年身体状态下滑非常大,体力也渐渐下降,还向教练提出退队,随后一直呆在家里打英雄联盟。就这样过了差不多一年,然后发现自己很有打游戏的天赋,打了差不多一年的英雄联盟就在一区上了钻石一,2016 年差不多就出过四次门,爸妈一直很失望,我不知道为什么会变成这样,撸管频率越来越多了,一天一次,就这样玩游戏的反应和手速越来越慢,从有个篮球梦到电竞职业梦,一次次地被撸管产生的后果给破坏。2017 很迷茫,游戏也不想玩了,更不用说出去运动,还有过自杀的念头,求大家救救我!
    \subparagraph{附评} 这位戒友算是一个篮球天才,能够在省队当队长是很不容易的,也许将来可以打上 CBA,可惜他染上了手淫恶习,身体素质下滑很厉害,最后离队了。我想中国有很多人才都是天赋异禀,出类拔萃,但是后来却泯然众人甚至过得很糟糕,不如普通人,这背后都有一个相同的原因,那就是沉迷色情与手淫。生活中我也见过一些人才,在某一方面很突出,的确很有天赋,可惜他的电脑和手机里存了太多黄片,那些黄片就像毒品一样,他一次次在虚无缥缈的快感中迷失自己,最后站在镜子前的就是一个猥琐憔悴被掏空的废人!他的命运开始多舛,生活开始不顺,脾气也变得暴躁,人生陷入了困境。这位戒友选择了打游戏,变成了宅男,爸妈很失望,这是在荒废自己的青春,很多人虽然热爱电竞,但父母往往不支持,自己要深思是否适合打职业电竞,这条路是否走得通,人要有自知之明!打职业电竞要付出很多很多,训练量也很大,长期久坐乃至熬夜是会把身体弄坏的,加之撸管恶习的摧残,身体垮下去就更快了。我个人不看好职业电竞,在打打杀杀的世界里只会增加自己的负能量,不少电竞选手狂妄得很,负能量很重。戒色的同时应该把网游也戒了,网游太耗费时间,现在不少网游还要花钱买装备,钱财也耗费了,很多网游还有色情元素,会让人起邪念,所以网游还是不玩为好。有的戒友会问我是否可以玩游戏,那要看什么游戏了,休闲益智的还是可以考虑的,但也不可沉迷其中。不管是篮球还是电竞,要靠它吃饭真的是很难的,需要极佳的天赋加努力,青少年阶段应该还是以学业为重,不要一直宅在家里,应该走出去多接触外界,多和人交流沟通,平时也要有良好的锻炼习惯,从年轻时就要有养生意识。天天宅在家里,游戏加撸管,这种生活实在太颓废了,父母也会极其失望,年轻时要多出去尝试,不要把自己局限于某一个领域,可以出去学点什么或者做一些实际的工作,慢慢积累人生经验,在实践中往往就能知道自己真正适合做什么,思想也会开始变得成熟起来,到时再回头想想过去的自己也许就会觉得可笑和幼稚,人生其实可以走得很宽广,关键自己要勇于迈出第一步,勇于去尝试,即使碰壁也没事,多碰壁慢慢就有经验了,将来可以更好地完善和提升自己。
\end{case}

\begin{case}
    我质疑过戒色吧,嘲笑过,直到我自己伤了我才信了,这里的人有大智慧、大善心!邪淫让我有了瘾,我每天在对抗大脑分泌的这种瘾,很难受,从一天一次到一天三次,甚至一天五次,越来越多,而身体越来越不行,最近一天一顿饭我却莫名其妙地发胖,失眠、脱发、腰疼、尿不尽、社交恐惧、耳鸣,这都无所谓,最要命的是我体力不行了,我是指着身体吃饭的!我昨天去卸车,从早上四点干到七点我就腰疼干不动了,甚至走路都觉得踩在棉花上,好像没有腰了,摔倒了两次!现在在躺着!我会不会完蛋了,我尿尿有些红色。
    \subparagraph{附评} 之前一位戒友说:“以前以为戒色很可笑,现在症状全出来了……”症状出来后就知道手淫是要还的!用痛苦加倍地还!很多无知的人被无害论洗脑,乍一听到戒色,他们会觉得很可笑,甚至觉得戒色的人都是脑子有问题,不正常,甚至恶意诽谤戒色吧,然而真正有修养的戒者是不会动怒的,只是笑而不语,因为症状迟早会告诉他们真相,等到症状爆发时,他们还笑得出来吗?!他们还敢嘲笑戒色吧吗?想象一下他们的嘲笑瞬间僵在脸上,眼睛中露出惊恐的神色,嘴角不停地抽搐着,为何?因为一记症状的闷棍正中他们的脑门!有些人还会用无害论和你辩论,引用所谓权威的言论来反驳你,显得趾高气昂不可一世,似乎他的每句话都充满了科学,好像爱因斯坦是他爹一样,但是当症状爆发时,他们就自己打脸了!他们的雄辩在症状面前根本不堪一击,他们的雄辩就像一个吹大的气球,被症状的针稍微一戳,就“啪”地一下猛然破裂,剩下他们悔恨无比、泪水长流的表情……我们不要和那些无知的人辩论,症状自会收拾他们,人就是这样,不到黄河心不死,不见棺材不掉泪,等到症状爆发时,他们自会来到戒色吧,这时才发现戒色是多么英明和正确。一个人一旦开始邪淫,就会在邪淫的泥潭里越陷越深,那种瘾是会逐步加重的,就像吸毒一样,刚开始剂量小,最后不满足吸食了,开始注射了,剂量越来越大,撸者会体验到那种极端失控的感觉,着了魔一样,根本停不下来,有时已经体验不到快感了,就是为了完成任务,那个任务只是一个射出的例行程序,一定要射出才算结束。心瘾越来越重,但是身体却越来越不行,这时候真的就危机四伏了,生活变成了八面埋伏,指不定什么症状会袭击你,最初是心魔袭击你,把你变成撸管肉机,然后症状袭击你,把你变成医院的常客,把你变成药罐子、病秧子。可怕!实在太可怕了!身体好时根本不觉得,等到身体崩溃了,那真是暗无天日、绝望无比的体验,真的生不如死!这位戒友一天吃一顿饭,他的做法是不对的,他是靠体力活吃饭的,应该保证一日三餐,营养一定要跟上,身体已经这么差了,还要应付体力活,生活开始变得艰难,他这种情况最好能辞职休养一段时间,不要带病干体力活,否则病情很可能会加重,弄不好会累出肝病。不管是体力活还是脑力活,一个好身体都是基础,家庭事业学业都需要一个好的身体,一个灵光的大脑,一个挺直的腰板,真的不能再撸了!那些嘲笑戒色吧的人,我从来没怪过他们,只是觉得他们很无知很可怜,当嘲笑的表情变成惶恐绝望时,那是怎样的痛苦呢?只有亲历者才明白。
\end{case}

\begin{case}
    我在床上躺着,输液的针眼在我干瘪的血管上排成一排,还没有愈合。前些日子我又高烧住院了,我感到浑身像被烤干了一样,一点火星就能把我点着,每一块骨头都像被醋泡了一样,又软又酸又痛。好想把他们拔出来擦一擦。持续的高烧不退让我的血压降不下来,脑袋里面的血管好像要涨开了一样。医生拿我也没办法,她说所有的药都用了,用酒精擦一擦吧。就这样我在医院里又高烧了四天,我排出的小便成墨绿色,医生说我高烧把小便里面的血烧得都变颜色了。我 SY 不到六年,在学校上课住校,一次重感冒高烧之后,出现了肉眼血尿,去医院检查,知道自己得了严重的肾病,已经出现了肾衰竭的症状,肌酐已经开始升高了。我的人生走进了地狱,我从小医院转到大医院,从我们省走到了全国各地。可是没有医生能够治得好。他们说如果能够延缓发展就不错了。渐渐地我的并发症越来越多,身上开始肿了,体重却越来越轻了,吃一点东西就想吐,一点力气也没有,睡觉就像要昏过去一样,醒来感觉好像刚刚走完沙漠似的,累得不行。我的眼睛也往下流一种粘液,视力下降得也很厉害。医生说这种病就是全身的器官都会受到影响,我还有各种各样的说不清的难受滋味,这些痛苦一点喘息的时间都不给我。我现在就一直吃药维持着,病情也一直在发展,根本遏制不住。我明白这就是我 SY 的恶果,后悔自责让我本来被疾病折磨的身体更难受了,我的内心承受不了这么严重的报应经常崩溃流泪,彻夜失眠。我对不起我的父母,对不起我自己,对不起所有关心过我期望过我的人,我好想回到从前,找到那个开始邪淫的自己,我要把他绑起来,我要贴着他的脸,看着他的眼睛,好好地给他讲讲可怕的报应。告诉他生活不是这么过的,人生不是那么活的,但是我回不到了,再也不能重来了。如今我和大家说我的经历,我希望大家吸取我的教训,悬崖勒马。多少个夜晚,我都希望一觉醒来自己的过去只是个梦啊,只是个假的梦,虚幻不实的梦。我还是那个没有染上恶习的自己,有一个美丽的梦想,在这个世界上的一角,尽自己的力铸造着自己的未来。
    \subparagraph{附评} 这个案例可以说是触目惊心!肾衰竭、尿毒症属于比较严重的疾病,肾衰竭要是发展成为尿毒症,简直就是噩梦,需要靠透析维持生命。肾衰竭和尿毒症,肾脏受损的程度是不一样的,肾衰竭肾脏受到的损伤要比尿毒症小,但是如果得不到控制,肾衰竭很快就可以发展到尿毒症。从中医角度解读肾衰竭的原因,\begin{description}
        \item[外邪侵袭] 肾病久治不愈,脏腑功能下降;
        \item[烦劳过度] 长期从事重体力劳动,疲劳过度则损伤脾肾;
        \item[] 酒色过度,损伤肾气,肾阳不足,命门火衰,火不暖土,脾肾两虚;
        \item[饮食不节] 饮食不节包括暴饮暴食、长期饥饿、偏食等;
        \item[情志所伤] 长期情志不舒,忧思恚怒,肝失疏泄,肝气郁结,三焦气机不畅
    \end{description} 撸者应该属于第三条病因,之前就有一些撸出肾衰竭、尿毒症的案例,还记得那个戒友躺在病床上一脸绝望的样子,他来到戒色吧发帖也是为了警示大家:疯狂手淫的恶果是很可怕的!每个人的体质是不一样的,体质薄弱加之伤得深了,后果真的不堪设想,活生生撸进了症状地狱,每天都过得异常煎熬,肾衰竭会严重影响患者的生活质量,并且带来沉重的经济负担。这位戒友撸龄不到六年,并不算长,但如果这六年的某个阶段很疯狂,不要命地撸,熬夜撸,并且不好好吃饭,作息不规律,这样身体真的会突然崩溃的,就像一根橡皮筋绷断一样,达到一个临界点,突然就不行了。有的人先天体质就差,根本承受不起这种疯狂的纵欲,几乎是自杀式的疯撸。撸者得个前列腺炎还算好的,得了神经症会很痛苦,而得了肾衰竭、尿毒症,那简直就是一场灾难。有调查显示,中国肾衰竭病人正越来越多,我国慢性肾病发病率为 10.8\%,在国内,终末期肾衰竭,称之为“尿毒症”,很多患者靠血液透析或者肾脏移植替代肾功能,在浙大一院的肾脏病中心,每年血液透析近 10 万人次,移植数量每年 300 人,尿毒症患者长期生存 20 年以上的占 68.9\%,这在国内是最好记录。在全身器官中,肾脏是储备功能最为强大、最“任劳任怨”的。因为肾脏的储备功能高达 80\%,只要单侧肾的 1/3 还保持着功能,就能使其正常“运转”。不像其他脏器稍有损害就反应强烈,而肾脏功能在一点点被损害的过程中,因为症状并不明显,往往会被忽视,等到肾脏功能丧失 70\% 以上后,人们才会察觉,此时,病情已很严重了。不少撸者还自以为身体还行,其实问题已经出现了,只是自己尚且感觉不明显,没有引起充分的重视,等到病情继续发展下去,就会感觉身体越来越不行。千里之堤溃于蚁穴,蚁穴很小,不容易察觉,随着时间一点点扩大,最后就会积重难返,导致灾难性后果。这位戒友最后的表述发人深省,如果能回到从前,找到那个开始邪淫的自己,把邪淫后遭受的痛苦报应讲给那个无知的孩子,也许就不会像现在这样遭罪了,然而无法回到从前,只能把自己的经历讲出来告诫大家。染上恶习前是纯真无邪的男孩,染上恶习后一切都变了,伤得深了,等待自己的就是无量的痛苦折磨。刚开始手淫时是不知道后来会发生什么的,处于极度无知的状态,这时候应该有过来人给予正确的引导和告诫,让其懂得手淫的危害,下决心尽早戒掉这个恶习。《文昌帝君训饬士子戒淫文》:“天道祸淫,其报甚速,人之不畏,梦梦无知,苟行检之不修,即灾殃之立至。”天道祸淫最速,万恶淫为首,心中淫念炽盛,疯狂沉迷色情与手淫,灾祸就离你不远了。等到自己遭到报应了,再看古人的告诫,就会觉得太对了!真的是至理名言,说得一点都没错!只是以前的自己狂妄无知,被色所迷,难以听进去,那些无知的撸者还是尽早回头吧!
\end{case}

\begin{case}
    飞翔老师您好!有一问想请教您!最近突然体会到自己最终破戒的情况往往不是念头直接地侵扰,而是一种感觉,就是感觉它存在那,但是不像念头直接上脑,它就像冰箱速冻,它不是直接变成冰,而是慢慢地慢慢地把你最终变成冰,刚开始我往往还招架得住,但是后来渐渐就不行了。而我发现它在那,但就是很难掐灭,故此想向您请教下,这种念头有没有什么办法可以在我觉察到它时,将它掐死?
    \subparagraph{附评} 微妙的感觉属于极其细微的念头,细微到还没有成形,就像雾一样。我之前也专门讲到过微妙的感觉,这种感觉就像冷空气一样慢慢渗透,刚开始出现,感觉不是很明显,你能发现它,知道它来了,但很难说清楚那种感觉,总之非常微妙。如果觉察力强大,也是能立刻消灭的,只要觉察力强到一定程度,任何念头、图像、微妙感觉都能立刻消灭,但如果你的觉察力不行,那就很难消灭它,你会被它占据,被它操控。《刘子·防欲》:“情欲之萌,如木之将蘖,火之始荧,手可掣而断,露可滴而灭。及其炽也,结条凌云,煽熛章华,虽穷力运斤,竭池灌火,而不能禁,其势盛也。”古人戒色觉悟很高,知道断念贵早!不能让它起势。木头的新芽好断,小火星也好灭,等到合抱之木就太难断了,发展到燎原大火就很难灭了。实战时发现早、断除早,就可以把握主动,当发现微妙的感觉上来时,立刻消灭之,如果觉察力不行,那就拼命念佛,即使你不信佛,也可以试试佛号的威力,佛号就是金刚王宝剑,拼命念佛即可消散那种微妙的感觉。南怀瑾先生:“实际上,你一观这个念头,这个念头已经跑掉了。”觉察力比较强的高手,一观,就解决问题了,一观,即是一记觉察,一记刀光,让入侵的念头瞬间消散,要达到这个境界,需要深入理解断念的理论,更需要勤奋练习观心断念,刚开始难以做到,后来慢慢就能做到了,到时断念就会很轻松,到了高阶只需保持一定的警惕即可,到时就可以戒得比较稳定。微妙感觉的入侵是实战中的一个难点,不少戒友都提到过这个问题,有的戒友可以立刻断掉念头、图像,但是更细微的微妙感觉,他就有点力不从心了,如果我们善于学习和总结实战经验,还是可以战胜微妙感觉的,关键要熟悉心魔的这种套路,知己知彼百战不殆,首先要熟悉,其次要学会对治,强化自己的断力。如果你对这种攻击方式了解得不深,实战中很可能就会不知所措,我在实战中与微妙感觉千百次地过招,深知微妙感觉渗透之厉害,有时一个迟疑,它就开始操控你的思想了,继而操控你的行为。很关键的一点就是保持警惕,不要认同,当你看到它了,不要跟随,不要卷入,只是看着,元音老人说过练就“不理睬”的功夫。念头的特性就是有生有灭,不管多强的念头,它出现后肯定会消融回空性,只要不理睬,它就无法操控你。后知后觉往往会陷入被动,最好是在它上来的第一刹那就把它“做掉”!第二刹那、第三刹那已经有点晚了,第一刹那是对治的黄金时间,不可错过,一定要把握这个刹那,在第一刹那时,它往往最弱小,之后的每个刹那就开始几何级数壮大变强,高手基本都是第一刹那解决战斗,够快够狠!做断念的狠角色!真正的快感来自于杀心魔!!!邪淫那点快感根本不值得一提!
\end{case}

下面步入正文。

这季是关于戒色执行力的,执行力就是把目标变成结果的能力,就是保质保量、不折不扣地完成学习或工作任务,把想法变成行动,把行动变成结果,没有任何借口坚决完成任务的能力。邪淫后精气神被摧残得很厉害,人的精力和脑力都会不同程度下降,这时候执行力也会随之下降,变得懒散拖延,不想做事,即使做事也往往半途而废,没有长性。有的人把自己撸成了废人,整天浑浑噩噩,对什么都没有热情,除了看黄手淫那会比较亢奋,其他时间都萎靡不振。这是因为伤精导致的执行力下降,随着坚持戒色养生,执行力会有所提升,我们自己也要学会培养强大的执行力。

执行力是非常重要的,小时候看打仗片,到了总攻的时候,必须坚决执行上级的命令,必须把目标攻下来,不惜一切代价,小时候觉得这种奋不顾身、拼死冲锋的精神很可贵,看着也热血沸腾,很有血性,很有英雄气概。战士们个个动作勇猛,如猛虎下山,战斗执行力非常之强。我们戒色也需要这样的执行力,特别是学习戒色文章更需要强大的执行力,学习需要热情与冲劲,更需要恒心,一天天坚持执行,注重积累,慢慢觉悟就会水涨船高。戒色过程中也可能出现暂时的退步,这时不必灰心,退步有时也是好事,为什么这样讲,因为有时退步只是一个调整,是在为更大的进步蓄力,就像起跳时要先蹲下来,下蹲正是在蓄力,是为了跳得更高。有时的确感觉进展不大,我也经历过那个阶段,但只要坚决执行学习,不中断,坚持下去突然就会柳暗花明。

执行力在管理界是提得最多的名词之一,老板或者主管会要求下级完全彻底地执行他们的指示,而戒色不仅要执行学习,也要执行学到的内容,比如看了断念的理论,不能停留在空谈,需要执行练习,练习使人强大,空谈只会沦为戒油子。之前有位戒友反馈说自己戒色笔记做得很多,但实战还是很差,虽然有一些戒色心得,但心魔入侵时还是显得那么无力,他的问题就是没有执行练习,他说他想练习断念口诀,可总是行动不起来,缺少执行力。有的戒友执行力就很强,看到理论,他就有转化成实践的冲动,他想马上付诸于实践,把想法变成行动。不少人喜欢找借口拖延,一再地拖延,不去面对,不去行动,很消极,有点逃避的感觉,信心也显得不足。他们不是很渴望强大起来,如果他们的渴望很强烈,就会迫不及待地执行练习,会很有冲劲。我们要有那种一条龙暴扣的执行力,冲劲要强,要狠,不顾一切,摧枯拉朽,就像打橄榄球时,对方每一位球员都想拉住你,而你万牛莫挽,势不可挡地向前冲去,要像撒开了腿向前冲刺的跑锋,撕开所有防守,就像抱着炸药包往前冲,拼死冲锋,那种执行力绝对强悍。当你不顾一切了,当你拼死了,心魔就尿裤子了……心魔看到你不要命了,它就颤抖了!不少戒友戒得死气沉沉的,缺少战斗意志,缺少亮剑精神,还没戒,气势上已经输掉一大半了,一开始就要拿出破釜沉舟、大干一场、同归于尽的气势!这是你死我活的战斗,根本容不得半点懦弱!必须强悍!

在断念实战方面我们也要具备超强的断念执行力,前几天看到一位戒友戒色四十多天破戒了,还是过去那个破戒流程,“前一秒抵抗,后一秒沦陷”,断念执行不力。大家都听过四个字:强制执行!不执行也得执行,当邪念入侵时,你要沉着应战,你平时的训练就为了实战的这一刻,要坚决把邪念断掉!戒色高手的断念执行力极强,他们不会让邪念起势,他们能够做到念起即断,这种强悍的执行力是不断训练出来的,也是在实战中不断培养出来的。要打造军人般的执行力,比如接到上级的死命令:明早必须攻下敌人的阵地,否则提头来见!这时候你会作何反应呢?你一定会鼓舞战士们的斗志,动员大家做好充分的战斗准备,一定要坚决执行上级的命令,坚决把敌人的阵地攻下来!当黎明来临,当嘹亮的冲锋号响起,战士们勇猛地跃出战壕,不顾一切地杀向敌阵,这是多么强大的执行力!杀声震天,白刃格斗,血光飞舞,敌尸满地,在敌人鬼哭狼嚎的惨叫声中越战越勇,戒色就要有杀红眼的执行力!邪念尽管上来,来一个,杀一个,杀光、杀净,杀透邪念几万重!杀他个片甲不留!冲破百万魔军!古德讲:“杀尽始安居!”戒色就是要杀念!不能空谈理论,要动真格的!在网游世界里打打杀杀,即使杀一万亿的怪物也不如在自己的头脑里杀一个念头怪,一开始他们就杀错了方向,最应该杀的就是自己的邪念,不仅是意淫、怂恿等,其他负面念头诸如嗔念、嫉妒念等也要统统杀光,一个不留!

戒色的威武之师,雄壮之师,钢铁之师,有着敢死队的精神面貌,眼神充满了坚毅与刚强。严格的纪律是成功的基础和胜利的保障,要严守戒色战场纪律,对境时警惕,断念时犀利,军令如山,坚决执行!拿出铁血精神,戒出开天辟地的气势,心魔狠,你比心魔更狠,心魔强,你比心魔更强!强过心魔!克住心魔!降伏心魔!剿杀心魔!要善于鼓舞自己的士气,过去打仗还有督战队,就是专门监督打仗的,有了监督,战斗执行力就上去了,贪生怕死、临阵脱逃的现象就会大幅减少,战斗就是要看执行力,执行差,自然会失败,要坚决执行到位!前辈说断念要快要狠,就要严格按照前辈说的去做,前辈说要每天练习观心断念,你就要勇猛精进地去执行,前辈说要坚持学习戒色文章不中断,你就要努力去做到,必须严格要求自己,真正拿出冲天的干劲,不折不扣地坚决执行!许世友将军在回忆录里写到:“最有力的命令就是自身的行动!我拔出雪亮的大刀,挺身冲入敌群,刀刀见红,砍得敌人血肉横溅,魂飞胆裂。”戒色就需要这种勇猛的狠劲,手榴弹和大刀开路,杀出一条血路!拿出强硬的战斗作风,彻底打垮心魔的进攻!越战越强,越战越狠!觉悟高、纪律严、士气旺、作风硬!

开始戒色,需要真正理解和领悟前辈的经验,应该反复研读戒色文章,多做笔记,多复习笔记,在坚持学习中一次次迎来顿悟,拍案叫绝、喜不自禁的顿悟,在一次次顿悟中,觉悟就像坐电梯一样上去了。平时要坚决执行对境实战的战场纪律,看到色弹马上避开,不看第二眼,色弹是来干掉你的,是绝对害人的东西。必须反复强化实战意识,从而在实战中能做出正确的选择,“内不随念转,外不为境迁”,这十个字要好好落实。有的戒友就是执行差,看到擦边图马上就陷进去了,放任自己经常浏览擦边新闻,这样是极其危险的,很容易出现破戒。看看有多少人还在贪恋女色,眼睛还在“抓”那些诱惑图片,第一眼、第二眼,一直盯着看,他们内心的贪恋还是很强的,要克服贪恋,必须多思维邪淫的危害,也可以多思维不净观、白骨观等,这样可以有效对治对女色的贪恋,不净观白骨观是为了对治贪恋、对治邪念,而不是仇视女性或者不尊重女性,这点要明确,不要理解偏了。戒色是为了更好地尊重女性,女性给了我们生命,抚养了我们,我们不应该对女性起邪念,邪淫就是不尊重女性,把女性当做发泄兽欲的工具,完全就是禽兽之举,甚至连禽兽都不如。有的戒友会说,对于那些诱惑我们的女子或者从事色情行业的女子,我们也要尊重吗?作为一个人,我们还是应该尊重,就像对于罪犯,国家还是在改造挽救,并不是仇视或者放弃。之前有的前辈说,那些女优像披着画皮的妖魔般引诱人堕落,从特定角度可以这样讲,这样讲是为了引起我们的警觉,从而远离黄片的诱惑,但并不是让我们去仇视她们。戒色是为了更好地控制自己的欲望,做一个负责任、有担当、有德行的正能量的人,不要让嗔恨偏激的心态进入自己的内心。

执行力强的一些表现和特征。

\paragraph{具有很高的自觉性,自动自发}

有的戒友不想看戒色文章,看一篇都不想,而有的戒友却渴望看戒色文章,极度渴望学习前辈的经验,完全自动自发,有着很高的自觉性,你觉得哪种人会戒色成功?不言自明!过去当学生党时老师也反复强调学习要有自觉性,要如饥似渴地吸收知识,现在想来,老师的强调是很对的,当你自觉去做一件事,往往能感受到其中的乐趣,也会热爱上它。当你的自觉性很强,你的执行力也会变得很强,自觉是积极主动的,不自觉就相当被动,懒得看戒色文章的人是难以成功的,心浮气躁的人也不会成功,学习应该要积极主动,自动自发,要渴望学习,当你迫不及待想学习时,你的执行力自然会变强,当你厌倦戒色文章,不想看戒色文章时,自己一定要注意调整,应该要养成良好的习惯,坚持学习不中断,这样才能保持比较好的戒色状态,觉悟也会不断提升。

\paragraph{注重细节,善于分析、总结和反省}

执行力强的人一定很注重细节,他之所以强和他不断优化与完善有关,人不是一生下来就有很强的执行力,而是后天习得的,通过学习、总结和反省来不断完善自己,提升自己。戒色方面认识是基础,态度是根本,行动是关键!最后还是要看执行力!首先通过学习戒色文章纠正自己的认识,对如何戒色有一个大体的认识,过去的思想误区都要纠正过来,这样戒起来就会顺利很多,其次要认识到态度是根本,有一个好的学习态度和一个不可动摇的戒色立场是非常重要的,一个人也许会破戒很多次,但只要立场和态度不变,还是有望东山再起而最终戒掉,就怕戒色立场出现动摇,一旦动摇就肯定会失败,而且动摇的人之后再想戒色,会很难找回良好的戒色状态,他会一直在怪圈中徘徊挣扎,不断地破戒。最近又看到有戒友发帖说自己道理都懂,但是还是破戒,这类戒友就是实战太差了,断念执行不力,还不是心魔的对手,光懂道理是不行的,因为实战是拼刺刀,不是讲道理,就像打仗时是要真刀真枪的,不是双方坐在那里讲道理,道理懂得再多,但是实战时一触即溃,那还是不行。有的戒友会把懂道理和戒色成功划等号,这其实是思想误区,懂道理最终还是为实战服务的,是为了更好地断念,光懂道理脱离实战,肯定会破戒,我的戒色文章非常重视断念的修炼,因为断念是戒色实战的核心!

\paragraph{对学习投入度高,悟性较好,注重积累,有恒心}

我了解过很多伟人,他们有一个共同的特点,那就是热爱学习,对学习的投入极高,毛主席常说:“我一生最大的爱好是读书。”在他去世前七小时,他仍在看书,书房的藏书达一万余种,近十万册。有一张照片,毛主席在床上看书,床上都堆满了书,可见他对学习的投入程度之高。新中国成立后,他总是挤出时间来读书,他的住所里,床上、办公桌上、休息间里,甚至卫生间里都放着书,一有空闲他就看书。每当沉浸在书中的时候,毛主席就忘了吃饭,工作人员催促他,他总是笑着说:“还有一点,看完再吃。”在游泳下水之前热身的几分钟里,有时他还要看几句名人的诗词。游泳上来后,顾不上休息,就又捧起了书本,连上厕所的几分钟时间,他也从不浪费掉。从行军打仗到新中国成立后去外地出差,毛主席都带着一大堆书。途中列车震荡颠簸,他全然不顾,阅读不辍。到了外地,同在北京一样,床上、办公桌上、茶几上、饭桌上都摆放着书,一有空闲就看起来。有一年夏天,他出差到武汉。在大“火炉”里,他每天晚上坚持看书,汗水不断地顺着脸颊往下淌。他风趣地对工作人员说:“读书学习也要付出一定的代价,流下汗水,才能学到知识!”毛主席晚年虽重病在身,仍不忘阅读。一次,他发烧到 39 度多,医生不准他看书。他难过地说:“我一辈子爱读书,现在你们不让我看书,叫我躺在这里,整天就是吃饭、睡觉,你们知道我是多么的难受啊!”工作人员不得已,只好把拿走的书又放在他身边,他这才高兴地笑了。1975 年,他的眼睛做手术后,视力有所恢复,又开始了大量阅读,有时竟然一天读上十几个小时,甚至躺在床上量血压时仍手不释卷。

曾国藩也特别爱读书,他参加会试不中,返家途中,到了南京的时候,路费花完了,于是他就去拜会一位做官的老乡,老乡很赏识他,就借给他 100 两银子,100 两银子在当时来说可是个大数目,他买了一套《二十三史》带回家,他的爷爷和父亲不但没有责备他,反而给他很大的鼓励。

\begin{quote}\it
    丙申年购《二十三史》,大人曰:“尔借钱买书,吾不惮极力为尔弥缝,尔能圈点一遍,则不负我矣。”嗣后每日圈点十页,间断不孝。(我在丙申年购置了一套《二十三史》,家父对我训诫道:“你为了买书而向别人借钱,我不惜一切地替你赔补还账,你若是能够仔仔细细地圈点阅读一遍,才算不辜负我的一番苦心啊。”从此以后,我每天都仔细读上十页,如果稍有间断,就是对父母不孝。)
\end{quote}

于是他就“侵晨起读,中夜而休,泛览百家,足不出庭户者几一年。”第二年,他赴京赶考,取得进士资格。曾国藩“日以读书为业”,读书只为进德修业,其孜孜不倦、刻苦读书的精神,为他仕途的畅通奠定了基础。曾国藩说,读书须立志有恒,“盖真能读书者,良亦贵乎强有力也”,他给自己规定,每日用楷书写日记,每日读史书十页,每日记茶余偶谈一则,这三件事未曾中断。在他晚年眼力很差的情况下,他读杜佑的《通典》,每天读两卷,薄的读三卷,其勤奋可见一斑。曾国藩读书有一套学问。譬如,读书要“一句不通,不看下句。今日不通,明日再读。今年不通明年再读。”又譬如,“读书不二,一书不点完,断不看他书。”

伟人有着良好的读书习惯,甚至有点嗜书如命的感觉,很善于利用零碎时间,不少戒友都抱怨自己忙,没时间学习戒色文章,其实这只是借口,再忙也可以挤出时间。有的戒友就比较会利用各种零碎时间来学习,比如散步时听戒色文章,一个五分钟的空闲拿出手机看戒色笔记,笔记有两份,一份电脑 txt 版,一份手记,五分钟就可以看几十条笔记,复习笔记是迎来顿悟的关键,很多人是在记戒色笔记,但是记完笔记从来不复习,这样留在脑子里的印象其实很浅薄,甚至记完笔记什么都忘记了,戒色笔记要不厌其烦地复习,关键要养成习惯,不能中断,这样才能战胜戒色厌倦期。如果中了中断魔,戒色状态就会一落千丈,坚持学习可以很好地保持良好的戒色状态,真正的戒色高手都热爱学习,热爱做笔记,他们能从学习中获得很大的乐趣,每一次醍醐灌顶的顿悟都是一次莫大的振奋。我们要全身心投入到戒色文章的学习当中去,有的人可以看黄一下午,那种劲头拿到学习戒色文章上来就好了,要有十足的冲劲打开戒色文章,要如饥似渴地吸收戒色知识,要坚决执行学习,不可中断。我之前学习戒色文章,投入相当高,有的时候一整天都在记戒色笔记,自己有戒色心得和感悟也马上写下来,往往在复习戒色笔记时,自己的灵感就开始涌现了,顿悟也接二连三地出现,觉悟很快就上去了。复习戒色笔记时还有一个很重要的技巧,那就是归纳类似的笔记,比如把断念的内容全部归纳在一起,当类似的内容归纳集中在一起了,就很容易产生突然看明白的感觉,当它们零碎地分散开,就不太容易产生那种感觉,我是很注重归纳笔记的,我还会在笔记中选笔记,优中选优,对给我感触很深的笔记再进行着重的复习。

像曾国藩都是有日课的,最有名的就是曾国藩的日课十二条,真的很有修为,终身坚持,不曾中断,可以说是坚持到死!曾国藩一生勤奋好学,以“勤”、“恒”两字激励自己,教育子侄。谓“百种弊病皆从懒生,懒则事事松弛”。制定计划和日课有助于执行力的培养,每天坚持完成日课,不中断,学习就怕中断,一中断,状态就找不到了,而每一次坚持就是在巩固和强化那个好状态。日课很重要,但不宜过多,过多自己会太累,到时容易起退心,之前有戒友就向我反馈过这个问题,他们的问题就是贪多,我刚开始也犯了这个毛病,所谓“贪多嚼不烂”,后来我精简了日课,现在可以保证每天都能完成。我现在的日课就是每天学习大德开示,还有念佛、念《阿弥陀经》,因为我修的是净土宗,每天都在坚持,应该坚持了有七年多了,我每天也会静坐片刻,保持恭敬,早起,这和曾国藩一样。我每天也在发感恩心、谦卑心、惭愧心等,每天还会抽空做做有氧运动和养生功法,还有来戒色吧答疑等,这些是我大致的日课,日课不在多,关键能保质保量地完成。一定要注重积累,前辈深知积累的重要性,量变产生质变,到时自然会有一连串的顿悟,觉悟和境界一下就上去了。很多戒友抱怨自己怎么老是失败,无法取得突破,看看他们的学习情况就知道了,缺少积累,三分钟热情,到了戒色厌倦期也不懂得调整,看戒色文章兴趣寥寥,没有冲劲,不做戒色笔记,即使做了也不复习。有的人戒了好几年都不行,试了很多方法都不行,心魔一来,还是马上破戒,他的实战表现没有根本的提升,还是跟着邪念跑,断念不力!学习提高觉悟,有了觉悟就可以更好地练习断念,断念变强了,再看戒色文章,就能更精细、更有针对性地进一步提高觉悟,两者是互相促进的。

\paragraph{有韧性,百折不挠,不会因为一时的失败而悲观失望}

韧性是指具备挫折忍耐力、压力承受力、自我控制和意志力等,一般指顽强持久的精神,坚韧不拔的意志,面对失败也能及时调整过来,而不会破罐破摔,懂得激励自己。戒色之后是可能会出现破戒的,关键要有韧性,不要被失败打倒,要善于总结和反省,从失败中汲取教训,这样下次才能戒得更好,破戒可以让你发现自己的不足之处,完善它你就会变得更强大。有韧性的人能在艰苦的、不利的情况下,克服外部和自身的困难,坚持完成任务,他们的执行力非常强,而且很稳定,即使受到外界干扰也能坚定地执行任务,在比较大的压力下也能不急不躁,心理素质相当好。具有韧性就能够经受住挫折的考验,决心固然重要,但唯有韧性才能让你不断向目标前进,百折不挠,像一颗子弹一样撞向目标,坚定无比。不管面对什么样的挑战,只要心理足够坚韧,就能轻松面对一切挑战,不断地靠近自己的梦想和目标。韧性表现为一种坚强的意志,一种乐观的心态,一种积极的态度,一种对目标的坚持,不管困难有多大,都能调整好心态轻松面对,戒色需要超强的韧性,钢铁般的意志力,戒色是一场硬仗,只有真正的硬汉才能戒到最后!戒到死!

\paragraph{强烈的戒色动机}

你对戒色成功的渴望程度有多少?强还是超强,还是强到爆炸?抑或是可有可无,甚至是没兴趣。据我的研究还有大量的反馈,真正戒色成功的人,他们的动机都是极其强烈的,他们在逆境中爆发出了强大的渴望蜕变的自我救赎的力量!没有强烈的戒色动机,就没有超强的执行力!真正渴望成功的人会疯狂学习戒色文章,就像抓住了最后一根救命稻草,就像几天没吃饭的人看到饭菜就拼命往嘴里送。我那时笔芯用完一根又一根,笔记记了一本又一本,不断复习笔记,总结笔记,在一次次顿悟后觉悟猛增,那种勇猛精进的势头就像打了兴奋剂一样,心中万马奔腾,各种灵感不断涌现,那种酣畅淋漓的畅快感唯有亲历者方能明白。在坚持学习中能感受到自己的觉悟在不断上升,这是很奇妙的事情,比如之前一直没搞懂的道理,现在突然就弄懂了,就像一座山被打穿一样,之前不明白前辈为何要那样说,后来经过自己反复学习,深入思考,以及结合实战的体会,突然某一天就顿悟了前辈的深意!悟道之乐绝非邪淫快感所能比拟,顿悟之时往往喜悦爆棚,那种发自内心的欢乐,那种手舞足蹈的情绪,就像发现一座金矿一样,比中大奖还高兴,你内心深深明白,这个领悟的价值是不可估量的!内心会逐渐生出一种战胜心魔的把握。前辈关键的话,每句背后都是一片汪洋大海,每句下面都是一座金矿,每句都是成百上千次实战的深刻体会与高度总结,所以一定要反复深入地研读,一遍遍去领会。

执行力的培养,具体来讲,可以从以下三个方面进行:

\begin{itemize}
    \item 培养自己的自觉习惯,不要拖延,立刻开始,积极主动,摒弃惰性
\end{itemize}

自觉的问题上面已经讲到了,要想戒色成功,就应该学会自觉,自觉学习,自觉复习,自觉练习断念,只要自觉起来,很多事情就会进入良性循环。关于拖延的问题,也是很多戒友经常反馈的,拖延是很多人都容易犯的毛病,我也犯过,有时只要有一个拖延的想法,半天乃至一天就过去了,到了晚上快睡觉时才发现自己错过了,当自己想做时已经晚了,来不及了,时间在不知不觉间就流逝了,然后就想着明天再做,明天还是在拖延,一点也不干脆果断。拖延其实就是执行力差的一个表现,执行力差的人往往时间观念差,有得过且过混日子的心态,缺少积极主动的行动力。我们一定要克服拖延,想学习就立马投入学习,要冲起来,能冲起来才能飞起来,就像飞机在跑道上达到一定速度,然后驾驶员一拉杆就飞起来了。只有当飞机速度增大到一定值,才可能产生足以支持飞机起飞的升力。执行力的提升需要我们改变心态,形成习惯,把消极被动的心态转变为积极主动的心态,面对学习或工作时要把执行变为自发自觉的行动。大家应该都知道麦迪 35 秒 13 分的壮举,那种积极主动的进攻,那种永不言败的心态,那种拼死一搏的勇气,那 35 秒的确堪称经典,是荡气回肠的逆袭和绝杀。我们戒色也要找到这种状态,火力全开,不顾一切地冲起来,在那种冲起来的状态下,你根本不知道什么叫拖延,什么叫惰性,因为你已经变得绝对的积极,绝对的主动,绝对的有执行力!而且是超强悍的执行力。

\begin{itemize}
    \item 建立固定模式,加强过程控制,标准执行,保质保量,彻底落实
\end{itemize}

不少戒友在戒色后不知道学什么,也不知道怎么学,还停留在靠毅力强戒的层次。戒色之后目标一定要明确,那就是战胜心魔!心魔也就是自己的邪念,也泛指一切负面念头。要战胜心魔必定要学习戒色文章,觉悟上去了才有望降伏心魔,学习最好能建立固定模式,何时学习,学习量多少,笔记多少,学习时间多少,最好能固定下来,每天坚决按标准执行,保质保量,真正彻底地落实。为了达到成功的目的,你应该要有固定的学习和训练模式,国家队都有固定的训练模式,早上练什么,下午练什么,训练量多少,训练强度如何,体能训练、力量训练,理疗恢复等,他们分得很细,严格按照制度和模式来执行,而且他们很注重全年训练周期的划分,其目的就是要采用一定的训练结构,科学系统地安排全年的训练任务。开始戒色,先学习理论知识,然后要练习断念,一定要注重练习,强者都是练出来的,有的戒友对练习断念重视不够,他只注重学习和做笔记,我建议每天背诵 500 遍以上的断意淫口诀,必须严格执行,拿出最大的热情来背诵,渐渐你就会发现自己的变化了,光说不练假把式。小学时的某个阶段,我们天天背诵乘法口诀,大家肯定不会忘记,背到滚瓜烂熟,都形成条件反射了,天天在那背,走路都在背,上课背,放学回家还在背,连做梦都在背。首先是极度熟悉口诀,然后就是自如地应用口诀,从背诵口诀到最后直接按照口诀的意思去做,其间有一个领悟和进步的过程,到了高阶是不用背了,只需觉察即可,到时就能真正做到“觉之即无”。刚开始需要大量背诵口诀,越多越好,有的人甚至在背诵过程中突然就有了顿悟,突然理解了断念理论的含义,当你有了更深的领悟后,就会发现之前自己其实懂得很肤浅。这十六个字的断念口诀极端重要,是修心最宝贵的口诀,参透这个口诀,熟练运用这个口诀,你就会越戒越强,这个口诀的每个字都是金子做的,每个字的分量都宛如千钧。

执行也需要加强过程控制,要不断跟进,有时完成一个任务的过程中会出现松懈的情况,刚开始执行还比较好,但后来就松懈了,这主要是过程中未严格控制造成的。行之有效的方法就是制定进度安排,明确到哪天需要完成什么任务,比如戒色文章一天看多少篇,从这个月到下个月能够看几篇,最好能预先计划下,然后在过程中自己要加强进度管控,在预定的时间内顺利完成学习量,甚至可以超出些,要随时检查计划实施的进度,久而久之,执行力就会得到很大的提升。如果没有时间限制,就可能会一直拖延下去,考试时都有一个时限,这样自己就有了紧迫感,知道答题不能拖延,必须要快,有了时限就能很好地激发自己的执行力。《戒为良药》1 - 60 季可以在一个月内看完,一天看两季,贵在坚持,多做笔记,只要肯认真学习,一定会取得很大的进步,就怕没恒心,一曝十寒,前辈的经验再好,但如果不肯学习,那也没办法,学习也要注重吸收率,加强复习和总结是关键。

刚开始戒色往往比较无知,脑子里还有很多思想误区,新人最需要的就是学习,学习,不断学习。前辈已经总结了成功的戒除经验,研读前辈的经验可以少走很多弯路,如果没有前辈的指导,要想真正戒除是很难的,真正成功的戒色前辈都善于学习、研究和总结。有的戒友戒了很久都不行,但有时看懂前辈的一句话也许就能悟进去,真正开窍后,戒起来就完全不同了,你只要领悟了核心,牢牢抓住了核心,那就无往不胜!之前有位戒友和我说,说他用了很多方法都不行,一会这个方法,一会那个方法,听说哪个好,就去学哪个,一开始觉得是好,自己也想宣传新方法,但后来心魔猛攻时,才发现自己还是那么菜鸟,根本无法做到念起即断!戒色方法和思路有很多,但万流归海,最终还是要在断念实战上检验,有的戒色文章还存在思想误区,会宣传看了文章后就不再起邪念了,一劳永逸,永远戒除了。其实不存在这种方法,戒色是持久战,邪念肯定会再次入侵,关键还是要看断念实战!如果你把握了断念实战,你就把握了戒色真正的核心。就像打仗的方法有很多,但殊途同归,都是为了消灭敌人,战胜敌人。

\paragraph{不找借口,坚决执行,迎难而上,拿出奋战到底的精神}

学习或者做事不找借口,哪怕是看似合理的借口,要体现一种迎难而上、开拓进取的精神。我聊过一些戒友,他们总是找各种借口不学习戒色文章,即使我叫他们学习,他们也不肯学习,说很多遍他们也不听,对于这类戒友我也没办法,也许要破戒很多次以后他们才能认识到学习的重要性,才能对戒色文章发生兴趣。伤精患者的心态往往比较浮躁,所以让他们静下心来学习是有一定难度的,但只要自己有学习的渴望和决心还是有望看进去的,随着坚持戒色养生,脑力也会逐步恢复,到时看戒色文章的效率就会提升很多。一位戒友反省说:“光是脑子里想,实际上执行力很差。就比如戒色,脑子里想着要做戒色笔记,要学戒色文章,要练习断念,但是却老是不开始行动,老想着:等会儿吧,明天吧,然后就一直拖着。”他这种心态不少戒友都有,习惯性拖延,不去执行,总找借口,有点逃避不想面对的感觉。要提升执行力需要狠下决心,必须马上执行,就像押赴刑场,立即执行枪决一样!把那个拖延、懒惰、找借口的自己枪毙掉,重生一个充满执行力、行动力的自己。

当你积极主动起来后,整个能量场就完全不同了,你可以试着改变下,你会发现自己进入了一个新的循环,整个能量变得积极进取了。想想当你找黄、看黄时,你的执行力有多么恐怖!简直达到了疯狂的程度!要把那种强大的执行力转移到学习戒色文章上来,不顾一切地学习戒色文章!万牛莫挽,冲劲十足!有时也可以把学习当做一种乐趣,而不是一种任务,可以换个角度来看戒色文章,这样自己更容易接受些,如果说任务,某些戒友可能会抗拒,如果说乐趣,他们就可能很感兴趣,实际上,学习的确能带来很大的乐趣,特别是当你顿悟时,那种乐趣和成就感非常强烈。之前有的戒友把戒色文章当小说看,这样他的心态就处于放松接受的状态,学习效率很高,很容易看进去,我们对戒色文章不要抗拒,而是要热爱,当你真正热爱了,自然就会很欢喜地去看,渐渐变得渴望学习,热爱学习,到时执行力自然就上去了。在戒色过程中肯定会碰到这样那样的各种问题以及一些困难等,问题和困难并不可怕,可怕的是逃避,不想面对困难。我们要勇于面对困难,要有“亮剑”的勇气,只要还有一口气在,我们就要奋勇拼杀。面对困难要迎难而上,困难就是为了被克服的,战胜了困难,你才能变得更强大。戒色需要一种强大的精神支撑,少了这种精神,就像人被抽走了脊梁骨,有了这种精神,你的执行力自然会很猛,你会变得势不可挡,你注定会成功!不可避免地成功!即使你不想成功,也不行!因为只要具备了这种精神,本身就是一种成功!这种精神就叫——奋战到底,永不言败!战斗到最后一个人,把最后一颗子弹留给自己!下死决心!拿出杀人的气势来杀念!杀光一切邪念!!!

\paragraph{总结}

戒色前辈赢在了执行,执行学习,执行练习,不找任何借口,坚决执行,不顾一切地执行,每天保质保量、不折不扣地完成,雷打不动,雷厉风行。必须知行合一,知而不行,终非真知,真知必将行猛!学习也要有一定计划,比如每天 30 条戒色笔记,坚持 10 天就是 300 条,坚持 30 天就是 900 条,每次记笔记前,再对之前的笔记进行复习、吸收、巩固,这样去执行,每天都要坚持,就像下了死命令那样坚决执行!绝不拖延,要迫不及待马上开始学习,带着很高的热情与冲劲去做,打开戒色文章要比打开黄片还要兴奋,还要迫不及待!执行力的重点是确保目标的明确性,执行计划的可行性,目标一旦明确,做起事情来就很有针对性,不会盲目行动,就可以调动全部的力量来导向那个目标,执行的计划也不能过高,过高而达不到容易丧失信心与动力。一旦目标确定,计划可行,就要不惜一切代价坚决完成预定目标,在执行的过程中也要保持专注,不要总是分心干其他事情,应该给自己设定时限,这样可以最大程度激发自己的执行动力。

从某种角度而言,执行力就是战斗力,真正的执行力是一种能够快速行动,遇到挫折继续勇往直前,直到达到一个高价值结果的能力,是一种很积极进取的能量状态,不是那种找借口拖延的消极能量状态。我们需要最为直接的力量——执行力!马上执行!马上行动起来!从椅子上跳起来!受够了消极不作为!必须立即执行,刻不容缓!带着全部的冲劲与热情开始学习和练习,不要三分钟热度,要猛火煮和慢火炖交替进行,不要中断!有句话是这样说的,“三分靠战略,七分靠执行”,再好的战略,如果不去执行,也只是一纸空文,毫无意义可言。我们一定要强化自己的执行力,执行必然会产生结果,可以根据结果来不断优化和完善,如果找借口拖延只会让自己越来越被动,对自己越来越不利,一定要转被动为主动,自己积极主动去执行,一定要自觉,没人会逼你学习戒色文章,也没人会逼你练习断念,别人最多只是建议,一切要靠自己的自觉,要高度自觉地去执行,成功者都是自觉的人,没有一个成功者是找借口拖延的,成功者都是强有力的执行者。现在中国很多地方都造了很多高层住宅,刚开始造房子会有一个规划,具体时限、何时竣工和进度安排等,一切都是有条不紊地执行着,那些民工每天在那干活,这其实就是一种长期的执行。有些戒友会有抗拒学习的心态,里面也有一种恐惧失败的心理,自己不愿面对。真正的强者不惧怕失败,他们很懂得从失败中学习,他们敢于执行,他们热爱学习,把那种抗拒学习的心态转变过来,能量就会变得积极主动起来,到时就会越执行越强,就像打一个游戏,越打水平越高。我们要做戒色的首席执行官,戒出人生的大境界、大格局、大场面!

下面分享一首戒色诗歌。

\begin{poem}[邪淫半兽人]
    \begin{multicols}{3}
        \begin{center}~\\
            最后一次射出后 \\ 他倒在床上 \\ 一切都变得没有意思了 \\ 心魔离去了 \\ 留下一具被掏空的躯壳 \\ 空虚和悔恨的感觉 \\ 瞬间将他填满 \\ 心魔又一次得逞了 \\ 刚才的疯狂历历在目 \\ 疯狂找黄、疯狂看黄 \\ 两只禽兽般的眼睛 \\ 死死盯住屏幕 \\ 对着屏幕疯狂套弄 \\ 着了魔一般 \\ 这种状态让他感到害怕 \\ 极端的失控 \\ 不掏空不爽 \\ 一定要射光才肯罢休 \\ 他就像被植入了木马病毒 \\ 向着自毁的方向一路狂撸 \\ 破戒后的第二天 \\ 症状彻底爆发了 \\ 他感受到了剧烈的痛苦 \\ 他知道要还了 \\ 邪淫肯定是要还的 \\ 逃不掉的 \\ 看着镜子里那个 \\ 猥琐、憔悴、颓废的自己 \\ 他怒吼一声,砸碎了镜子 \\ 他说:“我彻底受够了!” \\ 他像一只烦躁的困兽 \\ 急需挣脱邪淫的牢笼 \\ 这不是他想要的生活 \\ 但愿这一切只是一场噩梦 \\ 醒来还是那个 \\ 纯真无邪的男孩
        \end{center}
    \end{multicols}
\end{poem}

下面推荐一本书。

\begin{book}[《你值得过更好的生活》,Robert Scheinfeld(罗伯特·沙因费尔德)]
    整个宇宙都是你创造的游乐场,游戏的名字叫做“找出你是谁”。真正的你是全知全能的,你有无限的丰盛、无限的创意和无限的力量。沙因费尔德在一百九十多个国家,帮助过许多人以更少的时间和努力享受更多乐趣的同时,创造出惊人的成果。他乐于帮助他人从自我限制中彻底解脱出来,并活出最充满力量的生活。一旦你到达那个点,你生命中的其他领域也将发生翻天覆地的变化,一切将超乎你的想象。这本书把生活和对真理的表述很好地融合在了一起,而且是从科学的角度来论述的,不少观点给人很大的启发,的确是一本很好的书籍,里面讲到:“意识创造你所体验的一切,即使最细微事项也是。”还讲到:“从那些限制模式中取回你的能力,并且瓦解限制模式。”限制模式也就是作茧自缚,如果你无法做念头的主人,就会导致作茧自缚,如果你无法断除邪念,邪念就会附体,然后奴役你,让你身不由己,这会导致很大的限制,瓦解限制模式就是要学会修心,学会观心断念。善知识只是给人解粘去缚,让人明白问题究竟出在哪,这本书目前有 1、2,网上有电子版,有兴趣的戒友可以下载看看。
\end{book}
