\subsection{耳鸣问题、戒色心态、搜集资料的重要性}

\subsubsection{耳鸣问题}

耳鸣是戒友中比较常见的问题,当然导致耳鸣有很多原因,一般戒友出现的则是肾虚型耳鸣,就是因为纵欲过度,另外加上生活习惯不正常,经常熬夜久坐,这样就特别容易出现耳鸣问题。

中医:肾开窍于耳。肾一虚,耳鸣问题就会找上门来。很多戒友被耳鸣问题所深深困扰,久治不愈,把希望寄托在药物身上,殊不知戒色才是根本前提,如果你一边吃药还一边 SY 和 YY,这样耳鸣还能看好吗?我曾经也被耳鸣问题所困扰,当然那时我处在比较无知的状态,后来我戒色后学会养生,积极锻炼半年,养足肾气,耳鸣就不药而愈了,没吃任何药物!最大的补药就是不泄为补。另外,遗精也算泄,对身体也是有伤害的,如果你肾气充足,偶尔一次遗精并不会感觉到什么,但是,如果你肾气亏损了,再遗精,就是雪上加霜。我回答的问题中,有不少戒友都被遗精问题所困扰,频繁遗精,身体越来越不行,中医有讲到:久遗八脉皆伤。所以耳鸣问题要恢复,也必须在减少遗精上下功夫,我第三季推荐的八段锦固肾功,建议大家坚持做,不少戒友都取得了良好的效果,大大减少了遗精次数,当然,前提是必须找对拉紧感,并且杜绝其他导致遗精的因素。

肾虚型耳鸣,一般患者听到的是蝉鸣、汽笛或者嗡嗡、嘶嘶的声音。我听到的是蝉鸣,好像耳朵里面养了一只知了,很尖细而悠长的那种,很烦人,有时会产生睡眠障碍。这种困扰其实在我高中时就出现了,只是当时并不严重,后来我有熬夜和久坐,耳鸣问题就变得严重起来了。一般肾虚型耳鸣的出现都会伴有其他肾虚症状,并不是单一的耳鸣问题,出现的症状一般分为以下六个方面:

\begin{multicols}{2}
    \begin{description}
        \item[脑力方面] 记忆力下降记忆力减退,注意力不集中,精力不足,工作效率降低。
        \item[情志方面] 情绪不佳情绪常难以自控,头晕,易怒,烦躁,焦虑,抑郁等。
        \item[意志方面] 缺乏自信信心不足,工作没热情,生活没激情,没有目标和方向。
        \item[性功能方面] 性欲降低,阳痿或举而不坚,遗精、滑精、早泄,显微镜检查可见精减少或精活动力减低,不育。
        \item[泌尿方面] 尿频,尿等待,小便清长,滴白,前列腺炎,精索静脉曲张。
        \item[其他方面] 早衰健忘失眠,食欲不振,骨骼与关节疼痛,腰膝酸软,不耐疲劳,乏力,视力减退。脱发白发头发脱落或须发早白,牙齿松动易落等。
    \end{description}
\end{multicols}

SY 导致肾虚,肾一虚,就会出现很多身心问题,因为 SY 摧残的是身心,耳鸣问题就是众多肾虚症状的一个表现而已,所以要彻底治愈耳鸣,必须要彻底戒掉 SY 和 YY,积极锻炼,按时饮食作息,养足肾气,肾气足,万邪熄。否则,吃再多药也无济于事,到后来可能药都吃疲了,耳鸣都还没好。很多戒友反映戒掉两个月、三个月,身体还没多大改善,我要说的是,光戒是不行的,必须学会养生之道,积极锻炼,这样肾气恢复才快,而且很多戒友伤精程度严重,恢复也相对较慢,我自己的感觉是,恢复速度就像头发生长的速度,真的很慢,但是如果你半年不剃头,你会发现头发很长了,两个月三个月并不会感觉有多长,所以肾虚要恢复,要做好持久战的心理准备,好好坚持每一天。另外,也要在尽量减少遗精次数上下功夫,因为频繁遗精对于恢复是很不利的。

\subsubsection{戒色心态}

再来谈下戒色心态的调整。

因为 SY 很多戒友的心理都出现了问题,出现悲观厌世,自暴自弃的人非常多,我曾经也有这种情绪上的困扰,看不到希望,做什么都没动力,缺乏自信,人很自卑乃至会自残。

出现这类心理问题也很正常,因为 SY 摧残的是身心,身体出问题,心理也会出问题。身体出问题,调理身体,心理出问题,也要学会及时调整。

大家一定要知道,这个世界上永远有两个我,一个消极的我,一个积极的我。当出现消极的我时,一定要尽快调整到积极的我。就像钟表走错了,赶紧调整到正确的时间。

这种调整心理状态的能力,在西方心理学可以归为 EQ(情商),也就是情绪智商,是近年来心理学家们提出的与智商(IQ)相对应的概念。它主要是指人在情绪、情感、意志、耐受挫折等方面的品质。总的来讲,人与人之间的情商并无明显的先天差别,更多与后天的培养息息相关。也就是说,情商是后天“习得”的,通过学习情商是可以提高的。

前段时间土豆吧主推荐的书《秘密》,就是一本很好的提高 EQ 的书,里面讲到的吸引力法则,就是让人多想好的方面,多想积极的方面,这样就更容易成功,如果一直想消极灰暗的方面,其实就是在对自己进行催眠,进行暗示,暗示自己会失败,这样失败的几率就更大,就像打仗时士气不行,这样战败的可能性就会很大。

情绪智商包含五个主要方面:

\begin{itemize}
    \item 认识自身的情绪,只有认识自己,才能成为自己生活的主宰。
    \item 能妥善管理自己的情绪,即能调控自己。
    \item 自我激励,它能够使人走出生命中的低潮,重新出发。
    \item 认知他人的情绪,这是与他人正常交往,实现顺利沟通的基础。
    \item 人际关系的管理,即领导和管理能力。
\end{itemize}

这些能力通过不断学习是可以获得的,建议多看这方面的书籍,书籍是人类进步的阶梯,不学习无法开悟,不学习戒色也不会成功。

\subsubsection{搜集资料的重要性}

最后来谈一下搜集第一手资料的重要性。

第一手资料就是 SY 戒友的经历和症状表现,这些资料这些案例非常非常珍贵,很多案例都很典型,大家最好能建立个文件夹,把看到的典型案例搜集起来,这样有助于自己对 SY 的危害更深入更透彻地认识。很多戒友会搜集 H 片,当然我也这样干过,搜集过几十部,多搜集一部,自己就多了一次放纵的机会,而搜集 SY 危害案例,多搜集一个案例,就是对自己多一次警告,警告自己不能放纵,放纵就和他一样,放纵就是在害自己。我每天上戒色吧看到好的案例就会搜集起来,阅戒友无数,什么症状都见过,很多症状自己都曾经有过,看得越多越全面,认识就越深刻。

不管做什么研究,要深入理解和认识,第一手资料第一手案例就显得无比重要,SY 有害和无害其实根本不需要争论,有句话叫:实践出真知。有没有害,实际真相如何,多看看身体垮掉的戒友是怎么说的,从这些受害者口中出来的就是最真的真相。适度无害论根本就是扯淡,我阅戒友无数,没见过几个能真正做到适度的,看到最多的就是“一发不可收拾”,好比打开了潘多拉的魔盒,打开就收不住,热爱运动和生活习惯良好的人出来的症状轻微,不爱运动,熬夜久坐的人出来的症状就严重。可以说,只要开始放纵,就没有人可以逃得掉,万法皆空,因果不空,SY 是在种恶因,恶因导致恶果,什么种子结什么果,SY 出来的就是恶果。

不管身体如何好,随着年纪的上升,症状会越来越明显,身体垮掉比较早的戒友,其实也是福分,为什么这么说呢,因为症状才是最好的老师,出症状了,就知道要戒色了,就认识到了真相。否则很多人在五十岁时身体才垮,那时要恢复就更难了,现在二十多岁垮掉,可以提早认识到危害的严重性,年纪轻也容易恢复些,否则五十多岁再垮掉,就难以恢复了。我聊到过一个五十岁的戒友,他作息饮食相当规律,不熬夜不久坐,每天晨跑锻炼,几十年如一日,身体非常强壮非常好,但只有一点他没做好,也没认识到危害,他说他几乎每天都过两次性生活,早上一次,晚上一次,持续了几十年,他说,到四十岁以后,他明显感觉身体大不如前,但并未引起重视,继续纵欲,到五十岁时突然身体出了很多症状,一查是植物神经紊乱和焦虑症,每天活在症状的地狱里,相当苦恼,这个案例就说明了,只要纵欲,身体迟早会出问题的,即使你作息饮食规律,即使你热爱运动,还是会出症状,随着年纪的上升,症状会越来越多。所以,必须戒掉 SY 恶习,即使婚后也要懂得保精,节制性生活,否则身体还是会出问题的。

前车之鉴,后事之师,这些案例就是最好的警示,最好的真相,最好的老师,希望大家能做个有心人,把搜集 H 片的邪恶兴趣转变成搜集 SY 案例。每一个案例都是在提醒你,每一个案例都是在警告你,警告你:千万不要 SY,千万不要纵欲,否则你也会和他一样。

另外,75 党一直在干撒播无害论的事情,因为青少年缺乏辨别力,所以一看到无害论就糊涂,分不清真假,所以,青少年更要多看案例,从前辈的经验教训中多学习,一定要把无害论从自己大脑中清除掉,那是有毒的思想,必须清除掉。
