\subsection{冬季戒色养生之五大要点}

\paragraph*{前言}

我之前说失败不可怕,一方面是转消极为积极,另外就是可以减轻他们的心理负担,很多人破戒后很沮丧,甚至破罐破摔,我这样说可以让他们以积极的心态来应对破戒,把关注的重点放在破戒后的总结和反省方面,而不是一味地自暴自弃。以什么态度应对破戒至关重要!消极的态度会导致消极的结果,而积极的态度就完全不同了,破戒后一定要好好总结经验教训,认真分析破戒的原因,并且加强学习提高自己的觉悟水平,这样下次才能戒得更好。就像一次考试,当考卷发下来,发现自己考了个不及格,这时候不应该自暴自弃,而是要认真分析考卷,真正明白自己错在哪,这样下次遇到类似的题型就不会再错了。

老师总是会对学生说,失败不可怕,一方面是安慰,另外就是可以起到鼓励的作用,让学生以积极的心态来应对失败,要学会从失败中汲取经验教训。乔布斯曾说:“你必须学会面对失败,如果你害怕失败,那就不会取得成功。”如果你认为失败仅仅是一种打击,那你很可能会被失败打倒;如果你觉得失败也是一种收获,那么你将从失败中学到很多,所谓吃一堑长一智。很多人正是从破戒的教训中强势崛起的,你们可以看看很多戒色两年以上的人,他们之前也曾失败过很多次,但是他们并未气馁,而是通过不断反省和总结,让自己的觉悟不断提升,这样才最终降伏了自己的心魔。

下面分享一些案例。

\begin{case}
    飞翔大哥,我很困惑,我已经手淫四年了,今年十九岁,最近才知道它的危害,已经戒一个月了,我困惑为啥前段时间手淫失去快感了,我害怕以后恢复不了,这是怎么回事啊?以后可以恢复快感吗?

    \textbf{附评} 手淫伤肾到一定程度,就可能出现失去快感的表现,之前也多有反馈。一般手淫几年后,就会出现早泄的倾向,越撸时间越短,本来可以十分钟的,后来变成了秒射,同时射距也会变短,甚至是流淌下来的,出现了射精无力的表现,另外就是快感消失,虽然有那种欲望,但是却没有一点快感,快感就像人间蒸发一般从他身上消失了。一旦出现快感消失,很多人都会变得异常恐慌,心里也非常担心,害怕自己恢复不了,也担心会影响到婚后的性生活。这位戒友的症状表现,通过坚持戒色养生,积极锻炼,会慢慢恢复正常的。年少无知多瞎撸,被无害论洗脑后更是不要命地撸,这样症状迟早会爆发出来的。有的少年迷信教科书,其实教科书也是人编写的,是人就可能出错!我国的生理教科书也一直在修改,相信将来无害论会得到彻底的纠正!现在很多教科书也不是说手淫无害,而是说不能过度,但编写教科书的人并未认识到手淫的高度成瘾性,而且手淫属于邪淫,是古圣先贤一致反对的邪淫!邪淫伤身败德,真的一次都不能有!
\end{case}

\begin{case}
    戒色真有好处,亲身经历,我有中度焦虑症抑郁症,第一次吃药治好了,后来发作后吃药两年只能维持在 80\% 左右,戒 SY 两个月,恢复 90\%,如果继续撸,可能跌至 60\%。

    \textbf{附评} 这是位神经症戒友,他恢复得不错,戒了两个月,恢复已经达到了百分之九十。他的经历也表明,如果不戒除 SY 恶习,神经症是难好的,也容易再度发作,很多人甚至几年十几年都与药为伍,彻底变成了药罐子。不仅钱花光了,身体的病也没有好起来,人生彻底变得灰暗无望。我认识的不少病友,他们吃药吃到最后,已经出现了耐药,吃一种不行了,再去看医生,医生就给他换另外一种,就像小白鼠一样尝试着各种药物,但每种药物只能管一阵,不久耐药了效果就不好了,而且很多西药的副作用很大,也不宜长期服用。神经症痊愈的基础就是戒色养生,如果不懂得戒色养生,神经症是极有可能复发的,很多病友就是在 SY 后复发的,SY 伤肝肾伤五脏实在太厉害了,很多人撸完濒死感就来了……
\end{case}

\begin{case}
    戒色 142 天,说些发际线的问题,我是戒色前,发现自己发际线往后退,成 M 形状,我当时很害怕,因为家里的父辈、伯伯叔叔都有秃,但我爸爸发质很好,而且还很少有白发!但遗传基因应该也有的!我后来就看到一个戒友说喝黑芝麻糊,喝得头发浓密了!我就死马当活马医,我也喝,每天早上起来喝一袋,喝了一个多月,感觉现在有效果了,感觉发际退后的地方开始生发了!期间我也保持戒撸,每天铁板桥,毕竟腰强则肾强!

    \textbf{附评} 这位戒友的脱发也开始恢复了,脱发的问题我以前的文章曾经讲到过的。光戒是不行的,恢复很慢,自己一定要注意养生恢复,戒色是系统工程,养生恢复也是系统工程。脱发的问题,是该加强食疗,黑芝麻是可以吃起来,但也不能吃太多,否则也可能导致晚上出现遗精,在食用量和食用时间方面要好好把握,这位戒友放在早上喝,这样很不错,晚上喝的话,弄不好会出现遗精。中医:发为肾之华!手淫伤肾到一定程度,头发肯定会出现坏的变化,比较常见的就是白发、脱发、发质下降(枯黄分叉变卷等),一般坚持戒色养生,注重食疗,这样坚持下去,头发就会大大改观的。头发真不能小看,一旦出现白发和脱发,对于自信的摧残是相当严重的,很多人在脱发后都陷入了极度的恐慌之中,生活一下失去了光彩。
\end{case}

\begin{case}
    今天照镜子我突然发现变得有精神很多了,感觉像换了个人似的,皮肤变得光滑细腻了,痘痘也基本没有了,毛孔也小了很多,感觉基本正常了,不过痘痕还是有点的,我说的话一点不假,当一个人的精气神回来了,你的神采和相貌都会发生巨大的变化,周围的人都说我变帅了!

    \textbf{附评} 这是戒友犹太人的恢复反馈,他说得非常好,精气神恢复后,那真是一个天一个地。不少人应该都喝过花茶,晒干的花朵在未泡开之前是干瘪萎缩的,失去神采的感觉,但是在水中一泡开,马上就不一样了,每个花瓣都舒展开来了,就像重新绽放了一般,一下就变得鲜活了。我们坚持戒色,当肾精慢慢养足,身心就像得到水分滋润的花朵,又开始重新绽放,那种优质的能量状态又重新回来了,完全和以前不一样了,可以说是焕然新生,说是巨大的变化,一点不为过。与清新纯净的灵魂相对应的就是变帅的容颜,与龌龊肮脏的灵魂相对应的就是猥琐丑陋的容貌,只要你坚持戒色,奇迹般的变化就会悄然发生,自信也会重新回到你的身上。\textit{精能生气,气能生神,则精气又生神之本也,保精以储气,储气以养神,此长生之要耳。(明代陈继儒《养生肤语》)} 精气神就是男人的第一件衣服,有精气神的男人浑身都散发着无限的活力。
\end{case}

\begin{case}
    谁来帮帮我!十六岁就有这么多毛病!手淫至少七年了吧,从 2012 年开始到现在耳鸣不断,现在晚上多梦睡不踏实,白天困得要死,感觉整个人都像在梦中似的!尤其是在晚上脑子空洞洞的!平时胡思乱想,一点小事都要想个没完没了!怎么办?我是今年才发现这个贴吧的!有人能救救我吗?明年就中考了,就我这状态……唉!

    \textbf{附评} 很多戒友在未发育前就开始撸了,只是还不能射出,但已经隐隐约约有那方面的感觉了。这位戒友只有十六岁,但是他已经开始耳鸣了!实在非常可怕,从 2012 年开始就已经有了伤精的症状表现了。中医:肾开窍于耳!伤精到一定程度,肯定会出现耳鸣的,严重的会出现听力下降,更严重的甚至会出现耳聋!\textit{精脱者,耳聋。(《黄帝内经》)} 当然要达到一定的伤精程度,才会出现耳聋的症状表现。\textit{耳属少阴肾经,肾之窍也。肾气实,则耳聪;肾气虚,则耳聋。(《医林绳墨》)} 人家都是二十岁、三十岁才开始耳鸣的,而现在这个网络时代,黄毒空前猛烈,肉弹异常凶猛,很多少年十几岁就症状缠身了!十几岁的花样年华,却被邪淫文化疯狂摧残,还没绽放就提前凋零了,真是可悲可叹!这位戒友不仅耳鸣不断,还出现了神衰的症状表现,还有强迫症的倾向,而他也只有十六岁!神经症缠身的戒友,多有喊救命,最后能救自己的,还是自己!一定要多学习提高觉悟,觉悟到了,自然可以降伏心魔。别人是可以帮你,但最后的主动权还掌握在自己手里,所谓天助自助者。神经症严重的话,是要积极治疗的,然后坚持戒色养生,身心会慢慢恢复正常的,恢复是有一个过程的,病去如抽丝,一定要保持耐心。
\end{case}

\begin{case}
    手淫七年,现在人鬼同途!走路打飘、全身无力、意志薄弱、声音低、腰酸腿软、容易想多、胆小怕事、外貌畸形塌陷、双眼无神游离,这些都是我的问题。

    \textbf{附评} 这位戒友的症状表现都很典型,SY 对身、心、容貌三方面的摧残是非常严重的。大家泄精后应该都有这样一个体会,那就是双腿发软、脚底无根、走路发飘,在这样的情况下参加剧烈运动,很容易出现骨折,很多人在球场上并未发生身体接触,就是自己的脚突然一扭,又或者突然摔倒,然后就骨折了,有的人还动手术打上了钢钉,吃了大痛苦。中医讲到肾主骨,肾虚的人有一个表现,那就是腰膝酸软,泄精后人就软掉了,硬的代价就是叫你软掉!JJ 一硬,再一泄,身体就软掉了,很多练武或者练散打的人都深有体会,那就是泄精后踢沙包,原来脚不疼的,现在一踢就疼,已经影响到了正常训练。我以前看过的一本少林练武秘籍,里面就明确提到要戒色保精,戒色后骨密度会增加,抗击打能力也会相应增强,而滥撸滥泄则会导致骨质疏松,让一个硬汉彻底变成软蛋!无欲则刚,多欲则软!中医也讲到肾主恐,伤肾到一定程度,就会莫名其妙地变得胆小怕事,而当养足肾气后,又会重新变得气壮山河气势如虹。外貌畸形塌陷的问题,在 \ref{79} 讲得比较详细,大家可以看看。一个邪淫的人他的双眼往往缺少定光,总是无神游离,带着一股邪气,很多人都出现了对视障碍,自信也遭到了极其严重的打击。
\end{case}

\begin{case}
    得了糖尿病,悔不当初,我才十六岁呀!因为熬夜 SY,每天晚上一直吃东西,才十六岁就得了糖尿病!现在戒色还有救么?如果现在戒了糖尿病会减轻或消失吗?

    \textbf{附评} 彭鑫博士就专门讲过纵欲会导致糖尿病,\textit{消渴者,原糖尿病知识其发动,此则肾虚所致,每发即小便至甜。(《外台秘要消渴门》)} 肾虚真的会百病丛生,当然饮食不节也是糖尿病的诱因之一,这位十六岁的戒友,熬夜 + SY + 饮食不节,这三样全部占全了!后果可想而知。\textit{五脏皆柔弱者,善病消瘅。(《灵枢五变篇》)} 而肾为五脏之根,手淫就是在砍伐五脏之根,加之其他致病因素的共同作用,这样是可能得上糖尿病的。熬夜对身体的伤害也非常严重,熬夜 + SY,这不是自废的节奏,这简直就是自杀的节奏,这是插在肾脏上的两把尖刀!新闻经常报道熬夜玩网游的人猝死在网吧,仅仅一个熬夜就可以让人猝死,如果再加上 SY 恶习,那伤害绝对是加倍的,年轻的身体是有本钱,但也经不起这样疯狂的摧残啊!何况很多人的先天体质本就薄弱,根本就伤不起,如果还这样疯狂纵欲,不出几年就会症状缠身,半条命都撸掉了!\textit{孔子曰:“……少之时,血气未定,戒之在色……”(《论语·季氏》)} 色是少年第一关,这关打不破,这辈子就可能彻底废了,将来的下场真的很惨!没得病时异常疯狂,所谓毁灭前必疯狂,得病了,就如晴天霹雳五雷轰顶,面对检查报告呆若木鸡,真是悔不当初啊!\textit{天下有极惨极烈,至大至深之祸,动辄丧身殒命,而人多乐于从事,以身殉之,虽死不悔者,其唯女色乎!(印光大师)} 无知的少年,被无害论洗脑的少年,也许等到症状突然降临,你才会幡然醒悟,希望那一天不会太晚。
\end{case}

\begin{case}
    我属于比较严重的神经症,还伴有躯体震颤反应和严重的社恐,以前觉得快完了,后来有缘接触了戒色吧,从九月初戒到现在的这段时间里,真的恢复了好多。我知道恢复是很漫长的,但我充满希望。

    \textbf{附评} 看到神经症戒友的恢复,我总是感到很欣慰,因为神经症的确很折磨人。记得以前我也有躯体震颤反应,晚上睡觉躺在床上,我还以为地震了,后来才知道是躯体震颤,根本不受自己的控制。神经症的日子是充满灰暗的,也是不堪回首的,很多人好几年都走不出神经症的阴霾,花钱无数,检查无数,依然无法痊愈,很多人都是带病生存的。神经症的康复时间一般在一年以上,坚持戒色养生半年以上会感觉好了很多,症状严重的话,是该配合积极治疗的,但三分治疗,七分戒色养生,痊愈的重头戏其实掌握在自己手中,一定要懂得戒色养生,否则神经症是很难真正痊愈的,总是暂时好点了,然而在纵欲之后,神经症又会变得不稳定,很容易出现复发。我曾经也深陷于神经症,但现在已经彻底痊愈了,戒色养生的几年间,再也没有复发过,如果不懂得戒色养生,神经症是很难好利索的。
\end{case}

\begin{case}
    我感觉我戒不掉的原因是我不是太相信 SY 的那些危害,可我本人去医院检查,慢性前列腺炎、弱精证、脾肾阳虚,我该怎么办?我是一个下根之人,求飞翔哥救救我!救救我!我 SY 快九年了。

    \textbf{附评} 有句话叫不见棺材不掉泪,但下根之人是见了棺材还执迷不悟,真有一种无可救药的感觉。但也不能全怪这类人,毕竟无害论的洗脑实在太厉害了,那些所谓的砖家打着科学的幌子,以权威的名义来向你灌输无害论,这时候很多人就变得是非不分了。请你扪心自问,你是相信一种错误的理论,还是相信触目惊心的事实?你是相信荒谬的无害论,还是相信我国传统的中医科学?你是相信道貌岸然的性学家,还是相信古圣先贤的高度智慧?我想真正有点善根的人,肯定会认清事实真相,古圣先贤远胜于所谓的性学家,性学家在古圣先贤面前连小学生都不如,性学家靠性吃饭,难道他们会自扇耳光吗?他们肯定会到处吹嘘性的好处,所谓王婆卖瓜,自卖自夸。无害论就是一个弥天大局,谁来入局,我来告诉你,那就是成万上亿的无知撸民!这位戒友已经症状缠身了,但他还是执迷不悟,真是下根之人,好在他还有自知之明,还是有救的。我们一定要多学习中医医理,多看受害者的案例,多学习戒色文章,这样就会慢慢认清手淫的真相,然后一定要下大决心来戒色,这样才有望戒除手淫恶习。
\end{case}

\begin{case}
    轻轻松松戒了 75 天结果破了,以前对着撸得很凶的某位女星的擦边新闻出现在了 QQ 空间右上角那一栏,当时就欲火燃起,意淫出现,断,再出现,再断,心魔不断怂恿我去打开看,十个小时之后,终于还是妥协了。好奇心 + 擦边新闻要人命啊!就像走火入魔一样控制不住。念头狂轰滥炸地催我去看,去点击,真的好难打。

    \textbf{附评} 这位戒友描述了他和心魔斗争的十个小时,能斗争十个小时也属不易,很多人几秒钟就败下阵来了。网上的擦边新闻是非常多的,一不小心就会中招,所以必须要提高警惕,千万不要去看第二眼,往往第二眼就着魔,第二眼就陷入。不要试图去看清,第一眼看见时马上就要避开,如果你想看看清楚,那十有八九会陷进去,擦边新闻擦边图就像黑洞一样,会把你吸进去。在对境实战时,一方面要管住自己的视线,不要让视线停留在擦边图上,另外就是要牢牢看住自己的念头,邪念一旦出现,如果不及时断掉,它就会由弱转强,直到最后完全控制你的身体,让你身不由己,当断不断,反受其害!一定要在小火星时灭掉它,否则等到欲火中烧,那就很难断了,到时候就会变得很挣扎,两股力量在内心中拔河一般,谁赢谁就获得身体的控制权!就这么直接,就这么残酷!要战胜心魔,断念的强化训练是必不可少的,一定要第一时间发现、第一时间断掉,这两个第一,必须要严格做到!做不到的话,最终被虐的就是你!禅宗把那些能做到对境不动心的人称为“大力人”。真正的大力士不是搬动巨石的人,真正的大力士是对境不动心的人。不动心就是一种力量,不动心就是一种至高的境界!不管什么诱惑当前,就是不动心,诱惑越猛,就越不动心,反向增强!魔多反使道心坚!
\end{case}

下面步入正文。

现在已经进入了三九天,中国大部分地区已经非常冷了,冬季戒色养生也有不少需要注意的地方。进入冬季后,气温明显下降,很多戒友的前列腺炎也开始反复了,频遗的问题也开始出现了,腰酸腰痛的表现也增多了。\textit{血气者,喜温而恶寒,寒则泣而不流,温则消而去之。(《灵枢调经论》)} 经脉喜温而恶寒,血气在经脉中,寒则泣涩,温则通利。天气一冷,气温一降,很多疾病都可能出现反复或者加重,所以冬季的养生非常之关键。春养肝,夏养心,秋养肺,冬养肾,每个季节都有养生的重点,冬季特别要注意养肾防寒。所谓“阴阳四时者,万物之始终也,死生之本也。逆之则灾害生,从之则疴疾不起,是谓得道”。

中医认为冬季是藏精的时节,精要藏得住,藏得深,就像藏一个宝贝一样,如果肾精能够单独分离出来,那最好藏在有五米钢板保护的银行金库内,谁也别想盗走!肾精是人体最宝贵的能量资源,就像人体的核能一样,就像液体钻石般宝贵,这是种子的强大能量,可以滋养五脏、荣华六腑、光润容颜。到了冬季应该好好养精蓄锐,休养生息,为来年储备雄厚的能量,这个冬季养得好,储备充分,来年春天就有打虎之势!就像学生备战高考,就像运动员备战奥运,现在的知识储备和训练储备,就是为了到时一鸣惊人一飞冲天。

每一天的戒色养生,就像在储蓄罐内丢进一个“精币”,每一次泄精,都相当于倒出一些“精币”,冬季是一个储精大季,一定要把肾精的耗损减至最低。在冬季尽量不要破戒,并且要严格控遗,注重养生之道,这样才能在来年焕发出优质的能量状态。这也像手机充电一样,精就是人体的电池,一天天积累,一天天储蓄,最后能量就会满格,浑身洋溢着劲爆的能量,每天都表现得精神抖擞、斗志昂扬,在学业和事业上放大招,冲劲十足,干劲冲天,学习成绩和工作业绩突飞猛进、势如破竹!正所谓:我辈岂是蓬蒿人,一戒手淫便化龙!

\textit{冬不藏精,春必病温。(《黄帝内经》)} 这就告诉我们,如果不懂得冬季养藏之道,在冬季依然精液频泄,那么身体必然会日趋衰弱,到来年春天很可能会爆发出各种伤精症状,人的精气神也会变得极度萎靡,就像行尸走肉一般,一看就是气色极差、鬼气缠身,让人看了很不舒服,甚至很厌恶。善养生者,必奉于藏!藏功一定要好,冬季要格外注重养藏,就像运动员格外注重冬训一样,通过冬季的养藏和积累,在来年就会迸发出超强的身心状态,有如焕然新生、脱胎换骨一般。很多戒友的问题,就是藏不住,宝贵的生命能量被心魔小偷给盗走了,这个小偷不仅白天偷,晚上等你睡着了,它还会进入你的梦里偷,虽然梦是假的,但梦遗却是真的在耗泄宝贵的肾精。这个世界就是一个盗精空间,一个人的精华被盗走了,剩下的就是一堆人渣,就像甘蔗被压榨过一样。戒色养生就是要懂得保精和宝精,这两个字是同音字,但是意思却不一样,一个是要懂得保护保养自己的肾精,另一个就是要懂得宝贝自己的肾精。历代中医和养生名家,都把肾精看作是极其宝贵的能量,绝对不可轻泄,肾精一泄,就像苹果腐烂一般,瞬间失去了新鲜度和光润感,能量被抽走了,只剩下一具猥琐无比的躯壳。

寒冬一来临,对我们全身的器官都是一次严酷的考验,在这个气温骤降、风大干燥的季节,很多疾病都会反复。前段时间有个戒友的神经症就反复了,反馈比较多的还是前列腺炎的反复,主要表现就是尿频、尿不尽、尿等待、夜尿次数增多等。前列腺是男性的“娇嫩”器官,受凉是诱发前列腺炎最常见的原因,一个男人再强壮,他的前列腺也是脆弱的。气温降下来后,很多人的鼻炎也会加重,记得以前每年冬天,我都觉得很难熬,因为那时我有严重的过敏性鼻炎和冻疮,而现在这两个困扰我的问题,基本上完全好了,鼻炎很少犯,冻疮很久没出现了,戒色养生的确不可思议。有鼻炎问题的戒友我建议可以试试艾灸,我一般都艾灸合谷穴和足三里,艾灸对于我鼻炎的康复绝对功不可没。自古扶阳有三法:第一为灼艾,第二为丹药,第三为附子。灼艾就是艾灸,\textit{阳气者,若天与日,失其所,则折寿而不彰。(《黄帝内经》)} 熬夜、手淫和严寒都伤人体的阳气,艾灸可以补充阳气,扶正祛邪,补益强身,能激活提高免疫系统功能,促进新陈代谢,阳强则寿,阳衰则夭。\textit{人到四十,阳气不足,损与日至。(《内经灵枢》)} 意思是随着年龄的增长,人的阳气会逐渐亏耗,熬夜和手淫都会加速这种亏耗,让人未老先衰,《扁鹊心书》重点倡导的就是扶阳。当然,艾灸只是一种恢复方法和治疗手段,艾灸的效果也是因人而异的,有的见效比较快,有的则比较慢,刚开始艾灸还可能出现排病反应,我们应该先充分了解艾灸的相关知识,特别是艾灸的一些禁忌要事先了解清楚,不要盲目操作。

人体血管对温度十分敏感,血管在冬季遇冷就会收缩,容易变脆。寒冷的刺激会使人的收缩压及舒张压上升,就像一个橡皮水管,在天冷时橡胶就会变脆,如果水压再一增大,那么很容易就会裂开,我们的血管也是如此。当气温大幅度下降,血压的波动性就会增大,脑出血等疾病的发病率也会大大提升。特别是中老年人,他们的血管弹性已经比较差了,所以更容易出现脑出血等意外情况。我所居住的小区,前段时间就有一个在喝酒时脑出血的,后来没抢救过来,直接挂了,六十岁不到,很可惜。冬季很多人都会遭遇人生的不幸,我们现在虽然年轻,但也应该要了解这方面的知识,毕竟家里都有父母、爷爷奶奶等长辈,你懂点这方面的知识,也可以向他们多多宣传,防患于未然。

这季就冬季戒色养生分享以下五个要点,具体如下。

\subsubsection{适量锻炼,严格控制出汗量}

“冬天动一动,少闹一场病;冬天懒一懒,多喝一碗药。”冬季锻炼,不仅能提高身体对寒冷的适应能力,还可以磨炼意志。坚持冬季锻炼的人,身体强健,御寒能力强,但是如果锻炼的方式和锻炼的强度安排不当,则容易导致感冒或者其他的身体不适。另外,超负荷的锻炼也会使机体过度疲劳,导致抵抗力下降,所谓过犹不及,所以冬季锻炼其实有着很深的学问。空气清新、阳光明媚的日子可以选择在户外活动,在大风降温或者冰雪连天的日子,则不妨在室内锻炼,可以灵活安排。冬季气候寒冷,肌肉韧带的弹性、伸展性降低,粘滞性增高,关节不灵活,常感身体僵硬活动不便,因而突然进行剧烈活动容易造成肌肉拉伤、扭伤等伤害。冬季锻炼,应该特别注意热身,先做一些较缓和、运动量较低的热身运动,热身运动能提高身体主要部位的体温,并且能使更多的血液流向肌肉,从而为身体进行强度更大的活动作好准备,轻微活动后的拉伸运动可以使筋腱更灵活,可以有效提高体温并增加关节的活动范围,从而可避免关节、韧带和肌肉的拉伤。另外热身运动还可以调节心理状态,并且可以提高神经系统的兴奋性,使大脑皮层处于最佳的兴奋状态,在这样的状态下进行锻炼,可达到事半功倍的效果。

冬季锻炼的方式,诸如太极拳、养生气功、慢跑、快走、跳绳等都是不错的选择,年轻人打打球也很不错,但要注意运动量,也要做好自身防护。动则生阳,阳虚的人应以动养为先,但切记不可过于剧烈,阴虚的人应以静养为先,但也需配合动养。动静结合,方合养生大道,具体是以静养为主、动养为辅,还是以动养为主、静养为辅,这要根据自己的身体情况来具体安排。锻炼的过程中也要学会控制出汗量,人体本身也有两个阀门来控制,一个是释放阀门,一个是收藏阀门,即三阳与三阴的开阖枢,两个阀门都是双向的,有开有关。出汗就是一种释放,大汗伤阳,对身体恢复很不利,偶尔一次大汗还不觉得,如果经常出大汗,人很快就会虚掉。之前就有戒友听说跑步好,就天天几千米,结果身体恢复反而不理想,最近新闻也报道过一个退役特种兵在跑马拉松时猝死,相信很多戒友都应该看过那条新闻。锻炼量和出汗量一定要严格掌握,并不是越多越好,这里面大有讲究。年轻人就是冲劲足,但是养生的意识很缺乏,也不懂得控制自己的出汗量,这样很可能会适得其反。我现在每次锻炼,对出汗量都有严格的控制,每次身体发热快要出汗时,我就会降低运动强度,从跑改为快走,这样就可以把出汗量控制在微汗的程度,冬季养藏,还是应该尽量避免出大汗。《黄帝内经》告诉我们冬季养生要“无泄皮肤”,意思是说不要使皮肤开泄而出汗,中医认为,“盖汗之为物,以阳气为运用,以阴精为材料”,汗液就是阳气蒸腾阴液所致,因此出汗过多也会损伤阳气。

\subsubsection{天人相应,早睡晚起,增加日晒量}

\textit{冬三月,此谓闭藏,水冰地坼,无扰乎阳,早卧晚起,必待日光。(《黄帝内经》)} 中医认为冬季养生应早睡晚起,日出而作,保证充足的睡眠,这样有利于阳气潜藏、阴精蓄积。(从生理角度来讲,阴精与阳气是互根互用、互存互亡的关系,阴精是阳气功能活动的物质基础。)冬日阳气肃杀,夜间尤甚,所以古人主张要“早卧晚起”,早睡以养阳气,晚起以固阴精。这里的早睡晚起,并不是让你睡到十点或者十一点再起来,而是相对于其他季节的晚起。中医也讲到久卧伤气,适当的睡眠对人体是非常有益的,睡眠少则会伤身,气血得不到补养,但是睡眠太多也会对健康造成不利的影响,容易出现头晕乏力的现象。冬天三个月,天地之气收敛,阳气潜藏于地下,阴气弥漫于天地间,所以此时的气温是四季中最低的,水开始结冰,大地也冻裂了,天地间的阳气都在潜藏,这时候一定要懂得养藏,不要去搅扰阳气。在这个寒冷的季节里,自然界阴盛阳衰,寒气袭人,极易损伤人体的阳气,所以冬季养生应从敛阴护阳出发。“三九、四九,关门缩手。”在冬天就应当保暖避寒,护藏阳气。

冬天一定要多晒晒太阳,晒太阳的时候又可以重点晒晒头部和背部。冬天晒太阳就是接天气,可以起到生发、滋养人体阳气的作用,让人体阴阳达到一个平衡。在冬天里晒太阳,对增加人体皮肤和内脏器官的血液循环、提高造血功能、调节中枢神经、增强人体各部位新陈代谢和免疫功能均大有益处。特别是对骨质疏松症,有着非常好的疗养效果。中医认为“头为诸阳之首”,是所有阳气汇聚的地方,凡五脏精华之血、六腑清阳之气,皆汇于头部。百会穴位于头顶正中(过两耳直上连线中点),是晒太阳的重点,让阳光洒满头顶,这样就可以通畅百脉、调补阳气。人体腹为阴,背为阳,很多经脉和穴位都在后背,晒这里能起到调理脏腑气血的作用。坐着、站着或者锻炼时,可以后背对着太阳。

\subsubsection{多做养生功法,严格控遗}

戒色后一定要在养生方面多下功夫,懂得养生的戒友会恢复得更快、更好,如果不会养生,很可能戒了很久,恢复还是很不理想。养生是门大学问,戒色后很多戒友都开始自觉钻研养生知识,这样可以大大提高他们的养生觉悟,养生觉悟上去了,对恢复是非常有利的,因为知道了什么该做,什么不该做,做什么会对身体健康造成潜在的危害,怎么做才能恢复得更好,这些都是学问。我一般都说戒色养生,这两个是同样重要的,光戒不行,必须要学会养生,其实说到底,戒色也属于养生的范畴,因为养生就包括戒色保精的内容,只不过传统戒色和专业戒色有所区别而已。中医养生,就是指通过各种方法颐养生命、增强体质、预防疾病,从而达到延年益寿的一种医事活动。现在养生也已经成为了一种文化现象,在现代社会,很多人都处于亚健康的身心状态,这时候也的确需要传统的养生智慧来给予指导。我之前也推荐过养生功法,一般以八段锦、站桩和打坐为主,当然养生功法有很多,大概有几十种乃至上百种,可以根据自己的兴趣爱好来选择练习。

进入冬季后,很多戒友也出现了频遗,如何控遗是戒色后的一大难关,很多戒友,包括资深的戒友都会栽在这个问题上。遗精虽然不算破戒,但是遗精后心理防线容易出现松动,邪念也会变得活跃,这时候就比较容易出现破戒的情况,频遗更是加大了破戒的可能性。出现频遗也很影响身体的恢复,毕竟遗精也是漏,在冬季这个养藏的季节里,这样频繁地漏,损失也是非常大的。这个冬天能量储备好了,来年春天才会迸发出超强的活力,这个冬天就像在为来年充电一样。在寒冷的冬季,我们应该好好养藏,尽量减少肾精的漏失,把种子的能量真正守住了,必须严格控遗,降低精耗,把损失降至最低。手机有省电模式,我们也要学会开启人体的省精模式,整个冬季,尽量不要破戒,然后要把遗精次数控制在相对较少的范围内。

\subsubsection{注重食疗,提升免疫力}

俗话说,冬天进好补,来年打老虎!做好食补,少进药铺。冬季有补冬之说,冬天是应该在食补方面特意加强,这样明年春天来临时,就会龙马精神,虎虎生威。但是食补方面的水也很深,有的人一听说进补,就以为吃山珍海味肥甘厚腻才叫补,结果往往是瞎补乱补一通,这样反而会适得其反,补出不少毛病。如何科学进补,也是一门值得研究的学问,要具备这方面的专业知识才行。补也要根据自己的体质来安排,千万不可盲目瞎补,适合别人的食材,很可能完全不适合你,别人吃了一点事都没有,而你吃了很可能就会出现腹胀便秘或者开始便溏。之前就有戒友说黑豆很好,黑豆的确不错,有豆中之王的美称,有解表清热、养血平肝、补肾壮阴、补虚黑发之功效。宋朝著名文学家苏东坡也曾记述,当时京城内外,少男少女都为了养颜美容而服食黑豆。但是黑豆不适于空腹食用,因黑豆含高纤维不易消化,若空腹食用会过度刺激胃壁,有胃病者易导致疼痛、肠阻塞、腹泻等症状,有的人吃了以后,肚子一直叫,感觉不大舒服,还有的人吃了以后会拉肚子。所以吃黑豆也要看个人的体质,服用量和服用方法方面也很有讲究。

在食疗方面,我首推粥疗,我以前也曾向大家推荐过粥疗,山药粥、红枣粥、百合粥、五谷养生粥、红豆薏米粥、核桃粥、燕麦粥、黑芝麻粳米粥、银耳白果粥、莲藕大米粥、南瓜小米粥、五仁粥、栗子桂圆粥、五豆粥、胡萝卜粥、芡实粥、仙人粥、黑米八宝粥等等,这些粥都不错,可以根据自己的体质来食用。就像印光大师所言,补物实在很多,并不一定要大荤大肉,每天大荤大肉,实在很危险,这个危险主要指的就是破戒的危险,另外一份研究报告也指出,长期高脂饮食(大荤)不仅会加重肾病患者已有的肾损害,也会导致健康人的肾脏受损。很多戒友在吃了大荤之后,邪念会变得异常活跃,这时候如果修心功夫还不够稳定,那就很容易出现破戒。据我深入的研究和体会,邪淫之人吃了大荤,破起戒来也很猛,泄完精后心里又想吃大荤来滋补自己的身体,从而陷入了恶性循环,越吃越想破,越破越想吃,掉进了怪圈。冬补我是不建议吃大荤的,肉还是尽量少吃,这样对控制欲望很有帮助。

分享一首粥疗歌:

\begin{multicols}{2}
    \begin{center}
        若要不失眠,煮粥添白莲。 \\ 要得皮肤好,米粥煮红枣。 \\ 气短体虚弱,煮粥加山药。 \\ 治理血小板,煮粥花生衣。 \\ 心虚气不足,桂圆煨米粥。 \\ 要治口臭症,荔枝粥除根。 \\ 清退高热症,煮粥加芦根。 \\ 血压高头晕,胡萝卜粥灵。 \\ 要保肝功好,枸杞煮粥妙。 \\ 口渴心烦躁,粥加猕猴桃。 \\ 防治脚气病,米糠煮粥饮。 \\ 肠胃缓泻症,胡桃米粥炖。 \\ 头昏多汗症,煮粥加苡仁。 \\ 便秘补中气,藕粥很相宜。 \\ 夏令防中暑,荷叶同粥煮。 \\ 若要双目明,粥中加旱芹。
    \end{center}
\end{multicols}

\subsubsection{心地善良,心灵纯净、心胸开阔,心情愉悦}

养生的最高境界就是养心,下士养身,中士养气,上士养心。看相也是如此,观相不如观气,观气不如观心。现代人就是心太累,各种烦恼,而且心也太脏了,充满了积垢,特别是黄毒的积垢,潜意识里太多太多了,很多人经常看H,把自己整个儿洗成了黄脑,看到女性马上就条件反射地起邪念,浑身上下笼罩着一股邪秽之气,让人厌恶。当今社会最需要的就是心灵纯净!心灵的清浊决定着我们的健康状况,因为身心是合一的,心脏了,身体也会跟着出问题;心灵的清浊也决定着我们的生活质量,更决定着我们的幸福指数,心灵的纯净和清澈比什么都重要!纯净的心灵不仅能造就高雅的气质,更重要的是能获得真正的大自由、大快乐,完全超脱了邪淫的作茧自缚;纯净的心灵不仅是健康、快乐、智慧的源头,更是人生最美妙、最高明的境界。

\textit{常观天下之人,凡气之温和者寿,质之慈良者寿,量之宽宏者寿,言之简默者寿。盖四者,仁之端也,故曰仁者寿。(《中外卫生要旨》)} 仁者养心,仁就是要做到温和、善良、宽宏、厚道。仁心仁德、养心立德是一个人健康的内在要素。\textit{德行不克,纵服玉液金丹,未能延年……道德日全,不祈善而有福,不求寿而自延,此养生之大旨也。(唐代大医孙思邈)} 中医认为德高者五脏淳厚,气血匀和,阴平阳秘,所以能健康长寿。荀子言:“\textit{有德则乐,乐则能久,}”孔子更精辟地指出:“\textit{大德必得其寿。}”在这个寒冷的冬季,我们应该保有一颗善心,多行善事,积善成德,善则生阳,助人即助己,为善最乐,妙不可言。《黄帝内经》也强调“德全不危也”,同时也说到“恬淡虚无,真气从之,精神内守,病安从来”。畅通的经络需要的是清净心,一切七情六欲都会破坏清净心,从而破坏经络的正常运行,久而久之,疾病必然会爆发出来。不仅要让心保持清净纯正,更要让心充满善意,善待他人,善待一切生灵,自己的心胸也要保持开阔,如海阔天空一般,是一种大场面、大境界,坚持戒色修善之人,一定会感受到真正的大快乐、大愉悦!

\paragraph*{总结}

从四季的角度来看,春天迸发的生机源于冬天的蓄藏和积累,因此,冬季是养藏的重要季节。精要藏得住,藏得深,藏得心魔找不到,肾精就是身体的王牌,有王牌才有底气。面对心魔的疯狂攻击,我们应该经常上演让人热血澎湃的绝杀表演,谱一曲惊心动魄荡气回肠的英雄史诗,彻底逆袭自己颓废惶恐的人生,完成宛若新生般的蜕变,翻开生命中最为光辉璀璨的一章,这一章的主题就叫戒色修善!

下面分享二首戒色诗歌。

\begin{poem}[撸没一生]
    \begin{multicols}{2}
        \centering~\\
        快感如此短暂 \\ 就像兔子的尾巴掠过秋天的草原 \\ 当他回过神来,症状已经汹涌来袭 \\ 此刻,他坐在医院走廊的长凳上 \\ 目光呆滞,凝望着地面光滑的大理石 \\ 检查报告出来了,慢前精索无精症 \\ 右肾结石伴囊肿,真有一种欲哭无泪的感觉 \\ 有些疾病意料之中,有些则完全没有心理准备 \\ 撸了十几年,报应终究还是来了 \\ 他见四下无人,仰天长叹:“都是报应啊!” \\ 出来撸迟早是要还的!用痛苦还!到医院还! \\ 不仅是这些检查出来的疾病,还有早泄阳痿, \\ 还有令人崩溃的神经症,失眠多梦,头晕耳鸣 \\ 几乎让人失去活下去的希望与勇气 \\ 邪淫的人生注定是惶恐而灰暗的人生 \\ 曾几何时,他活在无撸的世界里 \\ 那时的每一天都那么单纯而美好 \\ 直到后来的一次不经意的摩擦 \\ 让他彻底落入了邪淫的魔窟 \\ 他开始变了,变成了另外一个人 \\ 变成了一个猥琐龌龊的存在 \\ 变成了一个丑不拉几的怪物 \\ 走出医院,他抬头望了望天空 \\ 天空灰暗无光,一如他此刻的心情 \\ 他已经把自己撸成了废人 \\ 而且检查吃药也花了无数的钱 \\ 病都没好,人倒是成了药罐子 \\ 这时候再想想手淫无害论 \\ 真是天大的讽刺,无害论只会害人 \\ 很多时候他都有一种不真实感 \\ 他知道这是神衰带来的错觉 \\ 但是他多么希望邪淫的十几年 \\ 只是一个荒唐的恶梦 \\ 从这个恶梦中醒来 \\ 他还是那个纯真的男孩 \\ 用最无邪的眼神 \\ 看着这个世界
    \end{multicols}
\end{poem}

\begin{poem}[皮之魔]
    \begin{multicols}{3}
        \centering~\\
        人体皮肤厚度 \\ 因人或部位不同 \\ 一般为 $0.5 \unit{\mm} \sim 4 \unit{\mm}$ \\ 有位戒友这样说 \\ 他被这层皮害惨了 \\ 一方面是女色之皮 \\ 另外就是自己的包皮 \\ 对着 H 片疯狂跑皮 \\ 人家马拉松在马路上跑 \\ 他在自己的 JJ 上跑 \\ 最后把自己跑废 \\ 把自己跑进医院 \\ 把脸跑丑、把身体跑差 \\ 把整个人生跑毁 \\ 这辈子就迷在皮里面了 \\ 其实揭开那层皮 \\ 里面真的是很恶心 \\ 简直令人作呕不忍直视 \\ 但是就是这层皮 \\ 瞒尽了天下英雄 \\ 有多少人为这层皮疯狂 \\ 有多少人为这层皮瞎撸 \\ 有多少人被这层皮废掉 \\ 实在难以统计 \\ 这层皮的魔力可见一斑 \\ 其实皮里面包裹的 \\ 都是什么啊 \\ 屎尿内脏脓血淋漓 \\ 皮下白骨 \\ 更是令人毛骨悚然 \\ 人就是一具臭皮囊 \\ 真是一皮障目不见恶心 \\ 没有人会迷恋粪便 \\ 但他们却如此迷恋造粪机 \\ 实在不可思议 \\ 这层皮让人走火入魔 \\ 这层皮让天之骄子泯然众人 \\ 这层皮太狠了,废人无量的皮 \\ 我们必须彻底看破这层皮 \\ 迷于皮的人生是荒唐的人生 \\ 邪淫的人生是不正常的人生 \\ 堂堂七尺男儿身 \\ 顶天立地撑乾坤 \\ 勇猛精进戒手淫 \\ 无愧祖先振雄风
    \end{multicols}
\end{poem}

(说明:不净观只是为了对治自己的邪念,并非不尊重女性,这点一定要认识清楚,否则很容易误解不净观。)
