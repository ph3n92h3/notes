\documentclass{article}

\usepackage[a4paper,scale=0.8]{geometry}

\usepackage{ctex} % [UTF8]

\setCJKmainfont{Noto Serif CJK SC}
\setCJKsansfont{Noto Sans CJK SC}
\setCJKmonofont{Noto Sans Mono CJK SC}

\usepackage{amssymb}
% \usepackage{algorithm2e}
\usepackage{cancel}
\usepackage{emoji}
\usepackage{extarrows}
% \usepackage{float}
\usepackage{framed}
\usepackage[colorlinks=true]{hyperref}
% \usepackage{minted}
\usepackage{physics}
% \usepackage{unicode-math}
\usepackage{upgreek}

\newcommand{\rme}{\mathrm{e}}
\newcommand{\rmi}{\mathrm{i}}

\title{电动力学}
\author{桜井\ 雪子}
\date{}

\begin{document}
\maketitle

這篇文章記錄一下我在電動力學課程進行的過程中做的事情,按道理講不算是一個 review,但是我尚無法想出一個更好的名字。

\section{指标运算推导矢量微分学公式}

由于电动力学一般都在三维欧氏空间讨论,故不区分上下指标,统一写成下指标。另外,我现在尚没有一个固定的习惯,有时我写具体指标,有时我写抽象指标。可以这么说,在这篇文章中,当我写的是具体指标时,我是抄的 \href{https://www.zhihu.com/people/Masaki.Ryuu}{知乎:東雲正樹};当我写抽象指标时,我是自己推的。(仅仅是因为我记不住希腊字母在具体指标中的顺序)

\begin{itemize}
    \item 我现在还发现一个使用抽象指标的好处——在打字的时候我可以少打一些。
\end{itemize}

为了正确地渲染公式,花了些力气,现在处于一个妥协的状态:对于行内公式,\texttt{\$...\$} 不能出现在空格后,但是那正是我写 \LaTeX 时所习惯的。等等,似乎可以???

\subsection{基本公式}

\begin{itemize}
    \item $$(\vec{a}\times\vec{b})_{\rho}=\varepsilon_{\mu\nu\rho}a_{\mu}b_{\nu}$$
    \item $$\varepsilon_{\mu\nu\rho}\varepsilon_{\mu\sigma\tau}=\delta_{\nu\sigma}\delta_{\rho\tau}-\delta_{\nu\tau}\delta_{\rho\sigma}$$
    \item $$\varepsilon_{\mu\nu\rho}\varepsilon_{\sigma\nu\rho}=2\delta_{\mu\sigma}$$
    \item $$\varepsilon_{\mu\nu\rho}\varepsilon_{\mu\nu\rho}=6$$
\end{itemize}

这里本应该有更多的,但是那些在下面用不到,详见\href{https://zhuanlan.zhihu.com/p/331738362}{知乎:東雲正樹的文章}

提一句,我们学校使用的教材是郭硕鸿先生写的《电动力学》,很多时候我这里写的式子的形式都是按照他的书 \emoji{book},我看书比较杂,没有自己的特定习惯。

\begin{quote}
    只要你自己足够强,就能用自己的符号! ——温伯格
\end{quote}

\subsection{開始表述}

\subsubsection{混合积}

$$\vec{c}\cdot(\vec{a}\times\vec{b})=\varepsilon_{\mu\nu\rho}a_{\mu}b_{\nu}c_{\rho}$$

\subsubsection{三重积}

$$\vec{c}\times(\vec{a}\times\vec{b})=(\vec{c}\cdot\vec{b})\vec{a}-(\vec{c}\cdot\vec{a})\vec{b}$$

$$\begin{aligned}
        \text{left: }[\vec{c}\times(\vec{a}\times\vec{b})]_e                        & =\varepsilon_{cde}c_{c}(\vec{a}\times\vec{b})_{d}=\varepsilon_{cde}c_{c}\varepsilon_{abd}a_{a}b_{b} \\
                                                                                    & =\varepsilon_{dec}\varepsilon_{dab}a_{a}b_{b}c_{c}                                                  \\
                                                                                    & =(\delta_{ae}\delta_{bc}-\delta_{ac}\delta_{be})a_{a}b_{b}c_{c}                                     \\
                                                                                    & =a_{e}b_{b}c_{b}-a_{a}b_{e}c_{a}                                                                    \\
        \text{right: }[(\vec{c}\cdot\vec{b})\vec{a}-(\vec{c}\cdot\vec{a})\vec{b}]_e & =(\vec{c}\cdot\vec{b})a_{e}-(\vec{c}\cdot\vec{a})b_{e}                                              \\
                                                                                    & =b_{b}c_{b}a_{e}-c_{a}a_ab_e\qquad\square
    \end{aligned}$$

似乎这个字间距有些不够优雅了,先这样吧 :-)

\begin{itemize}
    \item 推论:$\nabla\times(\nabla\times\vec{f})=\nabla(\nabla\cdot\vec{f})-\nabla^2\vec{f}$
\end{itemize}

\subsubsection{几个热身的例子}

\begin{itemize}
    \item $$\nabla(\varphi\psi)=\varphi\nabla\psi+\psi\nabla\varphi$$
\end{itemize}

这个我觉得显然吧?

$$[\nabla(\varphi\psi)]_{a}=\partial_{a}(\varphi\psi)=\varphi\partial_{a}(\psi)+\partial_{a}(\varphi)\psi=[\varphi\nabla\psi+\psi\nabla\varphi]_{a}\qquad\square$$

\begin{itemize}
    \item $$\nabla\cdot(\varphi\vec{f})=(\nabla\varphi)\vec{f}+\varphi\nabla\cdot\vec{f}$$
\end{itemize}

这个说实话我也觉得有点显然……

$$\nabla\cdot(\varphi\vec{f})=\partial_{a}(\varphi f_{a})=\partial_{a}(\varphi)f_{a}+\varphi\partial_{a}(f_{a})=(\nabla\varphi)\vec{f}+\varphi\nabla\cdot\vec{f}\qquad\square$$

\begin{itemize}
    \item $$\nabla\times(\varphi\vec{f})=(\nabla\varphi)\times\vec{f}+\varphi\nabla\times\vec{f}$$
\end{itemize}

说实话这个跟上面那个差不多……

$$\begin{aligned}
        \qty[\nabla\times(\varphi\vec{f})]_{c} & =\varepsilon_{abc}\partial_{a}(\varphi f_{b})                                            \\
                                               & =\varepsilon_{abc}\partial_{a}(\varphi)f_{b}+\varphi\varepsilon_{abc}\partial_{a}(f_{b}) \\
                                               & =[(\nabla\varphi)\times\vec{f}+\varphi\nabla\times\vec{f}]_{c}\qquad\square
    \end{aligned}$$

\subsubsection{真刀真枪}

\begin{itemize}
    \item $$\nabla\cdot(\vec{f}\times\vec{g})=(\nabla\times\vec{f})\cdot\vec{g}-\vec{f}\cdot(\nabla\times\vec{g})$$
\end{itemize}

这个开始才有点用得上指标运算。

$$\begin{aligned}
        \nabla\cdot(\vec{f}\times\vec{g}) & =\partial_{c}[\vec{f}\times\vec{g}]_{c}=\partial_{c}[\varepsilon_{abc}f_{a}g_{b}]    \\
                                          & =(\varepsilon_{abc}\partial_{c}f_{a})g_{b}+f_{a}(\varepsilon_{abc}\partial_{c}g_{b}) \\
                                          & =(\varepsilon_{cab}\partial_{c}f_{a})g_{b}-f_{a}(\varepsilon_{cba}\partial_{c}g_{b}) \\
                                          & =[\nabla\times\vec{f}]_{b}g_{b}-f_{a}[\nabla\times\vec{g}]_{a}                       \\
                                          & =(\nabla\times\vec{f})\cdot\vec{g}-\vec{f}\cdot(\nabla\times\vec{g})\qquad\square
    \end{aligned}$$

\begin{itemize}
    \item $$\nabla\times(\vec{f}\times\vec{g})=(\vec{g}\cdot\nabla)\vec{f}+(\nabla\cdot\vec{g})\vec{f}-(\vec{f}\cdot\nabla)\vec{g}-(\nabla\cdot\vec{f})\vec{g}$$
\end{itemize}

$$\begin{aligned}
        \text{left: }[\nabla\times(\vec{f}\times\vec{g})]_{e}                                                                             & =\varepsilon_{cde}\partial_{c}[\vec{f}\times\vec{g}]_{d}=\varepsilon_{cde}\partial_{c}(\varepsilon_{abd}f_{a}g_{b})    \\
                                                                                                                                          & =\varepsilon_{cde}\varepsilon_{abd}\partial_{c}(f_{a}g_{b})=\varepsilon_{dec}\varepsilon_{dab}\partial_{c}(f_{a}g_{b}) \\
                                                                                                                                          & \xlongequal{2.}(\delta_{ea}\delta_{cb}-\delta_{eb}\delta_{ca})[(\partial_{c}f_{a})g_{b}+f_{a}(\partial_{c}g_{b})]      \\
                                                                                                                                          & =(\partial_{b}f_{e})g_{b}+f_{e}(\partial_{b}g_{b})-(\partial_{a}f_{a})g_{e}-f_{a}(\partial_{a}g_{e})                   \\
        \text{right: }[(\vec{g}\cdot\nabla)\vec{f}+(\nabla\cdot\vec{g})\vec{f}-(\vec{f}\cdot\nabla)\vec{g}-(\nabla\cdot\vec{f})\vec{g}]_e & =(\vec{g}\cdot\nabla)f_{e}+(\nabla\cdot\vec{g})f_{e}-(\vec{f}\cdot\nabla)g_{e}-(\nabla\cdot\vec{f})g_{e}               \\
                                                                                                                                          & =g_a(\partial_{a}f_{e})+f_{e}(\partial_{a}g_{a})-f_{a}(\partial_{a}g_{e})-(\partial_{a}f_{a})g_{e}
    \end{aligned}$$

\begin{itemize}
    \item $$\nabla(\vec{f}\cdot\vec{g})=\vec{f}\times(\nabla\times\vec{g})+(\vec{f}\cdot\nabla)\vec{g}+g\times(\nabla\times\vec{f})+(\vec{g}\cdot\nabla)\vec{f}$$
\end{itemize}

$$\begin{aligned}
        \text{left: }  & [\nabla(\vec{f}\cdot\vec{g})]_{e}                                                                                                                                                             \\
        =              & \partial_{e}(f_ag_a)                                                                                                                                                                          \\
        =              & (\partial_{e}f_{a})g_{a}+f_{a}(\partial_{e}g_{a})                                                                                                                                             \\
        \text{right: } & [\vec{f}\times(\nabla\times\vec{g})+(\vec{f}\cdot\nabla)\vec{g}+g\times(\nabla\times\vec{f})+(\vec{g}\cdot\nabla)\vec{f}]_{e}                                                                 \\
        =              & \varepsilon_{cde}f_{c}(\varepsilon_{abd}\partial_{a}g_{b})+(f_{a}\partial_{a})g_{e}+\varepsilon_{cde}g_{c}(\varepsilon_{abd}\partial_{a}f_{b})+(g_{a}\partial_{a})f_{e}                       \\
        =              & (\varepsilon_{dec}\varepsilon_{dab})f_{c}\partial_{a}g_{b}+(f_{a}\partial_{a})g_{e}+(\varepsilon_{dec}\varepsilon_{dab})g_{c}\partial_{a}f_{b}+(g_{a}\partial_{a})f_{e}                       \\
        =              & (\delta_{ea}\delta_{cb}-\delta_{eb}\delta_{ca})f_{c}\partial_{a}g_{b}+(f_{a}\partial_{a})g_{e}+(\delta_{ea}\delta_{cb}-\delta_{eb}\delta_{ca})g_{c}\partial_{a}f_{b}+(g_{a}\partial_{a})f_{e} \\
        =              & f_{b}\partial_{e}g_{b}-\bcancel{f_{a}\partial_{a}g_{e}}+\bcancel{(f_{a}\partial_{a})g_{e}}+g_{b}\partial_{e}f_{b}-\bcancel{g_{a}\partial_{a}f_{e}}+\bcancel{(g_{a}\partial_{a})f_{e}}         \\
        =              & f_{b}\partial_{e}g_{b}+g_{b}\partial_{e}f_{b}\qquad\square
    \end{aligned}$$

\begin{itemize}
    \item 多说一句,这个删除线的效果需要宏包 \texttt{cancel}
\end{itemize}

\section{四維語言表述}

from 《微分幾何入門與廣義相對論》 by 梁燦彬,使用几何高斯制。

\subsection{电磁场张量 $F_{ab}$}

\subsubsection{电场:$E_{a}=F_{ab}Z^b$,磁场:$B_a=-^*F_{ab}Z^b=-\frac{1}{2}\varepsilon_{abcd}F^{cd}Z^b$}

\begin{itemize}
    \item $$\Rightarrow E_{i}=F_{i0},\ B_{1}=F_{23},\ B_{2}=F_{31},\ B_{3}=F_{12}$$
    \item $$\begin{aligned}
                  E_1'=E_1,\  & E_2'=\gamma(E_2-vB_3),\  & E_3'=\gamma(E_3+vB_2) \\
                  B_1'=B_1,\  & B_2'=\gamma(B_2+vE_3),\  & E_3'=\gamma(B_3+vE_2)
              \end{aligned}$$
\end{itemize}

\subsubsection{4 电流密度:$J^a=\rho_0U^a$}

\begin{itemize}
    \item $$J^a=\rho_0U^a=\rho_0\gamma(Z^a+u^a)=\rho Z^a+j^a$$
\end{itemize}

\subsubsection{Maxwell Eqns.}

\begin{enumerate}
    \item $$\partial^aF_{ab}=-4\pi J_b$$
    \item $$\partial_{[a}F_{bc]}=0$$
\end{enumerate}

\begin{itemize}
    \item 从这里推回 3 维形式之后会补充(可能直接就鸽了 :)
\end{itemize}

\subsubsection{4 维洛伦兹力:$F^a=qF^a_{\ b}U^b$,电荷运动方程:$qF^a_{\ b}U^b=U^b\partial_b P^a$。}

\subsubsection{能动张量:$T_{ab}=\frac{1}{4\pi}(F_{ac}F_{b}^{\ c}-\frac{1}{4}\eta_{ab}F_{cd}F^{cd})=\frac{1}{8\pi}(F_{ac}F_{b}^{\ c}+^*F_{ac}^*F_{b}^{\ c})$}

\subsubsection{势}

\begin{itemize}
    \item $\mathrm{d}\boldsymbol{F}=0\Rightarrow\boldsymbol{F}=\mathrm{d}\boldsymbol{A}\Rightarrow F_{ab}=\partial_aA_b-\partial_bA_a$
    \item $A_a=-\phi(\mathrm{d}t)_a+a_a$
    \item 规范变换:$\boldsymbol{F}=\mathrm{d}\boldsymbol{A}\Rightarrow\boldsymbol{F}=\mathrm{d}\tilde{\boldsymbol{A}}=\mathrm{d}(\boldsymbol{A}+\mathrm{d}\chi)$
    \item Maxwell Eqns in one: $\partial^a\partial_aA_b=-4\pi J_b$
\end{itemize}

Loading...

\end{document}