\documentclass{article}

\usepackage[a4paper,scale=0.8]{geometry}

\usepackage{ctex}

% \usepackage{algorithm2e}
% \usepackage{amsfonts}
% \usepackage{amsmath}
% \usepackage{amssymb}
% \usepackage{cancel}
% \usepackage{emoji}
\usepackage{extarrows}
% \usepackage{float}
% \usepackage{framed}
% \usepackage[colorlinks]{hyperref}
% \usepackage{mathrsfs}
% \usepackage{mathtools}
% \usepackage{minted}
% \usepackage{multicol}
% \usepackage{physics}
% \usepackage{pifont}
% \usepackage{unicode-math}
% \usepackage{upgreek}
\usepackage{xcolor}

\newcommand{\rme}{\mathrm{e}}
\newcommand{\rmi}{\mathrm{i}}

\title{微积分}
\author{桜井\ 雪子}
\date{}

\begin{document}

\maketitle

为快速准备高数比赛而作。


\section{极限}

\subsection{小量展开公式}

\begin{align*}
    \sin x          & = x - \frac{x^3}{6} + \frac{x^5}{120}   \\
    \cos x          & = 1 - \frac{x^2}{2} + \frac{x^4}{24}    \\
    \tan x          & = x + \frac{x^3}{3} + \frac{2 x^5}{15}  \\
    \mathrm{e}^x    & = 1 + x + \frac{x^2}{2} + \frac{x^3}{6} \\
    \ln (1 + x)     & = x - \frac{x^2}{2} + \frac{x^3}{3}     \\
    \frac{1}{1 - x} & = 1 + x + x^2                           \\
    \frac{1}{1 + x} & = 1 - x + x^2
\end{align*}

\subsection{Stirling 公式}

\[ n! = \sqrt{2 \pi n} \left(\frac{n}{e}\right)^n + \frac{\theta_n}{12 n}, \theta \in (0, 1) \]

\subsection{Stolz 公式}

\[ \lim_{n \to \infty} \frac{x_n}{y_n} = \lim_{n \to \infty} \frac{x_{n+1} - x_n}{y_{n+1} - y_n} \]

\subsection{间断点}

\begin{itemize}
    \item 第一类间断点 = 可去间断点 + 跳跃间断点
    \item 第二类间断点 = 无穷间断点 + 震荡间断点
\end{itemize}

\section{导数}

\subsection{定义}

\[ f'(x) \equiv \lim_{\Delta x \to 0} \frac{f(x + \Delta x) - f(x)}{\Delta x} \]

\subsection{高阶导数公式}

\begin{align*}
    \left[\frac{1}{x + a}\right]^{(n)}   & = (-1)^n n! \frac{1}{(x + a)^{n+1}}             \\
    \left[\begin{aligned}
                  \sin \\ \cos
              \end{aligned} (a x + b)\right]^{(n)} & = a^n \begin{aligned}
                                                           \sin \\ \cos
                                                       \end{aligned} (a x + b + n \pi / 2)
\end{align*}

找微分中值定理的本质方法是微分方程法

\subsection{在不同的地方展开}

\section{不定积分}

\[ \int \frac{1}{1 + x^2} \mathrm{d}x = \arctan x + C, \int \frac{1}{\sqrt{1 - x^2}} \mathrm{d}x = \arcsin x + C \]

\begin{itemize}
    \item 分部积分
    \item 如果都堆在分母上用倒代换,如果是一个完整精致的根号把它整个代换
    \item 分段函数记得衔接好分段点
    \item 隐函数的积分:写成参数方程 $y = y(x) \Rightarrow x = x(t), y = y(t)$
    \item \[ \int \frac{a \sin x + b \cos x}{c \sin x + d \cos x} \mathrm{d}x = \int \frac{A (c \sin x + d \cos x) + B (c \sin x + d \cos x)'}{c \sin x + d \cos x} \mathrm{d}x = A x + B \ln |c \sin x + d \cos x| + C \]
    \item \[ \int \frac{1}{\sin^m x \cos^n x} \mathrm{d}x = \int \frac{\sin^2x + \cos^2 x}{\sin^m x \cos^n x} \mathrm{d}x = \int \left[\frac{1}{\sin^{m-2} x \cos^n x} + \frac{1}{\sin^m x \cos^{n-2} x}\right] \mathrm{d}x \]
\end{itemize}

\section{定积分}

\begin{itemize}
    \item 奇偶性
    \item 区间再现 \[ \int_a^b f(x) \mathrm{d}x = \int_a^b f(a + b - x) \mathrm{d}x \]
    \item \[ \int g(x) f(x) \mathrm{d}x \xlongequal{F(x) \equiv \int f(x) \mathrm{d}x} \left.\int g \mathrm{d}F = g F\right| - \int F \mathrm{d}g \]
\end{itemize}

\subsection{H\"{o}lder / Cauchy-Schwartz 不等式}

\[ \left| \int_a^b f(x) g(x) \mathrm{d} x \right| \leq \left( \int_a^b |f(x)|^p \mathrm{d} x \right)^{\frac{1}{p}} \cdot \left( \int_a^b |g(x)|^q \mathrm{d} x \right)^{\frac{1}{q}} \Rightarrow \left| \int_a^b f(x) g(x) \mathrm{d} x \right|^2 \leq \left( \int_a^b |f(x)|^2 \mathrm{d} x \right) \cdot \left( \int_a^b |g(x)|^2 \mathrm{d} x \right) \]

\section{多元微分}

\section{解析几何}

\subsection{平面}

平面方程

\begin{align*}
    \text{一般式:} & A x + B y + C z + D = 0                     \\
    \text{点法式:} & A (x - x_0) + B (y - y_0) + C (z - z_0) = 0 \\
    \text{截距式:} & \frac{x}{a} + \frac{y}{b} + \frac{z}{c} = 1
\end{align*}

点到平面距离

\[ d = \frac{|A x_0 + B y_0 + C z_0 + D|}{\sqrt{A^2 + B^2 + C^2}} \]

\subsection{直线}

直线方程

\begin{align*}
    \text{一般式:} & \begin{cases}
                      A_1 x + B_1 y + C_1 z + D_1 = 0 \\
                      A_2 x + B_2 y + C_2 z + D_2 = 0
                  \end{cases}                           \\
    \text{对称式:} & \frac{x - x_0}{m} = \frac{y - y_0}{n} = \frac{z - z_0}{p} \\
    \text{参数式:} & \begin{cases}
                      x = x_0 + m t \\
                      y = y_0 + n t \\
                      z = z_0 + p t
                  \end{cases}
\end{align*}

点到直线的距离

\[ d = \frac{\left|\overrightarrow{(x, y, z) - (x_0, y_0, z_0)} \times (m, n, p)\right|}{|(m, n, p)|} \]

直线到直线的距离

\[ d = \frac{\left|\left[(m_1, n_1, p_1) \times (m_2, n_2, p_2) \right] \cdot \overrightarrow{(x_1, y_1, z_1) - (x_2, y_2, z_2)}\right|}{\left|(m_1, n_1, p_1) \times (m_2, n_2, p_2)\right|} \]

\section{重积分}

\begin{itemize}
    \item 直角坐标 / 极坐标转换
    \item 交换积分次序「空间直角坐标、平面极坐标」
    \item 关于 $x = y$ 对称
    \item 圆、椭圆、球、椭球等的形心
    \item \[ f''_{xy}(x, y) \mathrm{d} y = \mathrm{d} f'_x(x, y) \]
\end{itemize}

\section{线面积分}

\subsection{人名公式}

\begin{itemize}
    \item 格林公式 \[ \int_\Omega \left(\frac{\partial Q}{\partial x} - \frac{\partial P}{\partial y}\right) \mathrm{d}x \mathrm{d}y = \int_{\partial \Omega} P \mathrm{d}x + Q \mathrm{d}y \]
    \item 高斯公式 \[ \int_\Omega \left(\frac{\partial P}{\partial x} + \frac{\partial Q}{\partial y} + \frac{\partial R}{\partial z}\right) \mathrm{d}x \mathrm{d}y \mathrm{d}z = \int_{\partial \Omega} P \mathrm{d}y \mathrm{d}z + Q \mathrm{d}z \mathrm{d}x + R \mathrm{d}x \mathrm{d}y \]
\end{itemize}

\subsection{二重积分的分部积分}

\begin{align*}
    \iint_D u \frac{\partial v}{\partial x} \mathrm{d} x \mathrm{d} y =                                                          & \oint_{\partial D} u v \mathrm{d} y - \iint_D v \frac{\partial u}{\partial x} \mathrm{d} x \mathrm{d} y                                                                    \\
    \iint_D u \frac{\partial v}{\partial y} \mathrm{d} x \mathrm{d} y =                                                          & - \oint_{\partial D} u v \mathrm{d} x - \iint_D v \frac{\partial u}{\partial y} \mathrm{d} x \mathrm{d} y                                                                  \\
    \Rightarrow \iint_D \left(x \frac{\partial f}{\partial x} + y \frac{\partial f}{\partial y}\right) \mathrm{d}x \mathrm{d}y = & \frac{1}{2} \iint_D \left[f'_x \frac{\partial (x^2 + y^2)}{\partial x} + f'_y \frac{\partial (x^2 + y^2)}{\partial y} \mathrm{d}x \mathrm{d}y\right]                       \\
    \xlongequal{\text{分部积分}}                                                                                                     & \frac{1}{2} \oint_{\partial D} (x^2 + y^2) f'_x \mathrm{d}y - (x^2 + y^2) f'_y \mathrm{d}x - \frac{1}{2} \iint_D (x^2 + y^2) (f''_{xx} + f''_{yy}) \mathrm{d}x \mathrm{d}y \\
    \xlongequal{x^2 + y^2 = a^2}                                                                                                 & \frac{a^2}{2} \oint_{\partial D} f'_x \mathrm{d}y - f'_y \mathrm{d}x - \frac{1}{2} \iint_D (x^2 + y^2) (f''_{xx} + f''_{yy}) \mathrm{d}x \mathrm{d}y                       \\
    =                                                                                                                            & \frac{a^2}{2} \iint_D (f''_{xx} + f''_{yy}) \mathrm{d}x \mathrm{d}y - \frac{1}{2} \iint_D (x^2 + y^2) (f''_{xx} + f''_{yy}) \mathrm{d}x \mathrm{d}y                        \\
    \xlongequal{f''_{xx} + f''_{yy} = \cdots}                                                                                    & \cdots \text{「最开始可以从极坐标开始做」}
\end{align*}

\section{级数}

\subsection{Leibniz 判别法}

$a_n$ 单调递减趋于零,则 $\sum_{n = 0}^{\infty} (-1)^n a_n$ 收敛

\subsection{幂级数收敛半径}

\[ \sum_{n = 0}^{\infty} a_n x^n \Rightarrow r = \lim_{n \to \infty} \frac{1}{\sqrt[n]{|a_n|}} = \lim_{n \to \infty} \frac{a_n}{a_{n+1}} \]

\begin{itemize}
    \item 和函数求导凑微分方程
\end{itemize}

\end{document}