\documentclass{article}

\usepackage[letterpaper,top=2cm,bottom=2cm,left=3cm,right=3cm,marginparwidth=1.75cm]{geometry}

\usepackage{ctex} % [UTF8]

\setCJKmainfont{Noto Serif CJK SC}
\setCJKsansfont{Noto Sans CJK SC}
\setCJKmonofont{Noto Sans Mono CJK SC}

\usepackage{mathrsfs}
\usepackage{physics}

\newcommand{\rme}{\mathrm{e}}
\newcommand{\rmi}{\mathrm{i}}

\title{数学物理方法复习}
\author{桜井\ 雪子}
\date{}

\begin{document}
\maketitle

\section{复变函数}

\subsection{复变函数}

\begin{itemize}
    \item 区分:$\arg$ (主值)和 $\mathrm{Arg}$
    \item $\ln z=\ln\|z\|+\mathrm{i}\ \mathrm{Arg}\ z$
    \item C - R 条件:$\begin{cases}\frac{\partial u}{\partial x}=\frac{\partial v}{\partial y},& \frac{\partial u}{\partial y}=-\frac{\partial v}{\partial x}\\ \frac{\partial u}{\partial \rho}=\frac{1}{\rho}\frac{\partial v}{\partial \varphi},& \frac{1}{\rho}\frac{\partial u}{\partial\varphi}=-\frac{\partial v}{\partial \rho}\end{cases}$
    \item 给实部求虚部
    \item Cauchy 定理:$\oint_lf(z)\mathrm{d}z=0$
    \item 一个重要的积分:$\frac{1}{2\pi\mathrm{i}}\oint_l\frac{1}{z-\alpha}\mathrm{d}z\begin{cases}0, l \text{ 不包围 }\alpha\\ 1, l\text{ 包围 }\alpha\end{cases}; \frac{1}{2\pi\mathrm{i}}\oint_l(z-\alpha)^n\mathrm{d}z=0, n\neq-1$
    \item Cauchy 公式:$f(z)=\frac{1}{2\pi\mathrm{i}}\oint_l\frac{f(\zeta)}{\zeta-z}\mathrm{d}\zeta, f^{(n)}(z)=\frac{n!}{2\pi\mathrm{i}}\oint_l\frac{f(\zeta)}{(\zeta-z)^{n+1}}\mathrm{d}z$
\end{itemize}

\subsection{幂级数展开}

\begin{itemize}
    \item 收敛半径:$R=\lim_{k\to\infty}\frac{\|a_k\|}{\|a_{k+1}\|}=\lim_{k\to\infty}\frac{1}{\sqrt[k]{\|a_k\|}}$
    \item Taylor series: $f(z)=\sum^\infty_{k=0}a_k(z-z_0)^k, a_k=\frac{f^{(k)}(z_0)}{k!}$
    \item 上手实操
\end{itemize}

\subsection{留数定理}

\begin{itemize}
    \item 留数定理:$\oint_lf(z)\mathrm{d}z=2\pi\mathrm{i}\sum^n_{j=1}\mathrm{Res}f(b_j)$
    \item 求留数:$\begin{cases}\text{单极点}:&\mathrm{Res}f(z_0)=a_1=\lim_{z\to z_0}(z-z_0)f(z)\\ m \text{阶极点}:&\mathrm{Res}f(z_0)=a_1=\lim_{z\to z_0}\frac{1}{(m-1)!}\{\frac{\mathrm{d}^{m-1}}{\mathrm{d}z^{m-1}}[(z-z_0)^mf(z)]\}\end{cases}$
    \item 计算实变函数定积分:\begin{enumerate}
              \item $\int^{\infty}_{-\infty}f(x)\mathrm{d}x=2\pi\mathrm{i}\{f(z) \text{在上半平面所有奇点的留数之和}\}$
              \item $\int^{2\pi}_0R(\cos\theta,\sin\theta)\mathrm{d}\theta=\oint_{\|z\|=1}R(\frac{z+z^{-1}}{2}, \frac{z-z^{-1}}{2\mathrm{i}})\frac{\mathrm{d}z}{\mathrm{i}z}$
              \item $\begin{cases}偶函数:\int^\infty_0F(x)\cos mx\mathrm{d}x=\pi\mathrm{i} \times \{F(z)\mathrm{e}^{\mathrm{i}mz}\text{在上半平面所有奇点的留数之和}\}\\ 奇函数:\int^\infty_0G(x)\sin mx\mathrm{d}x=\pi \times \{G(z)\mathrm{e}^{\mathrm{i}mz}\text{在上半平面所有奇点的留数之和}\}\end{cases}$
              \item 实轴上有单极点:$+\pi\mathrm{i}\cdot$ 实轴上奇点留数之和
          \end{enumerate}
\end{itemize}

\subsection{积分变换}

\begin{enumerate}
    \item 傅里叶级数(周期为$2l$)\begin{itemize}
              \item $$f(x)=a_0+\sum^\infty_{k=1}(a_k\cos \frac{k\pi x}{l}+b_k\sin\frac{k\pi x}{l})$$ $$a_0=\frac{1}{2l}\int^l_{-l}f(\xi)\mathrm{d}\xi,a_k=\frac{1}{l}\int^l_{-l}f(\xi)\cos\frac{k\pi x}{l}\mathrm{d}\xi,b_k=\frac{1}{l}\int^l_{-l}f(\xi)\sin\frac{k\pi x}{l}\mathrm{d}\xi$$
              \item 奇函数,偶函数
              \item $$f(x)=\sum^\infty_{k=-\infty}c_k\mathrm{e}^{\mathrm{i}\frac{k\pi x}{l}},c_k=\frac{1}{2l}\int^\infty_{-\infty}f(\xi)\mathrm{e}^{-\mathrm{i}\frac{k\pi x}{l}}\mathrm{d}\xi$$
          \end{itemize}
    \item 傅里叶变换 \begin{itemize}
              \item $$f(x)=\int^\infty_0A(\omega)\cos(\omega x)\mathrm{d}\omega+\int^\infty_0B(\omega)\sin(\omega x)\mathrm{d}\omega$$ $$A(\omega)=\frac{1}{\pi}\int^\infty_{-\infty}=f(\xi)\cos(\omega x)\mathrm{d}\xi,B(\omega)=\frac{1}{\pi}\int^\infty_{-\infty}=f(\xi)\sin(\omega x)\mathrm{d}\xi$$
              \item 奇函数,偶函数
              \item $$f(x)=\int^\infty_{-\infty}F(\omega)\mathrm{e}^{\mathrm{i}\omega x}\mathrm{d}\omega,F(\omega)=\frac{1}{2\pi}\int^\infty_{-\infty}f(x)\mathrm{e}^{-\mathrm{i}\omega x}\mathrm{d}x$$
          \end{itemize}
    \item $\delta$ 函数:$\mathscr{F}[\delta(x)]=\frac{1}{2\pi},\mathscr{F}[\mathrm{H}(x)]=\frac{1}{2}\delta(\omega)-\frac{\mathrm{i}}{2\pi}\mathscr{P}\frac{1}{\omega}$
    \item 拉普拉斯变换 $$\bar f(p)=\int^\infty_0f(t)\mathrm{e}^{-pt}\mathrm{d}t$$
\end{enumerate}
最重要的公式:
$$\mathscr{L}[\mathrm{e}^{-\lambda t}f(t)]=\bar f(p+\lambda), \mathscr{L}[f(t-t_0)]=\mathrm{e}^{-pt_0}\bar{f}(p)$$
注意两种积分变换的卷积的不同:积分限、系数。

\section{数理方程}

\subsection{数理方程入门}

\subsubsection{三类数学物理方程}

\begin{itemize}
    \item 振动方程:$u_{tt}-a^2\nabla^2u=0$
    \item 扩散方程:$u_{t}-a^2\nabla^2u=0$
    \item Laplace equation: $\nabla^2u=0$
\end{itemize}

\subsubsection{三类定解条件}

一定要自己回推导定解条件!

\subsubsection{D 'Alembert formula}

$$ u*{tt}-a^2u*{xx}=0, u(x, t)\|_{t=0}=\varphi(x), u_t(x, t)\|_{t=0}=\psi(x) $$
$$ u=\frac{1}{2}[\varphi(x+at)+\varphi(x-at)]+\frac{1}{2a}\int^{x+at}_{x-at}\psi(\xi)\mathrm{d}\xi $$
类似问题的转化:
\begin{itemize}
    \item 对于半无限长问题,对初始条件进行奇偶延拓
    \item 对微分算符进行因式分解
    \item 低阶问题转化成 D 'Alembert formula 问题形式
\end{itemize}

\subsection{分离变数法}

首先就是要注意 $n=0$ 可不可取吧……
\begin{itemize}
    \item 两个初始条件都是 0 阶就不可取,其他好像都可取
    \item 柱坐标沿轴线不变的 Laplace equation 通解:$$u(\rho, \varphi)=C_0+D_0\ln\rho+\sum^\infty_{m=1}\rho^m(A_m\cos m\varphi+B_m\sin m\varphi)+\sum^\infty_{m=1}\rho^{-m}(C_m\cos m\varphi+D_m\sin m\varphi)$$
\end{itemize}

\subsubsection{特殊问题:}

\begin{enumerate}
    \item 方程非齐次、边界条件齐次 \begin{itemize}
              \item 模仿常数变易法
              \item 这时候边界条件不为 0 时,可以分解方程为一个是方程非齐次、边界条件为 0,一个是方程齐次、边界条件不为 0
              \item 冲量定理法
          \end{itemize}
    \item 方程齐次,边界条件非齐次 \begin{itemize}
              \item 先拆出来一个 $w(x, t)$ 把非齐次的边界条件顶住,得到一个关于 $v(x, t)$ 的边界条件其次的问题
              \item 这个问题的方程可能就是非齐次的了,转化成了上面那种情况,可能需要再拆一次方程
          \end{itemize}
    \item 泊松方程 \begin{itemize}
              \item 先找一个特解把非齐次项顶住,得到一个 Laplace equation 问题
              \item 再次转化为第一种情况的问题
          \end{itemize}
\end{enumerate}

\subsection{二阶常微分方程级数解法}

\subsubsection{几个方程}

\begin{itemize}
    \item 球函数方程:$-{\frac {1}{\sin \theta }}{\frac {\partial }{\partial \theta }}\sin \theta {\frac {\partial Y}{\partial \theta }}-{\frac {1}{\sin ^{2}\theta }}{\frac {\partial ^{2}Y}{\partial \varphi ^{2}}}=l(l+1)Y$
    \item 连带勒让德方程:$\frac{\mathrm{d}}{\mathrm{d}x}[(1-x^2)\frac{\mathrm{d}\Theta}{\mathrm{d}x}]+[l(l+1)-\frac{m^2}{1-x^2}]\Theta=0$
    \item 勒让德方程:$\frac{\mathrm{d}}{\mathrm{d}x}[(1-x^2)\frac{\mathrm{d}\Theta}{\mathrm{d}x}]+l(l+1)\Theta=0$
    \item 贝塞尔方程:$x^2\frac{\mathrm{d}^2R}{\mathrm{d}x^2}+x\frac{\mathrm{d}R}{\mathrm{d}x}+(x^2-m^2)R=0$,虚宗量贝塞尔方程:$x^2\frac{\mathrm{d}^2R}{\mathrm{d}x^2}+x\frac{\mathrm{d}R}{\mathrm{d}x}-(x^2+m^2)R=0$
    \item 球贝塞尔方程:$r^2\frac{\mathrm{d}^2R}{\mathrm{d}r^2}+2r\frac{\mathrm{d}R}{\mathrm{d}r}+[k^2r^2-l(l+1)]R=0$,$l$ 阶球贝塞尔方程化为 $l+1/2$ 阶贝塞尔方程
\end{itemize}
关系:
\begin{itemize}
    \item 球坐标 Laplace 方程 = 欧拉型方程 + 球函数方程 = 欧拉型方程 + (连带勒让德方程 + 谐振方程)
    \item 柱坐标 Laplace 方程 = 谐振方程 +超谐振方程+ 贝塞尔方程/虚宗量贝塞尔方程
    \item 波动方程 = 超谐振方程 + 亥姆霍兹方程
    \item 输运方程 = 指数衰减方程 + 亥姆霍兹方程
    \item 球坐标亥姆霍兹方程 = $l$ 阶球贝塞尔方程 + 连带勒让德方程 + 谐振方程
    \item 柱坐标亥姆霍兹方程 = $m$ 阶贝塞尔方程 + 谐振方程 + 超谐振方程
\end{itemize}
复变函数的常微分方程:$\frac{\mathrm{d}^2w}{\mathrm{d}z^2}+p(z)\frac{\mathrm{d}w}{\mathrm{d}z}+q(z)w=0, w(z_0)=C_0, w'(z_0)=C_1$

\subsubsection{常点邻域}

\begin{itemize}
    \item 常点:$p(z)$ 和 $q(z)$ 在 $z_0$ 的邻域内解析
    \item 设 $y(x)=\sum^\infty_{k=0}a_kx^k$
\end{itemize}

\subsubsection{正则奇点邻域}

\begin{itemize}
    \item 奇点: $z_0$ 是 $p(z)$ 和 $q(z)$ 的奇点
    \item 两个线性独立解:$$w_1(z)=\sum^\infty_{k=-\infty}a_k(z-z_0)^{s_1+k}, w_2(z)=\sum^\infty_{k=-\infty}b_k(z-z_0)^{s_2+k}\text{ or }w_2(z)=Aw_1(z)\ln(z-z_0)+\sum^\infty_{k=-\infty}b_k(z-z_0)^{s_2+k}$$
    \item 正则奇点:$$p(z)=\sum^\infty_{k=-1}p_k(z-z_0)^{k}, p(z)=\sum^\infty_{k=-2}q_k(z-z_0)^k$$
    \item 正则奇点的两个线性独立解:$$w_1(z)=\sum^\infty_{k=0}a_k(z-z_0)^{s_1+k}, w_2(z)=\sum^\infty_{k=0}b_k(z-z_0)^{s_2+k}\text{ or }w_2(z)=Aw_1(z)\ln(z-z_0)+\sum^\infty_{k=0}b_k(z-z_0)^{s_2+k}$$ $$s(s-1)+sp_{-1}+q_{-2}=0, s_1>s_2$$ 若 $s_1-s_2\neq$ 整数则取第一个 $w_2(z)$,否则取第二个
\end{itemize}

\subsubsection{贝塞尔函数}

\begin{enumerate}
    \item $\nu$ 阶贝塞尔方程,$\nu$ 非整数 \begin{itemize}
              \item $\nu$ 阶贝塞尔函数:$J_\nu(x)=\sum^\infty_{k=0}(-1)^k\frac{1}{k!\Gamma(\nu+k+1)}(\frac{x}{2})^{2k+\nu}$
              \item $\nu$ 阶诺伊曼函数:$N_\nu(x)=\frac{J_\nu(x)\cos(\nu\pi)-J_{-\nu}(x)}{\sin(\nu \pi)}$
              \item 通解:$y(x)=C_1J_\nu(x)+C_2J_{-\nu}(x)$ 或 $y(x)=C_1J_\nu(x)+C_2N_\nu(x)$
          \end{itemize}
    \item 整数 $m$ 阶贝塞尔方程 \begin{itemize}
              \item $m$ 阶诺伊曼函数:$N_m(x)=\lim_{\nu\to m}N_\nu(x)=\lim_{\nu\to m}\frac{J_\nu(x)\cos(\nu\pi)-J_{-\nu}(x)}{\sin(\nu \pi)}$
              \item 通解:$y(x)=C_1J_m(x)+C_2N_m(x)$
          \end{itemize}
    \item $l+1/2$ 阶贝塞尔方程 \begin{itemize}
              \item 通解:$y(x)=C_1J_{l+1/2}(x)+C_2J_{-(l+1/2)}(x)$
              \item $J_{1/2}(x)=\sqrt{\frac{2}{\pi x}}\sin x, J_{-(1/2)}(x)=\sqrt{\frac{2}{\pi x}}\cos x$
          \end{itemize}
    \item $x=0$ 处的自然边界条件 \begin{itemize}
              \item 去掉 $N_0(x), N_m(x), N_\nu(x), J_{-\nu}(x)$,留下 $J_0(x), J_m(x), J_\nu(x)$
          \end{itemize}
    \item $\nu$ 阶虚宗量贝塞尔方程 \begin{itemize}
              \item 做变换:$x=-i\xi$
          \end{itemize}
\end{enumerate}

\subsubsection{广义傅里叶级数}

$$\int^b_ay_m(x)y_n(x)\rho(x)\mathrm{d}x=N_m^2\delta_{mn}$$
$$f(x)=\sum^\infty_{n=1}f_ny_n(x)$$
$$f_m=\frac{1}{N_m^2}\int^b_a\rho(\xi)y_m(\xi)f(\xi)\mathrm{d}\xi$$

\subsection{球函数}

\subsubsection{勒让德多项式}

\begin{enumerate}
    \item 表达式 $$P_l(x)=\sum^{[l/2]}_{k=0}(-1)^k\frac{(2l-2k)!}{k!2^l(l-k)!(l-2k)!}x^{l-2k}={\frac {1}{2^{l}}}\sum _{k=0}^{[l/2]}(-1)^{k}{\binom {l}{k}}{\binom {2l-2k}{l}}x^{l-2k}$$ $$P_{2n+1}(0)=0, P_{2n}(0)=(-1)^n\frac{(2n)!}{(2^nn!)^2}$$ $$P_l(1)=1, P_l(-1)=(-1)^l$$ $$P_{2n}(-x)=P_{2n}(x), P_{2n+1}(-x)=-P_{2n+1}(x)$$ $$P_0(x)=1, P_1(x)=x, P_2(x)=\frac{1}{2}(3x^2-1)$$
    \item 微分式 $$P_l(x)=\frac{1}{2^ll!}\frac{\mathrm{d}^l}{\mathrm{d}x^l}(x^2-1)^l$$
    \item 积分式 $$P_l(x)=\frac{1}{2\pi\mathrm{i}}\frac{1}{2^l}\oint\frac{(z^2-1)^l}{(z-x)^{l+1}}\mathrm{d}z$$
    \item 正交归一 $$\int^1_{-1}P_k(\xi)P_l(\xi)\mathrm{d}\xi=\frac{2}{2l+1}\delta_{kl}$$
    \item generalized Fourier series $$f(x)=\sum^\infty_{l=0}f_lP_l(x)$$ $$f_l=\frac{2l+1}{2}\int^1_{-1}P_l(x)f(x)\mathrm{d}x$$
    \item 母函数 $$r<R,\frac{1}{\sqrt{R^2-2rR\cos\theta+r^2}}=\sum^\infty_{l=0}\frac{r^l}{R^{l+1}}P_l(\cos\theta)$$ $$r>R,\frac{1}{\sqrt{R^2-2rR\cos\theta+r^2}}=\sum^\infty_{l=0}\frac{R^l}{r^{l+1}}P_l(\cos\theta)$$
    \item 递推关系 $$(l+1)P_{l+1}(x)-(2l+1)xP_l(x)+lP_{l-1}(x)=0$$ $$\int^1_{-1}xP_l(x)P_k(x)\mathrm{d}x=
              \begin{cases}
                  \frac{2(k+1)}{[2(k+1)+1](2k+1)}, & l=k+1 \\
                  \frac{2k}{4k^2-1},               & l=k-1
              \end{cases}$$
\end{enumerate}

\subsubsection{轴对称球坐标 Laplace equation}

\begin{itemize}
    \item 特解:$u(r, \theta)=\sum^\infty_{l=0}[A_lr^l+B_lr^{-(l+1)}]P_l(\cos\theta)$
    \item 两个衔接条件:电势连续,无面电荷密度时电位移矢量法向分量连续
    \item 电像法
\end{itemize}

$$\int^1_{-1}P_l(x)\mathrm{d}x=\int^1_{-1}P_0(x)P_l(x)\mathrm{d}x$$

\subsubsection{连带勒让德方程}

连带勒让德方程做代换:$\Theta(x)=(1-x^2)^{m/2}y(x)$,则 $y(x)$ 满足逐项微分 $m$ 次的勒让德方程

\subsubsection{连带勒让德函数}

\begin{enumerate}
    \item 表达式 $$P^m_l(x)=(1-x^2)^{m/2}P^{[m]}_l(x),\ m=0, 1, 2, \dots,l,\ l=0, 1, 2, \dots$$
    \item 微分式 $$P^m_l(x)=(1-x^2)^{m/2}P^{[m]}_l(x)=(1-x^2)^{m/2}\frac{1}{2^ll!}\frac{\mathrm{d}^{l+m}}{\mathrm{d}x^{l-m}}(x^2-1)^l$$
    \item 积分式
    \item 正交归一 $$\int^1_{-1}P^m_l(\xi)P^m_k(x)\mathrm{d}x=\frac{(l+m)!}{(l-m)!}\frac{2}{2l+1}\delta_{lk}$$
    \item generalized Fourier series $$f(x)=\sum^\infty_{l=m}f_lP^m_l(x)$$ $$f_l=\frac{2l+1}{2}\frac{(l-m)!}{(l+m)!}\int^1_{-1}P^m_l(x)f(x)\mathrm{d}x$$
\end{enumerate}

\subsubsection{一般球坐标 Laplace equation}

特解:
$$u(r, \theta, \varphi)=\sum^\infty_{l=0}\sum^l_{m=0}r^l[A^m_l\cos m\varphi+B^m_l\sin m\varphi]P^m_l(\cos\theta)+\sum^\infty_{l=0}\sum^l_{m=0}r^{-(l+1)}[C^m_l\cos m\varphi+D^m_l\sin m\varphi]P^m_l(\cos\theta)$$

\subsubsection{一般球函数}

\begin{enumerate}
    \item 表达式 \begin{itemize}
              \item $Y(\theta, \varphi)=P^m_l(\cos\theta)(A\cos m\varphi+B\sin m\varphi), l=0, 1, 2, \dots, m=0, 1, 2, \dots,l$,$2l+1$ 个
              \item 复数形式:$Y(\theta, \varphi)=P^{\|m\|}_l(\cos\theta)\mathrm{e}^{\mathrm{i}m\varphi}, l=0, 1, 2, \dots, m=0, 1, 2, \dots,l$
          \end{itemize}
    \item 正交归一 \begin{itemize}
              \item 这里说的实数形式的模说的是 $Y^m_l(\theta, \varphi)=P^m_l(\cos\theta)\cos(m\varphi)$
              \item $$\iint_sY^m_l(\theta, \varphi)Y^{n^*}_l(\theta, \varphi)\sin\theta\mathrm{d}\theta\mathrm{d}\varphi=
                        \begin{cases}\frac{(l+m)!}{(l-m)!}\frac{2\pi(1+\delta_{m0})}{2l+1}, & \text{实数形式} \\
             \frac{(l+\|m\|)!}{(l-\|m\|)!}\frac{4\pi}{2l+1},        & \text{复数形式}
                        \end{cases}$$
          \end{itemize}
    \item generalized Fourier series \begin{itemize}
              \item 实数形式
              \item 复数形式
          \end{itemize}
    \item 正交归一化球函数
\end{enumerate}

\subsection{柱函数}

\subsubsection{三类柱函数}

\begin{enumerate}
    \item 表达式 \begin{itemize}
              \item $\nu$ 阶贝塞尔函数:$J_\nu(x)=\sum^\infty_{k=0}(-1)^k\frac{1}{k!\Gamma(\nu+k+1)}(\frac{x}{2})^{2k+\nu}$
              \item $\nu$ 阶诺伊曼函数:$N_\nu(x)=\frac{J_\nu(x)\cos(\nu\pi)-J_{-\nu}(x)}{\sin(\nu \pi)}$
              \item 两种汉克尔函数:$H^{(1, 2)}(x)=J_\nu(x)(+, -)\mathrm{i}N_\nu(x)$
          \end{itemize}
    \item 自然边界条件 \begin{itemize}
              \item $x=0$(柱内):去掉 $N_0(x), N_m(x), N_\nu(x), J_{-\nu}(x)$,留下 $J_0(x), J_m(x), J_\nu(x)$
              \item $x=\infty$(柱外):保留 $J_\nu(x)和N_\nu(x)$ 或 $H^{(1)}(x)和H^{(2)}(x)$
          \end{itemize}
    \item 递推关系 $$Z_{\nu-1}(x)-Z_{\nu+1}(x)=2Z'_\nu(x)$$ $$Z_{\nu-1}(x)+Z_{\nu+1}(x)=2\nu\frac{Z_\nu(x)}{x}$$
    \item 计算积分:$$\frac{\mathrm{d}}{\mathrm{d}x}[x^\nu J_\nu(x)]=x^\nu J_{\nu-1}(x)$$ $$\frac{\mathrm{d}}{\mathrm{d}x}[x^{-\nu}J_\nu(x)]=-x^{-\nu}J_{\nu+1}(x)$$
\end{enumerate}

\subsubsection{贝塞尔方程}

关于通解类型的选择:
\begin{itemize}
    \item 柱侧的齐次边界条件:$\mu\geq 0$
    \item 上下底面的齐次边界条件:$\mu\leq 0$
    \item 原因:$\mu<0$ 时 $R(\rho)$ 恒不为 0
\end{itemize}
性质:
\begin{enumerate}
    \item 三类齐次边界条件:$[\alpha\frac{\mathrm{d}R(\rho)}{\mathrm{d}\rho}+\beta R(\rho)]\|_{\rho=a}=0$ \begin{itemize}
              \item $R(\rho)\|_{\rho=a}=J_m(\sqrt{\mu}\rho)\|_{\rho=a}=0: \sqrt{\mu^{(m)}_n}a=x^{(m)}_n,\mu^{(m)}_n=(\frac{x^{(m)}_n}{a})^2, R(\rho)=J_m(\frac{x^{(m)}_n}{a}\rho)$
              \item $R'(\rho)\|_{\rho=a}=J_m'(\sqrt{\mu}\rho)\|_{\rho=a}=0: \sqrt{\mu^{(m)}_n}a=\tilde x^{(m)}_n,\mu^{(m)}_n=(\frac{\tilde x^{(m)}_n}{a})^2, R(\rho)=J_m(\frac{\tilde x^{(m)}_n}{a}\rho)$
              \item $[R(\rho)+HR'(\rho)]\|_{\rho=a}=[J_m(\sqrt{\mu}\rho)+H\sqrt{\mu}J_m'(\sqrt{\mu}\rho)]\|_{\rho=a}=0$,利用递推关系导出:$J_m'(x)=m\frac{J_m(x)}{x}-J_{m+1}(x)$,进而有 $J_m(\sqrt\mu a)=\frac{H\sqrt\mu J_{m+1}(\sqrt\mu a)}{1+Hm/a}$,根为 $\hat x^{(m)}_n$,则 $\mu^{(m)}_n=(\frac{\hat x^{(m)}}{a})^2$
          \end{itemize}
    \item 正交 $$\int^a_0J_m(\sqrt{\mu_n}\rho)J_m(\sqrt{\mu_l}\rho)\rho\mathrm{d}\rho=0, n\neq l$$
    \item 三类边界条件下的模 $$\int^a_0[J_m(\sqrt{\mu^{(m)}_n}\rho)]^2\rho\mathrm{d}\rho=[N^{(m)}_n]^2=\frac{1}{2}(a^2-\frac{m^2}{\mu^{(m)}_n})[J_m(\sqrt{\mu^{(m)}_n}a)]^2+\frac{1}{2}a^2[J'_m(\sqrt{\mu^{(m)}_n}a)]^2$$ \begin{itemize}
              \item 第一类边界条件下的:$[N^{(m)}_n]^2=\frac{1}{2}a^2[J_{m+1}(\sqrt{\mu^{(m)}_n}a)]^2$
              \item 第二类边界条件下的:$[N^{(m)}_n]^2=\frac 1 2(a^2-\frac{m^2}{\mu^{(m)}_n})[J_m(\sqrt{\mu^{(m)}_n}a)]^2$
              \item 第三类边界条件下的:$[N^{(m)}_n]^2=\frac 1 2(a^2-\frac{m^2}{\mu^{(m)}_n}+\frac{a^2}{\mu^{(m)}_nH})[J_m(\sqrt{\mu^{(m)}_n}a)]^2$
          \end{itemize}
    \item generalized Fourier series $$f(x)=\sum^\infty_{n=1}f_nJ_m(\sqrt{\mu^{(m)}_n}\rho)$$ $$f_n=\frac{1}{[N^{(m)}_n]^2}\int^a_0f(\rho)J_m(\sqrt{\mu^{(m)}_n}\rho)\rho\mathrm{d}\rho$$
    \item 母函数 $$\mathrm{e}^{\frac{1}{2}x(z-\frac{1}{z})}=\sum^\infty_{m=-\infty}J_m(x)z^m$$
    \item 加法公式 $$J_m(a+b)=\sum^\infty_{k=-\infty}J_k(a)J_{m-k}(b)$$
\end{enumerate}
讲义上求解了一个问题,我觉得这部分还是要再看看书

\subsubsection{虚宗量贝塞尔方程}

\subsubsection{球贝塞尔方程}

$x=kr, R=\sqrt{\frac{\pi}{2x}}y(x)$:

$$x^2\frac{\mathrm{d}^2y}{\mathrm{d}x^2}+2x\frac{\mathrm{d}y}{\mathrm{d}x}+[x^2-(l+\frac{1}{2})^2]y=0$$

\begin{enumerate}
    \item 通解表达式 $$y(x)=A_lj_l(x)+B_ln_l(x), j_l(x)=\sqrt{\frac{\pi}{2x}}J_{l+1/2}(x), n_l(x)=\sqrt{\frac{\pi}{2x}}N_{l+1/2}(x)$$
    \item 递推关系
    \item 初等函数表示
    \item $x\to0,x\to\infty$ 的行为:$x\to 0, j_0(0)=1, j_l(0)=0(l为自然数), n_l(0)\to\infty$
    \item 本征值问题与 generalized Fourier series:模可以直接转化成\ 贝塞尔函数\ 的模
\end{enumerate}

\end{document}