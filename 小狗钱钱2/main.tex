\documentclass{ctexart}

\usepackage[a4paper,scale=0.8]{geometry}

\usepackage[colorlinks]{hyperref}

\title{Notes on 《小狗钱钱 2》}
\author{桜井\ 雪子}
\date{}

\begin{document}

\maketitle
\tableofcontents

《小狗钱钱》是一本金融科普读物,而它的续作《小狗钱钱 2》则是一本关于道德的书。这本书内容非常精彩,但是应该对其中包含的基督教成分提高警惕。

\section{奖学金}

古代硬币有正反两面,其中一面是国王的头像,这代表着世俗的权力,另一面则是上帝的形象。

\section{白色石头}

我们应该首先看到积极的方面。

你应该更自制、更谦虚,特别是要更加成熟一点儿,这首先就表现在说话不要总是冒冒失失的。……这不能因人而异。虽然没人要求你一定要那样,但你仍然应该总是保持礼貌。因为不友善一直就是缺乏教养和内心脆弱的标志,而且也显得很愚蠢。……尊重他人和对人有礼貌是成功的重要因素。如果有人不喜欢你,很大程度上是因为你表现得不够友好。这会损害一个人的声誉,所以你不能那么做。另外,彬彬有礼也可以让以前并不信任你的人接纳你。

你不能总是抱怨命运的不公。在我们的生活中不是所有的事情都是公平的,不公平的情况很常见。但这并不意味着你就应该放弃,因为成功并不取决于你是否受到了公平的对待。……我和你一样也很厌恶不公平。虽然我一直能够公平地对待他人,然而我却并没有能力阻止别人不公平地对待我自己。所以,如果真的有人对我不公,我不会立即发怒,然后选择放弃。

永远不要在背后说别人的坏话。……在背后说别人坏话从根本上是不好的。……嘲笑他人是一件令人遗憾的事情。当你把目光都集中在别人的错误和缺点上时,就自然会忽略他们身上闪光的一面。而一个人只有具备了审视美好事物的能力,世界对他来说才会变得更加美好。除此之外,在背后议论他人也会给你身边的人留下不好的印象。因为人们会认为,你既然能够在背后说别人的坏话,就能在其他场合说他们的坏话。……我觉得,你无论如何都不可能只取笑自己不喜欢的人吧。

要是真的遇到不好的事情,我又该怎么做呢?……那就最好什么都不说。

硬币的一面印有上帝的形象,它表示在这个世界上的人并不是孤单的,每个人都为他人担负着职责。

金钱是生活中重要的东西,但却不是生活的全部。金钱或者一切可以用来消费的东西,对于我来说都像是甜甜圈外面那个圆圈。……我们只能看到那个甜甜圈的圆圈,但它其实还有其他的组成部分。……一个甜甜圈是由外面的圆圈和中间的圆孔共同组成的。……一个圆孔确实是空的。但当你掉进洞里的时候,你才会体验到这个空的孔是一个实实在在的存在。……甜甜圈中间的圆孔代表着人类的内心,可是这内心本身却是无形的。许多人并不关心自己的内心,就是因为看不到它。对于他们来说,只有看得见的成功才是重要的。但你如果想要变得幸福,就不能只重视物质上的成功,还必须培养自己具有优秀的内心。……优秀的内心就是那些人们不能用金钱买到的东西,但缺少了它们人们就无法获得幸福。这个内心就是你的品格,谦虚、感恩、尊敬老人及同情弱者都属于一个人的优秀品格。培养优秀品格的前提是,你应该认识到自己并不是孤零零地生活着,而是与他人同在。因此,你应该助人为乐,要通过自己的努力使得世界在他人眼中变得更加
美好。……没有圆孔的甜甜圈也就不再是甜甜圈了。同样的道理,缺少优秀品格的人其本质也是空虚的。只考虑金钱的人是不会真正幸福的。……绝不能忽视圆圈,否则的话内心也无法彰显出来。完满而幸福的人都是两者兼备的。

\begin{quote}
    赐他一块白石,石上写着新名。
\end{quote}

人类的内心指的就是他们的品格。而其他所有的东西都是外在的。诸如谦虚、感恩、尊重他人、有同情心、助人为乐及使他人感到快乐等都是重要的品格特质。我们不仅要学习外部的知识,还必须培养自己的优秀品格。……人们必须懂得每个人都要担负对这个世界的职责。如果不那样做的话就只会是一事无成。只有当你意识到并且履行了那份职责,你才能找到属于你自己的那块白色石头。那块石头上写有你新的名字,而这名字就是一个全新的幸福生活的标志。

\section{放大镜}

职责并非独自而来,它其实已经随身携带了你所急需的帮助。

助人为乐是生活中最美好的事。……无偿地使他人快乐或为他人付出。没有什么比这更美好了。……不仅仅意味着要给予他人欢乐,更要在他们遇到困难之时去帮助他们。

在你的生命中没有什么比寻找白色石头更重要的事情了。如果不去寻找,当然可以避免很多危险与问题,可你也将无法获得你本来能够得到的幸福。

奇迹到底是什么呢?奇迹其实就是那些成年人无法理解而又被人们称为幸福的东西。人们一旦拥有巨大的幸福,就会说那是一个奇迹。

人们通常并不把自己已经理解的事情看作是奇迹。但是每一次的帮助却都是奇迹。唯有依靠这些帮助,我们才能完成本来一个人无法完成的事。我们不理解某些事情,并不是因为它们是超越自然属性的东西,而是因为我们还不清楚如何解释它们。

时不我待。要试着享受每一种环境。无论在什么情况下都要试着去发现那些积极的东西。

只想一味尝试的人终归一事无成。

\section{论文}

失败者总是无法成功,那是因为他们习惯于放弃;胜利者最终总是成功,则是因为他们永不放弃。……在我们的脑袋里面住着巨人与侏儒两种灵魂。我们总是能听到他们的话语。侏儒总是对我们小声说:‘放弃吧,你无法成功。’而巨人则鼓励我们不要放弃。

……自己也有正反两个方面。而只有当你将自己的正反两面都注意到的时候,你才有机会体验到真正的幸福。……首先看一看你那外在的一面吧。你已经知道了,为了满足物质上的欲望并在这个世界上继续生存,你就必须得到金钱。而那内在的一面却能让你意识到上帝的力量,你必须学会心存感恩和乐于助人。只要去帮助他人,神性就会在你内心生长。……这内外两个方面都属于你的本质,两者都十分重要,它们也都能使你最终满足自身的需求。因此,千万不要忽视这两个方面,否则的话你就不可能幸福。

每当有人想寻找属于自己的白色石头的时候,那些拥有黑色石头的人就会试图阻止他们。

\section{前往加利福尼亚}

吹牛会让一个人变得不再讨人喜欢,即使他所说的并非毫无根据。

\section{寄宿学校}

害怕并不糟糕,糟糕的是因为害怕的缘故就不去做我们本应做的事情。

我总是尽量让自己看到他人的长处。对某人作出判断是很容易的一件事,但谁又能知道,如果我们生长在那种类型的家庭里,是不是也会变成他那个样子呢?

我认为找出不同事物的相同点与不同点很重要。虽然我们国籍不同,但却都是人类,因此在本质上是一样的。……人不可能只有一个归属,而是有两个或者更多。……如果我们过于重视它们的不同之处,争执就是不可避免的。但要是我们能意识到所有的宗教都具有相同的目标,取而代之的则是宽容和团结。……我认为世上并没有鼓励暴力和战争的宗教。所有宗教都规劝人们要爱好和平、有同情心及助人为乐。但总是有人歪曲和滥用宗教教义,并以此来为他们的暴力与战争行为辩护。

\section{好老师}

一门课开始的时候总会有些协议与约定,我们称其为承诺。任何人都不应该违反承诺。……我们必须遵守承诺。迟到这件事不但会妨碍我们所有人,还意味着迟到者不尊重大家的时间。……与他人的约定也一样。这种约定从某种意义上来讲也是承诺。如果你来晚了,这就表示你的言行不一,对方就不会再信任你,他会认为你不尊重他的时间。

看来有些事情在每个国家都一样:哪里都有可爱的人,当然也有不那么可爱的人。不可能所有人都是友好的。这不以人的意志为转移。有些人连自己都不喜欢,因而也不会去喜欢别人。

不受不公平之事的影响,这难道只意味着要去承担职责而不是别的什么东西吗?……是的,但比这还要更多。我很愿意为你解释一下。你知道吗,当事情并不像预想中那样顺利的时候,很多人会怎么做?他们通常都会简单地将其归因于环境或者说这是别人的责任。但即使有这样或那样的借口,他们最终还是会失去权利。……如果你不去做某件事的原因不是出于自己的意愿,而是别人要你这么做,那么你就已经失去了自己的权利。……你在把责任推脱给别人的同时,也把权利转让给了他。“答案”这个词就隐藏在“职责”当中。

可以这么说,橘子本身并不在乎你怎么做,它总是提供给你橘子汁。它的答案就是橘子汁。它总是按照自己的意愿给出答案,无论人们怎么做都一样。它绝不会说:“因为有人敲打我,所以这次榨出的是水。

你首先不应该受到不公平之事的影响。生活本身就是不公平的。……没有人是完全相同的,因此对所有人来说演讲比赛的起始点也不可能一模一样。生活中的所有事情都是如此。……因此,人们不应该受到不公平现象的影响,而必须勇敢地承担职责并把精力集中在自身优势上。

\section{危险}

\section{好老师的秘密}

您为什么不每天把所有卡片都看一遍呢?……我当然也可以那么做。可遗憾的是大脑总是一个单行道,它只能集中精力于一件事物。

每当我们想要集中精力去帮助他人时,就会把‘帮助’这个念头存储进我们的潜意识里。这样的话,‘帮助’就不再仅仅是一种自然而然的行为。只要对某种知识进行认真研究和具体实践,我们就总是能够很好地掌握它。所以即使我第二天有意识地关注下一张卡片,前一天的那条准则也会在不经意间得到强化。通过这种方法,久而久之那些准则就成为了我生活的一部分。

你必须学会勇于承担自己的职责。也就是说,你要将注意力更多地集中在自己的能力上。与因他人的成绩感到灰心丧气相比,你更应该想办法改进自己的成绩。

\begin{itemize}
    \item 我有一个强烈的愿望,希望其他人能够像我一样生活美好而幸福。
    \item 我不会伤害任何人。我克制自己,不介入任何争端。
    \item 我谦虚有礼,尊重他人。我并不是永远正确。
\end{itemize}


如果有人想给你们一些东西,而你们却不接受,那么这些东西又是属于谁的呢?……当然还属于那个给予者。……愤怒和仇恨也是同样的道理。如果我们不接受,它们就还是属于原来的人所有。

友好的力量比暴力更强大。

\begin{itemize}
    \item 遇事我能自我抉择。我能自行判断对某种情况应该作何反应。
    \item 我不受不公平之事的影响,而是将注意力集中在我能做的事情、我知道的知识和我拥有的东西之上。
    \item 我把责任推托给别人的同时,也把相应的权利转交给了对方。
\end{itemize}

\section{7 条准则}

演讲内容应该是一个人的真实经历,这一点是决定性的。因为只有这样,听众才会感到与演讲者非常亲近。

\begin{itemize}
    \item 我只称赞他人。如果确实无法称赞他人,那就最好什么都不说。
    \item 我尽量不批评他人。如果不得不批评,也要用非常礼貌和友善的方式。
    \item 我将注意力集中在他人的优点和闪光点上。
\end{itemize}

当你只看到光明的时候,你就会成为光明;而当你只看到垃圾的时候,你就会成为垃圾。

当你不确定怎样才能遵守这条准则的时候,就想一想那些榜样,然后问自己:“如果是这个人的话,他会怎么做呢?”这样你就能立即回到正确的道路上来。

\begin{itemize}
    \item 我祝愿自己遇到过的所有人都能一切顺利。
    \item 我送给某人礼物,因为我想表达自己对他的好感。
    \item 最美好的事情莫过于帮助他人。我总是在想自己能够帮助谁,没有什么比这更令人快乐。
\end{itemize}

当我心情不好的时候,我会去想一想此时能够帮助谁或者可以给谁带去欢乐。要是这样的话,自己也会感觉好一些。

\begin{itemize}
    \item 我总是心怀感恩,哪怕是对看似寻常的事情。
    \item 即便遇到了困难,我还是会关注值得感激的事物。
    \item 我对身边的人都充满感激之情,并非常享受和他们共度的美好时光。
\end{itemize}

生活中有许多我们无法理解的事情,有时候想得再多也没有用,但我们必须振作起来把生活交给我们的事情做到最好。

\begin{itemize}
    \item 如果我骄傲自满,那无异于说自己不必再学任何东西了。因此我应该保持谦恭好学的态度。
    \item 我不仅要阅读好的书籍、写成功日记和知识笔记,还要尽可能多地向他人学习。
    \item 我不拿自己和别人比较,而是尽我所能做到最好。
\end{itemize}

我总是不断地学习,这是因为我要成为我有能力成为的那种人。

要是在某一天我没有学到些什么,这天对于我来说就是不完整的一天。有位音乐家曾说过:“如果有一天你不再相信自己可以进步,从此以后你演奏的就只是同一首乐曲。”这句话的意思是:只要能坚持不懈地学习,生活就总是多姿多彩的。

\begin{itemize}
    \item 我能否成功总是取决于自身培养出的习惯。
    \item 如果我是一个非常自律的人,我就能比那些虽有天赋但却懒惰散漫的人获得更多的成功。
    \item 我总是很守时。我信守为他人作出的承诺。
\end{itemize}

我究竟有能力成为哪种人呢?……我也无法用几句话就解释清楚。每个人都必须自己去寻找答案。

\section{演讲比赛}

在生活中我们总是爱抱怨,总会想生活是多么不公平呀。或者更为糟糕的是:我们觉得自己的存在没有价值。但其实对任何人来说都一样:没有人会失去自身的价值。我们总是为了那些喜欢我们的人无私地存在。人类的价值并不取决于能够达到什么目标或是拥有多少财富,而是取决于你到底是一个什么样的人。

\section{事件}

\section{回家}

激烈争执的出现频率之高真是令人惊奇,而本来这是很容易就能避免的。很多时候只需要斜眼一瞥就足够引起一次争吵了。有时候我们觉得一个人高傲,但这只不过是他缺乏安全感的表现罢了。

有一种东西叫直觉。我们可以将其理解成一种认为我们应该做某事或者认为某事不正确的可靠感觉。这种感觉,或者叫内在声音,是每个人都拥有的,但大多数人却都没想到去听一听。

一旦我们坐船到达对岸,也就不再需要那只船了。……要是我还想回去,也就是说回到我先前出发的岸边去,又该怎么办呢?……这在生活中是不可能的。生活始终向前行进。你可以暂时休息一下或者安静地聚集新的能量,但你永远也无法再回去了。

\section{道别}

与其为那些已经不在的事物悲伤,倒不如为那些令我们感到愉快的时光而高兴。

永远不要说“永远不”。

\begin{quote}
    友好亲和、勇于承担、善待他人、帮助给予、感恩之心、勤学不辍、值得信赖
\end{quote}

\end{document}
