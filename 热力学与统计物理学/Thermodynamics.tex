区分:状态参量、态函数。

\section{热力学基本方程}

\subsection{导论}

三个重要参数:$\left.\pdv{V}{T}\right|_p \left.\pdv{T}{p}\right|_V \left.\pdv{p}{V}\right|_T = - 1 \Rightarrow \alpha = \beta \kappa_T p$

\begin{itemize}
    \item 等压膨胀:$\alpha = \frac{1}{V} \left.\pdv{V}{T}\right|_p$
    \item 等容压缩:$\beta = \frac{1}{p} \left.\pdv{p}{T}\right|_V$
    \item 等温压缩:$\kappa_T = - \frac{1}{V} \left.\pdv{V}{p}\right|_T$
\end{itemize}

\begin{framed}
    理想气体三个系数。
    \[
        p V = \nu R T
    \]
    \[
        \alpha = \frac{1}{V} \left.\pdv{V}{T}\right|_p = \frac{1}{T},\ \beta = \frac{1}{p} \left.\pdv{p}{T}\right|_V = \frac{1}{T},\ \kappa_T = - \frac{1}{V} \left.\pdv{V}{p}\right|_T = \frac{1}{p}
    \]
\end{framed}

\begin{framed}
    物态方程 $p (V_m - b) = R T \exp [-a / (V_m R T)]$ 的 $\alpha$.

    \[
        \alpha = \frac{1}{V_m} \left.\pdv{V_m}{T}\right|_p = \dots
    \]
\end{framed}

\[
    \dd V = \left.\pdv{V}{T}\right|_p \dd{T} + \left.\pdv{V}{p}\right|_T \dd{p} = \alpha V \dd{T} - \kappa_T V \dd{p} \Rightarrow \frac{\dd{V}}{V} = \alpha \dd{T} - \kappa_T \dd{p}
\]

\begin{framed}
    $\alpha = 1 / T,\ \kappa_T = 1 / p$ 推导状态方程。

    \[
        \frac{\dd{V}}{V} = \alpha \dd{T} - \kappa_T \dd{p} = \frac{\dd{T}}{T} - \frac{\dd{p}}{p} \Rightarrow \dd{(\frac{p V}{T})} = 0
    \]
\end{framed}

\begin{framed}
    $\alpha = \beta = 1 / T$ 推导状态方程。

    \[
        \dd T = \left.\pdv{T}{V}\right|_p \dd{V} + \left.\pdv{T}{p}\right|_V \dd{p} = \frac{1}{\alpha V} \dd{V} + \frac{1}{\beta p} \dd{p}
    \]

    \[
        \dd T = \frac{T}{V} \dd{V} + \frac{T}{p} \dd{p} \Rightarrow \dd{(\frac{p V}{T})} = 0
    \]
\end{framed}

\begin{framed}
    $\alpha = \frac{R}{p V_m} + \frac{a}{V_m T^2},\ \kappa_T = \frac{R T}{p^2 V_m}$ 推导物态方程。

    \[
        \frac{\dd{V_m}}{V_m} = \alpha \dd{T} - \kappa_T \dd{p} = (\frac{R}{p V_m} + \frac{a}{V_m T^2}) \dd{T} - \frac{R T}{p^2 V_m} \dd{p} \Rightarrow \dd{\left[p \left(V_m + \frac{a}{T}\right) - R T\right]} = 0
    \]
\end{framed}

\subsection{热力学第一定律}

热容:稳定系统 $C_p > C_V$

\begin{itemize}
    \item 热容:$C \equiv \dv{Q}{T}$
    \item 等容热容:$C_V = \left.\pdv{U}{T}\right|_V$
    \item 等压热容:$C_p = \left.\pdv{H}{T}\right|_p = \left.\pdv{(U + pV)}{T}\right|_p$
\end{itemize}

\begin{framed}
    多方过程 $p V^n = \text{const}$。
    \[
        p V^n = \text{const} \Rightarrow \frac{\dd{p}}{p} + n \frac{\dd{V}}{V} = 0,\ p V = \nu R T \Rightarrow \frac{\dd{p}}{p} + \frac{\dd{V}}{V} = \frac{\dd{T}}{T}
    \]
\end{framed}

% \begin{framed}
%     根据第一定律证明两条绝热线不可能相交。

%     若相交,可构造不吸放热、只做功的循环,违反开尔文表述。
% \end{framed}

\subsection{热力学第二定律}

\begin{framed}
    根据第二定律证明两条绝热线不可能相交。

    若相交,取一条等温线与它们交于此交点上方,则可构造不吸放热、只做功的循环,违反开尔文表述。
\end{framed}

热力学第二定律的数学表述:
\begin{itemize}
    \item 卡诺定理:$1 - \frac{Q_2}{Q_1} \leq 1 - \frac{T_1}{T_2}$
    \item 克劳修斯不等式:$\oint \frac{\dd{Q}}{T} \leq 0$
    \item 熵:$\dd{S} = \dd{Q_{\text{rev}}} / T$,注意可逆过程
    \item 熵增加原理:孤立系统的熵永不减少。
\end{itemize}

\begin{framed}
    求理想气体的熵。
    \[
        \nu C_{Vm} \dd{T} = T \dd{S} - p \dd{V} \Rightarrow \dd{S} = \nu C_{Vm} \frac{\dd{T}}{T} + \frac{p}{T} \dd{V} = \nu C_{Vm} \frac{\dd{T}}{T} + \nu R \frac{\dd{V}}{V}
    \]
    or:
    \[
        \dd{S} = \nu C_{Vm} \frac{\dd{T}}{T} + \nu R \frac{\dd{V}}{V} = \nu C_{Vm} \frac{\dd{T}}{T} + \nu R (\frac{\dd{T}}{T} - \frac{\dd{p}}{p}) = \nu C_{pm} \frac{\dd{T}}{T} - \nu R \frac{\dd{p}}{p}
    \]
\end{framed}

\begin{framed}
    求理想气体自由膨胀过程中的熵变。

    只需注意到对于自由膨胀过程有 $\dd{T} = 0$.
\end{framed}

\begin{framed}
    已知 $U = b V T^4 = 3 p V,\ S(T = 0) = 0$,求 $S(T)$.
    \[
        \dd{U} = b T^4 \dd{V} + 4 b V T^3 \dd{T} = T \dd{S} - p \dd{V} = T \dd{S} - \frac{b T^4}{3} \dd{V}
    \]\[
        \dd{S} = \frac{4}{3} b T^3 \dd{V} + 4 b T^2 V \dd{T} = \dd{\left[\frac{4}{3} b T^3 V\right]}
    \]
\end{framed}

\subsection{热力学基本方程}

\begin{align*}
    \dd{U(S, V)}                      & = + T \dd{S} - p \dd{V} \\
    \dd{(U + pV)} = \dd{H(S, p)}      & = + T \dd{S} + V \dd{p} \\
    \dd{(U - TS)} = \dd{F(T, V)}      & = - S \dd{T} - p \dd{V} \\
    \dd{(U - TS + pV)} = \dd{G(T, p)} & = - S \dd{T} + V \dd{p}
\end{align*}

\begin{itemize}
    \item 通过全微分得到状态参量和态函数偏导数的关系;
    \item 通过交换偏导次序得到 Maxwell 关系。
\end{itemize}

\section{均匀闭系的热力学性质}

\subsection{Maxwell 关系}

\[
    \pdv{f(x, y)}{x}{y} = \pdv{f(x, y)}{y}{x}
\](有一个记忆法则来着,很管用)\[
    \left.\pdv{T}{V}\right|_S = - \left.\pdv{p}{S}\right|_V, \left.\pdv{T}{p}\right|_S = + \left.\pdv{V}{S}\right|_p, \left.\pdv{S}{V}\right|_T = + \left.\pdv{p}{T}\right|_V, \left.\pdv{S}{p}\right|_T = - \left.\pdv{V}{T}\right|_p
\]

还可以利用\[
    \frac{\partial x \partial y}{\partial^2 f(x, y)} = \frac{\partial y \partial x}{\partial^2 f(x, y)}
\]得到四个倒过来的关系。

\subsection{Maxwell 关系的典型应用}

\subsubsection{复合函数法}

\begin{enumerate}
    \item $U(T, V) = U[S(T, V), V]$\begin{itemize}
              \item 定容热容:$C_V = \left.\pdv{U}{T}\right|_V = \left.\pdv{U}{S}\right|_V \left.\pdv{S}{T}\right|_V = T \left.\pdv{S}{T}\right|_V$
              \item $\left.\pdv{U}{V}\right|_T = \left.\pdv{U(T, V)}{V}\right|_T = \left.\pdv{U[S(T, V), V]}{V}\right|_T = \left.\pdv{U}{S}\right|_V \left.\pdv{S}{V}\right|_T + \left.\pdv{U}{V}\right|_S = T \left.\pdv{p}{T}\right|_V - p$
          \end{itemize}
    \item $H(T, p) = H[S(T, p), p]$\begin{itemize}
              \item 定压热容:$C_p = \left.\pdv{H}{T}\right|_p = \left.\pdv{H}{S}\right|_p \left.\pdv{S}{T}\right|_p = T \left.\pdv{S}{T}\right|_p$
              \item $\left.\pdv{H}{p}\right|_T = \left.\pdv{H(T, p)}{p}\right|_T = \left.\pdv{H[S(T, p), p]}{p}\right|_T = \left.\pdv{H}{S}\right|_p \left.\pdv{S}{p}\right|_T + \left.\pdv{H}{p}\right|_S = - T \left.\pdv{V}{T}\right|_p + V$
          \end{itemize}
    \item 迈耶公式\begin{itemize}
              \item $S(T, p) = S[T, V(T, p)] \Rightarrow \left.\pdv{S}{T}\right|_p = \left.\pdv{S}{T}\right|_V + \left.\pdv{S}{V}\right|_T \left.\pdv{V}{T}\right|_p = \left.\pdv{S}{T}\right|_V + \left.\pdv{p}{T}\right|_V \left.\pdv{V}{T}\right|_p$
              \item $C_p - C_V = T \left[\left.\pdv{S}{T}\right|_p - \left.\pdv{S}{T}\right|_V\right] = T \left.\pdv{S}{V}\right|_T \left.\pdv{V}{T}\right|_p = T \left.\pdv{p}{T}\right|_V \left.\pdv{V}{T}\right|_p = \alpha \beta p V T = - T \frac{\left(\left.\pdv{p}{T}\right|_V\right)^2}{\left.\pdv{p}{V}\right|_T}$
          \end{itemize}
\end{enumerate}

\subsubsection{Jacobi 行列式法}

唯一需要特殊记忆的性质:$\pdv{(x, v)}{(u, v)} = \left.\pdv{x}{u}\right|_v$,其他的:反对称、倒数、链式法则

\[
    \frac{C_p}{C_V} = \frac{\left.\pdv{S}{T}\right|_p}{\left.\pdv{S}{T}\right|_V} = \frac{\pdv{(S, p)}{(T, p)}}{\pdv{(S, V)}{(T, V)}} = \frac{\pdv{(T, V)}{(T, p)}}{\pdv{(S, V)}{(S, p)}} = \frac{\left.\pdv{V}{p}\right|_T}{\left.\pdv{V}{p}\right|_S} = \frac{- \frac{1}{V} \left.\pdv{V}{p}\right|_T}{- \frac{1}{V} \left.\pdv{V}{p}\right|_S} \equiv \frac{\kappa_T}{\kappa_S}
\]

\begin{framed}
    $p = f(V) T$,试证明其内能以 $(V, T)$ 为参数时不显含体积。
    \[
        \left.\pdv{U(V, T)}{V}\right|_T = \left.\pdv{U[S(V, T), V]}{V}\right|_T = \dots = T \left.\pdv{p}{T}\right|_T - p = T f(V) - p = 0
    \]
\end{framed}

\begin{framed}
    证明:一个均匀物体在准静态等压过程中熵随体积的增减取决于等压条件下温度随体积的增减。
    \begin{align*}
        \left.\pdv{S(p, V)}{V}\right|_p & = \left.\pdv{S(T(p, V), V)}{V}\right|_p = \left.\pdv{S}{V}\right|_T + \left.\pdv{S}{T}\right|_V \cdot \left.\pdv{T}{V}\right|_p = \left.\pdv{p}{T}\right|_V + \frac{C_V}{T} \cdot \left.\pdv{T}{V}\right|_p \\
        \left.\pdv{S(p, V)}{V}\right|_p & = \left.\pdv{S(T(p, V), p)}{V}\right|_p = \left.\pdv{S}{T}\right|_p \left.\pdv{T}{V}\right|_p = \frac{C_p}{T} \left.\pdv{T}{V}\right|_p
    \end{align*}
\end{framed}

\begin{framed}
    证明:$\dd{S} = \frac{C_p}{T} \dd{T} - \left.\pdv{V}{T}\right|_p \dd{p}$
    \[
        \dd{S} = \left.\pdv{S}{T}\right|_p \dd{T} + \left.\pdv{S}{p}\right|_T \dd{p} = \frac{C_p}{T} \dd{T} - \left.\pdv{V}{T}\right|_p \dd{p}
    \]
\end{framed}

\begin{framed}
    证明:$\left.\pdv{C_V}{V}\right|_T = T \left.\pdv[2]{p}{T}\right|_V,\ \left.\pdv{C_p}{p}\right|_T = - T \left.\pdv[2]{V}{T}\right|_p$
    \[
        \left.\pdv{C_V}{V}\right|_T = \left.\pdv{}{V}\right|_T
        \left(T\left.\pdv{S}{T}\right|_V\right) = T \left.\pdv{}{T}\right|_V \left.\pdv{S}{V}\right|_T = T \left.\pdv{}{T}\right|_V \left.\pdv{p}{T}\right|_V = T \left.\pdv[2]{p}{T}\right|_V,\ \dots
    \]
\end{framed}

\begin{framed}
    证明 van de Waals 气体的等容热容与体积无关。
    \[
        \left.\pdv{C_V}{V_m}\right|_T = T \left.\pdv[2]{p}{T}\right|_{V_m} = T \left.\pdv[2]{}{T}\right|_{V_m}\left[\frac{R T}{V_m - b} - \frac{a}{V_m^2}\right] = 0
    \]
\end{framed}

\begin{framed}
    求van de Waals气体的内能和熵。
    \[
        \dd{U(T, V_m)} = C_{Vm} \dd{T} + \left[T \left.\pdv{p}{T}\right|_{V_m} - p\right] \dd{V_m} = C_{Vm} \dd{T} + \left[T \frac{R}{V_m - b} - p\right] \dd{V_m} = C_{Vm} \dd{T} + \frac{a}{V_m^2} \dd{V_m}
    \]\[
        \dd{S(T, V_m)} = \frac{C_{Vm}}{T} \dd{T} + \frac{R}{V_m - b} \dd{V_m}
    \]
\end{framed}

\subsection{特性函数}

实验测量得到的量(认为已知):$f(p, V, T) = 0$, $C_V = C_V(T, V)$, $C_p = C_p(T, p)$;

核心的两个态函数:$U$, $S$。

\begin{align*}
    \dd{U(T, V)} & = \left.\pdv{U}{T}\right|_V \dd{T} + \left.\pdv{U}{V}\right|_T \dd{V} = C_V \dd{T} + \left[T \left.\pdv{p}{T}\right|_V - p\right] \dd{V}   \\
    \dd{S(T, V)} & = \left.\pdv{S}{T}\right|_V \dd{T} + \left.\pdv{S}{V}\right|_T \dd{V} = \frac{C_V}{T} \dd{T} + \left.\pdv{p}{T}\right|_V \dd{V}            \\
    \dd{H(T, p)} & = \left.\pdv{H}{T}\right|_p \dd{T} + \left.\pdv{H}{p}\right|_T \dd{p} = C_p \dd{T} + \left[- T \left.\pdv{V}{T}\right|_p + V\right] \dd{p} \\
    \dd{S(T, p)} & = \left.\pdv{S}{T}\right|_p \dd{T} + \left.\pdv{S}{p}\right|_T \dd{p} = \frac{C_p}{T} \dd{T} - \left.\pdv{V}{T}\right|_p \dd{p}
\end{align*}

\begin{framed}
    已知某固体物态方程为:$V(T, p) = V(T_0, 0) \left[1 + \alpha (T - T_0) - \kappa_T p\right]$。证明其等压热容只与温度有关,与压强无关。设等压热容为 $C_p$,求焓和熵。
    \[
        \left.\pdv{C_p}{p}\right|_T = - T \left.\pdv[2]{V}{T}\right|_p = 0
    \]
    \begin{align*}
        \dd{H(T, p)} & = C_p \dd{T} + \left[- T V(T_0, 0) \alpha + V\right] \dd{p} = C_p \dd{T} + V(T_0, 0) \left(1 - \alpha T_0 - \kappa_T p\right) \dd{p} \\
        \dd{S(T, p)} & = \frac{C_p}{T} \dd{T} - V(T_0, 0) \alpha \dd{p}
    \end{align*}
\end{framed}

已知某个态函数?那么它的偏导数,即两个状态参量也已知。现在你需要做的是把其他的热力学量表示成这些量的组合:已知态函数 + 这个态函数的两个自变量,亦即将两个已知的状态参量替换为偏导数形式。

特性函数是联系热力学和统计物理学的桥梁。

\begin{multicols}{2}
    $F(T, V)$:
    \begin{itemize}
        \item $S(T, V) = -\pdv{F}{T},\ p(T, V) = - \pdv{F}{V}$
        \item $U = F + T S = F - T \pdv{F}{T} = - T^2 \pdv{}{T} \left(\frac{F}{T}\right)$
        \item $H = U + p V = F - T \pdv{F}{T} - V \pdv{F}{V}$
        \item $G = F + p V = F - V \pdv{F}{V}$
    \end{itemize}
    $G(T, p)$:
    \begin{itemize}
        \item $S(T, p) = - \pdv{G}{T},\ V(T, p) = \pdv{G}{p}$
        \item $H = G + T S = G - T \pdv{G}{T} = - T^2 \pdv{}{T} \left(\frac{G}{T}\right)$
        \item $U = H - p V = G - T \pdv{G}{T} - p \pdv{G}{p}$
        \item $F = G - p V = G - p \pdv{G}{p}$
    \end{itemize}
\end{multicols}

\subsection{绝热降温与节流降温}

\subsubsection{绝热降温:等熵过程}

\begin{align*}
    \left.\pdv{T}{p}\right|_S & = \pdv{(T, S)}{(p, S)} = \pdv{(T, S)}{(T, p)} \pdv{(T, p)}{(p, S)} = -\left.\pdv{S}{p}\right|_T \frac{T}{C_p} = \frac{T}{C_p} \left.\pdv{V}{T}\right|_p = \frac{V T \alpha}{C_p}     \\
    \left.\pdv{T}{V}\right|_S & = \pdv{(T, S)}{(V, S)} = \pdv{(T, S)}{(T, V)} \pdv{(T, V)}{(V, S)} = - \left.\pdv{S}{V}\right|_T \frac{T}{C_V} = - \frac{T}{C_V} \left.\pdv{p}{T}\right|_V = - \frac{p T \beta}{C_V}
\end{align*}

\subsubsection{节流降温:等焓过程}

节流过程是绝热不可逆过程。

Joule-Thomson 系数:
\[
    \mu_{\text{JT}} = \left.\pdv{T}{p}\right|_H = \pdv{(T, H)}{(p, H)} = \pdv{(T, H)}{(T, p)} \pdv{(T, p)}{(p, H)} = \left.\pdv{H}{p}\right|_T \cdot \frac{1}{- C_p} = \frac{1}{C_p} \left[T \left.\pdv{V}{T}\right|_p - V\right] = \frac{V}{C_p} (\alpha T - 1)
\]

\begin{itemize}
    \item 制冷区/节流降温:$\mu_{\text{JT}} > 0$
    \item 制热区/节流升温:$\mu_{\text{JT}} < 0$
    \item 反转曲线:$\mu_{\text{JT}} = 0 \Rightarrow T \left.\pdv{V}{T}\right|_p - V = 0$,即 $(p, T)$ 图上的一条线。
\end{itemize}

\begin{framed}
    证明:$\left.\pdv{T}{p}\right|_S - \left.\pdv{T}{p}\right|_H > 0$
    \[
        \left.\pdv{T}{p}\right|_S = \left.\pdv{V}{S}\right|_p,\ \left.\pdv{T}{p}\right|_H = - \left.\pdv{T}{H}\right|_p \left.\pdv{H}{p}\right|_T = - \frac{1}{C_p} \left[- T \left.\pdv{V}{T}\right|_p + V\right]
    \]
    又 $C_p > 0$,即证:
    \[
        0 < C_p \left.\pdv{V}{S}\right|_p - \left[T \left.\pdv{V}{T}\right|_p - V\right] = T \left.\pdv{S}{T}\right|_p \left.\pdv{V}{S}\right|_p - \left[T \left.\pdv{V}{T}\right|_p - V\right] = V
    \]
\end{framed}

\section{单元复相系的热力学性质}

\subsection{开系的热力学基本方程}

\subsubsection{化学势}
\begin{itemize}
    \item 在四个态函数的全微分中加入一项 $+ \mu \dd{N}$
    \item 由上一步可得:$\mu \equiv \left.\pdv{U}{N}\right|_{S, V} \equiv \left.\pdv{H}{N}\right|_{S, p} \equiv \left.\pdv{F}{N}\right|_{T, V} \equiv \left.\pdv{G}{N}\right|_{T, p}$
    \item 再次根据交换偏导次序结果不变,可以得到一些 Maxwell 关系
    \item $G = \mu N$ 的来源/逻辑:由广延性证明 $U = T S - p V + \mu N$,再利用 $G = U + p V - T S$
\end{itemize}

\subsubsection{热力学巨势}
\begin{itemize}
    \item $J \equiv F - \mu N = F - G = - p V$
    \item $\dd{J} = -S \dd{T} - p \dd{V} - N \dd{\mu}$
    \item 由此可以用 $J$ 的偏导数表示一些状态参量
\end{itemize}

\subsection{热动平衡}

\subsubsection{热动平衡判据}

\begin{itemize}
    \item (态函数)平衡和稳定判据:$\delta (\star) = 0, \delta^2 (\star) \neq 0$,熵极大,其他极小。
    \item 虚变动:$\sum \delta (\text{广延量}) = 0, \delta (\text{强度量})_i = 0$
\end{itemize}

\subsubsection{热动平衡条件}

\begin{itemize}
    \item 平衡:$T_{\text{A}} = T_{\text{B}}, p_{\text{A}} = p_{\text{B}}, \mu_{\text{A}} (T, p) = \mu_{\text{B}} (T, p)$
    \item 稳定:$C_p > C_V > 0, \kappa_T > \kappa_S > 0$
\end{itemize}

\subsection{Clausius-Clapeyron 方程}

\[
    - S_{0\text{A}} \dd{T} + V_{0\text{A}} \dd{p} = \dd{\mu_\text{A}} = \dd{\mu_\text{B}} = - S_{0\text{B}} \dd{T} + V_{0\text{B}} \dd{p}
\]

\[
    \dv{p}{T} = \frac{S_{0\text{B}} - S_{0\text{A}}}{V_{0\text{B}} - V_{0\text{A}}} \equiv \frac{L_{0, \text{A} \rightarrow \text{B}} / T}{V_{0\text{B}} - V_{0\text{A}}} = \frac{L_{m, \text{A} \rightarrow \text{B}}}{T (V_{m\text{B}} - V_{m\text{A}})}
\]

(上式通常为正;也可以用热机循环推导)

饱和蒸气压方程
\begin{itemize}
    \item A相固/液,B相气,$V_{m\text{A}} \ll V_{m\text{B}}$,i.e. $V_{m\text{B}} - V_{m\text{A}} = V_{m\text{B}}$
    \item 理想气体:$p V_{m\text{B}} = R T$
\end{itemize}

\[
    \frac{\dd{p}}{p} = \frac{L_m(T)}{R T^2} \dd{T} \xrightarrow{p V_m = R T} \frac{1}{V_m} \dv{V_m}{T} = \frac{1}{T} (1 - \frac{L_m}{R T})
\]

\begin{align*}
    \dv{L_m}{T} = \dv{\left[T \left(\Delta S_m\right)\right]}{T} = \Delta S_m + T \Delta \left(\dv{S_m(T, p)}{T}\right) = & \Delta S_m + T \Delta \left(\left.\pdv{S_m}{T}\right|_p + \left.\pdv{S_m}{p}\right|_T \dv{p}{T}\right)                                  \\
    =                                                                                                                     & \Delta S_m + T \Delta \left(\frac{C_{pm}}{T} - \left.\pdv{V_m}{T}\right|_p \cdot \frac{L_m}{T (V_{m \text{B}} - V_{m \text{A}})}\right) \\
    =                                                                                                                     & \Delta S_m + \Delta C_{pm} - \left.\pdv{\Delta V_m}{T}\right|_p \cdot \frac{L_m}{\Delta V_m}                                            \\
    \xlongequal{p V_m = R T}                                                                                              & \Delta C_{pm}
\end{align*}

\subsection{气液相变理论}

\begin{itemize}
    \item van der Walls 方程:$\left(p + \frac{a}{V_m^2}\right) \left(V_m - b\right) = R T$
    \item 临界点(指温度再高就没有气液共存了):$\left.\pdv{p}{V_m}\right|_T = 0,\ \left.\pdv[2]{p}{V_m}\right|_T = 0$
    \item 看一下关于稳定性的分析
\end{itemize}

还有等面积法则什么的……