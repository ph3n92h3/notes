\documentclass{ctexart}

\usepackage[a4paper,scale=0.8]{geometry}

\usepackage[colorlinks]{hyperref}
\usepackage{siunitx}

\title{戒为良药}
\author{飞翔}
\date{}

\begin{document}

\maketitle
\tableofcontents

\section{卷首篇:我的戒色经验谈}

\paragraph{前言}

每天来戒色吧回帖解答,很多戒友都认识我,想必大家对于我的戒色经历也比较感兴趣,我今天就把我自己详细的戒色过程和大家分享一下。希望可以帮助到大家。我戒色最初的动机和大家是一样的,就是因为 SY 导致了身体江河日下,还有就是道德层面的自责,觉得自己怎么能干这么龌龊的事情。

\paragraph{罪恶起源}

第一次 SY 来自摩擦的舒服,接下来就是“一发不可收拾”,成瘾了,一旦成瘾就无法自拔,被瘾控制,身不由己。我来戒色吧看到比较多的字眼就有“一发不可收拾”,这六个字就是 SY 这个行为最好的概括,一发不可收拾,好比决堤的洪水。所以,所谓“适度无害论”根本就是扯淡,只要开始就停不下来,就好像潘多拉的魔盒,一旦打开就收不住。

\paragraph{症状篇}

我的 SY 史有 15 年,那时因为我热爱运动,除了前列腺炎外,并无特别难受的地方,不过那时的尿频症状的确折磨了我很久。另外,我的容貌也发生了很大的改变,就是痤疮很严重,并且脸部气色变得很不好,给人一种颓废的感觉,双眼无神,年纪轻轻眼袋也出来了,那时我鼻炎也很严重,本来就是过敏体质,加上不断地 SY 消耗肾精,我的身体一直没有真正健康过,总是 SY 后感觉不行了,然后觉得要收敛了,于是戒个十几天,加上积极锻炼,感觉身体恢复不少,肾气一恢复,马上又开始堕落了,而且是变本加厉,越陷越深,经常连续 SY,这样我身体就更差了。

SY 对人的摧残是身心两方面的,自从 SY 后,我心理也发生了很大的变化,变得急躁易怒,没有耐心,记忆力、意志力、注意力都全面下降了,更可怕的是,那时的我还没有真正意识到 SY 的害处,看杂志上写着无害论,本来我戒色的立场就不坚定,看了无害论,我更加沉迷其中了。在那个怪圈里,我挣扎了好久,始终出不来,那个怪圈有种很强大的魔力,始终牢牢控制着我,那时我最高戒断过 28 天,那时的媒体根本就没有戒色相关的内容,有的尽是让人堕落的内容,那时的年轻人缺少这方面的正确引导。

因为 SY 导致了我脑力的下降,我偏科更严重了,最后考了一所很普通的大学,大学里我继续堕落,但因为是集体住宿,我 SY 行为有所收敛,并且大学里经常打篮球,所以大学时期我身体有段时间恢复得非常好,尿频症状自动消失,脸部皮肤和气色都变好了,心理方面的问题也随之化解,重新变得自信和朝气蓬勃了。毕业后工作开始谈女朋友,周围没人告诉我 SY 有害,没人告诉我婚前性行为会有恶报,大家谈论的都是堕落的内容,整个社会的大环境都是堕落的,戒色的声音实在是太微弱了。现在的年轻人,你和他谈戒色,他可能会以为你脑子有问题,给你的回答往往是人生在世就要好好享受,及时行乐。这其实就是人性的弱点:不见棺材不掉泪。那时的我也是这样想的,大家都谈女朋友发生关系,我觉得很正常啊!现在回过头来想想,简直是太无知太肤浅太不懂事,正因为如此肤浅和无知才会让我在怪圈里无法自拔,直到身体开始彻底垮掉。

\paragraph{以病入道}

因为工作后饮食作息不规律,并且开始熬夜,工作压力也很大,纵欲也没停过,我身体开始垮掉了。我得了神经衰弱和焦虑症,没有得过此病的人是无法真正理解的,那是一种让人彻底崩溃的感觉,生不如死的感觉,没有人可以理解你,除了你自己和病友能理解,连医生也无法理解你。自从得了此病,我才明白为什么有人会选择自杀,因为这个病实在太痛苦了,那时我也有自杀的念头,还好没有实施。在此,我要说明一下,焦虑情绪和焦虑症是两码事,焦虑情绪每个正常人都会有,而焦虑症则是有一大堆千奇百怪的躯体症状反复折磨你,简直是人间地狱,生不如死。

得了焦虑症后我辞职了,每天就是在病友群里聊天,然后正是从那时起,我开始自学中医知识,广泛地听名家讲座,看了非常多的书籍,也开始接触佛理,经过和无数的病友聊天,我终于明白了一个道理:这个病和 SY 还有熬夜有非常密切的关系。我开始搜集病友资料,搜集了上千份,后来我又开始搜集 SY 病友的资料,也搜集了上千份,我研究了好几年,在中医里找到了答案,其实西医里也有答案,那些西医无害论其实早已经过时了,现在西医最新研究成果和我国几千年的传统中医是相符的,现代科学也已经证明:一个生物个体的寿命和他的繁殖行为密切相关,就是次数越多寿命越短。这和我国药王孙思邈所说的精少则病、精尽则死是完全相符的。人在健康的状态下是不知悔恨的,不见棺材不掉泪嘛,你对健康的人说戒色,简直是对牛弹琴,当身体垮掉,他自然会改变观点而同意你的看法。很多戒友都有脱发症状,在头发很浓密时,他是不知道戒色的,当出现秃顶时,其实已经晚了,要恢复起码要几年。我也出现过脱发现象,戒掉后自然慢慢恢复浓密了。

我是以病入道的,我现在也已经信佛,自己也研究佛理,宗教信仰实在是一种很强大的力量,我现在才真正体会到,换做以前的我,肯定认为是迷信,不科学,现在的我不会这样认为了,因为据我了解很多科学家都是有宗教信仰的,而且不少科学家对佛教的评价都相当高,爱因斯坦也非常认同佛教。现在的我不仅信佛,中医养生我也一直在研究,伤精患者的恢复也是系统工程,懂得中医养生无疑对于恢复是很有帮助的。

\paragraph{恢复情况}

恢复情况是大家比较关注的,我就尽量写得详细点,在戒色吧回答最多的问题之一就是:能不能恢复?多久能恢复?我的答案是:肯定能恢复,多久能恢复,因人而异,每个人体质不同,会养生之道和热爱运动的人恢复会比较快。SY 对人的摧残是身心两方面的,在中医来讲,心不对,身就会不对,身不对,心也会不对,是互相影响的。所以你 SY,你心理肯定会出问题,而你心理问题得不到缓解,又会反过来影响身体,从而陷入了恶性循环。戒色就是一种历练,是一种蜕变,是一种升华,戒掉后你才能真正把握自己的人生,而不是做欲望的奴隶,无法自拔直至身体彻底垮掉。我在戒色吧看到无数神衰患者,非常多,吧主也有神衰经历,强迫症的也非常多,焦虑症的也不少,其实这些问题大同小异,都差不多,只是症状表现有所不同,下面谈谈我的恢复情况。

身体方面的恢复:戒色一年后,我焦虑症自愈了,这一年我积极锻炼,学会了养生,根本没吃任何药物,原来我依赖过半年的药物,那是一种惨痛的经历,一种药耐药了,医生就给你换另外一种,最后自己彻底沦为了药罐子。戒色后,我尿频腰痛也自己好了,几十种躯体症状也自动消失了,我研习中医后才知道,肾虚百病丛生,当你肾气足了,病邪自然消失。原来出现脱发倾向,每天一大把,中医:发为肾之华,戒色后积极锻炼,现在一天最多掉 5 根,属于正常的范围。脸上的气色也好了,恢复了阳光,眼睛有神了,中医讲:人体五脏六腑之精气皆上注于目。一个人眼睛有没有神采,就可以看出他五脏的健康状况,而肾藏精,藏五脏六腑精华之气,肾虚了,五脏就虚了,五脏虚了,眼睛自然就会变得暗淡无神,是一环扣一环的。我的手也有握力了,手有没有握力,其实反映的是肝脏的健康状况,中医:肝在力为握,手的握力越大,肝气就越足,而肝肾同源,你肾水不足,肝木就燥,一是脾气容易变得急躁易怒,二就是握力会变小。我原来睡眠状况也不好,经常失眠多梦,现在睡眠已经很好了。还有焦虑症患者比较常见的肌肉跳,躯体震颤感,现在也已经完全消失。耳鸣也好了,中医:肾开窍于耳。耳鸣患者那是相当多,我戒掉半年后耳鸣自愈。现在的我感觉已经脱胎换骨,重生了一般。以前 SY 恶习像条绳索把我捆着,现在感觉到的是真正的自由,而不是放纵的自由,放纵的自由只会导致一个结果:痛苦。

心理方面的恢复:因为 SY 我变得自卑、自闭乃至出现过自残倾向,因为老戒不掉,就有自残的冲动,恨自己没用。因为 SY,容貌也变丑了,戒色吧里很多戒友都觉得自己变丑了,其实这种感觉一点没错,你精气神没了,人能不丑吗?就像干瘪的篮球,拍也拍不起来,简直废掉了。人也是一样,精气一走漏,人的容貌马上就会走下坡路,紧接着人就自卑了,颓废了,失去了原来的底气和自信。这种容貌的变化,在中医的面诊学里讲得很明白的,脸是五脏的镜子,你五脏虚了,脸上自然会有变化,懂医理的明眼人一看便知。我戒掉后,注重养生之道,不断积精累气养精蓄锐,现在我的容貌气质已经彻底恢复阳光健康,以前的我可以用一个词来形容:灰暗。现在的我又重新变得“明亮”起来了,自卑自闭自残的倾向也随之消失了,感觉像换了一个人,感觉自己的底气变足了,一点也不怕了。肾虚的人基本都会变得胆小,因为肾主恐,恐又伤肾,因为 SY 变得胆小怕事的人特别多,在我研究的病例中相当普遍。面对镜子:自信、阳光、健康的我又回来了,那个自甘堕落、猥琐丑陋的男人消失了,那种丑陋是一种灵魂加肉体的丑陋,就是 SY 的恶果。

\paragraph{成功经验篇}

我之所以能戒掉,是因为我彻底觉悟了,曾经我也失败过无数次,没有人指导我,没有人点醒我,因为那时的我很无知,没有戒色知识,没有养生意识,没有明白佛理,就是一个词:强戒。以为靠意志力就能戒除,其实光靠毅力强戒是很难成功的,必须从“净化心灵”上入手,也就是从根上入手,脑控制手,SY 行为是脑发出的指令,如果你有很多邪淫念头,你就会自然而然去 SY,所以必须学会修心,断除邪念,净化大脑里的思想,通过戒色文章的反复学习、反复灌输来不断强化正气,久而久之,SY 自然就能戒除。

戒色第一步:改造思想,就像戒毒一样,你去戒毒所,第一步,其实就是向你灌输正确的知识,让你明白危害,明白真相,清除你大脑里那些错误的想法和思想。很多戒友之所以反复失败,就是因为改造思想的失败。靠毅力强戒,只能成功一时,很快就会败下来,千万不能放松对戒色文章的学习,我现在每天还是会看看戒色文章和中医养生知识。这些善知识就是一道防火墙,防淫如防火,有了防火墙,就不怕了。以前我强戒,满脑子全是 YY,现在通过彻底改造思想,我一天也不会有一个 YY 念头,念头没了,自然就不会有 SY 行为。所以,我希望给大家指一条明路,少走弯路。我研究过很多戒色成功者,没有一个例外,这些人都是自我净化思想特别厉害的人,你和这些成功的人交谈,他们和你说戒色的道理可以口若悬河,滔滔不绝,他们向你灌输时,其实自己也是在强化灌输的内容,这样他们就更不容易破戒。一句话:要戒色成功,就要彻底改造思想,否则只有失败。

我研究戒色,也研究过戒毒、戒烟、戒酒,戒网瘾,戒购物瘾,这些我都研究过,成瘾这个现象如同过敏,可以有各种刺激物和源头,所以,给自己一个无毒环境很关键。成瘾现象按性质分,可以分为生理成瘾和心理成瘾。人的成瘾机制和多巴胺密切相关,多巴胺是负责在大脑细胞之间传递信息的一种递质,是大脑快感中心的源泉之一,因为它主要负责传递亢奋和快感的信息,成瘾者的多巴胺分泌往往处于失控状态。戒色必须纠正这种失控状态,戒色就是戒性瘾,很多人戒不掉就是心瘾太重,潜意识深处有太多的黄毒,要清除掉这些黄毒,唯有彻底改造思想,否定那些让你上瘾的内容。看到美女如果第一反应是起邪念,那就完蛋了,如果第一反应是白骨观和不净观,那就对了,久而久之,你的定力自然会变强。白骨观和不净观是为了对治自己的邪念,并非不尊重女性,这点要明确。我们要不断培养自己的浩然正气,在面对任何女色时,都不能起一丝一毫的邪念,起邪念就是在污浊自心。记住,是你控制欲望,而不能让欲望控制你,你不能做欲望的奴隶!

\paragraph{公式}

经过我几年的研究,我得出了一个公式:熬夜+纵欲+久坐=完蛋。光 SY,不熬夜,不久坐,那伤害还小一点,如果有熬夜和久坐,神经衰弱肯定会找上门来。另外,如果你热爱运动,症状也不会很重,恢复也比较快。熬夜和纵欲在中医来讲是伤精最严重的两种方式,而久坐伤肾,又熬夜又久坐又 SY,身体垮掉得就会特别快。希望大家能牢记这个公式,引以为戒,这个公式是我和广大病友用最惨痛的经历换来的,也可以说是用钱换来的,因为不少病友看病至少花了几万,十几万的也有不少。

\paragraph{恢复方法}

看了我上面的文字,大家应该能看出恢复的方法了,那就是戒色后积极锻炼,要学会养生之道,根据自己的兴趣来选择适合自己的运动方式,强度一定要把握好,过犹不及。如果症状比较明显,则要配合积极治疗。根据我对神衰患者的研究,很多患者都是通过站桩和打坐,让自己好起来的,我自己也有亲身体验,补元气的确很有效果,但贵在坚持,也要得其要领才行。另外,食疗也很重要,彭鑫博士推荐的食疗方案不错,大家可以搜着看下,大家最关心的问题之一就是:如何补。大家都知道自己虚了,就会问如何补才能恢复。其实我可以明确告诉大家,最大的补就是“不泄为补”,最好的药就是“戒为良药”,药补不如食补,食补不如动补,要让自己动起来,当然不是过度运动而是适量的运动。

戒友普遍存在一个认识上的误区,那就是完全靠医生,觉得看了医生一切问题都可以解决,其实这种想法是错误的,三分治,七分养,如果你不懂得养生,那么药物的疗效真的是很有限的,比如给你吃补肾的药物,而你却还在 SY,这样反而会适得其反。药物帮助你恢复肾气,一恢复就乱来,肾气永远不会满,要“上补下不漏”才行。伤精很严重的戒友,如果你的症状比较明显,我建议还是要积极治疗的,积极的治疗有助于身体的恢复,但千万不要依赖上药物,那是另外一种怪圈,重心应该放在戒色和养生上,这才是恢复的王道。

\paragraph{关于遗精问题}

遗精不算破戒,但频繁遗精就要引起重视了,导致频遗有很多种原因,如果频遗一直持续,那最好去看看中医调理。我再推荐一个八段锦里的动作,叫“双手攀足固肾腰”,大家可以百度搜下,这个动作很简单,但效果真的很好,我每天都压个几百下,遗精次数明显减少了。这个方法在很多养生的文章里都有讲到,动作原理不变,但叫法有很多种,原理就是拉伸大腿后侧的膀胱经,而膀胱经与肾经相表里,从而起到了固精关的作用。

减少遗精的注意要点:
\begin{enumerate}
    \item 不要盖太厚
    \item 不要吃太油太荤
    \item 不要熬夜
    \item 白天不要太劳累
    \item 睡前不要喝太多水
    \item 不要饮酒
    \item 睡前不要打坐
\end{enumerate}

\paragraph{关于春梦问题}

不少戒友深受春梦的困扰,这其实还是潜意识里黄毒在作祟,梦是潜意识的窗口,你潜意识里不干净,在睡觉时就会从梦中跑出来。要克服春梦多的问题,那就必须深度净化潜意识,如何净化?就是不断重复地看戒色文章,不断给自己灌输戒色的内容,不断重复、重复、再重复,必须加大灌输的强度和力度,这样坚持一段时间,你就会发现自己的春梦变少了。

\paragraph{关于勃起障碍}

很多戒友由于性功能下降而戒色,希望能尽快恢复。首先我们要读懂身体的信息,这样才会做出正确的选择。勃起障碍其实是身体自我保护机制的一个报警,是身体在告诉你不要再纵欲了,要好好休养生息了。原理类似于身体受寒,毛孔自动关闭一样。勃起障碍就是身体在告诉你该停止了,不能再纵欲了。不少戒友不懂这个道理,反而找邪淫的东西看,希望自己能行,这样反而会雪上加霜。根据我的研究,有阳痿早泄倾向的戒友都有一个共同的经历,那就是:强行 SY 或者连续多次 SY。中医的医案早就讲过,这就是导致肾气大衰的最关键的一个因素,我自己总结了一句话叫:欲不可强,越强越亏。SY 好比购物,花的是健康,购买的是短暂的快感。很多人就是在自己的贪欲下阳痿的,就是因为不懂医理,狂撸滥泄,希望这段文字能够点醒广大戒友。

\paragraph{戒色的归宿}

戒色不是让你当和尚,而是让你学会控制自己的欲望,加强自身的道德修养,戒色吧提倡的是杜绝婚前性行为,婚后节制,这才是正道。这既符合传统文化,也符合中医养生之道。

\paragraph{附:戒色的十个阶段}

\begin{enumerate}
    \item 第一个阶段,是意识到身体不行了,想戒色了,但看了无害论立场又不坚定了,觉得适度无害,结果还是继续 SY,这个阶段是最原始最初级的戒色阶段,比较无知和幼稚。结果只有一个:失败!
    \item 第二个阶段,稍微懂一点 SY 的危害,但不深刻,还是:失败!
    \item 第三个阶段,学习别人的戒色经历,尝试突破最大戒色天数,结果是戒色的时间纪录是突破了,但还是突破不了怪圈,还是:失败
    \item 第四个阶段,迷茫阶段,经过无数次失败会自我怀疑,还能不能戒掉,这个时候已经不太想戒了,不战自败!还是:失败
    \item 第五个阶段,身体再次发出警告,明显感觉身体垮掉了,再次燃起戒心,无奈只有雄心壮志,而不懂方法。还是:失败
    \item 第六个阶段,开始彻底认清无害论的真面目,所谓适度无害只能骗骗小朋友,这时候思想虽有飞跃,但还是没明白戒不掉的根本原因。还是:失败
    \item 第七个阶段,悟道阶段,这个阶段学会了专业戒色,超越了强戒和盲戒的层次,定力大增,戒色天数突破百天大关。但还是:失败
    \item 第八个阶段,觉悟增长阶段,通过不断学习戒色文章,戒色的觉悟不断增长,突破 200 天不在话下。无奈百密一疏,可能由于某天打开网页不慎,或者是放松了对戒色文章的学习而功亏一篑,还是:失败
    \item 第九个阶段,大成阶段,觉悟由于不断完善和提升,已经达到相对完美的程度。戒色修为更上一层楼,诱惑的对境根本动不了他了,看到等于没看到,别人喜欢,他害怕,他的反应和普通人相反,从而做到了绝不动心,此阶段已经彻底摆脱SY恶习。
    \item 第十个阶段,化境阶段。“本来无一物,何处惹尘埃”,已经到了“破我无我“之大境界,大巧若拙,大智若愚,就像金庸武侠最厉害的武功高手乃“扫地僧”,深藏不露,大隐隐于市。达到此境界者,是真正的“高人”。
\end{enumerate}

\subparagraph{tips} 每个戒友都可以找到自己所在的阶段,就像打游戏有自己的等级,等你境界修到了,自然而然就会成功,就怕你没悟性,那就很难戒除了,所谓会者不难,难者不会。希望大家不要停止学习戒色文章的脚步,哪天你顿悟了,就能做到了,个别悟性高的人可以完成“跳阶”,好比学习好就可以完成跳级,学习差就只能留级。

\section{飞翔经验}

\subsection{屡戒屡败的根本原因——修心不到位}

本人在戒色吧泡了一年左右,看见无数戒友的失败,很多戒友每次都信誓旦旦地要戒,决心极大,没过多久就败下阵来,破戒的人会很懊丧,自我否定乃至颓废,甚至破罐破摔。当然逃避的人也有不少,就是干脆认为戒不掉就不戒了。

几乎所有人都是因为身体江河日下,才想到戒色,但是每次戒又出不了怪圈,这个怪圈好像有种魔法,始终牢牢控制着你,你就像一个提线木偶被控制着,SY 这个恶习似乎很难戒掉,每次戒都无法突破怪圈。

很多戒友破戒后总是把原因归咎于没毅力,没自制力,这其实还是认识不深刻,以我的经验来讲,仅仅靠意志力和毅力来强戒,难度极大,除非这个人是“戒色烈士,宁死不手”,但这种忠烈之人极少。戒色不能一味靠毅力,靠毅力强戒基本都会失败,只会越戒越差,甚至放弃。

鉴于大家屡戒屡败,屡败屡戒,我想给大家指条明路,我正是通过这个方法彻底戒除了 SY。有句话说得好:成功总有方法,只是你没有找到。当你屡次失败,就要想想失败的深层次原因,而不是失败了就敷衍了事,草率地归咎于自己没毅力,这样的认识下次还是会失败,出不了怪圈。

我在这里要向大家普及一个很重要的概念:修心!相信大家都听说过这个词,修行的核心是修心!这里的心指的就是念头,戒色要学会控制念头!真正懂得修心,才有望戒除 SY 恶习,否则只是在外围打转,还没真正入门。有的戒友会靠转移注意力和充实生活来戒色,在学会修心之前,基本都是这样做的,这是治标不治本的方法,关键还是在于修心,因为心是根本,是念头导致了行为,所以必须在根本上下手,这样才能真正戒除。

有的戒友之所以屡戒屡败,一方面是有思想误区,觉悟不够完善,另外最根本的原因就是不会修心,或者修心不到位,总是 YY,贪恋心很重。念头一起,不知断除,反而跟随,最后欲火中烧,欲罢不能,不得不破。这个破戒流程,相信很多戒友已经重复很多次了,就是因为修心不到位。我们要学会修心,要学会观察自己的念头,不要跟着邪念跑,要学会识别邪念,及时断除邪念。从第 17 季开始,我会系统地介绍如何修心,如何控制自己的念头,这是戒色实战方面最关键的知识。这种知识非常宝贵,真正掌握了,就有望主宰自己的内心。

刚开始有些新人很浮躁,没有认真学习戒色文章,就一味靠毅力强戒,发毒誓,结果必然是失败,因为他们的思想觉悟没有提升,也不懂得修心,只是一味蛮干,肯定失败。脑控制手,是脑海中的念头导致了 SY 的行为,所以必须要学会修心,对治 YY。修心不到位肯定是不行的,很多戒友屡戒屡败,一方面对境实战太差,看到诱惑就盯着看,不知避开,对境是很大的考验,一定要学会避开!菩萨见欲,如避火坑!在这个色情泛滥的时代,一定要谨慎上网,管住自己的视线。另外一方面就是断念不行,念头上头时,无法立刻断除,这样很容易就会失去控制。

要想成功戒除,首先要通过学习戒色文章改造思想,基本每个戒友都被无害论误导过,脑子里有很多误区,所以新人开始戒色,一定要多学习戒色文章,改造自己的思想,认识危害,克服贪恋,好好下决心戒。佷多人看了很多黄片,甚至看了十几年,满脑子的邪念,看见异性就会起邪念,猥琐龌龊之极!充满了负能量!这就是长期用邪淫的内容给自己洗脑的结果,把自己的脑整个儿洗成了黄脑。很多人看无害论,也是在给自己洗脑,让那些错误的理论占据自己的大脑,从而变得是非不分。戒色就是要改造思想,建立正知正见,学会修心,这样才能从那个怪圈中出来。

我们戒色要坚持学习戒色文章,吧主和戒友会分享好的戒色文章,这都是宝贵的学习资源,好的戒色文章应该反复研读,多做笔记。大家一定要有学习的日课,不断坚持下去,不要中断,坚持学习一段时间,自己的觉悟就会有明显的提升,很多思想误区也会得到纠正,也知道戒色的重点在哪里,前辈会反复强调那些重点,帮助你加深认识,提高觉悟。我戒到现在,还是经常会看戒色文章和戒色笔记,我也一直在答疑,不仅是行善积德,对于保持良好的戒色状态也是非常重要的。我们要正己化人,帮助更多的人,自己戒掉了,也要帮助大家一起戒掉。

修心不仅是对治邪淫的念头,也要学会对治其他的负面念头,比如贪念、嗔恨、嫉妒、傲慢、抱怨等等,修心的范围应该扩大化,这样才能充满正能量。有的人虽然戒色了,但是其他负面念头还是很重,这样戒到一定时间,就可能因为这些负面念头而间接导致破戒,比如和人吵架生气了,内心失衡,就想通过看黄手淫来发泄,这样就会重新掉入怪圈。我们戒色一定要学会保持心平气和,避免嗔恨的心态,内心一定要祥和稳定。也不能嫉妒别人,有了这种念头,心里就不会安宁,就会生起烦恼,你自己的相续就被染污了。若是你再诋毁对方的话,你就跟对方结了恶缘,种了恶因,将来你会因此遭受痛苦。看到别人成功时,应当生起欢喜心,由衷地随喜!戒色后一定要谦虚,看到不少戒友戒了一段时间就开始骄傲了,结果就是“骄兵必败”,产生了轻敌思想,,放松了警惕,必然破戒。

对于总是失败的戒友,希望你们不要灰心,不要气馁,也不要自暴自弃,前辈也曾失败过很多次,关键要认真反省和总结,完善自己的觉悟,强化自己的修心功夫,这样才能越戒越好,才能突破怪圈,战胜 SY 恶习不在话下。你们在学习戒色文章后,也要把戒色的知识再教给别的戒友,这样既可以帮助别人,自己也加深了印象。

记住,这是一个过来人的告诫。

\subsection{屡戒屡败的思想误区之为恢复性功能而戒}

这篇文章就戒色动机专门谈一下,因为今天我一上来回答问题,就遇见 2 个因为动机不对而深陷怪圈的戒友,而以往我每次上来回答问题,几乎每天都能碰到因为戒色动机存在思想误区而导致的屡戒屡败,无法胜数。这个问题很严重,我想有必要讲一下。

为了恢复性功能而戒,这个动机乍一听上去,好像没什么问题,很多戒友都因为长期 SY,性功能出现了下降的趋势,有的甚至已经出现早泄症状和阳痿了,有的则是在通往早泄和阳痿的路上,最糟糕的是,他们还没认识到问题的严重性,就像温水煮青蛙一样,这些“青蛙”还在缓慢加热的锅里放松地游泳。

《黄帝内经》有云:生病起于过用!我们的身体就像一台机器,如果只是使用而不注重保养,那么零件很快就会磨损折旧乃至报废。我们的身体有一定的自我修复功能,所以,很多戒友在戒除 SY 一段时间后,感觉身体和性功能都有所恢复,但正是这个阶段是破戒最危险的时候,因为肾气一恢复,下面就会有一定的反应,邪念也开始滋生,那些以“为恢复性功能而戒”的戒友就开始蠢蠢欲动,起了“测试心”,想看看自己的恢复程度。正是这个想法让他们又一次掉进了 SY 的怪圈泥潭,一发不可收拾,甚至是变本加厉。

其实这就是典型的动机不对,从根上就错了,结出的果子当然就是错的,这种戒色动机从一开始就注定失败,必然的失败!因为他们是为了将来更好地放纵而戒,是为了更好地满足自己的欲望而戒,从一开始,他们戒 SY 的动机就是错的,动机错,结果就错,而且注定是越陷越深,不能自拔。

所以,建立起一个正确的戒色动机是关键也是基础,这就像一座大厦的地基一样,地基一定要正确稳固,否则就是在沙子上建房子,说倒就倒。

我提倡的动机是:

\begin{enumerate}
    \item 为了恢复身心健康:这是很多戒友戒色的最初动机,因为身体已经撸垮了;
    \item 中医养生:中医提倡的是保精惜精,学点中医养生知识,会更加有利于戒色;
    \item 宗教信仰:比如佛教是反对邪淫和婚前性行为的,提倡的是婚后节制。
\end{enumerate}

当然,还有一些动机也不错,比如要为家人争气,要实现自己的人生理想等。

就是“为了恢复性功能而戒”不可取,当你建立起正确合理的动机后,性功能自然而然就能恢复,但是你仅仅是为了性功能而戒,就会重新掉进陷阱里,这是必然的,是成百上千戒友的经历所揭示的真理。

希望我的这篇文章能带给广大戒友一些有益的思考,对待戒 SY 能有更清醒的认识。

有什么问题可以咨询我,我会以我的丰富经验给予你解答,我希望自己能帮助到更多的戒友跳出怪圈,重新找回阳光健康的自己。

\subsection{教你彻底摆脱频遗的烦恼}

这篇文章就遗精问题专门深入谈一下,希望可以给大家指一条明路,相信有悟性的戒友读完后就可以彻底摆脱频遗的烦恼了。请仔细详读之。

我回答过的几百个问题中,遗精问题绝对占很大的比例,只要你开始戒色基本都会受到遗精的困扰。我每天都会遇见提问遗精的戒友,最少不低于 3 个,最多一天有 10 个左右。其中很大一部分戒友真可谓深受遗精的困扰,但却是踏破铁鞋无觅处,怎么也找不到办法减少遗精的次数,控制不了遗精,本来肾气已经养起来了,一泄,就好比“一夜回到解放前”,很是困扰!很是颓丧!

曾经我也和大家一样,坚决戒色,却被遗精问题所深深困扰,遗精虽算不上破戒,但是遗精后基本都会感觉到身体不行,特别是对于那些肾气亏损很厉害的戒友,本已亏损,再一遗精,实则雪上加霜,遗精完第二天明显感觉不舒服,有人甚至感觉浑身散架了。遗精和 SY 虽然泄漏精气的形式不同,但结果是差不多的,精气都漏掉了,精气一走漏,身体就会感觉不行。如果你身体非常健康,肾气很充足,偶尔一次遗精是基本感觉不到什么的,就怕肾气已经亏损,再一遗精,反应就会很明显,随着年纪的上升,遗精后的不适感会更明显。而且,遗精对戒色的信心也是一种打击,虽然遗精不算破戒,但是精气走漏后,人的底气就会感觉下降,就像泄了气的皮球,中医养生讲究“积精累气”,对于泄漏精气是相当忌讳的。

道教把遗精叫“走丹”,本来是要“练精化气”的,现在精走漏了,怎么化气?拿什么去化气?百日筑基,这 100 天是不能遗精的,而 100 天不遗精,普通人能做到的真的很少很少,只有得道之人才能做到。其实说破了,就好比捅破一层窗户纸,知道后你会觉得太简单了,但是,如果你不知道方法,那就很难减少遗精的次数,100 天不遗精更是不要去奢望了,这里特别要提醒的是,这 100 天不遗精不是你刚开始戒色时 100 天不遗精,因为据我研究,很多人因为长年 SY 恶习,导致肾气大亏,戒色后再次出现遗精可能会在半年乃至 8 个月后,我这里说的 100 天不遗精是针对戒色后第二阶段而言的。

遗精曾经困扰了我大半年,这大半年我每个月遗精 3 - 5 次,有时连续两个晚上遗精,虽然不是很多次,但对于本已经亏损的身体,的确是“伤不起”了,记得有段时间我恢复不错,结果一次遗精就让我身体的状况下降不少。这大半年我每次遗精都做记录,几月几号遗精的,在几点遗精的,有梦还是无梦,我都做了详细的记录,每次遗精后我都在网上找方法,真是看了无数的文章和帖子,有药补的,有各种各样动作的,有暗示疗法的,但还是没办法减少遗精,有段时间我也困惑于是否真的是“精满自溢”,后来我否定了,因为的确有人通过练功,大大减少了遗精次数,乃至杜绝了遗精,的确有高人能做到,我当时在想,如果有哪位高人能指点我一下就好了。据我了解,某些气功是可以杜绝遗精的,但必须有明师指导,自己是不能瞎练的,弄不好会出偏,练出毛病来,所以我也不敢轻易去练习。气功里的站桩也可以减少遗精,但要做到 100 天不遗精是很难的。遗精在修道界是一道“铁门槛”,不知有多少人过不去这道坎,这道坎过不去,身体的体质要有质的飞跃,是相当难的,因为你补进去的东西,都被你遗精漏掉了,好比水箱,上面放水,下面漏水,这个水箱永远满不了。

而现在我终于摸索到了这个方法,也可以说是“悟到了”,因为以前我看过类似文章有讲到,但当时就是无法理解里面的深意,而且很多文章写得很繁琐,要求颇多,你一看这种文章,就会感觉摸不到头脑,你会觉得越看越复杂,自己很难完全掌握,即使你做了,也不一定能得其要领,很难完全做对,结果往往是遗精照旧。其实一篇文章几千字,最后一句话就可以概括,所谓“假传万卷书,真传一句话”,这句话你悟透了,就彻底明白了,根本无须看那么多文章,越看越糊涂。如果你悟不透,那层窗户纸就好比一座山,如果你悟透了,就像捅破一层窗户纸那样简单。

自从我悟透那个道理,到现在我已经有 3 个月没遗精了,肾气越来越充足,精神非常好。今天我就来捅破这一层窗户纸,希望大家竖起耳朵听以下的内容。

听到了,听懂了,你就是在省钱,不用吃各种补药,也不用花钱去医院。

首先我们来认识一个概念,这个概念就是“精关”,这个精关好比堤坝,如果堤坝够牢固,洪水是冲不垮的,就怕精关不固,遗精就会很频繁。如果你精关非常牢固,即使你相火妄动,也是很难冲破的。这就好比矛和盾的关系,如果盾够牢固,你矛再厉害也捅不破,所以如何加固精关,如何加固这个“盾”,就是我们必须要解决的问题。我摆脱遗精的困扰,就因为发现了一句口诀,这个口诀来自八段锦,叫“双手攀足固肾腰”,这个动作形式非常简单,双腿站直用手触地,在广播体操中有类似动作,其实学校里的广播体操正是脱胎于我国的传统气功八段锦。千万千万不要小看了这个动作,这个动作里面有“大玄机”,这个动作通过拉伸大腿后侧的膀胱经从而起到了固精关的作用,原理就是肾经和膀胱经相表里,它俩功能相连、气血相通,所以你拉伸膀胱经就能作用到肾经,从而就能加固精关,做这个动作就像在给精关上紧螺丝,精关加固了,频遗问题就会随之化解了。我过去看了无数的文章,当中也包括八段锦的介绍,但那时并未引起我足够的重视,也就是“没开悟”,没领悟其中的精髓,没彻底明白这个动作的威力!现在我终于意识到了这个动作的实战威力,比吃任何补药都有效!!!而且这个动作很简单,也容易坚持做。明白玄机的人会把它奉为至宝,不明白玄机的人就会不重视它,忽视它。

\paragraph{详细做法}

我教给大家的方法就是:只做八段锦的一个动作,就是专门针对遗精的“双手攀足固肾腰”。坚决把这个动作做到位,所谓做到位就是要找对拉紧感,感觉对了才是最重要的,做这个动作时要感觉到大腿后侧的韧带好像橡皮筋一样拉紧、拉直、拉长,找到这个感觉后,要不断强化这个感觉,这样就好比在给精关上螺丝加固,我每天做 500 次,晚上上床睡觉前必做 200 次,我建议晚上睡觉前一定要做这个动作,这个动作就好比在给精关上锁。你睡前不做这个动作,那就容易出现遗精,因为精关很可能处在比较松垮的状态,必须通过做这个动作来加固加紧。但睡前做固肾功也不能搞太累,如果太累了,也可能导致出现遗精,这点要格外注意。刚开始很多人都没基础,有的人体能也比较差,所以刚开始做固肾功,次数可以少一些,比如从 30 个开始,也可以分组做,比如你做 60 个,可以分三组,每组 20 个,组间休息 1 分钟。刚开始做固肾功的第二天,身体可能会出现酸痛的反应,注意休养几天即可恢复正常,到时再做就不会有酸痛的表现了。

\paragraph{精}

最后我再来阐述一下精的特质,精是一种神奇物质,施之则生人,留之则生己。精有一个特点,那就是精是可以“内化”的,通过功法的修炼达到练精化气,还精补脑。所谓“精满则溢”的理论只适合没有修炼过功法的普通人,因为没有修炼过功法,所以他的认识就只能局限在“精满则溢”这个层面,而没有更深刻的体验。就好像看到白天鹅就认为世界上只有白天鹅,其实世界上还有黑天鹅,只是他没有看到罢了。

\paragraph{后记}

一般戒色到一定阶段,几乎每位戒友都会出现频遗的烦恼,我们一定要学会控制自己的遗精次数,尽量把遗精频率控制在一月 3 次以内,希望我这篇文章能给深陷频遗烦恼的戒友带来启示,希望你们早日摆脱频遗的困扰。


\paragraph{固肾功成功控遗案例反馈}

\begin{description}
    \item[案例 1] 我之前在您的帖子中看到一个双手攀足固肾腰的动作,话说这个动作真是摆脱遗精的神技啊!我坚持做这个动作,每天睡觉前 200 个,竟然四十多天都没有遗精。而没做这个动作的时候一个星期就遗精了 2 次,这个动作真是强大。然而,我不争气,戒了 44 天的时候,破戒了。我现在已经能保证完完全全戒除,已经信心满满了。
    \item[案例 2] 刚戒色的第一个月,遗精了 8 次,差点没被吓死,看了飞翔哥的文章才又重拾信心。再加上自己有练习八段锦,第二个月竟然一次没有遗精过。现在的心态也比较好了!!!真的很感谢飞翔哥和戒色吧。希望大家也能像我一样,走在正确的道路上,共创我们美好的未来。
    \item[案例 3] 撸龄 8 年!矮、挫、穷。心中的女神都睡在别人怀里,陪在别人身边,很悲伤。看了很多飞翔老师的帖子,我才知道自己毁掉了宝贵的年华。今天是我戒色的第三十一天,第七天的时候遗精了,然后就按照飞翔老师的方法,成功控制了遗精。从第七天后到现在,未曾再次遗精,梦里也很安然,不再有春梦。身体明显感觉有精神了,腿软问题没有了,腰酸背痛的症状也没有了。面貌开始清晰,看上去很干净,从前好猥琐!思维也清晰了,雄心壮志回来了。现在我做着自己喜欢的事,赚进让我自由的钱。我终于明白,这才叫真实的生活。而从前那段撸管自卑悲伤的日子,一去不复返了。
    \item[案例 4] 看了戒色吧以后才开始戒色,后来也是频繁遗精,搞得我很恼火啊!我记得最严重的时候,大概是半个月吧,就遗精了 7 次,那是最多的一个月了,晚上冷到了就会遗精,身体真的是很恼火的,很虚弱啊,太虚弱了。连中午睡觉的时候也滑精了。那时候感觉想死啊!但是生的欲望非常强烈, 于是鼓起了勇气给我妈说了,我说我在戒色,而且遗精非常频繁,感觉人要虚脱了,我妈就带着我去医院看了老中医,那老中医小时候我就认识,现在很老了 80 岁了吧,但是还在上班, 院长叫他每天上午去医院坐坐接一下诊之类的。我给老中医说了我的情况,后来开了药叫我别紧张,会好的,就这样,吃了他的药之后,好了很多,一个月遗精才 2 次吧,固肾功也就放松了没做了,之后停药了。但是后来停药之后就反弹了,感觉又开始频繁遗精了,我就在想,能不能不吃药啊,于是就坚持做翔哥的固肾功。现在每天都做啊,以前没坚持,隔几天就会遗精,现在天天做,基本上个把月没遗精了吧。呵呵。不过我感觉我有一个小优点才让我身体开始恢复的,就是在戒色吧戒色了以后,从没破过戒。那些频繁遗精的朋友请坚持戒色,身体会感觉得到的,你对它好它会知道的。固肾功请永远坚持。
\end{description}

\subsection{警惕欲望休眠期和戒色厌倦情绪}

每天都有无数破戒的戒友,破戒后的懊恼和对自己的失望是不难理解的,相信每个戒友都反复经历过。从开始发誓戒除到屡戒屡败,再到最后的彻底戒除,是一个漫长的过程,这个过程就是改造思想意识的学习过程,是一个思想认识不断提高的过程。这季就欲望休眠期和戒色厌倦情绪专门谈一下。

反复破戒的戒友,如果你有点中医常识,你就会发现:破戒也是有套路的,也是有规律可循的。破戒的规律和“肾气值”密切相关,当你感觉身体不行了,你就会本能地想戒色,当你戒色一段时间,肾气开始有所恢复,就特别容易破戒。刚开始戒色会经历一段欲望休眠期,很多人在这个阶段心瘾不重,还是能自我控制的,但欲望休眠期特别容易放松警惕,以为戒色成功了,其实欲望只是暂时休眠,等你肾气恢复到一定程度,它就会苏醒,这时候破戒往往是变本加厉地 SY,前功尽弃。欲望休眠期因人而异,有人只有 3 天的休眠,有人是 20 天,有人是 60 天,据我了解,欲望休眠期平均为 3 周左右,也就是 21 天左右。欲望休眠期接下去就是破戒高峰期,破完戒就是心理后悔期,这时候自我否定和后悔感会比较强烈。

这篇文章主要就是要让大家认识到欲望休眠期,在这个时期要提高警惕,保持警觉,不要放松学习,只有不断学习戒色知识才能提高你的戒色觉悟,才能提高你的戒色定力等级,戒色的定力等级和打网游练级是一个道理,刚开始等级低,被心魔虐,等你不断学习戒色知识,定力等级上去了,心魔就不是你对手了,就动不了你了。如果你不学习戒色知识,不开悟,定力等级永远那么低,遇见心魔,结果可想而知,遇见一次失败一次,看到黄源一点定力都没有,一看到心马上乱,马上跟着点击,这就完了,没有免疫力和抵抗力。

\begin{center}
    发戒心 $\to$ 欲望休眠期 $\to$ 肾气有所恢复 $\to$ 破戒高峰期 $\to$ 心理后悔期
\end{center}

这就是破戒的过程,也就是怪圈,很多人几年,甚至十几年都出不了这个怪圈,我曾经就是十几年出不了这个怪圈,因为那时我没开悟,没有学习戒色知识和养生知识的意识,就是强戒和盲戒,所谓强戒就是以为靠意志力就能戒掉,其实强戒注定失败,因为戒色境界根本没有提升。盲戒,就是不学习戒色知识瞎戒,盲戒也注定失败,因为戒色境界也没实质的提升。

人脑“中黄毒”的机制和电脑“中病毒”的机制很相似,电脑中病毒后会影响系统的运行,同样,人脑中黄毒后,身体健康就会出问题,一般最先出现问题的是泌尿系统疾病,以前列腺炎为主。然后肾一虚,肾虚百病丛生,什么毛病都可能出现,脑力下降也很普遍,中医:肾上通于脑。SY 伤肾必伤脑力,记忆力和注意力都会下降。脑力不行,学业和事业都会受到很大影响。

再来谈谈戒色厌倦情绪。

戒色厌倦情绪实在是太普遍了,就像厌倦一件衣服,一道菜,一个手机一样。戒色知识看多了也是会让人厌倦的,一旦出现厌倦情绪,戒色就失败了一半,所以一定要学会调整心态和情绪,做好情绪管理,一出现马上调整,养成良好的阅读和学习习惯,做到每天学习戒色知识不放松,一般养成习惯后就不大容易厌倦了,就好像刷牙一样,习惯了就成自然了,哪天不刷牙也许你就会觉得不舒服,戒色知识的学习也要找到这种状态,当然不一定是戒色知识,养生类知识也很不错,因为养生和戒色是相通的。另外,一篇戒色文章其实可以看很多遍,因为温故而知新,你看得越多理解的程度就越深,而不是浅尝则止,看过了就忘。

\paragraph{结语}

孙子兵法有云,知己知彼百战百胜,要戒色就要知道怎么破戒的,为何出不了破戒的怪圈,这是每个戒友都要深入思考的问题,等你想明白了,戒色的境界就上去了,如果你还停留在戒色初级阶段,强戒盲戒,那注定只有失败。只有不断学习,才能开悟,哪天你顿悟了,境界就不同了,定力等级就上去了,离彻底戒除就不远了。加油!

\subsection{怎样补才能补到位}

今天专门来谈如何“补”的问题,这个问题是戒友们关注度非常高的一个问题,和戒一样,补也存在非常多的思想误区。这篇文章就来详细谈一下,希望能帮到大家。

补是门大学问,里面的水很深,不是想象的那么简单。很多戒友因为长年 SY 恶习,身体严重亏损,自然第一反应就是补,总想吃点什么补一下,自己不敢乱补的就会去看中医,吃中药来补肾气。其实补是应该的,但也要学会如何高效率地补,而不是低效的补,里面有很多讲究。不管是自己吃补药或者食补,还是找中医开药方,补的第一前提就是:修心功夫要到位。因为你吃补药后,肾气会有所恢复,肾气一恢复,欲望就会起来,这时候非常容易破戒,一破戒等于在泄漏肾气,这样补的效果就会大打折扣了,得不偿失。所以,如果你意识不到修心的重要性,而去一味地补肾气,结果就是时好时坏,身体还是没有多大的恢复,甚至会出现这样的情况,因为吃太多补药,反而出现了一些副作用,身体更加不行。而且补药也是会耐药的,吃多了你就会发现,效果没刚吃时好了。这其实就是补的误区之一:只知补,不知修心。所以大家一定要注重修心,多学习戒色文章,把修心功夫提起来,只有通过学习才能增加戒色的定力值,有了定力,YY 自然会少很多,甚至可以做到没有 YY,这样的心理状态再去补,效果就会加倍,否则上补下漏,肾气始终满不了。

补的误区之二:忽视吃饭的重要性。

其实最补益精气的不是人参,不是鹿茸,不是山药,不是黑豆,不是任何一种补药,而是大米!假如补精是在造房子,大米的作用相当于地基,如果你不好好吃饭,那么你吃再多其他的补肾食物,也不会收到多大的效果,因为你补的地基不稳。

大米有很多做法,我比较认可的是吃粥养生。

养生名著《老老恒言》中说:“每日空腹,食淡粥一瓯,能推陈致新,生津快胃,所益非细。”宋代大诗人陆游专作《食粥》,其诗写道:“世人个个学长年,不悟长年在目前,我得宛丘平易法,只将食粥致神仙。”陆游寿逾八秩,可见其食粥的补养之效。传统中医认为,食粥能滋生精液,培养胃气,助消化、且营养俱存。

唐朝医学家孙思邈,因少年多病而学医,并以佛家与道家的智慧来养生,活到一百多岁,他亦主张清晨食白粥。另外,又将中药煮粥,利用“米气”与水分作“药引”,根据五脏六腑的“生物时钟”,去调理身体,治疗疾病。

所以,大家不要舍本求末,把最重要的,也是最补益精气的大米给轻视或者忽略掉,好好吃饭比什么都重要。

补的误区之三:不看中医瞎补,乱补。

这种误区在戒友中相当普遍,有人不想去看医生,嫌麻烦,有人则怕难为情,原因很多,最后就是自己网上查补药,看自己的症状更适合哪种补药,然后去药房买,其实这种做法是有失偏颇的,肾虚可不像感冒,到药店买个感冒药就 OK 了,肾虚是需要对症治疗的,中医讲究:同病异治,异病同治。即使相同的症状,因为个体差异,药方也不尽相同,具体要望闻问切之后才有明确的判断。所以,最好是去正规的大医院去看,找有经验的老中医看比较保险。另外,也不要把全部希望压在医生和药上面,三分治,七分养,如果你不学习养生知识,没学会养生之道,那么药的作用非常有限,比如你在吃补药,同时又在熬夜久坐,这样补的效果就很有限了。另外,很多人身体很虚,是不适合大补的,因为身体“虚不受补”,脾肾阳虚,吃下去也吸收不了,反而成了胃肠的负担,所以最好找有经验的医生把脉看一下,不要擅自瞎补。

补的误区之四:不运动。

大家一说到补,第一反应就是吃。其实吃并不是最高明的补,最高明的补是运动!药补不如食补,食补不如动补。动补这个词太好了,运动就能帮助你身体恢复,比药的疗效还好。

中医讲三阳开泰,善则升阳,喜则升阳,动则升阳。又讲阳强则寿,“阳气者,若天与日,失其所,则折寿而不彰”,所以运动这种补药效果非常好,只要运动不过量,对于身体的恢复是很有帮助的。大家戒色的误区就是不运动,抱怨戒色后怎么身体恢复情况不佳,其实你扪心自问一下,你运动了吗,你学会养生之道了吗?光戒是远远不够的,必须动起来,必须学会养生之道,这样恢复才快。三阳开泰里面还有个“善则升阳”,这里面有很深刻的养生哲学,多做善事,包括佛教的放生,其实对你身体的恢复是有很大帮助的,做善事发慈悲心,你身体的阳气自然就补足了。

我推荐的最佳补法:就是打坐和站桩。

我曾经试过很多补药,都不如打坐和站桩好!打坐和站桩既可以补元气,又有利于修心,一举两得,不用花钱,不必担心副作用。很多人也许会认为修炼气功会出偏,气功的确有出偏的情况,但我说的打坐和站桩不是那种容易出偏的类型,我说的是养生类的打坐和养生桩,是不会出偏的。

最大的补药:不泄为补!可以说这四个字就是补法的王道,否则你大吃大补几十天,连续两天遗精就会让你“一夜回到解放前”,所以,要补到位,必须在如何减少遗精次数上狠下功夫,把遗精次数减少到最少,杜绝各种容易引起遗精的原因,再配合上积极锻炼,这样身体恢复才比较快。特别是对于长期 SY,肾气透支严重,身体症状繁多的戒友来说,如何减少遗精更是重中之重,是必须思考和研究的一个问题。

最后补充说一下前列腺炎的问题,这个问题是 SY 戒友最容易遇见的问题之一,非常普遍,基本人人都遇见过。很多戒友思想上存在误区,认为前列腺炎和感冒一样,看医生吃下药就会彻底好了,其实这是完全错误的,好多帖子里都有这样的戒友,看了很多次医生,花了上万的医疗费,但还是看不好,原因何在?其实如果懂点中医常识就知道了,SY 伤了肾气,肾气损伤后就会出现前列腺炎,医生给你治病,帮你恢复肾气,前列腺炎暂时好了,但是你又 SY,肾气又伤,所以前列腺炎特别容易复发,复发率在 90\% 以上,原因就是没有明白医理,思想认识上有问题。只有彻底戒掉 SY,前列腺炎才有望真正康复。这种康复其实就是思想认识上的飞跃,就是认识到了根本原因所在,注重养生了,节约使用肾气,这样前列腺炎才不容易复发。否则在这个问题上认识不清,你有得好跑医院了,花在检查和药费上的钱就会越来越多。

\subsection{前列腺炎、精索静脉曲张、早泄阳痿的恢复}

前列腺炎和精索静脉曲张是 SY 戒友比较普遍的疾病,特别是前列腺炎,因为 SY 伤了肾气导致尿频的戒友实在太多了。其实得精索静脉曲张的人也有很多,只是很多人并不知道自己已经得上了,精索静脉曲张可以无任何不适症状,但确实已经得上了,去医院照个 B 超就知道,一般精索静脉曲张 99\% 发生在左侧,也就是左边的睾丸上,严重的有明显的蚯蚓状曲张血管团块,自己都可以摸到,并且有坠胀感,轻微的精索静脉曲张可以无任何不适,我得的就是轻微的,得了十几年自己都不知道,去医院照个 B 超才知道已经得上了,有人是左侧,有人是两侧都有,一般左侧的人比较多。今天就根据我的研究和体验来谈下前列腺炎和精索静脉曲张,千万不能轻视这两个病。

我得前列腺炎很早,在 SY 一年不到,就出现了尿频,尿急的症状,最多一个晚上十几次厕所,白天喝了水没多久就要上厕所,根本兜不住,后来我才知道,这种兜不住尿的情况在老年人当中非常多,因为老年人肾气大衰,失去了固摄能力,所以一有尿就憋不住,反观很多小孩子,肾气特别足,一泡尿甚至可以憋上一小时,所以,尿频其实表示你肾气已经虚衰了,不能再 SY 了,是身体在向你发出警告了,如果你听不懂这个警告,一意孤行,就会引起严重的前列腺疾病,不仅是前列腺有问题,全身都可能出现症状,肾虚百病丛生,前列腺炎只是最早出现的症状之一。

一般出现了尿频症状,是个人都应该知道是 SY 导致的,因为没 SY 前是没有尿频症状的,这时候大多数人都会悬崖勒马,开始自觉戒除 SY,戒掉十几天,肾气一恢复,马上又开始 SY,然后又出现尿频症状。很多人深深被尿频症状所困扰,就会去医院求医治疗,一般去医院看医生会开两种药,前列康和消炎药,有医德的医生会关照你不要 SY 和久坐,实际情况是很多医生都没意识到 SY 的危害,很多医生都是 SY 无害论的认同者,真是可悲。

我现在彻底戒掉 SY 后,尿频症状再也没犯过,学了中医医理就知道,肾气足,万邪熄,肾气虚,什么毛病都可能找上门来。很多戒友因为前列腺炎看了很多医生,花掉了上万的医疗费,还是看不好,久治不愈,总是暂时治好了,然后 SY 后又复发了,复发率那是相当高,其实复发的根本原因还是认识上存在误区,很多戒友把前列腺炎当成感冒,以为吃两粒药就搞定了,不会再犯了,其实这种认识是大错特错的,前列腺炎的发病正是由于 SY 伤了肾气,吃药是在帮助你恢复肾气,肾气一恢复,你再 SY 伤肾气,这样肾气又伤,就会再次复发,如果你还没认识到肾气的重要性,还是把前列腺炎等同于感冒,那你一辈子都无法根治前列腺炎了,到 40 岁以后可能就前列腺增生了,或者发展为更严重的前列腺疾患。所以,要彻底根治前列腺炎,只有一条路,那就是彻底戒掉 SY 和 YY ,然后积极锻炼,不熬夜不久坐,养成良好的生活方式和作息饮食习惯。对于已经结婚的朋友,如果你前列腺炎比较严重,我建议你最好禁欲一段时间,把肾气养足,肾气养得很足后,再节制地使用肾气,意思就是节制地过性生活,这样就可以避免前列腺炎的复发,黄帝内经有云:生病起于过用!所以,如何把握一个度,真是一门学问,过犹不及。

很多人会问禁欲有害吗?特别是新人会这样问,我的回答是禁欲无害,如果有害,那和尚岂不是都是病秧子了吗?事实是很多和尚都活到了 100 岁以上,虚云法师 120 岁,本焕长老 106 岁,佛门长寿的人很多,所以禁欲对身体有害纯属无稽之谈。如果要说禁欲有害,只有一种情况,那就是修心功夫不到位,天天 YY,想 SY 又不敢 SY,这样憋着有可能会憋出毛病,如果修心功夫到位,禁欲是不存在危害的。

再补充一点,就是 YY 的危害,很多新人会问:光看不撸算破戒吗,光看不撸有害吗?其实 YY 对人体的损耗也是非常严重的,甚至比 SY 还厉害,这在中医里讲得很明白,不少戒友光看不撸后照镜子马上发现自己的精气神萎靡了,下降了。这就是中医讲的:心动则精自走。只要有 YY,精气就会自动走漏,SY 属于明耗,而 YY 属于暗耗,暗耗的危害更严重,希望广大戒友能深刻地认识到这点,加强学习,加强修心,做到彻底杜绝 YY,做到念起即断,念起不随,念起即觉,觉之即无。

下面进入到核心内容:

前列腺炎是会导致不孕不育的,当然不是每个前列腺炎患者都会不孕不育,具体还要看你的前列腺炎严重程度,而且前列腺炎还会导致精子的质量不行,精子的质量不行,生出来的后代就达不到优生优育的标准,可能将来你的孩子一出生体质就不行,先天就不足。很多戒友对报应说是持怀疑态度的,其实因为你年轻时无知放纵自己,将来很可能就丧失了生育能力,这其实就是报应了。纵欲主义时代,不孕不育是非常非常多的,困扰着很多人,你现在这个阶段可能困扰于 SY 后引起的症状,将来到结婚了,就会困扰于精子的质量不行无法致孕带来的烦恼,而且中医认为:肾上通于脑!SY 伤肾必伤脑力,脑力不行,学业和事业都会受到很大的影响。所以,最好能杜绝婚前性行为,好好养足自己的肾气,到将来结婚后可以生个健康的宝宝。我每天在贴吧回答问题,估计到现在已经回答了上千个问题了,其中因为婚前放纵自己,很多戒友已经丧失性能力了,出现早泄和阳痿的戒友太多太多了,试想你还没结婚,身体就不行了,结婚后怎么过适当的性生活,如果你老婆不理解你,很可能就闹离婚了,没结婚就把自己给废了,将来怎么办?这是一个非常严峻的问题,一个非常现实的问题,所以提倡杜绝婚前性行为是非常非常重要的,不仅影响你将来婚姻生活的质量,还影响你下一代的优生优育。

再来谈下精索静脉曲张,这个病也是不孕不育的罪魁祸首之一,非常影响精子的质量,而得上精索静脉曲张的原因基本都是因为长期 SY 和久坐。这也是报应之一。

很多戒友都不知道自己得上了,所以容易被忽视,我得了十几年自己都不知道,因为我是轻微的,并没有感觉到任何不适,去医院检查才知道,那段时间我又是前列腺炎又是精索静脉曲张,可想而知,精子的质量有多差,达不到致孕的标准。我那时虽然是轻微的精索静脉曲张,但医生还是建议我手术的,医生说了一句让我气馁的话,他说即使做了手术,也不能保证你可以恢复生育能力,还得看运气!!!后来我学习了中医医理才知道,做手术只能帮你结扎,但不能帮助你恢复肾气,如果你肾气虚衰,即使做了手术也无济于事,所以注意养足肾气才是治愈的关键,否则你即使做了手术,暂时恢复了,如果你再 SY 伤了肾气,精索静脉曲张还是会复发,到时候欲哭无泪,怪谁?

最后再来谈下早泄阳痿倾向的恢复,其实我在以前写的文章《屡戒屡败的思想误区之一:为恢复性功能而戒》里已经说得很明白了,早泄阳痿倾向是人体自我保护机制的表现,原理类似于毛孔遇冷空气自动关闭一样,阳痿早泄是身体在自保了,很多戒友并不明白这个道理,还去找邪淫的内容看,透支自己的肾气,这样就会加重早泄阳痿倾向。很多戒友出现早泄阳痿倾向,第一反应不是戒色,而是要补要吃,或者去看医生,希望恢复坚挺,其实吃是能帮助你恢复肾气的,但肾气一恢复,你再 SY,再伤肾气,就会再次出现早泄阳痿,你这时再吃相同的补药效果就不明显了,因为补药也会耐药的。所以,吃是不能解决根本问题的,必须戒色,必须通过戒色和积极锻炼,正确的生活习惯来养足肾气,肾气养足了,切记不可试,一试就又掉进 SY 的陷阱,又会陷入恶性循环。而且早泄阳痿的恢复速度很慢,少则 3 个月,多则一年以上,所以要做好持久战的准备,一点一点把肾气养足,不要再干损耗肾气的事情了,很多人只知道性会损耗肾气,其实损耗肾气的方式有很多,纵欲、熬夜、生气、久坐、吃冷饮、过于劳累,用力过猛,吹空调,这些都会损耗肾气,当你肾气充足时,并不会有多大感觉,就像钱多时用掉点不会觉得有什么,当你肾气不足时,再这样干,就是雪上加霜了。中医把肾气比作健康货币,平时要注意储存,不要一味透支,要懂得养生养肾气,好好呵护肾气,这样才能保持住健康的精神面貌。

回答了这么多问题,我深感广大戒友不只是身体有症状,更重要的是认识上有问题,认识上存在误区,就很难戒掉,认识上有误区,前列腺炎和早泄阳痿就别想恢复了,所以大家必须多学习戒色知识,包括中医养生知识,来纠正自己思想认识上的误区,当你头脑中形成了正确的认识,正确的想法,你就知道怎么做了,而不是一味把希望寄托在药上,寄托在医生身上,那是缘木求鱼,因为肾虚引起的疾病不是靠药靠医生就能治好的,三分治,七分养,关键是要学会养生,学会养肾气,肾气足,万邪熄。千万不要把肾虚当成感冒,两粒药就可以摆平,肾虚这个毛病更多靠养,不是靠药。

\paragraph{结语}

现在的年轻一代正在走我以前走过的弯路错路,因为中国教育在这方面可以说是空白的,学校里是学不到的,所以思想上存在误区的戒友非常多,认识上有误区,就会在 SY 陷阱越陷越深,不能自拔,不少戒友上瘾后,把自己搞得像行尸走肉,人不人,鬼不鬼,粉色鸦片的威力实在不容小觑,我聊过的戒友,有不少都想自杀,因为 SY 加熬夜,他们已经得上更可怕的神衰和焦虑症,植物神经紊乱,强迫症,抑郁症,这对于只是尿频困扰的戒友是不敢想象的。我现在这样写文章,就是希望更年轻的一代不要再重复我走过的错路,那是一条不堪回首的错路。希望这篇文 章能带给大家有益的启示,戒色战壕里的各位戒友,加油!

\subsection{戒色三阶段以及熬夜久坐的深入分析}

在进入第 7 季正文之前,先把遗精问题再谈一下,前面第 3 季我写了一篇文章向大家推荐八段锦里的固肾功,很多戒友都做了这个动作,我得到的反馈有两种:一种是做了没多大效果,另一种则是做了以后效果非常好,不少人已经突破了最大不遗精天数,有人是 20 天还没遗精,有人是一个多月才遗精 1 次,比起以前一月 3 次以上要好很多。

为何有人做了没多大效果呢,其实有其自身原因,很多人做这个动作并未找到感觉,只是马虎草率了事,做了 200 个还没找到强烈的拉紧感就上床睡觉了,怎么能保证效果呢?必须找对感觉,找准拉紧感,不断强化拉紧拉长拉直的感觉,带着强烈的拉紧感上床睡觉,这样才有效果,另外,导致遗精的其他因素也有很多,都要注意一一避免,这样才能最大限度地减少遗精次数,其他导致遗精的因素有:白天 YY,白天劳累,喝酒,吃肉太多,趴着睡,裸睡,晒被子,盖太厚,睡前打坐,内裤太紧等,这些因素都要注意一一避免。

把遗精次数减少到最少,这对于身体的恢复是非常有利的,否则你吃再多补药,频繁遗精都会给你泄掉,所以,如何减少遗精次数是广大戒友必须面对和思考的一个问题,我推荐的八段锦固肾功只要做对感觉,对固精关是非常有效的,关键是找对感觉。我不敢说做了固肾功就能保证你永不遗精,但我可以保证,只要你找对感觉,并且避免掉其他导致遗精的因素,那就可以把遗精次数控制在一月 1 次,也就是一年 12 次,这样的遗精频率对于身体恢复是非常有利的。

这个动作刚开始做时,会有一个肌肉酸痛的表现,在运动生理学上叫“延迟性肌肉酸痛”,一般 1 - 3 周内会自动消失,适应强度后,再做同样的动作就不会感到酸痛了,这需要一个过程,只要坚持做这个动作,酸痛会自己消失的,所以出现酸痛表现的戒友不用担心。另外,一些韧带练过的戒友,手掌可以摸地,可以适当加大点难度,比如单脚脚尖翘起,用双手摸单脚,或者可以加个小板凳,在小板凳上做固肾功。做这个动作时切记不可动作过猛,过快,否则容易导致头晕,最好是找到适合自己的节奏,既找准感觉,又找对节奏,这样做才是正确的。这个固肾功是我亲身体验并且不断摸索总结的,所以我可以把最直接的经验和大家分享,希望大家多加尝试,坚持每天都做这个动作,时间长了自然会找到感觉的,有的人有基础有天赋,很快能找到找对感觉,有的人可能反应迟钝些,需要假以时日才能找到感觉。

下面切入第 7 季正题

我在第四季发过一张图,里面讲到了戒色的三个阶段。分别是:

\begin{description}
    \item[初级阶段] 盲戒强戒,不学习。
    \item[中级阶段] 学戒。开始懂得学习戒色知识的重要性。
    \item[高级阶段] 悟戒。学而有所悟,才有可能戒色成功。
\end{description}

大家可以对照一下自己所在的阶段,很多戒友自己戒了很多年,还停留在初级阶段,破戒了还是认为自己毅力不行,其实靠毅力的强戒注定失败,要戒色成功必须多学习戒色知识和戒色文章来提高觉悟,提高戒色定力等级,有了定力,才能降伏心魔,否则,心魔一来,不是你降伏它,是见了心魔没抵抗力,见了心魔就投降。戒来戒去,戒了很多年,依然失败,这就好像你在学校念书,不学习,永远在小学一年级,如果你不断学习,就可以升上去,可以念到初中,高中,大学。戒色也是如此,必须不断学习增加定力,当你的定力如小树苗时,大风一吹就倒,当你的定力如山时,别说台风飓风吹不动,就是原子弹也动不了你。所以,大家破戒了不要怪外界环境的诱惑太大,诚然这个纵欲主义时代诱惑太多,但是,如果你定力过关,一切诱惑都是动不了你的,看见当没看见,绝不起心动念。要提升定力等级,必须大量学习戒色文章和戒色知识,学得越多,看得越多,懂得就越多,这种学习是每天必须进行的,不是你看了一篇文章知道了 SY 的害处就不再学习了,而是要每天!每天不断学习戒色文章,如果你有宗教信仰,那就更好了,宗教信仰是戒色的一大助力,有宗教信仰的人更容易戒色成功。

当你意识到学习戒色知识的重要性后,就会每天自觉搜索戒色文章学习,学得越多,定力等级增长得就越快,学到一定程度,自然而然就会开悟,一开悟,戒色境界就会大幅度提升,戒色的天数就会大幅度增加,当然,这也要看个人的悟性,有悟性的人可以完成跳级,悟性差的人可能会留级。有的人戒色 200 多天依然失败,原因不是放松学习就是厌倦学习,一旦懈怠和放松警惕,心魔就会乘虚而入,如果你定力等级还没达到“大成”,那就很容易一时糊涂,鬼使神差般地破戒,身不由己。其实学到一定程度,有了足够的见识以后,不一定要找新的戒色文章了,找几篇自己认为比较好的文章天天看即可,温故而知新,就像佛教每日的早课晚课一样。

我解答了上千的问题,深感大多数戒友还处在初级阶段,特别是新人,不学习,失败就在所难免!所以我的解答里有很多都建议多学习戒色文章和戒色知识,只有学习才能开悟,曾经我不学习,最多只戒过 28 天,相信很多戒友都比我好,但是我一旦开悟,定力等级就大幅度上升,可以说是一次成功,我现在还处在学习状态,开悟状态,每天都有新的认识和体悟。我是彻底开悟了,当你们彻底开悟时,就能做到彻底戒掉。我现在的状态是 YY 也不会有,因为我深知 YY 是在暗耗精气,当你通过学习中医医理知道这一层知识后,我相信很多戒友都能做到尽量避免 YY,通过不断学习,你知道的知识越多,就会戒得越专业越规范,而不是停留在瞎戒盲戒。

下面再来谈谈戒色后的身体恢复,第 7 季主要谈熬夜和久坐的危害。

不少戒友反映戒色后身体并未恢复多少,或者感觉身体恢复很慢,其中有频繁遗精的因素,还有不良生活习惯的因素,当然还和运动习惯密切相关。

据我研究,戒色后积极锻炼的戒友身体恢复比不锻炼的戒友恢复得更快,更好。这就是中医讲的:动则升阳,阳强则寿。当然我指的是适量运动,不是过度运动,因为过犹不及。

不少戒友虽然戒色了,但还在熬夜和久坐,他们对熬夜和久坐缺乏深刻的认识,或者他们习惯于熬夜和久坐,惯性力量太大,一时无法改变,但我要说的,不改变也要改变,要有壮士断腕的魄力,必须改正过来,变则通,否则贻害非浅。

熬夜,很多戒友认为通宵才算熬夜,其实我指的熬夜是 23 点以后睡就算熬夜了,如果你超过 23 点睡觉其实身体就很难修复了,因为中医医理有云:23 点子时是人体一阳生的时刻,这个时间必须是熟睡的状态,这样身体才能更好地修复,否则你错过了这个时间段,身体就很难修复了,这个道理好比你错过了最后一班公车,只有明天才能坐上了,而且中医认为,你熬夜一天,是需要很久才能恢复的,你熬夜一天的潜在损害可能需要 100 天才能真正调整过来。很多人并不知道这个医理,把熬夜当成家常便饭,而且往往是熬夜加久坐再加 SY,没多久就把自己身体给废掉了,出现了神经衰弱的症状,心理问题也很严重,这其实就是一个恶性循环,一直熬夜晚睡慢性积累的必然恶果。戒色后一定也要戒掉熬夜,因为熬夜伤精也是非常厉害的,这个医理必须认识到,如果你认识不深刻,就会继续熬夜,千万不要小看熬夜的威力,这种威力类似温水煮青蛙,不知不觉就让你废掉。最好能在9点半上床睡觉,这是最好的。很多戒友会说做不到,那我要说的是,必须想方设法做到不熬夜,否则你就亏大了。我聊过的戒友,不熬夜的明显比熬夜的恢复要快。

再来谈谈久坐,很多戒友对久坐有危害可能第一次听见,其实久坐的危害也非常之大,久坐可以导致腰椎病和颈椎病,久坐伤肾伤脾,伤运化,影响消化能力,压迫膀胱经,危害真是非常之大,容易得上前列腺炎和精索静脉曲张,而且阳主动,阴主静,久坐属阴,我们生活的地方叫阳间,你一直久坐就会导致身体阴气太重,在中医经典里就有讲到久坐折寿,阴气重到最后就会归阴,意思就是到阴间去,就是死掉了,所以我们要动起来,每 40 分钟就要起来活动 10 分钟,不要一坐几小时,那样危害实在太大。这种危害和熬夜一样属于慢性积累的危害,自己当时的感觉并不明显,但只要持续一段时间,恶果就会显现出来。所以大家对久坐和熬夜一定一定要认识到位,否则对你身体恢复实在太不利了。切记!

结语:热爱运动,不熬夜,不久坐的戒友恢复情况比较乐观,光戒是不行的,要学会养生之道。希望这篇文章能带给大家有益的启示,大家加油!

\subsection{身高问题、脱发问题、痤疮和戒断反应}

这季就身高、脱发、痤疮、戒断反应这四个方面谈一下,前面三个是戒友比较关心的问题,而戒断反应则是戒色后比较常见的问题,详细论述如下:

很多戒友都会问,SY 究竟会不会影响身高,我的回答是:SY 的确会影响到身高,会影响到骨骼的发育。因为中医:肾主骨。问身高问题比较多的是 90 后,因为他们这一代正处在发育期,80 后问得比较少,因为 80 后基本都长定型了。很多人会产生这样的疑问,大家都 SY,为何有的人还是长得很高,为何有的人就长不高,其实这个问题很好回答,因为影响身高的因素有很多,SY 只是其中一个因素,很多人虽然 SY 频繁,但其他影响身高的因素他都避免掉了,这样对身高的影响就不明显。就拿我来说吧,我爸妈都不高,我妈 160,我爸 170,而我 186,虽然我发育期在频繁 SY,而且我的遗传基因也不好,但我其他方面做得很好,所以我依然能长到 186,虽然我能长到 186,但是我明显感觉我的骨密度不行,有点骨质疏松,运动时容易崴脚和骨折。后来我学了医理才知道,肾气不足会导致骨质疏松,而且一旦受伤,也不容易好彻底。据我了解,很多人经过一段时间的精心调理,虽然外表并没有长胖的迹象,但是一测体重,的确增重了,哪里增重了呢?其实就是骨密度,原来骨质疏松,经过调理,骨密度上去了,体重就上去了,外表是看不出来的。

我总结的影响身高的因素如下:

\begin{enumerate}
    \item 遗传基因
    \item 是否积极锻炼
    \item 是否熬夜久坐
    \item 营养如何
    \item 睡眠质量如何
    \item 消化能力如何
    \item 是否 SY
    \item 心理是否健康
\end{enumerate}

影响身高的因素一般就这 8 个方面,很多文章只谈到 5 个方面,有的文章认为遗传占主要因素,有的文章则认为遗传只占 30\%,我比较倾向遗传只占 30\% 的论点,因为我就是活生生的例子,我发育期虽然频繁 SY,但我其他方面做得很好,第一项和第七项我得分低,也就是遗传和 SY 这两项我得分比较低,但是我其他 6 项得分都很高,我发育期经常打篮球,而且是户外阳光下的篮球,多晒太阳有助于增高,因为阳光的照射能促进人体合成维生素 D,有利于骨骼中钙质的积累、沉淀,使骨骼快速生长,机体快速长高。积极锻炼对骨骼的发育是非常有好处的,是会促进骨骼发育的,我也从来没有熬夜和久坐,我天性爱动,很少一坐几小时,一般 9 点半就睡觉了,睡眠质量很高,消化能力也很强,吃得下,睡得好,很多人能吃,但消化不行,吃下去也不吸收,所以消化能力非常重要。另外,心理健康也很重要,中医讲:七情致病,情绪会导致疾病,会导致内分泌紊乱,这样也是会影响到身高的。90 后的戒友如果你想长高最好在这 8 个方面好好下功夫,SY 当然应该要戒掉,这样更有利于长高,戒掉 SY,长出来的骨密度也高。


下面谈下脱发问题。

脱发问题一般是 SY 10 年以上的戒友会遇到的问题,80 后的戒友遇到脱发问题可能性比较大,导致脱发的因素也是有很多的,SY 导致的肾虚型脱发就是比较普遍的原因。脱发问题是非常愁人的问题,因为头发对于一个人的外貌至关重要,出现秃顶倾向给人的感觉就是未老先衰,精气神不行,很多戒友每天掉 100 根以上的头发,看到梳子上和脸盆里的头发,人会产生一种恐慌感,对于秃顶的恐慌,心乱如麻,对头发会变得异常敏感、异常在乎,掉一根头发都会让人神经质好几个小时,生活在一种担心、恐慌的情绪中,这样的心理又会加重脱发,因为中医:恐伤肾,发为肾之华。所以就会陷入恶性循环,到处找治疗脱发的药物,但不管何种药物效果都不会理想,于是心理压力会变得很大,很在意别人的眼光,也害怕照镜子,一看头发又少了,连自杀的心都会有,脱发问题就是让人处在一种心理困境之中,无法自拔,而很多人会把压力转变成 SY 行为,这样头发就会掉得更厉害,最后真有可能变成“地中海”。

很多戒友的发际线已经开始后移了,有人则是头顶先秃,我以前就是头顶先秃,有时一天能掉几百根头发,一梳头发,肩膀上就几百根,太恐怖了!然后这样掉了一个月,头顶就有点看见头皮了,好在当时我已经看清无害论的真面目,并且对中医医理有了比较深入的理解,于是我赶紧戒掉 SY,积极锻炼,按时作息,慢慢把肾气养足,差不多花了半年时间,我掉头发恢复到每天 5 根以内,现在梳头发,梳子上看不见有头发,枕头上也很少有头发了。

我现在更不敢 SY 了,因为我深知危害之惨烈。很多戒友处在无知状态,无知者无畏,对SY导致的恶果认识不深不全面,甚至不少戒友还迷在无害论的泥潭里,脱发了也不知道真正原因,继续 SY,把压力变成 SY 行为,这样只会越陷越深,陷入恶性循环,头发就更难恢复了,因为认识上有盲区,有误区,这样使用任何药物都不会有多大用处,永远无法治根,肾气不养足,头发问题就无法解决。就像一盆植物,外行看到叶子枯萎了,就会认为是叶子的问题,于是把叶子摘掉,但是没过多久,叶子全枯萎了,植物就死掉了,换做内行,一看就知道是根的问题,马上打开土壤,在根上用药,这样植物才会继续存活,头发也是这个道理,一定要认识到根上的原因,不能迷在表面,靠几个所谓的产品,吹得天花乱坠的产品,就能治愈脱发,那是不现实的。即使你去植发了,只要肾气不足,依然会继续脱发,除非你定期去植发,掉了就植,那样费用太大了,一般人用不起。


导致脱发的因素如下:

\begin{enumerate}
    \item 遗传因素(父母)
    \item 雄性激素分泌过高(雄秃)
    \item 熬夜久坐(伤精伤肾)
    \item 纵欲过度(肾虚导致的肾秃)
    \item 饮食习惯(吃得太咸容易脱发)
    \item 心理因素(压力过大脱发)
    \item 疾病因素(很多疾病导致脱发)
    \item 清洁因素(头皮屑堵塞毛孔)
    \item 季节性脱发(一般夏季容易脱发)
    \item 营养性脱发(严重营养不良导致脱发)
    \item 物理化学脱发(外在刺激影响脱发)
\end{enumerate}

再来谈下青春痘痤疮。

很多人认为过了青春期,痤疮就会自己好了,其实这种看法是错误的,我见过很多人 30 多岁 40 多岁,依然一脸粉刺痤疮青春痘,那其实已经不叫青春痘了,就是内分泌紊乱造成的,不是你过了青春期就能好的,一般的痤疮,只要注意休息和饮食清淡,很快就能好,但是 SY 导致的顽固性痤疮就很难好了,非常顽固,必须戒掉 SY,积极锻炼,养足肾气才有望痊愈,否则那种五脏失调的状态可能会一直持续下去,伴随终身,脸是毁容了,彻底废掉了,精气神更别谈了,自卑至极,一脸脓包粉刺,像个怪物,惨不忍睹。和发育前那种清爽肤质,一个天一个地,一 SY,就出油,去医院就说痤疮,溢脂性皮炎,开药调理,时好时坏,因为依然在 SY,只要还在 SY,吃药就无效,只能暂时缓解,无法去根。不少戒友都变丑了,变丑的占绝大多数,因为 SY 导致肾气亏损,肾气一走漏,五脏功能就容易紊乱,五脏一失调,就会表现在脸上,因为脸是五脏的镜子,脸的气色就会不行,感觉萎靡猥琐,得上痤疮的可能性就会很大。

总之,要彻底治愈顽固性痤疮,必须戒掉 SY,养成良好的作息饮食习惯,积极锻炼,这样才有望恢复健康肤质。否则,免谈。

另外,有的人虽然没有得上顽固性痤疮,但脸部的气色感觉像鬼一样,没有阳光的感觉,只有晦暗,甚至会出现凹陷,感觉就像骷髅包了一层皮,没有生气没有活力,未老先衰。年纪轻轻纵欲过度,把自己搞得人不人,鬼不鬼,就像泄了气的皮球。那种感觉就像旧社会吸鸦片上瘾的人,网瘾 + 烟瘾 + 性瘾,这样的身体能好吗?所以戒为王道,不戒就是在害自己。有的人身体好,报应出来就晚,但只要在纵欲,迟早会出来的,没有人可以逃得掉。不注重养生,身体就特别容易出问题,人的身体就像机器一样,需要保养,不保养就容易磨损折旧,好比一辆新的自行车,注意保养,一年后还可以 9 成新,但如果你不保养,一年后可能只有 5 成新了,或者已经坏掉了。这就是保养意识的重要性,不注重保养,不会长久。

最后再来谈谈戒断反应。

我研究过很多成瘾行为,包括酗酒、抽烟、吸毒、网瘾、性瘾、购物上瘾等,凡是成瘾的行为要戒掉几乎都会出现戒断反应,有的人表现严重,有的人表现轻微。而 SY 导致的性瘾,戒断后也会出现戒断反应,绝大多数戒友在戒掉后都会出症状,而 SY 时却没表现出来,从戒友的发言来看,绝大部分的戒断反应表现为泌尿系统的问题,以前列腺炎为主,其次为睡眠障碍和烦躁情绪等。

戒断反应可以这样理解,和中医的排病反应有相似之处,很多中药吃下去会有排病反应,就是反而严重了,这其实是好现象,是正邪在交战,只要继续吃药,正气就会占上风,这样疾病就会慢慢痊愈,很多人不懂得排病反应,感觉严重了,就以为吃错了,就不吃了,这其实就是耽误治疗了,古语有云:为人子弟不可不知医,因为你知道医理了,才能更好地配合医生治疗。

戒断反应的机制在西医的研究来讲就是:由于长期 SY 后,突然停止引起的适应性反跳。例如酒精戒断后出现的是兴奋、失眠,甚至癫痫发作等症状群。吸烟者在强制戒烟之后也可能会出现诸如焦躁不安、失眠、食欲增强、吐黑灰色痰、血压升高以及心律不齐等戒断反应,会产生极大的痛苦,但是这种反应大多数会随着体质的恢复而逐渐消失。一般 SY 戒断后,出现戒断反应的时间为一个月内,有的人戒了几天就会出现戒断反应,出现戒断反应不要害怕,积极锻炼,注意休养,戒断反应会自动消失的。

\subsection{耳鸣问题、戒色心态、搜集资料的重要性}

耳鸣是戒友中比较常见的问题,当然导致耳鸣有很多原因,一般戒友出现的则是肾虚型耳鸣,就是因为纵欲过度,另外加上生活习惯不正常,经常熬夜久坐,这样就特别容易出现耳鸣问题。

中医:肾开窍于耳。肾一虚,耳鸣问题就会找上门来。很多戒友被耳鸣问题所深深困扰,久治不愈,把希望寄托在药物身上,殊不知戒色才是根本前提,如果你一边吃药还一边 SY 和 YY,这样耳鸣还能看好吗?我曾经也被耳鸣问题所困扰,当然那时我处在比较无知的状态,后来我戒色后学会养生,积极锻炼半年,养足肾气,耳鸣就不药而愈了,没吃任何药物!最大的补药就是不泄为补。另外,遗精也算泄,对身体也是有伤害的,如果你肾气充足,偶尔一次遗精并不会感觉到什么,但是,如果你肾气亏损了,再遗精,就是雪上加霜。我回答的问题中,有不少戒友都被遗精问题所困扰,频繁遗精,身体越来越不行,中医有讲到:久遗八脉皆伤。所以耳鸣问题要恢复,也必须在减少遗精上下功夫,我第三季推荐的八段锦固肾功,建议大家坚持做,不少戒友都取得了良好的效果,大大减少了遗精次数,当然,前提是必须找对拉紧感,并且杜绝其他导致遗精的因素。

肾虚型耳鸣,一般患者听到的是蝉鸣、汽笛或者嗡嗡、嘶嘶的声音。我听到的是蝉鸣,好像耳朵里面养了一只知了,很尖细而悠长的那种,很烦人,有时会产生睡眠障碍。这种困扰其实在我高中时就出现了,只是当时并不严重,后来我有熬夜和久坐,耳鸣问题就变得严重起来了。一般肾虚型耳鸣的出现都会伴有其他肾虚症状,并不是单一的耳鸣问题,出现的症状一般分为以下六个方面:

\begin{description}
    \item[脑力方面] 记忆力下降记忆力减退,注意力不集中,精力不足,工作效率降低。
    \item[情志方面] 情绪不佳情绪常难以自控,头晕,易怒,烦躁,焦虑,抑郁等。
    \item[意志方面] 缺乏自信信心不足,工作没热情,生活没激情,没有目标和方向。
    \item[性功能方面] 性欲降低,阳痿或举而不坚,遗精、滑精、早泄,显微镜检查可见精减少或精活动力减低,不育。
    \item[泌尿方面] 尿频,尿等待,小便清长,滴白,前列腺炎,精索静脉曲张。
    \item[其他方面] 早衰健忘失眠,食欲不振,骨骼与关节疼痛,腰膝酸软,不耐疲劳,乏力,视力减退。脱发白发头发脱落或须发早白,牙齿松动易落等。
\end{description}

SY 导致肾虚,肾一虚,就会出现很多身心问题,因为 SY 摧残的是身心,耳鸣问题就是众多肾虚症状的一个表现而已,所以要彻底治愈耳鸣,必须要彻底戒掉 SY 和 YY,积极锻炼,按时饮食作息,养足肾气,肾气足,万邪熄。否则,吃再多药也无济于事,到后来可能药都吃疲了,耳鸣都还没好。很多戒友反映戒掉 2 个月,3 个月,身体还没多大改善,我要说的是,光戒是不行的,必须学会养生之道,积极锻炼,这样肾气恢复才快,而且很多戒友伤精程度严重,恢复也相对较慢,我自己的感觉是,恢复速度就像头发生长的速度,真的很慢,但是如果你半年不剃头,你会发现头发很长了,2 个月 3 个月并不会感觉有多长,所以肾虚要恢复,要做好持久战的心理准备,好好坚持每一天。另外,也要在尽量减少遗精次数上下功夫,因为频繁遗精对于恢复是很不利的。

再来谈下戒色心态的调整。

因为 SY 很多戒友的心理都出现了问题,出现悲观厌世,自暴自弃的人非常多,我曾经也有这种情绪上的困扰,看不到希望,做什么都没动力,缺乏自信,人很自卑乃至会自残。

出现这类心理问题也很正常,因为 SY 摧残的是身心,身体出问题,心理也会出问题。身体出问题,调理身体,心理出问题,也要学会及时调整。

大家一定要知道,这个世界上永远有两个我,一个消极的我,一个积极的我。当出现消极的我时,一定要尽快调整到积极的我。就像钟表走错了,赶紧调整到正确的时间。

这种调整心理状态的能力,在西方心理学可以归为 EQ(情商),也就是情绪智商,是近年来心理学家们提出的与智商(IQ)相对应的概念。它主要是指人在情绪、情感、意志、耐受挫折等方面的品质。总的来讲,人与人之间的情商并无明显的先天差别,更多与后天的培养息息相关。也就是说,情商是后天“习得”的,通过学习情商是可以提高的。

前段时间土豆吧主推荐的书《秘密》,就是一本很好的提高 EQ 的书,里面讲到的吸引力法则,就是让人多想好的方面,多想积极的方面,这样就更容易成功,如果一直想消极灰暗的方面,其实就是在对自己进行催眠,进行暗示,暗示自己会失败,这样失败的几率就更大,就像打仗时士气不行,这样战败的可能性就会很大。

情绪智商包含五个主要方面:

\begin{enumerate}
    \item 认识自身的情绪,只有认识自己,才能成为自己生活的主宰。
    \item 能妥善管理自己的情绪,即能调控自己。
    \item 自我激励,它能够使人走出生命中的低潮,重新出发。
    \item 认知他人的情绪,这是与他人正常交往,实现顺利沟通的基础。
    \item 人际关系的管理,即领导和管理能力。
\end{enumerate}

这些能力通过不断学习是可以获得的,建议多看这方面的书籍,书籍是人类进步的阶梯,不学习无法开悟,不学习戒色也不会成功。

最后来谈一下搜集第一手资料的重要性。

第一手资料就是 SY 戒友的经历和症状表现,这些资料这些案例非常非常珍贵,很多案例都很典型,大家最好能建立个文件夹,把看到的典型案例搜集起来,这样有助于自己对 SY 的危害更深入更透彻地认识。很多戒友会搜集 H 片,当然我也这样干过,搜集过几十部,多搜集一部,自己就多了一次放纵的机会,而搜集 SY 危害案例,多搜集一个案例,就是对自己多一次警告,警告自己不能放纵,放纵就和他一样,放纵就是在害自己。我每天上戒色吧看到好的案例就会搜集起来,阅戒友无数,什么症状都见过,很多症状自己都曾经有过,看得越多越全面,认识就越深刻。

不管做什么研究,要深入理解和认识,第一手资料第一手案例就显得无比重要,SY 有害和无害其实根本不需要争论,有句话叫:实践出真知。有没有害,实际真相如何,多看看身体垮掉的戒友是怎么说的,从这些受害者口中出来的就是最真的真相。适度无害论根本就是扯淡,我阅戒友无数,没见过几个能真正做到适度的,看到最多的就是“一发不可收拾”,好比打开了潘多拉的魔盒,打开就收不住,热爱运动和生活习惯良好的人出来的症状轻微,不爱运动,熬夜久坐的人出来的症状就严重。可以说,只要开始放纵,就没有人可以逃得掉,万法皆空,因果不空,SY 是在种恶因,恶因导致恶果,什么种子结什么果,SY 出来的就是恶果。

不管身体如何好,随着年纪的上升,症状会越来越明显,身体垮掉比较早的戒友,其实也是福分,为什么这么说呢,因为症状才是最好的老师,出症状了,就知道要戒色了,就认识到了真相。否则很多人在 50 岁时身体才垮,那时要恢复就更难了,现在 20 多岁垮掉,可以提早认识到危害的严重性,年纪轻也容易恢复些,否则 50 多岁再垮掉,就难以恢复了。我聊到过一个 50 岁的戒友,他作息饮食相当规律,不熬夜不久坐,每天晨跑锻炼,几十年如一日,身体非常强壮非常好,但只有一点他没做好,也没认识到危害,他说他几乎每天都过 2 次性生活,早上一次,晚上一次,持续了几十年,他说,到 40 岁以后,他明显感觉身体大不如前,但并未引起重视,继续纵欲,到 50 岁时突然身体出了很多症状,一查是植物神经紊乱和焦虑症,每天活在症状的地狱里,相当苦恼,这个案例就说明了,只要纵欲,身体迟早会出问题的,即使你作息饮食规律,即使你热爱运动,还是会出症状,随着年纪的上升,症状会越来越多。所以,必须戒掉 SY 恶习,即使婚后也要懂得保精,节制性生活,否则身体还是会出问题的。

前车之鉴,后事之师,这些案例就是最好的警示,最好的真相,最好的老师,希望大家能做个有心人,把搜集 H 片的邪恶兴趣转变成搜集 SY 案例。每一个案例都是在提醒你,每一个案例都是在警告你,警告你:千万不要 SY,千万不要纵欲,否则你也会和他一样。

另外,75 党一直在干撒播无害论的事情,因为青少年缺乏辨别力,所以一看到无害论就糊涂,分不清真假,所以,青少年更要多看案例,从前辈的经验教训中多学习,一定要把无害论从自己大脑中清除掉,那是有毒的思想,必须清除掉。

\subsection{SY 变丑详细论述和神衰、焦虑症、社恐的康复}

如果你在戒色吧泡久了就会知道,SY 变丑是极其普遍的现象,基本每个人都变丑了,但是,每个人变丑程度不一样,有的人变丑严重,有的人变丑轻微。

影响变丑的因素如下:

\begin{enumerate}
    \item 先天体质(先天体质好,变丑程度轻)
    \item 运动因素(不热爱运动的人,变丑程度严重)
    \item 熬夜久坐(有不良生活习惯的人,变丑程度重)
    \item SY 上瘾程度(上瘾严重的人变丑程度严重)
    \item 营养吸收(营养跟不上,吸收能力差,变丑严重)
    \item 其他瘾(如有网瘾、烟瘾,酒瘾,变丑更严重)
\end{enumerate}

很多人变丑了都不知道原因,其实只要在 SY,变丑就是必然的,SY 泄掉的就是人体最宝贵的精气神,精气神没了,人就蔫掉了,大家都知道选美比赛,一,比五官身材,二,比的就是精气神,三,比的是谈吐修养。如果一个选手有黑眼圈,并且面色暗沉晦暗,带着粉刺痤疮青春痘,评委会选这样的选手吗?一个人可以不漂亮,但一定要有精气神。很多人身体底子好,热爱运动,没有不良生活习惯,变丑程度就轻微,但只要你仔细观察,还是能发现问题的,我以前一个朋友,他身体很好,也一直 SY,变丑虽不严重,但你看他眼神就知道不对,他眼神总是有点呆滞,无神采。中医:五脏六腑之精气皆上注于目!而肾藏精,藏五脏六腑精华之气,SY 伤肾泄掉精气后,人的眼神就容易暗淡,没有神采。

SY 后皮肤也特别容易出问题,SY 导致内分泌紊乱,皮肤出油严重,痤疮青春痘频发,而且是顽固性的,几年乃至十几年都好不了,因为一直在SY。就医治疗,吃药涂药也不能根治,因为 SY 恶习还在泄精气。毛孔也容易粗大,本来发育前的肤质是格外清爽的,无油的,细腻的,SY 后肤质就差太多了,毛孔粗大污秽,看了就让人恶心。脸部气色也很差,一点不阳光,肤色暗沉,好像怎么洗也洗不干净,其实是真的洗不干净,因为这种脏是由内而外的脏,你外面暂时洗干净了,但里面还是不干净,是怎么洗怎么搓,也弄不干净的,SY 伤的是肾气,而肾脏有一个特别重要的功能,这个功能就是:过滤血液。通俗点说就是“洗血”,尿毒症的人肾脏不行了,就靠机器透析血液,人体的肾脏一天要把周身的血液过滤几十遍,以保持血液的纯净新鲜,就是在你睡觉的时候,肾脏还是在帮你“洗血”。SY 恶习伤肾,也会影响到肾脏的过滤功能,血液洗不干净,这样反映到脸上,就是脸洗不干净,看上去很脏,很晦暗。

很多人觉得自己并没有变丑,其实已经变丑了,你把自己前几年的照片拿出来比比就知道了,SY 的变丑并不是五官的显著变化,而是精气神的微妙变化,我打个比方你就懂了,比如你去水果店买苹果,你总要挑苹果吧,你肯定会挑新鲜的苹果,就是那种光鲜饱满的苹果,对于颜色暗沉,形态萎缩的苹果肯定是不要的,因为你知道那种苹果不新鲜了,口感肯定不好了。同理,SY 后的你就像“过期的苹果”,虽然五官形态没多大变化,但一看就知道,和过去那种饱满的精气神差远了。当然,当精气神亏损到一定程度,也是会造成五官的变化的,不少戒友眼眶深陷,颧骨突起,或者眼睛变小了,这种情况都有。

现在的化妆术真的很神奇,化妆后丑女能变美女,真是人类第八大奇迹,化妆除了能改变眼形、唇形、眉形以外,还能修饰脸型,欺骗性很强,很多女人化妆前后判若两人,化妆术除了改变形态以外,还有一个很重要的作用,就是“提亮”,提亮是什么意思呢,其实就是“补气血”,“补精气神”,很多女人其实精气神已经萎靡了,肤质很差,但是化妆后,精气神立马就上去了,本来暗沉粗糙的肤质,一下变得透亮细腻了,和婴儿的差不多。但是,一卸妆又被打回原形,所以有听说,很多女人结婚后,睡觉都不卸妆,怕被老公看到自己真面目。而男人最好的化妆品其实就是戒色和运动,加上学会养生之道,不熬夜不久坐,3 个月左右精气神就能恢复不少,比在脸上涂粉好多了,是真的精气神,而不是虚假的遮盖提亮。

脱发问题我在前面的文章中也讲到过了,脱发对于一个人的外貌是莫大的伤害,头发一秃就显老了,而且脱发问题恢复比较慢,很多戒友思想也存在严重误区,这样头发就更难恢复了,女孩子一般是不会喜欢秃顶的男人的,除非这个男人很有钱,这种情况,不知是喜欢你的人,还是喜欢你的钱,值得怀疑。而且有了女人后,脱发就更别想好了,过性生活是会不断耗损肾精的,肾精没了,头发就别想恢复到以前那种浓密的状态了。当然,除了脱发,白发问题也是一大困扰,白发较多,也会影响到人的自信。

总之,SY 后变丑是必然的,随着年龄的上升,这种变丑的倾向会更明显。所以要尽量避免 SY,积极锻炼,按时作息,学会养生之道,养足肾气,养颜和养生是相通的,美是由内而外的,不能只做表面功夫,切记。中医:脸是五脏的镜子。脸上不同的区域对应着五脏,这季我会分享 2 张中医面部全息图,大家可以看看,中医专门有面诊和色诊,中医“望而知之谓之神,闻而知之谓之圣,问而知之谓之工,脉而知之谓之巧。”真正高明的中医看你的脸色就知道你五脏的问题,大家都知道扁鹊见蔡桓公,名医扁鹊就是色诊高手,一看就知道身体潜在的疾病。

以下是我搜集的 SY 变丑真实案例,因为变丑的案例实在太多,我就选了十例供大家参考:

\begin{enumerate}
    \item 嗯,SY 变丑这个我深有体会,男人最重要的不是长相,是精气神,我本来长的就不咋地,但是初中那时候不 SY,整个人看起来也很精神,脸上清亮有光彩,也从来没觉得自己丑过,初三开始 SY 之后到高中再到现在自己真是慢慢变丑了,面色晦暗,神形猥琐,戒了一段时间之后就能感到自己变帅些,一破戒就又变回去了。
    \item 从初中的时候开始 SY,到现在已经 4、5 年了,几乎每天一次,手淫真的很害人!我以前的身体很好,冬天的时候都在屋子里光膀子。但是现在手脚总是冰凉。就算在热水里泡也没有用 以前初中的时候就 180 cm 了,到现在都高三了。个子居然没有长多少而且我以前脸上从来没长痘痘。这几年长了好多。SY 久了也会变丑的。好多人见到我都说我变了样子。以前初中还有几个女生追过我。。到现在一个都没了。手淫还会让人变得很没有自信。让人变得很黑暗。SY 久了视力还会下降,大脑会变得迟钝,耳朵总听不清楚东西,其实不是耳朵的毛病,而是肾虚的原因。
    \item 我是名上高二 18 岁男生,从初一开始 SY 已 4 年之久,所表现的症状有:精神不好,上课困倦睡觉,注意力不集中、感觉智商变低、黑眼圈、眼睛没神、皮肤暗黄粗糙、感觉连长相都变丑了(原来长得很帅气)鼻炎、做事没毅力、性格变得易怒、多疑、(原来的我性格开朗,聪明,心中阳光)现在尽胡思乱想、做事犹豫、请问各位前辈们我还能恢复到原来的那个“我”吗?请您给点建议,谢谢。
    \item 我才 18 岁,SY 已经二年多了,以前一天一次有时甚至一天三四次,现在不仅身体差了,样子都变了,SY 人会变丑,绝对没有错!以前都说我是小正太,现在头大身子小,人看起来好像天天没睡觉似的,皮肤会发黄,记忆力会下降,人会越来越自卑。
    \item 我就是 SY 变丑的,而且染上心理障碍,愤恨!!!阳光的生活不去感觉,自甘堕落的悲哀!誓要改变!
    \item 曾经看吧里有人问 sy 会不会使人变丑?关于这个问题我特意做过对比,前段时间拿我 15、6 岁时候的照片和现在现在的照片比了比,简直判若两人!一个英俊帅气、天真漂亮(实事求是),一个脸型浮肿、满脸死气,不信你也可以对比一下(不要照镜子,照片最明显)。用别人的评价就是——”长残了”。
    \item 我小时候长得还算帅,可是到了初中我老 sy,现在越来越丑!那些小时候没我帅的从来不 sy,现在越来越帅,我郁闷呀!
    \item 我的脸上现在的确有很多很痘痘和痤疮,而且还有很多白发,第一张是我以前,第二张是 SY 以后的我,没错我变丑了!一切都晚了!我后悔当初无知的我,我好想回到从前,重新做人。我现在就是个黑暗少年,SY 毁了我的人生,本该非常有前途的我变成了现在这个样子。
    \item 我 SY 10 年了,已是行尸走肉,如今一事无成,相貌变丑,这都是 SY 所赐!没谈过恋爱,没女孩子喜欢、懦弱、没血性、不男人,这也是 SY 所赐!
    \item 距上次开贴已经九十多天了,在这段日子里,虽然没彻底戒除,但 SY 频率较从前大大减少,也可以算是一个小小的成功吧。可是我身体的衰颓还在继续,我的原本帅气的相貌仍在变丑,每思及此,便痛心疾首。再次立誓开贴,自己的目标是 100 天。我已经没有退路了,一定要坚持下去。我相信我的身体会慢慢变好的,我的相貌会慢慢变庄严的,我的学业、事业会有建树的,我也会找到女朋友的。也欢迎各位兄弟与我一起努力吧。
\end{enumerate}

附手淫变丑类型二十三种

\begin{enumerate}
    \item 痤疮青春痘
    \item 脱发白发
    \item 气色萎靡晦暗
    \item 眼袋黑眼圈
    \item 眼眶深陷,颧骨突起
    \item 眼睛无神空洞,眉毛散乱
    \item 牙齿不整,易脱落
    \item 发质下降枯黄变卷,皮肤出油
    \item 毛孔粗大,肤质奇差,皮肤松弛
    \item 皱纹增多,嘴唇变厚
    \item 猥琐表情,眼睛变小
    \item 驼背,身体不对称等
    \item 脸部变形(浮肿瘦削等)
    \item 眼皮变化(双变单或者多眼皮)
    \item 眼球浑浊,出现血丝
    \item 脸色变灰暗(带鬼气)
    \item 明显感觉变老
    \item 骨骼细小(发育障碍)
    \item 眼睛带邪光,没定力
    \item 太阳穴凹陷,苹果肌凹陷
    \item 脸部出现斑点,黑痣
    \item 脸部显脏(很难洗净)
    \item 气质严重下降
\end{enumerate}

下面再来谈下神衰、焦虑症、植物神经紊乱、社恐。

这四个病症其实都大同小异,只是侧重点有所不同,共同点就是有一大堆躯体症状反复纠缠你,好比人间地狱,更准确地说应该是症状地狱。没有经历过的人很难明白其中的真切感受,就好像你向一个没见过大象的人描述大象,无论描述得多么细致,都略显苍白。我作为一个深刻体验者,也作为一个痊愈者,我希望把我痊愈的经历和研究的结果和大家分享,关于焦虑症和植物神经紊乱我研究了 2 年多,那时我因为严重社恐足不出户,每天十几个 QQ 群一起聊,聊了上千个病友,男的居多,有上千个,女的较少,大概 100 多个,可以说病因无一例外如下:

\begin{description}
    \item[男病友] 熬夜、久坐、纵欲导致的。
    \item[女病友] 生气、压力大、久坐。(也包括家庭变故,如亲人去世的打击)
\end{description}

男病友在一段时间的放纵生活后突然发病,发病往往有一个导火索,一个刺激,但根本原因还是身体虚了。女病友发病几乎都是生气,一生气,植物神经就紊乱了,女病友本来就有特殊生理期,气血容易虚衰,如果再一生气,就容易气出毛病来。对于男病友,如果想要痊愈,一定要学会养生之道,最好能彻底戒色,否则很难痊愈,对于女病友,则是一定要注重修心,中医讲七情致病,不良情绪就可以导致疾病,所以一定要好好管理自己的情绪,不要生气,要学会心平气和地为人处事。

一般得了此类病症的病友会有两个倾向:

\begin{enumerate}
    \item 疑病
    \item 敌对
\end{enumerate}

有疑病倾向的戒友太多太多,大多数病友检查费都上万,检查费十几万的都有。检查了一圈什么也没检查出来,但是症状却是很严重的,让人崩溃让人恐慌的,在这种情况下,就会反复检查,恨不得把医院所有检查器械都过一遍,甚至检查一遍还不够,还要检查好几遍,仍不放心,因为症状是明显的,但怎么会检查不出来呢,会产生这种疑问。其实这还是认识上有误区,像神衰,焦虑症,植物神经紊乱,基本都是功能性的问题为主,根本检查不出什么的,做再多检查也是枉然。

敌对倾向就更普遍了,去患者群聊天,经常有人吵起来,原因就是肾水不足,肝火就大,会常常发无名火,喜欢抬杠,没事找事,很多人在发完脾气后又感到后悔,其实发怒伤人更伤自己,怒伤肝。所以我建议病友一定要学会修心,控制自己的情绪,管理好自己的情绪,这样更有利于康复。

另外大家在群里都会聊症状,不聊病因,这是最可悲的,如果对病因认识不清,就非常难痊愈,认识到病因了,反其道行之,就能痊愈,很多病友就是聊症状,聊症状是可以起到缓解紧张情另外大家在群里都会聊症状,不聊病因,这是最可悲的,如果对病因认识不清,就非常难痊愈,认识到病因了,反其道行之,就能痊愈,很多病友就是聊症状,聊症状是可以起到缓解紧张情作用,但对于痊愈并无实质性的帮助,要痊愈,必须认识到病因!上像戴了个帽子,像个紧箍咒,然后睡眠严重障碍,一直失眠,但他还是没认识到病因,还是想和我聊症状,后来我告诉他病因,他才恍然大悟,如果不明白病因,不改变生活习惯,不戒掉 SY 恶习,要痊愈真的是太难太难了,很多人十几年都没出来,这十几年每天都靠药维持着,简直难以想象,我当年也吃了半年药,每天都吃药,严重依赖上了药物,那种经历太可怕了,吃药当家常便饭了,不吃药就难受,其实吃了也没好多少,就是那种怪圈似的心理状态,后来我对药彻底失望了,就不吃了,戒色后积极锻炼,学会养生之道,半年就好了一半,一年后基本无症状,到现在就彻底恢复了,身心都回到了正常的轨道。要痊愈,必须认识正确,否则如果认识上有盲点,有误区,就万难痊愈了。

其实神衰、焦虑症、植物神经紊乱、社恐,可以算一个病,就是神经出问题了,表现都大同小异,在中医上来说,就是身体虚掉了,或者气血瘀滞了。


一般症状表现为:
\begin{enumerate}
    \item 头部症状,包括失眠在内。
    \item 心脏症状,严重的心脏官能症。
    \item 肠胃症状,肠胃功能的紊乱。
    \item 皮肤症状,包括荨麻疹、肌肉跳、刺痛、蚁走感等。
    \item 全身症状,没一个地方舒服的。
\end{enumerate}

最后再补充一下婚前性行为的危害,这几天我回答问题,遇见好几个戒友,因为婚前放纵,他们现在身体已经废掉了,出现了阳痿早泄倾向,家里又催着结婚,所有心理很矛盾,这类戒友就是“未婚先虚”,就是没结婚,身体已经不行了,肾气亏损严重,以这种身体状况去结婚,能保证正常性生活的质量吗?很多戒友生育功能都产生了障碍,出现了不孕不育,而且不仅仅是阳痿问题,全身都有症状,身体一塌糊涂,在“虚则亢”状态没有引起警惕,我说过,虚则亢下面就是阳痿早泄倾向,很多戒友在虚则亢这个阶段,每天一次,甚至一天几次,还以为自己很行,以为自己身体很强壮,其实是身体虚衰的表现,老子说:知常曰明,不知常妄作凶。常指的就是规律。很多戒友不明白这个道理,疯狂纵欲,结果呢?没多久身体就垮掉了,未婚先虚,再结婚,一碰女人,结果就是“虚上加虚”,将来可能会出现更严重的疾病。所以,一定要杜绝婚前性行为,否则后果很严重,多看看出现问题的戒友,引以为戒,防患未然,此乃高明之举!

\subsection{破戒后的心态调整、精不液化、JJ长度问题}

戒色吧每天都有破戒的戒友,破戒后发帖总是后悔得不行,有些戒友会把破戒叫“阵亡”,这种叫法我觉得很好,戒色就像一场战役,大家都是一条战壕里的兄弟,看到别人阵亡的确很痛心,当然更要引以为戒,保持警惕,粉色子弹不断从对面阵地射过来,戒色文章就是我们的防弹衣,我们必须多学习戒色文章提高觉悟和定力等级,否则如果你放松警惕,下一个阵亡的可能就是你了。看到不良图片和视频马上要避开,畏色如畏虎,避色如避箭,这种防范意识一定要有。要戒掉,意识一定要强,意识是通过学习获得的,有了意识,有了警惕,戒色才有望成功。

对于破戒的戒友,也不用灰心丧气,没有人可以一次戒除成功,在成功前必定有无数次的失败,我在彻底戒除前,也失败了无数次,我已经记不清我破过多少次戒了,我有十几年都没走出那个怪圈,因为那时的我处在无知的状态,没人指点我,没人告诉我该怎么做,后来我通过学习开悟后,定力就上去了,大概学习了一年多的戒色文章和养生知识,认识有了极大的飞跃,有了更深刻的思考和认识,是在这种思想状态下才彻底戒掉的。

一般戒色后容易犯 2 类错误:

\begin{enumerate}
    \item 只戒不学
    \item 只戒不养
\end{enumerate}

只戒不学就是强戒和盲戒,强戒和盲戒注定失败,因为定力没提高,欲望休眠期过后就会进入破戒高峰期,到了破戒高峰期,破戒的欲望挡都挡不住,往往是连续破戒,前功尽弃。我在前面的文章中无数次地提到了学习戒色文章的重要性,因为不学习你就不知道,不学习你觉悟就不会提高,不学习就无法开悟。当你学到一定程度,定力就会升级,定力上去了,就能降伏心魔,否则见一次心魔,就失败一次。刚开始戒,难免定力等级低,假设心魔等级 100,你定力等级只有 15,你怎么能和心魔抗衡?只有通过学习你的定力才会上去,定力到了一定程度,心魔就动不了你了。失败不可怕,可怕的是不学习,不学习的结果就是,还会继续不断地破戒,出不了怪圈。

只戒不养在戒友中也相当普遍,很多戒友缺乏的正是养生意识,养生意识也是通过学习来获得的,多看名家讲座视频和养生类书籍,这对于身体的恢复是大有好处的。很多人光戒不养生,照样久坐熬夜抽烟,不注重情绪管理,这样的人恢复情况不会乐观,也容易出现反复,而有些戒友戒掉后积极锻炼,注重养生之道,不熬夜不久坐,这样恢复就比较快了。纵欲是伐生,戒色是养生,但戒色只是养生的第一步,光戒是不行的,光戒远远不够,因为伤肾气的方式不仅仅只有 SY 一种,熬夜、久坐、生气、吃冷饮、吹空调等,同样也很伤肾气。

中医专门有讲到五劳七伤,五劳:久视伤血,久卧伤气,久坐伤肉,久立伤骨,久行伤筋,是谓五劳所伤。
七伤:大饱伤脾,大怒气逆伤肝,强力举重久坐湿地伤肾,形寒饮冷伤肺,忧愁思虑伤心,风雨寒暑伤形,恐惧不节伤志。

五劳七伤也要注意避免,养生是一门很深刻的学问,是点滴的功夫,各方面都要注意,等你养生的意识提高后,对你身体更好地恢复是极其有利的。因为你知道了什么该做,什么不该做,伤肾气的事情绝对不做。

破戒后的类型:

\begin{enumerate}
    \item 越挫越勇,斗志依然
    \item 信心受到打击,不积极了
    \item 起了退心,做了逃兵
\end{enumerate}

戒友在破戒后要好好忏悔下,但不要太自责,心理压力也不要太大,一切可以重新开始,把戒色文章捡起来是真的,虽然破戒了,但你要看到,只要坚持戒色文章的学习,终有一天定力会达到的,定力修到了,自然就会彻底戒掉,就像爬山一样,只要坚持下去,终究会爬到山顶的,很多人因为中途劳累就不爬了,真正有决心和勇气的人会坚持到底,绝不会轻易放弃戒色。

经过我的研究,很多戒友之所以放弃,是因为对戒断反应认识不足,一般戒掉后都会出症状,坚持戒色,按时作息,积极锻炼,戒断症状就会消失的,很多人即使戒了半年乃至一年也是会出现反复的,出现反复很正常,不用担心,即使正常人也有身体不适的时候,这和外感有关,还和季节的转换有关,每个季节人体的阳气水平都不同,还有遗精后也容易出现反复,所以要尽量减少遗精次数。另外,有老婆和女友的戒友也容易动摇决心,因为有女人是戒色的一大障碍,有了女人,身体症状就很难恢复了。还有一部分戒友,对 SY 的恶果认识不深,还在认同适度无害论,这类戒友也比较容易起退心。

下面再来谈下精不液化的问题。

精不液化的问题出现得也比较多,也比较普遍,射出来的东西像果冻或者结晶体,很多戒友看到了都吓一跳,很恐慌,其实这种精不液化的现象非常多,在戒色吧,几乎每天都能看见有这类提问。引起精不液化的主要病因是前列腺炎,精不液化是会影响到生育功能和精子质量的,很多人去检查不是弱精、死精、就是畸形的精子,甚至是无精。

在中医来讲,精不液化的主要病因如下:

\begin{enumerate}
    \item 先天肾阳不足,或后天失养,大病久病,戕伐肾阳
    \item 寒邪外袭,损伤肾阳,均可使精寒凝,不得液化。
    \item 酒色房劳过度,频施伐泄,或劳心太甚,或五志化火
    \item 平素嗜食辛辣、醇甘厚腻,湿热内蕴,或外感湿毒
    \item 过食寒凉冷饮,损伤脾阳,或他病伤及脾阳,脾虚及肾
    \item 气虚血瘀或血淤体质,精窍淤阻,精亦不液化
\end{enumerate}

一般戒友出现精不液化的主要原因就是第 3 点:纵欲过度。纵欲过度后就容易出现这类问题,而 SY 只要开始,就会一发不可收拾,几乎没人能做到不过度,这个“度”也没几个人能知道。所以必须彻底戒掉 SY 恶习,否则身体就万难痊愈了,生育功能都会受到影响,我在贴吧回答问题,已经遇见无数因为 SY 恶习而导致无法生育的案例了,唯有戒掉 SY,学会养生,养足肾气,精子质量才有望恢复,否则真的就废掉了,怎么办?!

下面来谈下 JJ 的长度问题,JJ 的长度是个敏感话题。男人对 JJ 长度是很在意的,普遍认为越大越好,JJ 大者有自尊,JJ 短小者自卑。我以前也这样认为,后来我悟道后就不这样认为了,JJ 长度里面有大玄机,不是想象的那么简单,如果不悟道,很少有人知道有这么回事,很少有人知道其背后的深刻道理。今天我就来谈谈这个话题。

JJ 是属肝的,因为肝主筋,很多戒友有反映 SY 后 JJ 短小了,没自信。其实 SY 的确能影响到 JJ 的发育,因为肝肾同源,SY 伤肾也伤肝,而 JJ 属肝,所以会影响到 JJ 的长度,但是影响 JJ 长度的还有基因等其他因素,如果你基因不行,再加上 SY 恶习,这样出现 JJ 短小的可能性就比较大了。

JJ 的长度一般 12 cm 就可以了,10 cm 也还行,JJ 不需要太大,JJ 太大容易招邪,《柳庄神相》里就有讲到:JJ 粗大者主下贱,大者招凶,人必贱。这和大家的认识完全相反,大家普遍认为越大越好,越大越爽,以大为好,以大为强,其实这种认识是比较肤浅的,为什么粗大者反而不好呢?为什么粗大者反而下贱呢?这里面就涉及到中医医理了。如果不懂医理,就会在认识上继续错下去,可能到死都不明白这个道理。

我曾经把 SY 比作购物,花的是肾气,购买的是短暂的快感,而 JJ 粗大者一般“购买力”都比较强,性欲比较旺盛,表面看这似乎是好事,但福兮祸所伏,购买的快感越多,花掉的肾气就越多,这样身体垮掉的可能性就越大。别看他现在强,因为纵欲,将来有他好受的,将来弄不好会早泄阳痿,各种疾病缠身,所以说欲不可强,越强越亏。粗大者招邪,这个邪指的就是邪淫,中医的圣经《黄帝内经》第一章《上古天真论》就专门谈到了纵欲的危害。古代名医经过上千年的经验总结,深知邪淫的危害,我看了很多医案,从古至今,因为纵欲身体垮掉的例子实在太多,简直多如牛毛,如果你有机会深入了解,就会知道邪淫到底有多厉害了,洪水猛兽其实并不夸张,当你肾气充足时,感觉不会很明显,当你肾气耗损到一定程度,恶果就会越来越明显,随着年龄的上升,各种疾病都会找上门来。肾气足,万邪熄,肾气虚,百病丛生!

那又为何“粗大者主下贱”呢?因为中医:肾上通于脑,SY 伤肾必伤脑力,注意力和记忆力乃至意志力都会不同程度下降,一个人脑力不行,各方面就不容易成功,主下贱也就必然了。那些真正的成功人士往往都是严格自律之人,不自律的人只会成功一时,不会长久。性放纵、性混乱直接造成的就是脑力不行,犹太人有严格的性禁忌,对性有极其严格的规定,这样的种族智商就比较高,全世界诺贝尔奖获奖者中 22\% 是犹太人。

得道之人绝对不会认为性欲强是好事,因为中医:虚则亢。虚则亢再发展下去就是早泄阳痿,各种疾病都会找上门来,这就叫“亢龙有悔”,得道之人也绝对不会认为JJ粗大是好事,反而会认为 JJ 粗大是坏事,背后的道理相信看了我这篇文章的戒友都会了解。

\subsection{脱发问题补充、戒色后如何更好更快地恢复}

我在第 8 季有讲到过脱发问题,脱发问题的确比较棘手,恢复速度也相对较慢。在有些脱发网站会看到,有人戒色 8 个月也没恢复多少,不仅没恢复,甚至还有继续加重的迹象,有脱发问题的戒友看到此类帖子,肯定会信心动摇,肯定会觉得没希望了,或者认为戒色后头发是无法恢复的,所谓能恢复只是一个谎言。对此,我可以百分百肯定地告诉大家,脱发绝对是可以恢复的,但是,如果你脱发的程度很严重,那恢复的难度就比较大了。

脱发要恢复,里面深有玄机,很多人以为戒色就 OK 了,其实不然!

脱发真正恢复的戒友相对较少,但不是没有,很多人虽然没有完全恢复,但至少改善了很多。你看到有些人的恢复情况不理想,所以,你就会觉得脱发是无法恢复的,信心自然就动摇了。为什么有的人戒色 8 个月还没恢复,其实是有原因的,脱发要恢复光戒色是远远不够的,必须要学会养生之道,因为伤肾气的方式不仅仅只有 SY 一种,很多人是戒色了,但我要问的是,你戒色彻底吗?你还有 YY 吗?YY 属于暗耗肾气,对于恢复是极其不利的,另外,你有熬夜、久坐、生气吗?这三者也非常伤精,还有压力大、吃冷饮、吹空调,这些都很伤肾气,是头发恢复的不利因素。

如果把人体比作一个肾气瓶,SY 只是这个瓶子的一个漏洞,你戒色,只是堵住了一个漏洞,然而别的漏洞还在漏肾气,这样头发能恢复吗?如果没有足够的养生意识,那头发真的很难恢复,很多人对外用药物非常痴迷,这就是为什么脱发产品如此畅销的原因,但这些产品的疗效真的很有限,毕竟只是治标不治根,根的问题就是你肾气虚了。

假设头发要恢复,所需的肾气值为 5000,这个数值是我打比方用的,只是希望大家能更容易明白。
-为减号

\begin{enumerate}
    \item SY(-600)
    \item 久坐(-300)
    \item 熬夜(-300)
    \item 生气(-300)
    \item 吃冷饮(-100)
    \item 吹空调(-100)
    \item 劳累(-100)
    \item 久视(-200)
    \item YY(-500)
    \item 网瘾(-300)
    \item 烟瘾(-300)
    \item 酒瘾(-300)
    \item 用力过猛(-200)
    \item 遗精(-500)
    \item 压力大(-300)
    \item 饮食偏咸(-200)
\end{enumerate}

暂列这 16 项,很多人的确在戒色,但所犯的错误正是“只戒不养”,在养生方面是文盲,在养生方面一窍不通,只知道戒色,把希望都压在戒色和外用药物上,这样脱发要恢复真的就比较难了,因为漏肾气的方式太多了,一不小心就漏肾气了,熬夜并不是指通宵,而是 23 点以后就算熬夜,很伤精,而很多人每天都是 23 点以后睡觉,这对身体恢复太不利了,作息是首先要调整过来的,这一关不过,头发就很难恢复了。久坐的问题可以每 40 分钟起来活动 10 分钟,尽量避免一坐几个小时。大家也看到了,头发要恢复,所要满足的条件是非常苛刻的,所以说真正能恢复的绝对是少数,但并不是不存在。

知道了伤肾气的方式后,要学会调整和避免,另外,就是要学会养生之道。养生是门学问,真的不是一句两句就可以说清楚的,脱发的朋友会很在意吃什么,还有用什么产品,但很少有人会问该怎么养生,因为他们压根儿没养生意识,所以,要恢复必须开始学会养生之道,如八段锦、站桩、穴位按摩、艾灸、经络操、中药调理、打坐、有氧运动等,还必须多看名家讲座视频和养生类的文章,多学习养生知识,对养生要有深刻的理解和认识,这样你才会知道什么该做,什么不该做,这样才能把肾气的损耗减低到最小,这样坚持一年,头发恢复的可能性才会很大,否则你只戒色,其他方面都在漏肾气,这怎么行?!头发恢复是一个缓慢的过程,一定要有耐心和信心,否则思想认识上存在误区,立场不坚定,就很难恢复了。这里还要说一下的就是情绪管理,中医讲恐伤肾,很多人一脱发就恐惧得不行,担心得不行,吃饭睡觉都不香,这种心态对头发恢复也是很不利的,必须学会调整心态,不要过分担心,有时要学会看淡一切,淡定一些,这样的心态才有利于恢复。

上面我例举的是伤肾气的方式,是减法,减肾气值,而你学会养生之道后,就是在做加法了,加肾气值,肾气值加到 5000,头发就恢复了,有一个过程,急不来。我控制住脱发倾向用了半年 ,到一年后头发才重新恢复浓密,当时我的心态很好,而且我戒色比较彻底,基本没 YY,而且我在杜绝遗精次数上下了很大功夫,固肾功每天睡前都做,总体控制在一个多月 1 次遗精,最多时 有 3 个多月没遗精。频繁遗精对头发恢复也是很不利的,戒色 8 个月头发没恢复,和遗精也是有很大关系的,毕竟遗精也是泄肾气,所以要尽量减少遗精次数才行。

有戒友会问我每天的养生功课,这季我就罗列出来,和大家分享一下:

\begin{enumerate}
    \item 打坐(我每天打坐一小时,打坐补元气第一)
    \item 站桩(站桩不是每天进行,打坐每天都坐)
    \item 八段锦和六字诀(八段锦的固肾功我每天都做,六字诀偶尔做)
    \item 中药泡脚(一周一次,艾叶泡脚)
    \item 经络拍打操(经常做)
    \item 艾灸穴位(保健灸经常做)
    \item 有氧运动(慢跑和球类,一周 1 - 3 次)
    \item 穴位按摩(经常做)
    \item 山药黑豆红枣(常吃)
    \item 饮食保持清淡
    \item 按时作息,不超过 23 点睡觉
\end{enumerate}

这基本就是我经常做的养生功课,漏肾气的行为我尽量避免,把肾气的损耗尽量减少到最低,也就是上面 16 项要尽量避免,而养生功课就是在给肾气加分,这样头发才有望恢复,头发要恢复,必须满足以下三个条件:

\begin{enumerate}
    \item 戒色彻底(YY也要尽量杜绝)
    \item 养生功课(养生功课是在给肾气加分)
    \item 养生意识(给肾气减分的行为尽量避免)
\end{enumerate}

头发要恢复必须做到这三点,特别是中度以上的脱发患者,重度脱发则更要注意养生了,否则真是万难恢复。以我的阅历来说,我见到头发恢复的戒友还是有的,但相对较少,为什么恢复的人数不多,一就是脱发需要戒色时间很长,至少一年左右,能彻底戒色一年的很少,能彻底戒色一年并且能控制遗精次数的更是少之又少;二就是养生功课,很多人光戒色不养生,要恢复就更难了。我的头发为什么能恢复?因为我在养生方面下了大功夫,中医书籍我一直在钻研,这样我懂得就比别人多,懂得就比别人深刻,知道什么该做,什么不该做,一点点赚肾气,尽量少花肾气,我的头发就是这样恢复的。如果你不学习,你永远不知道,所以我一直强调学习。

我建议大家看的养生书籍和视频如下:

\begin{enumerate}
    \item 曲黎敏的书和视频
    \item 武国忠的书和视频
    \item 中里巴人的书和视频
    \item 单桂敏的《灸除百病》
    \item 《养生堂》节目,有非常多的名家
\end{enumerate}

养生其实是一通百通的,你只要通了一个人的书和视频,再看别人的书和视频,就会觉得很好理解了,入门后就会越学越精。

大家看到这,应该知道了,我这季讲的恢复方法并不是只是针对脱发,SY 出现的任何问题,任何症状,包括心理问题,要恢复必须通过两方面的努力来实现,一就是彻底戒色,二就是学会养生之道。这样身体恢复才比较快,否则很可能出现的结果就是,很多人戒色半年也没多大起色,戒色一年恢复还不行,因为其他方面漏得实在太多了。只知其一,不知其二。只知戒色,不知养生。

我非常喜欢《黄帝内经》里的一句话,叫“知之则强,不知则老”,你知道了这个奥妙的道理,你就会变强,别人不知道,还在漏肾气,而你知道了,就可以避免,所以你能变强,别人在无意中削弱了自己的肾气值,却还什么都不知道。结果就是,你身体和头发恢复了,而别人却很难恢复,因为别人不知道!

相信有脱发困扰的戒友看了我这篇文章,对脱发如何恢复,会有更深入更清晰的认识和理解。我并不反对使用脱发产品,但一定要注意戒色和养生,否则头发无法真正恢复。

\subsection{婚前性行为对人生的危害}

常驻戒色吧的戒友肯定知道,很多人其实已经未婚先废了,就是因为沉迷手淫恶习,自己一个人疯狂手淫,撸了十几年,身体已经严重透支,就像一张被刷爆的信用卡。很多人已经出现了不少伤精的症状表现,慢前和精索比较常见,有些戒友甚至染上了神经症,感觉活着很痛苦,还有一些戒友的精子质量已经非常差了,不是弱精症就是无精症,已经失去生育功能了。婚前的放纵也会导致性功能的下降,很多戒友都比较关注自己的性功能,在发现自己因为手淫恶习而出现早泄和阳痿时,他们心里就会比较慌,希望自己可以尽早恢复。生育功能和性功能不行,也是婚姻不和谐的隐患,之前就有好几位已婚戒友因为这两个问题和老婆离婚了。

婚前性行为属于邪淫的范畴,邪淫伤身败德,为害尤烈,很多人还没结婚就把自己的身体搞得危如累卵,这样结婚后再过性生活,那更是雪上加霜了。国家有黄金储备,身体有肾精储备,在结婚前真的不能乱来,肾为五脏之根,纵欲掏空五脏精华,肾虚百病丛生!在刚开始的时候,身体底子尚厚,基本没多大感觉,但随着日积月累的耗损,最后必然就是症状缠身。我们戒色之后,一定要弄清楚什么是邪淫的行为,只有知道了邪淫的具体定义,这样才能更好地规避邪淫。

邪淫的定义:

\begin{enumerate}
    \item 若于非时、非处、非女、处女、他妇、若属自身,是名邪淫。\begin{description}
              \item[非时] 指戒律所规定的不应行淫的日期及时间;
              \item[非处] 指戒律所规定的不应行淫的处所及“三道”(三道即:口腔、尿道、肛门);
              \item[非女] 指除女性以外的,各种不同性征的人、畜之类;
              \item[处女] 指未婚女(婚前性行为);
              \item[他妇] 指有夫之妇;
              \item[属自身] 指自身行淫,比如手淫、意淫等,均为邪淫。
          \end{description}
    \item 除了夫妻之间的正淫外,一切不受国家法律或社会道德所承认的男女关系,都称之为邪淫。
\end{enumerate}

即使夫妻之间,亦有限制:

\begin{description}
    \item[非时邪淫] 佛菩萨的纪念日,每月的六斋日,不得行淫;父母的生日、亲属(父母、兄弟、姐妹等)的死亡之日,不得行淫;月经期间、妊娠中、产前产后,不得行淫。
    \item[非处邪淫] 除了卧室以外,不得行淫;除了生殖道,不得行淫(如在口腔、肛门、身体其他部位等)。
    \item[非量邪淫] 婚后放纵,不知节制。
\end{description}

邪淫的具体分类很多,比如婚外恋、婚前性行为、未婚同居、手淫、磨床夹腿、讲黄色笑话、沉迷意淫、看黄片黄图等色情内容、同性恋、乱伦、包二奶、一夜情、卖淫嫖娼、恋物癖、制作传播色情的内容、向别人介绍邪淫或者转载邪淫的内容等等。

孔子曰:“君子有三畏:畏天命,畏大人,畏圣人之言。”

圣贤教育已经告诉我们什么是邪淫,什么是正淫,什么该做,什么不该做,已经很明确地告诉我们了。如果是正人君子,那么他肯定会畏圣人之言,因为圣人所言都是久经验证的真理,真理是永远不会过时的,如果你违背了圣人所讲的道理,那么肯定会自食恶果,所谓不听老人言,吃亏在眼前。能够听到圣贤的教诲,那真是莫大的福报,这个社会上有多少人都没有这个福报,他们经常在犯身语意的邪淫而不自知,还以为很对很正常,其实他们的想法完全是颠倒的,和圣贤教育是完全违背的。天道福善祸淫,报应丝毫不爽,犯邪淫者终将要遭到惨痛的报应。

戒除邪淫是君子都应该遵循的道德依归。在邪淫泛滥的时代,若能守持不邪淫戒,那么就能从根本上维护自己的身心健康,以及家庭的安定和睦。婚前性行为,这早已是现代社会司空见惯的事,很多年轻人都是如此,记得在上个世纪七十年代和八十年代,那时婚前性还是相当忌讳的,到了九十年代黄片开始横行,加之性学家歪理邪说的毒害,社会风气开始每况愈下,到了现在,婚前性行为已经相当普遍了,而且现在的年轻人对婚前性已经完全没有了羞耻感,他们觉得婚前性很正常,甚至有人觉得没有婚前性行为是一种耻辱,真是和过去完全颠倒了。

现代社会的年轻人在性方面完全就是被误导的,他们被邪淫文化洗脑了,他们不是以邪淫为耻,他们是以邪淫为荣,吹嘘标榜自己邪淫的经历,还把邪淫的明星称之为老师!把一夜情当作时尚!真是非常无知而可悲的一代,也可以说是被邪淫文化腐蚀的一代,垮掉的一代,缺少正气、正能量的一代。现代社会的年轻人,你去观察好了,已经很少能看到充满凛然正气的眼神了,很多年轻人都是眼神空洞,一副被掏空的伤精者的面容,还有不少年轻人的眼神充满了戾气和邪气,气质极其猥琐,令人见之厌恶。

婚前性行为其实不是一种负责任的行为,很多人都是始乱终弃,感觉没了,也就分手了,只是一时的玩玩,他们当初的目的也就是玩玩,并不以结婚为目的。婚前性行为会导致多方面的严重后果,会给个人、家庭和社会带来种种不幸,这是一种长远的影响,邪淫的报应终究会显现的。在大多数民族和国家中,婚前性行为是不为社会舆论、宗教信仰、道德伦理和法律所允许的。避免婚前性行为,不仅是法律和道德的要求,也是今后建立稳固家庭,使婚姻美满幸福的需要。

婚前性行为对人生危害,主要有以下六条:

\begin{enumerate}
    \item 在结婚前就把自己掏空了,症状缠身,苦不堪言。年纪大了之后,家里也开始逼婚了,但苦于自己身体不行,性功能差,甚至不孕不育,所以处于非常尴尬的境地。戒色吧很多戒友就是处于如此的境地,家人一个劲地逼婚,但是自己身体的症状很多,而且已经出现早泄阳痿了,精子质量也非常差,这时候真叫一个难堪。之前疯狂手淫把自己身体搞废了,现在要面对婚姻时,突然发现自己不行了,真是相当悲催的人生,而且性功能不行和精子质量差,这种事情也不能让女方家人知道,否则也容易被他们看不起,甚至还会要求离婚,因为他们会觉得你是废人!和废人结婚是没有任何希望和前途的,还不如让女儿趁早离婚另嫁他人。之前有位戒友就因为精子质量差而离婚的,女方根本无法怀孕,最后就以离婚收场,那位戒友真的很痛苦也很后悔。

          很多戒友都是学生党,因为沉迷于手淫恶习,不久就发现自己脑力大幅度下降了,脑力严重下降后,原来能理解的题,现在完全看不懂了,结果可想而知,成绩也随之一落千丈。脑力不行了,整个人生都会变得很灰暗,脑力在的时候,干什么都思维敏捷,脑力射掉后,干什么都感觉有心无力,总感觉自己的脑子像一团浆糊一样,反应很迟钝很慢。有的戒友得上神衰后,那更是整天头脑昏沉,没有真实感,就像活在梦中一样,真的变成了行尸走肉。脑力就是竞争力,肾上通于脑,男人拼的就是肾!肾虚百病丛生,所以不管是学生党还是工作族,都应该好好珍惜自己的性能量,一定要尽量避免发生婚前性行为,要把自己宝贵的性能量用作正途,使性能量转化为学业和事业的强大支持和动力。
    \item 婚前性行为更容易导致双方分手,很多年轻人对于责任感没有任何概念,他们追寻的是爱情的新鲜感,一旦发生了婚前性行为,时间长了自然就心生厌倦,到时候争吵是不可避免的,因争吵而分手的也极其多见。没有责任感的爱情,就像快餐一样,吃完就扔掉,想用性拴住男人是不切实际的想法,因为男人是容易厌倦的动物,双方谈恋爱一定要以责任感为基础,然后再谈感觉,否则一旦感觉没了,那么分手就势在必然了。现在很多年轻人的爱情观都存在问题,他们完全忽视了责任感,因为没有责任感,所以就会始乱终弃。爱情的感觉是很不靠谱的,爱情也是有保鲜期的,过了保鲜期,不可避免就会进入厌倦期,就像你厌倦一部手机一样,刚开始很喜欢某部手机,时间长了就没有任何感觉了,爱情其实也是如此。所以,稳定的婚姻应该以责任感为基础,然后才是彼此的感觉,责任感意味着稳定而长久,责任感也是彼此交往的大前提,如果没有这个前提,那么最终的结果往往不会很幸福。很多人就是迷失在了感觉里,他们不断追逐感觉,最后发现感觉是那么虚无缥缈和不可捉摸,曾经那么喜欢对方,最后却陷入了频繁的争吵,乃至要以分手结束。感觉迟早会消失,只有加入了责任感,这样才能让感觉继续保鲜。
    \item 婚前性的透支会影响以后的婚姻质量。一个人在结婚前就把自己掏空了,你觉得后果会是什么呢?就像一辆破车一样,摇摇晃晃地开进婚姻,然后不久就抛锚了,很多人都陷入了很深很深的困境当中,整个人生都变得灰暗而颓废。在结婚前疯狂看黄疯狂手淫,有着十几年的手淫史,身体五脏六腑早已被掏狠了、掏空了,这时候进入婚姻,还要面对婚后的性生活,在这个节骨眼上,很多人的身体不干了,要么是神经症爆发、慢前加重,要么是彻底阳痿,又或者得上了其他的慢性病。身心健康不行了,婚姻也会随之危机四伏,之前就有好几位戒友因为性功能不行而离婚的,当你性功能废掉了,老婆也可能给你戴绿帽子,到时候你就苦大了。
    \item 婚前性行为也容易使女方意外怀孕,假如女孩经历多次人工流产,那么很有可能会使子宫穿孔破裂,引起大出血,从而危及生命。堕胎对女方的危害很大,可能造成的身心危害有:\begin{enumerate}
              \item 细菌感染;
              \item 将来习惯性流产;
              \item 终身不孕;
              \item 手术意外,危及生命;
              \item 腰痛、身体虚弱、得慢性的妇科病;
              \item 家庭不和,事业不顺;
              \item 造成心理创伤,如愧疚、悔恨、怨天尤人、烦躁、郁闷等各种心理疾病,甚至会出现自杀的倾向;
              \item 有时还会梦见小孩在受苦哀嚎,或梦见有一个不认识的小孩常来跟着自己。
          \end{enumerate}
    \item 医生曾这样告诫:千万别把堕胎当儿戏,否则将会付出沉重的代价,如大出血、妇科炎症、终身不孕甚至死亡。人工流产手术后可能发生月经失调、子宫腔粘连及子宫内膜异位症等不良后果,对今后的生育可能会产生不良的影响。更可怕的是,人工流产可导致终生不孕,尤其是经历多次流产的人,更容易发生此类情况。由于反复地钳刮子宫内膜,使子宫壁变薄,内膜越来越少,导致月经过少,甚至闭经。受精卵着床而没有良好的“土壤”,使之不能发育成胚胎,因而终生不孕。并且往往易于造成生殖器官炎症、子宫和其他内脏损伤,或大出血,甚至会危及生命。有不少戒友都让女友堕胎过了,有的甚至堕胎过多次,堕胎其实就是一种杀业,将来的果报是很不好的,所以一定要尽量避免婚前性行为,并且要好好忏悔,不要再犯此类错误。
    \item 婚前性行为也可能染上性病,有些戒友不仅自己手淫,还会去嫖娼,这样就有可能染上各种性病,之前就有一位戒友因为放纵而染上了尖锐湿疣,真是饱受各种痛苦和难以言说的尴尬,花了很多治疗费,但也好不彻底,总是要面临复发的可能,搞得他想死的心都有了,感觉对人生彻底绝望了。还有的人染上了淋病、梅毒,甚至还有染上艾滋病的。有些人虽然没有染上艾滋病,但他得了恐艾,活得也异常痛苦和折磨,真可谓惶惶不可终日,过着苟且偷生的生活。最近几年开始流行约炮,这种性乱的活动最终必将把自己一炮打入医院,让自己成为邪淫的炮灰,最后肯定会一炮把自己打入地狱!实在非常可怕!性乱的一代,根本无颜面对列祖列宗,这样瞎搞简直连禽兽都不如了!现在很多年轻人反以之为荣,简直愚痴到了极点!也堕落到了极点!现在中国的艾滋病已呈爆发趋势,到处瞎搞的人,弄不好真会得上艾滋病,到时候就彻底傻眼了。
    \item 离婚率高。婚前性行为看似自由开放,实乃后患无穷,据研究统计:婚前有性行为的夫妻,离婚率远远高于守贞的夫妻。婚前有性行为的人,责任感往往容易缺失,他们追求的更多的是“感觉”,有感觉就在一起,没感觉就分手,而爱情的感觉是会随着时间的延长而消失的,到时候如果没有责任感的入主,那么必然就会导致吵架离婚。婚姻的基础其实就是责任感,就像一座大厦的地基一样,婚前放纵之人往往缺少应有的责任感,很多人只是图一时之快,即使在进入婚姻之后,他们依然会延续婚前放纵的模式,不仅自己还会沉迷手淫,甚至还会发生婚外情、嫖娼等行为,到时候就会和妻子上演离婚大战。
\end{enumerate}

相关文献:

《生命时报》于 2006 年 11 月 8 日报道,“日益猖獗的性病和数量庞大的单亲子女让美国政府头痛,为保证国民健康和提高国民素质,美国政府斥巨资 5000 万美元的“禁欲教育项目”推行婚前禁欲计划。政府要求年轻人到 30 岁后再开始性生活。近日,美国联邦政府通知各州,可以使用联邦专项补助金来鼓励那些 20 多岁的成年人,在婚前保持禁欲生活。”

最近 10 年,原本性观念开放的美国人,开始不断推广“禁欲教育”,使青少年的性观念慢慢回归传统。据了解,近几年,美国卫生福利部已将 500 万联邦教育基金拨给禁欲教育项目,这是唯一一个获奖的教育项目。

美国国家禁欲教育组织(NAEA)主要帮助州立及联邦政府普及性知识,解答青少年在性方面的疑问,然后对这些问题进行数据研究和分析,从而帮助青少年树立健康的性观念。禁欲教育组织目前主要涉及 13 个主题,统计数字表明,目前在全美高中生中,至少有 70\% 的学校讲授了其中的 8 个主题:禁欲是避孕的最有效方法(87\%),抵抗来自已发生过性行为的同龄人的压力(83\%),拥有多个性伴侣的风险(81\%),同学如何影响他人做出关于性行为的决定(80\%)等。至少 3/5 的高中还会讲婚姻与承诺(69\%)、安全套的功效(65\%)这两个主题。在小学里,老师最常讲的主题是人体发育。

禁欲教育组织还会到全美各个城市,根据学校要求及孩子的年龄提供不同课程。这些课程大部分是短期的,课时不超过两个月,对授课老师的基本要求是心理学专业研究生。老师们组成团队到各个学校进行演讲,各个学校还可以申请让本校老师参加培训。2011 年,老师在大学里上过一门“人类性行为”的课,曾对全班 150 多名同学(其中约有一半是女生)进行了一次统计,发现大概 80\% 的女生都是处女,她们大多是 90 后。这个数字比往年高出许多,老师对此也很震惊,她说,这个数字每年都在增加,这或许就是推广禁欲教育的成果。

后记:

孔子曰:“君子有三戒。少之时,血气未定,戒之在色;及其壮也,血气方刚,戒之在斗;及其老也,血气既衰,戒之在得。”戒色是君子第一修为,色也是少年第一关,这关一定要打过,印光大师云:“色欲一事,乃举世人之通病。不特中下之人被色所迷;即上根之人,若不战兢自持,乾惕在念,则亦难免不被所迷。试观古今多少出格豪杰,固足为圣为贤,只由打不破此关,反为下愚不肖,兼复永堕恶道者,盖难胜数。”《分别善恶报应经》亦云:“复次邪欲报有十种。何等为十?一欲心炽盛。二妻不贞良。三不善增长。四善法消灭。五男女纵逸。六资财密散。七心多疑虑。八远离善友。九亲族不信。十命终三涂。” 我们一定要学会控制自己的欲望,而不是做欲望的奴隶,我们要学会转化自己的性能量,把性能量用于自己的学业和事业,千万不可滥撸滥泄,那样只会掏空自己的五脏六腑,也会泄掉自己的脑力,到时候就悲催了。让我们向婚前性行为说不,让我们守住自己最后的底线,让我们做一个守贞之人,把最好的自己留到结婚后,避免未婚先废。

\subsection{黑眼圈眼袋、白发、痔疮、鼻炎问题详解}

最近在戒色吧浏览帖子,发现一个好现象,很多戒友的觉悟有了明显的提升,从回答问题就可以看出来。其实这些戒友通过学习就是在“长觉悟”,在围棋界把棋艺的进步叫“长棋”,在戒色方面就叫“长觉悟”,觉悟上去了,定力自然就上去了,通过不断学习戒色文章和养生知识,境界就会明显高于戒色新人,这样坚持学习下去,不断开悟,离彻底戒色就不会遥远了。

戒 SY 修的就是定力等级,定力等级达到了,自然而然就会戒掉,就怕不学习和放松警惕,那样永远戒不掉。

新人最大的问题就是存在很多思想误区,任何的怀疑和犹豫不定必须要克服掉,通过提问来消除心中的疑虑,这样戒色的信心才会坚固,否则心中还有疑问和困惑,这样是不利于戒色的。比如很多新人出现戒断反应时就想到了禁欲有害,认为会憋出毛病,如果在这个问题上认识不清,那就很容易出现破戒,其他思想误区还有很多,这些思想误区必须得到纠正,要戒色必须先改造思想,否则根本就无法戒色成功。

下面进入正题。

黑眼圈和眼袋这两个问题我都有过,眼袋问题是 19 岁开始的,黑眼圈在初中时就有过,因为我热爱运动,所以黑眼圈问题并不严重,根据我的仔细观察和体验,热爱运动的人出现黑眼圈的情况比较少,即使出现,只要积极锻炼和注意休息,黑眼圈很快能下去,就怕不爱动又熬夜久坐,这样的人黑眼圈就比较顽固了。我推荐大家的运动方式最好是阳光下的有氧运动,中医认为,晒太阳可以温煦体内的阳气,是养生必不可少的手段。晒太阳能够帮助人体获得维生素 D,可以提高人体的免疫力,预防各种疾病。

晒太阳也是有技巧的,在红日初升的清晨,适宜把两个手掌心(劳宫穴)对着太阳,做深呼吸,这样可以养心、肺之阳。在红日当空的午时,尤其是冬天的午时,是晒太阳最宝贵的时间,适宜把帽子脱掉,让阳气从头顶(百会穴)吸收进去,这样可以养心脑之阳,然后低下头,让阳光从颈后(风池穴)吸收进来,风池穴是人体卫外阳气的源头。

当然,晒太阳也要注意一个量,也不能晒得太多。一天 1 小时左右即可,不要整天猫着腰在电脑前久坐,这样对身体恢复是很不利的。要多接触大自然,多晒太阳,多运动。

在中医来讲,黑眼圈多因肾气虚损、精气不足、脉络失畅、目失所养所致。所以黑眼圈要恢复,一定要注意养生,积极锻炼,这样黑眼圈恢复才比较快。

记得高考后那个暑假我 SY 比较频繁,暑假过后照镜子就发现,有眼袋了!一有眼袋,人就显老几岁,感觉没精神很萎靡。那时的我虽然没有多少养生知识,但我知道运动可以帮助我身体恢复,于是大学时我经常慢跑,一般每次跑 20 圈左右,跑得很慢。这样跑了 20 天,再照镜子:眼袋没了,皮肤也好了很多,气色明显好了。但是,那时候我定力尚浅,还处在强戒盲戒阶段,于是没多久又开始 SY,一 SY 再照镜子,眼袋又回来了,所以那几年我比较苦恼,和眼袋斗争了很久,但眼袋问题一直解决不了,因为我还一直在 SY,本来好不容易通过运动和休息把眼袋消下去了,但一放纵,眼袋就又出来了,不仅眼袋出来,脸部气色也下降不少,就像过期的水果,那种枯萎颓败的感觉。我那时有了解过祛眼袋手术,有外切口和内切口两种,同时我也了解到,祛眼袋手术并不能帮你彻底祛除眼袋,还是有可能会复发的,如果你继续 SY,眼袋还是会出来的。

眼袋的出现多是因为胃燥化水功能出现衰退,胃机能差,承泣穴、四白穴阻塞造成的。而中医认为:肾为胃关!《素问》曰:“肾者,胃之关也,关门不利,故聚水而从其类也。上下溢于皮肤,故为浮肿。浮肿者,聚水而生病也”。肾不仅为胃关,肾还藏五脏六腑精华之气,SY 导致肾虚,肾一虚,全身都容易出问题,眼袋问题仅仅是肾虚的一个表现而已,虽然眼袋是在胃经上,位于承泣穴的位置,但是根本原因还是肾的问题,肾一虚,五脏功能就紊乱了。

现在我彻底戒掉 SY 和 YY 后,眼袋问题已经远离我了,虽然不是非常平,但也属于和年龄相符的状态。眼袋一旦出现,要恢复的前提就是彻底戒色,否则真的很难恢复。

下面谈一下白发问题。

我也有过白发困扰,但我不是少白头,青春期我 SY 频繁,但那时我作息规律,营养不错,白发有,但是很少。后来我又 SY 又熬夜,而且饮食不规律,有时一天只吃一顿,这样没多久,我白发就冒出来了很多,左边和右边各几十根,不仅出现白发还开始大量脱发,每天 100 根以上,那时的我真的快崩溃了,我知道这就是报应,自作自受,怪不了别人,只能怪自己放纵自己。

导致白发的因素:

\begin{enumerate}
    \item 精神因素精神紧张、忧愁伤感、焦虑不安、恐慌惊吓等都是造成少白头的原因。现代医学认为,不良的精神因素,会造成供应毛发营养的血管发生痉挛,使毛囊、毛球部的色素细胞分泌黑色素的功能发生障碍,影响黑色素颗粒的形成和运送。
    \item 营养失调实验证明,黑鼠如果一直进食缺乏叶酸、泛酸、维生素等的食物,鼠毛便会变成灰白色。另外,头发色素颗粒的颜色,往往和它含的金属有关。黑头发中的色素颗粒含有铜、钴、铁等元素,假如缺少这些元素,往往出现白发。此外。缺少蛋白质、严重营养不良等,也可长白发。
    \item 患慢性疾病一些人患有植物神经功能失调、甲状腺功能亢进、肺结核、伤寒、内分泌障碍等,也会出现白发。这是因为疾病破坏或干扰了毛囊、毛球色素细胞的生长发育,使它失去分泌黑色素的能力,阻碍黑色素颗粒的形成。
    \item 遗传因素少年白发也有一定的先天因素,在父母或家族血统中有类似的情况发生。
    \item 生活恶习如熬夜、久坐、纵欲,这些习惯都伤肾气,中医:发为肾之华。有这些恶习的人更容易出现白发。
\end{enumerate}

我现在的头发已经恢复,戒色半年后,我能看到发根长出来的是黑色的头发。我非常注重养生,这也是我恢复的有利因素,很多人只戒不养,要恢复相对较难。黑色食物我也吃过,但并没有刻意去吃,黑芝麻吃过,黑豆吃过,香菇也常吃。我戒色比较彻底,又注意养生,这才是我能恢复的根本原因。在精神方面,我没给自己太多压力,情绪管理我做得很好,基本很少有不良情绪,不生气,不抱怨,不嫉妒,以一种心平气和的心态度过每一天。很多戒友都会急躁易怒,这很正常,因为肾气不足,情绪上的表现就容易急躁易怒,所以一定要注意情绪管理,让自己的情绪保持稳定,不良情绪就像地震一样,对身心是有很大的负面影响的,危害实在不容小觑,对身体恢复很不利。记得以前我隔壁邻居家的男孩,比我大几岁,他就是少白头,他没遗传基因,家里吃得也很好,但他家庭不太和睦,父母经常吵架,搞得他压力非常大,经常处在家庭暴力的阴影中,这其实就是导致白发的精神因素,家庭不和睦,心理压力过大。

白发要恢复,道理其实很简单,但做起来并不简单,就是反其道行之!杜绝那些导致白发的因素,彻底戒色,学会养生之道,这样白发问题就能慢慢恢复,有一个过程,你必须在养生方面懂得更多,更深刻,这样恢复的可能性才比较大,否则光戒不养,远远不够,具体的养生方法我在第12季有专门提到,大家可以看看。

下面再来谈下痔疮问题:

痔疮问题也是很普遍的问题,俗话说,十男九痔!还有一句话是这样说的:十女十痔。女人得痔疮的比例也非常高,因为女性有特殊生理期,更容易气血双亏,再一久坐,痔疮问题就会找上门来了。我得痔疮是在 20 岁左右,当时我频繁 SY,然后有一段时间我坐沙发上,沙发是下陷的,这样坐了一段时间,就感觉下面有东西了,从那时起我就得痔疮了,去医院检查是内痔,配的药涂了就能消下去,但不能去根,痔疮也给我生活带来了很大的烦恼,冷辣不能吃,所以那时我在吃的方面很小心,一吃辣就会发作,大便容易下血。后来我学中医才知道,SY 会导致气虚,你气足的时候坐沙发没事,因为气推血行,气足坐沙发不会引起气血瘀滞,但是当你一旦气虚了,再坐下陷的沙发,那就容易导致气血瘀滞了,痔疮就容易得了。

痔疮的解剖学原因:人在站立或坐位时,肛门直肠位于下部。由于重力和脏器的压迫,静脉向上回流颇受障碍。

我现在痔疮已经彻底好了,我是无心插柳柳成荫,我痔疮好的过程有点奇特,本来我以为除了做手术,痔疮是不会好的。当然做手术后也是可能复发的,因为从医学角度来说,痔疮是由于静脉血液回流不畅所致,所以,手术只能把你现有的痔疮给解决掉,但是保证不了你血液回流不畅的问题能恢复。

我记得我 2 年前开始中药泡脚,用的是艾叶,淘宝上买的,当时我看了单桂敏的博客,里面提到了根治痔疮的方法,就是艾灸痔疮。方法是坐在一个挖空的凳子上,下面用艾灸熏烤痔疮,这样大概一周左右,痔核就会脱落,痔疮就痊愈了,当然这个过程有点痛苦,不是每个人都能接受的。我看了以后就没敢尝试,当时我就艾叶泡脚,没想到连续艾叶泡脚三天,我就感觉痔疮外翻了,变大了,我就知道是艾叶的温经通络在起作用,当痔疮外翻时,我的确很痛苦,坐都坐不下去,一坐就痛,只能歪着坐。那时我已经开始艾灸,当然没敢直接艾灸痔疮,那时我艾灸神阙穴,也就是肚脐眼,泡脚三天痔疮外翻,第四天艾灸肚脐眼,艾灸了 40 分钟,我就感觉到痔疮要破了,我马上一提肛,痔疮果然破了,痔核掉了,从那以后我痔疮就彻底好了,再没犯过。

有痔疮困扰的戒友可以试试我这个方法,当然,治疗痔疮的方法很多,我只是把我的经验拿出来和大家分享,希望能帮到大家。

最后再来谈下鼻炎的问题。

我是过敏性体质,是一个过敏性鼻炎患者,有时打喷嚏可以连续打十几个,我记得从小时候开始,我鼻子就不好,那时虽然没 SY,但是因为先天不足,先天就容易外感风寒,所以我小时候经常感冒,还得了哮喘。但是,那时候我鼻炎没那么严重,进入青春期以后,开始频繁 SY,鼻炎就突然加重了,一年鼻子也很少有通气的时候,很痛苦。用了很多治疗方法都不见效,后来去做了下鼻甲切除手术,才缓解了不少,但还是不行,一到季节转换就容易出问题。

那时候我处在无知的状态,根本不知道是 SY 加重了我的鼻炎,现在学习中医医理后就知道了,中医:精虚鼻渊。精气一虚,就容易导致鼻炎,因为肾主纳气,肺主出气。先天不足的人再一 SY,鼻炎就会加重,因为先天不足代表的意思就是伤不起了,每个人体质不同,有人先天禀赋深厚,不伤到一定程度是不会出症状的,像我就是先天不足,稍微一伤,就出症状了。记得有位戒友说过,原来体质超好,冬天赤膊都不觉得冷,后来频繁 SY 伤了肾阳后,夏天都手脚冰凉,就是肾阳虚了。你一年两年 SY 身体没问题,但只要继续 SY,伤到一定程度肯定会出症状,有的人热爱运动,作息规律,症状出来就轻微,但是无论身体如何强壮,到了 40 岁以后身体都会走下坡路,到时候很多潜在的疾病,也就是隐疾,就会集中爆发出来,就是年轻时放纵太过导致的。年轻时阳气足不觉得,年纪一上去,很多毛病就显现出来了。

我现在依然有鼻炎,但相对以前要好很多了,以前几乎每天都难受,现在很少有难受的时候了,而且现在我学会了艾灸,鼻子不舒服时艾灸合谷穴和足三里,很快就能恢复通畅。其实因为 SY 导致鼻炎加重的人有很多,我认识的朋友当中就有好几个,我表弟以前鼻子没问题,后来也学会了 SY,没几年,也得上了鼻炎。SY 导致肾气亏损,身体的抵抗力就会大大下降,内虚后,外邪就容易乘虚而入,这样很多疾病的发病率就提高了。所以,一定要戒掉 SY,养足肾气。否则真的会百病丛生!切记。

\subsection{腿软问题、出油问题、紧张障碍、阴囊潮湿}

腿软现象在戒友中相当普遍,戒色吧经常会看到有戒友发帖说自己腿软腿无力,在中医来讲其实很好理解,因为肾主骨,而且肾经是从脚底涌泉穴开始往上走的,所以 SY 后乃至遗精后,很容易出现腿软现象,腿软现象在西医上也已经证实,SY 会导致骨质疏松,SY 会导致骨密度的下降。中医又讲到:腰为肾之府,膝为肾之路。肾虚有一个普遍症状就是腰膝酸软,出现腰痛的戒友也有很多。很多人弯腰没多久,就酸痛得不行。这其实就是肾精不足的典型表现了,是身体在给你信号,不能再纵欲了。

我初中和高中 SY 频繁,那时我打篮球经常崴脚,好几次崴脚都是在没有任何身体接触的情况下崴到的,那时 SY 完总觉得腿软无力,支撑力不稳,但并不知道是 SY 导致的,也不懂得医理。现在研习了中医医理才明白背后的道理。

前段时间看 NBA,正好看到公牛队的罗斯扭伤膝盖,伤情是左膝前交叉韧带撕裂,余下赛季报废了。罗斯受伤也很诡异,没有任何身体接触,就持球往前一跳,膝盖就扭伤了,看到他这样受伤,我敢说比赛前天晚上罗斯不是遗精了,就是纵欲了,加上频繁赛事的疲劳积累,这种莫名其妙的受伤就这样发生了。史泰龙的《洛奇 1》里面有一个场景我现在还记得,洛奇的教练告诉洛奇不要碰女人,碰女人腿就软了,会影响到比赛状态。竞技体育是比较忌讳比赛前碰女人的,一碰女人腿就软了,很多动作都做不到位,也容易受伤。我看过拳王泰森的纪录片,在讲到他在日本输给道格拉斯的时候,泰森说,他输掉了比赛就是因为他放纵了,以一种很不好的身体状态去比赛,后果可想而知,被实力大大不如自己的道格拉斯击败。

SY 后或者遗精后,一定要注意休息,不要从事太剧烈的运动,以防受伤。

下面来谈下出油问题。

SY 后出现出油现象的戒友非常多,因为 SY 导致内分泌紊乱,就会出现出油现象,当然不全是出油,有部分戒友则是表现为皮肤干燥,内分泌紊乱一般分两类,一类出油,一类皮肤发干。总的来说,以出油现象居多,和每个人的体质不同有关。

一般出油分两类:

\begin{enumerate}
    \item 脸部皮肤出油
    \item 头皮出油
\end{enumerate}

脸部皮肤出油容易造成青春痘和痤疮,而且 SY 导致的痤疮一般很顽固,属于顽固性痤疮,不戒掉 SY 真的很难好。脸部皮肤总是油光光的,也是会影响到美观的。而头皮出油则比较容易造成脂溢性脱发,患者一般头发细软,有的还伴有头皮脂溢性皮炎症状。脱发症状我在前面的文章有专门讲到,一般轻中度脱发相对容易恢复些,重度脱发就比较难恢复了,因为有一个词叫“积重难返”,有条件的患者可以去尝试植发。我戒掉 SY 后,明显感觉皮肤出油少了,头皮屑也减少很多,这就是内分泌调整过来了,所以要减少出油状况,一定要戒掉 SY,不要只做表面功夫,要从根上去调理。有戒友会问,可不可以用些去油的产品,我的回答是当然可以,但一定要注重戒色,否则只是治标不治根,只有通过坚持戒色,内分泌才会自动调整正常,否则用再多的产品也只能是缓解,无法治愈。

接下来谈下紧张障碍的问题

紧张障碍分为:

\begin{enumerate}
    \item 小便障碍
    \item ML 障碍
\end{enumerate}

很多戒友都有反映:在有人时无法小便,没人时正常。还有戒友反映 SY 时没事,ML 时就出现了勃起障碍,这种紧张障碍其实就是肾气受损的表现,肾气伤到一定程度就会出现这种现象,我自己也经历过这两个现象,一般戒掉一段时间,积极锻炼,按时作息,养足肾气,紧张障碍的问题就会缓解不少乃至自愈。所以有这种情况的戒友,也不必太烦恼,戒掉 SY 养足肾气,是可以恢复的。

最后再来谈下阴囊潮湿的问题。

这个问题也非常普遍,绝大多数 SY 戒友都存在这个现象。有了阴囊潮湿,就感觉很不舒服,也是一种扰人的烦恼。

阴囊潮湿是指由于脾虚肾虚、药物过敏、缺乏维生素、真菌滋生等原因引起的男性阴囊糜烂、潮湿、瘙痒等症状,是一种男性特有的皮肤病。阴囊潮湿、瘙痒属于慢性前列腺炎的典型临床表现,常因慢性前列腺炎导致的植物神经功能紊乱所造成的,如果前列腺炎治疗不当或不彻底,阴囊潮湿、瘙痒是不会消失的。大部分阴囊潮湿患者都伴随着前列腺炎,主要是前列腺炎导致的植物神经功能紊乱所造成的,随着前列腺炎的好转和治愈,阴囊潮湿会间断出现直至消失。阴囊潮湿是慢性前列腺炎的典型症状之一,也与长期久坐,热的环境中工作以及精索静脉曲张有关。如果您是前列腺炎患者,阴囊潮湿就与前列腺炎有关。

我以前就有过阴囊潮湿,还出过湿疹,非常不舒服,总是潮湿的,像水一样,睾丸也松弛下垂。一般有前列腺炎的人都有阴囊潮湿,当然如果你有阴囊潮湿,也可以去做个前列腺液的检查,看看有没有前列腺炎。我彻底戒掉 SY 后,积极锻炼,按时作息,养足肾气,阴囊潮湿现象基本消失了,睾丸也收上去了,前列腺炎也好了,所以出现阴囊潮湿的戒友,最好能坚持戒色,这样才有望解决这个烦恼。

\subsection{遗精问题补充、无害论、禁欲有害论、婚后次数}

遗精问题我在第 3 季和第 7 季都有讲到,遗精是个大问题,频繁遗精太伤身体了,而且遗精后也容易破戒,思想容易发生动摇,所以要尽量减少遗精次数,这样才有利于身体更好地恢复。关于遗精问题我推荐的是固肾功,根据反馈情况,分为 2 类,一类做了效果很好,不少戒友真的做到了一月以上一次遗精,一般是 30 +,也有戒友做到了 40 +,原来是一个月 8 次,所以固肾功有效是不容置疑的,还有一类就是做了遗精照旧,效果不大,很是苦恼。为什么别人能做到 30 +,而我没做到呢?没做到的戒友一定要问下自己这个问题。

这季我就这个问题再深入论述一下,希望可以帮助大家更深入地认识遗精问题。

做了固肾功遗精照旧的朋友,让我想起了以前上学时体育课,体育老师教给大家一个动作,有运动天赋的人很快就能把握要领,很快就能把动作做到位,而其他很多人则是做了很多遍,依然做不像样,依然做不到位,老师教的内容是一样的,但学生的接受能力和接受程度是不同的。但只要你坚持练习总有一天会找对感觉的,天赋高的人可能马上就能找对感觉,天赋差的人,也许需要一周乃至一个月才能逐渐找对感觉,这是其一。

其二,很多人韧带已经很好了,能摸到地,但仅仅摸到地是不行的,一定要找对拉紧感,强化拉紧感,很多戒友一摸到地就结束了,以为到位了,其实才仅仅开始,必须冲击到自己的极限,然后把极限做 10 次以上,这样才比较到位,不能一拉到极限就结束。能摸到地了,就要手掌摸地,手掌摸地了,就要站在小凳子上摸地,一步步来。很多戒友会说,我韧带不好,摸不到地,这其实并没有关系,关键是拉紧感,你韧带不好,但你能找到拉紧感,并且强化拉紧感,久而久之,自然就能摸到地了,不仅摸到地,功夫深了,手掌摸地也会很轻松的,贵在坚持,是一个渐进的过程,很少有人可以一下就手掌摸地。

其三,导致遗精的因素非常多,不是做了固肾功就万事大吉了,如果你犯了其他导致遗精的因素,一样还会遗精。前 2 天回答了一个戒友的问题,他的情况是这样的,固肾功坚持做,的确减少了遗精次数,但最近几天频繁遗精,搞得很苦恼,他来问我怎么回事?我就帮他分析了,让他回忆下这几天干什么了?然后他就说最近感冒了,并且去打篮球了,然后我就和他说:答案找到了,你遗精的案子破了。他最近频繁遗精,原因其实很简单,但是如果你不懂得这个道理,可能一辈子都不会明白怎么回事。他感冒,本来身体已经虚了,再去打篮球出大汗,中医:大汗伤阳,劳伤阳气。他感冒后打球,正是犯了中医讲的“虚虚之祸”,劳累是晚上遗精的诱因,很多人白天运动过度,出完汗很疲倦,晚上就遗精了,这种劳累后晚上遗精的情况非常非常之多,我搜集的案例中占了很多,年轻人比较容易犯的错误就是:过度运动,只懂得放,不懂得收。当身体虚的时候,一定要注意休息,中医有讲到:汗者,精气也。打篮球出汗量比较大,一下把精气给伤了,晚上就容易出现遗精。关于劳累遗精,我自己也是有过体验的,在身体虚的时候,我去蹲了几组深蹲,晚上就遗精了,所以,当身体虚的时候,尽量不要做剧烈的运动,不要做容易出大汗的运动,否则晚上很可能会出现遗精。打篮球也是要注意把握强度的,也要讲究时机,身体虚或者身体不适的时候不要去打,以静养为主,可以做些强度比较低的锻炼,比如散步等。

导致遗精的因素如下:

\begin{enumerate}
    \item 白天 YY
    \item 白天劳累
    \item 喝酒
    \item 吃肉太多
    \item 吃补药
    \item 趴着睡
    \item 裸睡
    \item 晒被子
    \item 盖太厚
    \item 睡前打坐
    \item 内裤太紧
    \item 顶着或者夹着被子
    \item 艾灸不当
    \item 打坐意守下丹田
    \item 熬夜久坐
    \item 睡前喝水太多
    \item 运动过度
    \item 生病
    \item 肾亏无梦而遗
    \item 饮食偏辣偏重
    \item 紧张(包括梦魇)
    \item 挤压(包括趴睡)
    \item 生气(导致气血紊乱)
\end{enumerate}

关于遗精问题,我到现在已经回答了几百个问题,总结的遗精原因基本就是这 23 条,作为戒色前辈,其实干的事情就是“试错”,把自己悟到的经验传授给戒友,特别是戒色新人,新人的思想误区实在太多,必须帮助新人建立正确的认知,否则他一辈子都不会明白。

很多戒友遗精后都会做记录,什么时间遗精的都会记录,但他们缺少的正是对遗精原因的深入认识和了解,如果你对原因认识不清,那就无法真正避免遗精,还是会犯同样的错误,而且是犯了错误还不知道怎么回事,还不明白怎么会遗精,我打个比方:要不被贼偷到,必须比贼精,贼是够精的,而你必须比贼更精,而且要精益求精。这样才能最大限度地减少遗精次数,如果你能把遗精频率控制在一月一次,那对你身体的恢复是非常有利的,否则吃再多中药,疗效都会打折,因为你漏得太厉害!

每次遗精都要像破案一样,找到蛛丝马迹,找到线索,找到导致遗精的因素,当然一月一次的遗精,就不需要刻意去找了,因为对于普通人来说,精满自溢这个理论是适用的,但一月 3 次以上,肯定不是精满自溢,而是病态的遗精了。只有真正找到原因,下次才能做到尽量避免,否则遗精后你不总结,不深入思考原因,结果就是遗精还会频繁。据我所知,被遗精困扰了十年的戒友也大有人在,搞得腰膝酸软,浑身无力,腰痛耳鸣,脑力下降,症状很多。

再分享 2 个案例,一个是猪腰子吃多了,晚上遗精,另一个是芝麻吃多了,晚上遗精。芝麻很好,但古代养生家每日只吃 2 丸,为什么不多吃呢?因为吃多了,一不容易消化吸收,二就是容易导致遗精,所以芝麻可以吃,但必须注意量,不要一吃一碗,那样很容易导致遗精,补的东西要注意适量。还有肉也应该少吃,否则容易出现频遗,也容易助长欲望。切记!

下面谈下无害论和禁欲有害论:

无害论其实是非常具有迷惑性和误导性的,因为大家都知道结婚后会有性生活,那么婚前为何不能有性生活呢?很多人思想上认识有误区,就会如此发问,并且会很认同适度无害。我当年也被这个理论迷惑过,深深地误导过,这种似是而非的理论对于青少年,对于阅历尚浅的戒友,毒害的威力是非常大的,让那些 SY 的人更加深陷其中,等身体出症状了,就悔之晚矣,我看到很多戒友的案例,都会把无害论给批判一通,认为自己当初就是被无害论给害了。所以,在无害论这个问题上,必须立场坚定明确,不能有疑惑,否则根本是戒不掉的,只会在怪圈中继续徘徊。

适度无害的破绽就是忽略了 SY 的成瘾性,SY 具有高度成瘾性,成瘾了就不是你能控制得了的,就像决堤的洪水,不少戒友一破就是 2 次,甚至连续好几天放纵,根本收不住。那为何结婚后又能过性生活呢?结婚后的性生活,当然一定要注意“节制”,古今无数医案都记载了房劳伤身导致疾病。戒色吧有不少戒友已经结婚,但也想戒色,为什么呢?因为身体亏损厉害,出了很多症状,必须戒了,否则就完蛋了。一般有老婆的戒友,我都建议要和老婆好好沟通下,取得理解,这样戒色才比较好,否则老婆误解你,容易造成家庭矛盾,等你禁欲一段时间把肾气养足了,到时候再过性生活也不迟,留着青山在不怕没柴烧。假如你提前把自己透支光了,将来就彻底阳痿了,阳痿也就罢了,还会有无数疾病找上门来。

结婚后的性生活方针:养足肾气、节制使用。如果已经出症状,就要禁欲一段时间,养足了,再使用。

杜绝婚前性行为:就是为了让你的肾气尽量不受损,否则很多戒友婚前乱来,把自己身体废掉了,结婚后怎么办?

关于无害论,我打个比方,大家看了就知道应该怎样对待无害论。这个比方是这样的:有两个人走过来和你说话,一个说大象是灰色的,一个说大象是蓝色的。说大象是灰色的,说得很有道理,说大象是蓝色的,说得更有道理,这时候你就犹豫了,不知道该相信谁,因为两方面都很有道理。这时候,我告诉你一个办法,那就是自己去看大象究竟是什么颜色的!事实胜于雄辩!同理,在戒色方面,你可以直接去找受害者的案例来研究,用案例来说话,从案例中发现真相,案例是第一手宝贵资料,反映的情况是最真实的。

关于禁欲有害论,戒友问得也不少。我可以明确告诉大家,禁欲无害,但有一个前提,就是必须尽量杜绝 YY。否则相火妄动,然后憋着,是会憋出毛病的,我在中医医案上,有看到不少此类案例,就是 YY 导致的问题。如果修心到位,是不存在这个问题的,很多和尚都活到了 100 岁以上,修心到位,禁欲对身体无害。不少戒友戒色后,身体出症状了,然后他就想到了禁欲有害的文章,马上又开始 SY 了,这其实是他没认识到戒断反应,以我的阅历,几乎每个戒友戒色后都会出点症状,有人轻微,有人严重,这是很正常的,正气一恢复,潜在的病邪自然会表现出来,坚持戒色,戒断症状会消失的,身体会越来越好的。

最后再来谈下婚后次数。

婚后过性生活的频率,也是大家比较关心的,大家可以参照药王孙思邈的文章:

\begin{quote}\it
    御女之法,能一月再泄,一岁二十四泄,皆得二百岁,有颜色,无疾病。若加以药,则可长生也。人年二十者,四日一泄;三十者,八日一泄;四十者,十六日一泄;五十者,二十日一泄;六十者,闭精勿泄,若体力犹壮者,一月一泄。
\end{quote}

药王这段话是对结婚后的人说的,因为古人结婚比较早,20 岁基本都结婚了,古人是不主张婚前性行为的,因为古人在这方面认识比较深刻,知道婚前乱来的恶果。

这个性生活频率,也是有一个前提的,那就是在你肾气充足的情况下才能这样过,否则你肾气已经亏损了,再这样过,那就会虚上加虚,身体万难恢复了。

我个人推荐的是一月 2 次,就是肾气充足的情况下,一月 2 次,这样对身体伤害较小。

\subsection{如何克服意淫、反复现象、嗜睡、压力导致的 SY}

意淫是大家开始戒色后普遍会遇到的难题,很多人可以不 SY,但是意淫就像脱缰的野马,几乎每几分钟就意淫一次,在中医来讲:心动则精自走。意淫属于暗耗精气神,也很伤身体,所以必须克服意淫。如果你满脑子意淫,总憋着不 SY,也容易出问题。要杜绝意淫,必须深刻地认识到意淫是什么,YY 其实就是念头,邪淫的念头。这个念头像一样东西,这个东西就是“火”!

有了这个认识,大家的觉悟就会上一个层次。接下来,大家再来思考一个问题:火在什么时候比较容易扑灭?有点常识的人都知道,火在刚开始起来时最容易扑灭,几乎不需要多大力气就可以扑灭。否则,当火星变成了大火,那根本无法控制,毛主席曾经说过:星星之火,可以燎原。一旦等它发展壮大了,就不是你可以控制的了,而在刚开始时,你还是可以完全控制的。我们的意淫也是如此,当邪淫的念头刚开始出现时,我们马上断掉,这叫“念起即断”,在它还没成气候时,把它扼杀在摇篮里。否则任其发展壮大,最后就是欲火焚身,极易导致破戒。所以,断意淫是有窍门的,这个诀窍就是断意淫的时机,当意淫刚出现,必须马上断掉,否则它就会“越烧越旺”,必须在它还是火星时灭掉它,不能犹豫,不能妥协,千万不能错过断意淫的最佳时机,切记!

断意淫口诀:念起即断,念起不随,念起即觉,觉之即无。这个口诀很管用,大家背熟它就可以形成条件反射,一有意淫念头,自动就会断掉。当然,前提是背得滚瓜烂熟,不断重复再重复,深入潜意识,这样就能形成断 YY 的条件反射。我们一定要保持极高的警惕和敏感,不能等到它发展壮大了,再去断,那时候就很难控制了。所谓不怕念起,就怕觉迟,就是这个道理。

下面介绍两种佛教修心的方法。

一种虚云法师开示过,就是不要去理会妄念,妄念来了,不要去管它,这其实就是念起不随,不要去跟随妄念,跟随妄念就是在强化它,不跟随就是断!很多戒友都是跟着邪念跑了很久,才突然发现自己是在意淫,这时候发现已经晚了,邪念已经起势了,就像小火星已经变成大火了,这时候断意淫就比较被动了,已经失去了最佳时机。

还有一种叫转念法,当意淫出现时,立刻转成佛号,当然不只是意淫,包括贪嗔痴等妄念,一旦出现,马上转成佛号,一句“阿弥陀佛”就转过来了,这种方法很好,我用得比较多,我本人信佛,这个方法对我很管用,有佛缘的戒友可以尝试下。

总之,要戒色成功,意淫关必须过。

身体要恢复,频遗关必须过。

戒色的确不是那么简单的,不是说我意志力强,就可以戒掉,戒色要成功,必须多学习戒色文章提高觉悟和定力,这样才有望彻底戒除,一旦停止学习和放松警惕,那就很容易破戒。戒色必须专业,什么叫专业,大家都知道有职业玩家和业余玩家之分,篮球有职业篮球和业余篮球之分,戒色也是如此,必须通过学习让自己戒得更专业更到位。不学习戒色文章,你永远无法变得专业,永远只能处在菜鸟级别,要想彻底戒色成功,真的很难。

真正有觉悟有善根的人,必定是每天坚持学习的人,每天都有所领悟,这样他的觉悟和定力提高得就非常快,我在戒色吧这么久,也的确发现不少戒友进步很快,觉悟很高。而有部分戒友,戒了很长时间还是不行,戒色意识还是很模糊,很多问题还是认识不清,这怎么能够成功?如果你对戒色还有疑惑,立场还不够坚定,那要戒色成功,无异于痴人说梦。只有通过学习改造思想,让思想认识有了飞跃,这样才有可能彻底戒除,永不复手!

下面谈下反复现象。

有些戒友会问,戒断反应和反复现象有何区别?这两者的共同点都是身体会出现症状反复,但两者还是有所区别的。区别如下:

戒断反应一般在戒色后一个月内会出症状,有些戒友戒色前没事,戒色后就出症状了,这就是戒色后正气有所恢复,潜在的病邪自然就会显现,坚持戒色,症状会逐渐消失的,有些戒友不懂戒断反应,一出现症状就慌了,还以为是禁欲有害,结果就又掉进 SY 陷阱不能自拔。所以,要戒色成功必须建立起正确的认知,否则只有失败。

而反复现象出现的时间跨度就比较大了,有人戒色后 3 个月出现症状反复,有人是戒色半年出现症状反复,有人一年左右也会出现症状反复,出现症状反复是很正常的,据我研究,很多戒友在遗精后都会出现症状反复,还有就是最近熬夜、劳累、久坐、生气、饮食不节、感冒等原因,都会导致症状反复的出现。这个道理其实很好理解,当你还没养足肾气时,遗精乃至不良生活习惯都会伤到肾气,那就很容易出现反复症状。还有就是季节转换的原因,因为每个季节人体的阳气水平都不同,一般在季节转换时容易出点症状。出现症状反复,不用担心,好好休息几天,按时作息,保持饮食清淡,症状会慢慢消失的。

再来谈下嗜睡症状

肾虚的表现有一个名词叫“嗜卧懒动”,不仅是 SY 后,乃至遗精后,人都会出现这种表现,就是变懒了变得爱睡觉,睡不醒。很多戒友因为睡不着,会通过 SY 来让自己产生困倦感,然后会更容易入睡,这种方法其实非常不好,当肾气伤到一定程度,就会出现睡眠障碍,到时候就是你睡前 SY,也不容易睡着了,所以必须改变这种睡前 SY 的习惯,必须及时纠正它,就像一棵树长歪了,必须把它弄正了,不能让它顺着歪劲长。

分享一个案例:

我 14 岁时开始有 SY 史,而且很频繁,不过后来知道了 SY 对身体不好就停止了,但是现在晚上经常睡不着,白天头晕脑涨,腰部酸痛,小便发黄,是不是我 SY 过度导致了肾虚啊!我该怎么办我已经有好几年没睡个好觉了!

有睡前 SY 习惯的戒友,必须纠正这一习惯,习惯的力量是非常强大的,但必须学会克服它,否则等待你的就是症状。SY 后嗜卧懒动,其实就是气虚的表现,有人 1 次不过瘾,连着来 2 次,这种人将来肯定会出现早泄阳痿倾向,因为“欲不可强、欲强则毁”,必须懂这个道理。

最后来谈下压力导致的 SY 问题

据我研究和总结,一般戒友中普遍会出现两种不良倾向:

\begin{description}
    \item[睡前 SY] 就是通过 SY 来让自己更容易入睡,这种习惯太伤肾气,积累到一定程度,反而睡不着。
    \item[压力导致的 SY] 就是把压力变成 SY 行为,这种情况也极其普遍。
\end{description}

睡前 SY 我上面讲到过了,下面就来谈谈压力导致的 SY。

压力导致的 SY,分为好几部分,有求职压力、学业压力、父母的压力、人际关系压力等,人活在这个世界上,会受到各方面的压力,所以有句话叫:生容易,活容易,生活不容易。

我遇见的戒友,其中有求职失败,或者被老板骂,被父母骂,心中情绪失控,精神压力非常大,在这种情况下寻求 SY 的慰藉,希望通过 SY 来摆脱不良情绪,说实话我以前也经常这样,被骂了或者没考好,晚上肯定 SY,好像 SY 是一个发泄的出口,就是希望 SY 把自己搞累搞空虚,然后什么也不想,一天就过去了。

这种发泄的方法其实是不可取的,因为这种方法对身体的健康是有损害的,虽然是可以起到缓解精神压力的作用,但总的来说是弊大于利,所以必须克服这种不良倾向,生活中有压力可以通过其他方式来疏导,并不一定要通过 SY,比如运动方式就比较好,当然要注意适量运动。我提倡的方法就是注重调心,压力使你的心理失衡,你要学会把它调整到平衡的状态,让自己保持在心平气和的状态,这种调心的能力就是情绪管理的能力,大家可以多看看情商方面的书,对大家更好地管理自己的情绪是很有帮助的。

我咨询的戒友中,很多人破戒就是情绪出了问题,生活中的不如意,生活中的压力,导致暂时的情绪失控,这种心理状态就比较容易破戒了,心里不爽,没处发泄,不如 SY 吧,就这样鬼使神差般地破戒 SY 了。

我写这篇文章,就是让大家能更清楚地认识到情绪管理的重要性,有一个稳定的心理状态,戒色才有望彻底成功,永不复手!

\subsection{前列腺炎补充、睡眠问题、赖床破戒、周末破戒}

最近解答戒友的提问,发现一个问题,有很多戒友因为 SY 得了前列腺炎,所以会比较关注前列腺的问题,会去前列腺的贴吧和前列腺的论坛学习,前列腺炎的贴吧就不用去了,虚假的广告信息满天飞,前列腺的论坛倒是可以去,但是那里也充斥着一种错误的思想倾向,这种思想倾向很具有迷惑性,一般的患者看了以后,就会被洗脑,会很认同那种思想,不仅认同,还会宣传那种思想,这种思想就是禁欲对前列腺会有损害,必须定期排精,否则前列腺炎会加重。

相信有点阅历的戒友都看过,我当年也被类似的观点害过,那时我戒色后,的确前列腺炎的症状有所加重,但那时我并不知道,这是戒断反应,正气一恢复,病邪自然会显现出来,只是暂时加重了,坚持戒色,积极锻炼,前列腺炎的症状会逐渐消失。很多戒友戒色后,发现出症状了,马上联想到禁欲有害,马上又开始 SY,结果是又掉进了 SY 陷阱不能自拔。而且他会变得心安理得,因为有文章说要定期排精,会觉得这样放纵,并没有什么不对。其实这真是大错特错了,我敢肯定,写出这篇文章的人认识上有盲区,他肯定没认识到戒断反应,也没有坚持戒色,只是被某种不成熟的理论所误导,结果写出这么一篇文章,以讹传讹,一传十,十传百,真是假传万卷书。现在前列腺炎患者普遍都接受这个不排精有害的思想,我要说的就是,这个思想是完全错误的理论,害人匪浅。

为什么这种错误的理论会大行其道呢?

当然也有人提出这是阴谋论,我想说这也是有可能的,很多男科医院靠什么赚钱?前列腺炎患者就是他们最大的收入之一,一进去就是一通检查,很多患者检查费都几千,治疗费上万的都大有人在,结果是什么?结果是,前列腺炎依然没根治,还是不行,花了上万的医疗费还是无法治愈,原因何在?很多人一辈子,直到死亡,都是前列腺炎患者,一辈子无法治愈。

一种思想可以毒害整整几代人,如果这种思想不纠正,可能会毒害几十代人,太可怕了。后面的推手是谁?我想有 3 种可能:

\begin{enumerate}
    \item 75 党撒播的(自慰专家一伙)
    \item 男科医院故意撒播的
    \item 认识上有盲区的人想当然编造的
\end{enumerate}

下面就是我对那篇文章第 11 点的指正,如下:

\begin{quote}
    11. 纵欲和禁欲都会造成前列腺的过度充血。所以,要自己掌握好这个尺度:没有性的生理需求时不要进行性交,而有了性的生理需求时就要顺其自然,让性冲动能量得到合理的渲泄。保证规律的性生活和性伴侣,没有结婚的朋友可以手淫,通常 4 - 7 天排精一次。
\end{quote}

网上有很多文章的论点都有问题,当然很多无害论也会出现这种似是而非的论点,对新人或者体验认识不深刻的人来说,是比较具有迷惑性的。

如何辨别真伪呢?只有一个办法,找过来人,找真实的案例。

作为过来人,我的论述如下:

\begin{description}
    \item[「禁欲会导致前列腺过度充血」] 这一条我不敢苟同,目前也没研究表明,禁欲会导致充血。
    \item[「而有了性的生理需求时就要顺其自然」] 这一条也有问题,很多人上瘾后,每天都有性的生理需求,难道每天都要 SY 吗?一个人必须学会克制自己的欲望,否则就会导致自毁,无数戒友的案例已经说明了这个问题。
    \item[「没有结婚的朋友可以手淫」] 这一点我更不敢认同了,很多人肾气都亏损了,根本伤不起了,这种情况难道还要 SY?这只会形成恶性循环,前列腺炎永远别想痊愈了。你自己想想,你 SY 前有前列腺炎吗?肯定没有。现在因为 SY 得了前列腺炎,还奢望通过 SY 来控制前列腺炎,这不荒谬吗?
    \item[要自己掌握好这个尺度] 写这句话的人肯定没认识到 SY 的高度成瘾性,一旦成瘾,自己根本无法控制,你是被瘾控制的,身不由己。
\end{description}

我原来前列腺炎也非常严重,尿频很严重,做过前列腺液的检查,显示有炎症,现在我禁欲 2 年多,尿频彻底好了,再也没腰痛了,小腹也不涨了。如果我相信这种站不住脚的歪理,那我现在还会尿频,永远别想好了。说实话,我以前相信过不排精有害的理论,结果是什么呢?十几年了前列腺炎都没好,看了很多次医生,吃了很多药,依然不行,只能缓解,无法去根!

很多人好了只是暂时的痊愈,并没有彻底地好,如果他再纵欲,肯定还会复发,这就是为什么前列腺炎的复发率为 90\% 以上,因为很多人认识上有误区,还在认可所谓“ 4 - 7 天排精一次”的荒谬理论。

吃药能痊愈吗?NO,相信有治疗经历的前列腺炎患者深有体会,吃药根本无法痊愈,只是暂时消炎了,无法去根。现在还发明出来乱七八糟的治疗器械和术语,什么个体分型疗法,什么最新智源肽通导技术,什么 STQ 强效脂膜渗透修复因子,相信将来肯定会有更新的器械和术语,但目的只有一个:就是让患者掏钱,而且先进的器械一般治疗费用都很厉害,一个疗程几千元是普遍的价位。如果真能彻底治愈,掏那么多钱也心甘情愿,但结果呢?根本无法治愈,照旧复发。

其实要治愈必须改造思想,三分治、七分养,有了前列腺炎是要积极治疗,但重心在养生上面,养生就包括了戒色,你必须养足肾气,当然养生深有讲究,我 12 季的文章有专门讲到养生,大家可以看看。

禁欲是否有害,我在 16 季的文章已经阐述过了,大家也可以看看。

要禁欲,必须学会调心,如果YY严重,是容易出现问题的,如果你修心功夫到位,禁欲是不会有害的,养足肾气,前列腺炎会自愈。

看到我这篇文章的慢前患者,希望你们有所觉悟,觉悟后就能省医药费,否则医院的泌尿外科,还会向你招手,我戒色前,经常跑泌尿外科,戒色后 2 年多,一次都没去过,医药费就这样省下来了,很多戒友反映SY后留不住钱,就是因为 SY 伤了肾气,医药费花钱如流水。切记!

下面谈下睡眠问题:出现睡眠障碍的戒友非常多,主要有以下几类:

\begin{enumerate}
    \item 失眠
    \item 难以入睡,入睡困难
    \item 多梦
    \item 早醒,包括半夜醒来
    \item 睡眠表浅,睡眠质量差
\end{enumerate}

睡眠的确是一个大问题,如果你睡眠质量不行,那么身体要恢复就相对比较难了,而导致睡眠障碍的原因也非常多,一般 SY 戒友主要的原因就是 SY 导致的心肾不交,还有就是久视伤肝血,还有就是压力大,另外熬夜久坐的人也极易出现睡眠问题。特别是熬夜,一般熬夜都要久视久坐,这样伤身体就比较厉害了。所以要恢复睡眠质量,必须调整作息,尽量在 23 点之前上床,睡不着就闭目养神,一定要养成正常的作息习惯,熬夜实在太伤精气神。

有的戒友晚上睡不着,一问,原来是喜欢睡午觉,有的人一睡午觉,晚上就难以入睡,这种经历我也有过,后来我就不睡午觉了,如果不是太累,就不用睡午觉了,否则晚上容易入睡困难。

如果是偶尔一次的睡眠问题,不用太担心,建议调整作息,不要搞得太累,并且要学会调心,不要给自己太大压力,这样睡眠问题就能自动调整过来。

如果是顽固性的睡眠障碍,比如严重失眠或者持续的早醒和多梦,这种情况就说明伤得比较严重了,最好能找个中医配合中药调理,然后自己要注意作息规律,不熬夜不久坐,积极锻炼,学会养生之道,这样坚持下去,大概需要半年时间才能调整过来。我当年多梦和失眠比较严重,天天做梦,有时就直接失眠了,非常困扰,后来我把作息调整过来了,然后学会了养生之道,用了一年多才恢复正常,因为我伤得比较严重,所以用了一年多。这一年就是禁欲加养生,药也没吃,就靠自己。当然,如果病情严重,那最好还是去找个好点的中医配合中药调理,这样恢复能更快些。

要让自己的身体恢复得更好,必须提高睡眠质量,很多人能吃能睡,积极锻炼,恢复很快。有的人睡不着或者睡眠质量差,这样要恢复就相对比较困难了。睡眠就像手机充电,电充不进去,何谈恢复?

最后来谈下赖床破戒和周末破戒:

很多戒友反映赖床后就容易破戒,还有一到周末就破戒,平时上学没事,其实这两种情况我当年也经历过,一般学生党比较容易出现这种问题。

经过我几年的研究和案例分析,发现人在一种心理状态下极易破戒,这种心理状态就是:无聊。

赖床时其实很无聊,就躺在床上瞎想,很多人这时候就喜欢 YY,或者有了晨勃,就关注 JJ 的长度和硬度,这样就极易破戒了。正确的做法是,有了晨勃,不要去关注,让其自生自灭,不关注不助念,它自然就下去了。周末时,人比较空,也比较容易出现无聊情绪,一等父母出门或者自己关在房间里,就开始动歪脑筋了,学生党定力尚浅,认识也不深刻,无聊时心魔就会跳出来,这时候就容易破戒,所以必须多学习戒色文章提高觉悟和定力,始终保持高度警觉,不能放松警惕。

当你通过学习获得足够的定力后,无聊时就不会破戒了,就能真正做到心地光明磊落,不欺暗室。身正不怕影子歪,当你的心正了,就不会在无聊时破戒了,否则你平时上课没事,一到周末无聊时,就容易出现破戒。我的建议就是:不要离开戒色文章,当然学习戒色文章也会出现厌倦情绪,我前面的文章有讲到,当你厌倦戒色文章时,必须学会克服厌倦情绪,保持住戒色的热情,你不一定要找新的戒色文章,每天看几遍你自己认为比较好的戒色文章即可。还有一种选择,就是看受害者的案例,受害者的案例就是警钟,就是最好的警示,你每天看受害者的经历,就能让你保持在警惕的状态,否则一旦你厌倦了戒色文章或者出现松懈情绪,那就极易破戒了。切记!

另外,有赖床习惯的戒友必须克服这一习惯,要克服必须先下决心,告诉自己明天一醒就下床,绝不赖床,决心一定要到位。出现无聊情绪时,要学会调整心态,让自己的生活充实起来,最好的充实就是学习,古人说学习可以“医愚”,我觉得很有道理,不学习不知道,你可以学习戒色文章,可以看受害者案例,也可以学些你自己感兴趣的内容,总之,就是要让自己充实起来。

结语:

今天我去李毅吧逛了一下,大概一分钟,那个地方的确比较乱,很多青少年被纵欲主义思想洗脑,真是深陷其中,他们就像“温水青蛙”,根本不知道邪淫的危害,以错为对,太可怕了,症状迟早会收拾他们的,希望他们能早日醒悟。在这里我要向宣传戒色的戒友表达深深的敬意,也许你们的宣传会引起不明真相的人的误解,但肯定会有人看到帖子有所觉悟的,肯定会有人回头的,很多人之所以不理解戒色,是因为他们还没认识到危害的严重性,还以为放纵是理所当然的事情,因为大家都在这样干,其实是大家都错了,错得离谱。宣传戒色,其实就是在告诉大家真相,让他们认识到危害,建立起正确的认知,否则等待他们的就是症状和医院,而且整个生命质量都会受到很大影响,因为 SY 摧残的是身心。

李毅吧给我的感觉,就像一条被污染的河流,而戒色吧则是一泓清泉,所谓近朱者赤近墨者黑,很多人在李毅吧并不觉得自己被污染了,如入芝兰之室,久而不闻其香;入鲍鱼之肆,久而不闻其臭。当一个人在一个环境里习惯了以后,就失去了敏感和警惕,就像在城市里生活久了,去山里一段时间,才会发现原来空气可以这样清新,原来空气可以这样纯净。

很多人会说,我们生活在和平年代,我并不这样认为,所谓侵略有 2 种,一种是发动战争,还有一种就是文化侵略,西方乃至岛国对中国的文化侵略实在是很厉害,在自己国家宣扬禁欲论,却大力生产堕落光盘,给别的国家的青少年看,削弱他们的脑力和体力,而中国现在的青少年普遍缺少信仰,自制力更是一塌糊涂,一旦开始 SY,身心都极易出现问题。国家由人组成,而青少年更是国家的希望所在,如果我们不能抵制纵欲主义文化侵略,中国将来的命运的确令人担忧。作为炎黄子孙,我们应该多宣扬正气,帮助身边的人觉醒,有时候你的几句话,就可以改变一个人一生的命运。

\subsection{晨勃问题、易漏问题、大便问题、睾丸小}

晨勃是戒友普遍关心的问题,很多戒友都把晨勃作为自己肾气是否恢复的标志,有晨勃就高兴,没晨勃就心事重重,患得患失,搞得一天都没好心情。其实身体是否恢复,从很多方面都可以表现出来,你自己照照镜子,眼睛是否有神采,气色如何,整个精气神如何,握力如何,腿是否有力,其他的不适症状是否缓解乃至消失,这些都可以看做身体是否恢复的指标,而晨勃只是众多恢复指标中的一个而已。

只要坚持戒色,晨勃是会恢复的,如果你只有十几岁,那晨勃恢复相对比较快,如果上了 25 岁或者伤得比较严重,那晨勃恢复就比较慢了。有的戒友是戒色养生 200 多天才恢复晨勃,我当初大概用了一年多才恢复晨勃,因为那时我已经 27 岁了,十几年的纵欲经历让我的肾气亏损严重,所以我用了一年多才恢复晨勃。我现在虽然恢复晨勃了,但已经不可能和十几岁时相提并论了,那时在发育期,吃得下,并且吃得好,那时晨勃质量高,只要 SY 不是很频繁,一周可以晨勃 5 天左右。但这种晨勃频率也只维持了 2 年左右,也就是刚发育那会。后来SY频繁,晨勃就很少了。现在我晨勃的频率大概是一周 3 次左右,有时质量高,有时质量一般,我现在一周吃 5 天素,所以一周能有 3 次晨勃已经很不错了。我以前出现过早泄和阳痿倾向,一般有早泄阳痿倾向的人,晨勃要恢复很难,至少需要大半年以上了。

我现在基本不关心晨勃了,有最好,没有也无所谓了,因为身体恢复并不是晨勃一个指标,其他方面我恢复得很好,而且影响晨勃的因素也很多,一般如下:

\begin{description}
    \item[饮食因素] 补肾食物的摄入量
    \item[锻炼因素] 适量的有氧运动和力量训练是可以帮助晨勃恢复的
    \item[情绪因素] 不良情绪会影响到晨勃,比如生气、压力。
    \item[年纪因素] 青春期更容易晨勃,年纪越大,晨勃越容易消失
    \item[体质因素] 先天体质好的人容易有晨勃
    \item[疾病因素] 没其他慢性疾病的人更容易有晨勃
    \item[撸龄因素] 一般有 5 年以上的人晨勃容易消失
    \item[季节因素] 不同季节晨勃的频率和强度都有差别
    \item[每月因素] 每个月的不同时间段,人体阳气是不同的,出现晨勃也会不同
    \item[憋尿因素] 有时憋尿可以导致晨勃的发生。
    \item[劳累因素] 过于劳累会伤到肾气,晨勃容易消失
    \item[睡眠因素] 睡眠质量高的人,更容易有晨勃
    \item[频遗因素] 频繁遗精后,晨勃容易消失
    \item[戒断因素] 刚开始戒,会有欲望休眠期,晨勃易消失,坚持下去又会恢复
    \item[] 是否有其他不良生活习惯,比如熬夜久坐抽烟,都会伤到肾气,影响到晨勃
\end{description}

晨勃是指无 YY 的晨勃,如果有 YY,就不叫晨勃了,那是你念头导致的勃起。晨勃其实很危险,为什么说晨勃危险呢?一般人并不知道这个道理。因为很多人定力尚浅,一有晨勃就挺高兴的,然后会趁着勃起之势,去摸自己的 JJ,这一摸就完蛋了,不知不觉就又开始 SY 了,这种晨勃破戒的情况在周末是非常普遍的。我现在晨勃,一不在乎,二不去看,三不去摸,任其自生自灭,如果你很在意晨勃,弄不好就会走火入魔。你不在意它,它自己就下去了,不会导致破戒,很多戒友有赖床习惯,这样如果有晨勃,就非常容易破戒了。切记!

很多人会说自己的晨勃时有时无,其实这个很正常的,你以为你永远都在发育期吗?就是你在发育期,要做到每天都晨勃也很难,因为影响晨勃的因素有很多,不可能一年 365 天,天天晨勃,那是不现实的,为何晨勃会时有时无呢,这和季节、生活习惯、饮食、年纪都有密切关系,一般吃得好的人,并且消化吸收能力强的人,更容易有晨勃。以前我经常吃荤菜,那段时间我晨勃相对多些,现在我一周吃 5 天素,晨勃相对较少了,但是心里却清净很多,不容易起邪念。戒色期间吃素不错,有利于控制欲望,吃肉多容易助长欲望,如果定力尚浅,那就很容易出现破戒。吃素是一种选择,要看个人的接受程度,我建议广大戒友从减少吃肉开始,能够完全吃素我也是非常支持的。

下面谈下易漏问题。

易漏问题,指的是长期纵欲,肾气不固,在很多情况下都容易漏精,具体如下:

\begin{enumerate}
    \item 小便后漏出
    \item 大便后漏出
    \item 和女友打电话漏出
    \item 滑精
\end{enumerate}

这种易漏问题,打个比方大家就明白了,就好比一个水龙头用久了,关不紧了,就老漏。小便后漏出的情况其实挺多的,其实就是肾气不固的表现,大便后漏出的情况更普遍,特别是刚开始戒色时,更容易出现这种情况。还有不少戒友反映,和女友打电话,没聊到敏感内容,但也漏了,这其实和潜在的条件反射有关,有女友的戒友,要戒色比较难,所以最好和女友沟通一下,尽量避免婚前性,自己则要加强修心,把握好交往分寸。滑精的戒友也比较多,如果偶尔一次滑精,倒不必太担心,如果持续滑精,那就要及时就医治疗了,另外,建议多做固肾功。一般坚持戒色养生,积极锻炼,小便后和大便后漏出的情况会自动消失的,我以前也出现过大小便后漏出的情况,后来我坚持做固肾功,这种情况就极少出现了。

当然,漏出的情况不止这 4 种,比如 YY 和紧张也会导致漏出的情况,即使你肾气充足,沉迷于 YY,也会漏出。所以要坚决杜绝 YY,做到念起即断,并且多学习戒色文章提高觉悟和定力,这样才有望真正戒除 SY 恶习。

再来谈下大便问题。

大便问题在戒色吧的帖子里也时常出现,因为中医:肾主两便,大便和小便,肾一虚,大小便就容易出问题,小便问题就不用多说了,相信有前列腺炎的戒友深有体会。

肾虚导致的大便问题一般分为2类:

\begin{enumerate}
    \item 腹泻或者大便不成形
    \item 便秘
\end{enumerate}

导致腹泻和便秘的原因有很多,其中就有肾虚这个原因。大家可能会记得小时候的大便都非常好,一是颜色好,二是形状好。小孩子阳气充足,童体未破,如果没有其他疾病,一般消化吸收能力都是非常好的,大了以后就不同了,一是 SY 纵欲,二是冷饮或者刺激性的食物吃多了,或者饮食不节、暴饮暴食等原因,都容易出现消化系统的功能障碍,去检查也检查不出来什么,就是功能紊乱了。小孩吃点冷饮没事,因为小孩阳气足,能化掉。纵欲以后就不行了,一吃就容易出现胃肠功能的紊乱。所以一定要注意忌口,不要贪吃冷辣的东西。如果你有腹泻或者便秘的烦恼,最好去就医治疗,然后配合戒色和养生,积极锻炼,注意保持饮食清淡,这样慢慢地就能调理过来。便秘和腹泻的问题我都经历过,便秘是 19 岁开始的,困扰了我 2 年多,就是因为纵欲加吃冷饮导致的,后来 26 岁又开始腹泻,就是吹空调、暴饮暴食、加上纵欲导致的,我彻底戒色后,经常艾灸足三里和神阙穴,慢慢地就调理过来了,用了 1 年多时间。

最后再来谈下睾丸小的问题。

睾丸小睾丸萎缩的问题,在戒友中也时常出现,SY 恶习导致睾丸的问题一般如下:

\begin{enumerate}
    \item 睾丸下垂,不收缩
    \item 睾丸萎缩变小
    \item 阴囊潮湿
    \item 精索
    \item 阴囊湿疹
    \item 阴囊肿大
    \item 阴囊瘙痒
    \item 睾丸疼痛
    \item 睾丸炎症
    \item 睾丸囊肿
    \item 隐睾
    \item 其他病变
\end{enumerate}

睾丸萎缩的戒友最好去医院做个检查,确诊一下病情的严重程度。

睾丸萎缩,是指男子睾丸缩小痿软病证,以一例或双侧睾丸萎缩,既小又软为特征,大多数能引起不育,多继发于腮腺炎或外伤,也有先天者。多因肾气亏乏,或病邪损伤引起。先天性睾丸发育不良者不易治愈,继发性睾丸萎缩者亦需耐心调治,中医辨证治疗可以取得一定效果。

SY 恶习导致肾气亏损,继而导致睾丸萎缩,这在戒友中屡见不鲜,所以一旦出现睾丸萎缩的情况,一要积极治疗,二要配合戒色和养生。这样把肾气养足后,才有望恢复正常。

有的戒友还会问到睾丸大小不一样的问题,一般正常人的睾丸都不是等大的,但如果差别特别大,那最好去做个检查,如果差别不大,那就不必太担心。

\subsection{破戒类型,射距奥秘,连续 2 次的恶果}

前言:

最近吧里有帖《光戒是没用的,伤精患者唯一的复原之路在此》,有不少戒友跑过来问我静阳子说得对不对,一看文章内容,很多戒友慌了,不静坐就好不了,这到底是真是假?我看了静阳子的发言,他推荐的“因是子静坐法”很好,打坐的确是补元气的一大妙法。当然打坐也分很多种,也有很多讲究,我自己每天也打坐的,但我学的是南怀瑾静坐。打坐的确对于戒色后身体的恢复很有帮助,但并不是唯一的恢复方式,也并不适合所有人,我聊过不少焦虑症患者,一坐就头晕得不行,后来我建议站养生桩,身体就恢复得越来越好了。

有人喜欢打坐,有人更喜欢站桩,有人则喜欢八段锦和六字诀,有人练习太极拳,有人练习易筋经,中国的养生功法很多,可以选择适合自己的功法。对于静阳子说的“运动锻炼更是扯淡”,我不敢苟同,虽然中医有讲到劳则耗气,但也讲到动则升阳,中医并不反对运动,而是提倡适量运动,一味打坐并不是最好的恢复方式,要动静结合,华佗的五禽戏就是让人们动起来,但华佗并没有叫大家举石锁,华佗五禽戏的运动量和运动方式是有益于身体健康的。

华佗名言:“人体欲得劳动,但不当使极耳,动摇则谷气得消,血脉流通,病不得生。譬如户枢,终不朽也。”

华佗这句话其实就是告诉我们运动的法则:要运动,但不要过度。在身体虚的时候,更要注意运动适量。我相信中国任何一个名中医都不会反对运动锻炼,他们反对的只是过度运动。就算是道士,也并不是完全靠打坐养生,很多有名的道士都习武,这样动静结合,更有利于养生修道。

静阳子说的“光戒是没用的”,这句我倒是比较认同,因为戒色是地基,养生才是造房子。光戒的确远远不够,必须学会养生之道,建立自己的养生意识,这样对恢复才比较有利。静阳子推荐的静坐法其实就是养生的一个法门,这个法门并没有错,大家可以尝试着练习。尝试了,你才知道效果如何,适不适合自己。当然,容易出偏的功法一定要谨慎练习,否则没毛病也会练出毛病来。

如何更好更快地恢复的确是戒色后的一个问题,但更重要的是做到彻底戒色,这才是最关键的。否则谈再多的恢复也等于 0,就像一边吃中药,一边 SY,上补下漏,根本无法真正痊愈。

让身体恢复的法门很多,静坐只是其中一个法门而已,前段时间土豆吧主推荐的拜忏也非常好,土豆吧主就是靠拜忏恢复了健康。法门很多,请选择适合自己的,有佛缘的可以尝试拜忏。另外,除了养生功法,多做适量的有氧运动对身体恢复也是极其有利的,比如慢跑或者快走等。

相信静阳子的初衷没错,他大部分的观点也没问题,就是语气上给人盛气凌人之感,大家可能不大能接受。他是为了大家好,但如果换一种平和的方式,也许大家更容易接受些。

对于静阳子,我表示感谢,他的心是好的,希望他能更好地帮助到大家。这里的氛围不是敌对,而是善意的沟通,通过沟通来互相促进。

下面言归正传。

这季就破戒类型,射距奥秘,连续 2 次的恶果这三个方面详细论述一下,具体如下:

破戒是每天都在发生的事情,破戒后往往很沮丧,这是不难理解的,当初戒色是雄心万丈,决心极大,没过多久就产生了动摇,一而再,再而三地破戒,自己都对自己失望了。身体症状告诉你不能再放纵了,再放纵就废掉了,但瘾告诉你继续 SY 吧,绝大多数戒友都是被瘾牢牢控制着,像一个提线木偶,成了欲望之奴,身不由己。明知道不对,但就是停不下来,在怪圈中苦苦挣扎。

失败并不可怕,可怕的是不学习,不学习觉悟和定力就不会提升。不学习的强戒,注定失败!有戒友会问,有人是一次戒除的吗?说实话,我到现在阅戒友上千,没见过哪个人是一次戒除成功的,都是反复破戒,不断总结经验教训,不断学习戒色文章提高定力和觉悟,这样才慢慢戒掉的。有人会说,我见过有人一次戒除的,其实,一次戒除这个概念有点模糊,是彻底觉悟后一次戒除还是第一次戒色就成功?我就是彻底觉悟后一次戒除成功的,我也属于一次戒除成功,但并不是第一次戒色就成功,之前也破戒过无数次。我第一次想戒色要追溯到初中了,那时候就觉得 SY 这个行为很龌龊,SY 后体质明显下降,十几岁时我就想戒色了,但那时没戒色资料,没人引导,也没学习戒色文章的意识,就是一味地强戒,结果就是不断破戒。我想大多数人尝试第一次戒 SY 都是在十几岁时,因为人是有戒色本能的,身体会告诉你不能再放纵了。而尝试第一次戒色,很多方面都无任何经验,属于戒色菜鸟级别,一没戒色知识,二没戒色意识,在这种情况下想一次戒除简直是天方夜谭。打个比方,让一个从没学过开车的人第一次开车,结果是什么呢?结果肯定是开得歪歪扭扭,甚至撞车。

戒色就像学开车,你必须掌握一整套戒色的理论和方法,然后熟能生巧,这样才有望戒色成功。你要说这个世界上是否真的有人第一次戒色就成功,我想即使有,这种人也是千里挑一的,善根极其深厚,而且前提是,在第一次戒色时,他就接触到了大量的戒色文章,并且善于学习,悟性极高,就像念书跳级一样,别人十年都没搞懂的道理,他一个月就弄明白了,这种人拥有不世出的戒色天赋,千里挑一不为过,甚至是万里挑一。

戒色必须专业,这季我总结了一些破戒的类型,大家可以参照地警惕自己,所谓“知己知彼百战百胜”,你知道了破戒类型,就会尽量保持警觉。就像知道了骗子的伎俩,就不会轻易上当,因为你知道那是骗术,不是骗钱,而是骗精!

\begin{description}
    \item[情绪破戒] 无聊情绪、愤怒压抑情绪、烦躁情绪、厌倦情绪等
    \item[疑惑破戒] 疑惑产生动摇,然后破戒
    \item[诱惑破戒] 图和视频乃至H段子
    \item[赖床破戒] 非常多见,必须纠正
    \item[压力破戒] 各方面的压力
    \item[YY 破戒] 一般YY是破戒的前奏,包括回忆和幻想
    \item[放松警惕破戒] 以为成功了,其实一放松就易破戒
    \item[狂欢破戒] 狂欢时容易放纵,所谓得意忘形
    \item[试定力破戒] 千万不可去试
    \item[晨勃破戒] 晨勃后摸 JJ,从而破戒
    \item[试性能力破戒] 特别是早泄阳痿患者,一试就破
    \item[鼓动破戒] 被邪友带坏,去不良场所
    \item[喝酒破戒] 酒是色媒人
    \item[遗精后破戒] 遗精容易产生思想动摇,从而破戒
    \item[周末破戒] 周末一个人无聊,心魔容易跑出来
    \item[补药破戒] 补药吃多了,不注重修心,容易破戒
    \item[吃肉破戒] 肉吃多了,不注重修心,容易助长欲望而破戒
    \item[无害论破戒] 看了无害论,容易破戒
    \item[有女友破戒] 有女友对定力的要求更高,也容易破戒
    \item[有老婆破戒] 有老婆对戒色定力要求也很高。
    \item[习惯性破戒] 当破戒变成习惯,产生强大惯性,后果很可怕
    \item[假期破戒] 比如暑假和寒假,比较空闲,容易失控
    \item[邪法破戒] 不管是 PC 肌抑或别的所谓 JJ 增大法,极易破戒
\end{description}

下面再来谈下射距的奥秘:

射距就是射精距离,很多戒友反映刚开始 SY 的几年,射距很远,到后来就越来越不行,是顺着 JJ 流淌下来的,明显后劲不足,没有冲力了,这其实就是肾气亏损的一个表现。精,一看颜色,二看量,三看浓度,四看射距。射距越远,就说明这个人肾气越足。这个道理非常像射箭,精就好比箭,而肾气则是射箭的那个人,肾气不足,射箭无力,自然就射不远了。养足肾气,射距才会增加。一旦射距变短乃至是流淌下来的,其实就是身体在给你信号了,告诉你肾气已经不足了,不能再放纵了。如果此时你没读懂身体的信号,一意孤行,结果真的就是不见棺材不掉泪。

很多人只看到精,却没看见背后“射箭的那个人”已经不行了,精相对比较容易恢复,好好休息几天,吃得好点,就能有所恢复,但背后“射箭的那个人”要恢复就很慢了,可能要几个月甚至更长的时间才能恢复元气。所以,大家不能把眼光仅仅局限在精,而要看到背后“射箭的那个人”。那个人已经不行了,射不远了,就说明要学会戒色了,不能再放纵了,要学会养生了,否则等待你的只有症状和医院。

那个射箭的人就是“肾气”,既然他射不动了,射不远了,疲惫了,就不要再让他射了,让他好好休养生息吧,否则真会累出病来,切记!

最后谈下连续 2 次的恶果。

如果你去过植物神经紊乱吧,就会知道 sg552 这个病友,他就是长期熬夜 + 连续 2 次 SY,导致的发病,出现了濒死感。连续 2 次实在是太伤身体了,中医早就有讲到:欲不可强。《千金方》有讲到:强力施泄,便成劳损。特别是连续 2 次,真是大忌讳。连续 2 次也非常容易出现前列腺炎和早泄阳痿的倾向,这其实就是人性的弱点贪婪导致的,对欲望的贪婪无度。要满足自己,结果是害了自己!

很多人之所以连续 2 次,其实就是一个原因:1 次不过瘾,1 次太短了,没爽够。正是这种对欲望的贪婪,导致了自毁。这种方式的自毁实在太多太多了,我们一定要时刻牢记中医欲不可强的教诲,不能去做傻事,你不知道这个道理,就会无知者无畏,结果害惨了自己,如果你知道了这个道理,还这样放纵,那完全就是自取灭亡!很多人无法战胜心瘾,就容易犯这种错误,就像中了魔法一样,破戒后才会悔悟。所以,我们一定要防微杜渐,好好管住自己的念头,尽量杜绝 YY,否则等火烧起来,就不是你所能控制的了。

结语:

夏季来临,不少戒友反映脱发增多,于是有些慌张,其实导致脱发的因素很多,季节因素就是其中一个原因,我最近脱发量也由原来的 5 根以内变成 20 根左右,这其实就是季节导致的,并不是戒 SY 导致的,所以在这一点上大家一定要认识清楚,否则容易产生信心动摇,从而转变成 SY 行为。我在导致破戒的因素中就讲到了“疑惑破戒”,如果你心中有疑惑,破戒的可能性就会很大。一般夏季人体汗腺油脂都分泌旺盛,加上紫外线强烈,会对头发造成损伤,这样就容易出现脱发增多的现象,不仅人脱发,你观察下动物,比如宠物狗,夏季脱毛也很厉害。

有位戒友说得好,戒色是一个系统工程。必须让自己戒得尽可能专业,必须让自己懂得更多,要让自己变成戒色内行才行,否则不学习,永远是外行,总是处在戒色菜鸟级别,要戒色成功就非常难了。当大家有了学习意识后,经过一段时间的学习,你会发现自己的觉悟和定力都有了提升,只要坚持学习,彻底开悟就不会遥远,彻底开悟后保持警惕,就是彻底的戒色成功。加油!

\subsection{撸完腹痛、意淫导致的疾病、睾丸的下垂}

前言:

最近有看到戒友反映,之所以破戒是因为看了暗示性的内容。

其实这种破戒方式就是诱惑性破戒,之前已经有很多戒友反映过类似的情况,有的戒友虽然开始戒色,但是看到杂志上有写到男女两性方面的内容,顿时很感兴趣,结果看着看着心理防线就放松警惕了,结果就破戒了。还有的戒友说,我不看黄的内容,我搜集清纯的女孩,搜集身材好的女孩的图片,不涉及黄色内容,这种戒友其实也很容易破戒,因为他还迷在女色里面,虽然是搜集清纯的图片,但常在河边走哪有不湿鞋,就像佛教所言,一着相就容易着魔。

还有不少戒友的头像乃至签名都有问题,这样去戒,其实还没戒就已经失败了一半,因为你不够决绝!要彻底戒色,就应该断得彻底,不仅是直接带黄的内容不能看,对于任何带有暗示的内容更不能去碰,一定要提高警觉。戒色的每一天,都要保持警惕,好好管住自己的心,降伏其心。

下面步入正文。

这季就撸完腹痛、意淫导致的疾病、睾丸的下垂这三个方面详细论述下,具体如下:

撸完腹痛的帖子在戒色吧不是每天出现,但的确是时常出现,撸完腹痛的情况我曾经也有过,出现腹痛的情况,一般考虑就是前列腺炎或者精索静脉曲张。一般撸完或者第二天容易出现腹痛的情况,如果腹痛比较严重而且症状持续,建议及时就医治疗,就算不是很严重,还可以忍着,我建议最好还是去医院做个检查,前列腺液和睾丸 B 超,看看具体的严重程度,也可以给自己一个警示。很多人虽然腹痛,但还能忍着,好几年都没去过医院,SY 后的不适症状,一般休养几天,是会慢慢缓解的,很多人这时就掉以轻心了,觉得没事,就一直拖着,等到很严重时才去医院,据我所知,很多戒友自己都能摸到睾丸上的曲张团块了,但就是不想去医院,学生党则是怕家人知道,不敢告诉家人。

一旦出现症状了,就应该有所觉悟了,人除了“不见棺材不掉泪”这个人性的弱点以外,还有“好了伤疤忘了痛”。很多人出症状后,知道收敛了,戒了十几天觉得没事了,于是又开始沉迷 SY,等到又出现症状,然后又开始戒色,就在这个怪圈里徘徊。结果是什么呢?身体的症状越来越多,年纪轻轻一身的病,这也不舒服,那也不舒服,人也变丑了。即使没毛病,给人的感觉也是那种精神萎靡,目光无神的颓废感觉。记得有一个戒友去面试,结果失败了,他把原因归咎于 SY,我想是有一定道理的,如果他精神很饱满,很自信,容光焕发地去面试,结果很可能就是录取了。面试官除了看你的能力,第一印象也是非常非常关键的,一个没有精气神的人,一个不自信的人去面试,是比较容易遭受失败的挫折的。SY 会导致自卑已经是不争的事实,因为人没有了底气,在肾虚的状态下,你无法把自己最佳的一面展现出来,而你那种颓废无力的精神面貌,很容易给面试官留下不佳的印象,印象分就会下降不少。

我前面的文章也讲到过前列腺炎和精索,这两个病都会导致不孕不育,影响精子质量。所以对待这 2 个病一定要及早引起重视,否则将来就悔之晚矣,记得另外一个戒友得了精索,没有生育功能,看病吃药和开刀加起来的治疗费用超过 5 万多。如果他能早日醒悟,及早地认识到 SY 的恶果,也不至于那么严重,这都是活生生的案例,活生生的教训,大家一定要牢记教训,这样才能防患于未然,以免重蹈覆辙。

下面谈下 YY 导致的疾病。

YY 能导致疾病,这个说法可能广大戒友第一次听到,其实在传统中医的医案上早有记载,中医:心动则精自走。YY 其实就是在暗耗精气,也很伤身体。有戒友会说,YY 流出的是透明的前列腺液,不是精液。

这里我要纠正很多戒友存在的思想误区:

其实前列腺液是精液的重要组成部分,约占精液的 15\% - 30\%,你漏前列腺液其实就在漏精,只不过你漏的不是精液的全部,而是精液的一部分。你 YY 虽然没射精,但同样也会走漏精气,漏前列腺液对身体的伤害也不容小觑,有戒友反映,光看不撸,然后照镜子发现自己精气神马上就萎靡了,更有戒友反映光看不撸后小腹疼痛,还有的戒友反映,昨天漏了前列腺液,今天就感觉不在状态,腿发软全身无力,甚至腰酸腰痛。

所以大家戒色必须重视断 YY,一定要控制住 YY,YY 是戒色后的一个大关口,很多人就是败在这个关口,虽然能做到不撸管,但做不到杜绝 YY。这样,对身体恢复很不利,而且也容易导致破戒。关于如何断 YY,我 17 季有专门讲到,不能控制 YY 的戒友可以看看,一定要学会克服 YY。把 YY 克服掉,你的戒色天数就会成倍地上涨,突破百天很容易,否则克服不了 YY,可能1个月都无法坚持。

还有一种情况漏前列腺液也比较严重,就是大便时和小便最后会漏出来点,这种情况可以通过坚持做固肾功来解决,效果还是很不错的,坚持做一段时间你会发现大便和小便时也不会漏了。

再来谈下睾丸下垂的问题。

睾丸下垂在戒友中相当普遍,纵欲过度就容易出现这个问题,一般纵欲过度会出现 2 种情况,一种是过度下垂,一种就是病态的紧缩,能缩成鹌鹑蛋大小,甚至能缩进腹部,这也是不正常的,大多数戒友出现比较多的情况就是睾丸下垂。

有一个词叫“缩如童子”,如果一个小男孩身体健康没有疾病,那么他的睾丸就会很紧缩很紧致,缩得很饱满。这其实就是肾气充足的表现,童体未破,也没有 YY,这样他的睾丸质量就比较好。很多戒友会说“热胀冷缩”,其实热胀冷缩属于外在刺激导致的睾丸变化,和肾气倒没多大关系。当然如果肾气亏损到一定程度造成严重下垂,那么,即使是冷刺激也无法使之上缩了。

一般成年人,都会出现下垂的情况,因为毕竟身体漏过了,只要不是过度下垂即可。根据我自己的亲身实践,在低头做固肾功时,再加上用意念缩肛缩睾,是可以起到紧缩睾丸的作用,对于睾丸过度下垂有治疗意义,当然也不一定要在做固肾功时缩肛缩睾,在平时站着坐着的时候也可以练缩肛功,我只是把固肾功和缩肛功结合在一起练了,因为这样练,我感觉效果更好些,能缩得更到位些。现在我一天中大部分的时间睾丸都比较紧缩,不像以前纵欲时那样下垂,那时我的睾丸下垂到能从内裤漏出来,现在缩得比较紧致,也比较饱满了。

后记:

贴吧有不少戒友反映,戒色后身体恢复很慢,有的戒色7个月身体还不行,有的戒色 1 年还不行,今天看见一个戒色 2 年身体还不行,戒色 2 年的那个戒友是因为频繁遗精,一个月遗精 8 次,这样的遗精频率身体还能恢复吗?中医:久遗八脉皆伤。遗精也是很伤身体的,戒色后必须要控制遗精次数,尽量控制在一月 3 次以内,最好是一月 1 次的频率,这样对恢复才比较有利。

戒色后要过 2 个大关:

\begin{enumerate}
    \item 就是频繁遗精关
    \item 就是 YY 关,YY 也很伤肾精
\end{enumerate}

另外,还要注重养生之道,光戒不行,戒色是地基,养生是造房子,养生的内容就比较多了,要自己不断去学习。戒色后一定要养成良好的生活习惯,不熬夜不久坐不久视,不要太劳累,按时作息饮食,积极锻炼,这都属于养生的范畴。通过不断学习养生知识,养生的意识有了较大提升,这样对戒色后的身体恢复会更有利。否则只戒不养,要恢复真的很难,特别是对于那些伤得比较重的戒友,尤其如此。

\subsection{做一个纯净频率的持有者}

\subsubsection{发出念头的波}

六祖慧能从五祖弘忍处继承衣钵,来到广州法性寺弘法,法性寺的主持方丈引宗法师正在讲经,风吹幡动,于是他问:“是风动还是幡动?”弟子中有说风动,也有说幡动的。慧能上前,合掌说:“不是风动,不是幡动,仁者心动。”

\begin{quotation}
    “一切物质都是波动的现象,皆是人们的错觉,宇宙间只有场和物质这两种东西,实际上只有场,物质不过是场里场强特别高的地方。”\hfill ——爱因斯坦

    “任何事物以及产生的效果都是波动的(vibration),没有任何实物存在,所有的实物都包含波动。”\hfill ——普朗克

    “这个世界是意识的全息图,现实是幻象,是存在于我们头脑中的全息幻影。”\hfill ——大卫艾克

    “客观现实并不存在,尽管宇宙看起来具体而坚实,其实宇宙只是一个幻象,一个巨大而细节丰富的全息相片(Hologram)。”\hfill ——物理学家 David Bohm

    《金刚经》云:凡所有相,皆是虚妄。
\end{quotation}

念头是一种能量的波动,宇宙的基础是振动的波形信息场,而念头就是振动的波形信息,只要振动,就会产生幻象。你起心动念,就是在发送波形信息,每种念头都有自己特定的频率,宇宙就是一个频率世界。我们小时候都是纯净频率的持有者,心地单纯而美好,干净而清澈,在发育前,我们很少会起邪淫的念头,那时的每一天,我们都畅享纯粹的大快乐,只是那时的我们并不懂得珍惜,那种纯粹的大快乐稍纵即逝,一旦失去就很难重新获得。自从沉迷手淫恶习后,我们变了,我们的心灵变得龌龊而肮脏,我们每天都在发送邪淫的念头,我们已经被污染了,变得污浊不堪,就像从一泓清泉变成了一条臭河,不仅令人生厌,也让自己感觉恶心,这还是以前那个纯净的自己吗?当心灵的纯度下降,随之而来的就是痛苦和颓废;当心灵的纯度恢复后,又可以重新获得纯粹而美好的大快乐。

如果我们投一块石头到水里,水面就会出现向外扩散的波纹,大石头会制造出大的波纹,小石头会制造出小的波轮,不管是大波纹还是小波纹,都可以在水面同时进行。我们的念波也是如此,每个人都在发送自己的波形信息,如果你发送的是邪淫的波形信息,那么解码出来的现实就是扭曲而痛苦的。科学家说,一切都在振动着,六祖慧能大师说,仁者心动!其实最根本的就是你自己的念头在动,只要念头在动,就会出现全息幻影的物质世界,就像你睡着时,只要念头在动,就会出现梦境一样。最最根本的就是你发出的念头,境由心造,你发出的念头频率决定着你生命的景象。

虚云法师有一次从寺庙回到他的小茅蓬,走到一半时,天就黑了。但是他没有起心动念,他不知道天黑,他看到天是亮的,路是看得清清楚楚的。途中碰到两个出家人,有一个出家人拿着灯笼,走到他面前说:“虚老,天这么黑了,你怎么没有个灯?”老和尚听到这句话,天马上黑掉了。心里没有杂念的人,在他的觉受里,天就是亮的,才起分别执著,就又回到无明里去了,顿时就一片漆黑了。

这个公案说明了一个事实,那就是:一切法从心想生!

圣人云:万法从心生,万法从心灭,皆由尔心,善恶也只由尔心,地狱天堂也只由尔心。

\subsubsection{转换你的频率}

\begin{quote}
    “思想或者心态就是磁铁,同类相吸就是法则。结果必然是,心态会吸引与其本质相呼应的状态。”\hfill ——查尔斯哈尼尔
\end{quote}

一连串的念头构成某种思想,思想是具有磁性的,也有着某种频率。当你思考时,那些思想就会发送到宇宙中,它们会像磁铁般吸引所有相同频率的同类事物,所谓人以类聚,物以群分。我们的念头会吸引相同频率的内容,就像我们所看电视的每个频道都有一个频率,当转到那个频率,我们就会看到电视的画面,我们通过挑选频道来选择频率,然后就能看到该频道的画面。如果想看不同的电视画面,我们就得切换频道,调到新的频率上。收音机也是如此,想听哪个台,直接调过去就可以听到,不同的频率对应不同的内容。

大家想一想,如果一个人一直在发送邪淫的念头,结果会是什么?邪念会吸引与其频率相同的内容,而那些内容的性质往往是充满惶恐和痛苦的。一切的始作俑者,就是你自己发出的念头!人体就是一个发射台,同时也是一个接收器,不同的念头对应不同的频率,你发出的频率创造了你自己的生命世界。你发出的频率是超越城市、国家和这个世界的,它会在整个宇宙中回荡,而你就是在用你的念头来发送那个频率!你就是发出频率的源头。

你的思想产生了频率,而思想又会吸引该频率上的同类事物,然后反作用力回到你的身上,从而变成你自己的生命内容。如果你想改变人生的境遇,那么只要改变你自己的思想,从而转换频道和频率,这样你的人生境遇就会大大改变,所谓境随心转。

\subsubsection{善恶之念皆有频率}

地藏经云:“南阎浮提众生起心动念,无不是业,无不是罪!”《太上感应篇》云:“见他色美,起心私之。”你只要动念了,就算犯邪淫了,动念之间,高下立判!

真正的戒是心戒!也就是要在起心动念上修,戒色要懂得控制自己的念头,也就是修心!很多戒友虽然在坚持戒色,但是他无法克服自己的意淫,意淫往往是破戒的前奏,如果无法做到念起即断,那么很可能就会出现破戒的情况。其实沉迷意淫就算破戒,而且意淫是暗漏,心动则精自走,沉迷意淫也会导致症状爆发。如果能严格做到念起即断,那么对身体就没有什么伤害,关键就是要把握断念的时机,必须要快,尽量做到在 1 秒内断掉意淫,真正的断念高手基本都是念起即断,也就在 0.1 秒左右。很多戒色新人无法做到 1 秒内断除意淫,他们往往是几分钟后才发现自己正在意淫,这时候已经太晚了,邪念往往已经起势了,就像小火星演变成了森林大火,到时就很难控制了。戒色新人对内心缺少观照和觉察,所以他们才会被心魔一次次打败,戒色后我们一定要学会观心和断念,必须牢牢看住自己的念头,念头起了要知道,到时候凛然一觉,念头就灭掉了。所谓念起即觉,觉之即无!千万不可跟着念头跑,切记!

达照法师:“只要你的心起一个念头,你这个念头在十方法界,就像一张网一样,把整个你的法界网罗在里面。起善的念头,你的法界呈现的就是善的法界;起恶的念头,你的法界就呈现的是恶的法界。我们凡夫愚痴,不知道自己当下这颗心起的念头对你的法界有这么巨大的影响,但是十方诸佛菩萨,历代祖师大德,诸位同参道友,用心精进、内心微细的人,立马可以感受得到。就像无线电波,马上能接收到、拦截到你这个信号。心只要动,就会造业,就会有结果。这个结果,就改变了我们将来的人生,将来的方向。每一个念头只要生起来,它的信号传遍的是整个法界。就算你一个人在房间里面生气,很生气,把杯子拿起来砸到地上,你觉得在房间里面没人看见,实际上十方诸佛菩萨以及你自己的生命真相,就展现出一个生气的完整的法界,一点都不会缺少。每一个念头都具备十方法界的这种普遍性。换句话说,我们心念的信号,随时都覆盖着整个法界。这覆盖率啊,是非常高的。”

宇宙全息论认为,宇宙是一个各部分之间全息关联的统一整体。在宇宙整体中,各子系与系统、系统与宇宙之间全息对应。在潜态信息上,子系包含着系统的全部信息,系统包含着宇宙的全部信息。在显态信息上,子系是系统的缩影,系统是宇宙的缩影。

宇宙全息论的基本原理是:从潜显信息总和上看,任一部分都包含着整体的全部信息。全息理论在中医方面运用得比较早,几千年前就知道这种知识了,很多人应该都看过中医的全息图片,比如手掌、耳朵、面部的不同部位,都对应着身体的五脏六腑,任何一个微小的部分都包含整体的全部信息。宇宙从根本来说,就是信息!念头就是信息!善恶之念皆有频率,善念对应的是高频率,恶念对应的是低频率,你每天都在发送念头(即振动的波形信息)。

大卫霍金斯博士通过 20 多年的研究表明,人的身体会随着精神状况而出现强弱的起伏。他把人的意识映射到 1 - 1000 的范围内,任何导致人的振动频率低于 200 的状态就会削弱身体,而从 200 到 1000 的频率则会使身体增强。物理学家已经证明,我们这个世界上所有的固体都是由旋转的粒子组成的,这些粒子有着不同的振动频率,粒子的振动使我们的世界表现成目前这个样子。我们的人身也是如此,科学家已经测量过人在不同精神状态下身体的振动频率,结果让人大开眼界。善良、诚实、同情、慈爱,这些正面的品质可以提升身体中粒子的振动频率,进而可以改善身心的健康。霍金斯博士遇到过的最高最快频率是 700,出现在他研究特蕾莎修女获 1997 年诺贝尔和平奖的时候。当特蕾莎修女走进屋子里的一瞬间,在场所有人的心中都充满了幸福,她的出现使人们几乎想不起任何杂念和怨恨。1000 被称为彻底开悟的状态,这是绝对力量的频率!

大卫霍金斯博士说:“很多人生病因为没有爱,只有痛苦和沮丧,振动频率低于 200 易得病。”博士每天接触到大量的病人,后来他发现,凡是生病的人都有很多负面的念头。人的振动频率如果是在 200 以上就不容易生病,通常这些人的振动频率低于 200,低于 200 发出的是哪些念头呢?比如意淫的念头、指责的念头、生气的念头、杀生害命的念头等,在发出这些负面念头的过程中就会消减自己很多的能量,所以当振动频率低于 200 时,这些人就很容易得上各种疾病。

\subsubsection{邪念会拉低你的频率}

邪念会导致最低的频率,邪念的性质就是扭曲的、破坏的,当你想着下流的邪念,当你沉迷于意淫,其实你就是在削弱自己。不仅是邪淫的念头,还包括其他的恶念、贪念、损人利己的念头,还有愤怒和傲慢的念头等,这些都会拉低你的频率。而温和、乐观、宽容、感恩、慈爱、恭敬、安详、平静和喜悦等,这些都会拉高你的频率。振动频率越高,就越能感觉到纯粹的大快乐,振动频率越低,就会被痛苦所包围。善念的能量频率远高于恶念,善举多了,我们自身的能量频率也会相应地提高。

一个人的振动频率比较高的话,他本身就已经具备了各种正面的特质,他的思想纯正,品行高尚,行为无私,这些都处处表现在一个人的一举一动,他会喜欢接触与他频率相当的内容,选择与他频率类似的人相处。我们每发出一个念头,就产生一种频率,思想高尚则频率高,思想低俗频率就低,很多邪淫的人,他们的振动频率相当低,就是因为他们一直在起邪念,然后做了很多邪淫的事情,其实这正是对自己的一种糟蹋和伤害,邪念会拉低一个人的振动频率,频率越低,就会越痛苦,各种身心症状都会随之爆发出来。

天堂就在我们自己的心中,我们只要把自己的频率提升,一切都会随之改变,频率越高就与天堂越接近,频率越低就与地狱越接近!

\subsubsection{造物主的秘密:频率决定物质形态}

\begin{quotation}
    “如果你想知道宇宙的秘密,请思考:能量的频率与共振。”\hfill ——尼古拉特斯拉

    我们这个世界中的一切物质,都是频率的显现

    如果改变频率,物质结构也会随之改变

    思想、感觉、情绪都有特定的频率

    我们时刻持有这些频率,并且与周围的频率共振
\end{quotation}

著名的沙粒形态与振动频率的实验,充分揭示了频率与物质形态的关系。不同的频率对应着不同的形态,念头就是一种振动的波形信息,起一个念头,就会产生一种频率,而这种频率会直接影响到一个人的容貌气质,所谓相由心生。很多手淫的人都变丑了,这在戒友的叙述中多有反映,在刚开始的时候,变丑还不是很明显,然而随着撸龄的延长、伤精程度的不断加深,很多原本很帅气的人就开始变得面目全非了,显得异常猥琐和丑陋,甚至向着怪物畸形的方向发展着。为什么会这样?其实从频率的角度是很好理解的,因为他发出的是邪淫的频率,而邪淫的频率对应的就是猥琐丑陋的相貌,所以沉迷邪淫的人必将变得猥琐而丑陋,这是必然的。

戒色之后,心灵得到净化,频率也会随之提升,这时候容貌气质又会向着好的方向发展。一个人发出的念头是会影响到他的容貌气质的,因为念头产生频率,而频率会影响到物质形态,就是这么微妙。很多人以前知道相由心生,但并不明白背后的物理学原理,看了这段文字阐述,相信很多人会有更深刻的理解和认识。你发出的念波正在影响着你的容貌,这都是潜移默化的,邪淫几年后,容貌气质就下降了,因为你的振动频率下降了,邪念会导致低频率,而低频率对应的就是猥琐和丑陋。

\subsubsection{提高振动频率的方法}

\begin{enumerate}
    \item 行善积德,多帮助别人,这样可以大大提升你的振动频率
    \item 多发忏悔心,改过迁善,正己化人
    \item 多看提升振动频率的书籍,多学习传统文化
    \item 多发恭敬心,多看大德开示
    \item 保持感恩的心,多孝顺父母
    \item 学会不抱怨,学会修忍辱,学会原谅别人
    \item 保持心平气和,内心平和是一种高频振动
    \item 尝试找寻宇宙人生的真相,找到属于自己的信仰
    \item 不要看黄,特别是擦边图和新闻
    \item 保持谦虚,谦己让人可以提高你的振动频率
    \item 学会打坐,积极锻炼,学会养生之道
    \item 和孩子在一起,像孩子一样纯真无邪
    \item 戒色、戒烟、戒酒、戒网瘾、戒各种恶习
    \item 不指责、不嫉妒、多鼓励、多表扬、多赞美
    \item 学习佛法,读佛经、念佛号,这样可以大大提高振动频率
    \item 学会修心,学会控念,这是提高振动频率最根本的方法
\end{enumerate}

后记:

我们都应该回归纯净的心灵,做一个纯净频率的持有者,拒绝邪淫乱世的污染,现代社会邪淫极为泛滥,而且无害论邪说也相当猖獗,能够保持纯净的频率真的相当不易。每个人都有自己的振动频率,就像每个人都有自己的指纹,而振动频率是可以提升的,当你戒掉邪淫,并且尽力行善,这样你的振动频率就会大大提升,振动频率越高,就越能感受到纯粹的大快乐,反之,振动频率越低,就会陷入无限的惶恐与痛苦之中。

\subsection{腰痛问题、虚胖和瘦弱问题、保持警惕的重要性}

前言:

最近提问白发问题的戒友有不少,白发要恢复大概需要 1 年左右,经常看我经验帖的戒友,应该会看到一个案例,有个戒友是戒色 200 多天白发全部变黑,而我当初是用了 1 年多时间,现在我的白发极少,大概 1 - 2 根,完全属于正常范围。白发要恢复唯有坚持,并且要:一,严格控制遗精次数;二,尽量杜绝 YY;三,注重养生之道。这三方面的功夫到了,白发自然是能恢复的,就怕很多人戒色了,遗精控制不了,YY 控制不了,养生方面又不懂,而且还经常破戒,这样白发要恢复,难度较大。

有的戒友,特别是症状还不算严重的戒友,会觉得一个戒友有十几种症状匪夷所思,这样的戒友我只能说阅历尚浅体会不深,我到现在看过几千个戒友案例,亲自沟通的戒友上千个,其中大部分的人其实身体的症状都很多,只是挑了几个比较明显的症状来问我。多看看受害者案例其实就明白了,有十几种乃至几十种身体不适的戒友大有人在。有的戒友感觉很细腻很敏感,能把那些不适的感觉都描述出来,而有的戒友后知后觉或者表达能力有限,无法把全部的不适都描述出来。那些伤得不重的戒友,或者后知后觉的戒友,就会觉得不可思议,这其实很好理解,当你伤到那种程度,自然会明白我说的都是真的,没体会过的人无法理解那种感受。你没吃过盐,就永远不知道盐的具体味道,听别人描述再多也显得苍白。

另外我再强调点,我并不是一开始戒色就成功,我刚开始戒色在初中时代,我那时什么都不懂,又没学习戒色文章的意识,又没人指点我,我那时就是瞎戒和强戒,根本不懂如何专业戒色,结果就是屡戒屡败,失败过无数次,在怪圈中挣扎了很多年,搞得一身的症状,如果我刚开始就能戒掉,就不会有以后那么多症状了。可惜我在十几年后才觉悟到学习戒色文章的重要性,还好不算太晚,现在身体经过几年的调理也已经恢复正常,当然和十几岁时无法比,因为现在我都 30 岁了。

关于动机我再阐述下,我写经验帖,一不为名,二不为利。完全是公益性质的,不求任何回报,我也不需要任何人崇拜我,有的戒友会叫我大师,我从来没放在心上,我和各位戒友都是平等的,我就像一个指路人,指路人和问路人是平等的关系,而不是居高临下的关系,我就是这样的一个平凡的指路人,你要问我如何戒色,我可以给你指一条明路。我希望帮到大家,仅此而已。

下面步入正题。

这季就腰痛问题、虚胖和瘦弱问题、保持警惕的重要性详细论述一下,具体如下:

\begin{quote}\it
    中医:腰为肾之府。
\end{quote}

腰痛问题在戒友当中非常普遍,相信大部分的戒友都有腰酸腰痛的经历,SY 后身体的一个表现就是:腰膝酸软,严重点的就会出现疼痛症状。当然出现腰痛的原因很多,一般 SY 导致的腰酸腰痛,戒掉一段时间并且注意休养,腰酸腰痛的情况是会慢慢缓解的,如果你的腰痛问题持久困扰你,我建议你最好去医院做个检查,看看是否有器质性的病变,比如腰椎盘问题和肾结石等。生活中有腰椎盘突出的人还是很多的,比如在搬运重物时不注意正确的姿势,就可能落下病根。

SY 后的腰痛还有一个表现,就是在弯腰久了以后,直不起来,感觉异常酸痛,我以前 SY 后洗衣服,每次洗完腰都非常酸痛,要休息一段时间才能缓过来,所以 SY 后会让你的腰变得很脆弱,本来弯腰不会出现酸痛问题,但由于你有 SY 恶习,弯腰久了就会容易出现酸痛的表现。这种弯腰久了,腰部出现酸痛的现象,有很多戒友都有反映过,我自己也有切实的体会,另外,SY 后人就软掉了,这时候去运动就要格外小心,因为在软掉的身体状态下从事剧烈运动,是非常容易受伤的,这个我也深有体会,以前频繁 SY 后去打篮球,骨折过 3 次,崴脚无数次。而且那种受伤都很莫名其妙,是自己把自己给扭伤了。

对于腰痛问题,其实很好解决,如果没器质性病变,一般戒色后注意休养,腰痛问题一般会不药而愈的。如果腰酸腰痛问题持续,建议去做个检查,没器质性病变就配合中药调理一下,如果有器质性病变,那就积极治疗,并且坚持戒色和养生,这样恢复才比较快。男人的腰实在是太重要了,一定要好好保护好自己的腰。

再来谈下虚胖和瘦弱问题。

SY 后一般会出现失调情况,表现在体重上,不是虚胖就是瘦弱。当然有的人虽然 SY,但有良好的运动习惯和饮食习惯,倒不一定会出现这个问题。我以前虽然频繁 SY,但我热爱运动,在学校的时候作息饮食相对比较规律,就没出现过体重失调的状况。

在中医上虚胖一般有 2 种原因:脾虚湿阻和脾肾阳虚。

一个人如果热爱运动,他的胖也不会是虚胖,虽然胖,但感觉也很强壮,并不是那种肉很松垮的虚胖。为什么热爱运动的人一般不会出现虚胖呢?因为中医:脾主四肢。《素问》云:“脾主身之肌肉。”即脾气健运,则肌肉丰盈而有活力。如脾有病,则肌肉萎缩不用。一个人热爱运动,可以起到健脾的作用,这样出现虚胖的可能性就比较小,脾气健运,肌肉就会富有弹性有活力。很多人会认为虚胖是脂肪的问题,其实不然,虚胖其实是脂肪和肌肉都软塌塌的,感觉没有弹性,很松垮的感觉。

要如何治疗虚胖呢?吃中药当然是一种方法,可以调理你的内脏。但最关键的还是建立良好的运动习惯和饮食习惯,这点太重要了!你可以不吃中药,但只要做好后面 2 点,依然可以把体重减下来,甚至可以变成有型的肌肉男。我以前曾做过健身教练,在减脂方面积累了丰富的经验,自己也有减脂的亲身经历。如果你有点健身的常识,肯定会知道一句话:三分练,七分吃。要增加肌肉,光练不行,关键是吃练都要到位,这样增肌才比较快。其实这句话对于减脂也是一样的,很多会员过来一顿狂跑狂练,但是回家后无法管住自己的嘴,吃得太多,把在健身房消耗的热量又给全部吃回来了,而且脾主四肢,一般运动完,胃口都会变好,特能吃,结果是怎么练怎么跑,就是瘦不下去,其实如果能严格控制饮食摄入,不需要练得太苦,体重一样会下来。

对于虚胖戒友,我的建议就是:

\begin{enumerate}
    \item 积极锻炼,以有氧运动为主,可以再做些力量训练。
    \item 严格控制饮食摄入,高热量的食物,夜宵零食等都要忌口。
\end{enumerate}

这 2 点做到位了,你的体重肯定会下来,如果经过一个阶段的努力,体重没下来,你就要反省这 2 点哪点没做好,没做到位。

另外,做些力量训练可以让你的肌肉更有型,很多戒友会迷恋力量训练,会想让自己变得更强壮,更 MAN,这种想法相当普遍,强壮的身体的确能给人一种安全感,而且自己也更自信,一般强壮的男人性能力也更强。但请记住一句话:福兮祸所伏!中医讲究欲不可强。很多人练强壮了,反而更会乱来,结果是年轻时放纵,40 岁以后就要买单了,到时候就苦大了。强壮并不等于健康和长寿,很多强壮的人都有很多毛病,甚至活不过 50 岁,而很多注重养生的人,力量指标很低,但依然可以活到 80 岁以上。

对于虚胖的戒友,我还有一点建议就是,你想减脂,必须让自己懂得更多,这和戒色一样,都讲究专业。你有了专业知识,就更容易入门,更容易成功。学习在任何一个行业,任何一件事上,都显得异常重要,不学不知道,学了就懂了,再加上自己的不断实践,不断领悟,要成功就不难了,所谓会者不难,难者不会,必须多学习。等你掌握了足够多的减脂专业知识,很多事情就会变得简单,如果你懂得不够深刻,在很多选择上就会犹豫或者茫然。

还有一类戒友会出现瘦弱的情况。

瘦弱和体质有一定关系,当然和 SY 导致的内脏功能紊乱也有密切联系,瘦弱的人胃口不好,吃不下去,也不消化,怎么吃也长不胖,这是一类;还有一类,就是很能吃,但就是吃不胖,这种人很可能消化吸收不好,或者是基础代谢太厉害。还有的戒友慢性腹泻,这种情况就更容易掉体重了。

瘦弱的人建议看中医调理下,然后积极锻炼,注重养生之道,慢慢是能调理过来的。

最后谈下保持警惕性的重要性。

不少戒友戒了上百天,结果还是破了,原因是什么?很多戒友在成功戒除一段时间后,容易出现放松警惕的现象,自以为成功了,没问题了,于是放松了,一,是放松对戒色文章的学习。二,就是放松了警惕。人一旦放松了警惕,就容易破戒,戒色就像走钢丝,是要每天保持警惕的,因为一不小心,你就下去了,一不小心,你就破戒了,好像鬼使神差一般。

我戒到现在依然每天保持警惕,就像麻雀的警觉,人走近麻雀,麻雀就飞掉了,麻雀是不会让人靠近它们的。戒色也是如此,戒到一定时间进入稳定期后,并不意味着不会破戒了,你只要放松警惕,依然会破戒。所以我们要让自己保持在警惕的状态,不要离开戒色文章,应该多看受害者案例。这种习惯最后就像每天刷牙一样,不一定需要多长时间,但你每天看看戒色内容,绝对有利于你保持住警惕的状态。这对于彻底戒色是必须需要的意识。

当然有的人觉悟极高,他既不看戒色文章,又不看受害者案例,他每天把自己过去的痛苦回忆一遍,这样也可以让他保持警惕。还有一种人也可以保持警惕,那就是有宗教信仰的人,戒色属于戒律的一种,如果他严守戒律,也是不会去破戒的,因为他知道,破戒比死难受,他有这个觉悟。

结语:

有戒友会问到,为何我周围的人都有 SY,为何他们身体还好,我怎么就一身的症状。这个问题有不少戒友问到过,这个也不难理解,人的体质因人而异,有的人先天体质很好,底子厚,又热爱运动,作息饮食规律,没有其他不良嗜好,这种人可以伤个几年,恶果出来相对较晚。而有的人天生体弱多病,体质不佳,又不爱动,喜欢久坐有网瘾或者烟瘾,经常熬夜,再加上频繁 SY,这类人出现症状就比较早,也比较严重。

还有一点,很多人虽然表面上看着还好,其实他也有很多症状,只是表面上你看不出来而已,有一个词叫“暗疾”。生活中有暗疾的人相当多,这都属于隐私,他不说,你还以为他很健康,其实不然。就拿我来说吧,我那时有前列腺炎,但别人并不知道,我也没跟朋友说过,而我在运动场上也生龙活虎,别人都以为我身体很好,其实不然,只有我自己知道我因为染上 SY 恶习,身体出现了很多不适症状,只是这些症状不是很严重,还能忍着,戒一段时间,这些症状都会有所缓解,所以那时并未引起足够的重视。而且,我得精索十几年,自己都不知道,因为我是轻微精索,并无特别的不适感,但已经得上了,去检查 B 超才知道得上了。关于精索问题,我前面写过一篇文章,不少戒友看过那篇文章就去检查了,结果不少戒友都检查出了精索,只是以前自己并不知道。

其实,因为 SY 恶习,多多少少都会出点症状的,即使你体质超好,没有其他不良嗜好,但随着你年龄的上升,很多潜伏的问题就会暴露出来,出来 SY 迟早是要还的。身体早垮掉未必是坏事,因为在你年轻时就垮掉,然后觉悟后彻底戒掉,身体的恢复能力还是不错的,就怕是 40 岁以后身体垮掉,那时身体的恢复能力已经大不如前了,恢复的速度就会很缓慢。

再来谈谈夏季做固肾功出汗的问题,现在炎炎夏日,稍微一动就汗流浃背,有时即使什么都不动,汗水都像下雨一样。我现在依然每天坚持练习固肾功,没有一天间断的,我的做法就是做的时候赤膊,并且准备好一条毛巾,做一组流汗了,马上擦掉,然后再做,这样出汗的困扰就能避免,否则刚洗完澡,一做固肾功,衣服又湿了,很不舒服,也不利于入睡。另外,夏季尽量少吹空调,吹空调吃冷饮都伤阳气,对身体的恢复很不利。切记。

最近有戒友向我反映“天数党”的问题,我个人倒不反对每天签到,但我觉得签到就像是打卡上班,你不能来打卡却不上班,具体对应到戒色上,就是你来签到,签到的目的是什么呢?签到的目的,其实就是每天学习戒色文章和多看受害者案例警醒自己,不能签到了就完了,那样你的觉悟和定力并没有提升,还是停留在强戒的层次。

一般新人容易犯这个错误:光签到不学习,看到别人开帖签到,自己也开个帖签到,却不注重戒色文章的学习,我当年戒色,也喜欢记录天数,那时是记录在笔记本上,我那时最多熬了 28 天,后来我彻底觉悟后,根本没想过天数,也没有了煎熬的感觉,戒到现在都没破过,因为我彻底觉悟了,并且每天保持警惕意识。我不记得我戒了多少天,但我记得我从几月份开始戒的,我的建议就是:不要太在意天数,而是要注重学习来提高戒色觉悟和戒色意识,这才是戒色的正道。

\subsection{戒色后脱发加重,瘦弱问题补充、脑力下降问题}

前言:

有戒友会觉得,怎么飞翔什么症状都经历过?好像不大可能,从而产生了质疑。其实经常看我经验帖的戒友肯定会知道,我并没有说自己什么症状都得过,比如我在帖子里说过我是睾丸下垂,并没有睾丸萎缩,我也没有虚胖过,我写经验帖本着实事求是的态度,经历过就经历过,没经历过也不会瞎编。虽然有些症状我没经历过,但我做过深入的研究和了解,打个比方,医生并不需要什么症状都得过一样,关键是对症状有深入的研究和认识,这才是最关键的。

一个人即使身体再虚,也不可能什么症状都得过,即使两个人都是肾阳虚,症状的侧重点也会有所不同。我 SY 史有十几年,应该说绝大部分的症状我都亲身经历过,并且我研究 SY 行为有好几年,基本什么症状都见过,也搜集了非常多的案例,聊过上千的戒友,对这方面了解比较深入全面,也一直在研习中医理论,这才有了我写经验帖的基础。

因为我写的帖子非常多,很多问题无法及时回答,也请大家多加谅解。

戒色吧里其实已经有了一批具备答疑能力的戒友在崛起,这部分戒友一直在坚持学习,觉悟有了不少提升,所以大家有问题也可以找他们回答。哪里都需要人才,戒色吧也是如此,需要回答问题的人才,也需要宣传的人才,也需要管理的人才,大家各司其职,把戒色吧建设好,帮助更多的人。关于宣传被骂的问题,也时有反映,我以前也做过宣传的工作,不过不是网页宣传,而是 QQ 群里的宣传,是在我十几个焦虑症群做过戒色养生的宣传。

宣传工作,一,讲究度人要随缘,不要勉强;二,就是讲究宣传策略,选择恰当的契入点;三,一定要控制自己的情绪,因为宣传戒色很容易遇见冥顽不化的人,就像秀才遇见兵,有理说不清,遇见这种人,点到为止即可,不要做过多的争辩,一历耳根,永为道种,他现在不觉悟,也许将来某一天会突然觉悟,后悔当初执迷不悟,没有听你的逆耳忠言。度人其实就是在度曾经的自己,曾经的你也许和现在的他们一样,不见棺材不掉泪,不撞南墙不回头,被纵欲主义思想控制着,对于戒色有误解,并且有排斥和抵触情绪。现在你觉悟了,看清了真相,你去度他们,有这种想法非常好,但一定要注意策略和方法,在宣传方面吧主吧零厚有很多经验,大家可以和他多沟通,加入宣传团队,多借鉴成功的宣传模式。

宣传戒色的意义非常巨大,有时候,你的一句话可能就能改变一个人一生的命运,就因为那一句话,点化了他,点醒了他。我将来肯定会回到宣传戒色这条路上来的,而我现在的主要使命就是多写经验帖。

下面进入正文。

这季就戒色后脱发加重,瘦弱问题补充、脑力下降问题这三个方面详细论述一下,具体如下:

最近反映戒色后脱发加重的戒友比较多,戒色后脱发非常容易起退心,就是不想戒了,心想戒色前没脱发,戒色后反而脱发严重了,这是怎么回事?我在前面的文章里也说到过,这就是疑惑破戒,如果有问题想不明白,是极易破戒的,而戒色后脱发加重,是非常容易让人产生恐慌、焦虑、绝望的情绪,这些不良情绪又极易导致情绪破戒,想通过 SY 来宣泄不良情绪,结果就是又陷入了恶性循环。

其实戒色后脱发加重的戒友,之所以不想戒了,他们怎么想的我很清楚,他们想再 SY,看看脱发能不能止住,这种想法有点荒唐,但却是真实的。

一般戒色后都容易出点症状,很多人戒色前没什么明显的症状,但一开始戒色,症状就出来了,很多戒色新人一看到症状出来,就慌了,以为是禁欲有害,马上又掉进了 SY 陷阱。其实刚开始戒色出点症状属于戒断反应,是非常常见的表现,坚持戒色养生,身体症状会慢慢缓解乃至消失。戒色后脱发加重也可以看做戒断反应,当然有人说是排病反应,也有一定道理,因为戒色后正气养起来后,潜在的病邪自然就会显现出来。

另外,我发现夏季反映脱发加重的戒友非常多,为何是夏季?我在炎热的夏季来临之前,每天脱发量在 5 根以内,而现在每天至少几十根,有时甚至在 30 根以上。其实,这就是季节性脱发,就是正常人,在夏季也会出现脱发加重的现象,我在前面的文章也说过,导致脱发的因素很多,必须都认识到,否则很容易产生恐慌情绪。一有恐慌情绪,就容易起退心。夏季出汗多,头皮也容易发痒,头皮一痒,脱发量就会增多,我最近洗头时用洗发膏洗两遍,头皮还是非常痒,于是我在清水里滴了几滴花露水洗头,基本就很少痒了,头不痒,脱发量就下来不少。

夏季脱发原因如下:

\begin{enumerate}
    \item 夏季脱发是由于强烈的阳光照射头部,使头发中的角蛋白断列,造成毛发的脱落。
    \item 夏季头皮容易出汗、出油、再加是灰尘多,毛发更易受到影响。
    \item 由于经常在空调室内外走动,造成血液循环不良,头发也更脆弱易掉。
    \item 夏季高温燥热会导致人体分泌出现失衡,从而引发脱发。
    \item 吃冷饮过多。假如夏天过多食用冰棍、冰淇淋等冷饮,头发容易脱落。
    \item 营养摄入减少。蛋白质是营养头发所必须的重要物质,但夏天由于天气闷热,人们喜欢清淡的食品,肉类食品摄取相对减少。假如人体的蛋白质供给不足,头发就容易脱落。
    \item 炎症刺激。夏季头皮易发生毛囊炎、黄癣或其他疖、痈等,使毛发基根不稳,也容易造成头发的脱落。
    \item 夏天天气酷热,水分蒸发量较大,如果得不到及时补充会导致皮脂腺分泌减少,从而引起人体毛发干枯和脱落。头发水分流失后,角质蛋白变性脆化,时间一长头发就会变得松脆易断,容易掉落。
    \item 受夏季天气影响,不少人睡眠质量较差,脾气变得急躁,这也是引起脱发的重要原因之一。
\end{enumerate}

夏季是个很特殊的季节,俗话说:夏季无病三分虚。因为夏季出汗量很大,而中医:汗者,精气也。汗出得太多,人就虚掉了。最近也有戒友反映戒色几个月恢复情况都不错,但最近症状却出现了反复,甚至出现了频繁遗精。这其实就是反复期的表现,很多戒友容易混淆戒断反应和症状反复期,我在这里再强调下,戒断反应一般出现在戒色后一个月内,而反复期则是戒色后任何时间段都可能出现的,有人戒色半年后没保养好又出现了症状的反复,有人则是一年多以后因为熬夜劳累或者频繁遗精,身体又出现了反复。而夏季之所以那么容易出现反复,原因就是夏季出汗量极大,无病三分虚,然后再吹空调吃冷饮,或者有熬夜久坐久视,这样就很容易出现症状的反复了。因为在身体虚的情况下,你再每天做伤阳气的事情,结果不出症状才怪,所以我很强调和重视养生,很多人犯的错误就是只戒不养,不懂养生,养生意识太差,一洗完澡就喜欢贪凉,往空调房钻,是舒服,但也埋下了病根,有的戒友冷饮也不忌口,每天都吃,这样对身体的恢复太不利,这都属于养生意识,你知道了,就不会那样做了。因为你知道那样做太伤阳气,阳气伤了,症状就容易出现反复了。

有戒友肯定会问,夏季能不能吹空调呢,我的建议是夏季可以吹空调,但不要洗完澡就吹或者不要把温度调得太低,否则寒气是会通过毛孔钻进身体的,我记得前几天,我到空调间去拿东西,那个空调间开得很低,我本来有点出汗,毛孔都在打开状态,猛然一进去受了寒气,结果那天我就腹泻了。空调是一个伟大的发明,但也是双刃剑,其实是很不利于养生的。大家要吹空调,一定要多加注意,不要一味贪凉而埋下病根。

戒色后出现脱发加重的戒友,一定要好好坚持戒色,不要恐慌,记住一句话就行了:戒色是不会错的。你恐慌只是暂时没想明白怎么回事,并不是戒色会伤到头发。

下面谈下瘦弱问题。

上季谈了虚胖的问题,瘦弱的问题谈得不够详细,这季好好谈一下。

瘦弱的戒友,如果你想变得强壮,我的建议是可以做些适量的力量训练,力量训练可以让你的胃口变好,吃得多,如果你消化吸收没有问题,很快就能强壮起来,当然最好有个人带你入门,这样训练起来才比较专业系统,而不是自己瞎练,如果你只盯着俯卧撑和仰卧起坐,要快速强壮起来有难度。力量训练也讲究专业,你如果进行专业的力量训练,你会很快强壮起来的,进步速度会很快。所以,最好周围有个人可以带带你。

我大学毕业是 186 的身高 62 公斤,后来做了健身教练后,曾经冲到 100 公斤的体重,给人的感觉是又胖又强壮,然后我再脱脂,多做高次数的训练,身体的线条就出来了,现在保持在 80 公斤左右,没有特别去吃高蛋白的食物,现在我就保持正常的清淡饮食。

健身界有句话叫:不怕练,就怕吃。训练只要入了门,其实就不是大问题了,关键就是吃。要强壮起来,热量摄入必须要足够,刚开始都是肌肉和脂肪一起长的,等到长到足够的体重再脱脂,这样线条就出来了。

大家可以看下健身的专业名词:

\begin{table}[ht]
    \centering
    \begin{tabular}{l l l}
        胸  & 坐姿卧推         & 3 组 \\
        背  & 绳索颈前下拉       & 3 组 \\
        肩膀 & 组合器械上的坐姿上推举  & 3 组 \\
        三头 & 绳索站姿下拉       & 3 组 \\
        二头 & 组合器械上的 2 头弯举 & 3 组 \\
        大腿 & 腿举           & 3 组 \\
        股二 & 俯卧腿弯举        & 3 组 \\
        小腿 & 组合器械坐姿提踵     & 3 组 \\
        腹肌 & 仰卧起坐         & 2 组 \\
           & 哑铃提踵         & 2 组
    \end{tabular}
\end{table}

对于从来没接触过健身的人来说,这些专业名词可以说都是很陌生的,所以我才说,要有一个人带带就好了,可以给你示范动作要领,并且告诉你很多健身的专业知识,这样你就慢慢入门了,入门后,进步是非常快的,每个月都有变化,力量也会增长很多。当然不要指望身体强壮了,前列腺炎就好了,事实表明,身体再强壮也会得前列腺炎,关键是戒色养生,一个人强壮了可以减轻前列腺炎的症状,但要治愈前列腺炎只有靠戒色和养生。据我所知,很多健美冠军身体都有病,有的冠军中风了,有的冠军有心脏病,所以,强壮并不代表健康,还是要靠那四个字:戒色养生。没有戒色养生意识,强壮起来反而是坏事,就像习武讲究武德,没有武德的人极易闯祸。

有的戒友瘦弱,但没消化问题,这类戒友只要健身入门后,很快就能强壮起来。有的戒友瘦弱,消化也有很大问题,有慢性腹泻或者消化吸收功能不好,这种情况,我建议先把肠胃调整正常再去做力量训练,那样体重增长起来才比较稳妥,否则你肠胃不好,一,胃口不佳。二,就是吸收不好。这样要增肌增重,难度就比较大了。

对于想强壮的瘦弱戒友,我的建议总结如下:

\begin{enumerate}
    \item 消化吸收没问题的,建议多学习健身专业知识,最好有个人带着你练,再多吃,身体很快即可强壮起来。
    \item 消化吸收有问题的戒友,建议先把消化吸收调整正常,这样再去练力量,体重就能稳步增加。
\end{enumerate}

最后谈下脑力下降的问题。

脑力下降在 SY 后是极其普遍的现象,中医:肾上通于脑。SY 伤肾必伤脑力,主要的表现就是记忆力、注意力、思考力、理解力都会不同程度下降,感觉人变笨了,反应迟钝了,没有耐心,变得浮躁易怒。脑力一下降,干什么都会出问题,对于学生族尤其如此,学生要考试,对于脑力要求比较高,而 SY 削减脑力,对于学生来说无疑是个灾难。

一般脑力要恢复大概需要半年以上的戒色养生,脑力恢复有一个缓慢的过程,要从以下几方面去做好:

\begin{enumerate}
    \item 要严格控制遗精次数,遗精也很伤身体伤脑力
    \item 要坚决杜绝 YY,YY 是暗耗精气
    \item 要学会养生之道,各种伤肾气的方式都要杜绝。
    \item 积极锻炼,以有氧运动为主
    \item 可以尝试打坐或者站桩等养生身功法,包括固肾功
    \item 营养要跟上,补脑的食品可以适量摄入些
\end{enumerate}

这 6 方面做好了,半年以后脑力可以恢复很多,脑力上去了,人就焕然一新了,以前很多看不懂的东西,你再看就能突然看懂了,就像突然开窍了一样,其实就是过去脑力下降看不懂,看不进去,而现在脑力恢复了,一下就看明白了,定力,注意力,记忆力,理解力都会上去。

《楞严经》卷六:“摄心为戒,因戒生定,因定发慧,是则名为三无漏学。”戒色是可以增长智慧的,佛教的理论其实和中医是完全相通的,所谓大道同源。脑力恢复了,其实就可以达到戒色改命,脑力不行,你干什么都不行,总觉得不在状态,而当你脑力好时,很多以前干不好的事情都会变得容易起来,你的命运就会随之改变。比如一次关键的考试,脑力好时一个结果,脑力不行,又是另外一个结果,一次重要的考试就可能改变一个人一生的走向,而面试也是如此,脑力好时对答如流,很有自信,脑力差时,可能对方问的问题你都无法弄明白,回答时驴唇不对马嘴。脑力对于一个人实在太重要了,必须戒掉 SY。

后记:

最近看见名为“汤姆叔叔”的戒友发的经验帖,我惊讶于他觉悟提升的速度,我发的文章他基本都看懂了,而且他也广泛涉猎了其他戒色文章,他觉悟上去了,对别的戒友也是一种激励。汤姆的思想已经有了很大的改变,我说过,戒色第一步就是改造思想,你必须通过学习让自己的觉悟有一个质的飞跃,认识上有一个提升,这样再去戒色,就比较容易成功,否则靠强戒,只会注定失败。

我把我自己的经验帖写出来,就是希望更多的戒友能真正看懂它,看懂了,你的觉悟就会有所提升,面对戒色后出现的各种问题,你就会有一个比较靠谱的答案,就像吃了定心丸,不会再随便动摇,不会再起退心。当然,好的戒色文章和养生文章很多,大家应该广泛地涉猎,广泛地学习,学无止境,不断提高。我的觉悟和认识,来自自己的深入体验和不断学习,我到现在每天都会抽空学习,佛法每天看,养生文章经常看,戒色案例天天研究,不断发现,不断思考,不断认识。

我把我自己悟到的内容写出来和大家分享,希望大家的觉悟能够不断提高,即使你每天进步一点点,一年后的总进步也是很惊人的,就怕你不想学习或者是厌倦学习,那样你想彻底戒色成功就比较难了。戒色,必须改造思想,你到任何戒色网站,看任何戒色文章,都是在向你灌输戒色思想和意识,就是在改造你的思想认识。当你意识到学习的重要性,就会自觉地去搜索好的戒色文章来不断学习,温故而知新,当你学进去了,觉悟就会蹭蹭地上升。你思想改造得越彻底,就越容易戒色成功。

不过,最后我要补充一点,当你思想改造成功了,觉悟已经很高了,依然要每天保持警惕,警惕意识非常重要,就像走钢丝,不能放松警惕。

另外,当你思想改造过以后,你就会发现你和周围人格格不入了,因为你充分认识到了 SY 的真相,真正看清了 SY 的真面目,而大部分人还在麻木不仁地纵欲,在通往自毁的道路上一路狂撸,压根儿没意识到症状就在不远处等着他。现在这个社会是被黄毒和纵欲主义思想污染过的世界,外加无害论的毒害,所以,注定大部分人看不清楚真相,这正好验证了那句话:真理掌握在少数人手中。我们注定是少数知道真相的那部分人,我们要做的就是,在恰当的时机,用恰当的方式去度他们,去度曾经的自己,度人即度己。

\subsection{直指阳痿、破戒时间的奥秘、SY 伤精分类}

前言:

戒色吧现在很多戒友都有了学习意识,不再一味地强戒,开始步入了戒色正轨,戒得越来越专业了,这是非常好的现象,一旦养成每天学习戒色文章的习惯,看进去了,真正学进去了,戒色觉悟和戒色意识就会突飞猛进。这个道理其实和打网游一样,你每天打网游,你的等级很快就能上去,如果你不修炼自己的等级,就不会有进步,始终是菜鸟。在游戏方面,就是被对手虐,在戒色方面,就是被心魔虐,因为心魔等级比你高,你等级太低,见一次失败一次,所以,必须通过不断学习提高觉悟等级,觉悟和意识有了,心魔就很难动你了。

我希望戒色吧有更多的戒友分享自己的经验,新人需要前辈的经验指导,新人最大的特点就是思想误区多,很多问题认识不清,而经验帖可以帮助他们更好地提高觉悟,把他们引导到一条正确的戒色道路上,坚定他们的信心和决心,避免走更多的弯路,从而提高戒色的成功率。

研究 SY 行为好几年,我发现成功的戒色模式只有一种:那就是通过不断学习提高觉悟。觉悟修到了,自然就戒掉了,等你觉悟非常高了,只要保持警惕,就可以彻底戒掉。而修炼觉悟就像爬山一样,很多人爬到一半,就不想爬了,就下去了,很多人虽然爬得慢,但一直坚持爬,一直在坚持学习戒色文章,不断提高觉悟,虽然进步缓慢,但一直在不断提高,这样他最终依然可以达到觉悟的顶峰,到时再强化一下警惕意识,就可以做到彻底戒掉。

我能有现在的觉悟,也是这样一步步上来的,我当初学习戒色文章是这样的:每天摘抄好的戒色句子,当然也包括养生类的内容,因为养生和戒色是相通的。我那时一年多时间摘抄了 6 大本,我对自己的要求是,每天进步一点点,这样一个月后,就是一个大进步。从开始摘抄戒色文章后,我的觉悟就有了飞升,很多以前没搞懂的问题一下全想明白了,头脑越来越清晰。后来我又看了不少中医医案的书籍,觉悟又有了进一步的提高,现在每天我都会抽空看看中医和佛法,还在不断提高自己的思想境界,学无止境,提高也是无止境的。

很多戒友会说自己看不进去,一看到长篇大论就烦,看一行就不想看了。这种戒友其实很多,因为肾精一亏,人的脑力就会下降,记忆力和注意力乃至理解力都会下降,更糟糕的是,纵欲后人的情绪会变得急躁易怒,出现“烦相”,做事学习都没有足够的定力和耐心。这种状态我以前也经历过,那时我看到满屏幕的字就烦,不想看。后来戒色一段时间,脑力有所恢复后,心又能重新静下来了,慢慢就看进去了,看进去后,就是越看越想看,并且把自己认为好的句子摘抄下来,不知不觉我的觉悟就有了飞升。把看过的文章和摘抄的句子时常看看,又能做到温故而知新,这样觉悟又会有提高。觉悟提高到一定程度,你就会发现,你对黄毒有免疫力了,而且能降伏心魔了,这其实就是觉悟提高的结果。

对于每天的学习量,我是这样安排的,根据自己的状态来定,如果这天我精神状态好,兴趣比较浓,就多看点,有时是十几页,有时是一章,有时兴趣浓,可以看大半本书。而有时我状态不是很好,兴趣匮乏,这种情况我安排的学习量,就是每天至少一页,看一页只需要几分钟,即使再没耐心,忍一下也能看完,这样我就能做到每天学习不间断,这样我的觉悟就在不断提高,最终就能迎来觉悟的大飞升。我建议各位戒友,每天尽量看一篇戒色文章,不一定要新的文章,温故知新也很好,不要离开戒色文章,你也可以把自己摘抄的内容反复看,这样你觉悟提高也会很快的。

有戒友会说,我喜欢看图片,不喜欢看长篇的文字,因为看不进去,烦!我的建议就是等你戒色一段时间,脑力有所恢复后,就要让自己学着看文字了,因为我做的图片只是戒色的辅助,最关键的还是文章,要专业系统地提高觉悟,必须看大量的文章,图片只是一个助力,文章是主,图片是次,这点一定要明确。

还是那句话:戒色神力来自学习!不学习不知道,不学习觉悟没提高。觉悟的持续提高,最终会让你战胜 SY,降伏心魔。

下面步入正题。

这季就阳痿问题、破戒时间的奥秘、SY 伤精分类三个方面展开详细的论述,具体如下:

我第 6 季的文章讲到过阳痿早泄的问题,现在 25 季了,可以写得再详细全面些。

我回答了这么多的问题,关于早泄、勃起不坚、阳痿的占了不少,伤精伤到一定程度,就会出现性功能障碍,我最早出现的是早泄,大概在频繁 SY 后两年就出现了,然后就有了勃起不坚,勃起不坚的现象大概持续了十几年,因为那时我热爱运动,作息饮食还算规律,无网瘾不抽烟,所以十几年都还能勃起,就是硬度不行。后来开始熬夜久坐后,阳痿就出现了,我那时阳痿还是很严重的,费了九牛二虎之力,并且要强刺激,才能勉强勃起,而且硬度也不行,很快就软下去了,后劲明显不足。

出现早泄和阳痿,男人就会很恐慌,因为性功能对于男人太重要了,应该说恐慌程度和脱发的恐慌程度不相上下,性功能不行在女人那里就会抬不起头,没自信,还会担心老婆出轨给自己戴绿帽子,所以早泄阳痿的男人,心理压力也非常大,焦虑情绪很严重。

最糟糕的是,很多人没有中医养生常识,一出现早泄阳痿,第一反应就是要吃要补,要壮阳,让自己重新坚挺,重新雄起,千万不能让自己软下去,在这种思想指导下,就是到处吃补药。有些补药刚开始吃效果还行,但补药也是会吃疲掉的,经常吃效果就不明显了,到时候就要换药吃,最后的结果就是吃遍很多名贵的补肾壮阳药,身体还是不行,并且很有可能,身体除了阳痿,还会出现很多其他症状,苦不堪言。其实,这类人因为没中医养生常识,从一开始就误读了身体的信号,古语有云:为人子弟不可不知医。如果你不懂医理,在很多方面的认识和选择上就会步入误区,最终害了自己。一个男人阳痿了,这个身体信号其实就是身体在求救,意思叫你不能再放纵自己了,原理类似于毛孔受寒自动关闭一样,很多人不懂这个道理,阳痿了反而吃补药放纵自己,根本就没有戒色养生意识,他们不懂最好的补药其实就是:不泄为补。因为无知,他们选择了错误的道路,靠补药来更好地纵欲,结果可想而知,将来真的有可能彻底阳痿,甚至会染上糖尿病或者中风。

因为肾气不足,身体选择阳痿来自保,你误读身体信号,反而靠吃药纵欲,必然招致恶果,这就是无知的代价。

戒友一旦出现早泄、勃起不坚、阳痿倾向,就要学会戒色养生了,如果症状严重建议去看中医调理,吃中药对身体恢复有利,但一定要注意修心,否则中药吃下去很有可能会助长欲望,结果就是一边吃中药一边纵欲,这种上补下漏的行为,对身体的恢复极其不利。如果你不能做到彻底戒色,而是边吃边漏,我敢肯定你的治疗效果不理想,没多久你就会发现自己又早泄阳痿了。这样就陷入了一种困境,其实就是他自己思想认识上有误区,如果这种思想误区得不到纠正,结果就是花了很多钱,吃了很多药,早泄阳痿还是依旧。所以要治好一个人的早泄阳痿,必须先让他开悟,给他讲道理,让他明白道理,开悟其实就能治病,不开悟,他这辈子就废了。不少戒友结婚前就把自己玩废了,出现了早泄阳痿,家里又催着结婚,这样搞得他很害怕结婚,早知今日何必当初。

不过大家也不用绝望,早泄,勃起不坚,阳痿都是可以恢复的,我就完全恢复了,因为我认识到位,思想认识上没有误区,养生功夫做得很好。最关键一点,我不会去试,因为一试就又掉进 SY 陷阱里去了。大家肯定会问,那怎么知道恢复了呢?其实是否恢复从晨勃的质量是完全可以看出来的,晨勃坚挺持久,硬度强,这都是身体恢复的表现。如果你去试,就非常有可能一而再地破戒,前功尽弃。很多人都喜欢试,结果就是一试无法自拔,又重新开始纵欲。

根据我的经验,早泄和勃起不坚的恢复时间大概需要半年以上,阳痿则需要一年以上的戒色养生,要恢复是各方面都要做好,最关键的前提就是严格控制遗精次数和杜绝 YY,这 2 点做到后,再在养生上下功夫,这样性功能是可以慢慢恢复的。我是过来人,只要你有信心和恒心,不断学习提高觉悟,做到彻底戒色,提高养生意识,性功能完全可以恢复。就怕你步入思想误区,那就很难恢复了。早泄阳痿都分轻度、中度和重度,如果是轻度的,恢复时间相对会快些,如果是重度的,那恢复时间会更长,恢复难度也更大,但只要坚持戒色养生,各方面做到位,依然是可以恢复的。

另外,有人是紧张性的早泄阳痿,自己 SY 没事,这种情况西医会归咎为心理问题,其实紧张心理的出现和肾气的亏损是密不可分的,因为身心是合一的,因为肾气的亏损,人的情绪心理都会出现相应的变化,有的人原来不紧张的,SY 后变得紧张烦躁易怒,这其实就是肾精亏损的表现,甚至小便都紧张,无人时才可以尿出。这类紧张性的障碍,通过坚持戒色养生也是可以恢复的,肾气养足,身心都会恢复到正常状态。

下面谈下破戒时间的奥秘。

这部分知识很少有人知道,一方面要熟谙中医医理,另一方面要有咨询反馈的案例,有了深入研究后才能体会到这部分知识。

在我咨询的案例中,有好几位戒友都反映过,为何每次破戒身体恢复的速度都不一样,有时破戒后几天就能恢复,有时破戒后一个月甚至几个月还有症状,为什么会这样?要知道这个问题的答案,就要了解人体的阳气水平,人体的阳气在每天的不同时间段都不一样,在每个月的不同时间段又不一样,在每个季节又不一样,如果你在阳气水平低的时候放纵,结果就比较难恢复,这个道理其实很好明白,比如你在手机费很少时打电话,很可能打一个电话你就欠费停机了,但在你还有很多话费时打电话,就没事。

人体的阳气水平也是在不断变化着,忽高忽低,总的变化规律如下:

每天的变化:子时一阳生,午时一阴生。故养生之道在子时不行房事,以免戕伐阳气。因为子时阳气刚刚生发起来,好比一个小火星,此时行房事就会很伤阳气,更容易出症状。阳气从子时生发起来,然后逐渐壮大,至午时盛极而衰,然后午时一阴生,然后阴气逐渐壮大,至 23 点子时盛极而衰,然后子时一阳生,这就是一个循环。而锻炼最好选择在上午,因为上午阳气处于上升阶段,这时候锻炼就是顺应天时,效果会更好,一般晚上入夜适合静养,因为晚上阴气重,这时候锻炼出汗,也很伤阳气。这个道理大家懂了以后,就可以更好地选择锻炼的时机。

古代十二个时辰:

\begin{description}
    \item[子:zi] (晚上11时正至凌晨1时正)
    \item[丑:chou] (凌晨1时正至凌晨3时正)
    \item[寅:yin] (凌晨3时正至早上5时正)
    \item[卯:mao] (早上5时正至早上7时正 )
    \item[辰:chen] (早上7时正至上午9时正)
    \item[巳:si] (上午9时正至上午11时正)
    \item[午:wu] (上午11时正至下午1时正)
    \item[未:wei] (下午1时正至下午3时正 )
    \item[申:shen] (下午3时正至下午5时正)
    \item[酉:you] (下午5 时正至晚上7时正)
    \item[戌:xu] (晚上7时正至晚上9时正)
    \item[亥:hai] (晚上9时正至晚上11时正)
\end{description}

每月的变化:

每个月的月亮都有阴晴圆缺,我们可以根据月亮的变化来判断一个月的阳气变化。月初之时,月牙微露,阳气开始渐渐生发释放,月相也慢慢由缺变圆,也就是上弦月,在上弦月慢慢变为满月的过程中,阳气是生发释放最为旺盛的时候,人体内的生命能量也最活跃,月满一过,重阳必阴,阳气逐渐地转入收藏状态,月相也渐渐由满变缺。到了二十二,二十三即成为下弦月。下弦月以后,月的亮区进一步缩小,直至三十,光亮皆无,只能看见月亮的影子,这个时候就叫晦。整个月象的变化,实际就是阳气变化的一个例证。所以,在阳气微弱或者阴气极重时,尽量不要进行性生活,否则是比较容易出症状的。

每年的变化:

夏至一阴生,冬至一阳生。

春生、夏长、秋收、冬藏。养生这个词,其实只是说了保养之道的四分之一,还有夏养长、秋养收、冬养藏。所以,冬天尽量不要过性生活,冬天是一个养精蓄锐为来年做准备的这么一个季节,如果在冬天放纵,来年拿出来生发的能量就少了,所以今年冬天你放纵了,明年你身体就很可能会出症状,特别不能在冬至日放纵,因为冬至一阳生,阳气正微弱,此时破戒更容易导致身体出症状,也很不利于恢复。冬藏精,藏好了,来年身体健康才有保障。

在古代专门有一门择日择时的学问,现在普及较少,很多年轻人都不知道,只有一些高人能真正了解那门学问,什么时候干什么事很重要,因为背后都有深刻的道理,戒友在不同季节破戒后,身体的恢复进度是不同的,因为每个季节人体的阳气水平都不同,你在阳气微弱时破戒,必然犯了忌讳,身体就较难恢复。如果你在阳气正旺时破戒,那还相对容易恢复些。

关于破戒时间的忌讳,相信很多戒友看过九毒日,九毒日,农历五月初五、初六、初七、十五、十六、十七以及二五、二六、二七,此九天为 “天地交泰九毒日”。五月份有「九毒日」,为纵欲大忌!所以古代有习俗,五月让妇女回娘家住一个月,九毒日更要慎重。九毒日背后肯定有其道理,我们宁可信其有不可信其无,因为据我研究,破戒后的报应有延迟现象,有时并不是马上就会出症状,可能会在几十天以后,甚至上百天才出症状,因为这里面有一个蝴蝶效应,而蝴蝶效应发生的过程也是需要时间的。

我们要懂得忌讳,尽量避免犯忌,这样对于我们身体的健康才比较有利,知之则强,不知则老。如果你懂得避免忌讳,这样就可以避免很多危机。

这方面的详细内容,我建议大家可以参阅《寿康宝鉴》,里面专门有“保身立命戒期及天地人忌”的内容,懂得这方面的知识真的很重要。

最后谈下 SY 伤精分类。

这个题目是一个戒友建议的,因为他症状比较多,也不知道自己伤到什么程度了,所以建议我能把伤精表现大致划分下,他自己也可以做到心中有数。如果你广泛地研究过伤精患者的案例,你一定会发现,很多人都是身兼多种伤精表现,身跨多种伤精分类,症状并不是单一的,以下是我对伤精表现的大致划分:

\begin{description}
    \item[泌尿系统疾病] 前列腺炎、尿道炎、精索等
    \item[呼吸系统疾病] 肺部疾病,也包括鼻炎,哮喘等
    \item[脑力严重下降] 注意力、记忆力、理解力不同程度下降
    \item[结石类] 肾结石、尿路结石等
    \item[皮肤症状] 各类皮肤疾病。包括痤疮,荨麻疹等
    \item[头发问题] 以脱发和白发为主
    \item[消化系统疾病] 肠炎,便秘腹泻,消化吸收弱等
    \item[频繁遗精问题] 频繁遗精也是一种病
    \item[变丑问题] 变丑的太多
    \item[性功能障碍] 早泄阳痿,勃起不坚
    \item[心脏问题] 心悸、早搏等
    \item[腰痛腿软] 腰痛腰酸,腰膝酸软
    \item[颈椎问题] 颈椎病的各种表现
    \item[耳鸣问题] 肾虚导致的耳鸣
    \item[多汗多油] 这两多比较常见
    \item[视力问题] 视力下降,飞蚊症等
    \item[其他疾病] 大概有上百种,肾虚百病丛生
\end{description}

很多症状即使不是 SY 直接导致的,也和 SY 密不可分,因为肾精是人体的健康货币。

1 - 17 类不涉及心理方面的疾病,虽然对戒友会产生困扰,但还不至于那么生不如死,当然这 17 类比较笼统,只能算大致分类。接下去主要是心理类的问题,如下:

\begin{enumerate}
    \item 焦虑症
    \item 抑郁症
    \item 神衰
    \item 植物神经紊乱
    \item 强迫症
    \item 偏执
    \item 自闭症
    \item 恐惧症(包括恐艾、社恐等)
    \item 疑病症
\end{enumerate}

当然,我这里讲的心理类问题,并不是纯心理的,是有一定躯体症状的,比如焦虑情绪并不等于焦虑症,焦虑症是有一系列躯体症状的。一旦染上这些问题,想死的非常多,去焦虑症群或者抑郁症群,经常有要死要活的,这些疾病的症状真是千奇百怪,无所不有,如果你不是这类患者,你和这类患者交谈,你也许会认为他说的是天方夜谭,但的确是这样,只有体验过才能理解患者的感受。我那时焦虑症神衰,天天感到绝望崩溃,多次有自杀的想法,因为那种状态简直就是生不如死,你会觉得死亡也许还好受些,就是那种感受,普通人是无法理解的。有的患者不敢出门不敢坐车,普通人就会觉得怎么那么胆小啊,不会吧。其实伤到一定程度,人的胆子真的会变得胆小如鼠,因为肾主恐,恐伤肾,肾虚到一定程度,人就会变得很胆小,和以前判若两人。我以前生病时就不敢出门,现在我恢复后,觉得不可思议,我那时怎么那么胆小呢?现在全明白了,就是肾精严重亏损,纵欲熬夜伤了神经。

所以,如果你是 1 - 17 类,那还算好,你不会那么想死,如果你还有严重的心理类疾病,问题就比较严重了,要恢复就更慢了,难度也更大,很多焦虑症患者 5 年甚至十几年都出不来,常年靠药维持着。其实焦虑症等心理疾病,是必须开悟才能出来的,靠药不行,药能缓解但无法真正治愈,必须要学会养生之道,必须懂得珍惜肾精,否则即使你暂时靠药控制住了,依然会很容易复发,因为你没认识到发病的真正原因。

结语:

偶尔有戒友会提问 SY 是否真的会导致运气的变差,表面看,运气和 SY 没啥必然关系,其实是有一定关系的,因为 SY 会导致脑力下降,脑力正常时,很多事情可以做得很好,而脑力下降时,很多本来可以做好的事情就做不好了,表面上看,好像自己运气变差了。比如一次考试,有的人会说,我当时怎么那么健忘啊,我当时怎么没想到啊,我运气也太差了,其实这种运气差的根源就在于脑力的下降,因为脑力下降,所以你不在状态,这时候人的运气自然就变差了。

再比如面试时,给面试官的第一印象是非常重要的,而一个沉迷 SY 的人眼睛无神,气色像鬼一样,甚至有黑眼圈和眼袋,一副衰样,没自信很自卑,这样的人去面试,成功率自然不会高,不少人面试了十几家,一次都没成功,就怪自己运气差,其实就是 SY 导致的变丑,精气神不行,如果你精神饱满,容光焕发,自信满满,这样给面试官的印象就会完全不同,你这种饱满的精神状态是可以感染到面试官的,他会觉得你很有朝气,而公司正是需要这样有朝气有活力的年轻人。

\subsection{戒色厌倦期、多汗问题、视力问题}

戒色厌倦期是一个不得不面对的问题,也是一个必须克服的问题,很多戒友都曾向我反映过这个问题。

如果大家常驻戒色吧,肯定会发现一个现象,那就是来来去去的人很多,很多人刚开始戒色决心极大,戒色热情高涨,无奈“一鼓作气,再而衰,三而竭”,没几周就消失了,又回到了过去,特别是破戒后的戒友,一旦破戒,戒色决心和热情都会大幅下滑,戒着戒着,就不想戒了,看不到希望,只有绝望和无奈。其实这就是戒色厌倦期,不管你是什么人,都会有厌倦情绪,这种厌倦情绪在其他领域也很普遍,比如你听一首歌,刚开始感觉非常好听,听几周你可能就厌倦了,再也找不到当初的那种感觉,剩下的就是厌倦和不耐烦。食欲方面也是一样的,刚开始吃某样美食很喜欢,叫你天天吃,可以吃得你想吐。手机也一样,刚开始特别想拥有一款手机,用久了就没那种感觉了,就会觉得这款手机也不过如此,再也没有当时那种激动的感觉。

应该这样说,人是一种容易产生厌倦的动物。

在戒色方面也不例外,很多戒友厌倦了学习,厌倦了戒色文章,懒得看,懒得学,不耐烦。一旦出现戒色厌倦情绪,是极容易导致失败的,这就像一支部队的士气不振,很多人不想打仗,想做逃兵,这样的部队根本没有战斗力,如果你出现厌倦情绪了,要学会马上克服,及时调整情绪,让自己的戒色斗志重新饱满起来。一支部队没士气,就是一盘散沙,一支部队有士气,凝聚力就强,个个如猛虎下山。佛教修行也讲究:勇猛精进,不能有懈怠心,要保持恒心。但是人难免有厌倦情绪,因为人是有惰性的,所以一旦出现厌倦情绪,我们要及时调整,就像一只表走慢了,我们要把它调整到正确时间。

下面就谈谈我的调整办法。

说实话,我也出现过厌倦情绪,但我很快就调整过来了,这里面涉及到一个概念,就是情商(EQ),情商就是情绪智商,情商包含了自制、热忱、坚持,以及自我驱动、自我鞭策的能力。情商和智商不同,情商是可以后天习得的,就是可以通过学习来获得的。当你通过学习知道该如何调整自己的情绪,那么你就能完全掌控自己的情绪,而不是被情绪控制。一个人的情绪是可以主宰一个人的行为的,很多戒友破戒都是情绪破戒,一出现无聊情绪和压抑情绪,马上转变成 SY 行为来发泄。如果你情商高,懂得调整情绪,就不会出现这种情况了,所以我们必须学会调整情绪,这在戒色的道路上是异常关键的,出现无聊情绪和压抑情绪,我们要及时调整,出现厌倦情绪时,我们也要及时调整,让自己的戒色斗志重新振作起来,戒好每一天。

不仅学习戒色文章会厌倦,解答问题也会产生厌倦,家兴有向我反映,说他对解答也厌倦了,出现了退隐的想法。这时候我建议他要注意休养,不要把自己搞得太累,在状态不好的时候就减少解答,否则带着心理疲劳去解答,只会感觉更累,这种累其实就是心累,心累是很容易起退心的,说白了,就是不想干了。

学习戒色文章也是如此,刚开始戒色热情高涨,你可以选择大量摄入戒色知识,但学到一定程度,你就有可能出现厌倦学习的情绪。这时候,你可以调整一下,把摄入量减少些,每天看一篇即可,或者每天看几个警示案例也可以。不学习的时候要注意保持高度警惕,不要让心魔钻了空子。等过了这段低谷期,你就能重新找回良好的戒色状态了。


具体的调整办法总结如下:(以跑步作比喻,希望大家能更好地理解)

\begin{enumerate}
    \item 一旦出现厌倦情绪,要学会分配体力,不要把自己搞得太累。就像跑 1500 米,跑不动了,不要不跑,而是跑慢点,继续前进。
    \item 要学会激励自己,激励自己就是在调整情绪,让自己的情绪重新振作起来。好比跑步时,你告诉自己:我能行,我能跑完,我能做到。
    \item 出现负面情绪时,要让自己保持积极乐观,不要沉迷于压抑颓废的情绪。好比跑不动时,你想放弃,有个声音告诉你坚持就是胜利。
\end{enumerate}

我说过,戒色后 YY 是一关,频繁遗精是一关,这是戒色道路上的两只拦路虎,如果你克服了这两关,就胜利了一半,如果你再能克服情绪关,那么你就能戒得更稳定,更长久。

下面再来谈下多汗的问题。

出汗过多的问题,近来提问的戒友比较多,现在虽已立秋,但天气还是很炎热,稍微一运动,还是很容易出汗的。出汗一定要分清是生理性出汗还是病理性出汗,病理性出汗也就是多汗症,原因就是中枢神经系统功能失调的表现,中医认为多汗是阴阳失调引起的。现在多汗是季节因素为主,如果你其他季节也这样容易出汗,那就不正常了,说明你身体需要调整了,最好去看下中医配合中药调理,然后坚持戒色养生,身体就能慢慢调整过来。

其实,伤到一定程度容易出现 2 种情况:

\begin{enumerate}
    \item 多汗
    \item 无汗
\end{enumerate}

一般以多汗比较常见,无汗相对少一些。SY 导致的症状一般都 2 个极端,有人出油多,有人皮肤发干,一般以出油多比较常见。医学上对于出汗是有明确分类的,有自汗、盗汗、头汗、半身汗、手足心汗等,最常见的还是自汗和盗汗。所谓自汗就是无缘无故、不自主地出汗,一般都是在白天并不炎热也没有运动的情况下发生的。盗汗医学上认为就是在夜间睡着了时候出汗,而睡醒了后汗就止了。而 SY 导致的肾虚,不管是肾阳虚还是肾阴虚都有多汗的症状,很多人出汗量极大,像全身洗过一样,而中医有讲到:汗者,精气也。大汗伤阳,如果出汗量太大,对于身体的恢复是极其不利的,所以有多汗症状的戒友最好能及时就医,配合中药调理,这样解决了多汗问题,相信你身体恢复也会上一个台阶的,会比以前恢复得更好,更稳定。

也有戒友说,为何运动出汗后很舒服,没有感觉任何不适。这只能说你阳气还是没伤到那种程度,等你伤到那种程度,再这样出大汗,对于健康是很不利的,所以身体虚的时候一般都建议静养,运动以散步为主。这个道理和遗精也有点类似,很多人身体好,遗精一次并没感觉到任何不适,但有的人以前放纵过度,现在一次遗精就马上会出症状,马上感觉不舒服。就是这个道理,伤到一定的程度,很多事情都和原来想象的不一样了。当然,适量的出汗对身体也是有利的,可以排出体内的邪毒,但是出汗量过大则会伤身体。很多人运动量一大,晚上就遗精了,有的人运动量一大,症状出现了反复,所以,当身体虚时,尽量以静养为主。当阳气慢慢养足了,可以适量做些锻炼,这样对于恢复才比较有利。我建议大家可以根据运动后身体的反应来调整,有的戒友运动后身体无任何不适,这种情况可以继续坚持锻炼,但一定要注意适量,并不是越多越好。而有的戒友运动过后有不适感,或者出现了症状反复,这种情况就说明你现在还不适宜这种运动量,可以先以静养为主,运动量先降下来,做做八段锦,还有散步,等养足肾气,再慢慢把运动量提起来,但也要注意运动适量。

对于 SY 导致的很多症状,我都是建议积极治疗的,并不是一味地要求你戒色养生,而不治疗。该治疗的治疗,再配合戒色养生,这样恢复才比较快。当然,也有不吃药,专靠戒色养生就能恢复的戒友,这类戒友一般在养生上都很有心得。这样养生功夫到位,对于恢复也是很有利的。很多戒友虽然积极治疗,但不注重戒色养生,这样的治疗效果就不会理想,因为要痊愈是三分治,七分养,养生才是重点,古代很多名医看病,对患者都有医嘱,比如:远房帏,节制饮食,按时起居,不要劳累等。很多戒友之所以慢前久治不愈,就是不懂得戒色养生,这样花费上万都看不好,如果你真正意识到了戒色养生的重要性,那结果就会完全不一样。我十几年的慢前,现在就痊愈了,化验正常,也没有症状了。所以,有症状,要治更要养,否则万难真正痊愈。很多人靠药暂时痊愈了,一放纵,马上就又复发了。

最后谈下视力问题。

\begin{quote}
    《太素脉》曰:眼乃一身之精华,不宜不秀。
\end{quote}

中医还讲到:五脏六腑之精气皆上注于目!所以一个人的眼睛就是自己五脏六腑功能的反映。而肾藏精,藏五脏六腑精华之气。所以 SY 伤了肾气以后,人容易出现以下几种眼部变化:

\begin{enumerate}
    \item 眼睛无神
    \item 眼睛无光
    \item 眼球混浊
    \item 眼睛血丝
    \item 眼睛变小
    \item 眼皮变化
    \item 黑眼圈眼袋
    \item 眼球呆滞,转动不灵
    \item 眼窝凹陷
    \item 视力下降
    \item 飞蚊症
    \item 结膜炎
\end{enumerate}

其中视力问题是广大戒友普遍比较关心的问题,特别是对于学生党而言,视力下降的确是一个很大的困扰。视力下降的原因很多,具体如下:

\begin{enumerate}
    \item 用眼习惯不好
    \item 用眼过度
    \item 遗传基因
    \item 沉迷 SY
    \item 眼部疾病等
\end{enumerate}

SY 应该不是视力下降的直接原因,应该是间接原因,很多人虽然不 SY,比如小学生,但他也会视力下降戴眼镜,这种情况就是遗传或者用眼习惯不好导致的。而有的人本来视力极佳,但自从沉迷 SY 后,眼睛视力下降就很厉害了,像这种情况就考虑是 SY 的间接原因,再加上用眼习惯不好,用眼过度,就比较容易出现视力问题了。通过戒色养生,视力状况是有望改善的,眼睛的不适感也会有所缓解。如果视力问题对你来说是个大困扰,那么将来也可以考虑激光矫正。

一般坚持戒色养生,人的眼睛会重新变得有神,有定力,有光彩。光彩是个很微妙的感觉,就像一个电灯泡的亮度一样,有的灯泡亮,有的灯泡暗。而人体阳气足,反映到眼睛上,就是眼睛明亮有光彩。反之,就是无神暗淡。你去观察下小孩,你就会发现小孩不仅肤质极好,另外一个很明显的特点就是眼睛明亮有神,给人一种清澈可爱的感觉。而成人的眼睛就差了很多,一方面身体漏过了,另外就是生活习惯,比如抽烟熬夜久坐等,这样身体里面伤掉了,自然会反映到眼睛上来。我那时沉迷 SY,眼睛无神,缺少定力,一种很没自信的感觉。现在完全不一样了,通过戒色养生,我的眼睛又重新变得有神起来,比起以前自信了许多,也敢和人直视了,有底气了。

我建议有视力困扰的戒友一定要建立起良好的用眼习惯,上网要控制时间,看书学习要保持良好的姿势和距离,注意休息。然后一定要戒掉 SY,学会养生,可以再加强些营养。这样视力下降的问题就会得到有效的控制。

\subsection{直指久坐、轻敌现象、尿泡沫问题}

前言:

最近前任大吧传奇玛雅人发了一个帖子,介绍了戒色心得,也梳理了戒色吧发展的历程,他是最早一批戒色吧的元老,为戒色吧做出了卓越的贡献,值得大家尊敬,他的亲身经历对大家戒色是一个很大的助力。关于 75 的论述,让大家明白了为何无害论会如此猖獗的原因,因为背后有人或者组织在专干撒播无害论的事情,无害论很多都是编造出来的,然后好好伪装一番,冠以某国的科学研究,对于无知的新人很有迷惑性。玛雅人关于 75 的论述,可以让大家更加清楚地认识到无害论的真面目,以及背后的推手。

其实无害论早已有之,我记得在 90 年代末时,我看杂志时,就有文章在宣扬无害论。我相信那时的无害论应该是作者的认识误区所致,不像现在的无害论,论调变得越来越夸张,越来越耸人听闻,好像不 SY 就会死一样。90 年代末的时候,家庭电脑还未普及,基本就接触不到有害论,更别提有人传授戒色的经验,那时的我就像井底之蛙,我记得那时我看到一个广告,是卖补肾药物的,上面有一句:肾虚,百病丛生。就这么一句,就是我那十几年唯一接触到的正面信息,而且只是说肾虚,并没有直指 SY。现在是网络时代,更多的青少年有机会通过网络这个平台接触到有害论,应该说,网络时代的青少年是幸运的,但也是不幸的,因为这个时代是黄毒泛滥的时代,90 年代有黄毒,但不像现在如此泛滥,上网到处是黄毒,门户网站也充斥着暗示性的内容,而且上网下视频太容易了。很多青少年都有网瘾,有网瘾的人久坐电脑前,体质本来就容易下降,再加上 SY 的摧残,身体出症状的可能性就太高了。

要戒色成功,必须认清无害论,这其实就是戒色的第一关,认不清无害论,迟早会产生动摇,很多戒友一看无害论,马上就产生了动摇,戒色决心不再那么坚定。所以,我们要远离无害论,远离邪知邪见,适度无害论的最大漏洞就是:它完全忽视了 SY 行为的高度成瘾性。人人都以为自己能做到适度,但我可以明确地告诉你,普通人根本就做不到适度,成瘾了再谈适度是荒谬的,另外就是没有一个人知道度在哪里,大家都知道开水烧到 \SI{100}{\degreeCelsius} 会沸腾,而 SY 伤到什么程度才会出症状,这个度只有鬼知道。因为每个人的体质是不同的,有的人一周 3 次没出症状,但有的人一周一次就出症状了,而那个一周 3 次没出症状的人,你敢说三年后他不出症状吗?他一周 3 次,可以保证他3年不出症状,因为他体质好,能够伤 3 年,但是伤到第 4 年就会出症状。还有人说,我能控制一周一次,但我要说的是,你能控制 YY 吗?YY 是暗漏,一般 SY 的人,经常会沉迷于 YY,他以为一周一次没事,其实他 YY 漏得更多。就这样,不知不觉间,症状就找上门来了!

我第一次来戒色吧时大概是在 1 年多以前,当时戒色吧的会员人数只有几百人,现在能接近 3 万,其实和大家的宣传是分不开的,戒色吧的繁荣是大家共同努力的结果。我和玛雅人没交流过,和土豆倒是交流过不少次,土豆虽然现在不当大吧了,但他还是常驻贴吧的,土豆能坚持到现在很不容易,这要克服很多阻力才能做到的,是要有大愿力才能做到的。戒色吧现在的整体氛围比之过去要好很多了,已经有一批有觉悟有学识的戒友在崛起,在带动戒色吧更好地发展。这在过去是不可想象的,过去很多问题发出来都是没人回答的,所以现在的新人是幸福的,戒色吧的前辈已经为你们铺好路了,你们一进来就能把你们引导到正确的戒色道路上来,可以避免走很多弯路。作为新人,你们要好好珍惜学习戒色文章的机会,让自己戒得更专业。

我那时来戒色吧,发现一个现象,很多帖子都有关于 SY 危害的论述,也有戒友发自己的经历来警 示大家,但是真正深入研究 SY 行为的很少,给出具体指导的也较少,很多疑问戒友都得不到满意的答案,大家应该知道,一有疑惑就容易起退心,一有疑惑也会导致破戒。那时,我因为得焦虑症和神衰,研究神经症已经很久了,经过和上千个神经症患者的聊天,经过对致病原因的大量比对和深入分析,终于得出了一个公式:熬夜 + 纵欲 + 久坐 = 完蛋。只要是男病友,基本都符合这个公式。我的研究是基于大量案例的,并不是凭空猜测的结果,做任何研究都要注意搜集第一手案例资料,然后经过深入分析才能找到其中的规律,我那时聊过的焦虑症患者上千,各个年龄层都有,最大的有 50 多岁,最小的 16 岁。然后我意识到,纵欲和这个病密切相关,肾为元气之根,脾为生气之源,纵欲伤了肾,久坐既伤肾又伤脾,而熬夜伤精更严重,这三者就像三把斧头砍向生命之树,不出症状才是怪事。接着,我就从研究焦虑症转向研究 SY 行为,因为这两者其实是共通的,现在 SY 戒友也聊了上千,搜集的案例有几千例,基本上什么症状我都见过。

我是过来人,也是痊愈者,曾几何时,我也处在无知幼稚的思想状态,那时的我也以为前列腺炎是无法痊愈的,那时的我也以为脱发了就难以恢复了,但现在我通过坚持戒色养生,这两样我都恢复了。一句话,如果你想痊愈,必须多和痊愈者交流,如果你一直和没痊愈的人交流,就会被灌输无法痊愈的想法。我的主张就是戒色一定要彻底,养生一定要到位,这两点做到了,就有了痊愈的基础,然后再配合积极治疗,很多疾病都是可以痊愈的,如果你这两点做不到,吃再好的药也难以痊愈,即使暂时痊愈了,也极有可能复发。

我注重戒色,更注重养生,一直看我文章的戒友应该会知道,我对养生是极其重视的,因为我发现,戒色并不代表万事大吉,相反,戒色仅仅是开始,很多戒友戒了 8 个月乃至 1 年,身体恢复情况都不理想,还是没有多大改善,其实就是不懂养生,没有养生的意识,在其他方面漏得太多,这个问题我在 12 季讲得比较详细,没看过的戒友可以看看,赶紧把养生知识学起来,这样对于你身体的恢复才会比较有利。再举个案例,有个戒友虽然戒色 2 年,但频遗关始终无法克服,这样戒了2年身体还是老样子,苦不堪言。所以,戒色后的频遗关和 YY 关必须过,否则对恢复实在太不利了。切记。

下面步入正题。

这季就久坐问题、轻敌现象、尿泡沫问题详细论述一下,具体如下:

久坐问题我在第 7 季曾经讲到过,现在觉得有必要再强调下,也可以讲得更深入些,曾经看过的一本养生书籍上写到:生病就在一穿一脱之间。意思就是你觉得热,把衣服脱了,然后就着凉了生病了,而我要说的就是:生病也在一站一坐之间。久坐伤人,是温水煮青蛙的方式,刚开始让你觉得很舒服,不知不觉间就开始伤你了。久坐,就这么一坐,再简单不过的姿势,就把肝肾脾给伤了!久坐必久视,否则你久坐在那干嘛?久坐伤肾,压迫膀胱经,久坐伤脾,脾主肌肉,所以久坐也伤运化,久视则伤肝,因为肝开窍于目。很多人对于久坐不以为然,其实中医早就有讲到久坐会折寿,而且现在西医的研究也证实了这一点,西医也有很多关于久坐危害的文章。很多戒友是戒色了,YY 也控制得不错,但是网瘾还是很强,有久坐久视的不良习惯,结果就是戒色半年乃至一年,恢复还是不理想。而且我发现,久坐会产生一种惯性,按照某些人的说法就是,有一双无形的手把自己强按在椅子上,站不起来,欲罢不能,不知不觉间几个小时就过去了。所以我们如果想要更好地恢复,必须深刻认识到久坐的危害:

\begin{enumerate}
    \item 肥胖,当摄入的热量大于消耗的热量时,体内的脂肪容易堆积,体重便会上升。肥胖是引发多种慢性病的危险因素。
    \item 颈椎病,人保持长时间坐姿,全身重量压在脊椎骨底端,加上肩膀和颈部长时间不活动,容易引起颈椎僵硬,严重者甚至导致脊椎变形而诱发弓背及骨质增生。
    \item 食欲不振、消化不良,久坐缺乏全身运动,会使胃肠蠕动减弱,消化液分泌减少,日久就会出现食欲不振、消化不良以及饱胀等症状。
    \item 肌肉萎缩,久坐可使体内携氧血液量减少,氧分压降低和携二氧化碳血液量增多,二氧化碳分压升高,引起肌肉酸痛、僵硬、萎缩。
    \item 妇科疾病,许多妇女得宫颈炎疾病并不是因为卫生习惯不好,而是与久坐有关。久坐会妨碍免疫细胞的生成,导致抵抗力下降,加上血液循环不通畅,容易引发宫颈炎等妇科疾病。
    \item 记忆力下降,久坐不动,血液循环减缓,则会导致大脑供血不足,伤神损脑,产生精神厌抑,表现为体倦神疲,精神萎靡,哈欠连天。久坐思虑耗血伤阴,会导致记忆力下降,注意力不集中。
    \item 前列腺炎:调查发现,慢性前列腺炎患者中,办公室职员、开车司机、电脑工作者等尤其是汽车司机(长途汽车司机)占较大比例,并且不易治愈。因为从事这些方面工作的人,要长时间久坐不动。
    \item 精索静脉曲张:研究显示久坐可引起精索,并且有可能会加重精索。
    \item 伤心,久坐不动对心脏工作需求量减少了,但是可致心脏功能降低,引起心肌萎缩,给高血压、冠状动脉栓提供了可乘之机。
    \item 伤筋,久坐不动,会使肌肉松弛,血脉不畅,发生瘀积,容易导致静脉曲张,还容易引起痔疮及坐板疮。因为久坐不动,人的全身重量几乎都压在屁股肌一点上,造成其被“压死”,血不畅通而坏死。
    \item 伤肌,久坐不活动,血液不畅,会使肌肉僵硬、酸痛、萎缩、失去力量和弹性而发生痉挛。一次,一个人打牌通宵不动,当他起身时,竟然跌倒在地。原来久坐不动,两腿僵硬不听使唤。
    \item 伤骨,长期坐着工作的人,由于全身重量压在脊椎骨底端,或坐姿不当,加上肩膀和颈部长时间不活动,会引起上端的四节脊椎骨僵硬,还会导致脊椎骨干燥而诱发多种脊椎病,最常见的是躬背及骨质增生。
    \item 伤脑,久坐不动则大脑供血会不足,当人突然站起时,会感到头晕、眼花甚至欲呕吐等。
    \item 伤神,久坐不动还会产生精神压抑,使人无精打采,倦怠无力,哈欠连天,有时还会引起虚心上火,出现牙、咽疼痛,耳鸣及便秘等症状。另外,久坐还会妨碍免疫细胞的生成等。
    \item 伤肾,久坐不动会压迫位于臀部和大腿部的膀胱经,造成膀胱经气血运行不畅,导致膀胱功能失常,而肾经与膀胱经相表里,这样就会引发肾功能异常,所谓“久坐伤肾”就是这个道理。而肾气不足慢慢就会导致气血双虚,出现皮肤瘙痒、面色苍白或黝黑、失眠多梦、心情烦躁、便秘、经血量少等。而这些问题反映在颜面上会表现为可怕的色斑。色斑的出现其实是身体在告诉我们:它的内部气血发生了瘀堵,即中医所说的气滞血瘀。
    \item 久坐使人的全身血管血容量(外周血容量)减少,心脏功能减退,加重中老年人的心脏病,提前发生动脉硬化、冠心病和高血压等病症。
    \item 久坐使胸腔血液不足,导致人的心、肺功能进一步降低,加重中老年人心脏病和肺系统疾病如肺气肿感染,迁延不愈等。
    \item 久坐还会使人的脑供血不足,导致脑供氧和营养物质减少,加重人体乏力、失眠、记忆力减退并增大患老年性痴呆症的可能性。
    \item 久坐不动会引发全身肌肉酸痛、脖子僵硬和头痛头晕,加重人的腰椎疾病和颈椎疾病。
    \item 久坐容易引起肠胃蠕动减慢,消化腺分泌消化液减少,出现食欲不振等症状,加重人的腹胀、便秘、消化不良等消化系统症状。
    \item 久坐可使直肠附近的静脉丛长期充血,淤血程度加重,从而使人的痔疮加重,导致大便出血、肛裂等症。
    \item 由于有些中老年人经常处于静态且不爱说话,会加速语言功能的衰退,并使大脑反应能力变得迟钝,这与“用进废退”理论是一致的。
    \item 久坐还会导致人的心理压抑,爱发无名之火,精神状态欠佳,对外界兴趣逐渐降低直至全无兴趣。
    \item 长期久坐易患肾结石、胆结石等结石类毛病。
\end{enumerate}

这 24 点是我摘抄总结的,基本上久坐的危害都在里面了,有句话叫:久坐能杀人!这句话并不是危言耸听,这种温水煮青蛙的杀人方式,的确是杀人于无形,不知不觉间你的健康状况就被削弱了,而且有的人比较无知,还不知道问题出在哪里。世界卫生组织报告指出,每年有 200 多万人因久坐少动而死亡。久坐能坐出病来,久坐会与死神亲密接触,是这样吗?是的,这千真万确!然而,现在坐着的人越来越多,坐的时间过长的人越来越多。有人进行过观察,办公室人员每天上班坐的时间平均为 5 小时以上,连续坐的时间平均为 2 小时以上,还有电脑族、开车族……

有的戒友会问,我工作学习没办法啊,每天都要久坐,怎么办呢?我给出的答案就是,每 40 分钟起来活动一下,给自己定个电脑闹钟,网上有专门的闹钟软件下载,大家可以下载一个,到时可以提醒你起来活动一下。另外,我可以透露一下我自己的办法,我在家都是站着上网的,就是在桌子上放个椅子,这样我就避免了久坐,然而又出现了一个新问题,那就是久站,中医:久站伤骨,所以我一般站一会就坐下休息会。这样方式我觉得挺适合自己的。大家有兴趣也可以尝试下。

还有的戒友会问,打坐算久坐吗?打坐是一种练功态,的确有高僧大德打坐入定几小时乃至半天,甚至几天的也有。因为是特殊的练功态,所以一般是无害的,我推荐是每天打坐一小时,你也可以根据自己的情况来安排,如果你境界达到了,打坐入定几小时也是没问题的。当然我也看到过打坐时间久了有害的文章,那篇文章讲的是双盘打坐时间久了,到老了容易出现腿脚方面的毛病。写那篇文章的是道教方面的知名人物,阅历深厚,他的意见大家可以参考下。我现在一般以散盘和单盘为主。

久坐这个问题怎么强调都不过分,希望大家能深刻认识到这个问题。戒色后,我是主张积极锻炼的,但运动要注意适量,并不是越多越好。实践也证明:经常做有氧运动的戒友恢复相对比较快,也比较理想。

下面来讲下轻敌现象。

轻敌的帖子在戒色吧里时有出现,比如发出来的论调是:戒色很简单嘛,戒色很容易嘛。一般发这样帖子的戒友,都处在欲望休眠期,看过我第 4 季文章的戒友应该会知道,欲望休眠期过后就是破戒高峰期,很多戒友在欲望休眠期还是能控制心瘾的,欲望也不重,YY 也不多,有的人欲望休眠期只有几天,有的人欲望休眠期有 1 个月左右,在欲望休眠期内,很多人容易滋生一种轻敌的情绪,就是觉得戒色有什么难的,现在我不是戒得很好嘛,发出来的帖子也是这种论调。

如果大家很细心,一定会发现过不了多久,发出来的帖子变成了这样:我太自信了,破戒了。我太大意了,破戒了。从刚开始的轻敌变成了破戒后的懊悔。所谓:骄兵必败。毛主席说过,战略上藐视敌人,战术上重视敌人。我们战略上可以藐视心魔,但战术上一定要重视。如果戒色那么简单,那怎么会有那么多人反复破戒?而且很多人都是研究生,都是高智商的人群。套用一句电影台词:你把戒色想简单了!要戒色成功必须专业系统地学习戒色知识,不断提高戒色觉悟和思想意识,这样才有望彻底戒除。

再来谈下尿泡沫的问题。

尿泡沫的问题,很多戒友都问到过,很多戒友都经历过,尿液中泡沫的形成,主要与尿液液体的表面张力有关。一般来说,液体表面张力越高,越容易形成泡沫。在各种情况下,尿液中的各种成分发生变化,如蛋白、黏液和有机物质增多,就可使尿液的表面张力增加,尿液中容易出现泡沫。而尿液中泡沫多,到底是不是病?则需视不同情况进行具体分析。

尿泡沫现象,我们也要分清是生理性的,还是病理性的。

如果属于下列情况,那么尿液中泡沫增多并不属于病变:

\begin{description}
    \item [尿道中存在精液成分] 男性如果尿道中存在精液成分,可以引起泡沫尿。如逆行射精(常见于糖尿病同时伴有自主神经功能紊乱症患者);经常性兴奋使尿道球腺分泌的黏液增多、遗精后等。
    \item [排尿过急] 排尿过急时尿液强力冲击液面,空气和尿液混合在一起,容易形成泡沫,但较易消散。此外,排尿时站得过高,在重力作用下,尿液对液面的冲击力较大,也容易形成泡沫。
    \item [尿液浓缩] 在饮水过少、出汗过多、腹泻等情况下,人体因水分不足引起尿液浓缩,造成尿液中蛋白及其他成分浓度较高,容易形成尿中泡沫增多。
    \item [其他原因] 便池中的消毒剂或去垢剂也是使尿液形成泡沫的原因之一。
\end{description}

偶尔出现的泡沫尿,多半是生理性的,大多能找到引起泡沫尿的诱因,如排尿过急、尿液浓缩等。如果去除上述诱因后泡沫尿消失,且没有伴随其他异常症状或疾病,则无需太担心。

以下情况则要警惕:

\begin{description}
    \item[蛋白尿] 尿液中蛋白含量异常升高是引起泡沫尿最常见的原因之一,也是各种疾病尤其是肾脏病的重要临床表现。各类原发性肾脏疾病,如各类原发性肾小球肾炎等和各类继发性肾脏损害,如糖尿病、高血压、痛风、肝炎等,均可以导致肾脏损害,尿液中蛋白增加。而多发性骨髓瘤、急性血管内溶血、白血病等,虽然肾功能正常,但因血液中出现大量异常蛋白,尿液中也有蛋白漏出,形成蛋白尿。有蛋白尿时一般都会出现泡沫尿,这种泡沫尿的特征是尿液表面漂浮着一层细小的泡沫,久不消失。
    \item[泌尿系统感染] 泌尿系统感染可以引起尿液中泡沫增多。常见的包括尿路感染、膀胱炎、前列腺炎等,大多同时伴随有尿频、尿急、尿痛等症状。
    \item[尿糖增多] 尿液中的有机物质(葡萄糖)和无机物质(各种矿物盐类),也可以使尿液张力增强而出现泡沫,但这种泡沫一般较大,且很快消失。糖尿病病人因血糖升高,继发尿糖升高,容易产生泡沫尿。
\end{description}

我自己的经历,我也出现过尿泡沫,一般泡沫久久不消失,考虑身体内有炎症的可能性,比如前列腺炎。当然,如果蛋白质摄入过多,也有出现尿泡沫的情况。如果偶尔出现一次,就不用太担心,如果尿泡沫的情况一直持续,或者经常出现,那最好去医院做个检查。

后记:

戒色后如何克服频遗关的确是戒友们比较关心的问题,大家都想尽量减少遗精次数,我当年也和大家一样,在网上搜索了无数的防遗的方法,经过我自己的切身体验,我最后还是选择了八段锦里的固肾功,这个功法有中医的理论支持,而且经过了无数人的验证,可靠性比较高,固肾功不仅对遗精有效,对于滴白问题也同样有效。我在第三季介绍了固肾功,经过很多戒友的反馈,很多人的确做到了减少遗精次数,把遗精频率控制在一月一次,这样的遗精频率对于恢复是比较有利的,当然还有不少戒友做了固肾功,但遗精照旧,一方面考虑是动作没做到位,另外就是其他导致遗精的因素没有注意避免。如果其他导致遗精的因素没有注意避免,那么即使做了固肾功还是有可能会遗精。比如有的戒友做了固肾功,但是白天运动过量导致劳累,这样晚上还是会遗精。还有的戒友 YY 关克服不了,遗精问题还是无法解决。

最近吧里有很多戒友分享了减少遗精的方法,对待这些方法我们可以去尝试,但请记住了,不要把全部希望都寄托在一种功法上,因为如果你不注意避免其他导致遗精的因素,那么依然有可能会遗精的,关于导致遗精的其他因素我专门写过一季,大家可以参照着看看。

\subsection{戒断反应、JJ 偏向问题、精液颜色改变}

戒断反应我以前的文章有讲过,最近问戒断反应的新人很多,我有一种恍惚的感觉,时空错位的感觉,戒色吧已经进入了轮回,不断有新人加入进来,问的问题都是前辈问过的问题,而以前的新人已经变成了老戒友,老人带新人,这种局面已经出现了。关于戒断反应,这季再好好讲讲,以便消除新人的顾虑,坚定他们的戒色信心,我想还是很有必要的。

戒色后,在思想方面,要过的第一关就是无害论!而在身体方面,要过的第一关就是戒断反应。绝大多数戒友在刚开始戒色后的一个月内都会出现戒断反应,没戒色前症状并不明显,戒色后反而出症状了,一出症状很多人就慌了,马上想到了禁欲有害,马上想到了禁欲对前列腺不好,结果又掉进了 SY 陷阱。其实这就是戒断反应,一般坚持戒色,戒断反应会逐渐缓解乃至消失的。如果新人不知道戒断反应,不了解戒断反应,那就很可能会起退心了。因为他慌了,他怕了,的确,被戒断症状缠着的感受不好过,但只要坚持,戒断反应会过去的。出现戒断反应时,要注意休养,不要太劳累,可以做些适量的锻炼。一般过了戒断反应,身体恢复会越来越好的,当然前提是你要学会养生之道,积极锻炼,这样恢复才有保障,不能老坐着,那样恢复进度会很慢的。

我研究过很多成瘾的行为,一般成瘾的行为戒断后,都会出点症状,戒断反应在瘾界比较普遍。

比如戒烟:吸烟者在强制戒烟之后也可能会出现诸如焦躁不安、心急、胸闷、咳嗽、短暂健忘、无精神、发胖、发抖、失眠、食欲增强、吐黑灰色痰、血压升高以及心律不齐等戒断反应,会产生极大的痛苦。但是这种反应大多数会随着体质恢复逐渐消失。

再比如戒酒:戒酒后精神症状表现为焦虑、抑郁、易激惹,重者出现幻觉、错觉、妄想、意识障碍等;神经系统症状表现为心悸、胸闷、大汗淋漓等植物神经症状,重者可有震颤、抽搐、癫痫样发作等;胃肠道症状表现为恶心、呕吐、腹痛腹泻等。

戒断反应属于适应性反跳,一般坚持戒色,戒断症状会逐渐消失的。一般戒色后的戒断反应表现如下:

\begin{description}
    \item[情绪障碍] 情绪不佳,急躁易怒,颓废悲观,兴趣丧失。
    \item[睡眠障碍] 戒断后反映睡眠障碍的戒友很多。以入睡困难,失眠多梦比较多见。
    \item[躯体症状] 以泌尿系统疾病为常见,比如前列腺炎加重。
\end{description}

还有的戒友会出现腰酸腰痛,全身无力,浑身难受,晨勃消失、脱发增多、精液改变等。

出现戒断反应不要慌,不要怕,好好坚持戒色,注意休养,很快就过去了,这是一道坎,大家都是这样过来的。

下面谈下 JJ 偏向问题。

我聊过上千个戒友,其中反映 JJ 偏向的不多,但时不时会冒出来几个,这个问题的确不容小觑,会影响到一个人的自信。大家先看 2 个案例:

\begin{enumerate}
    \item 男性,25岁,未婚,无性生活,阴茎弯曲偏左,弯曲角度很大,好几年了,因为不好意思就一直没到医院治疗。
    \item 飞翔大哥,我 JJ 向左弯 20 度左右,尿线向左偏一点,都是原来 SY 用右手造成的,怎么办?
\end{enumerate}

我遇见的戒友中有 JJ 偏左的,也有 JJ 偏右的,有的甚至偏右 \SI{45}{\degree},国外有一篇戒色文章就说到,SY 会导致 JJ 发生偏向问题,但也要看具体情况而言,因为其他很多人也有 SY 恶习,但他们并未出现这个问题,我想 SY 应该是一个诱因,出现 JJ 偏向应该是很多因素共同作用才会出现这种情况。比如你在 JJ 发育时撸管,撸管方向偏左或者偏右,或者摩擦时习惯往一个方向用力,在 JJ 发育时这样很可能就会影响到 JJ 的生长方向,就像一棵小树,在它还小时把它弄弯,结果会怎样呢?它就会顺着歪劲长,长大了也是歪的,而大树则没这个问题,因为大树的生长方向已经基本定型了。

出现这个问题如何治疗,那要看是不是影响你的自信,还有是否会影响到将来的性生活,如果影响到了,一般医生会建议做矫正手术的。所以有这方面问题的戒友,应该及时就医治疗,听取医生的专业建议。同时,不要再 SY 了,好好养身体,SY 导致的症状太多了。等出了症状,其实有点晚了,中医讲究治未病,我们应该防患于未然!

最后谈下精液颜色。

一般 SY 伤了肾气以后,精液也会随之发生改变,反映精液改变的极多,应该是人人都会遇见的一个问题。一般会发生如下几种情况:

\begin{enumerate}
    \item 颜色的改变,变得很黄或者黄绿色
    \item 精不液化,出现结晶体或者果冻状
    \item 血精,出现血精的原因很多,以炎症和结石较常见
    \item 精液偏稀薄偏少,黏稠度下降
    \item 精液排出过多过浓,也属病态,提示炎症的存在
    \item 精子质量下降(无精、死精、活力低、畸形等)
\end{enumerate}

我看过的文章,提到精液的正常颜色,一般有两种说法:1. 乳白色或者淡黄色;2. 灰白色或者淡黄色。如果严格地说,乳白色也不正常,提示炎症的可能性。从大家的提问来看,出现果冻状和结晶体比较常见,其次就是变黄,如果是淡黄色,是正常的,如果变得很黄,或者出现黄绿色了,那就要引起重视了。还有一些戒友会出现血精的现象,精发红或者带粉红,出现血精的情况建议及时就医检查治疗。

刚开始戒色,精液是容易出现改变的,可以看作戒断反应。随着坚持戒色养生,精液的质量会慢慢好起来的,还有就是不注意养生也容易出现精液的改变,比如酗酒抽烟,久坐熬夜,吹空调等。养成良好的生活习惯是非常重要的,对健康有直接的影响。

中医理论有讲到:“种子之法,男必先养其精”。要想优生优育,男人应该好好保养身体,该戒的都戒掉,把精子质量养好,这样生出来的孩子才会比较健康。很多戒友有去做精子检查,有部分戒友出现了弱精症,出现无精症的也有,还有的就是精子活力不行,死精,畸形的不少。一般导致精子质量不行的主要原因,就是前列腺炎或者精索静脉曲张,而 SY 是可以导致这 2 个疾病的,然后就会影响到你精子的质量。很多人沉迷 SY,把精子质量搞得很差,有的出现了不孕不育,有的虽然还有怀孕的功能,但是继续 SY 下去,弄不好也会出现不孕不育。所以,戒掉 SY 是有百利无一害。不戒掉,有百害无一利。如果真要说 SY 有什么好处的话,那就是可以发泄你的情绪和压力,可以让你暂时放松下来,但结果就是陷入恶性循环,得不偿失。

希望大家能好好坚持戒色养生,把精子质量养好,把最好的自己留到结婚后,千万不要还没结婚,就把自己给废了。精就是种子,种子好,将来你的后代才会更健康。如果种子不好,你的后代就可能体质不佳,出现多病甚至夭折的可能。

后记:

在回帖时,有看到戒友讲到:男女居室,人之大伦。孤阴不生,独阳不长,人道不可废者。这个戒友引用了古代房事养生的理论,有一定道理。但如果阅历不够深厚,没有自己的思考和理解,就会认为何必禁欲呢?不是说孤阴不生,独阳不长吗?禁欲是不是会导致短寿呢?不碰女人,阴阳不调和,那不就是“独阳不长”吗?如果我没有广泛的阅历,也会这样认为,但我深入中医理论,发现这句话并不是绝对的,是有前提条件的。

中医有讲到:精少则病。也讲到:肾水无封藏太过之病,肾水愈能封藏,阳根愈坚固也!大家看到这句话肯定会觉得,这句话不是和“独阳不长”矛盾的吗?我举个案例大家就明白了,虚云法师一生没碰女人,活到了 120 岁,佛门活过 100 岁的有不少,如果说“独阳不长”,那和尚不碰女人,怎么可能活过 100 岁?道教修炼,也提到了要完全禁欲,自古没有漏精的神仙。白玉蟾有说过:点滴精液不漏,即是登天行梯。

所以,“独阳不长”并不是绝对的真理,并不适合修道者,对于凡夫还是比较适合的,因为凡夫修心不到位,会有 YY,大家都知道,YY 憋着不好,YY 本身就是暗漏,很多戒友 YY 后就出现了症状的反复,而修道者修心功夫到位,并不会存在这个问题。

不过,凡夫也不能有念头就行房事,还要看自己的身体状态,如果身体虚着,已经症状缠身了,是要禁欲养身体的,等把肾气养足,彻底恢复健康后,才能节制地过性生活。古代很多名医看病,都要求患者远房帏的,就是远离房事,否则身体是万难痊愈的,药的疗效也会大打折扣。现代很多中医不讲了,很遗憾。

\subsection{戒色状态的调整、发育障碍、紧张性遗精}

前言:

最近浏览贴吧的帖子,看到很多戒友破戒,原因还是过不了 YY 关,而戒色要成功,必须过 YY 关,可以这样说,如果你能够杜绝 YY,其实就成功了一半,你的戒色天数就会猛增!如果你过不了 YY 关,等待你的必然就是反复的失败。失败不可怕,可怕的是不学习,只有不断学习提高觉悟,不断总结经验教训,才能戒得越来越好。失败了不要灰心,要看到自己的进步,提高觉悟有一个过程,当你觉悟修到了,自然就戒掉了,所谓:水到渠成!

断 YY 贵早,就是要断得早,一定要先知先觉。

有的戒友和我说,欲火中烧了怎么办?做俯卧撑可以压制欲望吗?我的回答就是,你断 YY 太晚了,YY 的念头刚起来时犹如小火星,断掉很容易,当 YY 的小火星变成燎原之势,变成欲火中烧,你想扑灭它,太难了,等到它发展壮大了,你才后知后觉,想到要去扑灭它,很有可能就会适得其反,越压制越旺,越压制越反弹。不怕念起,就怕觉迟。在 YY 的念头刚出来时 0.01 秒就要察觉到,马上断掉它,不犹豫,不妥协,不拖延,要果断,干脆利落!手起刀落!断得早!断得狠!断得快!你不降伏心魔,心魔就降伏你,没有第三种可能,被心魔降伏,就等着被症状虐吧!

关于断 YY,我 17 季的文章写得比较详细,大家可以再看下,我所写的文章中,17 季是比较重要的一季,因为里面涉及到如何断 YY,过了 YY 关,戒色才能日趋稳定,否则就是戒了很久,还是失败,还是打不过心魔。

我上季有张图片讲到,戒色如下棋,坐在你对面的就是心魔。和现实中的下棋一样,刚开始你和一个高手下棋,只有被虐的份,怎么下,都下不过。因为人家是高手,有着丰富的实战经验。而你是新手,什么都不懂,还存在很多思想误区,在这种情况下,如何能下赢心魔?心魔段位比你高,而你只是一个菜鸟。其实,要下赢心魔,只有华山一条路,那就是不断学习提高觉悟,等你觉悟修到了,段位上去了,再战心魔,你就会发现心魔不是你对手了,你可以降伏心魔了,可以战胜它了。戒色要成功,就是一个觉悟不断提高的过程,等你有了学习意识,有了良好的学习习惯后,离戒色成功就不远了!

还有一个问题,就是勃起时,流出的透明液体,到底是什么?我研究过的文章,一般有 2 种说法,一种是前列腺液,一种是尿道球腺分泌的液体,有润滑的作用,有的戒友会叫它润滑液。尿道球腺液,它透明而粘稠,可拉长成丝,有润滑尿道的作用,并构成射出精液的最初部分,也是组成精浆的成分之一。而前列腺液也是精液的重要组成部分。所以不管流出的是前列腺液还是尿道球腺液,其实都对身体会造成损伤。很多戒友,光看不撸后就出现了症状反复,YY 是暗漏,中医有讲到:心动则精自走。我看过的中医医案中,就有记载 YY 导致疾病的案例,YY 是能够致病的。这点千真万确!光看不撸之后,会有什么后果,你可以照照镜子,你会发现自己的气色下降了,然后很有可能会出现睾丸痛,小腹涨,或者尿频的现象。这就是 YY 暗漏的后果,这是大量戒友反馈的结论。大家一定要坚决杜绝 YY,做到彻底戒色!

下面进入正文。

这季就戒色状态的调整,发育障碍和紧张性遗精详细论述一下,具体如下:

最近反映戒色状态不佳的戒友有很多,比如出现乏力,懒散,情绪低落,动力丧失等现象。

这种情况我也有过,出现这种表现和季节是有密切关系的,“春困秋乏夏打盹儿,睡不醒的冬三月”,季节转换对人的心理和生理都有着深刻的影响。夏季天气炎热晚上休息不好,白天容易犯困不难理解,然而到了秋高气爽金秋时节,人们为什么还会出现秋乏现象呢? 因为,在炎热的夏天,人的身体大量出汗造成了水盐代谢失调,肠胃功能减弱,心血管系统的负担加重,人的身体处于过度消耗阶段。夏去秋来,气候由炎热变得凉爽宜人,人体出汗也明显减少,人的机体进入到了一个周期性的休整阶段,水盐代谢开始恢复平衡,人的心血管系统的负担也得到缓解,消化系统功能也日渐正常,然而此时人们的身体却有一种说不出来的疲惫感,这就是人们常说的“秋乏”。其实这是不同季节人体的自然生理反应。经过一段时间的调整,秋乏现象会自然而然地消除。

所以,如果你最近戒色状态不佳,建议注意休养,不要搞得太累,注意情绪管理,慢慢就会过去的,戒色的好状态会回来的。自己一定要学会调整,这异常关键。人的戒色状态是起伏的,这很像运动员比赛状态的起伏,梅里特现在跑 12 秒 80,他下次还能跑 12 秒 80 吗?也许状态就回落了。戒色也是如此,刚开始不少戒友看了几篇戒色文章,马上决心极大,戒色热情高涨,但是过了十几天,马上就丧失热情了,找不到动力了。看的时候疯狂看,不看的时候几周都不看戒色文章。这样觉悟的提高就没有持续性可言了,就像爬山,爬到一半就半途而废。所以,我们要戒色成功,必须尽量保持稳定的戒色状态,刚开始时热情高涨,可以多摄入些戒色文章,当热情消退时,我们不要不看戒色文章,而是把摄入量减少些,每天按时看些戒色文章,就像每天刷牙一样,养成习惯,习惯成自然。有了稳定的戒色状态,觉悟的持续提高就有保障了,就像你打网游,不是打一天就休息十几天,而是每天打,这样练级才快。当然,我是不赞成沉迷网游的,只是打个比方而已。

导致戒色状态发生变化主要有以下几点:

\begin{description}
    \item [戒色厌倦期] 26 季的文章有专门讲到
    \item [季节因素] 季节转变是可以影响到人的情绪和生理的,从而影响到戒色状态,这点是比较容易被忽视的
    \item [看了无害论] 一看无害论就容易动摇,从而产生退心和疑心
    \item [做事的计划性] 做事有计划的戒友,会戒得更稳定,每天给自己安排多少量的学习,他心中有数
    \item [生活琐事] 如果琐事很多,是容易导致分心,或者心一累,就不想看戒色文章了
    \item [工作学业压力] 人在压力大的情况下,情绪容易紊乱,从而影响到戒色状态
    \item [有老婆或者女友] 这类戒友的戒色状态,是很难稳定的,具体什么原因,我不说大家也应该知道
    \item [环境污染] 比如寝室有人看黄,或者因为应酬要去那些场所,这都会影响戒色状态
    \item [遗精后] 遗精后出现破戒的情况还是很多的,遗精后是容易产生思想动摇的,一定要保持警惕
    \item [饮食因素] 肉吃多了,补药吃多了,容易产生妄念,从而影响戒色状态
\end{description}

我们要戒色成功,必须要让自己的戒色状态保持稳定,就像走钢丝一样,要让自己保持稳定,找到一个平衡点,这样才能戒得长久,戒得稳定。

下面谈下发育障碍。

我搜集了几千个受害者案例,聊过上千个戒友,其中反映发育障碍的还是很多的,主要集中在以下几点:

\begin{enumerate}
    \item JJ 发育障碍
    \item 身高发育障碍
    \item 面容发育障碍(看上去偏小很多)
\end{enumerate}

前 2 点,相信大家比较容易理解,第 3 点,肯定有人会说,面容偏小不是好事吗?其实面容偏小,不一定是好事,很多戒友正在为此发愁,在同龄人当中面容偏小,容易让人看不起,而且二十几岁的人,看上去像高中生或者初中生,将来找工作也会遇见麻烦,用人单位会觉得你不够成熟,无法胜任工作。这里的偏小不是偏小一二岁,而是在 5 岁以上。SY 后,面容会出现 2 种倾向,一种明显变老,一种就是发育延迟,面容偏小。这 2 种都会给人带来烦恼,而SY后变老变丑比较多见一些。

JJ 发育障碍比较常见,JJ 属肝,而 SY 伤肝肾,是会影响到 JJ 发育的,有的人本来体质就不行,再加上疯狂沉迷 SY,结果就会影响到 JJ 的发育,出现短小的现象,搞得自己很自卑。很多戒友在 \SI{10}{\centi\metre} 以下,有的则在 \SI{8}{\centi\metre} 以下,有的则是太细,也是非常让人烦恼的问题。出现 JJ 发育障碍,应该要赶紧戒掉 SY,注意调理身体,这样是会有所改善的,至少勃起质量会有所提升。

身高发育障碍,我在第 8 季有专门讲到,有身高烦恼的戒友可以看看。影响身高的因素很多,SY 是其中一个因素,沉迷 SY 是有可能影响骨骼发育的,因为中医:肾主骨。沉迷 SY 恶习会导致骨骼发育障碍,从而影响到身高。但并不是所有 SY 的人都会影响到身高,因为影响身高的因素有很多,SY 只是其中一个因素,如果其他因素都做得很好,比如吃得很好,运动到位,按时作息,基因优异,这样还是可以长到理想的身高,但可以肯定的是,SY 绝对会影响到骨密度,SY 会导致腿软,在剧烈运动时,容易出现骨折骨裂的现象。

SY 的的确确会影响到发育,但每个人体质不同,后天的营养和生活运动习惯的不同,导致影响程度也不同,有的人热爱运动,无不良嗜好,按时作息饮食,营养跟得上,虽然他也 SY,但影响程度比较轻微,而有的人天生体质不佳,再加上久坐熬夜,不运动,吃得也不好,这样就很有可能会出现严重的发育障碍。所以在发育期的男孩们,一定要及早认识到 SY 的危害,免得将来欲哭无泪,过了发育期就定型了,很多东西很难改变了。

最后谈下紧张性遗精。

紧张性遗精还是很多的,经常会出现这类问题的戒友。

我知道紧张会导致遗精,是在初中时代,记得当时期中考试,有一个同学就在考场遗精了,因为他还有题目没答完,但时间已经到了,他一紧张,就漏了,当时这件事在同学之间还传为奇谈,觉得不可思议。而我现在做 SY 行为的研究,才发现紧张性遗精其实还是很多的。

大家先看几个案例:

\begin{enumerate}
    \item 飞翔大哥!请问一下晚上睡觉做的不是春梦,而是梦见在考试没写完,一着急就忍不住射了。这种情况是遗精还是滑精?
    \item 飞翔哥,我是一名大四学生,快要考司考了,这几个月倍感煎熬,从今年一月戒到现在,破了很多次,由于考前特别紧张,最近一连四天都遗精,包括昨晚,我对未来失去信心了,希望你帮帮我!再遗精下去整个人都没了!
    \item 飞翔哥,昨晚遗精了,没有坚持到 100 天,遗完后非常沮丧,是梦遗,梦见自己被老师点唱歌,我不知道唱什么好,一着急,就遗了,一遗就醒了,才四点半。当时很沮丧,没有了原来的兴奋感。我还要继续努力,坚持固肾功,向下一个 100 天奋斗。
\end{enumerate}

我初中那个同学,是白天清醒时紧张遗精,而很多戒友是晚上做梦一紧张就遗精,或者白天紧张情绪很强烈,日有所思夜有所梦,晚上也会出现紧张性遗精。出现这种情况,大家不要慌,一定要学会管理自己的情绪,多给自己正面的暗示,不要多想。再加上坚持练习固肾功和积极的治疗,这种紧张性遗精是可以克服的。记得以前有一个戒友总是因为梦魇而遗精,一直很困扰他,后来我建议他积极治疗并且坚持练习固肾功,平时则要学会情绪管理,让自己保持在心平气和的心理状态,后来他反馈遗精比以前好了很多。

不少人一紧张就六神无主,心烦意乱,这时候你应该告诉自己,深呼吸,保持从容,保持心理状态的安详。多给自己这方面的暗示,时间久了,自然会克服紧张情绪的。保持淡定是非常重要的,你如果在白天能够保持从容淡定,慢慢是会影响到潜意识的,到时候,在梦里出现紧张性遗精的情况也会慢慢减少乃至消失。这是一个心结,你必须学会解开它,而不是越搞越紧张。

后记:

很多戒友存在把不良情绪转变成 SY 的行为定势,就是一遇到挫折或者烦恼的事情,马上就想通过 SY 来发泄,有这方面倾向的戒友一定要学会情绪管理,纠正这种错误的行为定势。我在破戒类型那季里面,专门有讲到情绪破戒,情绪破戒太普遍,但危害实在太大了,会让你陷入恶性循环。管理好自己的情绪,对于戒色成功是必须要具备的素质。有的戒友被爸妈骂,然后情绪压力很大,回房间就破戒了;有的戒友被老板骂,情绪失控,想不通,回家就破戒了;还有的戒友是乐观情绪破戒,比如高考后,一放松就破戒了。或者为了庆祝胜利,庆祝成功,这时候人也有放纵的倾向。还有就是无聊情绪,周末一个人在家无聊,心魔就跑出来了,然后鬼使神差般地就破戒了。情绪破戒太普遍,我们一定要意识到管理情绪的重要性,好好管理自己的情绪,让自己戒得更稳定。

\subsection{直指焦虑症,神经症,深挖神经症}

\subsubsection{前言}

这几天看到一个帖子,是前列腺炎戒友,他说他有腹痛的情况,每次排精后就不痛了。我的建议就是,不要希望通过 SY 排精来缓解炎症。有炎症应该积极治疗,控制住病情,然后好好坚持戒色养生,这才是身体恢复的正道,否则,你靠 SY 排精来缓解炎症,这只会让你陷入恶性循环,身体会一步步垮掉的,前列腺炎永远别想好了。我们治疗前列腺炎,指导思想一定要正确,前列腺炎主要靠养,其次才是靠治,三分治,七分养,治疗控制病情,然后就是要靠戒色养生了,如果你没有戒色养生意识,前列腺炎就极有可能复发,为什么前列腺炎的复发率在 90\% 以上,原因就是他们的指导思想出现了误区,还以为靠药能治好,其实前列腺炎靠药是无法根治的,前列腺炎是养好的,必须学会养生之道,建立起戒色养生的意识。这样前列腺炎才能慢慢痊愈,一般慢前患者的恢复时间在一年左右。如果你一边吃药一边 SY,那花多少钱都治不好了,很多戒友花了好几万,用了很多种治疗手段,先进的机器都给用上,各种疗法都给用上,还是不行,就是不懂得戒色养生,钱花冤枉了,身体依然好不了。医生也不会告诉你为什么治不好,一方面,有的医生思想存在误区,认可无害论。另一方面,有的医生也希望你治不好,这样医院就可以搞创收了。很多医院,医生看病是吃回扣的,你觉得他会和你讲 SY 有害吗?

另外,大家非常关心的晨勃问题,其实也可以用雄性激素的分泌来解释,一般清晨 5 点左右,雄性激素分泌达到高峰,然后就会出现晨勃现象。而在发育期时,雄性激素分泌极其旺盛,这样就可能会每天都有晨勃,记得我刚发育那会,的确有一段时间每天都会出现晨勃,后来沉迷 SY,晨勃现象就减少很多了,直到后来彻底消失了,当然影响晨勃的因素有很多,比如饮食因素,吃肉比较多,吃补肾的食物比较多,晨勃就容易出现。还有就是情绪因素,情绪和晨勃是有密切关联的,这点很多人容易忽视,情绪可以影响雄性激素的分泌,从而影响晨勃。比如你生活中遇见不开心的事情,然后这种悲观的情绪是会导致雄性激素分泌下降的,这样就可能出现晨勃消失的现象。情绪和激素是互相影响的,很多时候情绪低落都与雄性激素分泌失调相关,激素是会反过来影响一个人的情绪和行为的。比如有的人今天醒来,感觉情绪很好,心里有愉悦感。但是过十几天后,情绪就突然变得很低落,这其实和体内的激素分泌密切相关。女性更年期,随着卵巢功能逐渐减退,雌激素减少,然后她的情绪和外貌都会发生变化,情绪的显著变化就是伴有敏感、多疑、烦躁、易怒等不良情绪。男性其实也有更年期,男性到了更年期,随着体内睾酮的减少,情绪和身体也会发生一系列的变化。所以说情绪和激素是互相作用的关系。

还有一个影响雄性激素分泌的显著因素就是运动习惯,适量的有氧运动和力量训练是可以促进雄性激素的分泌的,我每次杠铃深蹲第二天早上就会出现晨勃,深蹲是可以刺激雄性激素分泌的,有氧运动也可以提升雄性激素的分泌水平,但有一个前提,就是不能过度锻炼,过度锻炼会导致雄性激素的下降,不管是有氧还是力量训练,过度了都会导致激素分泌下降。季节因素也不可忽略,现在秋季,一年中雄性激素分泌最高,我感觉自己晨勃次数多了很多。

如果从生理角度去理解,可以说,人的本质就是激素,人体会分泌 75 种以上的激素,它们在人体内扮演着各自的角色。体内荷尔蒙浓度高的女性,比体内荷尔蒙浓度低的同龄女性看起来要年轻很多。而体内雄性激素分泌旺盛的男性,进取心和攻击倾向,也比一般男性要强很多。

有戒友会说,戒色后晨勃会不会恢复,因为他沉迷 SY,晨勃已经消失很久了,我的回答就是,坚持戒色养生,晨勃是会恢复的,我以前晨勃曾一度消失很久,后来坚持戒色养生又恢复了。还有的戒友会问,SY 到底会导致雄性激素分泌增加还是减少,其实这个问题很好回答,SY 导致的失调问题,基本都走 2 个极端,有的戒友沉迷 SY 后,雄性激素分泌减少了,比如我,SY 后感觉胡子少了。而有的戒友则是 SY 后,胡子体毛生长更加旺盛。之所以出现 2 个极端,其实和个人的体质有关,中医有把人分成 9 种体质,SY 后的激素分泌失调,要看个人体质而定,在有的人身上表现为雄性激素分泌增加,在有的人身上表现就是雄性激素分泌减少。分泌增加其实也不是好事,容易造成雄秃的可能性。我所看过的案例,以 SY 后雄性激素分泌减少比较常见,大概占 70\%,还有 30\% 就是 SY 后雄性激素分泌增加。

关于晨勃问题,大家也不必强求天天要有晨勃,随着年龄的增长,你会发现晨勃会慢慢减少的,晨勃和多种因素有关,关于晨勃我们要懂得顺其自然,不要强求。我现在主要关注的并不是晨勃的频率,而是晨勃的质量,一次质量好的晨勃,可以超过 100 次差的晨勃。持久坚挺的晨勃为好的晨勃,也可以作为早泄是否恢复的指标,早泄是不可以随便试的,可以从晨勃质量中观察到是否恢复了,如果你去试,很可能又掉进 SY 的陷阱。很多人虽然有晨勃,但晨勃质量很差,晨勃不坚不持久,硬度不行后劲不足,而且很多人晨勃是有 YY 的晨勃,并不是真正意义上的晨勃。另外,憋尿的晨勃也不是真正意义上的晨勃。

下面步入正题,这季就一个主题:直指焦虑症,神经症,深挖神经症。具体论述如下。

\subsubsection{关于神经症}

神经症,全称是神经官能症,对于很多戒友可能很陌生,因为很多戒友还没伤到那个程度,难以理解神经症到底是什么感觉。没有经历过神经症的戒友可以大致了解下,增长一下自己的见识,也可以警示一下自己。

神经症是伤精患者的一道分水岭,伤到神经了,要恢复相对就比较慢了,轻微的神经症还好恢复些,严重的至少一年以上才有望恢复。这个病是需要悟道才能好的,靠吃药是极难痊愈的,很多人始终没有开窍,始终不明白为什么,就像进入一座迷宫,进去后就迷在里面了,几年乃至十几年都出不来,每天都靠药维持着,生活质量大受影响。很多特别严重的,会出现自杀倾向,媒体上经常有抑郁症焦虑症自杀的新闻,有的人有钱有名,资产上亿,结果他也自杀了,普通人想不明白,为什么会自杀呢?其实一旦得上了严重的神经症,想自杀的极多,我也曾经出现过自杀的念头,那时的我,每天活在症状地狱里,没经历过的人,的确很难体会到那种感觉,那是对身心的极大摧残,很多人得病后,性格都发生了巨变,判若两人。

我那时分别出现过恐惧症(社恐、恐艾、恐癌)、疑病症、强迫症、焦虑症、神经衰弱、胃肠神经症、心脏神经症,头也昏沉过,也头痛过,还有全身游走性刺痛,简直就是地狱般的刺痛,简直就是酷刑,躺在床上我老出现地震的感觉,后来才发现是严重的躯体震颤感,肌肉也乱跳,我那时真的快崩溃了,全身没一个地方舒服的,我之前不相信有人间地狱,得了神经症,我信了。身处症状地狱,的确是一种莫大的折磨。

神经症其实非常强调体验,如果你没体验过,你是无法真正理解那种感觉的,就像我和你说蹦极的感觉,你只是听我说,但没有去尝试过,这样你的体验和感受就不会很深刻,也不会很直观。很多医生没得过神经症,他们对神经症的理解来自于书本,而我可以很负责地说,书本上很多理论还不够完善,特别是西医在这方面的理论还处在研究阶段,所以很多人对神经症会产生误解,包括很多医生。不可否认的一个事实就是,庸医多,真正明白的医生少之又少,看几十个医生,能遇见一个真正懂的,就算是你的造化了。神经症患者被误诊是极其普遍的,有的患者心脏不舒服,当心脏病来治疗,而不是当神经症来治疗,这样治了大半年,花了好几万,还是看不好,也没有多大缓解,搞到最后才知道是得了神经症,而之前一直当心脏病来误诊,所以,神经症患者对于医生是颇有微词的。对于神经症,我推荐还是看中医比较好,西医的药副作用大,也容易造成依赖,很多病友都依赖上了药物,药物上瘾,不吃不舒服,其实吃了也没好多少,在一种恶性循环中苦苦挣扎。

神经官能症导致的常见疾病:

\begin{enumerate}
    \item 慢性咽喉炎、口腔溃疡;
    \item 肠易激综合症、结肠炎、慢性胃炎;
    \item 神经性头痛、头晕、头昏、失眠 、多梦;
    \item 抑郁、焦虑、恐惧、强迫、疑病症;
    \item 多汗、虚汗、盗汗、怕冷、怕风;
    \item 心脏神经官能症、胃神经官能症;
    \item 脖子肌肉僵硬 、关节游走性疼痛、幻肢痛;
    \item 记忆差、反应迟钝、神经衰弱;
    \item 早泄阳痿、易感冒、免疫力低下。
\end{enumerate}

神经官能症患者有的人以胃肠神经官能症为主,有的以心脏官能症为主,有的则是头部极其不舒服,还有的患者症状很多,基本都有。我聊过的病友中,很多人都有十几种乃至几十种症状,千奇百怪,出症状的规律是此起彼伏、层出不穷,搞得自己很恐慌,其实根源只有一个:那就是植物神经功能紊乱导致了免疫系统的功能紊乱。那是什么导致植物神经紊乱了呢?我当时也很困扰,一直在找原因,最后经过我的深入研究,聊过上千的病友,得出一个公式:熬夜 + 纵欲 + 久坐 = 完蛋。光纵欲,那需要伤到一定程度才会出现神经症,我过去十几年频繁 SY,只是前列腺炎和精索,虽然人也变丑好多,但是神经一直是好的,后来我久坐熬夜,不好好吃饭,然后就出现神经症了。应该这样说,神经症的出现是众多因素共同作用的结果。女病友则是生气、压力大、熬夜导致的,还有精神刺激,比如家庭变故等。

\subsubsection{真假神衰}

很多戒友因为沉迷 SY,出现了记忆力和理解力大幅度下降的情况,这种情况,有的戒友就会觉得自己得了神衰,其实据我观察,很多戒友只是脑力下降而已,不一定是神衰。

选了几个典型神衰的表现,大家可以对照下:

\begin{enumerate}
    \item 本人男,25 岁,由于小时候(12 岁)一次偶然机会不幸染上 SY,SY 2 年后开始出现不适症状:头晕耳鸣、全身乏力、失眠、视觉昏暗,感觉眼前的世界很不真实,好像自己活在做梦一样,腰部酸疼,精神萎靡,胸闷气短,心慌心悸,记忆力下降,脸色黑黄,瞌睡倦怠,额头两侧脱发,手心脚心出汗,大便不成形,还有泌尿方面的疾病如尿频,尿不尽。
    \item 飞翔大哥你好,小弟 SY 最少有十年的历史了。今年 21 周岁。今年五月末突然发现脑力不足,精神恍惚,觉得整个世界有空虚感。因为正在备战考研,真的很无助,身体一直还不错,也经常打球啥的。经过了解我大概确定应该是有神经衰弱,最严重的时候晚上根本无法入睡,每天就两个多小时,晚上盗汗,人消瘦了五六斤,尤其是脊背瘦多了,肚子上还好,下眼皮突然变黑,白天觉得眼睛不好使,眼睛老是被重压着很难睁开,白天就像喝醉酒似的,天哪!真不敢想象,那时候都不敢上街,害怕被车撞了……
    \item 我今年 20 岁身体发黄瘦弱,5 年前开始无节制的 SY,起初身体没什么症状,可是 1 年后的一天在上楼途中突然脑后一沉,然后就开始头昏目眩,开始只是认为是发烧引起的,可是后来一直是这样没有好过,直至现在还是这样。5 年期间人一天一天的衰老下去,全身无力、精神萎靡、记忆力减退、注意力不能集中,休息睡眠更是差得要命!人慢慢地变成了废人,也看过不少大夫吃过不少药,可是都不见效。
\end{enumerate}

\subsubsection{感恩焦虑症}

虽然焦虑症给了我很大的痛苦体验,但我还是要感恩焦虑症,因为没有焦虑症,就没有现在的我,是焦虑症度了我,是焦虑症让我开始自学中医,是焦虑症让我开始悟道的生活,把我的精神世界提升到了另一个境界,塞翁失马焉知非福,是焦虑症点化了我。我是以病入道的,据我所知弘一法师李叔同也得过神经症,神经症是他出家的一个助缘,文摘如下:

\begin{quote}
    今得阅弘一法师述,高胜进笔记的《我在西湖出家的经过》,始知其绝食出家的本因:患有神经衰弱症。其文曰:“到了民国五年的夏天,我因为看到日本杂志中,有说及关于断食方法的,谓断食可以治疗各种疾病。当时我就起了一种好奇心,想来断食一下。因为我那个时候患有神经衰弱症,若实行断食后,或者可以痊愈亦未可知”。
\end{quote}

\subsubsection{濒死感体验}

焦虑症,严重的会有惊恐发作濒死体验,先选 2 个案例,大家可以看下:

\begin{enumerate}
    \item 今年 1 月独自上街,突然觉得胸闷、心悸心慌、呼吸困难,濒死感,随即送医院,当时测血压正常,心跳 100 / 分,心电图正常。此后不敢独自上街,或在人多的地方或与人聚餐是就容易紧张,现在还不定时发作,发作时气透不过来的感觉、心悸、出很多汗、全身发冷、濒死感、不真实感,要发疯感或失去控制感,每次发作约一刻钟左右,非常难受,现在还不时有头晕,脖子痛。做过 SCL-90 的测试,恐怖重,焦虑中,其余轻度升高。看过中医说,舌淡苔薄白,脉沉偏细数。我有 2 型糖尿病,担心服西药副作用大,给身体负担更重,但服用中药、针灸、耳穴压豆,但效果一般,担心吃西药副作用大,成瘾或依赖,希望在医生指导用药,降低我的忧虑。
    \item 我得焦虑症 4 年了,经常惊恐发作,发作时心跳加速,达到 140、150 次,一开始还以为是心脏病,但做了 5 次 B 超和冠脉 CT、2 次心机酶、3 次心电图、100 多次 24 小时动态心电图、X 射线甲亢胸片,全正常。发作时感觉从血管里发冰、发冷,然后莫名的心跳加速,非常恐惧!简直就象要死了!!发作时心跳得超级快,但打出来的心电图却只是心动过速,医生诊断惊恐发作。哎!也没给任何药物,大概过了 20 分钟,心跳就自动恢复了,恢复到 70 次左右了。痛苦啊!
\end{enumerate}

有病友会问我,惊恐发作会不会死,这个问题其实不好回答,因为死人不会说话,能救活的人基本都会说濒死发作不会死。全国每年 55 万人猝死,你觉得里面是否有焦虑症患者呢?

\subsubsection{SY是会伤神经的}

SY 的确会伤到神经,但要伤到一定程度,如果再有熬夜久坐,这样出现神经症的概率会加大很多。伤到神经是非常痛苦的,那是一个真正的症状地狱,那是一个深渊,不被理解的深渊。掉进去后,也很难爬出来,必须彻底悟道才有可能真正摆脱它。

\subsubsection{得病过程}

我频繁 SY 十几年,因为热爱运动,按时作息,合理饮食,所以没有出现过神经症。后来开始熬夜久坐了,就不对了,身体一天天不行,被神衰、焦虑、植物神经紊乱折磨着,过着生不如死的生活。而我得病前后就一个月,也就是说,我放纵了一个月,然后就出现了神经症!而且更让人难以置信的是,那段放纵的时间,我每天都做几百个俯卧撑,卧推有 \SI{120}{\kilo\gram},衣服脱了很强壮的感觉,一点都不像会生病的感觉,但我也得上了神经症,而且是那种生不如死的神经症,并伴有严重的恐惧、焦虑,几十种躯体症状轮番折磨我,后来我研习中医才知道,强壮并不代表健康,很多强壮的人其实都有毛病,属于暗疾,只是别人不知道。真正会养生的人,才是健康的人。强壮的人,不是,外强中干的很多。

\subsubsection{误诊经历}

我开始是胃肠神经官能症,然后去消化内科看,当消化问题治疗,但效果很不好,后来才知道被误诊了,其实是神经的问题,还看过皮肤科,然后按皮肤病治疗,效果也不好,西医分科有缺点也有优点,缺点就是神经症这类病,很容易被误诊,有的心脏官能症的病友,一误诊就是大半年,花费几万都看不好,光检查费都上万。后来我自己查,才慢慢觉得自己很可能是神经症,再和病友一交流就更加确定了,然后我去神经内科一看,确诊为焦虑症。

\subsubsection{疑病倾向}

得了神经症后,出现疑病的病友极多,因为疑病,总是不断地去医院检查,检查费惊人,关于这点,其实医院也承认,疑病倾向的人,的确在为医院搞创收。这些人都以为自己快不行了,以为自己得了大病,希望能检查出来,检查后,一切正常,心理舒缓些,过段时间又不舒服了,然后再去检查,就这样很多人的钱都花在了检查上,其实他们不知道,是神经的问题,根本就检查不出什么器质性问题。当然,该检查的检查下,也没必要经常去检查,检查一次排除了器质性的问题,就不用经常去检查了。一定要认识到问题的所在,是神经出了问题。

\subsubsection{注定被误解}

身体出现很多不适症状,第一反应肯定去医院检查,但是神经症检查很多次,你会发现基本没什么大的问题,但是症状是明显的,让人崩溃的,让人恐慌绝望的。但检查出来,却没多大问题,这种情况下,家人就会认为你装病,或者说,病是你想出来的,是你自己多想了,其实你没病。这种对神经症的误解太多太多了,不仅家人会误解你,很多医生也会误解你,因为医生也会认为,病是你想出来的,这样很多病友,处在无法被人理解的困境中,好在现在有网络,才能找到同病相怜的病友,否则一直这样不被理解,真的有可能走上绝路,很多病友之所以那么喜欢聊天聊症状,就是图一个心理安慰,知道有很多人和自己一样。其实光聊症状也不行,好不了,必须要找到真正的病因,病因认识不正确,真的万难痊愈了。有的病友,检查费好几万,甚至十几万的都有,还有的病友经常抢救,经常住院,医生也拿他没办法,检查不出什么毛病,但身体就是不行。有病友会问我,他的身体到底怎么了,后来我和他说,其实神经症基本都是功能性的问题,又不是器质性的问题,基本检查不出什么问题,打个比方,你一部自行车骑了一年多,很旧了,你这时候感觉旧车和新车已经没法比了,旧车骑着很累很费劲,但是你把旧车送去检查,零件都正常,就是旧了,功能没新的时候好了,其实神经症就是这种状况,功能大不如前了。

还有一种误解,来自普通人,因为抑郁症和焦虑症,一般人都以为是心理出了问题,和躯体没任何关系,是一个人想多了导致的,其实这真是大错特错了,很多人发病前正常得不得了,乐观向上,突然的焦虑症把他击垮了。所谓心理疾病,并不是单纯的心理问题,而是有生理基础的,不断 SY 伤肾气,熬夜久坐也伤肾伤脾伤肝,然后就出现了神经症,如果你单单认定为心理疾病,是想出来的,这是不正确的,现在普遍这样认为,说是心理疾病,其实是错的,我是以一个深刻的体验者和研究者的身份来讲的。我觉得没深刻体验过的人,很多想法都是有偏颇的,包括很多医学教材上的论点,都不是正确的,并不符合真正的事实。就像无害论一样,现在无害论也上教材了,难道它是正确的吗?事实证明它是错误的,尽信书不如无书,前人正确的论点,我们要继承,如果是错误的,我们要有勇气去纠正。事实证明,心理疾病并不是简单的心理问题,并不是想多了。打个比方,惊弓之鸟的成语大家都知道,鸟本来有伤,然后听见弓的声音,一紧张,伤口破裂掉下来了,大家想一想,如果这只鸟没伤,能掉下来吗?肯定不会,其实神经症也是如此,如果你肾气充足,会得神经症吗?中医:肾气足,万邪熄。所以说,心理疾病的发作是有生理基础的,原因是你身体虚掉了,然后给一个刺激就突然发病了。心理疾病也不是那样简单,是有一大堆躯体症状的,焦虑情绪不代表焦虑症,抑郁情绪也不代表抑郁症。有的病友是这样反映的,他原来是非常乐观的人,是突然发病后才变得焦虑,并不是焦虑导致焦虑症,而是发病后开始出现焦虑、恐惧、强迫等倾向。我自己也是这样,我根本没多想,一直很乐观向上,结果熬夜久坐纵欲,然后就得了焦虑症。

如果说,是不是有想出来的焦虑症,我想如果真有这种情况,也是占很小比例,比如有的人天生体质极差,肾气严重不足,这样受点刺激也是有可能出现焦虑症的,但我通过上千的案例比对和分析,可以说 99\% 都是有生理基础的,而这种发病的生理基础并不是先天体质不好,而是后天放纵自己和不良的生活习惯导致的,并不是想出来的病。

\subsubsection{吃药回忆}

吃药,是我不堪回首的一段经历,因为我严重依赖过药物,把药当饭吃!我现在家里还有一大箱没吃完的药,很多都过期了,我还留着,就是提醒自己,我曾经活在症状地狱里。我不想再放纵自己,我不想再重返那个生不如死的症状地狱。我那时谷维素吃了几十瓶,没多大效果,黛力新天天吃,刚开始吃,效果不错,吃多了就耐药了,然后看医生,就给我换药,然后再继续依赖药物,光吃药就花费了我不少钱。关键是吃了还不能痊愈,最多只能缓解,那真叫一个生不如死。年纪轻轻,靠药活着,不吃药,比死都难受。普通人无法理解那种感受,只有病友可以理解。

\subsubsection{直指病因}

病因在上面的文章已经提到,男病友就是:熬夜 + 纵欲 + 久坐导致的,这是我聊了上千个病友得出的答案。肾为元气之根,熬夜、纵欲、久坐就像三把斧子砍向生命之树。当然个别人吸毒、酗酒、暴怒,这也很伤身体,也很容易导致神经症的出现,另外遗传、压力大、家庭变故等因素也可能诱发神经症。

\subsubsection{康复过程}

神经症的康复过程,基本就是 2 个方向:

\begin{enumerate}
    \item 治心
    \item 治身
\end{enumerate}

身心是合一的,治身可以影响心理,同样,治疗心理也可以影响身体,心理的健康也是极其重要和关键的,所以,得神经症后,很多病友都开始信佛,信佛对修心是非常好的,通过修心来影响身体的健康。我现在也信佛了,当然我有佛缘,没佛缘的病友可以多学习其他方面的传统文化,也是很利于修心的,比如《弟子规》就很好。心对,身也会跟着对。

治身方面我推荐中医调理,然后自己要多学习养生知识,可以选择适合自己的养生功法坚持练习,这样身心同治,神经症是可以慢慢恢复的,严重的至少需要戒色养生 1 年以上,很多病友不懂得戒色,不理解戒色,这类病友其实根本就没有认识到保精的重要性,肾为先天之本,脾为后天之本,要恢复,必须懂得保精,也要懂得养脾。这样各方面都做好了,养功到位,神经症才有望恢复,否则很多病友五年没恢复,甚至十几年没恢复的都有,就是不懂得养生,不重视保精!还在漏,还在 SY,还在自我摧残,这类病友就是没有悟道,不悟道,神经症太难好了,即使暂时痊愈了,也极易复发,很多病友在放纵后又复发了,非常常见。所以,神经症病友必须认识到戒色的重要意义,否则你的神经症太难好了,症状地狱会让你生不如死的。戒色养生是痊愈的基础,否则吃再多的药,也会被你漏掉。

\subsubsection{就医指导}

我推荐看中医,而且是百里挑一的老中医,好中医!因为庸医太多,你去看医生,最好先在网上了解哪个医生好,哪个医生的医术高超,可以去所在城市的医院网站或者中医 QQ 群了解,否则盲目就医,很可能会遇见庸医,有的庸医看不好就罢了,甚至还会向你灌输无害论,或者说你的病是想出来的,这类庸医和患者之间缺少真正的理解,他们的医术很值得怀疑。好中医永远是少数,得真传有心得的更是少之又少,所以一定要找好中医看。

再打个比方,一个中医学博士,和一个初中学历,但行医 40 年的老中医,你会选择谁?我想懂行的人肯定会选择老中医,老中医阅历深厚,在治疗方面有着丰富的经验,这在学校里和书本上是学不到的,只有通过实践和反复体验才能悟得真机。否则纸上谈兵,永远成不了好中医。有很多科班出身的中医,你要说理论,你不会比过他,他背书能力强,口若悬河,但我要问的是,真正的实践经验有多少呢?中医是讲究师承的,能得到大师指点,口传心授,这样进步提高才会比较快。真正的好中医永远是少数,他知道的比普通中医要多要深。

\subsubsection{最后总结}

我是以病入道的,应该算体验派,自己有深刻体验,然后再开始研习中医理论,我也是痊愈者,因为悟道所以痊愈。明道了才知道该怎么去做,我希望神经症患者看到这篇文章,能得到有益的启发,我是从症状地狱爬出来的人,太清楚那种感觉,希望我的文章能帮到你们,按照我指的路去坚持,相信你们会痊愈的。但恢复是一个缓慢的过程,需要保持耐心。加油!

\end{document}
