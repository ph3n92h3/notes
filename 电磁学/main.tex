\documentclass{ctexbeamer}

\usetheme{Goettingen}
\usecolortheme{crane}
\usefonttheme{professionalfonts}
\setbeamercolor{item}{fg=black}
\setbeamertemplate{footline}[frame number]
\setbeamertemplate{itemize items}[circle]

\usepackage{emoji}
\usepackage{esint}
\usepackage{extarrows}
\usepackage{multicol}
\usepackage{physics}

\title{电磁学期末梳理}
\author{冯哲}
\date{2021 年 12 月 25 日}
\institute{河海大学 理学院}

\begin{document}

\frame{\titlepage}

\begin{frame}
    \frametitle{目录}
    \begin{multicols}{2}
        \tableofcontents
    \end{multicols}
\end{frame}

\section{静电场}

\subsection{库仑定律}

\begin{frame}[c]
    \frametitle{库仑定律}

    $$\vec{F} =\frac{1}{4\pi\varepsilon_0}\frac{q_1q_2}{r^2}\hat{r}$$

    \begin{description}
        \item[成立条件] 真空,点电荷,静止
        \item[库仑力的叠加原理] 分立电荷的叠加,连续电荷的叠加
    \end{description}
\end{frame}

\subsection{电场强度、电势}

\begin{frame}[c]
    \frametitle{电场强度、电势}

    电场强度:
    \begin{description}
        \item[定义] $\vec{E}=\frac{\vec{F}}{q_0}$
        \item[求] $\vec{E}=\frac{1}{4\pi\varepsilon_0}\cdot[\sum_i\frac{q_i}{r_i^2}\hat{r}_i\text{ or }\int_{V\text{ or }S\text{ or }L}\frac{\dd q}{r^2}\hat{r}]$
    \end{description}

    \begin{description}
        \item[电场的叠加原理] 分立电荷的叠加,连续电荷的叠加
    \end{description}
\end{frame}

\begin{frame}[c]
    \frametitle{电场强度、电势}

    电势:
    \begin{description}
        \item[定义] $V=\frac{W}{q}=\int^{\vec{r}}_{\text{电势零点}}\vec{E}\cdot\dd\vec{l}$
        \item[求] $V=\frac{1}{4\pi\varepsilon_0}\cdot[\sum_i\frac{q_i}{r_i}\text{ or }\int_{V\text{ or }S\text{ or }L}\frac{\dd q}{r}]$
    \end{description}

    \begin{itemize}
        \item $\vec{E}=-\nabla V$,注意曲线坐标系下的 $\nabla$ 的形式!
    \end{itemize}

    \begin{description}
        \item[电势的叠加原理] 分立电荷的叠加,连续电荷的叠加
        \item[电偶极子] 求场、求势、求受力、求力矩
    \end{description}
\end{frame}

\subsection{静电场的高斯定理、环路定理}

\begin{frame}[c]
    \frametitle{静电场的高斯定理、环路定理}

    \begin{itemize}
        \item 电通量、电力线的概念
    \end{itemize}

    静电场的高斯定理:
    \begin{description}
        \item[积分式] $\varoiint_S\vec{E}\cdot\dd\vec{s}=\Phi_{\vec{E}}=\frac{Q_{\text{内}}}{\varepsilon_0}$
        \item[微分式] $\nabla\cdot\vec{E}=\frac{\rho}{\varepsilon_0}$
    \end{description}

    求解电场:
    \begin{itemize}
        \item 注意利用对称性选取高斯面:球面、柱面……
        \item 注意利用基本模型:球、平面、直线
    \end{itemize}
\end{frame}

\begin{frame}[c]
    \frametitle{静电场的高斯定理、环路定理}

    静电场的环路定理:
    \begin{description}
        \item[积分式] $\oint_L\vec{E}\cdot\dd\vec{l}=0$
        \item[微分式] $\nabla\times\vec{E}=0$
    \end{description}
\end{frame}

\subsection{导体、电介质}

\begin{frame}[c]
    \frametitle{导体、电介质}

    导体:电荷可以自由移动
    \begin{itemize}
        \item 电荷只分布在外表面
        \item 外表面处的电场:$\vec{E}=\frac{\sigma}{\varepsilon_0}$
        \item 导体等势……
    \end{itemize}
    电容器:
    \begin{itemize}
        \item 几种简单的电容器的电容公式:球形、柱形、平行板
        \item 求解:假设电荷 $\pm Q\Rightarrow \vec{E}\Rightarrow V\Rightarrow C$
        \item 串并联
    \end{itemize}
\end{frame}

\begin{frame}[c]
    \frametitle{导体、电介质}

    电介质:电场强度、电极化强度、电位移矢量、自由电荷、极化电荷

    定义:
    \begin{description}
        \item[电极化强度] $\vec{P}=\sum_i\vec{p}_i/\Delta V$
        \item[电位移矢量] $\vec{D}=\varepsilon_0\vec{E}+\vec{P}$
    \end{description}

    关系:
    \begin{itemize}
        \item $\vec{P}\xlongequal{\chi_e=\varepsilon_r-1}\chi_e\varepsilon_0\vec{E}, \vec{D}=\varepsilon_0\vec{E}+\vec{P}=\varepsilon_r\varepsilon_0\vec{E}$
        \item $\sigma_{\text{极化}}=\vec{P}\cdot\hat{n}=P_n, \rho_{\text{极化}}=-\nabla\cdot\vec{P}$,注意曲线坐标系下 $\nabla$ 的形式!
        \item $\oiint_S\vec{E}\cdot\dd\vec{s}=Q_{\text{自由}}+Q_{\text{极化}}, \oiint_S\vec{D}\cdot\dd\vec{s}=Q_{\text{自由}}, \oint_L\vec{E}\cdot\dd\vec{l}=0$
    \end{itemize}

    求解:
    $$q_{\text{自由}}\xLongrightarrow{\vec{D}\text{' Guass}}\vec{D}\xLongrightarrow{\vec{D}=\varepsilon_r\varepsilon_0\vec{E}}\vec{E}\xLongrightarrow{\vec{P}=\chi_e\varepsilon_0\vec{E}}\vec{P}\xLongrightarrow{\sigma_{\text{极化}}=\vec{P}\cdot\hat{n}=P_n, \rho_{\text{极化}}=-\nabla\cdot\vec{P}} q_{\text{极化}}$$
\end{frame}

\section{恒磁场}

\subsection{安培定律、毕奥-萨伐尔定律}

\begin{frame}
    \frametitle{安培定律}

    $$\dd\vec{F}_{12}=\frac{\mu_0}{4\pi}\frac{I_2\dd\vec{l}_2\times(I_1\dd\vec{l}_1\times\hat{r}_{12})}{r_{12}^2}$$

    $$\vec{F}_{12}=\frac{\mu_0}{4\pi}\int_{L_1}\int_{L_2}\frac{I_2\dd\vec{l}_2\times(I_1\dd\vec{l}_1\times\hat{r}_{12})}{r_{12}^2}$$

    \begin{itemize}
        \item 两个电流元之间的相互作用不满足牛顿第三定律
    \end{itemize}
\end{frame}

\begin{frame}
    \frametitle{毕奥-萨伐尔定律}

    $$\dd\vec{B}=\frac{\mu_0}{4\pi}\frac{I\dd\vec{l}\times\hat{r}}{r^2}$$

    $$\vec{B}=\frac{\mu_0}{4\pi}\int_L\frac{I\dd\vec{l}\times\hat{r}}{r^2}$$

    \begin{itemize}
        \item $\dd\vec{F}_{12}=I_2\dd\vec{l}_2\times\vec{B}$
        \item 求解简单但是对称性不高的电流分布的磁场:有限长直电流、圆环电流轴线、有限长螺线管
    \end{itemize}

    对于运动电荷:$\int I\dd\vec{l}=\bar{I}\bar{\vec{L}}=\frac{q}{t}\bar{\vec{L}}=q\vec{v}$

    $\Rightarrow$ 关于电流元的公式怎么换成关于运动电荷的?:$I\dd\vec{l}\rightarrow q\vec{v}$,并把积分号去掉

    $$\vec{B}=\frac{\mu_0}{4\pi}\frac{q\vec{v}\times\hat{r}}{r^2}, v\ll c$$
\end{frame}

\subsection{静磁场的高斯定理和安培环路定理}

\begin{frame}
    \frametitle{静磁场的高斯定理}

    $$\oiint_S\vec{B}\cdot\dd\vec{s}=\Phi_{\vec{B}}=0$$
    $$\nabla\cdot\vec{B}=0$$
\end{frame}

\begin{frame}
    \frametitle{静磁场的安培环路定理}

    $$\oint_L\vec{B}\cdot\dd\vec{l}=\mu_0\sum_iI_i$$
    $$\nabla\times\vec{B}=\mu_0\vec{j}$$

    \begin{itemize}
        \item 利用对称性求解磁场:无限长直导线、无限长载流圆柱、长直螺线管、无限大载流导体板
    \end{itemize}
\end{frame}

\begin{frame}
    \frametitle{磁矢势}

    定义:(不唯一)
    \begin{itemize}
        \item $\vec{B}=\nabla\times\vec{A}$
        \item $\oint_L\vec{A}\cdot\dd\vec{l}=\oiint_S\vec{B}\cdot\dd\vec{s}$
    \end{itemize}

    求解:$$\vec{A}(\vec{r})=\frac{\mu_0}{4\pi}\oint_L\frac{I\dd\vec{l}}{r}$$
\end{frame}


\subsection{磁场对电流、运动电荷的作用}

\begin{frame}
    \frametitle{磁场对电流、运动电荷的作用}

    为什么静电场没有这样一个类似的内容呢?

    静电场对应的内容就是:$\vec{F}=q\vec{E}$

    只不过是在磁场的情况下对应的式子是包含了矢量之间的运算,单独写出以示强调
\end{frame}

\begin{frame}
    \frametitle{安培力公式}

    安培力公式 = 安培定律 - 毕奥-萨伐尔定律:$$\dd\vec{F}=I\dd\vec{l}\times\vec{B}$$

    磁偶极子:定义、在外磁场中受力矩($\vec{L}=\vec{m}\times\vec{B}$)
\end{frame}

\begin{frame}
    \frametitle{洛伦兹力公式}

    $I\dd\vec{l}\rightarrow q\vec{v}$,并把积分号去掉:
    $$\vec{F}=q\vec{v}\times\vec{B}$$

    后面有一系列关于这个公式的应用
\end{frame}

\subsection{磁介质}

\begin{frame}
    \frametitle{磁介质}

    一些概念:顺磁介质、抗磁介质、铁磁介质

    定义:
    \begin{description}
        \item[磁化强度矢量] $\vec{M}=\frac{\sum\vec{m}_{\text{分子}}}{\Delta V}$
        \item[磁场强度] $\vec{H}=\frac{\vec{B}}{\mu_0}-\vec{M}$
    \end{description}

    关系:
    \begin{itemize}
        \item $\vec{M}\xlongequal{\chi_m=\mu_r-1}\chi_m\vec{H}, \vec{B}=\mu_0(\vec{H}+\vec{M})=\mu_0\mu_r\vec{H}$
        \item 磁化电流体密度:$\vec{j}_{\text{磁化}}=\nabla\times\vec{M}$,磁化电流面密度:$\vec{\kappa}_{\text{磁化}}=\vec{M}\times\vec{n}$
        \item $\oint_L\vec{B}\cdot\dd\vec{l}=\mu_0(\sum I_{\text{自由}}+\sum I_{\text{磁化}}), \oint_L\vec{H}\cdot\dd\vec{l}=\sum I_{\text{自由}}, \oiint_S\vec{B}\cdot\dd\vec{s}=0$
    \end{itemize}

    求解:
    $$I_{\text{自由}}\xLongrightarrow{\vec{H}\text{'s 环路}}\vec{H}\xLongrightarrow{\vec{B}=\mu_0\mu_r\vec{H}}\vec{B}\xLongrightarrow{\vec{M}=\chi_m\vec{H}}\vec{M}\xLongrightarrow{\vec{j}_{\text{磁化}}=\nabla\times\vec{M}, \vec{\kappa}_{\text{磁化}}=\vec{M}\times\vec{n}}\vec{I}_{\text{磁化}}$$
\end{frame}

\section{电磁感应}

\subsection{电磁感应定律}

\begin{frame}[c]
    \frametitle{电磁感应定律、动生电动势}

    电磁感应定律:
    $$\mathcal{E}=-\frac{\dd\Psi}{\dd t}=-N\frac{\dd\Phi}{\dd t}=-N\frac{\dd(\vec{B}\cdot\vec{S})}{\dd t}=-N(\frac{\dd\vec{B}}{\dd t}\cdot\vec{S}+\vec{B}\cdot\frac{\dd \vec{S}}{\dd t})=\text{感生}+\text{动生}$$

    \begin{itemize}
        \item 楞次定律判断感应电动势方向
    \end{itemize}

    动生电动势:
    $$\mathcal{E}_{\text{动生}}=\int_L(\vec{v}\times\vec{B})\cdot\dd\vec{l}$$

    \begin{itemize}
        \item 平动的、转动的……
    \end{itemize}
\end{frame}

\begin{frame}[c]
    \frametitle{感生电动势、感生电场}

    感生电动势:
    $$\begin{aligned}
            \mathcal{E}_{\text{感生}} & =-\frac{\dd}{\dd t}\iint_S\vec{B}\cdot\dd\vec{s}=\iint_S(-\frac{\partial\vec{B}}{\partial t})\cdot\dd\vec{s}                                                \\
                                    & =-\frac{\dd}{\dd t}\oint_L\vec{A}\cdot\dd\vec{l}=\oint_L(-\frac{\partial\vec{A}}{\partial t})\cdot\dd\vec{l}\equiv\oint_L\vec{E}_{\text{感生}}\cdot\dd\vec{l}
        \end{aligned}$$

    感生电场:$\vec{E}_{\text{感生}}=-\frac{\partial \vec{A}}{\partial t}$

    求感生电场:$\oint_L\vec{E}_{\text{感生}}\cdot\dd\vec{l}=\iint_S(-\frac{\partial\vec{B}}{\partial t})\cdot\dd\vec{s}$

    \begin{itemize}
        \item 静电场、感生电场、电场、恒磁场的比较
    \end{itemize}
\end{frame}

\subsection{自感、互感}

\begin{frame}[c]
    \frametitle{自感、互感}

    对应于静电学当中的电容

    \begin{description}
        \item[自感] $L=\frac{\Phi_m}{I}$
        \item[自感电动势] $\mathcal{E}_L=-L\frac{\dd I}{\dd t}$
        \item[互感] $M=\frac{\Phi_{21}}{I_1}=\frac{\Phi_{12}}{I_2}$,若涉及到铁磁质
        \item[互感电动势] $\mathcal{E}_{12}=-M\frac{\dd I_2}{\dd t}, \mathcal{E}_{21}=-M\frac{\dd I_1}{\dd t}$
    \end{description}
    计算自感、互感:
    $$\text{设一个电流}I\Rightarrow\vec{B}\Rightarrow\Phi\Rightarrow L/M$$
\end{frame}

\section{电磁场}

\subsection{电磁场的能量}

\begin{frame}[c]
    \frametitle{电磁场的能量}

    \begin{description}
        \item[只有电荷] $W_e=\frac{1}{2}\int\rho_eU\dd v=\frac{1}{2}\sigma_eU\dd s$
        \item[能量密度] $w=w_e+w_m=\frac{1}{2}\varepsilon_0 E^2+\frac{1}{2}\frac{B^2}{\mu_0}$
        \item[元件] 电容:$W_e=\frac{1}{2}\frac{Q^2}{C}=\frac{1}{2}CU^2$\\电感:$W_m=\frac{1}{2}\sum^k_{i=1}L_iI_i^2+\frac{1}{2}\sum^k_{i=1, j\neq i}M_{ij}I_iI_j$
    \end{description}
\end{frame}

\subsection{全文背诵环节}

\begin{frame}[c]
    \frametitle{麦克斯韦方程组:积分形式}

    $$\begin{aligned}
             & \oiint_S\vec{E}\cdot\dd\vec{s}=\frac{Q}{\varepsilon}=\frac{1}{\varepsilon}\iiint\rho\dd v                                                                                                                                    \\
             & \oiint_S\vec{B}\cdot\dd\vec{s}=0                                                                                                                                                                                             \\
             & \oint_L\vec{E}\cdot\dd\vec{l}=-\iint_S\frac{\partial\vec{B}}{\partial t}\cdot\dd\vec{s}                                                                                                                                      \\
             & \oint_L\vec{B}\cdot\dd\vec{l}=\mu_0\iint_S\vec{j}\cdot\dd\vec{s}+\mu_0\varepsilon_0\iint\frac{\partial\vec{E}}{\partial t}\cdot\dd\vec{s}=\mu_0\iint(\vec{j}+\varepsilon_0\frac{\partial\vec{E}}{\partial t})\cdot\dd\vec{s}
        \end{aligned}$$
\end{frame}

\begin{frame}[c]
    \frametitle{麦克斯韦方程组:微分形式}

    $$\begin{aligned}
             & \nabla\cdot\vec{E}=\frac{\rho}{\varepsilon_0}                                         \\
             & \nabla\cdot\vec{B}=0                                                                  \\
             & \nabla\times\vec{E}=-\frac{\partial \vec{B}}{\partial t}                              \\
             & \nabla\times\vec{B}=\mu_0\vec{j}+\mu_0\varepsilon_0\frac{\partial\vec{E}}{\partial t}
        \end{aligned}$$
\end{frame}

\subsection{电磁波}

\begin{frame}[c]
    \frametitle{电磁波}

    无源空间:
    $$\begin{aligned}
             & \nabla\cdot\vec{E}=0                                                     \\
             & \nabla\cdot\vec{B}=0                                                     \\
             & \nabla\times\vec{E}=-\frac{\partial \vec{B}}{\partial t}                 \\
             & \nabla\times\vec{B}=\mu_0\varepsilon_0\frac{\partial\vec{E}}{\partial t}
        \end{aligned}\Rightarrow\begin{aligned}
             & \nabla^2\vec{E}-\frac{1}{c^2}\frac{\partial^2\vec{E}}{\partial t}=0 \\
             & \nabla^2\vec{B}-\frac{1}{c^2}\frac{\partial^2\vec{B}}{\partial t}=0
        \end{aligned}, c=\frac{1}{\sqrt{\mu_0\varepsilon_0}}$$

    横波:$\vec{B}\propto\vec{k}\times\vec{E}$
\end{frame}

\begin{frame}[c]
    \frametitle{电磁波}

    \begin{description}
        \item[能量密度] $w=\frac{1}{2}\vec{D}\cdot\vec{E}+\frac{1}{2}\vec{B}\cdot\vec{H}$
        \item[能流密度] (坡印廷矢量)$\vec{S}=\vec{E}\times\vec{H}$
        \item[光强] $I=\frac{1}{2}\sqrt{\frac{\varepsilon}{\mu}}E_0^2$
        \item[动量密度] $\vec{g}=\vec{S}/c=\vec{E}\times\vec{H}/c$
    \end{description}
\end{frame}

\begin{frame}[c]
    \frametitle{End}
    Merry Christmas!\emoji{christmas-tree}
\end{frame}

\end{document}